\chapter{Ethics} % (fold)
\label{cha:ethics}

THE ETHICS OF ARISTOTLE


INTRODUCTION

The _Ethics_ of Aristotle is one half of a single treatise of which his
_Politics_ is the other half. Both deal with one and the same subject.
This subject is what Aristotle calls in one place the "philosophy of
human affairs;" but more frequently Political or Social Science. In the
two works taken together we have their author's whole theory of human
conduct or practical activity, that is, of all human activity which
is not directed merely to knowledge or truth. The two parts of this
treatise are mutually complementary, but in a literary sense each
is independent and self-contained. The proem to the _Ethics_ is an
introduction to the whole subject, not merely to the first part; the
last chapter of the _Ethics_ points forward to the _Politics_, and
sketches for that part of the treatise the order of enquiry to be
pursued (an order which in the actual treatise is not adhered to).

The principle of distribution of the subject-matter between the two
works is far from obvious, and has been much debated. Not much can be
gathered from their titles, which in any case were not given to them by
their author. Nor do these titles suggest any very compact unity in the
works to which they are applied: the plural forms, which survive so
oddly in English (Ethic_s_, Politic_s_), were intended to indicate the
treatment within a single work of a _group_ of connected questions. The
unity of the first group arises from their centring round the topic of
character, that of the second from their connection with the existence
and life of the city or state. We have thus to regard the _Ethics_ as
dealing with one group of problems and the _Politics_ with a second,
both falling within the wide compass of Political Science. Each of these
groups falls into sub-groups which roughly correspond to the several
books in each work. The tendency to take up one by one the various
problems which had suggested themselves in the wide field obscures both
the unity of the subject-matter and its proper articulation. But it is
to be remembered that what is offered us is avowedly rather an enquiry
than an exposition of hard and fast doctrine.

Nevertheless each work aims at a relative completeness, and it is
important to observe the relation of each to the other. The distinction
is not that the one treats of Moral and the other of Political
Philosophy, nor again that the one deals with the moral activity of the
individual and the other with that of the State, nor once more that the
one gives us the theory of human conduct, while the other discusses its
application in practice, though not all of these misinterpretations are
equally erroneous. The clue to the right interpretation is given by
Aristotle himself, where in the last chapter of the _Ethics_ he is
paving the way for the _Politics_. In the _Ethics_ he has not confined
himself to the abstract or isolated individual, but has always thought
of him, or we might say, in his social and political context, with a
given nature due to race and heredity and in certain surroundings.
So viewing him he has studied the nature and formation of his
character--all that he can make himself or be made by others to be.
Especially he has investigated the various admirable forms of human
character and the mode of their production. But all this, though it
brings more clearly before us what goodness or virtue is, and how it is
to be reached, remains mere theory or talk. By itself it does not
enable us to become, or to help others to become, good. For this it is
necessary to bring into play the great force of the Political Community
or State, of which the main instrument is Law. Hence arises the demand
for the necessary complement to the _Ethics, i.e._, a treatise devoted
to the questions which centre round the enquiry; by what organisation
of social or political forces, by what laws or institutions can we best
secure the greatest amount of good character?

We must, however, remember that the production of good character is not
the end of either individual or state action: that is the aim of the one
and the other because good character is the indispensable condition and
chief determinant of happiness, itself the goal of all human doing. The
end of all action, individual or collective, is the greatest happiness
of the greatest number. There is, Aristotle insists, no difference of
kind between the good of one and the good of many or all. The sole
difference is one of amount or scale. This does not mean simply that the
State exists to secure in larger measure the objects of degree which the
isolated individual attempts, but is too feeble, to secure without it.
On the contrary, it rather insists that whatever goods society alone
enables a man to secure have always had to the individual--whether he
realised it or not--the value which, when so secured, he recognises them
to possess. The best and happiest life for the individual is that which
the State renders possible, and this it does mainly by revealing to him
the value of new objects of desire and educating him to appreciate them.
To Aristotle or to Plato the State is, above all, a large and powerful
educative agency which gives the individual increased opportunities of
self-development and greater capacities for the enjoyment of life.

Looking forward, then, to the life of the State as that which aids
support, and combines the efforts of the individual to obtain happiness,
Aristotle draws no hard and fast distinction between the spheres of
action of Man as individual and Man as citizen. Nor does the division of
his discussion into the _Ethics_ and the _Politics_ rest upon any such
distinction. The distinction implied is rather between two stages in the
life of the civilised man--the stage of preparation for the full life of
the adult citizen, and the stage of the actual exercise or enjoyment of
citizenship. Hence the _Ethics_, where his attention is directed upon
the formation of character, is largely and centrally a treatise on Moral
Education. It discusses especially those admirable human qualities which
fit a man for life in an organised civic community, which makes him "a
good citizen," and considers how they can be fostered or created and
their opposites prevented.

This is the kernel of the _Ethics_, and all the rest is subordinate to
this main interest and purpose. Yet "the rest" is not irrelevant; the
whole situation in which character grows and operates is concretely
conceived. There is a basis of what we should call Psychology, sketched
in firm outlines, the deeper presuppositions and the wider issues of
human character and conduct are not ignored, and there is no little of
what we should call Metaphysics. But neither the Psychology nor the
Metaphysics is elaborated, and only so much is brought forward as
appears necessary to put the main facts in their proper perspective
and setting. It is this combination of width of outlook with close
observation of the concrete facts of conduct which gives its abiding
value to the work, and justifies the view of it as containing
Aristotle's Moral Philosophy. Nor is it important merely as summing up
the moral judgments and speculations of an age now long past. It seizes
and dwells upon those elements and features in human practice which are
most essential and permanent, and it is small wonder that so much in it
survives in our own ways of regarding conduct and speaking of it. Thus
it still remains one of the classics of Moral Philosophy, nor is its
value likely soon to be exhausted.

As was pointed out above, the proem (Book I., cc. i-iii.) is a prelude
to the treatment of the whole subject covered by the _Ethics_ and the
_Politics_ together. It sets forth the purpose of the enquiry, describes
the spirit in which it is to be undertaken and what ought to be the
expectation of the reader, and lastly states the necessary conditions
of studying it with profit. The aim of it is the acquisition and
propagation of a certain kind of knowledge (science), but this knowledge
and the thinking which brings it about are subsidiary to a practical
end. The knowledge aimed at is of what is best for man and of the
conditions of its realisation. Such knowledge is that which in its
consumate form we find in great statesmen, enabling them to organise and
administer their states and regulate by law the life of the citizens
to their advantage and happiness, but it is the same kind of knowledge
which on a smaller scale secures success in the management of the family
or of private life.

It is characteristic of such knowledge that it should be deficient
in "exactness," in precision of statement, and closeness of logical
concatenation. We must not look for a mathematics of conduct. The
subject-matter of Human Conduct is not governed by necessary and uniform
laws. But this does not mean that it is subject to no laws. There
are general principles at work in it, and these can be formulated in
"rules," which rules can be systematised or unified. It is all-important
to remember that practical or moral rules are only general and always
admit of exceptions, and that they arise not from the mere complexity
of the facts, but from the liability of the facts to a certain
unpredictable variation. At their very best, practical rules state
probabilities, not certainties; a relative constancy of connection is
all that exists, but it is enough to serve as a guide in life. Aristotle
here holds the balance between a misleading hope of reducing the
subject-matter of conduct to a few simple rigorous abstract principles,
with conclusions necessarily issuing from them, and the view that it is
the field of operation of inscrutable forces acting without predictable
regularity. He does not pretend to find in it absolute uniformities, or
to deduce the details from his principles. Hence, too, he insists on the
necessity of experience as the source or test of all that he has to
say. Moral experience--the actual possession and exercise of good
character--is necessary truly to understand moral principles and
profitably to apply them. The mere intellectual apprehension of them is
not possible, or if possible, profitless.

The _Ethics_ is addressed to students who are presumed both to have
enough general education to appreciate these points, and also to have a
solid foundation of good habits. More than that is not required for the
profitable study of it.

If the discussion of the nature and formation of character be regarded
as the central topic of the _Ethics_, the contents of Book I., cc.
iv.-xii. may be considered as still belonging to the introduction and
setting, but these chapters contain matter of profound importance and
have exercised an enormous influence upon subsequent thought. They lay
down a principle which governs all Greek thought about human life, viz.
that it is only intelligible when viewed as directed towards some end or
good. This is the Greek way of expressing that all human life involves
an ideal element--something which it is not yet and which under certain
conditions it is to be. In that sense Greek Moral Philosophy is
essentially idealistic. Further it is always assumed that all human
practical activity is directed or "oriented" to a _single_ end, and that
that end is knowable or definable in advance of its realisation. To know
it is not merely a matter of speculative interest, it is of the highest
practical moment for only in the light of it can life be duly guided,
and particularly only so can the state be properly organised and
administered. This explains the stress laid throughout by Greek Moral
Philosophy upon the necessity of knowledge as a condition of the best
life. This knowledge is not, though it includes knowledge of the nature
of man and his circumstances, it is knowledge of what is best--of man's
supreme end or good.

But this end is not conceived as presented to him by a superior power
nor even as something which _ought_ to be. The presentation of the Moral
Ideal as Duty is almost absent. From the outset it is identified with
the object of desire, of what we not merely judge desirable but actually
do desire, or that which would, if realised, satisfy human desire. In
fact it is what we all, wise and simple, agree in naming "Happiness"
(Welfare or Well-being)

In what then does happiness consist? Aristotle summarily sets aside the
more or less popular identifications of it with abundance of physical
pleasures, with political power and honour, with the mere possession of
such superior gifts or attainments as normally entitle men to these,
with wealth. None of these can constitute the end or good of man as
such. On the other hand, he rejects his master Plato's conception of a
good which is the end of the whole universe, or at least dismisses it
as irrelevant to his present enquiry. The good towards which all human
desires and practical activities are directed must be one conformable to
man's special nature and circumstances and attainable by his efforts.
There is in Aristotle's theory of human conduct no trace of Plato's
"other worldliness", he brings the moral ideal in Bacon's phrase down to
"right earth"--and so closer to the facts and problems of actual human
living. Turning from criticism of others he states his own positive view
of Happiness, and, though he avowedly states it merely in outline his
account is pregnant with significance. Human Happiness lies in activity
or energising, and that in a way peculiar to man with his given nature
and his given circumstances, it is not theoretical, but practical: it is
the activity not of reason but still of a being who possesses reason and
applies it, and it presupposes in that being the development, and
not merely the natural possession, of certain relevant powers and
capacities. The last is the prime condition of successful living
and therefore of satisfaction, but Aristotle does not ignore other
conditions, such as length of life, wealth and good luck, the absence or
diminution of which render happiness not impossible, but difficult of
attainment.

It is interesting to compare this account of Happiness with Mill's
in _Utilitarianism_. Mill's is much the less consistent: at times
he distinguishes and at times he identifies, happiness, pleasure,
contentment, and satisfaction. He wavers between belief in its general
attainability and an absence of hopefulness. He mixes up in an arbitrary
way such ingredients as "not expecting more from life than it is capable
of bestowing," "mental cultivation," "improved laws," etc., and in fact
leaves the whole conception vague, blurred, and uncertain. Aristotle
draws the outline with a firmer hand and presents a more definite ideal.
He allows for the influence on happiness of conditions only partly, if
at all, within the control of man, but he clearly makes the man positive
determinant of man's happiness he in himself, and more particularly
in what he makes directly of his own nature, and so indirectly of his
circumstances. "'Tis in ourselves that we are thus or thus" But once
more this does not involve an artificial or abstract isolation of the
individual moral agent from his relation to other persons or things from
his context in society and nature, nor ignore the relative dependence of
his life upon a favourable environment.

The main factor which determines success or failure in human life is the
acquisition of certain powers, for Happiness is just the exercise or
putting forth of these in actual living, everything else is secondary
and subordinate. These powers arise from the due development of certain
natural aptitudes which belong (in various degrees) to human nature as
such and therefore to all normal human beings. In their developed
form they are known as virtues (the Greek means simply "goodnesses,"
"perfections," "excellences," or "fitnesses"), some of them are
physical, but others are psychical, and among the latter some, and these
distinctively or peculiarly human, are "rational," _i e_, presuppose the
possession and exercise of mind or intelligence. These last fall into
two groups, which Aristotle distinguishes as Goodnesses of Intellect and
Goodnesses of Character. They have in common that they all excite in us
admiration and praise of their possessors, and that they are not natural
endowments, but acquired characteristics But they differ in important
ways. (1) the former are excellences or developed powers of the
reason as such--of that in us which sees and formulates laws, rules,
regularities systems, and is content in the vision of them, while the
latter involve a submission or obedience to such rules of something
in us which is in itself capricious and irregular, but capable of
regulation, viz our instincts and feelings, (2) the former are acquired
by study and instruction, the latter by discipline. The latter
constitute "character," each of them as a "moral virtue" (literally "a
goodness of character"), and upon them primarily depends the realisation
of happiness. This is the case at least for the great majority of men,
and for all men their possession is an indispensable basis of the
best, _i e_, the most desirable life. They form the chief or central
subject-matter of the _Ethics_.

Perhaps the truest way of conceiving Aristotle's meaning here is to
regard a moral virtue as a form of obedience to a maxim or rule of
conduct accepted by the agent as valid for a class of recurrent
situations in human life. Such obedience requires knowledge of the rule
and acceptance of it _as the rule_ of the agent's own actions, but not
necessarily knowledge of its ground or of its systematic connexion with
other similarly known and similarly accepted rules (It may be remarked
that the Greek word usually translated "reason," means in almost all
cases in the _Ethics_ such a rule, and not the faculty which apprehends,
formulates, considers them).

The "moral virtues and vices" make up what we call character, and the
important questions arise: (1) What is character? and (2) How is it
formed? (for character in this sense is not a natural endowment; it is
formed or produced). Aristotle deals with these questions in the reverse
order. His answers are peculiar and distinctive--not that they are
absolutely novel (for they are anticipated in Plato), but that by him
they are for the first time distinctly and clearly formulated.

(1.) Character, good or bad, is produced by what Aristotle calls
"habituation," that is, it is the result of the repeated doing of acts
which have a similar or common quality. Such repetition acting upon
natural aptitudes or propensities gradually fixes them in one or other
of two opposite directions, giving them a bias towards good or evil.
Hence the several acts which determine goodness or badness of character
must be done in a certain way, and thus the formation of good character
requires discipline and direction from without. Not that the agent
himself contributes nothing to the formation of his character, but that
at first he needs guidance. The point is not so much that the process
cannot be safely left to Nature, but that it cannot be entrusted to
merely intellectual instruction. The process is one of assimilation,
largely by imitation and under direction and control. The result is a
growing understanding of what is done, a choice of it for its own sake,
a fixity and steadiness of purpose. Right acts and feelings become,
through habit, easier and more pleasant, and the doing of them a "second
nature." The agent acquires the power of doing them freely, willingly,
more and more "of himself."

But what are "right" acts? In the first place, they are those that
conform to a rule--to the right rule, and ultimately to reason. The
Greeks never waver from the conviction that in the end moral conduct is
essentially reasonable conduct. But there is a more significant way of
describing their "rightness," and here for the first time Aristotle
introduces his famous "Doctrine of the Mean." Reasoning from the analogy
of "right" physical acts, he pronounces that rightness always means
adaptation or adjustment to the special requirements of a situation. To
this adjustment he gives a quantitative interpretation. To do (or to
feel) what is right in a given situation is to do or to feel just the
amount required--neither more nor less: to do wrong is to do or to
feel too much or too little--to fall short of or over-shoot, "a mean"
determined by the situation. The repetition of acts which lie in the
mean is the cause of the formation of each and every "goodness of
character," and for this "rules" can be given.

(2) What then is a "moral virtue," the result of such a process duly
directed? It is no mere mood of feeling, no mere liability to emotion,
no mere natural aptitude or endowment, it is a permanent _state_ of the
agent's self, or, as we might in modern phrase put it, of his will,
it consists in a steady self-imposed obedience to a rule of action
in certain situations which frequently recur in human life. The rule
prescribes the control and regulation within limits of the agent's
natural impulses to act and feel thus and thus. The situations fall into
groups which constitute the "fields" of the several "moral virtues",
for each there is a rule, conformity to which secures rightness in
the individual acts. Thus the moral ideal appears as a code of
rules, accepted by the agent, but as yet _to him_ without rational
justification and without system or unity. But the rules prescribe no
mechanical uniformity: each within its limits permits variety, and the
exactly right amount adopted to the requirements of the individual
situation (and every actual situation is individual) must be determined
by the intuition of the moment. There is no attempt to reduce the rich
possibilities of right action to a single monotonous type. On the
contrary, there are acknowledged to be many forms of moral virtue, and
there is a long list of them, with their correlative vices enumerated.

The Doctrine of the Mean here takes a form in which it has impressed
subsequent thinkers, but which has less importance than is usually
ascribed to it. In the "Table of the Virtues and Vices," each of the
virtues is flanked by two opposite vices, which are respectively the
excess and defect of that which in due measure constitutes the virtue.
Aristotle tries to show that this is the case in regard to every virtue
named and recognised as such, but his treatment is often forced and the
endeavour is not very successful. Except as a convenient principle
of arrangement of the various forms of praiseworthy or blameworthy
characters, generally acknowledged as such by Greek opinion, this form
of the doctrine is of no great significance.

Books III-V are occupied with a survey of the moral virtues and vices.
These seem to have been undertaken in order to verify in detail the
general account, but this aim is not kept steadily in view. Nor is there
any well-considered principle of classification. What we find is a sort
of portrait-gallery of the various types of moral excellence which
the Greeks of the author's age admired and strove to encourage. The
discussion is full of acute, interesting and sometimes profound
observations. Some of the types are those which are and will be admired
at all times, but others are connected with peculiar features of Greek
life which have now passed away. The most important is that of Justice
or the Just Man, to which we may later return. But the discussion is
preceded by an attempt to elucidate some difficult and obscure points in
the general account of moral virtue and action (Book III, cc i-v). This
section is concerned with the notion of Responsibility. The discussion
designedly excludes what we may call the metaphysical issues of the
problem, which here present themselves, it moves on the level of thought
of the practical man, the statesman, and the legislator. Coercion and
ignorance of relevant circumstances render acts involuntary and exempt
their doer from responsibility, otherwise the act is voluntary and the
agent responsible, choice or preference of what is done, and inner
consent to the deed, are to be presumed. Neither passion nor ignorance
of the right rule can extenuate responsibility. But there is a
difference between acts done voluntarily and acts done of _set_ choice
or purpose. The latter imply Deliberation. Deliberation involves
thinking, thinking out means to ends: in deliberate acts the whole
nature of the agent consents to and enters into the act, and in a
peculiar sense they are his, they _are_ him in action, and the most
significant evidence of what he is. Aristotle is unable wholly to avoid
allusion to the metaphysical difficulties and what he does here say upon
them is obscure and unsatisfactory. But he insists upon the importance
in moral action of the agent's inner consent, and on the reality of his
individual responsibility. For his present purpose the metaphysical
difficulties are irrelevant.

The treatment of Justice in Book V has always been a source of great
difficulty to students of the _Ethics_. Almost more than any other part
of the work it has exercised influence upon mediaeval and modern thought
upon the subject. The distinctions and divisions have become part of the
stock-in-trade of would be philosophic jurists. And yet, oddly enough,
most of these distinctions have been misunderstood and the whole purport
of the discussion misconceived. Aristotle is here dealing with justice
in a restricted sense viz as that special goodness of character which
is required of every adult citizen and which can be produced by early
discipline or habituation. It is the temper or habitual attitude
demanded of the citizen for the due exercise of his functions as taking
part in the administration of the civic community--as a member of the
judicature and executive. The Greek citizen was only exceptionally, and
at rare intervals if ever, a law-maker while at any moment he might
be called upon to act as a judge (juryman or arbitrator) or as an
administrator. For the work of a legislator far more than the moral
virtue of justice or fairmindedness was necessary, these were requisite
to the rarer and higher "intellectual virtue" of practical wisdom. Then
here, too, the discussion moves on a low level, and the raising of
fundamental problems is excluded. Hence "distributive justice" is
concerned not with the large question of the distribution of political
power and privileges among the constituent members or classes of the
state but with the smaller questions of the distribution among those of
casual gains and even with the division among private claimants of a
common fund or inheritance, while "corrective justice" is concerned
solely with the management of legal redress. The whole treatment is
confused by the unhappy attempt to give a precise mathematical form to
the principles of justice in the various fields distinguished. Still it
remains an interesting first endeavour to give greater exactness to some
of the leading conceptions of jurisprudence.

Book VI appears to have in view two aims: (1) to describe goodness of
intellect and discover its highest form or forms; (2) to show how this
is related to goodness of character, and so to conduct generally. As all
thinking is either theoretical or practical, goodness of intellect has
_two_ supreme forms--Theoretical and Practical Wisdom. The first, which
apprehends the eternal laws of the universe, has no direct relation to
human conduct: the second is identical with that master science of human
life of which the whole treatise, consisting of the _Ethics_ and the
_Politics_, is an exposition. It is this science which supplies the
right rules of conduct Taking them as they emerge in and from practical
experience, it formulates them more precisely and organises them into a
system where they are all seen to converge upon happiness. The mode in
which such knowledge manifests itself is in the power to show that such
and such rules of action follow from the very nature of the end or good
for man. It presupposes and starts from a clear conception of the end
and the wish for it as conceived, and it proceeds by a deduction which
is dehberation writ large. In the man of practical wisdom this process
has reached its perfect result, and the code of right rules is
apprehended as a system with a single principle and so as something
wholly rational or reasonable He has not on each occasion to seek and
find the right rule applicable to the situation, he produces it at
once from within himself, and can at need justify it by exhibiting its
rationale, _i.e._ , its connection with the end. This is the consummate
form of reason applied to conduct, but there are minor forms of it, less
independent or original, but nevertheless of great value, such as the
power to think out the proper cause of policy in novel circumstances or
the power to see the proper line of treatment to follow in a court of
law.

The form of the thinking which enters into conduct is that which
terminates in the production of a rule which declares some means to the
end of life. The process presupposes _(a)_ a clear and just apprehension
of the nature of that end--such as the _Ethics_ itself endeavours to
supply; _(b)_ a correct perception of the conditions of action, _(a)_ at
least is impossible except to a man whose character has been duly formed
by discipline; it arises only in a man who has acquired moral virtue.
For such action and feeling as forms bad character, blinds the eye of
the soul and corrupts the moral principle, and the place of practical
wisdom is taken by that parody of itself which Aristotle calls
"cleverness"--the "wisdom" of the unscrupulous man of the world. Thus
true practical wisdom and true goodness of character are interdependent;
neither is genuinely possible or "completely" present without the other.
This is Aristotle's contribution to the discussion of the question, so
central in Greek Moral Philosophy, of the relation of the intellectual
and the passionate factors in conduct.

Aristotle is not an intuitionist, but he recognises the implication in
conduct of a direct and immediate apprehension both of the end and of
the character of his circumstances under which it is from moment to
moment realised. The directness of such apprehension makes it analogous
to sensation or sense-perception; but it is on his view in the end due
to the existence or activity in man of that power in him which is the
highest thing in his nature, and akin to or identical with the divine
nature--mind, or intelligence. It is this which reveals to us what is
best for us--the ideal of a happiness which is the object of our real
wish and the goal of all our efforts. But beyond and above the practical
ideal of what is best _for man_ begins to show itself another and still
higher ideal--that of a life not distinctively human or in a narrow
sense practical, yet capable of being participated in by man even under
the actual circumstances of this world. For a time, however, this
further and higher ideal is ignored.

The next book (Book VII.), is concerned partly with moral conditions, in
which the agent seems to rise above the level of moral virtue or fall
below that of moral vice, but partly and more largely with conditions in
which the agent occupies a middle position between the two. Aristotle's
attention is here directed chiefly towards the phenomena of
"Incontinence," weakness of will or imperfect self-control. This
condition was to the Greeks a matter of only too frequent experience,
but it appeared to them peculiarly difficult to understand. How can a
man know what is good or best for him, and yet chronically fail to act
upon his knowledge? Socrates was driven to the paradox of denying the
possibility, but the facts are too strong for him. Knowledge of the
right rule may be present, nay the rightfulness of its authority may be
acknowledged, and yet time after time it may be disobeyed; the will may
be good and yet overmastered by the force of desire, so that the act
done is contrary to the agent's will. Nevertheless the act may be the
agent's, and the will therefore divided against itself. Aristotle is
aware of the seriousness and difficulty of the problem, but in spite of
the vividness with which he pictures, and the acuteness with which he
analyses, the situation in which such action occurs, it cannot be said
that he solves the problem. It is time that he rises above the abstract
view of it as a conflict between reason and passion, recognising that
passion is involved in the knowledge which in conduct prevails or is
overborne, and that the force which leads to the wrong act is not blind
or ignorant passion, but always has some reason in it. But he tends to
lapse back into the abstraction, and his final account is perplexed and
obscure. He finds the source of the phenomenon in the nature of the
desire for bodily pleasures, which is not irrational but has something
rational in it. Such pleasures are not necessarily or inherently bad, as
has sometimes been maintained; on the contrary, they are good, but only
in certain amounts or under certain conditions, so that the will is
often misled, hesitates, and is lost.

Books VIII. and IX. (on Friendship) are almost an interruption of the
argument. The subject-matter of them was a favourite topic of ancient
writers, and the treatment is smoother and more orderly than elsewhere
in the _Ethics_. The argument is clear, and may be left without
comment to the readers. These books contain a necessary and attractive
complement to the somewhat dry account of Greek morality in the
preceding books, and there are in them profound reflections on what may
be called the metaphysics of friendship or love.

At the beginning of Book X. we return to the topic of Pleasure, which
is now regarded from a different point of view. In Book VII. the
antagonists were those who over-emphasised the irrationality or badness
of Pleasure: here it is rather those who so exaggerate its value as to
confuse or identify it with the good or Happiness. But there is offered
us in this section much more than criticism of the errors of others.
Answers are given both to the psychological question, "What is
Pleasure?" and to the ethical question, "What is its value?" Pleasure,
we are told, is the natural concomitant and index of perfect activity,
distinguishable but inseparable from it--"the activity of a subject at
its best acting upon an object at its best." It is therefore always
and in itself a good, but its value rises and falls with that of the
activity with which it is conjoined, and which it intensifies and
perfects. Hence it follows that the highest and best pleasures are those
which accompany the highest and best activity.

Pleasure is, therefore, a necessary element in the best life, but it is
not the whole of it nor the principal ingredient. The value of a life
depends upon the nature and worth of the activity which it involves;
given the maximum of full free action, the maximum of pleasure necessary
follows. But on what sort of life is such activity possible? This leads
us back to the question, What is happiness? In what life can man find
the fullest satisfaction for his desires? To this question Aristotle
gives an answer which cannot but surprise us after what has preceded.
True Happiness, great satisfaction, cannot be found by man in any form
of "practical" life, no, not in the fullest and freest exercise possible
of the "moral virtues," not in the life of the citizen or of the
great soldier or statesman. To seek it there is to court failure and
disappointment. It is to be found in the life of the onlooker, the
disinterested spectator; or, to put it more distinctly, "in the life of
the philosopher, the life of scientific and philosophic contemplation."
The highest and most satisfying form of life possible to man is "the
contemplative life"; it is only in a secondary sense and for those
incapable of their life, that the practical or moral ideal is the best.
It is time that such a life is not distinctively human, but it is the
privilege of man to partake in it, and such participation, at however
rare intervals and for however short a period, is the highest Happiness
which human life can offer. All other activities have value only because
and in so far as they render _this_ life possible.

But it must not be forgotten that Aristotle conceives of this life as
one of intense activity or energising: it is just this which gives it
its supremacy. In spite of the almost religious fervour with which he
speaks of it ("the most orthodox of his disciples" paraphrases his
meaning by describing its content as "the service and vision of God"),
it is clear that he identified it with the life of the philosopher, as
he understood it, a life of ceaseless intellectual activity in which at
least at times all the distractions and disturbances inseparable from
practical life seemed to disappear and become as nothing. This ideal was
partly an inheritance from the more ardent idealism of his master Plato,
but partly it was the expression of personal experience.

The nobility of this ideal cannot be questioned; the conception of the
end of man or a life lived for truth--of a life blissfully absorbed in
the vision of truth--is a lofty and inspiring one. But we cannot resist
certain criticisms upon its presentation by Aristotle: (1) the relation
of it to the lower ideal of practice is left somewhat obscure; (2) it is
described in such a way as renders its realisation possible only to a
gifted few, and under exceptional circumstances; (3) it seems in various
ways, as regards its content, to be unnecessarily and unjustifiably
limited. But it must be borne in mind that this is a first endeavour to
determine its principle, and that similar failures have attended the
attempts to describe the "religious" or the "spiritual" ideals of
life, which have continually been suggested by the apparently inherent
limitations of the "practical" or "moral" life, which is the subject of
Moral Philosophy.

The Moral Ideal to those who have most deeply reflected on it leads
to the thought of an Ideal beyond and above it, which alone gives it
meaning, but which seems to escape from definite conception by man.
The richness and variety of this Ideal ceaselessly invite, but as
ceaselessly defy, our attempts to imprison it in a definite formula or
portray it in detailed imagination. Yet the thought of it is and remains
inexpungable from our minds.

This conception of the best life is not forgotten in the _Politics_ The
end of life in the state is itself well-living and well-doing--a life
which helps to produce the best life The great agency in the production
of such life is the State operating through Law, which is Reason backed
by Force. For its greatest efficiency there is required the development
of a science of legislation. The main drift of what he says here is that
the most desirable thing would be that the best reason of the community
should be embodied in its laws. But so far as that is not possible, it
still is true that anyone who would make himself and others better must
become a miniature legislator--must study the general principles of law,
morality, and education. The conception of [Grek: politikae] with which
he opened the _Ethics_ would serve as a guide to a father educating his
children as well as to the legislator legislating for the state. Finding
in his predecessors no developed doctrine on this subject, Aristotle
proposes himself to undertake the construction of it, and sketches in
advance the programme of the _Politics_ in the concluding sentence of
the _Ethics_ His ultimate object is to answer the questions, What is the
best form of Polity, how should each be constituted, and what laws and
customs should it adopt and employ? Not till this answer is given will
"the philosophy of human affairs" be complete.

On looking back it will be seen that the discussion of the central topic
of the nature and formation of character has expanded into a Philosophy
of Human Conduct, merging at its beginning and end into metaphysics
The result is a Moral Philosophy set against a background of Political
Theory and general Philosophy. The most characteristic features of this
Moral Philosophy are due to the fact of its essentially teleological
view of human life and action: (1) Every human activity, but especially
every human practical activity, is directed towards a simple End
discoverable by reflection, and this End is conceived of as the object
of universal human desire, as something to be enjoyed, not as something
which ought to be done or enacted. Anstotle's Moral Philosophy is not
hedonistic but it is eudæmomstic, the end is the enjoyment of Happiness,
not the fulfilment of Duty. (2) Every human practical activity derives
its value from its efficiency as a means to that end, it is good or bad,
right or wrong, as it conduces or fails to conduce to Happiness Thus his
Moral Philosophy is essentially utilitarian or prudential Right action
presupposes Thought or Thinking, partly on the development of a clearer
and distincter conception of the end of desire, partly as the deduction
from that of rules which state the normally effective conditions of
its realisation. The thinking involved in right conduct is
calculation--calculation of means to an end fixed by nature and
foreknowable Action itself is at its best just the realisation of a
scheme preconceived and thought out beforehand, commending itself by its
inherent attractiveness or promise of enjoyment.

This view has the great advantage of exhibiting morality as essentially
reasonable, but the accompanying disadvantage of lowering it into a
somewhat prosaic and unideal Prudentialism, nor is it saved from this
by the tacking on to it, by a sort of after-thought, of the second and
higher Ideal--an addition which ruins the coherence of the account
without really transmuting its substance The source of our
dissatisfaction with the whole theory lies deeper than in its tendency
to identify the end with the maximum of enjoyment or satisfaction, or to
regard the goodness or badness of acts and feelings as lying solely in
their efficacy to produce such a result It arises from the application
to morality of the distinction of means and end For this distinction,
for all its plausibility and usefulness in ordinary thought and speech,
cannot finally be maintained In morality--and this is vital to its
character--everything is both means and end, and so neither in
distinction or separation, and all thinking about it which presupposes
the finality of this distinction wanders into misconception and error.
The thinking which really matters in conduct is not a thinking which
imaginatively forecasts ideals which promise to fulfil desire, or
calculates means to their attainment--that is sometimes useful,
sometimes harmful, and always subordinate, but thinking which reveals
to the agent the situation in which he is to act, both, that is, the
universal situation on which as man he always and everywhere stands,
and the ever-varying and ever-novel situation in which he as this
individual, here and now, finds himself. In such knowledge of given
or historic fact lie the natural determinants of his conduct, in such
knowledge alone lies the condition of his freedom and his good.

But this does not mean that Moral Philosophy has not still much to
learn from Aristotle's _Ethics_. The work still remains one of the best
introductions to a study of its important subject-matter, it spreads
before us a view of the relevant facts, it reduces them to manageable
compass and order, it raises some of the central problems, and makes
acute and valuable suggestions towards their solution. Above all, it
perpetually incites to renewed and independent reflection upon them.

J. A. SMITH


  The following is a list of the works of Aristotle:--

  First edition of works (with omission of Rhetorica, Poetica, and
  second book of Economica), 5 vols by Aldus Manutius, Venice, 1495 8,
  re impression supervised by Erasmus and with certain corrections by
  Grynaeus (including Rhetorica and Poetica), 1531, 1539, revised 1550,
  later editions were followed by that of Immanuel Bekker and Brandis
  (Greek and Latin), 5 vols. The 5th vol contains the Index by Bomtz,
  1831-70, Didot edition (Greek and Latin), 5 vols 1848 74

  ENGLISH TRANSLATIONS Edited by T Taylor, with Porphyry's
  Introduction, 9 vols, 1812, under editorship of J A Smith and
  W D Ross, II vols, 1908-31, Loeb editions Ethica, Rhetorica,
  Poetica, Physica, Politica, Metaphysica, 1926-33

  Later editions of separate works
  _De Anima_ Torstrik, 1862, Trendelenburg, 2nd edition, 1877,
  with English translation, L Wallace, 1882, Biehl, 1884, 1896, with
  English, R D Hicks, 1907
_Ethica_ J S Brewer (Nicomachean), 1836, W E Jelf, 1856, J F T Rogers,
1865, A Grant, 1857 8, 1866, 1874, 1885, E Moore, 1871, 1878, 4th
edition, 1890, Ramsauer (Nicomachean), 1878, Susemihl, 1878, 1880,
revised by O Apelt, 1903, A Grant, 1885, I Bywater (Nicomachean), 1890,
J Burnet, 1900

_Historia Animalium_ Schneider, 1812, Aubert and Wimmer, 1860;
Dittmeyer, 1907

_Metaphysica_ Schwegler, 1848, W Christ, 1899

_Organon_ Waitz, 1844 6

_Poetica_ Vahlen, 1867, 1874, with Notes by E Moore, 1875, with English
translation by E R Wharton, 1883, 1885, Uberweg, 1870, 1875, with
German translation, Susemihl, 1874, Schmidt, 1875, Christ, 1878, I
Bywater, 1898, T G Tucker, 1899

_De Republica Athenientium_ Text and facsimile of Papyrus, F G Kenyon,
1891, 3rd edition, 1892, Kaibel and Wilamowitz-Moellendorf, 1891, 3rd
edition, 1898, Van Herwerden and Leeuwen (from Kenyon's text), 1891,
Blass, 1892, 1895, 1898, 1903, J E Sandys, 1893

_Politica_ Susemihl, 1872, with German, 1878, 3rd edition, 1882,
Susemihl and Hicks, 1894, etc, O Immisch, 1909

_Physica_ C Prantl, 1879

_Rhetorica_ Stahr, 1862, Sprengel (with Latin text), 1867, Cope and
Sandys, 1877, Roemer, 1885, 1898

ENGLISH TRANSLATIONS OF ONE OR MORE WORKS De Anima (with Parva
Naturalia), by W A Hammond, 1902 Ethica Of Morals to Nicomachus, by
E Pargiter, 1745, with Politica by J Gillies, 1797, 1804, 1813, with
Rhetorica and Poetica, by T Taylor, 1818, and later editions Nicomachean
Ethics, 1819, mainly from text of Bekker by D P Chase, 1847, revised
1861, and later editions, with an introductory essay by G H Lewes
(Camelot Classics) 1890, re-edited by J M Mitchell (New Universal
Library), 1906, 1910, by R W Browne (Bohn's Classical Library),
1848, etc, by R Williams, 1869, 1876, by W M Hatch and others (with
translation of paraphrase attributed to Andronicus of Rhodes), edited
by E Hatch, 1879 by F H Peters, 1881, J E C Welldon, 1892, J Gillies
(Lubbock's Hundred Books) 1893 Historia Animalium, by R Creswell (Bonn's
Classical Library) 1848, with Treatise on Physiognomy, by T Taylor,
1809 Metaphysica, by T Taylor, 1801, by J H M Mahon (Bohn's Classical
Library), 1848 Organon, with Porphyry's Introduction, by O F Owen
(Bohn's Classical Library), 1848 Posterior Analytics, E Poste, 1850, E S
Bourchier, 1901, On Fallacies, E Poste, 1866 Parva Naturaha (Greek and
English), by G R T Ross, 1906, with De Anima, by W A Hammond, 1902 Youth
and Old Age, Life and Death and Respiration, W Ogle 1897 Poetica, with
Notes from the French of D Acier, 1705, by H J Pye, 1788, 1792, T
Twining, 1789, 1812, with Preface and Notes by H Hamilton, 1851,
Treatise on Rhetorica and Poetica, by T Hobbes (Bohn's Classical
Library), 1850, by Wharton, 1883 (see Greek version), S H Butcher, 1895,
1898, 3rd edition, 1902, E S Bourchier, 1907, by Ingram Bywater, 1909 De
Partibus Animalium, W Ogle, 1882 De Republica Athenientium, by E Poste,
1891, F G Kenyon, 1891, T J Dymes, 1891 De Virtutibus et Vitus, by W
Bridgman, 1804 Politica, from the French of Regius, 1598, by W Ellis,
1776, 1778, 1888 (Morley's Universal Library), 1893 (Lubbock's Hundred
Books) by E Walford (with Æconomics, and Life by Dr Gillies), (Bohn's
Classical Library), 1848, J E. C. Welldon, 1883, B Jowett, 1885, with
Introduction and Index by H W C Davis, 1905, Books i iii iv (vii)
from Bekker's text by W E Bolland, with Introduction by A Lang, 1877.
Problemata (with writings of other philosophers), 1597, 1607, 1680,
1684, etc. Rhetorica, A summary by T Hobbes, 1655 (?), new edition,
1759, by the translators of the Art of Thinking, 1686, 1816, by D M
Crimmin, 1812, J Gillies, 1823, Anon 1847, J E C Welldon, 1886, R C
Jebb, with Introduction and Supplementary Notes by J E Sandys, 1909 (see
under Poetica and Ethica). Secreta Secretorum (supposititious work),
Anon 1702, from the Hebrew version by M Gaster, 1907, 1908. Version by
Lydgate and Burgh, edited by R Steele (E E T S), 1894, 1898.

LIFE, ETC J W Blakesley, 1839, A Crichton (Jardine's Naturalist's
Library), 1843, JS Blackie, Four Phases of Morals, Socrates, Aristotle,
etc, 1871, G Grote, Aristotle, edited by A Bain and G C Robertson, 1872,
1880, E Wallace, Outlines of the Philosophy of Aristotle, 1875, 1880,
A Grant (Ancient Classics for English readers), 1877, T Davidson,
Aristotle and Ancient Educational Ideals (Great Educators), 1892, F
Sewall, Swedenborg and Aristotle, 1895, W A Heidel, The Necessary
and the Contingent of the Aristotelian System (University of Chicago
Contributions to Philosophy), 1896, F W Bain, On the Realisation of the
Possible, and the Spirit of Aristotle, 1899, J H Hyslop, The Ethics of
the Greek Philosophers, etc (Evolution of Ethics), 1903, M V Williams,
Six Essays on the Platonic Theory of Knowledge as expounded in the later
dialogues and reviewed by Aristotle, 1908, J M Watson, Aristotle's
Criticism of Plato, 1909 A E Taylor, Aristotle, 1919, W D Ross,
Aristotle, 1923.




ARISTOTLE'S ETHICS




BOOK I


Every art, and every science reduced to a teachable form, and in like
manner every action and moral choice, aims, it is thought, at some good:
for which reason a common and by no means a bad description of the Chief
Good is, "that which all things aim at."

Now there plainly is a difference in the Ends proposed: for in some
cases they are acts of working, and in others certain works or tangible
results beyond and beside the acts of working: and where there are
certain Ends beyond and beside the actions, the works are in their
nature better than the acts of working. Again, since actions and arts
and sciences are many, the Ends likewise come to be many: of the healing
art, for instance, health; of the ship-building art, a vessel; of
the military art, victory; and of domestic management, wealth; are
respectively the Ends.

And whatever of such actions, arts, or sciences range under some one
faculty (as under that of horsemanship the art of making bridles, and
all that are connected with the manufacture of horse-furniture in
general; this itself again, and every action connected with war, under
the military art; and in the same way others under others), in all such,
the Ends of the master-arts are more choice-worthy than those ranging
under them, because it is with a view to the former that the latter are
pursued.

(And in this comparison it makes no difference whether the acts of
working are themselves the Ends of the actions, or something further
beside them, as is the case in the arts and sciences we have been just
speaking of.)

[Sidenote: II] Since then of all things which may be done there is some
one End which we desire for its own sake, and with a view to which we
desire everything else; and since we do not choose in all instances with
a further End in view (for then men would go on without limit, and so
the desire would be unsatisfied and fruitless), this plainly must be the
Chief Good, _i.e._ the best thing of all.

Surely then, even with reference to actual life and conduct, the
knowledge of it must have great weight; and like archers, with a mark in
view, we shall be more likely to hit upon what is right: and if so, we
ought to try to describe, in outline at least, what it is and of which
of the sciences and faculties it is the End.

[Sidenote: 1094b] Now one would naturally suppose it to be the End
of that which is most commanding and most inclusive: and to this
description, [Greek: _politikae_] plainly answers: for this it is that
determines which of the sciences should be in the communities, and which
kind individuals are to learn, and what degree of proficiency is to be
required. Again; we see also ranging under this the most highly esteemed
faculties, such as the art military, and that of domestic management,
and Rhetoric. Well then, since this uses all the other practical
sciences, and moreover lays down rules as to what men are to do, and
from what to abstain, the End of this must include the Ends of the rest,
and so must be _The Good_ of Man. And grant that this is the same to
the individual and to the community, yet surely that of the latter is
plainly greater and more perfect to discover and preserve: for to do
this even for a single individual were a matter for contentment; but to
do it for a whole nation, and for communities generally, were more noble
and godlike.


[Sidenote: III] Such then are the objects proposed by our treatise,
which is of the nature of [Greek: _politikae_]: and I conceive I shall
have spoken on them satisfactorily, if they be made as distinctly clear
as the nature of the subject-matter will admit: for exactness must not
be looked for in all discussions alike, any more than in all works
of handicraft. Now the notions of nobleness and justice, with the
examination of which _politikea_ is concerned, admit of variation
and error to such a degree, that they are supposed by some to exist
conventionally only, and not in the nature of things: but then, again,
the things which are allowed to be goods admit of a similar error,
because harm cornes to many from them: for before now some have perished
through wealth, and others through valour.

We must be content then, in speaking of such things and from such data,
to set forth the truth roughly and in outline; in other words, since
we are speaking of general matter and from general data, to draw also
conclusions merely general. And in the same spirit should each person
receive what we say: for the man of education will seek exactness so far
in each subject as the nature of the thing admits, it being plainly much
the same absurdity to put up with a mathematician who tries to persuade
instead of proving, and to demand strict demonstrative reasoning of a
Rhetorician.

[Sidenote: 1095a] Now each man judges well what he knows, and of these
things he is a good judge: on each particular matter then he is a good
judge who has been instructed in _it_, and in a general way the man of
general mental cultivation.

Hence the young man is not a fit student of Moral Philosophy, for he has
no experience in the actions of life, while all that is said presupposes
and is concerned with these: and in the next place, since he is apt to
follow the impulses of his passions, he will hear as though he heard
not, and to no profit, the end in view being practice and not mere
knowledge.

And I draw no distinction between young in years, and youthful in temper
and disposition: the defect to which I allude being no direct result of
the time, but of living at the beck and call of passion, and following
each object as it rises. For to them that are such the knowledge comes
to be unprofitable, as to those of imperfect self-control: but, to
those who form their desires and act in accordance with reason, to have
knowledge on these points must be very profitable.

Let thus much suffice by way of preface on these three points, the
student, the spirit in which our observations should be received, and
the object which we propose.

[Sidenote: IV] And now, resuming the statement with which we commenced,
since all knowledge and moral choice grasps at good of some kind or
another, what good is that which we say [Greek: _politikai_] aims at?
or, in other words, what is the highest of all the goods which are the
objects of action?

So far as name goes, there is a pretty general agreement: for HAPPINESS
both the multitude and the refined few call it, and "living well" and
"doing well" they conceive to be the same with "being happy;" but about
the Nature of this Happiness, men dispute, and the multitude do not in
their account of it agree with the wise. For some say it is some one of
those things which are palpable and apparent, as pleasure or wealth or
honour; in fact, some one thing, some another; nay, oftentimes the same
man gives a different account of it; for when ill, he calls it health;
when poor, wealth: and conscious of their own ignorance, men admire
those who talk grandly and above their comprehension. Some again held it
to be something by itself, other than and beside these many good things,
which is in fact to all these the cause of their being good.

Now to sift all the opinions would be perhaps rather a fruitless task;
so it shall suffice to sift those which are most generally current, or
are thought to have some reason in them.

[Sidenote: 1095b] And here we must not forget the difference between
reasoning from principles, and reasoning to principles: for with good
cause did Plato too doubt about this, and inquire whether the right road
is from principles or to principles, just as in the racecourse from the
judges to the further end, or _vice versâ_.

Of course, we must begin with what is known; but then this is of two
kinds, what we _do_ know, and what we _may_ know: perhaps then as
individuals we must begin with what we _do_ know. Hence the necessity
that he should have been well trained in habits, who is to study, with
any tolerable chance of profit, the principles of nobleness and justice
and moral philosophy generally. For a principle is a matter of fact,
and if the fact is sufficiently clear to a man there will be no need in
addition of the reason for the fact. And he that has been thus trained
either has principles already, or can receive them easily: as for him
who neither has nor can receive them, let him hear his sentence from
Hesiod:

  He is best of all who of himself conceiveth all things;
  Good again is he too who can adopt a good suggestion;
  But whoso neither of himself conceiveth nor hearing from
  another
  Layeth it to heart;--he is a useless man.

[Sidenote: V] But to return from this digression.

Now of the Chief Good (_i.e._ of Happiness) men seem to form their
notions from the different modes of life, as we might naturally expect:
the many and most low conceive it to be pleasure, and hence they are
content with the life of sensual enjoyment. For there are three lines of
life which stand out prominently to view: that just mentioned, and the
life in society, and, thirdly, the life of contemplation.

Now the many are plainly quite slavish, choosing a life like that of
brute animals: yet they obtain some consideration, because many of the
great share the tastes of Sardanapalus. The refined and active again
conceive it to be honour: for this may be said to be the end of the life
in society: yet it is plainly too superficial for the object of our
search, because it is thought to rest with those who pay rather than
with him who receives it, whereas the Chief Good we feel instinctively
must be something which is our own, and not easily to be taken from us.

And besides, men seem to pursue honour, that they may *[Sidenote: 1096a]
believe themselves to be good: for instance, they seek to be honoured
by the wise, and by those among whom they are known, and for virtue:
clearly then, in the opinion at least of these men, virtue is higher
than honour. In truth, one would be much more inclined to think this
to be the end of the life in society; yet this itself is plainly not
sufficiently final: for it is conceived possible, that a man possessed
of virtue might sleep or be inactive all through his life, or, as a
third case, suffer the greatest evils and misfortunes: and the man who
should live thus no one would call happy, except for mere disputation's
sake.

And for these let thus much suffice, for they have been treated of at
sufficient length in my Encyclia.

A third line of life is that of contemplation, concerning which we shall
make our examination in the sequel.

As for the life of money-making, it is one of constraint, and wealth
manifestly is not the good we are seeking, because it is for use, that
is, for the sake of something further: and hence one would rather
conceive the forementioned ends to be the right ones, for men rest
content with them for their own sakes. Yet, clearly, they are not the
objects of our search either, though many words have been wasted on
them. So much then for these.

[Sidenote: VI] Again, the notion of one Universal Good (the same, that
is, in all things), it is better perhaps we should examine, and discuss
the meaning of it, though such an inquiry is unpleasant, because they
are friends of ours who have introduced these [Greek: _eidae_]. Still
perhaps it may appear better, nay to be our duty where the safety of the
truth is concerned, to upset if need be even our own theories, specially
as we are lovers of wisdom: for since both are dear to us, we are bound
to prefer the truth. Now they who invented this doctrine of [Greek:
_eidae_], did not apply it to those things in which they spoke of
priority and posteriority, and so they never made any [Greek: _idea_] of
numbers; but good is predicated in the categories of Substance, Quality,
and Relation; now that which exists of itself, _i.e._ Substance, is
prior in the nature of things to that which is relative, because this
latter is an off-shoot, as it were, and result of that which is; on
their own principle then there cannot be a common [Greek: _idea_] in the
case of these.

In the next place, since good is predicated in as many ways as there are
modes of existence [for it is predicated in the category of Substance,
as God, Intellect--and in that of Quality, as The Virtues--and in that
of Quantity, as The Mean--and in that of Relation, as The Useful--and in
that of Time, as Opportunity--and in that of Place, as Abode; and
other such like things], it manifestly cannot be something common and
universal and one in all: else it would not have been predicated in all
the categories, but in one only.

[Sidenote: 1096b] Thirdly, since those things which range under one
[Greek: _idea_] are also under the cognisance of one science, there
would have been, on their theory, only one science taking cognisance of
all goods collectively: but in fact there are many even for those which
range under one category: for instance, of Opportunity or Seasonableness
(which I have before mentioned as being in the category of Time), the
science is, in war, generalship; in disease, medical science; and of the
Mean (which I quoted before as being in the category of Quantity), in
food, the medical science; and in labour or exercise, the gymnastic
science. A person might fairly doubt also what in the world they mean by
very-this that or the other, since, as they would themselves allow, the
account of the humanity is one and the same in the very-Man, and in any
individual Man: for so far as the individual and the very-Man are both
Man, they will not differ at all: and if so, then very-good and any
particular good will not differ, in so far as both are good. Nor will it
do to say, that the eternity of the very-good makes it to be more good;
for what has lasted white ever so long, is no whiter than what lasts but
for a day.

No. The Pythagoreans do seem to give a more credible account of the
matter, who place "One" among the goods in their double list of goods
and bads: which philosophers, in fact, Speusippus seems to have
followed.

But of these matters let us speak at some other time. Now there is
plainly a loophole to object to what has been advanced, on the plea that
the theory I have attacked is not by its advocates applied to all good:
but those goods only are spoken of as being under one [Greek: idea],
which are pursued, and with which men rest content simply for their own
sakes: whereas those things which have a tendency to produce or preserve
them in any way, or to hinder their contraries, are called good because
of these other goods, and after another fashion. It is manifest then
that the goods may be so called in two senses, the one class for their
own sakes, the other because of these.

Very well then, let us separate the independent goods from the
instrumental, and see whether they are spoken of as under one [Greek:
idea]. But the question next arises, what kind of goods are we to call
independent? All such as are pursued even when separated from other
goods, as, for instance, being wise, seeing, and certain pleasures and
honours (for these, though we do pursue them with some further end in
view, one would still place among the independent goods)? or does it
come in fact to this, that we can call nothing independent good except
the [Greek: idea], and so the concrete of it will be nought?

If, on the other hand, these are independent goods, then we shall
require that the account of the goodness be the same clearly in all,
just as that of the whiteness is in snow and white lead. But how stands
the fact? Why of honour and wisdom and pleasure the accounts are
distinct and different in so far as they are good. The Chief Good then
is not something common, and after one [Greek: idea].

But then, how does the name come to be common (for it is not seemingly a
case of fortuitous equivocation)? Are different individual things called
good by virtue of being from one source, or all conducing to one end, or
rather by way of analogy, for that intellect is to the soul as sight to
the body, and so on? However, perhaps we ought to leave these questions
now, for an accurate investigation of them is more properly the business
of a different philosophy. And likewise respecting the [Greek: idea]:
for even if there is some one good predicated in common of all things
that are good, or separable and capable of existing independently,
manifestly it cannot be the object of human action or attainable by Man;
but we are in search now of something that is so.

It may readily occur to any one, that it would be better to attain a
knowledge of it with a view to such concrete goods as are attainable and
practical, because, with this as a kind of model in our hands, we shall
the better know what things are good for us individually, and when we
know them, we shall attain them.

Some plausibility, it is true, this argument possesses, but it is
contradicted by the facts of the Arts and Sciences; for all these,
though aiming at some good, and seeking that which is deficient, yet
pretermit the knowledge of it: now it is not exactly probable that all
artisans without exception should be ignorant of so great a help as this
would be, and not even look after it; neither is it easy to see wherein
a weaver or a carpenter will be profited in respect of his craft by
knowing the very-good, or how a man will be the more apt to effect cures
or to command an army for having seen the [Greek: idea] itself. For
manifestly it is not health after this general and abstract fashion
which is the subject of the physician's investigation, but the health
of Man, or rather perhaps of this or that man; for he has to heal
individuals.--Thus much on these points.


VII

And now let us revert to the Good of which we are in search: what can it
be? for manifestly it is different in different actions and arts: for it
is different in the healing art and in the art military, and similarly
in the rest. What then is the Chief Good in each? Is it not "that for
the sake of which the other things are done?" and this in the healing
art is health, and in the art military victory, and in that of
house-building a house, and in any other thing something else; in short,
in every action and moral choice the End, because in all cases men do
everything else with a view to this. So that if there is some one End of
all things which are and may be done, this must be the Good proposed by
doing, or if more than one, then these.

Thus our discussion after some traversing about has come to the same
point which we reached before. And this we must try yet more to clear
up.

Now since the ends are plainly many, and of these we choose some with
a view to others (wealth, for instance, musical instruments, and, in
general, all instruments), it is clear that all are not final: but the
Chief Good is manifestly something final; and so, if there is some one
only which is final, this must be the object of our search: but if
several, then the most final of them will be it.

Now that which is an object of pursuit in itself we call more final than
that which is so with a view to something else; that again which is
never an object of choice with a view to something else than those which
are so both in themselves and with a view to this ulterior object: and
so by the term "absolutely final," we denote that which is an object of
choice always in itself, and never with a view to any other.

And of this nature Happiness is mostly thought to be, for this we choose
always for its own sake, and never with a view to anything further:
whereas honour, pleasure, intellect, in fact every excellence we choose
for their own sakes, it is true (because we would choose each of these
even if no result were to follow), but we choose them also with a view
to happiness, conceiving that through their instrumentality we shall be
happy: but no man chooses happiness with a view to them, nor in fact
with a view to any other thing whatsoever.

The same result is seen to follow also from the notion of
self-sufficiency, a quality thought to belong to the final good. Now
by sufficient for Self, we mean not for a single individual living a
solitary life, but for his parents also and children and wife, and,
in general, friends and countrymen; for man is by nature adapted to a
social existence. But of these, of course, some limit must be fixed: for
if one extends it to parents and descendants and friends' friends,
there is no end to it. This point, however, must be left for future
investigation: for the present we define that to be self-sufficient
"which taken alone makes life choice-worthy, and to be in want of
nothing;" now of such kind we think Happiness to be: and further, to
be most choice-worthy of all things; not being reckoned with any other
thing, for if it were so reckoned, it is plain we must then allow it,
with the addition of ever so small a good, to be more choice-worthy than
it was before: because what is put to it becomes an addition of so much
more good, and of goods the greater is ever the more choice-worthy.

So then Happiness is manifestly something final and self-sufficient,
being the end of all things which are and may be done.

But, it may be, to call Happiness the Chief Good is a mere truism, and
what is wanted is some clearer account of its real nature. Now this
object may be easily attained, when we have discovered what is the work
of man; for as in the case of flute-player, statuary, or artisan of any
kind, or, more generally, all who have any work or course of action,
their Chief Good and Excellence is thought to reside in their work, so
it would seem to be with man, if there is any work belonging to him.

Are we then to suppose, that while carpenter and cobbler have certain
works and courses of action, Man as Man has none, but is left by Nature
without a work? or would not one rather hold, that as eye, hand, and
foot, and generally each of his members, has manifestly some special
work; so too the whole Man, as distinct from all these, has some work of
his own?

What then can this be? not mere life, because that plainly is shared
with him even by vegetables, and we want what is peculiar to him. We
must separate off then the life of mere nourishment and growth, and next
will come the life of sensation: but this again manifestly is common to
horses, oxen, and every animal. There remains then a kind of life of
the Rational Nature apt to act: and of this Nature there are two parts
denominated Rational, the one as being obedient to Reason, the other as
having and exerting it. Again, as this life is also spoken of in two
ways, we must take that which is in the way of actual working, because
this is thought to be most properly entitled to the name. If then the
work of Man is a working of the soul in accordance with reason, or at
least not independently of reason, and we say that the work of any given
subject, and of that subject good of its kind, are the same in kind (as,
for instance, of a harp-player and a good harp-player, and so on in
every case, adding to the work eminence in the way of excellence; I
mean, the work of a harp-player is to play the harp, and of a good
harp-player to play it well); if, I say, this is so, and we assume the
work of Man to be life of a certain kind, that is to say a working of
the soul, and actions with reason, and of a good man to do these things
well and nobly, and in fact everything is finished off well in the way
of the excellence which peculiarly belongs to it: if all this is so,
then the Good of Man comes to be "a working of the Soul in the way of
Excellence," or, if Excellence admits of degrees, in the way of the best
and most perfect Excellence.

And we must add, in a complete life; for as it is not one swallow or one
fine day that makes a spring, so it is not one day or a short time that
makes a man blessed and happy.

Let this then be taken for a rough sketch of the Chief Good: since it
is probably the right way to give first the outline, and fill it in
afterwards. And it would seem that any man may improve and connect
what is good in the sketch, and that time is a good discoverer and
co-operator in such matters: it is thus in fact that all improvements
in the various arts have been brought about, for any man may fill up a
deficiency.

You must remember also what has been already stated, and not seek
for exactness in all matters alike, but in each according to the
subject-matter, and so far as properly belongs to the system. The
carpenter and geometrician, for instance, inquire into the right line in
different fashion: the former so far as he wants it for his work, the
latter inquires into its nature and properties, because he is concerned
with the truth.

So then should one do in other matters, that the incidental matters may
not exceed the direct ones.

And again, you must not demand the reason either in all things
alike, because in some it is sufficient that the fact has been well
demonstrated, which is the case with first principles; and the fact is
the first step, _i.e._ starting-point or principle.

And of these first principles some are obtained by induction, some by
perception, some by a course of habituation, others in other different
ways. And we must try to trace up each in their own nature, and take
pains to secure their being well defined, because they have
great influence on what follows: it is thought, I mean, that the
starting-point or principle is more than half the whole matter, and that
many of the points of inquiry come simultaneously into view thereby.


VIII

We must now inquire concerning Happiness, not only from our conclusion
and the data on which our reasoning proceeds, but likewise from what
is commonly said about it: because with what is true all things which
really are are in harmony, but with that which is false the true very
soon jars.

Now there is a common division of goods into three classes; one being
called external, the other two those of the soul and body respectively,
and those belonging to the soul we call most properly and specially
good. Well, in our definition we assume that the actions and workings of
the soul constitute Happiness, and these of course belong to the soul.
And so our account is a good one, at least according to this opinion,
which is of ancient date, and accepted by those who profess philosophy.
Rightly too are certain actions and workings said to be the end, for
thus it is brought into the number of the goods of the soul instead of
the external. Agreeing also with our definition is the common notion,
that the happy man lives well and does well, for it has been stated by
us to be pretty much a kind of living well and doing well.

But further, the points required in Happiness are found in combination
in our account of it.

For some think it is virtue, others practical wisdom, others a kind of
scientific philosophy; others that it is these, or else some one of
them, in combination with pleasure, or at least not independently of it;
while others again take in external prosperity.

Of these opinions, some rest on the authority of numbers or antiquity,
others on that of few, and those men of note: and it is not likely that
either of these classes should be wrong in all points, but be right at
least in some one, or even in most.

Now with those who assert it to be Virtue (Excellence), or some kind of
Virtue, our account agrees: for working in the way of Excellence surely
belongs to Excellence.

And there is perhaps no unimportant difference between conceiving of
the Chief Good as in possession or as in use, in other words, as a mere
state or as a working. For the state or habit may possibly exist in a
subject without effecting any good, as, for instance, in him who is
asleep, or in any other way inactive; but the working cannot so, for it
will of necessity act, and act well. And as at the Olympic games it is
not the finest and strongest men who are crowned, but they who enter the
lists, for out of these the prize-men are selected; so too in life, of
the honourable and the good, it is they who act who rightly win the
prizes.

Their life too is in itself pleasant: for the feeling of pleasure is a
mental sensation, and that is to each pleasant of which he is said to be
fond: a horse, for instance, to him who is fond of horses, and a sight
to him who is fond of sights: and so in like manner just acts to him who
is fond of justice, and more generally the things in accordance with
virtue to him who is fond of virtue. Now in the case of the multitude of
men the things which they individually esteem pleasant clash, because
they are not such by nature, whereas to the lovers of nobleness those
things are pleasant which are such by nature: but the actions in
accordance with virtue are of this kind, so that they are pleasant both
to the individuals and also in themselves.

So then their life has no need of pleasure as a kind of additional
appendage, but involves pleasure in itself. For, besides what I have
just mentioned, a man is not a good man at all who feels no pleasure in
noble actions, just as no one would call that man just who does not feel
pleasure in acting justly, or liberal who does not in liberal actions,
and similarly in the case of the other virtues which might be
enumerated: and if this be so, then the actions in accordance with
virtue must be in themselves pleasurable. Then again they are certainly
good and noble, and each of these in the highest degree; if we are to
take as right the judgment of the good man, for he judges as we have
said.

Thus then Happiness is most excellent, most noble, and most pleasant,
and these attributes are not separated as in the well-known Delian
inscription--

"Most noble is that which is most just, but best is health; And
naturally most pleasant is the obtaining one's desires."

For all these co-exist in the best acts of working: and we say that
Happiness is these, or one, that is, the best of them.

Still it is quite plain that it does require the addition of external
goods, as we have said: because without appliances it is impossible, or
at all events not easy, to do noble actions: for friends, money, and
political influence are in a manner instruments whereby many things
are done: some things there are again a deficiency in which mars
blessedness; good birth, for instance, or fine offspring, or even
personal beauty: for he is not at all capable of Happiness who is very
ugly, or is ill-born, or solitary and childless; and still less perhaps
supposing him to have very bad children or friends, or to have lost good
ones by death. As we have said already, the addition of prosperity of
this kind does seem necessary to complete the idea of Happiness; hence
some rank good fortune, and others virtue, with Happiness.

And hence too a question is raised, whether it is a thing that can be
learned, or acquired by habituation or discipline of some other kind, or
whether it comes in the way of divine dispensation, or even in the way
of chance.

Now to be sure, if anything else is a gift of the Gods to men, it is
probable that Happiness is a gift of theirs too, and specially because
of all human goods it is the highest. But this, it may be, is a question
belonging more properly to an investigation different from ours: and it
is quite clear, that on the supposition of its not being sent from the
Gods direct, but coming to us by reason of virtue and learning of a
certain kind, or discipline, it is yet one of the most Godlike things;
because the prize and End of virtue is manifestly somewhat most
excellent, nay divine and blessed.

It will also on this supposition be widely participated, for it may
through learning and diligence of a certain kind exist in all who have
not been maimed for virtue.

And if it is better we should be happy thus than as a result of chance,
this is in itself an argument that the case is so; because those things
which are in the way of nature, and in like manner of art, and of every
cause, and specially the best cause, are by nature in the best way
possible: to leave them to chance what is greatest and most noble would
be very much out of harmony with all these facts.

The question may be determined also by a reference to our definition of
Happiness, that it is a working of the soul in the way of excellence or
virtue of a certain kind: and of the other goods, some we must have to
begin with, and those which are co-operative and useful are given by
nature as instruments.

These considerations will harmonise also with what we said at the
commencement: for we assumed the End of [Greek Text: poletikae] to be
most excellent: now this bestows most care on making the members of the
community of a certain character; good that is and apt to do what is
honourable.

With good reason then neither ox nor horse nor any other brute animal
do we call happy, for none of them can partake in such working: and for
this same reason a child is not happy either, because by reason of his
tender age he cannot yet perform such actions: if the term is applied,
it is by way of anticipation.

For to constitute Happiness, there must be, as we have said, complete
virtue and a complete life: for many changes and chances of all kinds
arise during a life, and he who is most prosperous may become involved
in great misfortunes in his old age, as in the heroic poems the tale is
told of Priam: but the man who has experienced such fortune and died in
wretchedness, no man calls happy.

Are we then to call no man happy while he lives, and, as Solon would
have us, look to the end? And again, if we are to maintain this
position, is a man then happy when he is dead? or is not this a complete
absurdity, specially in us who say Happiness is a working of a certain
kind?

If on the other hand we do not assert that the dead man is happy, and
Solon does not mean this, but only that one would then be safe in
pronouncing a man happy, as being thenceforward out of the reach of
evils and misfortunes, this too admits of some dispute, since it is
thought that the dead has somewhat both of good and evil (if, as we must
allow, a man may have when alive but not aware of the circumstances),
as honour and dishonour, and good and bad fortune of children and
descendants generally.

Nor is this view again without its difficulties: for, after a man has
lived in blessedness to old age and died accordingly, many changes may
befall him in right of his descendants; some of them may be good and
obtain positions in life accordant to their merits, others again quite
the contrary: it is plain too that the descendants may at different
intervals or grades stand in all manner of relations to the ancestors.
Absurd indeed would be the position that even the dead man is to change
about with them and become at one time happy and at another miserable.
Absurd however it is on the other hand that the affairs of the
descendants should in no degree and during no time affect the ancestors.

But we must revert to the point first raised, since the present question
will be easily determined from that.

If then we are to look to the end and then pronounce the man blessed,
not as being so but as having been so at some previous time, surely it
is absurd that when he _is_ happy the truth is not to be asserted of
him, because we are unwilling to pronounce the living happy by reason of
their liability to changes, and because, whereas we have conceived of
happiness as something stable and no way easily changeable, the fact is
that good and bad fortune are constantly circling about the same people:
for it is quite plain, that if we are to depend upon the fortunes of
men, we shall often have to call the same man happy, and a little while
after miserable, thus representing our happy man

  "Chameleon-like, and based on rottenness."

Is not this the solution? that to make our sentence dependent on the
changes of fortune, is no way right: for not in them stands the well, or
the ill, but though human life needs these as accessories (which we have
allowed already), the workings in the way of virtue are what determine
Happiness, and the contrary the contrary.

And, by the way, the question which has been here discussed, testifies
incidentally to the truth of our account of Happiness. For to nothing
does a stability of human results attach so much as it does to the
workings in the way of virtue, since these are held to be more abiding
even than the sciences: and of these last again the most precious
are the most abiding, because the blessed live in them most and most
continuously, which seems to be the reason why they are not forgotten.
So then this stability which is sought will be in the happy man, and
he will be such through life, since always, or most of all, he will be
doing and contemplating the things which are in the way of virtue: and
the various chances of life he will bear most nobly, and at all times
and in all ways harmoniously, since he is the truly good man, or in the
terms of our proverb "a faultless cube."

And whereas the incidents of chance are many, and differ in greatness
and smallness, the small pieces of good or ill fortune evidently do not
affect the balance of life, but the great and numerous, if happening for
good, will make life more blessed (for it is their nature to contribute
to ornament, and the using of them comes to be noble and excellent), but
if for ill, they bruise as it were and maim the blessedness: for they
bring in positive pain, and hinder many acts of working. But still, even
in these, nobleness shines through when a man bears contentedly many and
great mischances not from insensibility to pain but because he is noble
and high-spirited.

And if, as we have said, the acts of working are what determine the
character of the life, no one of the blessed can ever become wretched,
because he will never do those things which are hateful and mean. For
the man who is truly good and sensible bears all fortunes, we presume,
becomingly, and always does what is noblest under the circumstances,
just as a good general employs to the best advantage the force he has
with him; or a good shoemaker makes the handsomest shoe he can out
of the leather which has been given him; and all other good artisans
likewise. And if this be so, wretched never can the happy man come to
be: I do not mean to say he will be blessed should he fall into fortunes
like those of Priam.

Nor, in truth, is he shifting and easily changeable, for on the one
hand from his happiness he will not be shaken easily nor by ordinary
mischances, but, if at all, by those which are great and numerous; and,
on the other, after such mischances he cannot regain his happiness in a
little time; but, if at all, in a long and complete period, during which
he has made himself master of great and noble things.

Why then should we not call happy the man who works in the way of
perfect virtue, and is furnished with external goods sufficient for
acting his part in the drama of life: and this during no ordinary period
but such as constitutes a complete life as we have been describing it.

Or we must add, that not only is he to live so, but his death must be in
keeping with such life, since the future is dark to us, and Happiness we
assume to be in every way an end and complete. And, if this be so, we
shall call them among the living blessed who have and will have the
things specified, but blessed _as Men_.

On these points then let it suffice to have denned thus much.


XI

Now that the fortunes of their descendants, and friends generally,
contribute nothing towards forming the condition of the dead, is plainly
a very heartless notion, and contrary to the current opinions.

But since things which befall are many, and differ in all kinds of ways,
and some touch more nearly, others less, to go into minute particular
distinctions would evidently be a long and endless task: and so it may
suffice to speak generally and in outline.

If then, as of the misfortunes which happen to one's self, some have a
certain weight and turn the balance of life, while others are, so to
speak, lighter; so it is likewise with those which befall all our
friends alike; if further, whether they whom each suffering befalls
be alive or dead makes much more difference than in a tragedy the
presupposing or actual perpetration of the various crimes and horrors,
we must take into our account this difference also, and still more
perhaps the doubt concerning the dead whether they really partake of any
good or evil; it seems to result from all these considerations, that if
anything does pierce the veil and reach them, be the same good or bad,
it must be something trivial and small, either in itself or to them; or
at least of such a magnitude or such a kind as neither to make happy
them that are not so otherwise, nor to deprive of their blessedness them
that are.

It is plain then that the good or ill fortunes of their friends do
affect the dead somewhat: but in such kind and degree as neither to make
the happy unhappy nor produce any other such effect.


XII

Having determined these points, let us examine with respect to
Happiness, whether it belongs to the class of things praiseworthy or
things precious; for to that of faculties it evidently does not.

Now it is plain that everything which is a subject of praise is praised
for being of a certain kind and bearing a certain relation to something
else: for instance, the just, and the valiant, and generally the good
man, and virtue itself, we praise because of the actions and the
results: and the strong man, and the quick runner, and so forth, we
praise for being of a certain nature and bearing a certain relation to
something good and excellent (and this is illustrated by attempts to
praise the gods; for they are presented in a ludicrous aspect by being
referred to our standard, and this results from the fact, that all
praise does, as we have said, imply reference to a standard). Now if
it is to such objects that praise belongs, it is evident that what is
applicable to the best objects is not praise, but something higher and
better: which is plain matter of fact, for not only do we call the gods
blessed and happy, but of men also we pronounce those blessed who most
nearly resemble the gods. And in like manner in respect of goods; no man
thinks of praising Happiness as he does the principle of justice, but
calls it blessed, as being somewhat more godlike and more excellent.

Eudoxus too is thought to have advanced a sound argument in support of
the claim of pleasure to the highest prize: for the fact that, though it
is one of the good things, it is not praised, he took for an indication
of its superiority to those which are subjects of praise: a superiority
he attributed also to a god and the Chief Good, on the ground that they
form the standard to which everything besides is referred. For praise
applies to virtue, because it makes men apt to do what is noble; but
encomia to definite works of body or mind.

However, it is perhaps more suitable to a regular treatise on encomia to
pursue this topic with exactness: it is enough for our purpose that from
what has been said it is evident that Happiness belongs to the class of
things precious and final. And it seems to be so also because of its
being a starting-point; which it is, in that with a view to it we all do
everything else that is done; now the starting-point and cause of good
things we assume to be something precious and divine.


XIII

Moreover, since Happiness is a kind of working of the soul in the way
of perfect Excellence, we must inquire concerning Excellence: for so
probably shall we have a clearer view concerning Happiness; and again,
he who is really a statesman is generally thought to have spent most
pains on this, for he wishes to make the citizens good and obedient
to the laws. (For examples of this class we have the lawgivers of the
Cretans and Lacedaemonians and whatever other such there have been.)
But if this investigation belongs properly to [Greek: politikae], then
clearly the inquiry will be in accordance with our original design.

Well, we are to inquire concerning Excellence, _i.e._ Human Excellence
of course, because it was the Chief Good of Man and the Happiness of Man
that we were inquiring of just now. By Human Excellence we mean not that
of man's body but that of his soul; for we call Happiness a working of
the Soul.

And if this is so, it is plain that some knowledge of the nature of the
Soul is necessary for the statesman, just as for the Oculist a knowledge
of the whole body, and the more so in proportion as [Greek: politikae]
is more precious and higher than the healing art: and in fact physicians
of the higher class do busy themselves much with the knowledge of the
body.

So then the statesman is to consider the nature of the Soul: but he must
do so with these objects in view, and so far only as may suffice for
the objects of his special inquiry: for to carry his speculations to a
greater exactness is perhaps a task more laborious than falls within his
province.

In fact, the few statements made on the subject in my popular treatises
are quite enough, and accordingly we will adopt them here: as, that
the Soul consists of two parts, the Irrational and the Rational (as to
whether these are actually divided, as are the parts of the body, and
everything that is capable of division; or are only metaphysically
speaking two, being by nature inseparable, as are convex and concave
circumferences, matters not in respect of our present purpose). And of
the Irrational, the one part seems common to other objects, and in fact
vegetative; I mean the cause of nourishment and growth (for such a
faculty of the Soul one would assume to exist in all things that receive
nourishment, even in embryos, and this the same as in the perfect
creatures; for this is more likely than that it should be a different
one).

Now the Excellence of this manifestly is not peculiar to the human
species but common to others: for this part and this faculty is thought
to work most in time of sleep, and the good and bad man are least
distinguishable while asleep; whence it is a common saying that during
one half of life there is no difference between the happy and the
wretched; and this accords with our anticipations, for sleep is an
inactivity of the soul, in so far as it is denominated good or bad,
except that in some wise some of its movements find their way through
the veil and so the good come to have better dreams than ordinary men.
But enough of this: we must forego any further mention of the nutritive
part, since it is not naturally capable of the Excellence which is
peculiarly human.

And there seems to be another Irrational Nature of the Soul, which yet
in a way partakes of Reason. For in the man who controls his appetites,
and in him who resolves to do so and fails, we praise the Reason or
Rational part of the Soul, because it exhorts aright and to the best
course: but clearly there is in them, beside the Reason, some other
natural principle which fights with and strains against the Reason. (For
in plain terms, just as paralysed limbs of the body when their owners
would move them to the right are borne aside in a contrary direction to
the left, so is it in the case of the Soul, for the impulses of men who
cannot control their appetites are to contrary points: the difference is
that in the case of the body we do see what is borne aside but in the
case of the soul we do not. But, it may be, not the less on that account
are we to suppose that there is in the Soul also somewhat besides the
Reason, which is opposed to this and goes against it; as to _how_ it is
different, that is irrelevant.)

But of Reason this too does evidently partake, as we have said: for
instance, in the man of self-control it obeys Reason: and perhaps in
the man of perfected self-mastery, or the brave man, it is yet more
obedient; in them it agrees entirely with the Reason.

So then the Irrational is plainly twofold: the one part, the merely
vegetative, has no share of Reason, but that of desire, or appetition
generally, does partake of it in a sense, in so far as it is obedient to
it and capable of submitting to its rule. (So too in common phrase we
say we have [Greek: _logos_] of our father or friends, and this in a
different sense from that in which we say we have [Greek: logos] of
mathematics.)

Now that the Irrational is in some way persuaded by the Reason,
admonition, and every act of rebuke and exhortation indicate. If then we
are to say that this also has Reason, then the Rational, as well as the
Irrational, will be twofold, the one supremely and in itself, the other
paying it a kind of filial regard.

The Excellence of Man then is divided in accordance with this
difference: we make two classes, calling the one Intellectual, and
the other Moral; pure science, intelligence, and practical
wisdom--Intellectual: liberality, and perfected self-mastery--Moral: in
speaking of a man's Moral character, we do not say he is a scientific
or intelligent but a meek man, or one of perfected self-mastery: and we
praise the man of science in right of his mental state; and of these
such as are praiseworthy we call Excellences.




BOOK II

Well: human Excellence is of two kinds, Intellectual and Moral: now the
Intellectual springs originally, and is increased subsequently, from
teaching (for the most part that is), and needs therefore experience
and time; whereas the Moral comes from custom, and so the Greek term
denoting it is but a slight deflection from the term denoting custom in
that language.

From this fact it is plain that not one of the Moral Virtues comes to be
in us merely by nature: because of such things as exist by nature, none
can be changed by custom: a stone, for instance, by nature gravitating
downwards, could never by custom be brought to ascend, not even if one
were to try and accustom it by throwing it up ten thousand times; nor
could file again be brought to descend, nor in fact could anything whose
nature is in one way be brought by custom to be in another. The Virtues
then come to be in us neither by nature, nor in despite of nature, but
we are furnished by nature with a capacity for receiving themu and are
perfected in them through custom.

Again, in whatever cases we get things by nature, we get the faculties
first and perform the acts of working afterwards; an illustration of
which is afforded by the case of our bodily senses, for it was not
from having often seen or heard that we got these senses, but just
the reverse: we had them and so exercised them, but did not have
them because we had exercised them. But the Virtues we get by first
performing single acts of working, which, again, is the case of other
things, as the arts for instance; for what we have to make when we
have learned how, these we learn how to make by making: men come to be
builders, for instance, by building; harp-players, by playing on the
harp: exactly so, by doing just actions we come to be just; by doing the
actions of self-mastery we come to be perfected in self-mastery; and by
doing brave actions brave.

And to the truth of this testimony is borne by what takes place in
communities: because the law-givers make the individual members good men
by habituation, and this is the intention certainly of every law-giver,
and all who do not effect it well fail of their intent; and herein
consists the difference between a good Constitution and a bad.

Again, every Virtue is either produced or destroyed from and by the very
same circumstances: art too in like manner; I mean it is by playing
the harp that both the good and the bad harp-players are formed: and
similarly builders and all the rest; by building well men will become
good builders; by doing it badly bad ones: in fact, if this had not been
so, there would have been no need of instructors, but all men would have
been at once good or bad in their several arts without them.

So too then is it with the Virtues: for by acting in the various
relations in which we are thrown with our fellow men, we come to be,
some just, some unjust: and by acting in dangerous positions and being
habituated to feel fear or confidence, we come to be, some brave, others
cowards.

Similarly is it also with respect to the occasions of lust and anger:
for some men come to be perfected in self-mastery and mild, others
destitute of all self-control and passionate; the one class by behaving
in one way under them, the other by behaving in another. Or, in one
word, the habits are produced from the acts of working like to them: and
so what we have to do is to give a certain character to these particular
acts, because the habits formed correspond to the differences of these.

So then, whether we are accustomed this way or that straight from
childhood, makes not a small but an important difference, or rather I
would say it makes all the difference.


II

Since then the object of the present treatise is not mere speculation,
as it is of some others (for we are inquiring not merely that we may
know what virtue is but that we may become virtuous, else it would have
been useless), we must consider as to the particular actions how we are
to do them, because, as we have just said, the quality of the habits
that shall be formed depends on these.

Now, that we are to act in accordance with Right Reason is a general
maxim, and may for the present be taken for granted: we will speak of it
hereafter, and say both what Right Reason is, and what are its relations
to the other virtues.

[Sidenote: 1104a]

But let this point be first thoroughly understood between us, that all
which can be said on moral action must be said in outline, as it were,
and not exactly: for as we remarked at the commencement, such reasoning
only must be required as the nature of the subject-matter admits of, and
matters of moral action and expediency have no fixedness any more than
matters of health. And if the subject in its general maxims is such,
still less in its application to particular cases is exactness
attainable: because these fall not under any art or system of rules, but
it must be left in each instance to the individual agents to look to the
exigencies of the particular case, as it is in the art of healing,
or that of navigating a ship. Still, though the present subject is
confessedly such, we must try and do what we can for it.

First then this must be noted, that it is the nature of such things to
be spoiled by defect and excess; as we see in the case of health and
strength (since for the illustration of things which cannot be seen we
must use those that can), for excessive training impairs the strength as
well as deficient: meat and drink, in like manner, in too great or too
small quantities, impair the health: while in due proportion they cause,
increase, and preserve it.

Thus it is therefore with the habits of perfected Self-Mastery and
Courage and the rest of the Virtues: for the man who flies from and
fears all things, and never stands up against anything, comes to be a
coward; and he who fears nothing, but goes at everything, comes to be
rash. In like manner too, he that tastes of every pleasure and abstains
from none comes to lose all self-control; while he who avoids all, as
do the dull and clownish, comes as it were to lose his faculties of
perception: that is to say, the habits of perfected Self-Mastery and
Courage are spoiled by the excess and defect, but by the mean state are
preserved.

Furthermore, not only do the origination, growth, and marring of the
habits come from and by the same circumstances, but also the acts of
working after the habits are formed will be exercised on the same: for
so it is also with those other things which are more directly matters of
sight, strength for instance: for this comes by taking plenty of food
and doing plenty of work, and the man who has attained strength is best
able to do these: and so it is with the Virtues, for not only do we by
abstaining from pleasures come to be perfected in Self-Mastery, but when
we have come to be so we can best abstain from them: similarly too with
Courage: for it is by accustoming ourselves to despise objects of fear
and stand up against them that we come to be brave; and [Sidenote(?):
1104_b_] after we have come to be so we shall be best able to stand up
against such objects.

And for a test of the formation of the habits we must [Sidenote(?): III]
take the pleasure or pain which succeeds the acts; for he is perfected
in Self-Mastery who not only abstains from the bodily pleasures but is
glad to do so; whereas he who abstains but is sorry to do it has not
Self-Mastery: he again is brave who stands up against danger, either
with positive pleasure or at least without any pain; whereas he who does
it with pain is not brave.

For Moral Virtue has for its object-matter pleasures and pains, because
by reason of pleasure we do what is bad, and by reason of pain decline
doing what is right (for which cause, as Plato observes, men should have
been trained straight from their childhood to receive pleasure and pain
from proper objects, for this is the right education). Again: since
Virtues have to do with actions and feelings, and on every feeling and
every action pleasure and pain follow, here again is another proof that
Virtue has for its object-matter pleasure and pain. The same is
shown also by the fact that punishments are effected through the
instrumentality of these; because they are of the nature of remedies,
and it is the nature of remedies to be the contraries of the ills they
cure. Again, to quote what we said before: every habit of the Soul by
its very nature has relation to, and exerts itself upon, things of the
same kind as those by which it is naturally deteriorated or improved:
now such habits do come to be vicious by reason of pleasures and pains,
that is, by men pursuing or avoiding respectively, either such as they
ought not, or at wrong times, or in wrong manner, and so forth (for
which reason, by the way, some people define the Virtues as certain
states of impassibility and utter quietude, but they are wrong because
they speak without modification, instead of adding "as they ought," "as
they ought not," and "when," and so on). Virtue then is assumed to be
that habit which is such, in relation to pleasures and pains, as to
effect the best results, and Vice the contrary.

The following considerations may also serve to set this in a clear
light. There are principally three things moving us to choice and three
to avoidance, the honourable, the expedient, the pleasant; and their
three contraries, the dishonourable, the hurtful, and the painful: now
the good man is apt to go right, and the bad man wrong, with respect
to all these of course, but most specially with respect to pleasure:
because not only is this common to him with all animals but also it is
a concomitant of all those things which move to choice, since both the
honourable and the expedient give an impression of pleasure.

[Sidenote: 1105a] Again, it grows up with us all from infancy, and so it
is a hard matter to remove from ourselves this feeling, engrained as it
is into our very life.

Again, we adopt pleasure and pain (some of us more, and some less) as
the measure even of actions: for this cause then our whole business must
be with them, since to receive right or wrong impressions of pleasure
and pain is a thing of no little importance in respect of the actions.
Once more; it is harder, as Heraclitus says, to fight against pleasure
than against anger: now it is about that which is more than commonly
difficult that art comes into being, and virtue too, because in that
which is difficult the good is of a higher order: and so for this
reason too both virtue and moral philosophy generally must wholly busy
themselves respecting pleasures and pains, because he that uses these
well will be good, he that does so ill will be bad.

Let us then be understood to have stated, that Virtue has for its
object-matter pleasures and pains, and that it is either increased or
marred by the same circumstances (differently used) by which it
is originally generated, and that it exerts itself on the same
circumstances out of which it was generated.

Now I can conceive a person perplexed as to the meaning of our
statement, that men must do just actions to become just, and those of
self-mastery to acquire the habit of self-mastery; "for," he would say,
"if men are doing the actions they have the respective virtues already,
just as men are grammarians or musicians when they do the actions of
either art." May we not reply by saying that it is not so even in the
case of the arts referred to: because a man may produce something
grammatical either by chance or the suggestion of another; but then only
will he be a grammarian when he not only produces something grammatical
but does so grammarian-wise, _i.e._ in virtue of the grammatical
knowledge he himself possesses.

Again, the cases of the arts and the virtues are not parallel: because
those things which are produced by the arts have their excellence in
themselves, and it is sufficient therefore [Sidenote: 1105b] that these
when produced should be in a certain state: but those which are produced
in the way of the virtues, are, strictly speaking, actions of a certain
kind (say of Justice or perfected Self-Mastery), not merely if in
themselves they are in a certain state but if also he who does them
does them being himself in a certain state, first if knowing what he is
doing, next if with deliberate preference, and with such preference for
the things' own sake; and thirdly if being himself stable and unapt to
change. Now to constitute possession of the arts these requisites are
not reckoned in, excepting the one point of knowledge: whereas for
possession of the virtues knowledge avails little or nothing, but the
other requisites avail not a little, but, in fact, are all in all, and
these requisites as a matter of fact do come from oftentimes doing the
actions of Justice and perfected Self-Mastery.

The facts, it is true, are called by the names of these habits when they
are such as the just or perfectly self-mastering man would do; but he is
not in possession of the virtues who merely does these facts, but he who
also so does them as the just and self-mastering do them.

We are right then in saying, that these virtues are formed in a man by
his doing the actions; but no one, if he should leave them undone, would
be even in the way to become a good man. Yet people in general do not
perform these actions, but taking refuge in talk they flatter themselves
they are philosophising, and that they will so be good men: acting in
truth very like those sick people who listen to the doctor with great
attention but do nothing that he tells them: just as these then cannot
be well bodily under such a course of treatment, so neither can those be
mentally by such philosophising.

[Sidenote: V] Next, we must examine what Virtue is. Well, since the
things which come to be in the mind are, in all, of three kinds,
Feelings, Capacities, States, Virtue of course must belong to one of the
three classes.

By Feelings, I mean such as lust, anger, fear, confidence, envy, joy,
friendship, hatred, longing, emulation, compassion, in short all such as
are followed by pleasure or pain: by Capacities, those in right of which
we are said to be capable of these feelings; as by virtue of which we
are able to have been made angry, or grieved, or to have compassionated;
by States, those in right of which we are in a certain relation good
or bad to the aforementioned feelings; to having been made angry, for
instance, we are in a wrong relation if in our anger we were too violent
or too slack, but if we were in the happy medium we are in a right
relation to the feeling. And so on of the rest.

Now Feelings neither the virtues nor vices are, because in right of the
Feelings we are not denominated either good or bad, but in right of the
virtues and vices we are.

[Sidenote: 1106_a_] Again, in right of the Feelings we are neither
praised nor blamed (for a man is not commended for being afraid or
being angry, nor blamed for being angry merely but for being so in a
particular way), but in right of the virtues and vices we are.

Again, both anger and fear we feel without moral choice, whereas the
virtues are acts of moral choice, or at least certainly not independent
of it.

Moreover, in right of the Feelings we are said to be moved, but in right
of the virtues and vices not to be moved, but disposed, in a certain
way.

And for these same reasons they are not Capacities, for we are not
called good or bad merely because we are able to feel, nor are we
praised or blamed.

And again, Capacities we have by nature, but we do not come to be good
or bad by nature, as we have said before.

Since then the virtues are neither Feelings nor Capacities, it remains
that they must be States.

[Sidenote: VI] Now what the genus of Virtue is has been said; but we
must not merely speak of it thus, that it is a state but say also what
kind of a state it is. We must observe then that all excellence makes
that whereof it is the excellence both to be itself in a good state and
to perform its work well. The excellence of the eye, for instance, makes
both the eye good and its work also: for by the excellence of the eye
we see well. So too the excellence of the horse makes a horse good, and
good in speed, and in carrying his rider, and standing up against the
enemy. If then this is universally the case, the excellence of Man, i.e.
Virtue, must be a state whereby Man comes to be good and whereby he will
perform well his proper work. Now how this shall be it is true we have
said already, but still perhaps it may throw light on the subject to see
what is its characteristic nature.

In all quantity then, whether continuous or discrete, one may take the
greater part, the less, or the exactly equal, and these either with
reference to the thing itself, or relatively to us: and the exactly
equal is a mean between excess and defect. Now by the mean of the thing,
_i.e._ absolute mean, I denote that which is equidistant from either
extreme (which of course is one and the same to all), and by the mean
relatively to ourselves, that which is neither too much nor too little
for the particular individual. This of course is not one nor the same to
all: for instance, suppose ten is too much and two too little, people
take six for the absolute mean; because it exceeds the smaller sum by
exactly as much as it is itself exceeded by the larger, and this mean is
according to arithmetical proportion.

[Sidenote: 1106_b_] But the mean relatively to ourselves must not be
so found ; for it does not follow, supposing ten minæ is too large a
quantity to eat and two too small, that the trainer will order his man
six; because for the person who is to take it this also may be too much
or too little: for Milo it would be too little, but for a man just
commencing his athletic exercises too much: similarly too of the
exercises themselves, as running or wrestling.

So then it seems every one possessed of skill avoids excess and defect,
but seeks for and chooses the mean, not the absolute but the relative.

Now if all skill thus accomplishes well its work by keeping an eye on
the mean, and bringing the works to this point (whence it is common
enough to say of such works as are in a good state, "one cannot add
to or take ought from them," under the notion of excess or defect
destroying goodness but the mean state preserving it), and good
artisans, as we say, work with their eye on this, and excellence, like
nature, is more exact and better than any art in the world, it must have
an aptitude to aim at the mean.

It is moral excellence, _i.e._ Virtue, of course which I mean, because
this it is which is concerned with feelings and actions, and in these
there can be excess and defect and the mean: it is possible, for
instance, to feel the emotions of fear, confidence, lust, anger,
compassion, and pleasure and pain generally, too much or too little,
and in either case wrongly; but to feel them when we ought, on what
occasions, towards whom, why, and as, we should do, is the mean, or in
other words the best state, and this is the property of Virtue.

In like manner too with respect to the actions, there may be excess and
defect and the mean. Now Virtue is concerned with feelings and actions,
in which the excess is wrong and the defect is blamed but the mean is
praised and goes right; and both these circumstances belong to Virtue.
Virtue then is in a sense a mean state, since it certainly has an
aptitude for aiming at the mean.

Again, one may go wrong in many different ways (because, as the
Pythagoreans expressed it, evil is of the class of the infinite, good
of the finite), but right only in one; and so the former is easy, the
latter difficult; easy to miss the mark, but hard to hit it: and for
these reasons, therefore, both the excess and defect belong to Vice, and
the mean state to Virtue; for, as the poet has it,

  "Men may be bad in many ways,
  But good in one alone."
Virtue then is "a state apt to exercise deliberate choice, being in the
relative mean, determined by reason, and as the man of practical wisdom
would determine."

It is a middle state between too faulty ones, in the way of excess on
one side and of defect on the other: and it is so moreover, because the
faulty states on one side fall short of, and those on the other exceed,
what is right, both in the case of the feelings and the actions; but
Virtue finds, and when found adopts, the mean.

And so, viewing it in respect of its essence and definition, Virtue is a
mean state; but in reference to the chief good and to excellence it is
the highest state possible.

But it must not be supposed that every action or every feeling is
capable of subsisting in this mean state, because some there are
which are so named as immediately to convey the notion of badness, as
malevolence, shamelessness, envy; or, to instance in actions, adultery,
theft, homicide; for all these and suchlike are blamed because they are
in themselves bad, not the having too much or too little of them.

In these then you never can go right, but must always be wrong: nor in
such does the right or wrong depend on the selection of a proper person,
time, or manner (take adultery for instance), but simply doing any one
soever of those things is being wrong.

You might as well require that there should be determined a mean state,
an excess and a defect in respect of acting unjustly, being cowardly, or
giving up all control of the passions: for at this rate there will be
of excess and defect a mean state; of excess, excess; and of defect,
defect.

But just as of perfected self-mastery and courage there is no excess and
defect, because the mean is in one point of view the highest possible
state, so neither of those faulty states can you have a mean state,
excess, or defect, but howsoever done they are wrong: you cannot, in
short, have of excess and defect a mean state, nor of a mean state
excess and defect.


VII

It is not enough, however, to state this in general terms, we must also
apply it to particular instances, because in treatises on moral conduct
general statements have an air of vagueness, but those which go into
detail one of greater reality: for the actions after all must be in
detail, and the general statements, to be worth anything, must hold good
here.

We must take these details then from the Table.

I. In respect of fears and confidence or boldness:

[Sidenote: 1107b]

The Mean state is Courage: men may exceed, of course, either in absence
of fear or in positive confidence: the former has no name (which is a
common case), the latter is called rash: again, the man who has too much
fear and too little confidence is called a coward.

II. In respect of pleasures and pains (but not all, and perhaps fewer
pains than pleasures):

The Mean state here is perfected Self-Mastery, the defect total absence
of Self-control. As for defect in respect of pleasure, there are really
no people who are chargeable with it, so, of course, there is really no
name for such characters, but, as they are conceivable, we will give
them one and call them insensible.

III. In respect of giving and taking wealth (a):

The mean state is Liberality, the excess Prodigality, the defect
Stinginess: here each of the extremes involves really an excess and
defect contrary to each other: I mean, the prodigal gives out too much
and takes in too little, while the stingy man takes in too much and
gives out too little. (It must be understood that we are now giving
merely an outline and summary, intentionally: and we will, in a later
part of the treatise, draw out the distinctions with greater exactness.)

IV. In respect of wealth (b):

There are other dispositions besides these just mentioned; a mean state
called Munificence (for the munificent man differs from the liberal, the
former having necessarily to do with great wealth, the latter with but
small); the excess called by the names either of Want of taste or
Vulgar Profusion, and the defect Paltriness (these also differ from the
extremes connected with liberality, and the manner of their difference
shall also be spoken of later).

V. In respect of honour and dishonour (a):

The mean state Greatness of Soul, the excess which may be called
braggadocio, and the defect Littleness of Soul.

VI. In respect of honour and dishonour (b):

[Sidenote: 1108a]

Now there is a state bearing the same relation to Greatness of Soul as
we said just now Liberality does to Munificence, with the difference
that is of being about a small amount of the same thing: this state
having reference to small honour, as Greatness of Soul to great honour;
a man may, of course, grasp at honour either more than he should or
less; now he that exceeds in his grasping at it is called ambitious, he
that falls short unambitious, he that is just as he should be has no
proper name: nor in fact have the states, except that the disposition of
the ambitious man is called ambition. For this reason those who are in
either extreme lay claim to the mean as a debateable land, and we call
the virtuous character sometimes by the name ambitious, sometimes by
that of unambitious, and we commend sometimes the one and sometimes
the other. Why we do it shall be said in the subsequent part of the
treatise; but now we will go on with the rest of the virtues after the
plan we have laid down.

VII. In respect of anger:

Here too there is excess, defect, and a mean state; but since they
may be said to have really no proper names, as we call the virtuous
character Meek, we will call the mean state Meekness, and of the
extremes, let the man who is excessive be denominated Passionate, and
the faulty state Passionateness, and him who is deficient Angerless, and
the defect Angerlessness.

There are also three other mean states, having some mutual resemblance,
but still with differences; they are alike in that they all have for
their object-matter intercourse of words and deeds, and they differ in
that one has respect to truth herein, the other two to what is pleasant;
and this in two ways, the one in relaxation and amusement, the other in
all things which occur in daily life. We must say a word or two about
these also, that we may the better see that in all matters the mean is
praiseworthy, while the extremes are neither right nor worthy of praise
but of blame.

Now of these, it is true, the majority have really no proper names, but
still we must try, as in the other cases, to coin some for them for the
sake of clearness and intelligibleness.

I. In respect of truth: The man who is in the mean state we will call
Truthful, and his state Truthfulness, and as to the disguise of truth,
if it be on the side of exaggeration, Braggadocia, and him that has it a
Braggadocio; if on that of diminution, Reserve and Reserved shall be the
terms.

II. In respect of what is pleasant in the way of relaxation or
amusement: The mean state shall be called Easy-pleasantry, and the
character accordingly a man of Easy-pleasantry; the excess Buffoonery,
and the man a Buffoon; the man deficient herein a Clown, and his state
Clownishness.

III. In respect of what is pleasant in daily life: He that is as he
should be may be called Friendly, and his mean state Friendliness: he
that exceeds, if it be without any interested motive, somewhat too
Complaisant, if with such motive, a Flatterer: he that is deficient and
in all instances unpleasant, Quarrelsome and Cross.

There are mean states likewise in feelings and matters concerning them.
Shamefacedness, for instance, is no virtue, still a man is praised for
being shamefaced: for in these too the one is denominated the man in the
mean state, the other in the excess; the Dumbfoundered, for instance,
who is overwhelmed with shame on all and any occasions: the man who is
in the defect, _i.e._ who has no shame at all in his composition, is
called Shameless: but the right character Shamefaced.

Indignation against successful vice, again, is a state in the mean
between Envy and Malevolence: they all three have respect to pleasure
and pain produced by what happens to one's neighbour: for the man who
has this right feeling is annoyed at undeserved success of others, while
the envious man goes beyond him and is annoyed at all success of others,
and the malevolent falls so far short of feeling annoyance that he even
rejoices [at misfortune of others].

But for the discussion of these also there will be another opportunity,
as of Justice too, because the term is used in more senses than one. So
after this we will go accurately into each and say how they are mean
states: and in like manner also with respect to the Intellectual
Excellences.

Now as there are three states in each case, two faulty either in the way
of excess or defect, and one right, which is the mean state, of course
all are in a way opposed to one another; the extremes, for instance, not
only to the mean but also to one another, and the mean to the extremes:
for just as the half is greater if compared with the less portion, and
less if compared with the greater, so the mean states, compared with the
defects, exceed, whether in feelings or actions, and _vice versa_. The
brave man, for instance, shows as rash when compared with the coward,
and cowardly when compared with the rash; similarly too the man of
perfected self-mastery, viewed in comparison with the man destitute of
all perception, shows like a man of no self-control, but in comparison
with the man who really has no self-control, he looks like one destitute
of all perception: and the liberal man compared with the stingy seems
prodigal, and by the side of the prodigal, stingy.

And so the extreme characters push away, so to speak, towards each other
the man in the mean state; the brave man is called a rash man by
the coward, and a coward by the rash man, and in the other cases
accordingly. And there being this mutual opposition, the contrariety
between the extremes is greater than between either and the mean,
because they are further from one another than from the mean, just as
the greater or less portion differ more from each other than either from
the exact half.

Again, in some cases an extreme will bear a resemblance to the mean;
rashness, for instance, to courage, and prodigality to liberality; but
between the extremes there is the greatest dissimilarity. Now things
which are furthest from one another are defined to be contrary, and so
the further off the more contrary will they be.

[Sidenote: 1109a] Further: of the extremes in some cases the excess,
and in others the defect, is most opposed to the mean: to courage, for
instance, not rashness which is the excess, but cowardice which is the
defect; whereas to perfected self-mastery not insensibility which is the
defect but absence of all self-control which is the excess.

And for this there are two reasons to be given; one from the nature of
the thing itself, because from the one extreme being nearer and more
like the mean, we do not put this against it, but the other; as, for
instance, since rashness is thought to be nearer to courage than
cowardice is, and to resemble it more, we put cowardice against courage
rather than rashness, because those things which are further from the
mean are thought to be more contrary to it. This then is one reason
arising from the thing itself; there is another arising from our own
constitution and make: for in each man's own case those things give the
impression of being more contrary to the mean to which we individually
have a natural bias. Thus we have a natural bias towards pleasures,
for which reason we are much more inclined to the rejection of all
self-control, than to self-discipline.

These things then to which the bias is, we call more contrary, and so
total want of self-control (the excess) is more contrary than the defect
is to perfected self-mastery.


IX

Now that Moral Virtue is a mean state, and how it is so, and that it
lies between two faulty states, one in the way of excess and another in
the way of defect, and that it is so because it has an aptitude to aim
at the mean both in feelings and actions, all this has been set forth
fully and sufficiently.

And so it is hard to be good: for surely hard it is in each instance to
find the mean, just as to find the mean point or centre of a circle is
not what any man can do, but only he who knows how: just so to be angry,
to give money, and be expensive, is what any man can do, and easy: but
to do these to the right person, in due proportion, at the right time,
with a right object, and in the right manner, this is not as before what
any man can do, nor is it easy; and for this cause goodness is rare, and
praiseworthy, and noble.

Therefore he who aims at the mean should make it his first care to keep
away from that extreme which is more contrary than the other to the
mean; just as Calypso in Homer advises Ulysses,

  "Clear of this smoke and surge thy barque direct;"

because of the two extremes the one is always more, and the other
less, erroneous; and, therefore, since to hit exactly on the mean is
difficult, one must take the least of the evils as the safest plan; and
this a man will be doing, if he follows this method.

[Sidenote: 1109b] We ought also to take into consideration our own
natural bias; which varies in each man's case, and will be ascertained
from the pleasure and pain arising in us. Furthermore, we should force
ourselves off in the contrary direction, because we shall find ourselves
in the mean after we have removed ourselves far from the wrong side,
exactly as men do in straightening bent timber.

But in all cases we must guard most carefully against what is pleasant,
and pleasure itself, because we are not impartial judges of it.

We ought to feel in fact towards pleasure as did the old counsellors
towards Helen, and in all cases pronounce a similar sentence; for so by
sending it away from us, we shall err the less.

Well, to speak very briefly, these are the precautions by adopting which
we shall be best able to attain the mean.

Still, perhaps, after all it is a matter of difficulty, and specially
in the particular instances: it is not easy, for instance, to determine
exactly in what manner, with what persons, for what causes, and for what
length of time, one ought to feel anger: for we ourselves sometimes
praise those who are defective in this feeling, and we call them meek;
at another, we term the hot-tempered manly and spirited.

Then, again, he who makes a small deflection from what is right, be it
on the side of too much or too little, is not blamed, only he who makes
a considerable one; for he cannot escape observation. But to what point
or degree a man must err in order to incur blame, it is not easy to
determine exactly in words: nor in fact any of those points which are
matter of perception by the Moral Sense: such questions are matters of
detail, and the decision of them rests with the Moral Sense.

At all events thus much is plain, that the mean state is in all things
praiseworthy, and that practically we must deflect sometimes towards
excess sometimes towards defect, because this will be the easiest method
of hitting on the mean, that is, on what is right.




BOOK III

I Now since Virtue is concerned with the regulation of feelings and
actions, and praise and blame arise upon such as are voluntary, while
for the involuntary allowance is made, and sometimes compassion is
excited, it is perhaps a necessary task for those who are investigating
the nature of Virtue to draw out the distinction between what is
voluntary and what involuntary; and it is certainly useful for
legislators, with respect to the assigning of honours and punishments.



III

Involuntary actions then are thought to be of two kinds, being
done either on compulsion, or by reason of ignorance. An action is,
properly speaking, compulsory, when the origination is external to the
agent, being such that in it the agent (perhaps we may more properly
say the patient) contributes nothing; as if a wind were to convey you
anywhere, or men having power over your person.

But when actions are done, either from fear of greater evils, or from
some honourable motive, as, for instance, if you were ordered to commit
some base act by a despot who had your parents or children in his power,
and they were to be saved upon your compliance or die upon your refusal,
in such cases there is room for a question whether the actions are
voluntary or involuntary.

A similar question arises with respect to cases of throwing goods
overboard in a storm: abstractedly no man throws away his property
willingly, but with a view to his own and his shipmates' safety any one
would who had any sense.

The truth is, such actions are of a mixed kind, but are most like
voluntary actions; for they are choiceworthy at the time when they are
being done, and the end or object of the action must be taken with
reference to the actual occasion. Further, we must denominate an action
voluntary or involuntary at the time of doing it: now in the given case
the man acts voluntarily, because the originating of the motion of his
limbs in such actions rests with himself; and where the origination is
in himself it rests with himself to do or not to do.

Such actions then are voluntary, though in the abstract perhaps
involuntary because no one would choose any of such things in and by
itself.

But for such actions men sometimes are even praised, as when they endure
any disgrace or pain to secure great and honourable equivalents; if
_vice versâ_, then they are blamed, because it shows a base mind to
endure things very disgraceful for no honourable object, or for a
trifling one.

For some again no praise is given, but allowance is made; as where a
man does what he should not by reason of such things as overstrain the
powers of human nature, or pass the limits of human endurance.

Some acts perhaps there are for which compulsion cannot be pleaded, but
a man should rather suffer the worst and die; how absurd, for instance,
are the pleas of compulsion with which Alcmaeon in Euripides' play
excuses his matricide!

But it is difficult sometimes to decide what kind of thing should be
chosen instead of what, or what endured in preference to what, and much
moreso to abide by one's decisions: for in general the alternatives are
painful, and the actions required are base, and so praise or blame is
awarded according as persons have been compelled or no.

1110b What kind of actions then are to be called compulsory? may we say,
simply and abstractedly whenever the cause is external and the agent
contributes nothing; and that where the acts are in themselves such
as one would not wish but choiceworthy at the present time and in
preference to such and such things, and where the origination rests with
the agent, the actions are in themselves involuntary but at the given
time and in preference to such and such things voluntary; and they are
more like voluntary than involuntary, because the actions consist of
little details, and these are voluntary.

But what kind of things one ought to choose instead of what, it is not
easy to settle, for there are many differences in particular instances.

But suppose a person should say, things pleasant and honourable exert
a compulsive force (for that they are external and do compel); at that
rate every action is on compulsion, because these are universal motives
of action.

Again, they who act on compulsion and against their will do so with
pain; but they who act by reason of what is pleasant or honourable act
with pleasure.

It is truly absurd for a man to attribute his actions to external things
instead of to his own capacity for being easily caught by them; or,
again, to ascribe the honourable to himself, and the base ones to
pleasure.

So then that seems to be compulsory "whose origination is from without,
the party compelled contributing nothing." Now every action of which
ignorance is the cause is not-voluntary, but that only is involuntary
which is attended with pain and remorse; for clearly the man who has
done anything by reason of ignorance, but is not annoyed at his own
action, cannot be said to have done it _with_ his will because he did
not know he was doing it, nor again _against_ his will because he is not
sorry for it.

So then of the class "acting by reason of ignorance," he who feels
regret afterwards is thought to be an involuntary agent, and him that
has no such feeling, since he certainly is different from the other, we
will call a not-voluntary agent; for as there is a real difference it is
better to have a proper name.

Again, there seems to be a difference between acting _because of_
ignorance and acting _with_ ignorance: for instance, we do not usually
assign ignorance as the cause of the actions of the drunken or angry
man, but either the drunkenness or the anger, yet they act not knowingly
but with ignorance.

Again, every bad man is ignorant what he ought to do and what to leave
undone, and by reason of such error men become unjust and wholly evil.

[Sidenote: 1111a] Again, we do not usually apply the term involuntary
when a man is ignorant of his own true interest; because ignorance which
affects moral choice constitutes depravity but not involuntariness: nor
does any ignorance of principle (because for this men are blamed)
but ignorance in particular details, wherein consists the action and
wherewith it is concerned, for in these there is both compassion and
allowance, because he who acts in ignorance of any of them acts in a
proper sense involuntarily.

It may be as well, therefore, to define these particular details; what
they are, and how many; viz. who acts, what he is doing, with respect to
what or in what, sometimes with what, as with what instrument, and with
what result (as that of preservation, for instance), and how, as whether
softly or violently.

All these particulars, in one and the same case, no man in his senses
could be ignorant of; plainly not of the agent, being himself. But
what he is doing a man may be ignorant, as men in speaking say a
thing escaped them unawares; or as Aeschylus did with respect to the
Mysteries, that he was not aware that it was unlawful to speak of them;
or as in the case of that catapult accident the other day the man said
he discharged it merely to display its operation. Or a person might
suppose a son to be an enemy, as Merope did; or that the spear really
pointed was rounded off; or that the stone was a pumice; or in striking
with a view to save might kill; or might strike when merely wishing to
show another, as people do in sham-fighting.

Now since ignorance is possible in respect to all these details in
which the action consists, he that acted in ignorance of any of them is
thought to have acted involuntarily, and he most so who was in ignorance
as regards the most important, which are thought to be those in which
the action consists, and the result.

Further, not only must the ignorance be of this kind, to constitute an
action involuntary, but it must be also understood that the action is
followed by pain and regret.

Now since all involuntary action is either upon compulsion or by reason
of ignorance, Voluntary Action would seem to be "that whose origination
is in the agent, he being aware of the particular details in which the
action consists."

For, it may be, men are not justified by calling those actions
involuntary, which are done by reason of Anger or Lust.

Because, in the first place, if this be so no other animal but man, and
not even children, can be said to act voluntarily. Next, is it meant
that we never act voluntarily when we act from Lust or Anger, or that we
act voluntarily in doing what is right and involuntarily in doing what
is discreditable? The latter supposition is absurd, since the cause
is one and the same. Then as to the former, it is a strange thing to
maintain actions to be involuntary which we are bound to grasp at: now
there are occasions on which anger is a duty, and there are things which
we are bound to lust after, health, for instance, and learning.

Again, whereas actions strictly involuntary are thought to be attended
with pain, those which are done to gratify lust are thought to be
pleasant.

Again: how does the involuntariness make any difference between wrong
actions done from deliberate calculation, and those done by reason of
anger? for both ought to be avoided, and the irrational feelings are
thought to be just as natural to man as reason, and so of course must be
such actions of the individual as are done from Anger and Lust. It is
absurd then to class these actions among the involuntary.

II

Having thus drawn out the distinction between voluntary and involuntary
action our next step is to examine into the nature of Moral Choice,
because this seems most intimately connected with Virtue and to be a
more decisive test of moral character than a man's acts are.

Now Moral Choice is plainly voluntary, but the two are not co-extensive,
voluntary being the more comprehensive term; for first, children and all
other animals share in voluntary action but not in Moral Choice; and
next, sudden actions we call voluntary but do not ascribe them to Moral
Choice.

Nor do they appear to be right who say it is lust or anger, or wish, or
opinion of a certain kind; because, in the first place, Moral Choice is
not shared by the irrational animals while Lust and Anger are. Next; the
man who fails of self-control acts from Lust but not from Moral Choice;
the man of self-control, on the contrary, from Moral Choice, not from
Lust. Again: whereas Lust is frequently opposed to Moral Choice, Lust is
not to Lust.

Lastly: the object-matter of Lust is the pleasant and the painful, but
of Moral Choice neither the one nor the other. Still less can it be
Anger, because actions done from Anger are thought generally to be least
of all consequent on Moral Choice.

Nor is it Wish either, though appearing closely connected with it;
because, in the first place, Moral Choice has not for its objects
impossibilities, and if a man were to say he chose them he would be
thought to be a fool; but Wish may have impossible things for its
objects, immortality for instance.

Wish again may be exercised on things in the accomplishment of which
one's self could have nothing to do, as the success of any particular
actor or athlete; but no man chooses things of this nature, only such as
he believes he may himself be instrumental in procuring.

Further: Wish has for its object the End rather, but Moral Choice the
means to the End; for instance, we wish to be healthy but we choose
the means which will make us so; or happiness again we wish for, and
commonly say so, but to say we choose is not an appropriate term,
because, in short, the province of Moral Choice seems to be those things
which are in our own power.

Neither can it be Opinion; for Opinion is thought to be unlimited in its
range of objects, and to be exercised as well upon things eternal and
impossible as on those which are in our own power: again, Opinion is
logically divided into true and false, not into good and bad as Moral
Choice is.

However, nobody perhaps maintains its identity with Opinion simply; but
it is not the same with opinion of any kind, because by choosing good
and bad things we are constituted of a certain character, but by having
opinions on them we are not.

Again, we choose to take or avoid, and so on, but we opine what a thing
is, or for what it is serviceable, or how; but we do not opine to take
or avoid.

Further, Moral Choice is commended rather for having a right object than
for being judicious, but Opinion for being formed in accordance with
truth.

Again, we choose such things as we pretty well know to be good, but we
form opinions respecting such as we do not know at all.

And it is not thought that choosing and opining best always go together,
but that some opine the better course and yet by reason of viciousness
choose not the things which they should.

It may be urged, that Opinion always precedes or accompanies Moral
Choice; be it so, this makes no difference, for this is not the point in
question, but whether Moral Choice is the same as Opinion of a certain
kind.

Since then it is none of the aforementioned things, what is it, or how
is it characterised? Voluntary it plainly is, but not all voluntary
action is an object of Moral Choice. May we not say then, it is "that
voluntary which has passed through a stage of previous deliberation?"
because Moral Choice is attended with reasoning and intellectual
process. The etymology of its Greek name seems to give a hint of it,
being when analysed "chosen in preference to somewhat else."


III

Well then; do men deliberate about everything, and is anything soever
the object of Deliberation, or are there some matters with respect to
which there is none? (It may be as well perhaps to say, that by "object
of Deliberation" is meant such matter as a sensible man would deliberate
upon, not what any fool or madman might.)

Well: about eternal things no one deliberates; as, for instance, the
universe, or the incommensurability of the diameter and side of a
square.

Nor again about things which are in motion but which always happen in
the same way either necessarily, or naturally, or from some other cause,
as the solstices or the sunrise.

Nor about those which are variable, as drought and rains; nor fortuitous
matters, as finding of treasure.

Nor in fact even about all human affairs; no Lacedæmonian, for instance,
deliberates as to the best course for the Scythian government to adopt;
because in such cases we have no power over the result.

But we do deliberate respecting such practical matters as are in our own
power (which are what are left after all our exclusions).

I have adopted this division because causes seem to be divisible into
nature, necessity, chance, and moreover intellect, and all human powers.

And as man in general deliberates about what man in general can effect,
so individuals do about such practical things as can be effected through
their own instrumentality.

[Sidenote: 1112b] Again, we do not deliberate respecting such arts or
sciences as are exact and independent: as, for instance, about written
characters, because we have no doubt how they should be formed; but we
do deliberate on all buch things as are usually done through our own
instrumentality, but not invariably in the same way; as, for instance,
about matters connected with the healing art, or with money-making; and,
again, more about piloting ships than gymnastic exercises, because the
former has been less exactly determined, and so forth; and more about
arts than sciences, because we more frequently doubt respecting the
former.

So then Deliberation takes place in such matters as are under general
laws, but still uncertain how in any given case they will issue,
_i.e._ in which there is some indefiniteness; and for great matters we
associate coadjutors in counsel, distrusting our ability to settle them
alone.

Further, we deliberate not about Ends, but Means to Ends. No physician,
for instance, deliberates whether he will cure, nor orator whether
he will persuade, nor statesman whether he will produce a good
constitution, nor in fact any man in any other function about his
particular End; but having set before them a certain End they look how
and through what means it may be accomplished: if there is a choice of
means, they examine further which are easiest and most creditable; or,
if there is but one means of accomplishing the object, then how it may
be through this, this again through what, till they come to the first
cause; and this will be the last found; for a man engaged in a process
of deliberation seems to seek and analyse, as a man, to solve a
problem, analyses the figure given him. And plainly not every search is
Deliberation, those in mathematics to wit, but every Deliberation is
a search, and the last step in the analysis is the first in the
constructive process. And if in the course of their search men come upon
an impossibility, they give it up; if money, for instance, be necessary,
but cannot be got: but if the thing appears possible they then attempt
to do it.

And by possible I mean what may be done through our own instrumentality
(of course what may be done through our friends is through our own
instrumentality in a certain sense, because the origination in such
cases rests with us). And the object of search is sometimes the
necessary instruments, sometimes the method of using them; and similarly
in the rest sometimes through what, and sometimes how or through what.

So it seems, as has been said, that Man is the originator of his
actions; and Deliberation has for its object whatever may be done
through one's own instrumentality, and the actions are with a view to
other things; and so it is, not the End, but the Means to Ends on which
Deliberation is employed.

[Sidenote: III3a]

Nor, again, is it employed on matters of detail, as whether the
substance before me is bread, or has been properly cooked; for these
come under the province of sense, and if a man is to be always
deliberating, he may go on _ad infinitum_.

Further, exactly the same matter is the object both of Deliberation
and Moral Choice; but that which is the object of Moral Choice is
thenceforward separated off and definite, because by object of Moral
Choice is denoted that which after Deliberation has been preferred to
something else: for each man leaves off searching how he shall do a
thing when he has brought the origination up to himself, _i.e_. to the
governing principle in himself, because it is this which makes the
choice. A good illustration of this is furnished by the old regal
constitutions which Homer drew from, in which the Kings would announce
to the commonalty what they had determined before.

Now since that which is the object of Moral Choice is something in our
own power, which is the object of deliberation and the grasping of the
Will, Moral Choice must be "a grasping after something in our own power
consequent upon Deliberation:" because after having deliberated we
decide, and then grasp by our Will in accordance with the result of our
deliberation.

Let this be accepted as a sketch of the nature and object of Moral
Choice, that object being "Means to Ends."

[Sidenote: IV] That Wish has for its object-matter the End, has been
already stated; but there are two opinions respecting it; some thinking
that its object is real good, others whatever impresses the mind with a
notion of good.

Now those who maintain that the object of Wish is real good are beset by
this difficulty, that what is wished for by him who chooses wrongly is
not really an object of Wish (because, on their theory, if it is an
object of wish, it must be good, but it is, in the case supposed, evil).
Those who maintain, on the contrary, that that which impresses the mind
with a notion of good is properly the object of Wish, have to meet this
difficulty, that there is nothing naturally an object of Wish but to
each individual whatever seems good to him; now different people have
different notions, and it may chance contrary ones.

But, if these opinions do not satisfy us, may we not say that,
abstractedly and as a matter of objective truth, the really good is the
object of Wish, but to each individual whatever impresses his mind with
the notion of good. And so to the good man that is an object of Wish
which is really and truly so, but to the bad man anything may be; just
as physically those things are wholesome to the healthy which are really
so, but other things to the sick. And so too of bitter and sweet, and
hot and heavy, and so on. For the good man judges in every instance
correctly, and in every instance the notion conveyed to his mind is the
true one.

For there are fair and pleasant things peculiar to, and so varying with,
each state; and perhaps the most distinguishing characteristic of the
good man is his seeing the truth in every instance, he being, in fact,
the rule and measure of these matters.

The multitude of men seem to be deceived by reason of pleasure, because
though it is not really a good it impresses their minds with the notion
of goodness, so they choose what is pleasant as good and avoid pain as
an evil.

Now since the End is the object of Wish, and the means to the End of
Deliberation and Moral Choice, the actions regarding these matters
must be in the way of Moral Choice, _i.e._ voluntary: but the acts of
working out the virtues are such actions, and therefore Virtue is in our
power.

And so too is Vice: because wherever it is in our power to do it is also
in our power to forbear doing, and _vice versâ_: therefore if the doing
(being in a given case creditable) is in our power, so too is the
forbearing (which is in the same case discreditable), and _vice versâ_.

But if it is in our power to do and to forbear doing what is creditable
or the contrary, and these respectively constitute the being good or
bad, then the being good or vicious characters is in our power.

As for the well-known saying, "No man voluntarily is wicked or
involuntarily happy," it is partly true, partly false; for no man is
happy against his will, of course, but wickedness is voluntary. Or must
we dispute the statements lately made, and not say that Man is the
originator or generator of his actions as much as of his children?

But if this is matter of plain manifest fact, and we cannot refer our
actions to any other originations beside those in our own power, those
things must be in our own power, and so voluntary, the originations of
which are in ourselves.

Moreover, testimony seems to be borne to these positions both privately
by individuals, and by law-givers too, in that they chastise and punish
those who do wrong (unless they do so on compulsion, or by reason of
ignorance which is not self-caused), while they honour those who act
rightly, under the notion of being likely to encourage the latter and
restrain the former. But such things as are not in our own power, _i.e._
not voluntary, no one thinks of encouraging us to do, knowing it to be
of no avail for one to have been persuaded not to be hot (for instance),
or feel pain, or be hungry, and so forth, because we shall have those
sensations all the same.

And what makes the case stronger is this: that they chastise for the
very fact of ignorance, when it is thought to be self-caused; to the
drunken, for instance, penalties are double, because the origination in
such case lies in a man's own self: for he might have helped getting
drunk, and this is the cause of his ignorance.

[Sidenote: III4_a_] Again, those also who are ignorant of legal
regulations which they are bound to know, and which are not hard to
know, they chastise; and similarly in all other cases where neglect is
thought to be the cause of the ignorance, under the notion that it was
in their power to prevent their ignorance, because they might have paid
attention.

But perhaps a man is of such a character that he cannot attend to such
things: still men are themselves the causes of having become such
characters by living carelessly, and also of being unjust or destitute
of self-control, the former by doing evil actions, the latter by
spending their time in drinking and such-like; because the particular
acts of working form corresponding characters, as is shown by those who
are practising for any contest or particular course of action, for such
men persevere in the acts of working.

As for the plea, that a man did not know that habits are produced
from separate acts of working, we reply, such ignorance is a mark of
excessive stupidity.

Furthermore, it is wholly irrelevant to say that the man who acts
unjustly or dissolutely does not _wish_ to attain the habits of these
vices: for if a man wittingly does those things whereby he must become
unjust he is to all intents and purposes unjust voluntarily; but he
cannot with a wish cease to be unjust and become just. For, to take the
analogous case, the sick man cannot with a wish be well again, yet in
a supposable case he is voluntarily ill because he has produced his
sickness by living intemperately and disregarding his physicians. There
was a time then when he might have helped being ill, but now he has let
himself go he cannot any longer; just as he who has let a stone out of
his hand cannot recall it, and yet it rested with him to aim and throw
it, because the origination was in his power. Just so the unjust man,
and he who has lost all self-control, might originally have helped being
what they are, and so they are voluntarily what they are; but now that
they are become so they no longer have the power of being otherwise.

And not only are mental diseases voluntary, but the bodily are so in
some men, whom we accordingly blame: for such as are naturally deformed
no one blames, only such as are so by reason of want of exercise, and
neglect: and so too of weakness and maiming: no one would think of
upbraiding, but would rather compassionate, a man who is blind by
nature, or from disease, or from an accident; but every one would blame
him who was so from excess of wine, or any other kind of intemperance.
It seems, then, that in respect of bodily diseases, those which depend
on ourselves are censured, those which do not are not censured; and if
so, then in the case of the mental disorders, those which are censured
must depend upon ourselves.

[Sidenote: III4_b_] But suppose a man to say, "that (by our own
admission) all men aim at that which conveys to their minds an
impression of good, and that men have no control over this impression,
but that the End impresses each with a notion correspondent to his own
individual character; that to be sure if each man is in a way the cause
of his own moral state, so he will be also of the kind of impression he
receives: whereas, if this is not so, no one is the cause to himself of
doing evil actions, but he does them by reason of ignorance of the true
End, supposing that through their means he will secure the chief good.
Further, that this aiming at the End is no matter of one's own choice,
but one must be born with a power of mental vision, so to speak, whereby
to judge fairly and choose that which is really good; and he is blessed
by nature who has this naturally well: because it is the most important
thing and the fairest, and what a man cannot get or learn from another
but will have such as nature has given it; and for this to be so given
well and fairly would be excellence of nature in the highest and truest
sense."

If all this be true, how will Virtue be a whit more voluntary than Vice?
Alike to the good man and the bad, the End gives its impression and is
fixed by nature or howsoever you like to say, and they act so and so,
referring everything else to this End.

Whether then we suppose that the End impresses each man's mind with
certain notions not merely by nature, but that there is somewhat also
dependent on himself; or that the End is given by nature, and yet Virtue
is voluntary because the good man does all the rest voluntarily, Vice
must be equally so; because his own agency equally attaches to the bad
man in the actions, even if not in the selection of the End.

If then, as is commonly said, the Virtues are voluntary (because we at
least co-operate in producing our moral states, and we assume the End
to be of a certain kind according as we are ourselves of certain
characters), the Vices must be voluntary also, because the cases are
exactly similar.

Well now, we have stated generally respecting the Moral Virtues, the
genus (in outline), that they are mean states, and that they are habits,
and how they are formed, and that they are of themselves calculated to
act upon the circumstances out of which they were formed, and that they
are in our own power and voluntary, and are to be done so as right
Reason may direct.

[Sidenote: III5_a_] But the particular actions and the habits are not
voluntary in the same sense; for of the actions we are masters from
beginning to end (supposing of course a knowledge of the particular
details), but only of the origination of the habits, the addition by
small particular accessions not being cognisiable (as is the case with
sicknesses): still they are voluntary because it rested with us to use
our circumstances this way or that.

Here we will resume the particular discussion of the Moral Virtues,
and say what they are, what is their object-matter, and how they stand
respectively related to it: of course their number will be thereby
shown. First, then, of Courage. Now that it is a mean state, in respect
of fear and boldness, has been already said: further, the objects of our
fears are obviously things fearful or, in a general way of statement,
evils; which accounts for the common definition of fear, viz.
"expectation of evil."

Of course we fear evils of all kinds: disgrace, for instance, poverty,
disease, desolateness, death; but not all these seem to be the
object-matter of the Brave man, because there are things which to fear
is right and noble, and not to fear is base; disgrace, for example,
since he who fears this is a good man and has a sense of honour, and he
who does not fear it is shameless (though there are those who call him
Brave by analogy, because he somewhat resembles the Brave man who agrees
with him in being free from fear); but poverty, perhaps, or disease, and
in fact whatever does not proceed from viciousness, nor is attributable
to his own fault, a man ought not to fear: still, being fearless in
respect of these would not constitute a man Brave in the proper sense of
the term.

Yet we do apply the term in right of the similarity of the cases; for
there are men who, though timid in the dangers of war, are liberal men
and are stout enough to face loss of wealth.

And, again, a man is not a coward for fearing insult to his wife or
children, or envy, or any such thing; nor is he a Brave man for being
bold when going to be scourged.

What kind of fearful things then do constitute the object-matter of the
Brave man? first of all, must they not be the greatest, since no man is
more apt to withstand what is dreadful. Now the object of the greatest
dread is death, because it is the end of all things, and the dead man is
thought to be capable neither of good nor evil. Still it would seem
that the Brave man has not for his object-matter even death in every
circumstance; on the sea, for example, or in sickness: in what
circumstances then? must it not be in the most honourable? now such is
death in war, because it is death in the greatest and most honourable
danger; and this is confirmed by the honours awarded in communities, and
by monarchs.

He then may be most properly denominated Brave who is fearless in
respect of honourable death and such sudden emergencies as threaten
death; now such specially are those which arise in the course of war.

[Sidenote: 1115b] It is not meant but that the Brave man will be
fearless also on the sea (and in sickness), but not in the same way as
sea-faring men; for these are light-hearted and hopeful by reason of
their experience, while landsmen though Brave are apt to give themselves
up for lost and shudder at the notion of such a death: to which it
should be added that Courage is exerted in circumstances which admit
of doing something to help one's self, or in which death would be
honourable; now neither of these requisites attach to destruction by
drowning or sickness.



VII


Again, fearful is a term of relation, the same thing not being so to
all, and there is according to common parlance somewhat so fearful as to
be beyond human endurance: this of course would be fearful to every
man of sense, but those objects which are level to the capacity of
man differ in magnitude and admit of degrees, so too the objects of
confidence or boldness.

Now the Brave man cannot be frighted from his propriety (but of course
only so far as he is man); fear such things indeed he will, but he will
stand up against them as he ought and as right reason may direct, with a
view to what is honourable, because this is the end of the virtue.

Now it is possible to fear these things too much, or too little, or
again to fear what is not really fearful as if it were such. So the
errors come to be either that a man fears when he ought not to fear at
all, or that he fears in an improper way, or at a wrong time, and so
forth; and so too in respect of things inspiring confidence. He is
Brave then who withstands, and fears, and is bold, in respect of right
objects, from a right motive, in right manner, and at right times:
since the Brave man suffers or acts as he ought and as right reason may
direct.

Now the end of every separate act of working is that which accords
with the habit, and so to the Brave man Courage; which is honourable;
therefore such is also the End, since the character of each is
determined by the End.

So honour is the motive from which the Brave man withstands things
fearful and performs the acts which accord with Courage.

Of the characters on the side of Excess, he who exceeds in utter absence
of fear has no appropriate name (I observed before that many states have
none), but he would be a madman or inaccessible to pain if he feared
nothing, neither earthquake, nor the billows, as they tell of the Celts.

He again who exceeds in confidence in respect of things fearful is rash.
He is thought moreover to be a braggart, and to advance unfounded claims
to the character of Brave: the relation which the Brave man really bears
to objects of fear this man wishes to appear to bear, and so imitates
him in whatever points he can; for this reason most of them exhibit a
curious mixture of rashness and cowardice; because, affecting rashness
in these circumstances, they do not withstand what is truly fearful.

[Sidenote: III6_a_] The man moreover who exceeds in feeling fear is a
coward, since there attach to him the circumstances of fearing wrong
objects, in wrong ways, and so forth. He is deficient also in feeling
confidence, but he is most clearly seen as exceeding in the case of
pains; he is a fainthearted kind of man, for he fears all things: the
Brave man is just the contrary, for boldness is the property of the
light-hearted and hopeful.

So the coward, the rash, and the Brave man have exactly the same
object-matter, but stand differently related to it: the two
first-mentioned respectively exceed and are deficient, the last is in a
mean state and as he ought to be. The rash again are precipitate, and,
being eager before danger, when actually in it fall away, while the
Brave are quick and sharp in action, but before are quiet and composed.

Well then, as has been said, Courage is a mean state in respect of
objects inspiring boldness or fear, in the circumstances which have been
stated, and the Brave man chooses his line and withstands danger either
because to do so is honourable, or because not to do so is base. But
dying to escape from poverty, or the pangs of love, or anything that is
simply painful, is the act not of a Brave man but of a coward; because
it is mere softness to fly from what is toilsome, and the suicide braves
the terrors of death not because it is honourable but to get out of the
reach of evil.


VIII

Courage proper is somewhat of the kind I have described, but there are
dispositions, differing in five ways, which also bear in common parlance
the name of Courage.

We will take first that which bears most resemblance to the true, the
Courage of Citizenship, so named because the motives which are thought
to actuate the members of a community in braving danger are the
penalties and disgrace held out by the laws to cowardice, and the
dignities conferred on the Brave; which is thought to be the reason
why those are the bravest people among whom cowards are visited with
disgrace and the Brave held in honour.

Such is the kind of Courage Homer exhibits in his characters; Diomed and
Hector for example. The latter says,

 "Polydamas will be the first to fix
  Disgrace upon me."

Diomed again,

 "For Hector surely will hereafter say,
  Speaking in Troy, Tydides by my hand"--

This I say most nearly resembles the Courage before spoken of, because
it arises from virtue, from a feeling of shame, and a desire of what is
noble (that is, of honour), and avoidance of disgrace which is base. In
the same rank one would be inclined to place those also who act under
compulsion from their commanders; yet are they really lower, because not
a sense of honour but fear is the motive from which they act, and what
they seek to avoid is not that which is base but that which is simply
painful: commanders do in fact compel their men sometimes, as Hector
says (to quote Homer again),

  "But whomsoever I shall find cowering afar from the fight,
  The teeth of dogs he shall by no means escape."

[Sidenote: III6_h_] Those commanders who station staunch troops by
doubtful ones, or who beat their men if they flinch, or who draw their
troops up in line with the trenches, or other similar obstacles,
in their rear, do in effect the same as Hector, for they all use
compulsion.

But a man is to be Brave, not on compulsion, but from a sense of honour.

In the next place, Experience and Skill in the various particulars is
thought to be a species of Courage: whence Socrates also thought that
Courage was knowledge.

This quality is exhibited of course by different men under different
circumstances, but in warlike matters, with which we are now concerned,
it is exhibited by the soldiers ("the regulars"): for there are, it
would seem, many things in war of no real importance which these have
been constantly used to see; so they have a show of Courage because
other people are not aware of the real nature of these things. Then
again by reason of their skill they are better able than any others to
inflict without suffering themselves, because they are able to use their
arms and have such as are most serviceable both with a view to offence
and defence: so that their case is parallel to that of armed men
fighting with unarmed or trained athletes with amateurs, since in
contests of this kind those are the best fighters, not who are the
bravest men, but who are the strongest and are in the best condition.

In fact, the regular troops come to be cowards whenever the danger is
greater than their means of meeting it; supposing, for example, that
they are inferior in numbers and resources: then they are the first to
fly, but the mere militia stand and fall on the ground (which as you
know really happened at the Hermæum), for in the eyes of these flight
was disgraceful and death preferable to safety bought at such a price:
while "the regulars" originally went into the danger under a notion
of their own superiority, but on discovering their error they took to
flight, having greater fear of death than of disgrace; but this is not
the feeling of the Brave man.

Thirdly, mere Animal Spirit is sometimes brought under the term Courage:
they are thought to be Brave who are carried on by mere Animal Spirit,
as are wild beasts against those who have wounded them, because in fact
the really Brave have much Spirit, there being nothing like it for going
at danger of any kind; whence those frequent expressions in Homer,
"infused strength into his spirit," "roused his strength and spirit," or
again, "and keen strength in his nostrils," "his blood boiled:" for all
these seem to denote the arousing and impetuosity of the Animal Spirit.

[Sidenote: III7_a_] Now they that are truly Brave act from a sense of
honour, and this Animal Spirit co-operates with them; but wild beasts
from pain, that is because they have been wounded, or are frightened;
since if they are quietly in their own haunts, forest or marsh, they do
not attack men. Surely they are not Brave because they rush into danger
when goaded on by pain and mere Spirit, without any view of the danger:
else would asses be Brave when they are hungry, for though beaten they
will not then leave their pasture: profligate men besides do many bold
actions by reason of their lust. We may conclude then that they are not
Brave who are goaded on to meet danger by pain and mere Spirit; but
still this temper which arises from Animal Spirit appears to be most
natural, and would be Courage of the true kind if it could have added
to it moral choice and the proper motive. So men also are pained by a
feeling of anger, and take pleasure in revenge; but they who fight from
these causes may be good fighters, but they are not truly Brave (in
that they do not act from a sense of honour, nor as reason directs, but
merely from the present feeling), still they bear some resemblance to
that character.

Nor, again, are the Sanguine and Hopeful therefore Brave: since their
boldness in dangers arises from their frequent victories over numerous
foes. The two characters are alike, however, in that both are confident;
but then the Brave are so from the afore-mentioned causes, whereas these
are so from a settled conviction of their being superior and not likely
to suffer anything in return (they who are intoxicated do much the
same, for they become hopeful when in that state); but when the event
disappoints their expectations they run away: now it was said to be the
character of a Brave man to withstand things which are fearful to man
or produce that impression, because it is honourable so to do and the
contrary is dishonourable.

For this reason it is thought to be a greater proof of Courage to be
fearless and undisturbed under the pressure of sudden fear than under
that which may be anticipated, because Courage then comes rather from a
fixed habit, or less from preparation: since as to foreseen dangers a
man might take his line even from calculation and reasoning, but in
those which are sudden he will do so according to his fixed habit of
mind.

Fifthly and lastly, those who are acting under Ignorance have a show
of Courage and are not very far from the Hopeful; but still they are
inferior inasmuch as they have no opinion of themselves; which the
others have, and therefore stay and contest a field for some little
time; but they who have been deceived fly the moment they know things to
be otherwise than they supposed, which the Argives experienced when they
fell on the Lacedæmonians, taking them for the men of Sicyon. We have
described then what kind of men the Brave are, and what they who are
thought to be, but are not really, Brave.

[Sidenote: IX]

It must be remarked, however, that though Courage has for its
object-matter boldness and fear it has not both equally so, but objects
of fear much more than the former; for he that under pressure of these
is undisturbed and stands related to them as he ought is better entitled
to the name of Brave than he who is properly affected towards objects
of confidence. So then men are termed Brave for withstanding painful
things.

It follows that Courage involves pain and is justly praised, since it
is a harder matter to withstand things that are painful than to abstain
from such as are pleasant.

[Sidenote: 1117_b_]

It must not be thought but that the End and object of Courage is
pleasant, but it is obscured by the surrounding circumstances: which
happens also in the gymnastic games; to the boxers the End is pleasant
with a view to which they act, I mean the crown and the honours; but the
receiving the blows they do is painful and annoying to flesh and blood,
and so is all the labour they have to undergo; and, as these drawbacks
are many, the object in view being small appears to have no pleasantness
in it.

If then we may say the same of Courage, of course death and wounds must
be painful to the Brave man and against his will: still he endures these
because it is honourable so to do or because it is dishonourable not to
do so. And the more complete his virtue and his happiness so much the
more will he be pained at the notion of death: since to such a man as
he is it is best worth while to live, and he with full consciousness is
deprived of the greatest goods by death, and this is a painful idea. But
he is not the less Brave for feeling it to be so, nay rather it may be
he is shown to be more so because he chooses the honour that may be
reaped in war in preference to retaining safe possession of these other
goods. The fact is that to act with pleasure does not belong to all the
virtues, except so far as a man realises the End of his actions.

But there is perhaps no reason why not such men should make the best
soldiers, but those who are less truly Brave but have no other good to
care for: these being ready to meet danger and bartering their lives
against small gain.

Let thus much be accepted as sufficient on the subject of Courage; the
true nature of which it is not difficult to gather, in outline at least,
from what has been said.

[Sidenote: X]

Next let us speak of Perfected Self-Mastery, which seems to claim the
next place to Courage, since these two are the Excellences of the
Irrational part of the Soul.

That it is a mean state, having for its object-matter Pleasures, we have
already said (Pains being in fact its object-matter in a less degree
and dissimilar manner), the state of utter absence of self-control has
plainly the same object-matter; the next thing then is to determine what
kind of Pleasures.

Let Pleasures then be understood to be divided into mental and bodily:
instances of the former being love of honour or of learning: it being
plain that each man takes pleasure in that of these two objects which he
has a tendency to like, his body being no way affected but rather his
intellect. Now men are not called perfectly self-mastering or wholly
destitute of self-control in respect of pleasures of this class: nor in
fact in respect of any which are not bodily; those for example who love
to tell long stories, and are prosy, and spend their days about
mere chance matters, we call gossips but not wholly destitute of
self-control, nor again those who are pained at the loss of money or
friends.

[Sidenote: 1118_a_]

It is bodily Pleasures then which are the object-matter of Perfected
Self-Mastery, but not even all these indifferently: I mean, that they
who take pleasure in objects perceived by the Sight, as colours, and
forms, and painting, are not denominated men of Perfected Self-Mastery,
or wholly destitute of self-control; and yet it would seem that one may
take pleasure even in such objects, as one ought to do, or excessively,
or too little.

So too of objects perceived by the sense of Hearing; no one applies the
terms before quoted respectively to those who are excessively pleased
with musical tunes or acting, or to those who take such pleasure as they
ought.

Nor again to those persons whose pleasure arises from the sense
of Smell, except incidentally: I mean, we do not say men have no
self-control because they take pleasure in the scent of fruit, or
flowers, or incense, but rather when they do so in the smells of
unguents and sauces: since men destitute of self-control take pleasure
herein, because hereby the objects of their lusts are recalled to their
imagination (you may also see other men take pleasure in the smell of
food when they are hungry): but to take pleasure in such is a mark of
the character before named since these are objects of desire to him.

Now not even brutes receive pleasure in right of these senses, except
incidentally. I mean, it is not the scent of hares' flesh but the eating
it which dogs take pleasure in, perception of which pleasure is caused
by the sense of Smell. Or again, it is not the lowing of the ox but
eating him which the lion likes; but of the fact of his nearness the
lion is made sensible by the lowing, and so he appears to take pleasure
in this. In like manner, he has no pleasure in merely seeing or finding
a stag or wild goat, but in the prospect of a meal.

The habits of Perfect Self-Mastery and entire absence of self-control
have then for their object-matter such pleasures as brutes also share
in, for which reason they are plainly servile and brutish: they are
Touch and Taste.

But even Taste men seem to make little or no use of; for to the sense of
Taste belongs the distinguishing of flavours; what men do, in fact, who
are testing the quality of wines or seasoning "made dishes."

But men scarcely take pleasure at all in these things, at least those
whom we call destitute of self-control do not, but only in the actual
enjoyment which arises entirely from the sense of Touch, whether in
eating or in drinking, or in grosser lusts. This accounts for the wish
said to have been expressed once by a great glutton, "that his throat
had been formed longer than a crane's neck," implying that his pleasure
was derived from the Touch.

[Sidenote: 1118b] The sense then with which is connected the habit of
absence of self-control is the most common of all the senses, and this
habit would seem to be justly a matter of reproach, since it attaches to
us not in so far as we are men but in so far as we are animals. Indeed
it is brutish to take pleasure in such things and to like them best of
all; for the most respectable of the pleasures arising from the touch
have been set aside; those, for instance, which occur in the course of
gymnastic training from the rubbing and the warm bath: because the touch
of the man destitute of self-control is not indifferently of _any_ part
of the body but only of particular parts.

XI

Now of lusts or desires some are thought to be universal, others
peculiar and acquired; thus desire for food is natural since every one
who really needs desires also food, whether solid or liquid, or both
(and, as Homer says, the man in the prime of youth needs and desires
intercourse with the other sex); but when we come to this or that
particular kind, then neither is the desire universal nor in all men is
it directed to the same objects. And therefore the conceiving of such
desires plainly attaches to us as individuals. It must be admitted,
however, that there is something natural in it: because different things
are pleasant to different men and a preference of some particular
objects to chance ones is universal. Well then, in the case of the
desires which are strictly and properly natural few men go wrong and all
in one direction, that is, on the side of too much: I mean, to eat and
drink of such food as happens to be on the table till one is overfilled
is exceeding in quantity the natural limit, since the natural desire
is simply a supply of a real deficiency. For this reason these men are
called belly-mad, as filling it beyond what they ought, and it is the
slavish who become of this character.

But in respect of the peculiar pleasures many men go wrong and in many
different ways; for whereas the term "fond of so and so" implies either
taking pleasure in wrong objects, or taking pleasure excessively, or as
the mass of men do, or in a wrong way, they who are destitute of all
self-control exceed in all these ways; that is to say, they take
pleasure in some things in which they ought not to do so (because they
are properly objects of detestation), and in such as it is right to take
pleasure in they do so more than they ought and as the mass of men do.

Well then, that excess with respect to pleasures is absence of
self-control, and blameworthy, is plain. But viewing these habits on the
side of pains, we find that a man is not said to have the virtue for
withstanding them (as in the case of Courage), nor the vice for not
withstanding them; but the man destitute of self-control is such,
because he is pained more than he ought to be at not obtaining things
which are pleasant (and thus his pleasure produces pain to him), and the
man of Perfected Self-Mastery is such in virtue of not being pained by
their absence, that is, by having to abstain from what is pleasant.

[Sidenote:III9a] Now the man destitute of self-control desires either
all pleasant things indiscriminately or those which are specially
pleasant, and he is impelled by his desire to choose these things in
preference to all others; and this involves pain, not only when he
misses the attainment of his objects but, in the very desiring them,
since all desire is accompanied by pain. Surely it is a strange case
this, being pained by reason of pleasure.

As for men who are defective on the side of pleasure, who take
less pleasure in things than they ought, they are almost imaginary
characters, because such absence of sensual perception is not natural to
man: for even the other animals distinguish between different kinds of
food, and like some kinds and dislike others. In fact, could a man be
found who takes no pleasure in anything and to whom all things are
alike, he would be far from being human at all: there is no name for
such a character because it is simply imaginary.

But the man of Perfected Self-Mastery is in the mean with respect to
these objects: that is to say, he neither takes pleasure in the things
which delight the vicious man, and in fact rather dislikes them, nor at
all in improper objects; nor to any great degree in any object of the
class; nor is he pained at their absence; nor does he desire them; or,
if he does, only in moderation, and neither more than he ought, nor at
improper times, and so forth; but such things as are conducive to health
and good condition of body, being also pleasant, these he will grasp at
in moderation and as he ought to do, and also such other pleasant things
as do not hinder these objects, and are not unseemly or disproportionate
to his means; because he that should grasp at such would be liking such
pleasures more than is proper; but the man of Perfected Self-Mastery
is not of this character, but regulates his desires by the dictates of
right reason.

XII

Now the vice of being destitute of all Self-Control seems to be more
truly voluntary than Cowardice, because pleasure is the cause of the
former and pain of the latter, and pleasure is an object of choice,
pain of avoidance. And again, pain deranges and spoils the natural
disposition of its victim, whereas pleasure has no such effect and is
more voluntary and therefore more justly open to reproach.

It is so also for the following reason; that it is easier to be inured
by habit to resist the objects of pleasure, there being many things of
this kind in life and the process of habituation being unaccompanied by
danger; whereas the case is the reverse as regards the objects of fear.

Again, Cowardice as a confirmed habit would seem to be voluntary in
a different way from the particular instances which form the habit;
because it is painless, but these derange the man by reason of pain so
that he throws away his arms and otherwise behaves himself unseemly,
for which reason they are even thought by some to exercise a power of
compulsion.

But to the man destitute of Self-Control the particular instances are on
the contrary quite voluntary, being done with desire and direct exertion
of the will, but the general result is less voluntary: since no man
desires to form the habit.

[Sidenote: 1119b]

The name of this vice (which signifies etymologically unchastened-ness)
we apply also to the faults of children, there being a certain
resemblance between the cases: to which the name is primarily applied,
and to which secondarily or derivatively, is not relevant to the present
subject, but it is evident that the later in point of time must get the
name from the earlier. And the metaphor seems to be a very good one;
for whatever grasps after base things, and is liable to great increase,
ought to be chastened; and to this description desire and the child
answer most truly, in that children also live under the direction of
desire and the grasping after what is pleasant is most prominently seen
in these.

Unless then the appetite be obedient and subjected to the governing
principle it will become very great: for in the fool the grasping after
what is pleasant is insatiable and undiscriminating; and every acting
out of the desire increases the kindred habit, and if the desires are
great and violent in degree they even expel Reason entirely; therefore
they ought to be moderate and few, and in no respect to be opposed
to Reason. Now when the appetite is in such a state we denominate it
obedient and chastened.

In short, as the child ought to live with constant regard to the orders
of its educator, so should the appetitive principle with regard to those
of Reason.

So then in the man of Perfected Self-Mastery, the appetitive principle
must be accordant with Reason: for what is right is the mark at which
both principles aim: that is to say, the man of perfected self-mastery
desires what he ought in right manner and at right times, which is
exactly what Reason directs. Let this be taken for our account of
Perfected Self-Mastery.




BOOK IV

I

We will next speak of Liberality. Now this is thought to be the mean
state, having for its object-matter Wealth: I mean, the Liberal man is
praised not in the circumstances of war, nor in those which constitute
the character of perfected self-mastery, nor again in judicial
decisions, but in respect of giving and receiving Wealth, chiefly the
former. By the term Wealth I mean "all those things whose worth is
measured by money."

Now the states of excess and defect in regard of Wealth are respectively
Prodigality and Stinginess: the latter of these terms we attach
invariably to those who are over careful about Wealth, but the former we
apply sometimes with a complex notion; that is to say, we give the name
to those who fail of self-control and spend money on the unrestrained
gratification of their passions; and this is why they are thought to be
most base, because they have many vices at once.

[Sidenote: 1120a]

It must be noted, however, that this is not a strict and proper use of
the term, since its natural etymological meaning is to denote him who
has one particular evil, viz. the wasting his substance: he is unsaved
(as the term literally denotes) who is wasting away by his own fault;
and this he really may be said to be; the destruction of his substance
is thought to be a kind of wasting of himself, since these things
are the means of living. Well, this is our acceptation of the term
Prodigality.

Again. Whatever things are for use may be used well or ill, and Wealth
belongs to this class. He uses each particular thing best who has the
virtue to whose province it belongs: so that he will use Wealth best
who has the virtue respecting Wealth, that is to say, the Liberal
man. Expenditure and giving are thought to be the using of money, but
receiving and keeping one would rather call the possessing of it. And so
the giving to proper persons is more characteristic of the Liberal man,
than the receiving from proper quarters and forbearing to receive
from the contrary. In fact generally, doing well by others is more
characteristic of virtue than being done well by, and doing things
positively honourable than forbearing to do things dishonourable;
and any one may see that the doing well by others and doing things
positively honourable attaches to the act of giving, but to that of
receiving only the being done well by or forbearing to do what is
dishonourable.

Besides, thanks are given to him who gives, not to him who merely
forbears to receive, and praise even more. Again, forbearing to receive
is easier than giving, the case of being too little freehanded with
one's own being commoner than taking that which is not one's own.

And again, it is they who give that are denominated Liberal, while they
who forbear to receive are commended, not on the score of Liberality but
of just dealing, while for receiving men are not, in fact, praised at
all.

And the Liberal are liked almost best of all virtuous characters,
because they are profitable to others, and this their profitableness
consists in their giving.

Furthermore: all the actions done in accordance with virtue are
honourable, and done from the motive of honour: and the Liberal man,
therefore, will give from a motive of honour, and will give rightly;
I mean, to proper persons, in right proportion, at right times, and
whatever is included in the term "right giving:" and this too with
positive pleasure, or at least without pain, since whatever is done in
accordance with virtue is pleasant or at least not unpleasant, most
certainly not attended with positive pain.

But the man who gives to improper people, or not from a motive of honour
but from some other cause, shall be called not Liberal but something
else. Neither shall he be so [Sidenote:1120b] denominated who does it
with pain: this being a sign that he would prefer his wealth to the
honourable action, and this is no part of the Liberal man's character;
neither will such an one receive from improper sources, because the so
receiving is not characteristic of one who values not wealth: nor again
will he be apt to ask, because one who does kindnesses to others does
not usually receive them willingly; but from proper sources (his own
property, for instance) he will receive, doing this not as honourable
but as necessary, that he may have somewhat to give: neither will he be
careless of his own, since it is his wish through these to help others
in need: nor will he give to chance people, that he may have wherewith
to give to those to whom he ought, at right times, and on occasions when
it is honourable so to do.

Again, it is a trait in the Liberal man's character even to exceed
very much in giving so as to leave too little for himself, it being
characteristic of such an one not to have a thought of self.

Now Liberality is a term of relation to a man's means, for the
Liberal-ness depends not on the amount of what is given but on the moral
state of the giver which gives in proportion to his means. There is then
no reason why he should not be the more Liberal man who gives the less
amount, if he has less to give out of.

Again, they are thought to be more Liberal who have inherited, not
acquired for themselves, their means; because, in the first place, they
have never experienced want, and next, all people love most their own
works, just as parents do and poets.

It is not easy for the Liberal man to be rich, since he is neither apt
to receive nor to keep but to lavish, and values not wealth for its own
sake but with a view to giving it away. Hence it is commonly charged
upon fortune that they who most deserve to be rich are least so. Yet
this happens reasonably enough; it is impossible he should have wealth
who does not take any care to have it, just as in any similar case.

Yet he will not give to improper people, nor at wrong times, and so on:
because he would not then be acting in accordance with Liberality, and
if he spent upon such objects, would have nothing to spend on those on
which he ought: for, as I have said before, he is Liberal who spends in
proportion to his means, and on proper objects, while he who does so
in excess is prodigal (this is the reason why we never call despots
prodigal, because it does not seem to be easy for them by their gifts
and expenditure to go beyond their immense possessions).

To sum up then. Since Liberality is a mean state in respect of the
giving and receiving of wealth, the Liberal man will give and spend on
proper objects, and in proper proportion, in great things and in small
alike, and all this with pleasure to himself; also he will receive from
right sources, and in right proportion: because, as the virtue is a mean
state in respect of both, he will do both as he ought, and, in fact,
upon proper giving follows the correspondent receiving, while that which
is not such is contrary to it. (Now those which follow one another come
to co-exist in the same person, those which are contraries plainly do
not.)

[Sidenote:1121a] Again, should it happen to him to spend money beyond
what is needful, or otherwise than is well, he will be vexed, but only
moderately and as he ought; for feeling pleasure and pain at right
objects, and in right manner, is a property of Virtue.

The Liberal man is also a good man to have for a partner in respect of
wealth: for he can easily be wronged, since he values not wealth, and
is more vexed at not spending where he ought to have done so than at
spending where he ought not, and he relishes not the maxim of Simonides.

But the Prodigal man goes wrong also in these points, for he is neither
pleased nor pained at proper objects or in proper manner, which will
become more plain as we proceed. We have said already that Prodigality
and Stinginess are respectively states of excess and defect, and this in
two things, giving and receiving (expenditure of course we class under
giving). Well now, Prodigality exceeds in giving and forbearing to
receive and is deficient in receiving, while Stinginess is deficient in
giving and exceeds in receiving, but it is in small things.

The two parts of Prodigality, to be sure, do not commonly go together;
it is not easy, I mean, to give to all if you receive from none, because
private individuals thus giving will soon find their means run short,
and such are in fact thought to be prodigal. He that should combine both
would seem to be no little superior to the Stingy man: for he may be
easily cured, both by advancing in years, and also by the want of means,
and he may come thus to the mean: he has, you see, already the _facts_
of the Liberal man, he gives and forbears to receive, only he does
neither in right manner or well. So if he could be wrought upon by
habituation in this respect, or change in any other way, he would be a
real Liberal man, for he will give to those to whom he should, and will
forbear to receive whence he ought not. This is the reason too why he is
thought not to be low in moral character, because to exceed in giving
and in forbearing to receive is no sign of badness or meanness, but only
of folly.

[Sidenote:1121b] Well then, he who is Prodigal in this fashion is
thought far superior to the Stingy man for the aforementioned reasons,
and also because he does good to many, but the Stingy man to no one,
not even to himself. But most Prodigals, as has been said, combine with
their other faults that of receiving from improper sources, and on this
point are Stingy: and they become grasping, because they wish to spend
and cannot do this easily, since their means soon run short and they are
necessitated to get from some other quarter; and then again, because
they care not for what is honourable, they receive recklessly, and from
all sources indifferently, because they desire to give but care not how
or whence. And for this reason their givings are not Liberal, inasmuch
as they are not honourable, nor purely disinterested, nor done in right
fashion; but they oftentimes make those rich who should be poor, and to
those who are quiet respectable kind of people they will give nothing,
but to flatterers, or those who subserve their pleasures in any way,
they will give much. And therefore most of them are utterly devoid
of self-restraint; for as they are open-handed they are liberal in
expenditure upon the unrestrained gratification of their passions, and
turn off to their pleasures because they do not live with reference to
what is honourable.

Thus then the Prodigal, if unguided, slides into these faults; but if he
could get care bestowed on him he might come to the mean and to what is
right.

Stinginess, on the contrary, is incurable: old age, for instance, and
incapacity of any kind, is thought to make people Stingy; and it is more
congenial to human nature than Prodigality, the mass of men being fond
of money rather than apt to give: moreover it extends far and has many
phases, the modes of stinginess being thought to be many. For as it
consists of two things, defect of giving and excess of receiving,
everybody does not have it entire, but it is sometimes divided, and one
class of persons exceed in receiving, the other are deficient in giving.
I mean those who are designated by such appellations as sparing,
close-fisted, niggards, are all deficient in giving; but other men's
property they neither desire nor are willing to receive, in some
instances from a real moderation and shrinking from what is base.

There are some people whose motive, either supposed or alleged, for
keeping their property is this, that they may never be driven to do
anything dishonourable: to this class belongs the skinflint, and every
one of similar character, so named from the excess of not-giving. Others
again decline to receive their neighbour's goods from a motive of fear;
their notion being that it is not easy to take other people's things
yourself without their taking yours: so they are content neither to
receive nor give.

[Sidenote:1122a] The other class again who are Stingy in respect of
receiving exceed in that they receive anything from any source; such as
they who work at illiberal employments, brothel keepers, and such-like,
and usurers who lend small sums at large interest: for all these receive
from improper sources, and improper amounts. Their common characteristic
is base-gaining, since they all submit to disgrace for the sake of gain
and that small; because those who receive great things neither whence
they ought, nor what they ought (as for instance despots who sack cities
and plunder temples), we denominate wicked, impious, and unjust, but not
Stingy.

Now the dicer and bath-plunderer and the robber belong to the class of
the Stingy, for they are given to base gain: both busy themselves and
submit to disgrace for the sake of gain, and the one class incur the
greatest dangers for the sake of their booty, while the others make gain
of their friends to whom they ought to be giving.

So both classes, as wishing to make gain from improper sources, are
given to base gain, and all such receivings are Stingy. And with good
reason is Stinginess called the contrary of Liberality: both because it
is a greater evil than Prodigality, and because men err rather in this
direction than in that of the Prodigality which we have spoken of as
properly and completely such.

Let this be considered as what we have to say respecting Liberality and
the contrary vices.

II

Next in order would seem to come a dissertation on Magnificence,
this being thought to be, like liberality, a virtue having for its
object-matter Wealth; but it does not, like that, extend to all
transactions in respect of Wealth, but only applies to such as are
expensive, and in these circumstances it exceeds liberality in respect
of magnitude, because it is (what the very name in Greek hints at)
fitting expense on a large scale: this term is of course relative: I
mean, the expenditure of equipping and commanding a trireme is not the
same as that of giving a public spectacle: "fitting" of course also is
relative to the individual, and the matter wherein and upon which he has
to spend. And a man is not denominated Magnificent for spending as he
should do in small or ordinary things, as, for instance,

  "Oft to the wandering beggar did I give,"

but for doing so in great matters: that is to say, the Magnificent man
is liberal, but the liberal is not thereby Magnificent. The falling
short of such a state is called Meanness, the exceeding it Vulgar
Profusion, Want of Taste, and so on; which are faulty, not because they
are on an excessive scale in respect of right objects but, because they
show off in improper objects, and in improper manner: of these we will
speak presently. The Magnificent man is like a man of skill, because he
can see what is fitting, and can spend largely in good taste; for, as
we said at the commencement, [Sidenote: 1122b] the confirmed habit is
determined by the separate acts of working, and by its object-matter.

Well, the expenses of the Magnificent man are great and fitting: such
also are his works (because this secures the expenditure being not great
merely, but befitting the work). So then the work is to be proportionate
to the expense, and this again to the work, or even above it: and the
Magnificent man will incur such expenses from the motive of honour, this
being common to all the virtues, and besides he will do it with pleasure
and lavishly; excessive accuracy in calculation being Mean. He will
consider also how a thing may be done most beautifully and fittingly,
rather, than for how much it may be done, and how at the least expense.

So the Magnificent man must be also a liberal man, because the liberal
man will also spend what he ought, and in right manner: but it is the
Great, that is to say tke large scale, which is distinctive of the
Magnificent man, the object-matter of liberality being the same, and
without spending more money than another man he will make the work more
magnificent. I mean, the excellence of a possession and of a work is not
the same: as a piece of property that thing is most valuable which is
worth most, gold for instance; but as a work that which is great and
beautiful, because the contemplation of such an object is admirable,
and so is that which is Magnificent. So the excellence of a work is
Magnificence on a large scale. There are cases of expenditure which we
call honourable, such as are dedicatory offerings to the gods, and the
furnishing their temples, and sacrifices, and in like manner everything
that has reference to the Deity, and all such public matters as are
objects of honourable ambition, as when men think in any case that it is
their duty to furnish a chorus for the stage splendidly, or fit out and
maintain a trireme, or give a general public feast.

Now in all these, as has been already stated, respect is had also to the
rank and the means of the man who is doing them: because they should be
proportionate to these, and befit not the work only but also the doer of
the work. For this reason a poor man cannot be a Magnificent man, since
he has not means wherewith to spend largely and yet becomingly; and if
he attempts it he is a fool, inasmuch as it is out of proportion and
contrary to propriety, whereas to be in accordance with virtue a thing
must be done rightly.

Such expenditure is fitting moreover for those to whom such things
previously belong, either through themselves or through their ancestors
or people with whom they are connected, and to the high-born or people
of high repute, and so on: because all these things imply greatness and
reputation.

So then the Magnificent man is pretty much as I have described him,
and Magnificence consists in such expenditures: because they are the
greatest and most honourable: [Sidenote:1123a] and of private ones such
as come but once for all, marriage to wit, and things of that kind; and
any occasion which engages the interest of the community in general, or
of those who are in power; and what concerns receiving and despatching
strangers; and gifts, and repaying gifts: because the Magnificent man
is not apt to spend upon himself but on the public good, and gifts are
pretty much in the same case as dedicatory offerings.

It is characteristic also of the Magnificent man to furnish his house
suitably to his wealth, for this also in a way reflects credit; and
again, to spend rather upon such works as are of long duration, these
being most honourable. And again, propriety in each case, because the
same things are not suitable to gods and men, nor in a temple and a
tomb. And again, in the case of expenditures, each must be great of its
kind, and great expense on a great object is most magnificent, that is
in any case what is great in these particular things.

There is a difference too between greatness of a work and greatness of
expenditure: for instance, a very beautiful ball or cup is magnificent
as a present to a child, while the price of it is small and almost
mean. Therefore it is characteristic of the Magnificent man to do
magnificently whatever he is about: for whatever is of this kind cannot
be easily surpassed, and bears a proper proportion to the expenditure.

Such then is the Magnificent man.

The man who is in the state of excess, called one of Vulgar Profusion,
is in excess because he spends improperly, as has been said. I mean in
cases requiring small expenditure he lavishes much and shows off out of
taste; giving his club a feast fit for a wedding-party, or if he has to
furnish a chorus for a comedy, giving the actors purple to wear in the
first scene, as did the Megarians. And all such things he will do, not
with a view to that which is really honourable, but to display his
wealth, and because he thinks he shall be admired for these things; and
he will spend little where he ought to spend much, and much where he
should spend little.

The Mean man will be deficient in every case, and even where he has
spent the most he will spoil the whole effect for want of some trifle;
he is procrastinating in all he does, and contrives how he may spend
the least, and does even that with lamentations about the expense, and
thinking that he does all things on a greater scale than he ought.

Of course, both these states are faulty, but they do not involve
disgrace because they are neither hurtful to others nor very unseemly.

III

The very name of Great-mindedness implies, that great things are its
object-matter; and we will first settle what kind of things. It makes no
difference, of course, whether we regard the moral state in the abstract
or as exemplified in an individual.

[Sidenote: 1123b] Well then, he is thought to be Great-minded who values
himself highly and at the same time justly, because he that does so
without grounds is foolish, and no virtuous character is foolish or
senseless. Well, the character I have described is Great-minded. The man
who estimates himself lowly, and at the same time justly, is modest; but
not Great-minded, since this latter quality implies greatness, just as
beauty implies a large bodily conformation while small people are neat
and well made but not beautiful.

Again, he who values himself highly without just grounds is a Vain
man: though the name must not be applied to every case of unduly
high self-estimation. He that values himself below his real worth is
Small-minded, and whether that worth is great, moderate, or small, his
own estimate falls below it. And he is the strongest case of this error
who is really a man of great worth, for what would he have done had his
worth been less?

The Great-minded man is then, as far as greatness is concerned, at
the summit, but in respect of propriety he is in the mean, because he
estimates himself at his real value (the other characters respectively
are in excess and defect). Since then he justly estimates himself at a
high, or rather at the highest possible rate, his character will have
respect specially to one thing: this term "rate" has reference of course
to external goods: and of these we should assume that to be the greatest
which we attribute to the gods, and which is the special object of
desire to those who are in power, and which is the prize proposed to the
most honourable actions: now honour answers to these descriptions, being
the greatest of external goods. So the Great-minded man bears himself as
he ought in respect of honour and dishonour. In fact, without need of
words, the Great-minded plainly have honour for their object-matter:
since honour is what the great consider themselves specially worthy of,
and according to a certain rate.

The Small-minded man is deficient, both as regards himself, and also
as regards the estimation of the Great-minded: while the Vain man is in
excess as regards himself, but does not get beyond the Great-minded
man. Now the Great-minded man, being by the hypothesis worthy of the
greatest things, must be of the highest excellence, since the better a
man is the more is he worth, and he who is best is worth the most: it
follows then, that to be truly Great-minded a man must be good,
and whatever is great in each virtue would seem to belong to the
Great-minded. It would no way correspond with the character of the
Great-minded to flee spreading his hands all abroad; nor to injure any
one; for with what object in view will he do what is base, in whose eyes
nothing is great? in short, if one were to go into particulars, the
Great-minded man would show quite ludicrously unless he were a good man:
he would not be in fact deserving of honour if he were a bad man, honour
being the prize of virtue and given to the good.

This virtue, then, of Great-mindedness seems to be a kind of ornament
of all the other virtues, in that it makes them better and cannot be
without them; and for this reason it is a hard matter to be really and
truly Great-minded; for it cannot be without thorough goodness and
nobleness of character.

[Sidenote:1124a] Honour then and dishonour are specially the
object-matter of the Great-minded man: and at such as is great, and
given by good men, he will be pleased moderately as getting his own, or
perhaps somewhat less for no honour can be quite adequate to perfect
virtue: but still he will accept this because they have nothing higher
to give him. But such as is given by ordinary people and on trifling
grounds he will entirely despise, because these do not come up to his
deserts: and dishonour likewise, because in his case there cannot be
just ground for it.

Now though, as I have said, honour is specially the object-matter of the
Great-minded man, I do not mean but that likewise in respect of wealth
and power, and good or bad fortune of every kind, he will bear himself
with moderation, fall out how they may, and neither in prosperity will
he be overjoyed nor in adversity will he be unduly pained. For not even
in respect of honour does he so bear himself; and yet it is the greatest
of all such objects, since it is the cause of power and wealth being
choiceworthy, for certainly they who have them desire to receive honour
through them. So to whom honour even is a small thing to him will all
other things also be so; and this is why such men are thought to be
supercilious.

It seems too that pieces of good fortune contribute to form this
character of Great-mindedness: I mean, the nobly born, or men of
influence, or the wealthy, are considered to be entitled to honour, for
they are in a position of eminence and whatever is eminent by good is
more entitled to honour: and this is why such circumstances dispose men
rather to Great-mindedness, because they receive honour at the hands of
some men.

Now really and truly the good man alone is entitled to honour; only if
a man unites in himself goodness with these external advantages he is
thought to be more entitled to honour: but they who have them without
also having virtue are not justified in their high estimate of
themselves, nor are they rightly denominated Great-minded; since perfect
virtue is one of the indispensable conditions to such & character.

[Sidenote:1124b] Further, such men become supercilious and insolent, it
not being easy to bear prosperity well without goodness; and not being
able to bear it, and possessed with an idea of their own superiority to
others, they despise them, and do just whatever their fancy prompts; for
they mimic the Great-minded man, though they are not like him, and they
do this in such points as they can, so without doing the actions which
can only flow from real goodness they despise others. Whereas the
Great-minded man despises on good grounds (for he forms his opinions
truly), but the mass of men do it at random.

Moreover, he is not a man to incur little risks, nor does he court
danger, because there are but few things he has a value for; but he will
incur great dangers, and when he does venture he is prodigal of his life
as knowing that there are terms on which it is not worth his while to
live. He is the sort of man to do kindnesses, but he is ashamed to
receive them; the former putting a man in the position of superiority,
the latter in that of inferiority; accordingly he will greatly overpay
any kindness done to him, because the original actor will thus be laid
under obligation and be in the position of the party benefited. Such men
seem likewise to remember those they have done kindnesses to, but not
those from whom they have received them: because he who has received is
inferior to him who has done the kindness and our friend wishes to be
superior; accordingly he is pleased to hear of his own kind acts but not
of those done to himself (and this is why, in Homer, Thetis does
not mention to Jupiter the kindnesses she had done him, nor did the
Lacedæmonians to the Athenians but only the benefits they had received).

Further, it is characteristic of the Great-minded man to ask favours not
at all, or very reluctantly, but to do a service very readily; and to
bear himself loftily towards the great or fortunate, but towards people
of middle station affably; because to be above the former is difficult
and so a grand thing, but to be above the latter is easy; and to be high
and mighty towards the former is not ignoble, but to do it towards those
of humble station would be low and vulgar; it would be like parading
strength against the weak.

And again, not to put himself in the way of honour, nor to go where
others are the chief men; and to be remiss and dilatory, except in the
case of some great honour or work; and to be concerned in few things,
and those great and famous. It is a property of him also to be open,
both in his dislikes and his likings, because concealment is a
consequent of fear. Likewise to be careful for reality rather than
appearance, and talk and act openly (for his contempt for others makes
him a bold man, for which same reason he is apt to speak the truth,
except where the principle of reserve comes in), but to be reserved
towards the generality of men.

[Sidenote: II25a] And to be unable to live with reference to any other
but a friend; because doing so is servile, as may be seen in that all
flatterers are low and men in low estate are flatterers. Neither is his
admiration easily excited, because nothing is great in his eyes; nor
does he bear malice, since remembering anything, and specially wrongs,
is no part of Great-mindedness, but rather overlooking them; nor does he
talk of other men; in fact, he will not speak either of himself or of
any other; he neither cares to be praised himself nor to have others
blamed; nor again does he praise freely, and for this reason he is
not apt to speak ill even of his enemies except to show contempt and
insolence.

And he is by no means apt to make laments about things which cannot be
helped, or requests about those which are trivial; because to be thus
disposed with respect to these things is consequent only upon real
anxiety about them. Again, he is the kind of man to acquire what
is beautiful and unproductive rather than what is productive and
profitable: this being rather the part of an independent man. Also slow
motion, deep-toned voice, and deliberate style of speech, are thought to
be characteristic of the Great-minded man: for he who is earnest about
few things is not likely to be in a hurry, nor he who esteems nothing
great to be very intent: and sharp tones and quickness are the result of
these.

This then is my idea of the Great-minded man; and he who is in the
defect is a Small-minded man, he who is in the excess a Vain man.
However, as we observed in respect of the last character we discussed,
these extremes are not thought to be vicious exactly, but only mistaken,
for they do no harm.

The Small-minded man, for instance, being really worthy of good deprives
himself of his deserts, and seems to have somewhat faulty from not
having a sufficiently high estimate of his own desert, in fact from
self-ignorance: because, but for this, he would have grasped after what
he really is entitled to, and that is good. Still such characters are
not thought to be foolish, but rather laggards. But the having such
an opinion of themselves seems to have a deteriorating effect on the
character: because in all cases men's aims are regulated by their
supposed desert, and thus these men, under a notion of their own want of
desert, stand aloof from honourable actions and courses, and similarly
from external goods.

But the Vain are foolish and self-ignorant, and that palpably: because
they attempt honourable things, as though they were worthy, and then
they are detected. They also set themselves off, by dress, and carriage,
and such-like things, and desire that their good circumstances may
be seen, and they talk of them under the notion of receiving
honour thereby. Small-mindedness rather than Vanity is opposed to
Great-mindedness, because it is more commonly met with and is worse.

[Sidenote:1125b] Well, the virtue of Great-mindedness has for its object
great Honour, as we have said: and there seems to be a virtue having
Honour also for its object (as we stated in the former book), which may
seem to bear to Great-mindedness the same relation that Liberality does
to Magnificence: that is, both these virtues stand aloof from what is
great but dispose us as we ought to be disposed towards moderate and
small matters. Further: as in giving and receiving of wealth there is
a mean state, an excess, and a defect, so likewise in grasping after
Honour there is the more or less than is right, and also the doing so
from right sources and in right manner.

For we blame the lover of Honour as aiming at Honour more than he ought,
and from wrong sources; and him who is destitute of a love of Honour as
not choosing to be honoured even for what is noble. Sometimes again we
praise the lover of Honour as manly and having a love for what is noble,
and him who has no love for it as being moderate and modest (as we
noticed also in the former discussion of these virtues).

It is clear then that since "Lover of so and so" is a term capable of
several meanings, we do not always denote the same quality by the term
"Lover of Honour;" but when we use it as a term of commendation we
denote more than the mass of men are; when for blame more than a man
should be.

And the mean state having no proper name the extremes seem to dispute
for it as unoccupied ground: but of course where there is excess and
defect there must be also the mean. And in point of fact, men do grasp
at Honour more than they should, and less, and sometimes just as they
ought; for instance, this state is praised, being a mean state in regard
of Honour, but without any appropriate name. Compared with what is
called Ambition it shows like a want of love for Honour, and compared
with this it shows like Ambition, or compared with both, like both
faults: nor is this a singular case among the virtues. Here the
extreme characters appear to be opposed, because the mean has no name
appropriated to it.


V

Meekness is a mean state, having for its object-matter Anger: and as the
character in the mean has no name, and we may almost say the same of the
extremes, we give the name of Meekness (leaning rather to the defect,
which has no name either) to the character in the mean.

The excess may be called an over-aptness to Anger: for the passion is
Anger, and the producing causes many and various. Now he who is angry at
what and with whom he ought, and further, in right manner and time, and
for proper length of time, is praised, so this Man will be Meek since
Meekness is praised. For the notion represented by the term Meek man is
the being imperturbable, and not being led away by passion, but being
angry in that manner, and at those things, and for that length of time,
which Reason may direct. This character however is thought to err rather
on [Sidenote:1126a] the side of defect, inasmuch as he is not apt to
take revenge but rather to make allowances and forgive. And the defect,
call it Angerlessness or what you will, is blamed: I mean, they who are
not angry at things at which they ought to be angry are thought to be
foolish, and they who are angry not in right manner, nor in right time,
nor with those with whom they ought; for a man who labours under this
defect is thought to have no perception, nor to be pained, and to have
no tendency to avenge himself, inasmuch as he feels no anger: now to
bear with scurrility in one's own person, and patiently see one's own
friends suffer it, is a slavish thing.

As for the excess, it occurs in all forms; men are angry with those with
whom, and at things with which, they ought not to be, and more than they
ought, and too hastily, and for too great a length of time. I do not
mean, however, that these are combined in any one person: that would
in fact be impossible, because the evil destroys itself, and if it is
developed in its full force it becomes unbearable.

Now those whom we term the Passionate are soon angry, and with people
with whom and at things at which they ought not, and in an excessive
degree, but they soon cool again, which is the best point about them.
And this results from their not repressing their anger, but repaying
their enemies (in that they show their feeings by reason of their
vehemence), and then they have done with it.

The Choleric again are excessively vehement, and are angry at
everything, and on every occasion; whence comes their Greek name
signifying that their choler lies high.

The Bitter-tempered are hard to reconcile and keep their anger for
a long while, because they repress the feeling: but when they have
revenged themselves then comes a lull; for the vengeance destroys their
anger by producing pleasure in lieu of pain. But if this does not happen
they keep the weight on their minds: because, as it does not show
itself, no one attempts to reason it away, and digesting anger within
one's self takes time. Such men are very great nuisances to themselves
and to their best friends.

Again, we call those Cross-grained who are angry at wrong objects, and
in excessive degree, and for too long a time, and who are not appeased
without vengeance or at least punishing the offender.

To Meekness we oppose the excess rather than the defect, because it is
of more common occurrence: for human nature is more disposed to take
than to forgo revenge. And the Cross-grained are worse to live with
[than they who are too phlegmatic].

Now, from what has been here said, that is also plain which was said
before. I mean, it is no easy matter to define how, and with what
persons, and at what kind of things, and how long one ought to be
angry, and up to what point a person is right or is wrong. For he that
transgresses the strict rule only a little, whether on the side of
too much or too little, is not blamed: sometimes we praise those who
[Sidenote:1126b] are deficient in the feeling and call them Meek,
sometimes we call the irritable Spirited as being well qualified for
government. So it is not easy to lay down, in so many words, for what
degree or kind of transgression a man is blameable: because the decision
is in particulars, and rests therefore with the Moral Sense. Thus much,
however, is plain, that the mean state is praiseworthy, in virtue of
which we are angry with those with whom, and at those things with which,
we ought to be angry, and in right manner, and so on; while the excesses
and defects are blameable, slightly so if only slight, more so if
greater, and when considerable very blameable.

It is clear, therefore, that the mean state is what we are to hold to.

This then is to be taken as our account of the various moral states
which have Anger for their object-matter.

VI

Next, as regards social intercourse and interchange of words and acts,
some men are thought to be Over-Complaisant who, with a view solely to
giving pleasure, agree to everything and never oppose, but think their
line is to give no pain to those they are thrown amongst: they, on
the other hand, are called Cross and Contentious who take exactly the
contrary line to these, and oppose in everything, and have no care at
all whether they give pain or not.

Now it is quite clear of course, that the states I have named are
blameable, and that the mean between them is praiseworthy, in virtue
of which a man will let pass what he ought as he ought, and also will
object in like manner. However, this state has no name appropriated, but
it is most like Friendship; since the man who exhibits it is just the
kind of man whom we would call the amiable friend, with the addition of
strong earnest affection; but then this is the very point in which it
differs from Friendship, that it is quite independent of any feeling or
strong affection for those among whom the man mixes: I mean, that he
takes everything as he ought, not from any feeling of love or hatred,
but simply because his natural disposition leads him to do so; he will
do it alike to those whom he does know and those whom he does not, and
those with whom he is intimate and those with whom he is not; only in
each case as propriety requires, because it is not fitting to care
alike for intimates and strangers, nor again to pain them alike.

It has been stated in a general way that his social intercourse will be
regulated by propriety, and his aim will be to avoid giving pain and to
contribute to pleasure, but with a constant reference to what is noble
and expedient.

His proper object-matter seems to be the pleasures and pains which arise
out of social intercourse, but whenever it is not honourable or even
hurtful to him to contribute to pleasure, in these instances he will run
counter and prefer to give pain.

Or if the things in question involve unseemliness to the doer, and this
not inconsiderable, or any harm, whereas his opposition will cause some
little pain, here he will not agree but will run counter.

[Sidenote:1127a] Again, he will regulate differently his intercourse
with great men and with ordinary men, and with all people according to
the knowledge he has of them; and in like manner, taking in any other
differences which may exist, giving to each his due, and in itself
preferring to give pleasure and cautious not to give pain, but still
guided by the results, I mean by what is noble and expedient according
as they preponderate.

Again, he will inflict trifling pain with a view to consequent pleasure.

Well, the man bearing the mean character is pretty well such as I have
described him, but he has no name appropriated to him: of those who try
to give pleasure, the man who simply and disinterestedly tries to be
agreeable is called Over-Complaisant, he who does it with a view to
secure some profit in the way of wealth, or those things which wealth
may procure, is a Flatterer: I have said before, that the man who is
"always non-content" is Cross and Contentious. Here the extremes have
the appearance of being opposed to one another, because the mean has no
appropriate name.



VII

The mean state which steers clear of Exaggeration has pretty much the
same object-matter as the last we described, and likewise has no name
appropriated to it. Still it may be as well to go over these states:
because, in the first place, by a particular discussion of each we shall
be better acquainted with the general subject of moral character, and
next we shall be the more convinced that the virtues are mean states by
seeing that this is universally the case.

In respect then of living in society, those who carry on this
intercourse with a view to pleasure and pain have been already spoken
of; we will now go on to speak of those who are True or False, alike in
their words and deeds and in the claims which they advance.

Now the Exaggerator is thought to have a tendency to lay claim to things
reflecting credit on him, both when they do not belong to him at all and
also in greater degree than that in which they really do: whereas the
Reserved man, on the contrary, denies those which really belong to
him or else depreciates them, while the mean character being a
Plain-matter-of-fact person is Truthful in life and word, admitting
the existence of what does really belong to him and making it neither
greater nor less than the truth.

It is possible of course to take any of these lines either with or
without some further view: but in general men speak, and act, and live,
each according to his particular character and disposition, unless
indeed a man is acting from any special motive.

Now since falsehood is in itself low and blameable, while truth is noble
and praiseworthy, it follows that the Truthful man (who is also in the
mean) is praiseworthy, and the two who depart from strict truth are both
blameable, but especially the Exaggerator.

We will now speak of each, and first of the Truthful man: I call him
Truthful, because we are not now meaning the man who is true in his
agreements nor in such matters as amount to justice or injustice (this
would come within the [Sidenote:1127b] province of a different virtue),
but, in such as do not involve any such serious difference as this, the
man we are describing is true in life and word simply because he is in a
certain moral state.

And he that is such must be judged to be a good man: for he that has a
love for Truth as such, and is guided by it in matters indifferent, will
be so likewise even more in such as are not indifferent; for surely he
will have a dread of falsehood as base, since he shunned it even in
itself: and he that is of such a character is praiseworthy, yet he leans
rather to that which is below the truth, this having an appearance of
being in better taste because exaggerations are so annoying.

As for the man who lays claim to things above what really belongs to him
_without_ any special motive, he is like a base man because he would
not otherwise have taken pleasure in falsehood, but he shows as a fool
rather than as a knave. But if a man does this _with_ a special motive,
suppose for honour or glory, as the Braggart does, then he is not
so very blameworthy, but if, directly or indirectly, for pecuniary
considerations, he is more unseemly.

Now the Braggart is such not by his power but by his purpose, that is to
say, in virtue of his moral state, and because he is a man of a certain
kind; just as there are liars who take pleasure in falsehood for its
own sake while others lie from a desire of glory or gain. They who
exaggerate with a view to glory pretend to such qualities as are
followed by praise or highest congratulation; they who do it with a view
to gain assume those which their neighbours can avail themselves of,
and the absence of which can be concealed, as a man's being a skilful
soothsayer or physician; and accordingly most men pretend to such things
and exaggerate in this direction, because the faults I have mentioned
are in them.

The Reserved, who depreciate their own qualities, have the appearance of
being more refined in their characters, because they are not thought to
speak with a view to gain but to avoid grandeur: one very common trait
in such characters is their denying common current opinions, as Socrates
used to do. There are people who lay claim falsely to small things and
things the falsity of their pretensions to which is obvious; these are
called Factotums and are very despicable.

This very Reserve sometimes shows like Exaggeration; take, for instance,
the excessive plainness of dress affected by the Lacedaemonians: in
fact, both excess and the extreme of deficiency partake of the nature of
Exaggeration. But they who practise Reserve in moderation, and in cases
in which the truth is not very obvious and plain, give an impression of
refinement. Here it is the Exaggerator (as being the worst character)
who appears to be opposed to the Truthful Man.

VIII

[Sidenote:II28a] Next, as life has its pauses and in them admits of
pastime combined with Jocularity, it is thought that in this respect
also there is a kind of fitting intercourse, and that rules may be
prescribed as to the kind of things one should say and the manner of
saying them; and in respect of hearing likewise (and there will be a
difference between the saying and hearing such and such things). It is
plain that in regard to these things also there will be an excess and
defect and a mean.

Now they who exceed in the ridiculous are judged to be Buffoons and
Vulgar, catching at it in any and every way and at any cost, and aiming
rather at raising laughter than at saying what is seemly and at avoiding
to pain the object of their wit. They, on the other hand, who would not
for the world make a joke themselves and are displeased with such as do
are thought to be Clownish and Stern. But they who are Jocular in good
taste are denominated by a Greek term expressing properly ease of
movement, because such are thought to be, as one may say, motions of the
moral character; and as bodies are judged of by their motions so too are
moral characters.

Now as the ridiculous lies on the surface, and the majority of men take
more pleasure than they ought in Jocularity and Jesting, the Buffoons
too get this name of Easy Pleasantry, as if refined and gentlemanlike;
but that they differ from these, and considerably too, is plain from
what has been said.

One quality which belongs to the mean state is Tact: it is
characteristic of a man of Tact to say and listen to such things as are
fit for a good man and a gentleman to say and listen to: for there are
things which are becoming for such a one to say and listen to in the way
of Jocularity, and there is a difference between the Jocularity of the
Gentleman and that of the Vulgarian; and again, between that of the
educated and uneducated man. This you may see from a comparison of the
Old and New Comedy: in the former obscene talk made the fun; in the
latter it is rather innuendo: and this is no slight difference _as
regards decency_.

Well then, are we to characterise him who jests well by his saying what
is becoming a gentleman, or by his avoiding to pain the object of his
wit, or even by his giving him pleasure? or will not such a definition
be vague, since different things are hateful and pleasant to different
men?

Be this as it may, whatever he says such things will he also listen to,
since it is commonly held that a man will do what he will bear to hear:
this must, however, be limited; a man will not do quite all that he will
hear: because jesting is a species of scurrility and there are some
points of scurrility forbidden by law; it may be certain points of
jesting should have been also so forbidden. So then the refined and
gentlemanlike man will bear himself thus as being a law to himself. Such
is the mean character, whether denominated the man of Tact or of Easy
Pleasantry.

But the Buffoon cannot resist the ridiculous, sparing neither himself
nor any one else so that he can but raise his laugh, saying things of
such kind as no man of refinement would say and some which he would not
even tolerate if said by others in his hearing. [Sidenote:1128b] The
Clownish man is for such intercourse wholly useless: inasmuch as
contributing nothing jocose of his own he is savage with all who do.

Yet some pause and amusement in life are generally judged to be
indispensable.

The three mean states which have been described do occur in life, and
the object-matter of all is interchange of words and deeds. They differ,
in that one of them is concerned with truth, and the other two with the
pleasurable: and of these two again, the one is conversant with
the jocosities of life, the other with all other points of social
intercourse.

IX

To speak of Shame as a Virtue is incorrect, because it is much more like
a feeling than a moral state. It is defined, we know, to be "a kind of
fear of disgrace," and its effects are similar to those of the fear of
danger, for they who feel Shame grow red and they who fear death turn
pale. So both are evidently in a way physical, which is thought to be a
mark of a feeling rather than a moral state.

Moreover, it is a feeling not suitable to every age, but only to youth:
we do think that the young should be Shamefaced, because since they live
at the beck and call of passion they do much that is wrong and Shame
acts on them as a check. In fact, we praise such young men as are
Shamefaced, but no one would ever praise an old man for being given
to it, inasmuch as we hold that he ought not to do things which cause
Shame; for Shame, since it arises at low bad actions, does not at all
belong to the good man, because such ought not to be done at all: nor
does it make any difference to allege that some things are disgraceful
really, others only because they are thought so; for neither should be
done, so that a man ought not to be in the position of feeling Shame. In
truth, to be such a man as to do anything disgraceful is the part of a
faulty character. And for a man to be such that he would feel Shame if
he should do anything disgraceful, and to think that this constitutes
him a good man, is absurd: because Shame is felt at voluntary actions
only, and a good man will never voluntarily do what is base.

True it is, that Shame may be good on a certain supposition, as "if a
man should do such things, he would feel Shame:" but then the Virtues
are good in themselves, and not merely in supposed cases. And, granted
that impudence and the not being ashamed to do what is disgraceful is
base, it does not the more follow that it is good for a man to do such
things and feel Shame.

Nor is Self-Control properly a Virtue, but a kind of mixed state:
however, all about this shall be set forth in a future Book.




BOOK V

[Sidenote:1129a] Now the points for our inquiry in respect of Justice
and Injustice are, what kind of actions are their object-matter, and
what kind of a mean state Justice is, and between what points the
abstract principle of it, i.e. the Just, is a mean. And our inquiry
shall be, if you please, conducted in the same method as we have
observed in the foregoing parts of this treatise.

We see then that all men mean by the term Justice a moral state such
that in consequence of it men have the capacity of doing what is
just, and actually do it, and wish it: similarly also with respect to
Injustice, a moral state such that in consequence of it men do unjustly
and wish what is unjust: let us also be content then with these as a
ground-work sketched out.

I mention the two, because the same does not hold with regard to States
whether of mind or body as with regard to Sciences or Faculties: I mean
that whereas it is thought that the same Faculty or Science embraces
contraries, a State will not: from health, for instance, not the
contrary acts are done but the healthy ones only; we say a man walks
healthily when he walks as the healthy man would.

However, of the two contrary states the one may be frequently known from
the other, and oftentimes the states from their subject-matter: if it be
seen clearly what a good state of body is, then is it also seen what a
bad state is, and from the things which belong to a good state of body
the good state itself is seen, and _vice versa_. If, for instance,
the good state is firmness of flesh it follows that the bad state is
flabbiness of flesh; and whatever causes firmness of flesh is connected
with the good state. It follows moreover in general, that if of two
contrary terms the one is used in many senses so also will the other be;
as, for instance, if "the Just," then also "the Unjust." Now Justice and
Injustice do seem to be used respectively in many senses, but, because
the line of demarcation between these is very fine and minute, it
commonly escapes notice that they are thus used, and it is not plain
and manifest as where the various significations of terms are widely
different for in these last the visible difference is great, for
instance, the word [Greek: klehis] is used equivocally to denote the
bone which is under the neck of animals and the instrument with which
people close doors.

Let it be ascertained then in how many senses the term "Unjust man" is
used. Well, he who violates the law, and he who is a grasping man, and
the unequal man, are all thought to be Unjust and so manifestly the Just
man will be, the man who acts according to law, and the equal man "The
Just" then will be the lawful and the equal, and "the Unjust" the
unlawful and the unequal.

[Sidenote:1129b] Well, since the Unjust man is also a grasping man, he
will be so, of course, with respect to good things, but not of every
kind, only those which are the subject-matter of good and bad fortune
and which are in themselves always good but not always to the
individual. Yet men pray for and pursue these things: this they should
not do but pray that things which are in the abstract good may be so
also to them, and choose what is good for themselves.

But the Unjust man does not always choose actually the greater part, but
even sometimes the less; as in the case of things which are simply evil:
still, since the less evil is thought to be in a manner a good and the
grasping is after good, therefore even in this case he is thought to be
a grasping man, i.e. one who strives for more good than fairly falls to
his share: of course he is also an unequal man, this being an inclusive
and common term.

We said that the violator of Law is Unjust, and the keeper of the Law
Just: further, it is plain that all Lawful things are in a manner
Just, because by Lawful we understand what have been defined by the
legislative power and each of these we say is Just. The Laws too give
directions on all points, aiming either at the common good of all, or
that of the best, or that of those in power (taking for the standard
real goodness or adopting some other estimate); in one way we mean by
Just, those things which are apt to produce and preserve happiness and
its ingredients for the social community.

Further, the Law commands the doing the deeds not only of the brave man
(as not leaving the ranks, nor flying, nor throwing away one's arms),
but those also of the perfectly self-mastering man, as abstinence from
adultery and wantonness; and those of the meek man, as refraining from
striking others or using abusive language: and in like manner in respect
of the other virtues and vices commanding some things and forbidding
others, rightly if it is a good law, in a way somewhat inferior if it is
one extemporised.

Now this Justice is in fact perfect Virtue, yet not simply so but as
exercised towards one's neighbour: and for this reason Justice is
thought oftentimes to be the best of the Virtues, and

  "neither Hesper nor the Morning-star
  So worthy of our admiration:"

and in a proverbial saying we express the same;

  "All virtue is in Justice comprehended."

And it is in a special sense perfect Virtue because it is the practice
of perfect Virtue. And perfect it is because he that has it is able to
practise his virtue towards his neighbour and not merely on himself; I
mean, there are many who can practise virtue in the regulation of their
own personal conduct who are wholly unable to do it in transactions with
[Sidenote:1130a] their neighbour. And for this reason that saying of
Bias is thought to be a good one,

  "Rule will show what a man is;"

for he who bears Rule is necessarily in contact with others, i.e. in a
community. And for this same reason Justice alone of all the Virtues is
thought to be a good to others, because it has immediate relation to
some other person, inasmuch as the Just man does what is advantageous to
another, either to his ruler or fellow-subject. Now he is the basest
of men who practises vice not only in his own person but towards his
friends also; but he the best who practises virtue not merely in his
own person but towards his neighbour, for this is a matter of some
difficulty.

However, Justice in this sense is not a part of Virtue but is
co-extensive with Virtue; nor is the Injustice which answers to it a
part of Vice but co-extensive with Vice. Now wherein Justice in this
sense differs from Virtue appears from what has been said: it is the
same really, but the point of view is not the same: in so far as it has
respect to one's neighbour it is Justice, in so far as it is such and
such a moral state it is simply Virtue.

II

But the object of our inquiry is Justice, in the sense in which it is
a part of Virtue (for there is such a thing, as we commonly say), and
likewise with respect to particular Injustice. And of the existence of
this last the following consideration is a proof: there are many vices
by practising which a man acts unjustly, of course, but does not grasp
at more than his share of good; if, for instance, by reason of cowardice
he throws away his shield, or by reason of ill-temper he uses abusive
language, or by reason of stinginess does not give a friend pecuniary
assistance; but whenever he does a grasping action, it is often in the
way of none of these vices, certainly not in all of them, still in
the way of some vice or other (for we blame him), and in the way of
Injustice. There is then some kind of Injustice distinct from that
co-extensive with Vice and related to it as a part to a whole, and some
"Unjust" related to that which is co-extensive with violation of the law
as a part to a whole.

Again, suppose one man seduces a man's wife with a view to gain and
actually gets some advantage by it, and another does the same from
impulse of lust, at an expense of money and damage; this latter will be
thought to be rather destitute of self-mastery than a grasping man, and
the former Unjust but not destitute of self-mastery: now why? plainly
because of his gaining.

Again, all other acts of Injustice we refer to some particular
depravity, as, if a man commits adultery, to abandonment to his
passions; if he deserts his comrade, to cowardice; if he strikes
another, to anger: but if he gains by the act to no other vice than to
Injustice.

[Sidenote:1131b] Thus it is clear that there is a kind of Injustice
different from and besides that which includes all Vice, having the same
name because the definition is in the same genus; for both have their
force in dealings with others, but the one acts upon honour, or wealth,
or safety, or by whatever one name we can include all these things, and
is actuated by pleasure attendant on gain, while the other acts upon all
things which constitute the sphere of the good man's action.

Now that there is more than one kind of Justice, and that there is one
which is distinct from and besides that which is co-extensive with,
Virtue, is plain: we must next ascertain what it is, and what are its
characteristics.

Well, the Unjust has been divided into the unlawful and the unequal, and
the Just accordingly into the lawful and the equal: the aforementioned
Injustice is in the way of the unlawful. And as the unequal and the more
are not the same, but differing as part to whole (because all more is
unequal, but not all unequal more), so the Unjust and the Injustice we
are now in search of are not the same with, but other than, those before
mentioned, the one being the parts, the other the wholes; for this
particular Injustice is a part of the Injustice co-extensive with Vice,
and likewise this Justice of the Justice co-extensive with Virtue.
So that what we have now to speak of is the particular Justice and
Injustice, and likewise the particular Just and Unjust.

Here then let us dismiss any further consideration of the Justice
ranking as co-extensive with Virtue (being the practice of Virtue in all
its bearings towards others), and of the co-relative Injustice (being
similarly the practice of Vice). It is clear too, that we must separate
off the Just and the Unjust involved in these: because one may pretty
well say that most lawful things are those which naturally result in
action from Virtue in its fullest sense, because the law enjoins the
living in accordance with each Virtue and forbids living in accordance
with each Vice. And the producing causes of Virtue in all its bearings
are those enactments which have been made respecting education for
society.

By the way, as to individual education, in respect of which a man is
simply good without reference to others, whether it is the province of
[Greek: politikhae] or some other science we must determine at a
future time: for it may be it is not the same thing to be a good man and
a good citizen in every case.

Now of the Particular Justice, and the Just involved in it, one species
is that which is concerned in the distributions of honour, or wealth, or
such other things as are to be shared among the members of the social
community (because in these one man as compared with another may have
either an equal or an unequal share), and the other is that which is
Corrective in the various transactions between man and man.

[Sidenote: 1131a] And of this latter there are two parts: because of
transactions some are voluntary and some involuntary; voluntary, such as
follow; selling, buying, use, bail, borrowing, deposit, hiring: and this
class is called voluntary because the origination of these transactions
is voluntary.

The involuntary again are either such as effect secrecy; as theft,
adultery, poisoning, pimping, kidnapping of slaves, assassination, false
witness; or accompanied with open violence; as insult, bonds, death,
plundering, maiming, foul language, slanderous abuse.

III

Well, the unjust man we have said is unequal, and the abstract "Unjust"
unequal: further, it is plain that there is some mean of the unequal,
that is to say, the equal or exact half (because in whatever action
there is the greater and the less there is also the equal, i.e. the
exact half). If then the Unjust is unequal the Just is equal, which all
must allow without further proof: and as the equal is a mean the Just
must be also a mean. Now the equal implies two terms at least: it
follows then that the Just is both a mean and equal, and these to
certain persons; and, in so far as it is a mean, between certain things
(that is, the greater and the less), and, so far as it is equal, between
two, and in so far as it is just it is so to certain persons. The Just
then must imply four terms at least, for those to which it is just are
two, and the terms representing the things are two.

And there will be the same equality between the terms representing the
persons, as between those representing the things: because as the latter
are to one another so are the former: for if the persons are not equal
they must not have equal shares; in fact this is the very source of all
the quarrelling and wrangling in the world, when either they who are
equal have and get awarded to them things not equal, or being not equal
those things which are equal. Again, the necessity of this equality of
ratios is shown by the common phrase "according to rate," for all agree
that the Just in distributions ought to be according to some rate:
but what that rate is to be, all do not agree; the democrats are for
freedom, oligarchs for wealth, others for nobleness of birth, and the
aristocratic party for virtue.

The Just, then, is a certain proportionable thing. For proportion does
not apply merely to number in the abstract, but to number generally,
since it is equality of ratios, and implies four terms at least (that
this is the case in what may be called discrete proportion is plain and
obvious, but it is true also in continual proportion, for this uses the
one [Sidenote: 1131b] term as two, and mentions it twice; thus A:B:C may
be expressed A:B::B:C. In the first, B is named twice; and so, if, as
in the second, B is actually written twice, the proportionals will be
four): and the Just likewise implies four terms at the least, and the
ratio between the two pair of terms is the same, because the persons and
the things are divided similarly. It will stand then thus, A:B::C:D, and
then permutando A:C::B:D, and then (supposing C and D to represent the
things) A+C:B+D::A:B. The distribution in fact consisting in putting
together these terms thus: and if they are put together so as to
preserve this same ratio, the distribution puts them together justly. So
then the joining together of the first and third and second and fourth
proportionals is the Just in the distribution, and this Just is the
mean relatively to that which violates the proportionate, for
the proportionate is a mean and the Just is proportionate. Now
mathematicians call this kind of proportion geometrical: for in
geometrical proportion the whole is to the whole as each part to each
part. Furthermore this proportion is not continual, because the person
and thing do not make up one term.

The Just then is this proportionate, and the Unjust that which violates
the proportionate; and so there comes to be the greater and the less:
which in fact is the case in actual transactions, because he who acts
unjustly has the greater share and he who is treated unjustly has the
less of what is good: but in the case of what is bad this is reversed:
for the less evil compared with the greater comes to be reckoned for
good, because the less evil is more choiceworthy than the greater, and
what is choiceworthy is good, and the more so the greater good.

This then is the one species of the Just.

IV

And the remaining one is the Corrective, which arises in voluntary as
well as involuntary transactions. Now this just has a different form
from the aforementioned; for that which is concerned in distribution of
common property is always according to the aforementioned proportion: I
mean that, if the division is made out of common property, the
shares will bear the same proportion to one another as the original
contributions did: and the Unjust which is opposite to this Just is that
which violates the proportionate.

But the Just which arises in transactions between men is an equal in a
certain sense, and the Unjust an unequal, only not in the way of that
proportion but of arithmetical. [Sidenote: 1132a ] Because it makes no
difference whether a robbery, for instance, is committed by a good man
on a bad or by a bad man on a good, nor whether a good or a bad man has
committed adultery: the law looks only to the difference created by the
injury and treats the men as previously equal, where the one does and
the other suffers injury, or the one has done and the other suffered
harm. And so this Unjust, being unequal, the judge endeavours to reduce
to equality again, because really when the one party has been wounded
and the other has struck him, or the one kills and the other dies, the
suffering and the doing are divided into unequal shares; well, the judge
tries to restore equality by penalty, thereby taking from the gain.

For these terms gain and loss are applied to these cases, though perhaps
the term in some particular instance may not be strictly proper, as
gain, for instance, to the man who has given a blow, and loss to him who
has received it: still, when the suffering has been estimated, the one
is called loss and the other gain.

And so the equal is a mean between the more and the less, which
represent gain and loss in contrary ways (I mean, that the more of good
and the less of evil is gain, the less of good and the more of evil is
loss): between which the equal was stated to be a mean, which equal we
say is Just: and so the Corrective Just must be the mean between loss
and gain. And this is the reason why, upon a dispute arising, men have
recourse to the judge: going to the judge is in fact going to the Just,
for the judge is meant to be the personification of the Just. And men
seek a judge as one in the mean, which is expressed in a name given by
some to judges ([Greek: mesidioi], or middle-men) under the notion that
if they can hit on the mean they shall hit on the Just. The Just is then
surely a mean since the judge is also.

So it is the office of a judge to make things equal, and the line, as it
were, having been unequally divided, he takes from the greater part that
by which it exceeds the half, and adds this on to the less. And when the
whole is divided into two exactly equal portions then men say they have
their own, when they have gotten the equal; and the equal is a mean
between the greater and the less according to arithmetical equality.

This, by the way, accounts for the etymology of the term by which we
in Greek express the ideas of Just and Judge; ([Greek: dikaion] quasi
[Greek: dichaion], that is in two parts, and [Greek: dikastaes] quasi
[Greek: dichastaes], he who divides into two parts). For when from one
of two equal magnitudes somewhat has been taken and added to the other,
this latter exceeds the former by twice that portion: if it had been
merely taken from the former and not added to the latter, then the
latter would [Sidenote:1132b] have exceeded the former only by that one
portion; but in the other case, the greater exceeds the mean by one, and
the mean exceeds also by one that magnitude from which the portion was
taken. By this illustration, then, we obtain a rule to determine what
one ought to take from him who has the greater, and what to add to him
who has the less. The excess of the mean over the less must be added to
the less, and the excess of the greater over the mean be taken from the
greater.

Thus let there be three straight lines equal to one another. From one of
them cut off a portion, and add as much to another of them. The whole
line thus made will exceed the remainder of the first-named line, by
twice the portion added, and will exceed the untouched line by that
portion. And these terms loss and gain are derived from voluntary
exchange: that is to say, the having more than what was one's own is
called gaining, and the having less than one's original stock is called
losing; for instance, in buying or selling, or any other transactions
which are guaranteed by law: but when the result is neither more nor
less, but exactly the same as there was originally, people say they have
their own, and neither lose nor gain.

So then the Just we have been speaking of is a mean between loss and
gain arising in involuntary transactions; that is, it is the having the
same after the transaction as one had before it took place.

[Sidenote: V] There are people who have a notion that Reciprocation is
simply just, as the Pythagoreans said: for they defined the Just simply
and without qualification as "That which reciprocates with another." But
this simple Reciprocation will not fit on either to the Distributive
Just, or the Corrective (and yet this is the interpretation they put
on the Rhadamanthian rule of Just, If a man should suffer what he hath
done, then there would be straightforward justice"), for in many
cases differences arise: as, for instance, suppose one in authority
has struck a man, he is not to be struck in turn; or if a man has
struck one in authority, he must not only be struck but punished also.
And again, the voluntariness or involuntariness of actions makes a
great difference.

[Sidenote: II33_a_] But in dealings of exchange such a principle of
Justice as this Reciprocation forms the bond of union, but then it must
be Reciprocation according to proportion and not exact equality, because
by proportionate reciprocity of action the social community is held
together, For either Reciprocation of evil is meant, and if this be
not allowed it is thought to be a servile condition of things: or else
Reciprocation of good, and if this be not effected then there is no
admission to participation which is the very bond of their union.

And this is the moral of placing the Temple of the Graces ([Greek:
charites]) in the public streets; to impress the notion that there may
be requital, this being peculiar to [Greek: charis] because a man ought
to requite with a good turn the man who has done him a favour and then
to become himself the originator of another [Greek: charis], by doing
him a favour.

Now the acts of mutual giving in due proportion may be represented
by the diameters of a parallelogram, at the four angles of which the
parties and their wares are so placed that the side connecting the
parties be opposite to that connecting the wares, and each party be
connected by one side with his own ware, as in the accompanying diagram.

[Illustration: Builder_Shoemaker House_Shoes.]

The builder is to receive from the shoemaker of his ware, and to give
him of his own: if then there be first proportionate equality, and
_then_ the Reciprocation takes place, there will be the just result
which we are speaking of: if not, there is not the equal, nor will the
connection stand: for there is no reason why the ware of the one may not
be better than that of the other, and therefore before the exchange is
made they must have been equalised. And this is so also in the other
arts: for they would have been destroyed entirely if there were not a
correspondence in point of quantity and quality between the producer and
the consumer. For, we must remember, no dealing arises between two of
the same kind, two physicians, for instance; but say between a physician
and agriculturist, or, to state it generally, between those who are
different and not equal, but these of course must have been equalised
before the exchange can take place.

It is therefore indispensable that all things which can be exchanged
should be capable of comparison, and for this purpose money has come
in, and comes to be a kind of medium, for it measures all things and so
likewise the excess and defect; for instance, how many shoes are equal
to a house or a given quantity of food. As then the builder to the
shoemaker, so many shoes must be to the house (or food, if instead of a
builder an agriculturist be the exchanging party); for unless there is
this proportion there cannot be exchange or dealing, and this proportion
cannot be unless the terms are in some way equal; hence the need, as was
stated above, of some one measure of all things. Now this is really
and truly the Demand for them, which is the common bond of all such
dealings. For if the parties were not in want at all or not similarly of
one another's wares, there would either not be any exchange, or at least
not the same.

And money has come to be, by general agreement, a representative of
Demand: and the account of its Greek name [Greek: nomisma] is this, that
it is what it is not naturally but by custom or law ([Greek: nomos]),
and it rests with us to change its value, or make it wholly useless.

[Sidenote: 1113b] Very well then, there will be Reciprocation when
the terms have been equalised so as to stand in this proportion;
Agriculturist : Shoemaker : : wares of Shoemaker : wares of
Agriculturist; but you must bring them to this form of proportion when
they exchange, otherwise the one extreme will combine both exceedings of
the mean: but when they have exactly their own then they are equal and
have dealings, because the same equality can come to be in their case.
Let A represent an agriculturist, C food, B a shoemaker, D his wares
equalised with A's. Then the proportion will be correct, A:B::C:D; _now_
Reciprocation will be practicable, if it were not, there would have been
no dealing.

Now that what connects men in such transactions is Demand, as being some
one thing, is shown by the fact that, when either one does not want the
other or neither want one another, they do not exchange at all: whereas
they do when one wants what the other man has, wine for instance, giving
in return corn for exportation.

And further, money is a kind of security to us in respect of exchange
at some future time (supposing that one wants nothing now that we shall
have it when we do): the theory of money being that whenever one brings
it one can receive commodities in exchange: of course this too is liable
to depreciation, for its purchasing power is not always the same,
but still it is of a more permanent nature than the commodities it
represents. And this is the reason why all things should have a price
set upon them, because thus there may be exchange at any time, and if
exchange then dealing. So money, like a measure, making all things
commensurable equalises them: for if there was not exchange there would
not have been dealing, nor exchange if there were not equality, nor
equality if there were not the capacity of being commensurate: it
is impossible that things so greatly different should be really
commensurate, but we can approximate sufficiently for all practical
purposes in reference to Demand. The common measure must be some one
thing, and also from agreement (for which reason it is called [Greek:
nomisma]), for this makes all things commensurable: in fact, all things
are measured by money. Let B represent ten minæ, A a house worth five
minæ, or in other words half B, C a bed worth 1/10th of B: it is clear
then how many beds are equal to one house, namely, five.

It is obvious also that exchange was thus conducted before the existence
of money: for it makes no difference whether you give for a house five
beds or the price of five beds. We have now said then what the abstract
Just and Unjust are, and these having been defined it is plain that
just acting is a mean between acting unjustly and being acted unjustly
towards: the former being equivalent to having more, and the latter to
having less.

But Justice, it must be observed, is a mean state not after the same
manner as the forementioned virtues, but because it aims at producing
the mean, while Injustice occupies _both_ the extremes.

[Sidenote: 1134_a_] And Justice is the moral state in virtue of which
the just man is said to have the aptitude for practising the Just in
the way of moral choice, and for making division between _, himself and
another, or between two other men, not so as to give to himself the
greater and to his neighbour the less share of what is choiceworthy and
contrariwise of what is hurtful, but what is proportionably equal, and
in like manner when adjudging the rights of two other men.

Injustice is all this with respect to the Unjust: and since the Unjust
is excess or defect of what is good or hurtful respectively, in
violation of the proportionate, therefore Injustice is both excess and
defect because it aims at producing excess and defect; excess, that is,
in a man's own case of what is simply advantageous, and defect of what
is hurtful: and in the case of other men in like manner generally
speaking, only that the proportionate is violated not always in one
direction as before but whichever way it happens in the given case. And
of the Unjust act the less is being acted unjustly towards, and the
greater the acting unjustly towards others.

Let this way of describing the nature of Justice and Injustice, and
likewise the Just and the Unjust generally, be accepted as sufficient.

[Sidenote: VI] Again, since a man may do unjust acts and not yet have
formed a character of injustice, the question arises whether a man is
unjust in each particular form of injustice, say a thief, or adulterer,
or robber, by doing acts of a given character.

We may say, I think, that this will not of itself make any difference; a
man may, for instance, have had connection with another's wife, knowing
well with whom he was sinning, but he may have done it not of deliberate
choice but from the impulse of passion: of course he acts unjustly, but
he has not necessarily formed an unjust character: that is, he may have
stolen yet not be a thief; or committed an act of adultery but still not
be an adulterer, and so on in other cases which might be enumerated.

Of the relation which Reciprocation bears to the Just we have already
spoken: and here it should be noticed that the Just which we are
investigating is both the Just in the abstract and also as exhibited in
Social Relations, which latter arises in the case of those who live in
communion with a view to independence and who are free and equal either
proportionately or numerically.

It follows then that those who are not in this position have not among
themselves the Social Just, but still Just of some kind and resembling
that other. For Just implies mutually acknowledged law, and law the
possibility of injustice, for adjudication is the act of distinguishing
between the Just and the Unjust.

And among whomsoever there is the possibility of injustice among these
there is that of acting unjustly; but it does not hold conversely that
injustice attaches to all among whom there is the possibility of acting
unjustly, since by the former we mean giving one's self the larger share
of what is abstractedly good and the less of what is abstractedly evil.

[Sidenote: 134_b_] This, by the way, is the reason why we do not allow
a man to govern, but Principle, because a man governs for himself and
comes to be a despot: but the office of a ruler is to be guardian of the
Just and therefore of the Equal. Well then, since he seems to have no
peculiar personal advantage, supposing him a Just man, for in this case
he does not allot to himself the larger share of what is abstractedly
good unless it falls to his share proportionately (for which reason he
really governs for others, and so Justice, men say, is a good not to
one's self so much as to others, as was mentioned before), therefore
some compensation must be given him, as there actually is in the shape
of honour and privilege; and wherever these are not adequate there
rulers turn into despots.

But the Just which arises in the relations of Master and Father, is not
identical with, but similar to, these; because there is no possibility
of injustice towards those things which are absolutely one's own; and
a slave or child (so long as this last is of a certain age and not
separated into an independent being), is, as it were, part of a man's
self, and no man chooses to hurt himself, for which reason there cannot
be injustice towards one's own self: therefore neither is there the
social Unjust or Just, which was stated to be in accordance with law and
to exist between those among whom law naturally exists, and these were
said to be they to whom belongs equality of ruling and being ruled.

Hence also there is Just rather between a man and his wife than between
a man and his children or slaves; this is in fact the Just arising in
domestic relations: and this too is different from the Social Just.

[Sidenote: VII] Further, this last-mentioned Just is of two kinds,
natural and conventional; the former being that which has everywhere the
same force and does not depend upon being received or not; the latter
being that which originally may be this way or that indifferently but
not after enactment: for instance, the price of ransom being fixed at
a mina, or the sacrificing a goat instead of two sheep; and again, all
cases of special enactment, as the sacrificing to Brasidas as a hero; in
short, all matters of special decree.

But there are some men who think that all the Justs are of this latter
kind, and on this ground: whatever exists by nature, they say, is
unchangeable and has everywhere the same force; fire, for instance,
burns not here only but in Persia as well, but the Justs they see
changed in various places.

Now this is not really so, and yet it is in a way (though among the gods
perhaps by no means): still even amongst ourselves there is somewhat
existing by nature: allowing that everything is subject to change, still
there is that which does exist by nature, and that which does not.

Nay, we may go further, and say that it is practically plain what among
things which can be otherwise does exist by nature, and what does not
but is dependent upon enactment and conventional, even granting
that both are alike subject to be changed: and the same distinctive
illustration will apply to this and other cases; the right hand is
naturally the stronger, still some men may become equally strong in
both.

[Sidenote: 1135_a_] A parallel may be drawn between the Justs which
depend upon convention and expedience, and measures; for wine and corn
measures are not equal in all places, but where men buy they are large,
and where these same sell again they are smaller: well, in like manner
the Justs which are not natural, but of human invention, are not
everywhere the same, for not even the forms of government are, and yet
there is one only which by nature would be best in all places.

Now of Justs and Lawfuls each bears to the acts which embody and
exemplify it the relation of an universal to a particular; the acts
being many, but each of the principles only singular because each is an
universal. And so there is a difference between an unjust act and the
abstract Unjust, and the just act and the abstract Just: I mean, a thing
is unjust in itself, by nature or by ordinance; well, when this has been
embodied in act, there is an unjust act, but not till then, only
some unjust thing. And similarly of a just act. (Perhaps [Greek:
dikaiopragaema] is more correctly the common or generic term for just
act, the word [Greek: dikaioma], which I have here used, meaning
generally and properly the act corrective of the unjust act.) Now as
to each of them, what kinds there are, and how many, and what is their
object-matter, we must examine afterwards.

[Sidenote: VIII] For the present we proceed to say that, the Justs
and the Unjusts being what have been mentioned, a man is said to act
unjustly or justly when he embodies these abstracts in voluntary
actions, but when in involuntary, then he neither acts unjustly or
justly except accidentally; I mean that the being just or unjust is
really only accidental to the agents in such cases.

So both unjust and just actions are limited by the being voluntary or
the contrary: for when an embodying of the Unjust is voluntary, then
it is blamed and is at the same time also an unjust action: but, if
voluntariness does not attach, there will be a thing which is in itself
unjust but not yet an unjust action.

By voluntary, I mean, as we stated before, whatsoever of things in his
own power a man does with knowledge, and the absence of ignorance as to
the person to whom, or the instrument with which, or the result with
which he does; as, for instance, whom he strikes, what he strikes him
with, and with what probable result; and each of these points again, not
accidentally nor by compulsion; as supposing another man were to seize
his hand and strike a third person with it, here, of course, the owner
of the hand acts not voluntarily, because it did not rest with him to do
or leave undone: or again, it is conceivable that the person struck may
be his father, and he may know that it is a man, or even one of the
present company, whom he is striking, but not know that it is his
father. And let these same distinctions be supposed to be carried into
the case of the result and in fact the whole of any given action. In
fine then, that is involuntary which is done through ignorance, or
which, not resulting from ignorance, is not in the agent's control or is
done on compulsion.

I mention these cases, because there are many natural *[Sidenote:
1135_b_] things which we do and suffer knowingly but still no one of
which is either voluntary or involuntary, growing old, or dying, for
instance.

Again, accidentality may attach to the unjust in like manner as to the
just acts. For instance, a man may have restored what was deposited
with him, but against his will and from fear of the consequences of
a refusal: we must not say that he either does what is just, or does
justly, except accidentally: and in like manner the man who through
compulsion and against his will fails to restore a deposit, must be said
to do unjustly, or to do what is unjust, accidentally only.

Again, voluntary actions we do either from deliberate choice or without
it; from it, when we act from previous deliberation; without it, when
without any previous deliberation. Since then hurts which may be done in
transactions between man and man are threefold, those mistakes which are
attended with ignorance are, when a man either does a thing not to the
man to whom he meant to do it, or not the thing he meant to do, or not
with the instrument, or not with the result which he intended: either he
did not think he should hit him at all, or not with this, or this is not
the man he thought he should hit, or he did not think this would be
the result of the blow but a result has followed which he did not
anticipate; as, for instance, he did it not to wound but merely to prick
him; or it is not the man whom, or the way in which, he meant.

Now when the hurt has come about contrary to all reasonable expectation,
it is a Misadventure; when though not contrary to expectation yet
without any viciousness, it is a Mistake; for a man makes a mistake when
the origination of the cause rests with himself, he has a misadventure
when it is external to himself. When again he acts with knowledge, but
not from previous deliberation, it is an unjust action; for instance,
whatever happens to men from anger or other passions which are necessary
or natural: for when doing these hurts or making these mistakes they act
unjustly of course and their actions are unjust, still they are not yet
confirmed unjust or wicked persons by reason of these, because the hurt
did not arise from depravity in the doer of it: but when it does arise
from deliberate choice, then the doer is a confirmed unjust and depraved
man.

And on this principle acts done from anger are fairly judged not to be
from malice prepense, because it is not the man who acts in wrath who
is the originator really but he who caused his wrath. And again,
the question at issue in such cases is not respecting the fact but
respecting the justice of the case, the occasion of anger being a notion
of injury. I mean, that the parties do not dispute about the fact, as in
questions of contract (where one of the two must be a rogue, unless real
forgetfulness can be pleaded), but, admitting the fact, they dispute on
which side the justice of the case lies (the one who plotted against the
other, _i.e._ the real aggressor, of course, cannot be ignorant), so
that the one thinks there is injustice committed while the other does
not.

[Sidenote: 11364] Well then, a man acts unjustly if he has hurt another
of deliberate purpose, and he who commits such acts of injustice is
_ipso facto_ an unjust character when they are in violation of the
proportionate or the equal; and in like manner also a man is a just
character when he acts justly of deliberate purpose, and he does act
justly if he acts voluntarily.

Then as for involuntary acts of harm, they are either such as are
excusable or such as are not: under the former head come all errors done
not merely in ignorance but from ignorance; under the latter all that
are done not from ignorance but in ignorance caused by some passion
which is neither natural nor fairly attributable to human infirmity.

[Sidenote: IX] Now a question may be raised whether we have spoken with
sufficient distinctness as to being unjustly dealt with, and dealing
unjustly towards others. First, whether the case is possible which
Euripides has put, saying somewhat strangely,

  "My mother he hath slain;  the tale is short,
  Either he willingly did slay her willing,
  Or else with her will but against his own."

I mean then, is it really possible for a person to be unjustly dealt
with with his own consent, or must every case of being unjustly dealt
with be against the will of the sufferer as every act of unjust dealing
is voluntary?

And next, are cases of being unjustly dealt with to be ruled all one way
as every act of unjust dealing is voluntary? or may we say that some
cases are voluntary and some involuntary?

Similarly also as regards being justly dealt with: all just acting is
voluntary, so that it is fair to suppose that the being dealt with
unjustly or justly must be similarly opposed, as to being either
voluntary or involuntary.

Now as for being justly dealt with, the position that every case of this
is voluntary is a strange one, for some are certainly justly dealt
with without their will. The fact is a man may also fairly raise this
question, whether in every case he who has suffered what is unjust is
therefore unjustly dealt with, or rather that the case is the same with
suffering as it is with acting; namely that in both it is possible to
participate in what is just, but only accidentally. Clearly the case of
what is unjust is similar: for doing things in themselves unjust is not
identical with acting unjustly, nor is suffering them the same as being
unjustly dealt with. So too of acting justly and being justly dealt
with, since it is impossible to be unjustly dealt with unless some one
else acts unjustly or to be justly dealt with unless some one else acts
justly.

Now if acting unjustly is simply "hurting another voluntarily" (by which
I mean, knowing whom you are hurting, and wherewith, and how you are
hurting him), and the man who fails of self-control voluntarily hurts
himself, then this will be a case of being voluntarily dealt unjustly
with, and it will be possible for a man to deal unjustly with himself.
(This by the way is one of the questions raised, whether it is possible
for a man to deal unjustly with himself.) Or again, a man may, by
reason of failing of self-control, receive hurt from another man acting
voluntarily, and so here will be another case of being unjustly dealt
with voluntarily. [Sidenote: 1136]

The solution, I take it, is this: the definition of being unjustly dealt
with is not correct, but we must add, to the hurting with the knowledge
of the person hurt and the instrument and the manner of hurting him, the
fact of its being against the wish of the man who is hurt.

So then a man may be hurt and suffer what is in itself unjust
voluntarily, but unjustly dealt with voluntarily no man can be: since no
man wishes to be hurt, not even he who fails of self-control, who really
acts contrary to his wish: for no man wishes for that which he does not
_think_ to be good, and the man who fails of self-control does not what
he thinks he ought to do.

And again, he that gives away his own property (as Homer says Glaucus
gave to Diomed, "armour of gold for brass, armour worth a hundred oxen
for that which was worth but nine") is not unjustly dealt with, because
the giving rests entirely with himself; but being unjustly dealt with
does not, there must be some other person who is dealing unjustly
towards him.

With respect to being unjustly dealt with then, it is clear that it is
not voluntary.

There remain yet two points on which we purposed to speak: first, is he
chargeable with an unjust act who in distribution has _given_ the larger
share to one party contrary to the proper rate, or he that _has_ the
larger share? next, can a man deal unjustly by himself?

In the first question, if the first-named alternative is possible and
it is the distributor who acts unjustly and not he who has the larger
share, then supposing that a person knowingly and willingly gives more
to another than to himself here is a case of a man dealing unjustly by
himself; which, in fact, moderate men are thought to do, for it is a
characteristic of the equitable man to take less than his due.

Is not this the answer? that the case is not quite fairly stated,
because of some other good, such as credit or the abstract honourable,
in the supposed case the man did get the larger share. And again, the
difficulty is solved by reference to the definition of unjust dealing:
for the man suffers nothing contrary to his own wish, so that, on this
score at least, he is not unjustly dealt with, but, if anything, he is
hurt only.

It is evident also that it is the distributor who acts unjustly and not
the man who has the greater share: because the mere fact of the abstract
Unjust attaching to what a man does, does not constitute unjust action,
but the doing this voluntarily: and voluntariness attaches to that
quarter whence is the origination of the action, which clearly is in the
distributor not in the receiver. And again the term doing is used in
several senses; in one sense inanimate objects kill, or the hand, or
the slave by his master's bidding; so the man in question does not act
unjustly but does things which are in themselves unjust.

[Sidenote: 1137a] Again, suppose that a man has made a wrongful award
in ignorance; in the eye of the law he does not act unjustly nor is
his awarding unjust, but yet he is in a certain sense: for the Just
according to law and primary or natural Just are not coincident: but, if
he knowingly decided unjustly, then he himself as well as the receiver
got the larger share, that is, either of favour from the receiver or
private revenge against the other party: and so the man who decided
unjustly from these motives gets a larger share, in exactly the same
sense as a man would who received part of the actual matter of the
unjust action: because in this case the man who wrongly adjudged, say a
field, did not actually get land but money by his unjust decision.

Now men suppose that acting Unjustly rests entirely with themselves,
and conclude that acting Justly is therefore also easy. But this is not
really so; to have connection with a neighbour's wife, or strike one's
neighbour, or give the money with one's hand, is of course easy and
rests with one's self: but the doing these acts with certain inward
dispositions neither is easy nor rests entirely with one's self. And in
like way, the knowing what is Just and what Unjust men think no great
instance of wisdom because it is not hard to comprehend those things
of which the laws speak. They forget that these are not Just actions,
except accidentally: to be Just they must be done and distributed in
a certain manner: and this is a more difficult task than knowing what
things are wholesome; for in this branch of knowledge it is an easy
matter to know honey, wine, hellebore, cautery, or the use of the knife,
but the knowing how one should administer these with a view to health,
and to whom and at what time, amounts in fact to being a physician.

From this very same mistake they suppose also, that acting Unjustly is
equally in the power of the Just man, for the Just man no less, nay even
more, than the Unjust, may be able to do the particular acts; he may be
able to have intercourse with a woman or strike a man; or the brave man
to throw away his shield and turn his back and run this way or that.
True: but then it is not the mere doing these things which constitutes
acts of cowardice or injustice (except accidentally), but the doing them
with certain inward dispositions: just as it is not the mere using or
not using the knife, administering or not administering certain drugs,
which constitutes medical treatment or curing, but doing these things in
a certain particular way.

Again the abstract principles of Justice have their province among those
who partake of what is abstractedly good, and can have too much or too
little of these. Now there are beings who cannot have too much of them,
as perhaps the gods; there are others, again, to whom no particle of
them is of use, those who are incurably wicked to whom all things are
hurtful; others to whom they are useful to a certain degree: for this
reason then the province of Justice is among Men.

[Sidenote: 1137b] We have next to speak of Equity and the Equitable,
that is to say, of the relations of Equity to Justice and the Equitable
to the Just; for when we look into the matter the two do not appear
identical nor yet different in kind; and we sometimes commend the
Equitable and the man who embodies it in his actions, so that by way of
praise we commonly transfer the term also to other acts instead of the
term good, thus showing that the more Equitable a thing is the better it
is: at other times following a certain train of reasoning we arrive at a
difficulty, in that the Equitable though distinct from the Just is yet
praiseworthy; it seems to follow either that the Just is not good or the
Equitable not Just, since they are by hypothesis different; or if both
are good then they are identical.

This is a tolerably fair statement of the difficulty which on these
grounds arises in respect of the Equitable; but, in fact, all these may
be reconciled and really involve no contradiction: for the Equitable is
Just, being also better than one form of Just, but is not better than
the Just as though it were different from it in kind: Just and Equitable
then are identical, and, both being good, the Equitable is the better of
the two.

What causes the difficulty is this; the Equitable is Just, but not the
Just which is in accordance with written law, being in fact a correction
of that kind of Just. And the account of this is, that every law is
necessarily universal while there are some things which it is not
possible to speak of rightly in any universal or general statement.
Where then there is a necessity for general statement, while a general
statement cannot apply rightly to all cases, the law takes the
generality of cases, being fully aware of the error thus involved; and
rightly too notwithstanding, because the fault is not in the law, or
in the framer of the law, but is inherent in the nature of the thing,
because the matter of all action is necessarily such.

When then the law has spoken in general terms, and there arises a
case of exception to the general rule, it is proper, in so far as the
lawgiver omits the case and by reason of his universality of statement
is wrong, to set right the omission by ruling it as the lawgiver himself
would rule were he there present, and would have provided by law had he
foreseen the case would arise. And so the Equitable is Just but better
than one form of Just; I do not mean the abstract Just but the error
which arises out of the universality of statement: and this is the
nature of the Equitable, "a correction of Law, where Law is defective by
reason of its universality."

This is the reason why not all things are according to law, because
there are things about which it is simply impossible to lay down a law,
and so we want special enactments for particular cases. For to speak
generally, the rule of the undefined must be itself undefined also, just
as the rule to measure Lesbian building is made of lead: for this rule
shifts according to the form of each stone and the special enactment
according to the facts of the case in question.

[Sidenote: 1138a] It is clear then what the Equitable is; namely that it
is Just but better than one form of Just: and hence it appears too who
the Equitable man is: he is one who has a tendency to choose and carry
out these principles, and who is not apt to press the letter of the law
on the worse side but content to waive his strict claims though backed
by the law: and this moral state is Equity, being a species of Justice,
not a different moral state from Justice.

XI

The answer to the second of the two questions indicated above, "whether
it is possible for a man to deal unjustly by himself," is obvious from
what has been already stated. In the first place, one class of Justs is
those which are enforced by law in accordance with Virtue in the most
extensive sense of the term: for instance, the law does not bid a man
kill himself; and whatever it does not bid it forbids: well, whenever a
man does hurt contrary to the law (unless by way of requital of hurt),
voluntarily, i.e. knowing to whom he does it and wherewith, he acts
Unjustly. Now he that from rage kills himself, voluntarily, does this
in contravention of Right Reason, which the law does not permit. He
therefore acts Unjustly: but towards whom? towards the Community, not
towards himself (because he suffers with his own consent, and no man can
be Unjustly dealt with with his own consent), and on this principle the
Community punishes him; that is a certain infamy is attached to the
suicide as to one who acts Unjustly towards the Community.

Next, a man cannot deal Unjustly by himself in the sense in which a man
is Unjust who only does Unjust acts without being entirely bad (for the
two things are different, because the Unjust man is in a way bad, as the
coward is, not as though he were chargeable with badness in the full
extent of the term, and so he does not act Unjustly in this sense),
because if it were so then it would be possible for the same thing to
have been taken away from and added to the same person: but this is
really not possible, the Just and the Unjust always implying a plurality
of persons.

Again, an Unjust action must be voluntary, done of deliberate purpose,
and aggressive (for the man who hurts because he has first suffered and
is merely requiting the same is not thought to act Unjustly), but here
the man does to himself and suffers the same things at the same time.

Again, it would imply the possibility of being Unjustly dealt with with
one's own consent.

And, besides all this, a man cannot act Unjustly without his act falling
under some particular crime; now a man cannot seduce his own wife,
commit a burglary on his own premises, or steal his own property. After
all, the general answer to the question is to allege what was settled
respecting being Unjustly dealt with with one's own consent.

It is obvious, moreover, that being Unjustly dealt by and dealing
Unjustly by others are both wrong; because the one is having less, the
other having more, than the mean, and the case is parallel to that of
the healthy in the healing art, and that of good condition in the art of
training: but still the dealing Unjustly by others is the worst of the
two, because this involves wickedness and is blameworthy; wickedness, I
mean, either wholly, or nearly so (for not all voluntary wrong implies
injustice), but the being Unjustly dealt by does not involve wickedness
or injustice.

[Sidenote: 1138b] In itself then, the being Unjustly dealt by is the
least bad, but accidentally it may be the greater evil of the two.
However, scientific statement cannot take in such considerations; a
pleurisy, for instance, is called a greater physical evil than a bruise:
and yet this last may be the greater accidentally; it may chance that a
bruise received in a fall may cause one to be captured by the enemy and
slain.

Further: Just, in the way of metaphor and similitude, there may be I do
not say between a man and himself exactly but between certain parts of
his nature; but not Just of every kind, only such as belongs to the
relation of master and slave, or to that of the head of a family. For
all through this treatise the rational part of the Soul has been viewed
as distinct from the irrational.

Now, taking these into consideration, there is thought to be a
possibility of injustice towards one's self, because herein it is
possible for men to suffer somewhat in contradiction of impulses really
their own; and so it is thought that there is Just of a certain kind
between these parts mutually, as between ruler and ruled.

Let this then be accepted as an account of the distinctions which we
recognise respecting Justice and the rest of the moral virtues.




BOOK VI


I having stated in a former part of this treatise that men should choose
the mean instead of either the excess or defect, and that the mean
is according to the dictates of Right Reason; we will now proceed to
explain this term.

For in all the habits which we have expressly mentioned, as likewise
in all the others, there is, so to speak, a mark with his eye fixed on
which the man who has Reason tightens or slacks his rope; and there is a
certain limit of those mean states which we say are in accordance with
Right Reason, and lie between excess on the one hand and defect on the
other.

Now to speak thus is true enough but conveys no very definite meaning:
as, in fact, in all other pursuits requiring attention and diligence on
which skill and science are brought to bear; it is quite true of course
to say that men are neither to labour nor relax too much or too little,
but in moderation, and as Right Reason directs; yet if this were all
a man had he would not be greatly the wiser; as, for instance, if in
answer to the question, what are proper applications to the body, he
were to be told, "Oh! of course, whatever the science of medicine, and
in such manner as the physician, directs."

And so in respect of the mental states it is requisite not merely that
this should be true which has been already stated, but further that it
should be expressly laid down what Right Reason is, and what is the
definition of it.

[Sidenote: 1139a] Now in our division of the Excellences of the Soul, we
said there were two classes, the Moral and the Intellectual: the former
we have already gone through; and we will now proceed to speak of the
others, premising a few words respecting the Soul itself. It was
stated before, you will remember, that the Soul consists of two parts,
the Rational, and Irrational: we must now make a similar division of the
Rational.

Let it be understood then that there are two parts of the Soul possessed
of Reason; one whereby we realise those existences whose causes cannot
be otherwise than they are, and one whereby we realise those which can
be otherwise than they are (for there must be, answering to things
generically different, generically different parts of the soul naturally
adapted to each, since these parts of the soul possess their knowledge
in virtue of a certain resemblance and appropriateness in themselves to
the objects of which they are percipients); and let us name the
former, "that which is apt to know," the latter, "that which is apt to
calculate" (because deliberating and calculating are the same, and no
one ever deliberates about things which cannot be otherwise than they
are: and so the Calculative will be one part of the Rational faculty of
the soul).

We must discover, then, which is the best state of each of these,
because that will be the Excellence of each; and this again is relative
to the work each has to do.

II

There are in the Soul three functions on which depend moral action and
truth; Sense, Intellect, Appetition, whether vague Desire or definite
Will. Now of these Sense is the originating cause of no moral action, as
is seen from the fact that brutes have Sense but are in no way partakers
of moral action.

[Intellect and Will are thus connected,] what in the Intellectual
operation is Affirmation and Negation that in the Will is Pursuit and
Avoidance, And so, since Moral Virtue is a State apt to exercise Moral
Choice and Moral Choice is Will consequent on deliberation, the Reason
must be true and the Will right, to constitute good Moral Choice, and
what the Reason affirms the Will must pursue. Now this Intellectual
operation and this Truth is what bears upon Moral Action; of course
truth and falsehood than the conclusion such knowledge as he has will be
merely accidental.

IV

[Sidenote:1140a] Let thus much be accepted as a definition of Knowledge.
Matter which may exist otherwise than it actually does in any given case
(commonly called Contingent) is of two kinds, that which is the object
of Making, and that which is the object of Doing; now Making and Doing
are two different things (as we show in the exoteric treatise), and
so that state of mind, conjoined with Reason, which is apt to Do, is
distinct from that also conjoined with Reason, which is apt to Make: and
for this reason they are not included one by the other, that is, Doing
is not Making, nor Making Doing. Now as Architecture is an Art, and is
the same as "a certain state of mind, conjoined with Reason, which is
apt to Make," and as there is no Art which is not such a state, nor any
such state which is not an Art, Art, in its strict and proper sense,
must be "a state of mind, conjoined with true Reason, apt to Make."

Now all Art has to do with production, and contrivance, and seeing how
any of those things may be produced which may either be or not be, and
the origination of which rests with the maker and not with the thing
made.

And, so neither things which exist or come into being necessarily, nor
things in the way of nature, come under the province of Art, because
these are self-originating. And since Making and Doing are distinct, Art
must be concerned with the former and not the latter. And in a certain
sense Art and Fortune are concerned with the same things, as, Agathon
says by the way,

  "Art Fortune loves, and is of her beloved."

So Art, as has been stated, is "a certain state of mind, apt to Make,
conjoined with true Reason;" its absence, on the contrary, is the same
state conjoined with false Reason, and both are employed upon Contingent
matter.

V

As for Practical Wisdom, we shall ascertain its nature by examining to
what kind of persons we in common language ascribe it.

[Sidenote: 1140b] It is thought then to be the property of the
Practically Wise man to be able to deliberate well respecting what is
good and expedient for himself, not in any definite line, as what is
conducive to health or strength, but what to living well. A proof of
this is that we call men Wise in this or that, when they calculate well
with a view to some good end in a case where there is no definite
rule. And so, in a general way of speaking, the man who is good at
deliberation will be Practically Wise. Now no man deliberates respecting
things which cannot be otherwise than they are, nor such as lie not
within the range of his own action: and so, since Knowledge requires
strict demonstrative reasoning, of which Contingent matter does not
admit (I say Contingent matter, because all matters of deliberation must
be Contingent and deliberation cannot take place with respect to things
which are Necessarily), Practical Wisdom cannot be Knowledge nor Art;
nor the former, because what falls under the province of Doing must be
Contingent; not the latter, because Doing and Making are different in
kind.

It remains then that it must be "a state of mind true, conjoined with
Reason, and apt to Do, having for its object those things which are good
or bad for Man:" because of Making something beyond itself is always the
object, but cannot be of Doing because the very well-doing is in itself
an End.

For this reason we think Pericles and men of that stamp to be
Practically Wise, because they can see what is good for themselves and
for men in general, and we also think those to be such who are skilled
in domestic management or civil government. In fact, this is the reason
why we call the habit of perfected self-mastery by the name which in
Greek it bears, etymologically signifying "that which preserves the
Practical Wisdom:" for what it does preserve is the Notion I have
mentioned, _i.e._ of one's own true interest, For it is not every kind
of Notion which the pleasant and the painful corrupt and pervert, as,
for instance, that "the three angles of every rectilineal triangle are
equal to two right angles," but only those bearing on moral action.

For the Principles of the matters of moral action are the final cause
of them: now to the man who has been corrupted by reason of pleasure or
pain the Principle immediately becomes obscured, nor does he see that it
is his duty to choose and act in each instance with a view to this final
cause and by reason of it: for viciousness has a tendency to destroy the
moral Principle: and so Practical Wisdom must be "a state conjoined with
reason, true, having human good for its object, and apt to do."

Then again Art admits of degrees of excellence, but Practical Wisdom
does not: and in Art he who goes wrong purposely is preferable to him
who does so unwittingly, but not so in respect of Practical Wisdom or
the other Virtues. It plainly is then an Excellence of a certain kind,
and not an Art.

Now as there are two parts of the Soul which have Reason, it must be the
Excellence of the Opinionative [which we called before calculative or
deliberative], because both Opinion and Practical Wisdom are exercised
upon Contingent matter. And further, it is not simply a state conjoined
with Reason, as is proved by the fact that such a state may be forgotten
and so lost while Practical Wisdom cannot.

VI

Now Knowledge is a conception concerning universals and Necessary
matter, and there are of course certain First Principles in all trains
of demonstrative reasoning (that is of all Knowledge because this is
connected with reasoning): that faculty, then, which takes in the first
principles of that which comes under the range of Knowledge, cannot be
either Knowledge, or Art, or Practical Wisdom: not Knowledge, because
what is the object of Knowledge must be derived from demonstrative
reasoning; not either of the other two, because they are exercised upon
Contingent matter only. [Sidenote: 1141a] Nor can it be Science which
takes in these, because the Scientific Man must in some cases depend on
demonstrative Reasoning.

It comes then to this: since the faculties whereby we always attain
truth and are never deceived when dealing with matter Necessary or even
Contingent are Knowledge, Practical Wisdom, Science, and Intuition, and
the faculty which takes in First Principles cannot be any of the three
first; the last, namely Intuition, must be it which performs this
function.

VII

Science is a term we use principally in two meanings: in the first
place, in the Arts we ascribe it to those who carry their arts to the
highest accuracy; Phidias, for instance, we call a Scientific or cunning
sculptor; Polycleitus a Scientific or cunning statuary; meaning, in this
instance, nothing else by Science than an excellence of art: in the
other sense, we think some to be Scientific in a general way, not in any
particular line or in any particular thing, just as Homer says of a man
in his Margites; "Him the Gods made neither a digger of the ground, nor
ploughman, nor in any other way Scientific."

So it is plain that Science must mean the most accurate of all
Knowledge; but if so, then the Scientific man must not merely know the
deductions from the First Principles but be in possession of truth
respecting the First Principles. So that Science must be equivalent
to Intuition and Knowledge; it is, so to speak, Knowledge of the most
precious objects, _with a head on_.

I say of the most precious things, because it is absurd to suppose
[Greek: politikae], or Practical Wisdom, to be the highest, unless it
can be shown that Man is the most excellent of all that exists in the
Universe. Now if "healthy" and "good" are relative terms, differing
when applied to men or to fish, but "white" and "straight" are the same
always, men must allow that the Scientific is the same always, but the
Practically Wise varies: for whatever provides all things well for
itself, to this they would apply the term Practically Wise, and commit
these matters to it; which is the reason, by the way, that they call
some brutes Practically Wise, such that is as plainly have a faculty of
forethought respecting their own subsistence.

And it is quite plain that Science and [Greek: politikae] cannot be
identical: because if men give the name of Science to that faculty which
is employed upon what is expedient for themselves, there will be many
instead of one, because there is not one and the same faculty employed
on the good of all animals collectively, unless in the same sense as you
may say there is one art of healing with respect to all living beings.

[Sidenote: 1141b] If it is urged that man is superior to all other
animals, that makes no difference: for there are many other things more
Godlike in their nature than Man, as, most obviously, the elements of
which the Universe is composed.

It is plain then that Science is the union of Knowledge and Intuition,
and has for its objects those things which are most precious in their
nature. Accordingly, Anexagoras, Thales, and men of that stamp, people
call Scientific, but not Practically Wise because they see them ignorant
of what concerns themselves; and they say that what they know is quite
out of the common run certainly, and wonderful, and hard, and very fine
no doubt, but still useless because they do not seek after what is good
for them as men.

But Practical Wisdom is employed upon human matters, and such as are
objects of deliberation (for we say, that to deliberate well is most
peculiarly the work of the man who possesses this Wisdom), and no man
deliberates about things which cannot be otherwise than they are, nor
about any save those that have some definite End and this End good
resulting from Moral Action; and the man to whom we should give the name
of Good in Counsel, simply and without modification, is he who in the
way of calculation has a capacity for attaining that of practical goods
which is the best for Man. Nor again does Practical Wisdom consist in
a knowledge of general principles only, but it is necessary that one
should know also the particular details, because it is apt to act, and
action is concerned with details: for which reason sometimes men who
have not much knowledge are more practical than others who have; among
others, they who derive all they know from actual experience: suppose a
man to know, for instance, that light meats are easy of digestion and
wholesome, but not what kinds of meat are light, he will not produce a
healthy state; that man will have a much better chance of doing so,
who knows that the flesh of birds is light and wholesome. Since then
Practical Wisdom is apt to act, one ought to have both kinds of
knowledge, or, if only one, the knowledge of details rather than
of Principles. So there will be in respect of Practical Wisdom the
distinction of supreme and subordinate.

VIII

Further: [Greek: politikhae] and Practical Wisdom are the same mental
state, but the point of view is not the same.

Of Practical Wisdom exerted upon a community that which I would call
the Supreme is the faculty of Legislation; the subordinate, which is
concerned with the details, generally has the common name [Greek:
politikhae], and its functions are Action and Deliberation (for the
particular enactment is a matter of action, being the ultimate issue of
this branch of Practical Wisdom, and therefore people commonly say, that
these men alone are really engaged in government, because they alone
act, filling the same place relatively to legislators, that workmen do
to a master).

Again, that is thought to be Practical Wisdom in the most proper sense
which has for its object the interest of the Individual: and this
usually appropriates the common name: the others are called respectively
Domestic Management, Legislation, Executive Government divided into two
branches, Deliberative and Judicial. Now of course, knowledge for
one's self is one kind of knowledge, but it admits of many shades of
difference: and it is a common notion that the man [Sidenote:1142a] who
knows and busies himself about his own concerns merely is the man of
Practical Wisdom, while they who extend their solicitude to society at
large are considered meddlesome.

Euripides has thus embodied this sentiment; "How," says one of his
Characters, "How foolish am I, who whereas I might have shared equally,
idly numbered among the multitude of the army ... for them that are busy
and meddlesome [Jove hates]," because the generality of mankind seek
their own good and hold that this is their proper business. It is then
from this opinion that the notion has arisen that such men are the
Practically-Wise. And yet it is just possible that the good of the
individual cannot be secured independently of connection with a family
or a community. And again, how a man should manage his own affairs is
sometimes not quite plain, and must be made a matter of inquiry.

A corroboration of what I have said is the fact, that the young come to
be geometricians, and mathematicians, and Scientific in such matters,
but it is not thought that a young man can come to be possessed of
Practical Wisdom: now the reason is, that this Wisdom has for its object
particular facts, which come to be known from experience, which a young
man has not because it is produced only by length of time.

By the way, a person might also inquire why a boy may be made a
mathematician but not Scientific or a natural philosopher. Is not this
the reason? that mathematics are taken in by the process of abstraction,
but the principles of Science and natural philosophy must be gained by
experiment; and the latter young men talk of but do not realise, while
the nature of the former is plain and clear.

Again, in matter of practice, error attaches either to the general rule,
in the process of deliberation, or to the particular fact: for instance,
this would be a general rule, "All water of a certain gravity is bad;"
the particular fact, "this water is of that gravity."

And that Practical Wisdom is not knowledge is plain, for it has to do
with the ultimate issue, as has been said, because every object of
action is of this nature.

To Intuition it is opposed, for this takes in those principles which
cannot be proved by reasoning, while Practical Wisdom is concerned with
the ultimate particular fact which cannot be realised by Knowledge but
by Sense; I do not mean one of the five senses, but the same by which
we take in the mathematical fact, that no rectilineal figure can be
contained by less than three lines, i.e. that a triangle is the ultimate
figure, because here also is a stopping point.

This however is Sense rather than Practical Wisdom, which is of another
kind.

IX

Now the acts of inquiring and deliberating differ, though deliberating
is a kind of inquiring. We ought to ascertain about Good Counsel
likewise what it is, whether a kind of Knowledge, or Opinion, or Happy
Conjecture, or some other kind of faculty. Knowledge it obviously is
not, because men do not inquire about what they know, and Good Counsel
is a kind of deliberation, and the man who is deliberating is inquiring
and calculating. [Sidenote:1142b]

Neither is it Happy Conjecture; because this is independent of
reasoning, and a rapid operation; but men deliberate a long time, and
it is a common saying that one should execute speedily what has been
resolved upon in deliberation, but deliberate slowly.

Quick perception of causes again is a different faculty from good
counsel, for it is a species of Happy Conjecture. Nor is Good Counsel
Opinion of any kind.

Well then, since he who deliberates ill goes wrong, and he who
deliberates well does so rightly, it is clear that Good Counsel is
rightness of some kind, but not of Knowledge nor of Opinion: for
Knowledge cannot be called right because it cannot be wrong, and
Rightness of Opinion is Truth: and again, all which is the object of
opinion is definitely marked out.

Still, however, Good Counsel is not independent of Reason, Does it
remain then that it is a rightness of Intellectual Operation simply,
because this does not amount to an assertion; and the objection to
Opinion was that it is not a process of inquiry but already a definite
assertion; whereas whosoever deliberates, whether well or ill, is
engaged in inquiry and calculation.

Well, Good Counsel is a Rightness of deliberation, and so the first
question must regard the nature and objects of deliberation. Now
remember Rightness is an equivocal term; we plainly do not mean
Rightness of any kind whatever; the [Greek: akrataes], for instance, or
the bad man, will obtain by his calculation what he sets before him as
an object, and so he may be said to have deliberated _rightly_ in one
sense, but will have attained a great evil. Whereas to have deliberated
well is thought to be a good, because Good Counsel is Rightness of
deliberation of such a nature as is apt to attain good.

But even this again you may get by false reasoning, and hit upon the
right effect though not through right means, your middle term being
fallacious: and so neither will this be yet Good Counsel in consequence
of which you get what you ought but not through proper means.

Again, one man may hit on a thing after long deliberation, another
quickly. And so that before described will not be yet Good Counsel, but
the Rightness must be with reference to what is expedient; and you must
have a proper end in view, pursue it in a right manner and right time.

Once more. One may deliberate well either generally or towards some
particular End. Good counsel in the general then is that which goes
right towards that which is the End in a general way of consideration;
in particular, that which does so towards some particular End.

Since then deliberating well is a quality of men possessed of Practical
Wisdom, Good Counsel must be "Rightness in respect of what conduces to a
given End, of which Practical Wisdom is the true conception." [Sidenote:
X 1143_a_] There is too the faculty of Judiciousness, and also its
absence, in virtue of which we call men Judicious or the contrary.

Now Judiciousness is neither entirely identical with Knowledge or
Opinion (for then all would have been Judicious), nor is it any one
specific science, as medical science whose object matter is things
wholesome; or geometry whose object matter is magnitude: for it has not
for its object things which always exist and are immutable, nor of those
things which come into being just any which may chance; but those in
respect of which a man might doubt and deliberate.

And so it has the same object matter as Practical Wisdom; yet the two
faculties are not identical, because Practical Wisdom has the capacity
for commanding and taking the initiative, for its End is "what one
should do or not do:" but Judiciousness is only apt to decide upon
suggestions (though we do in Greek put "well" on to the faculty and its
concrete noun, these really mean exactly the same as the plain words),
and Judiciousness is neither the having Practical Wisdom, nor attaining
it: but just as learning is termed [Greek: sunievai] when a man uses
his knowledge, so judiciousness consists in employing the Opinionative
faculty in judging concerning those things which come within the
province of Practical Wisdom, when another enunciates them; and not
judging merely, but judging well (for [Greek: eu] and [Greek: kalos]
mean exactly the same thing). And the Greek name of this faculty is
derived from the use of the term [Greek: suvievai] in learning: [Greek:
mavthaveiv] and [Greek: suvievai] being often used as synonymous.

[Sidenote: XI] The faculty called [Greek: gvomh], in right of which we
call men [Greek: euyvomoves], or say they have [Greek: gvomh], is "the
right judgment of the equitable man." A proof of which is that we most
commonly say that the equitable man has a tendency to make allowance,
and the making allowance in certain cases is equitable. And [Greek:
sungvomae] (the word denoting allowance) is right [Greek: gvomh] having
a capacity of making equitable decisions, By "right" I mean that which
attains the True. Now all these mental states tend to the same object,
as indeed common language leads us to expect: I mean, we speak of
[Greek: gnomae], Judiciousness, Practical Wisdom, and Practical
Intuition, attributing the possession of [Greek: gnomae] and Practical
Intuition to the same Individuals whom we denominate Practically-Wise
and Judicious: because all these faculties are employed upon the
extremes, i.e. on particular details; and in right of his aptitude
for deciding on the matters which come within the province of the
Practically-Wise, a man is Judicious and possessed of good [Greek:
gnomae]; i.e. he is disposed to make allowance, for considerations of
equity are entertained by all good men alike in transactions with their
fellows.

And all matters of Moral Action belong to the class of particulars,
otherwise called extremes: for the man of Practical Wisdom must know
them, and Judiciousness and [Greek: gnomae] are concerned with matters
of Moral Actions, which are extremes.

[Sidenote:1143b] Intuition, moreover, takes in the extremes at both
ends: I mean, the first and last terms must be taken in not by reasoning
but by Intuition [so that Intuition comes to be of two kinds], and that
which belongs to strict demonstrative reasonings takes in immutable,
i.e. Necessary, first terms; while that which is employed in practical
matters takes in the extreme, the Contingent, and the minor Premiss: for
the minor Premisses are the source of the Final Cause, Universals being
made up out of Particulars. To take in these, of course, we must have
Sense, i.e. in other words Practical Intuition. And for this reason
these are thought to be simply gifts of nature; and whereas no man is
thought to be Scientific by nature, men are thought to have [Greek:
gnomae], and Judiciousness, and Practical Intuition: a proof of which is
that we think these faculties are a consequence even of particular ages,
and this given age has Practical Intuition and [Greek: gnomae], we say,
as if under the notion that nature is the cause. And thus Intuition is
both the beginning and end, because the proofs are based upon the one
kind of extremes and concern the other.

And so one should attend to the undemonstrable dicta and opinions of the
skilful, the old and the Practically-Wise, no less than to those which
are based on strict reasoning, because they see aright, having gained
their power of moral vision from experience.

XII

Well, we have now stated the nature and objects of Practical Wisdom and
Science respectively, and that they belong each to a different part
of the Soul. But I can conceive a person questioning their utility.
"Science," he would say, "concerns itself with none of the causes of
human happiness (for it has nothing to do with producing anything):
Practical Wisdom has this recommendation, I grant, but where is the need
of it, since its province is those things which are just and honourable,
and good for man, and these are the things which the good man as such
does; but we are not a bit the more apt to do them because we know them,
since the Moral Virtues are Habits; just as we are not more apt to be
healthy or in good condition from mere knowledge of what relates to
these (I mean, of course, things so called not from their producing
health, etc., but from their evidencing it in a particular subject),
for we are not more apt to be healthy and in good condition merely from
knowing the art of medicine or training.

"If it be urged that _knowing what is_ good does not by itself make a
Practically-Wise man but _becoming_ good; still this Wisdom will be no
use either to those that are good, and so have it already, or to those
who have it not; because it will make no difference to them whether they
have it themselves or put themselves under the guidance of others who
have; and we might be contented to be in respect of this as in respect
of health: for though we wish to be healthy still we do not set about
learning the art of healing.

"Furthermore, it would seem to be strange that, though lower in the
scale than Science, it is to be its master; which it is, because
whatever produces results takes the rule and directs in each matter."

This then is what we are to talk about, for these are the only points
now raised.

[Sidenote:1144a] Now first we say that being respectively Excellences
of different parts of the Soul they must be choiceworthy, even on the
supposition that they neither of them produce results.

In the next place we say that they _do_ produce results; that Science
makes Happiness, not as the medical art but as healthiness makes health:
because, being a part of Virtue in its most extensive sense, it makes a
man happy by being possessed and by working.

Next, Man's work _as Man_ is accomplished by virtue of Practical Wisdom
and Moral Virtue, the latter giving the right aim and direction, the
former the right means to its attainment; but of the fourth part of the
Soul, the mere nutritive principle, there is no such Excellence, because
nothing is in its power to do or leave undone.

As to our not being more apt to do what is noble and just by reason of
possessing Practical Wisdom, we must begin a little higher up, taking
this for our starting-point. As we say that men may do things in
themselves just and yet not be just men; for instance, when men do what
the laws require of them, either against their will, or by reason of
ignorance or something else, at all events not for the sake of the
things themselves; and yet they do what they ought and all that the good
man should do; so it seems that to be a good man one must do each act in
a particular frame of mind, I mean from Moral Choice and for the sake of
the things themselves which are done. Now it is Virtue which makes the
Moral Choice right, but whatever is naturally required to carry out
that Choice comes under the province not of Virtue but of a different
faculty. We must halt, as it were, awhile, and speak more clearly on
these points.

There is then a certain faculty, commonly named Cleverness, of such a
nature as to be able to do and attain whatever conduces to _any_ given
purpose: now if that purpose be a good one the faculty is praiseworthy;
if otherwise, it goes by a name which, denoting strictly the ability,
implies the willingness to do _anything_; we accordingly call the
Practically-Wise Clever, and also those who can and will do anything.

Now Practical Wisdom is not identical with Cleverness, nor is it without
this power of adapting means to ends: but this Eye of the Soul (as we
may call it) does not attain its proper state without goodness, as we
have said before and as is quite plain, because the syllogisms into
which Moral Action may be analysed have for their Major Premiss, "since
----------is the End and the Chief Good" (fill up the blank with just
anything you please, for we merely want to exhibit the Form, so that
anything will do), but _how_ this blank should be filled is seen only by
the good man: because Vice distorts the moral vision and causes men to
be deceived in respect of practical principles.

It is clear, therefore, that a man cannot be a Practically-Wise,
without being a good, man.

XIII

[Sidenote:1144b] We must inquire again also about Virtue: for it may be
divided into Natural Virtue and Matured, which two bear to each other a
relation similar to that which Practical Wisdom bears to Cleverness, one
not of identity but resemblance. I speak of Natural Virtue, because men
hold that each of the moral dispositions attach to us all somehow by
nature: we have dispositions towards justice, self-mastery and courage,
for instance, immediately from our birth: but still we seek Goodness
in its highest sense as something distinct from these, and that these
dispositions should attach to us in a somewhat different fashion.
Children and brutes have these natural states, but then they are plainly
hurtful unless combined with an intellectual element: at least thus much
is matter of actual experience and observation, that as a strong body
destitute of sight must, if set in motion, fall violently because it has
not sight, so it is also in the case we are considering: but if it can
get the intellectual element it then excels in acting. Just so the
Natural State of Virtue, being like this strong body, will then
be Virtue in the highest sense when it too is combined with the
intellectual element.

So that, as in the case of the Opinionative faculty, there are two
forms, Cleverness and Practical Wisdom; so also in the case of the Moral
there are two, Natural Virtue and Matured; and of these the latter
cannot be formed without Practical Wisdom.

This leads some to say that all the Virtues are merely intellectual
Practical Wisdom, and Socrates was partly right in his inquiry and
partly wrong: wrong in that he thought all the Virtues were merely
intellectual Practical Wisdom, right in saying they were not independent
of that faculty.

A proof of which is that now all, in defining Virtue, add on the "state"
[mentioning also to what standard it has reference, namely that] "which
is accordant with Right Reason:" now "right" means in accordance with
Practical Wisdom. So then all seem to have an instinctive notion that
that state which is in accordance with Practical Wisdom is Virtue;
however, we must make a slight change in their statement, because that
state is Virtue, not merely which is in accordance with but which
implies the possession of Right Reason; which, upon such matters, is
Practical Wisdom. The difference between us and Socrates is this: he
thought the Virtues were reasoning processes (_i.e._ that they were all
instances of Knowledge in its strict sense), but we say they imply the
possession of Reason.

From what has been said then it is clear that one cannot be, strictly
speaking, good without Practical Wisdom nor Practically-Wise without
moral goodness.

And by the distinction between Natural and Matured Virtue one can
meet the reasoning by which it might be argued "that the Virtues are
separable because the same man is not by nature most inclined to all at
once so that he will have acquired this one before he has that other:"
we would reply that this is possible with respect to the Natural Virtues
but not with respect to those in right of which a man is denominated
simply good: because they will all belong to him together with the one
faculty of Practical Wisdom. [Sidenote:1145a]

It is plain too that even had it not been apt to act we should have
needed it, because it is the Excellence of a part of the Soul; and that
the moral choice cannot be right independently of Practical Wisdom and
Moral Goodness; because this gives the right End, that causes the doing
these things which conduce to the End.

Then again, it is not Master of Science (i.e. of the superior part of
the Soul), just as neither is the healing art Master of health; for it
does not make use of it, but looks how it may come to be: so it commands
for the sake of it but does not command it.

The objection is, in fact, about as valid as if a man should say
[Greek: politikae] governs the gods because it gives orders about all
things in the communty.


APPENDIX

On [Greek: epistaemae], from I. Post. Analyt. chap. i. and ii.

(Such parts only are translated as throw light on the Ethics.)

All teaching, and all intellectual learning, proceeds on the basis
of previous knowledge, as will appear on an examination of all. The
Mathematical Sciences, and every other system, draw their conclusions in
this method. So too of reasonings, whether by syllogism, or induction:
for both teach through what is previously known, the former assuming
the premisses as from wise men, the latter proving universals from
the evidentness of the particulars. In like manner too rhetoricians
persuade, either through examples (which amounts to induction), or
through enthymemes (which amounts to syllogism).

Well, we suppose that we _know_ things (in the strict and proper sense
of the word) when we suppose ourselves to know the cause by reason
of which the thing is to be the cause of it; and that this cannot be
otherwise. It is plain that the idea intended to be conveyed by the term
_knowing_ is something of this kind; because they who do not really know
suppose themselves thus related to the matter in hand and they who
do know really are so that of whatsoever there is properly speaking
Knowledge this cannot be otherwise than it is Whether or no there is
another way of knowing we will say afterwards, but we do say that we
know through demonstration, by which I mean a syllogism apt to produce
Knowledge, i.e. in right of which through having it, we know.

If Knowledge then is such as we have described it, the Knowledge
produced by demonstrative reasoning must be drawn from premisses _true_
and _first_, and _incapable of syllogistic proof_, and _better known_,
and _prior in order of time_, and _causes of the conclusion_, for so the
principles will be akin to the conclusion demonstrated.

(Syllogism, of course there may be without such premisses, but it will
not be demonstration because it will not produce knowledge).

_True_, they must be, because it is impossible to know that which is not.

_First_, that is indemonstrable, because, if demonstrable, he cannot be
said to _know_ them who has no demonstration of them for knowing such
things as are demonstrable is the same as having demonstration of them.

_Causes_ they must be, and _better known_, and _prior_ in time,
_causes_, because we then know when we are acquainted with the cause,
and _prior_, if causes, and _known beforehand_, not merely comprehended
in idea but known to exist (The terms prior, and better known, bear two
senses for _prior by nature_ and _prior relatively to ourselves_ are not
the same, nor _better known by nature_, and _better known to us_ I mean,
by _prior_ and _better known relatively to ourselves_, such things as
are nearer to sensation, but abstractedly so such as are further
Those are furthest which are most universal those nearest which are
particulars, and these are mutually opposed) And by _first_, I mean
_principles akin to the conclusion_, for principle means the same as
first And the principle or first step in demonstration is a proposition
incapable of syllogistic proof, i. e. one to which there is none prior.
Now of such syllogistic principles I call that a [Greek: thxsis] which
you cannot demonstrate, and which is unnecessary with a view to learning
something else. That which is necessary in order to learn something else
is an Axiom.

Further, since one is to believe and know the thing by having a
syllogism of the kind called demonstration, and what constitutes it to
be such is the nature of the premisses, it is necessary not merely to
_know before_, but to _know better than the conclusion_, either all or
at least some of, the principles, because that which is the cause of a
quality inhering in something else always inheres itself more as the
cause of our loving is itself more lovable. So, since the principles are
the cause of our knowing and behoving we know and believe them more,
because by reason of them we know also the conclusion following.

Further: the man who is to have the Knowledge which comes through
demonstration must not merely know and believe his principles better
than he does his conclusion, but he must believe nothing more firmly
than the contradictories of those principles out of which the contrary
fallacy may be constructed: since he who _knows_, is to be simply and
absolutely infallible.




BOOK VII



I

Next we must take a different point to start from, and observe that of
what is to be avoided in respect of moral character there are three
forms; Vice, Imperfect Self-Control, and Brutishness. Of the two former
it is plain what the contraries are, for we call the one Virtue, the
other Self-Control; and as answering to Brutishness it will be most
suitable to assign Superhuman, i.e. heroical and godlike Virtue, as, in
Homer, Priam says of Hector "that he was very excellent, nor was he like
the offspring of mortal man, but of a god." and so, if, as is commonly
said, men are raised to the position of gods by reason of very high
excellence in Virtue, the state opposed to the Brutish will plainly be
of this nature: because as brutes are not virtuous or vicious so neither
are gods; but the state of these is something more precious than Virtue,
of the former something different in kind from Vice.

And as, on the one hand, it is a rare thing for a man to be godlike (a
term the Lacedaemonians are accustomed to use when they admire a man
exceedingly; [Greek:seios anhæp] they call him), so the brutish man is
rare; the character is found most among barbarians, and some cases of it
are caused by disease or maiming; also such men as exceed in vice all
ordinary measures we therefore designate by this opprobrious term. Well,
we must in a subsequent place make some mention of this disposition,
and Vice has been spoken of before: for the present we must speak of
Imperfect Self-Control and its kindred faults of Softness and Luxury, on
the one hand, and of Self-Control and Endurance on the other; since we
are to conceive of them, not as being the same states exactly as Virtue
and Vice respectively, nor again as differing in kind. [Sidenote:1145b]
And we should adopt the same course as before, i.e. state the phenomena,
and, after raising and discussing difficulties which suggest themselves,
then exhibit, if possible, all the opinions afloat respecting these
affections of the moral character; or, if not all, the greater part and
the most important: for we may consider we have illustrated the matter
sufficiently when the difficulties have been solved, and such theories
as are most approved are left as a residuum.

The chief points may be thus enumerated. It is thought,

I. That Self-Control and Endurance belong to the class of things good
and praiseworthy, while Imperfect Self-Control and Softness belong to
that of things low and blameworthy.

II. That the man of Self-Control is identical with the man who is apt to
abide by his resolution, and the man of Imperfect Self-Control with him
who is apt to depart from his resolution.

III. That the man of Imperfect Self-Control does things at the
instigation of his passions, knowing them to be wrong, while the man of
Self-Control, knowing his lusts to be wrong, refuses, by the influence
of reason, to follow their suggestions.

IV. That the man of Perfected Self-Mastery unites the qualities of
Self-Control and Endurance, and some say that every one who unites these
is a man of Perfect Self-Mastery, others do not.

V. Some confound the two characters of the man who has _no_
Self-Control, and the man of _Imperfect Self-Control_, while others
distinguish between them.

VI. It is sometimes said that the man of Practical Wisdom cannot be a
man of Imperfect Self-Control, sometimes that men who are Practically
Wise and Clever are of Imperfect Self-Control.

VII. Again, men are said to be of Imperfect Self-Control, not simply
but with the addition of the thing wherein, as in respect of anger, of
honour, and gain.

These then are pretty well the common statements.

II

Now a man may raise a question as to the nature of the right conception
in violation of which a man fails of Self-Control.

That he can so fail when _knowing_ in the strict sense what is right
some say is impossible: for it is a strange thing, as Socrates thought,
that while Knowledge is present in his mind something else should
master him and drag him about like a slave. Socrates in fact contended
generally against the theory, maintaining there is no such state as that
of Imperfect Self-Control, for that no one acts contrary to what is best
conceiving it to be best but by reason of ignorance what is best.

With all due respect to Socrates, his account of the matter is at
variance with plain facts, and we must inquire with respect to the
affection, if it be caused by ignorance what is the nature of the
ignorance: for that the man so failing does not suppose his acts to be
right before he is under the influence of passion is quite plain.

There are people who partly agree with Socrates and partly not: that
nothing can be stronger than Knowledge they agree, but that no man acts
in contravention of his conviction of what is better they do not agree;
and so they say that it is not Knowledge, but only Opinion, which the
man in question has and yet yields to the instigation of his pleasures.

[Sidenote:1146a] But then, if it is Opinion and not Knowledge, that is
it the opposing conception be not strong but only mild (as in the case
of real doubt), the not abiding by it in the face of strong lusts would
be excusable: but wickedness is not excusable, nor is anything which
deserves blame.

Well then, is it Practical Wisdom which in this case offers opposition:
for that is the strongest principle? The supposition is absurd, for
we shall have the same man uniting Practical Wisdom and Imperfect
Self-Control, and surely no single person would maintain that it is
consistent with the character of Practical Wisdom to do voluntarily what
is very wrong; and besides we have shown before that the very mark of
a man of this character is aptitude to act, as distinguished from
mere knowledge of what is right; because he is a man conversant with
particular details, and possessed of all the other virtues.

Again, if the having strong and bad lusts is necessary to the idea of
the man of Self-Control, this character cannot be identical with the man
of Perfected Self-Mastery, because the having strong desires or bad ones
does not enter into the idea of this latter character: and yet the man
of Self-Control must have such: for suppose them good; then the moral
state which should hinder a man from following their suggestions must be
bad, and so Self-Control would not be in all cases good: suppose them on
the other hand to be weak and not wrong, it would be nothing grand; nor
anything great, supposing them to be wrong and weak.

Again, if Self-Control makes a man apt to abide by all opinions without
exception, it may be bad, as suppose the case of a false opinion: and
if Imperfect Self-Control makes a man apt to depart from all without
exception, we shall have cases where it will be good; take that of
Neoptolemus in the Philoctetes of Sophocles, for instance: he is to be
praised for not abiding by what he was persuaded to by Ulysses, because
he was pained at being guilty of falsehood.

Or again, false sophistical reasoning presents a difficulty: for because
men wish to prove paradoxes that they may be counted clever when they
succeed, the reasoning that has been used becomes a difficulty: for the
intellect is fettered; a man being unwilling to abide by the conclusion
because it does not please his judgment, but unable to advance because
he cannot disentangle the web of sophistical reasoning.

Or again, it is conceivable on this supposition that folly joined with
Imperfect Self-Control may turn out, in a given case, goodness: for by
reason of his imperfection of self-control a man acts in a way which
contradicts his notions; now his notion is that what is really good is
bad and ought not to be done; and so he will eventually do what is good
and not what is bad.

Again, on the same supposition, the man who acting on conviction pursues
and chooses things because they are pleasant must be thought a better
man than he who does so not by reason of a quasi-rational conviction but
of Imperfect Self-Control: because he is more open to cure by reason of
the possibility of his receiving a contrary conviction. But to the man
of Imperfect Self-Control would apply the proverb, "when water chokes,
what should a man drink then?" for had he never been convinced at all
in respect of [Sidenote: 1146b] what he does, then by a conviction in a
contrary direction he might have stopped in his course; but now though
he has had convictions he notwithstanding acts against them.

Again, if any and every thing is the object-matter of Imperfect and
Perfect Self-Control, who is the man of Imperfect Self-Control simply?
because no one unites all cases of it, and we commonly say that some men
are so simply, not adding any particular thing in which they are so.

Well, the difficulties raised are pretty near such as I have described
them, and of these theories we must remove some and leave others as
established; because the solving of a difficulty is a positive act of
establishing something as true.

III

Now we must examine first whether men of Imperfect Self-Control act with
a knowledge of what is right or not: next, if with such knowledge, in
what sense; and next what are we to assume is the object-matter of the
man of Imperfect Self-Control, and of the man of Self-Control; I mean,
whether pleasure and pain of all kinds or certain definite ones; and as
to Self-Control and Endurance, whether these are designations of the
same character or different. And in like manner we must go into all
questions which are connected with the present.

But the real starting point of the inquiry is, whether the two
characters of Self-Control and Imperfect Self-Control are distinguished
by their object-matter, or their respective relations to it. I mean,
whether the man of Imperfect Self-Control is such simply by virtue of
having such and such object-matter; or not, but by virtue of his being
related to it in such and such a way, or by virtue of both: next,
whether Self-Control and Imperfect Self-Control are unlimited in their
object-matter: because he who is designated without any addition a man
of Imperfect Self-Control is not unlimited in his object-matter, but has
exactly the same as the man who has lost all Self-Control: nor is he so
designated because of his relation to this object-matter merely (for
then his character would be identical with that just mentioned, loss
of all Self-Control), but because of his relation to it being such
and such. For the man who has lost all Self-Control is led on with
deliberate moral choice, holding that it is his line to pursue pleasure
as it rises: while the man of Imperfect Self-Control does not think that
he ought to pursue it, but does pursue it all the same.

Now as to the notion that it is True Opinion and not Knowledge in
contravention of which men fail in Self-Control, it makes no difference
to the point in question, because some of those who hold Opinions have
no doubt about them but suppose themselves to have accurate Knowledge;
if then it is urged that men holding Opinions will be more likely than
men who have Knowledge to act in contravention of their conceptions,
as having but a moderate belief in them; we reply, Knowledge will not
differ in this respect from Opinion: because some men believe their
own Opinions no less firmly than others do their positive Knowledge:
Heraclitus is a case in point.

Rather the following is the account of it: the term _knowing_ has two
senses; both the man who does not use his Knowledge, and he who does,
are said to _know_: there will be a difference between a man's acting
wrongly, who though possessed of Knowledge does not call it into
operation, and his doing so who has it and actually exercises it: the
latter is a strange case, but the mere having, if not exercising,
presents no anomaly.

[Sidenote:1147a] Again, as there are two kinds of propositions affecting
action, universal and particular, there is no reason why a man may not
act against his Knowledge, having both propositions in his mind, using
the universal but not the particular, for the particulars are the
objects of moral action.

There is a difference also in universal propositions; a universal
proposition may relate partly to a man's self and partly to the thing in
question: take the following for instance; "dry food is good for every
man," this may have the two minor premisses, "this is a man," and "so
and so is dry food;" but whether a given substance is so and so a man
either has not the Knowledge or does not exert it. According to these
different senses there will be an immense difference, so that for a
man to _know_ in the one sense, and yet act wrongly, would be nothing
strange, but in any of the other senses it would be a matter for wonder.

Again, men may have Knowledge in a way different from any of those which
have been now stated: for we constantly see a man's state so differing
by having and not using Knowledge, that he has it in a sense and also
has not; when a man is asleep, for instance, or mad, or drunk: well, men
under the actual operation of passion are in exactly similar conditions;
for anger, lust, and some other such-like things, manifestly make
changes even in the body, and in some they even cause madness; it is
plain then that we must say the men of Imperfect Self-Control are in a
state similar to these.

And their saying what embodies Knowledge is no proof of their actually
then exercising it, because they who are under the operation of these
passions repeat demonstrations; or verses of Empedocles, just as
children, when first learning, string words together, but as yet know
nothing of their meaning, because they must grow into it, and this is a
process requiring time: so that we must suppose these men who fail in
Self-Control to say these moral sayings just as actors do. Furthermore,
a man may look at the account of the phænomenon in the following way,
from an examination of the actual working of the mind: All action may
be analysed into a syllogism, in which the one premiss is an universal
maxim and the other concerns particulars of which Sense [moral or
physical, as the case may be] is cognisant: now when one results from
these two, it follows necessarily that, as far as theory goes the mind
must assert the conclusion, and in practical propositions the man must
act accordingly. For instance, let the universal be, "All that is
sweet should be tasted," the particular, "This is sweet;" it follows
necessarily that he who is able and is not hindered should not only
draw, but put in practice, the conclusion "This is to be tasted." When
then there is in the mind one universal proposition forbidding to taste,
and the other "All that is sweet is pleasant" with its minor "This is
sweet" (which is the one that really works), and desire happens to be in
the man, the first universal bids him avoid this but the desire leads
him on to taste; for it has the power of moving the various organs:
and so it results that he fails in Self-Control, [Sidenote:1147b] in a
certain sense under the influence of Reason and Opinion not contrary in
itself to Reason but only accidentally so; because it is the desire that
is contrary to Right Reason, but not the Opinion: and so for this reason
brutes are not accounted of Imperfect Self-Control, because they have
no power of conceiving universals but only of receiving and retaining
particular impressions.

As to the manner in which the ignorance is removed and the man of
Imperfect Self-Control recovers his Knowledge, the account is the same
as with respect to him who is drunk or asleep, and is not peculiar to
this affection, so physiologists are the right people to apply to. But
whereas the minor premiss of every practical syllogism is an opinion on
matter cognisable by Sense and determines the actions; he who is under
the influence of passion either has not this, or so has it that his
having does not amount to _knowing_ but merely saying, as a man when
drunk might repeat Empedocles' verses; and because the minor term
is neither universal, nor is thought to have the power of producing
Knowledge in like manner as the universal term: and so the result which
Socrates was seeking comes out, that is to say, the affection does not
take place in the presence of that which is thought to be specially
and properly Knowledge, nor is this dragged about by reason of the
affection, but in the presence of that Knowledge which is conveyed by
Sense.

Let this account then be accepted of the question respecting the failure
in Self-Control, whether it is with Knowledge or not; and, if with
knowledge, with what kind of knowledge such failure is possible.

IV

The next question to be discussed is whether there is a character to be
designated by the term "of Imperfect Self-Control" simply, or whether
all who are so are to be accounted such, in respect of some particular
thing; and, if there is such a character, what is his object-matter.

Now that pleasures and pains are the object-matter of men of
Self-Control and of Endurance, and also of men of Imperfect Self-Control
and Softness, is plain.

Further, things which produce pleasure are either necessary, or objects
of choice in themselves but yet admitting of excess. All bodily things
which produce pleasure are necessary; and I call such those which relate
to food and other grosser appetities, in short such bodily things as
we assumed were the Object-matter of absence of Self-Control and of
Perfected Self-Mastery.

The other class of objects are not necessary, but objects of choice in
themselves: I mean, for instance, victory, honour, wealth, and such-like
good or pleasant things. And those who are excessive in their liking for
such things contrary to the principle of Right Reason which is in their
own breasts we do not designate men of Imperfect Self-Control simply,
but with the addition of the thing wherein, as in respect of money, or
gain, or honour, or anger, and not simply; because we consider them as
different characters and only having that title in right of a kind of
resemblance (as when we add to a man's name "conqueror in the Olympic
games" the account of him as Man differs but little from the account
of him as the Man who conquered in the Olympic games, but still it is
different). And a proof of the real [Sidenote: 1148a] difference between
these so designated with an addition and those simply so called is this,
that Imperfect Self-Control is blamed, not as an error merely but also
as being a vice, either wholly or partially; but none of these other
cases is so blamed.

But of those who have for their object-matter the bodily enjoyments,
which we say are also the object-matter of the man of Perfected
Self-Mastery and the man who has lost all Self-Control, he that pursues
excessive pleasures and too much avoids things which are painful (as
hunger and thirst, heat and cold, and everything connected with touch
and taste), not from moral choice but in spite of his moral choice and
intellectual conviction, is termed "a man of Imperfect Self-Control,"
not with the addition of any particular object-matter as we do in
respect of want of control of anger but simply.

And a proof that the term is thus applied is that the kindred term
"Soft" is used in respect of these enjoyments but not in respect of any
of those others. And for this reason we put into the same rank the man
of Imperfect Self-Control, the man who has lost it entirely, the man
who has it, and the man of Perfected Self-Mastery; but not any of those
other characters, because the former have for their object-matter the
same pleasures and pains: but though they have the same object-matter,
they are not related to it in the same way, but two of them act upon
moral choice, two without it. And so we should say that man is more
entirely given up to his passions who pursues excessive pleasures, and
avoids moderate pains, being either not at all, or at least but little,
urged by desire, than the man who does so because his desire is very
strong: because we think what would the former be likely to do if he had
the additional stimulus of youthful lust and violent pain consequent on
the want of those pleasures which we have denominated necessary?

Well then, since of desires and pleasures there are some which are in
kind honourable and good (because things pleasant are divisible, as we
said before, into such as are naturally objects of choice, such as
are naturally objects of avoidance, and such as are in themselves
indifferent, money, gain, honour, victory, for instance); in respect of
all such and those that are indifferent, men are blamed not merely for
being affected by or desiring or liking them, but for exceeding in any
way in these feelings.

And so they are blamed, whosoever in spite of Reason are mastered by,
that is pursue, any object, though in its nature noble and good; they,
for instance, who are more earnest than they should be respecting
honour, or their children or parents; not but what these are good
objects and men are praised for being earnest about them: but still they
admit of excess; for instance, if any one, as Niobe did, should fight
even against the gods, or feel towards his father as Satyrus, who got
therefrom the nickname of [Greek: philophator], [Sidenote: 1148b]
because he was thought to be very foolish.

Now depravity there is none in regard of these things, for the reason
assigned above, that each of them in itself is a thing naturally
choiceworthy, yet the excesses in respect of them are wrong and matter
for blame: and similarly there is no Imperfect Self-Control in respect
of these things; that being not merely a thing that should be avoided
but blameworthy.

But because of the resemblance of the affection to the Imperfection of
Self-Control the term is used with the addition in each case of the
particular object-matter, just as men call a man a bad physician, or bad
actor, whom they would not think of calling simply bad. As then in these
cases we do not apply the term simply because each of the states is not
a vice, but only like a vice in the way of analogy, so it is plain that
in respect of Imperfect Self-Control and Self-Control we must limit the
names to those states which have the same object-matter as Perfected
Self-Mastery and utter loss of Self-Control, and that we do apply it to
the case of anger only in the way of resemblance: for which reason, with
an addition, we designate a man of Imperfect Self-Control in respect of
anger, as of honour or of gain.

V

As there are some things naturally pleasant, and of these two kinds;
those, namely, which are pleasant generally, and those which are so
relatively to particular kinds of animals and men; so there are others
which are not naturally pleasant but which come to be so in consequence
either of maimings, or custom, or depraved natural tastes: and one may
observe moral states similar to those we have been speaking of, having
respectively these classes of things for their object-matter.

I mean the Brutish, as in the case of the female who, they say, would
rip up women with child and eat the foetus; or the tastes which are
found among the savage tribes bordering on the Pontus, some liking raw
flesh, and some being cannibals, and some lending one another their
children to make feasts of; or what is said of Phalaris. These are
instances of Brutish states, caused in some by disease or madness; take,
for instance, the man who sacrificed and ate his mother, or him who
devoured the liver of his fellow-servant. Instances again of those
caused by disease or by custom, would be, plucking out of hair, or
eating one's nails, or eating coals and earth. ... Now wherever nature
is really the cause no one would think of calling men of Imperfect
Self-Control, ... nor, in like manner, such as are in a diseased state
through custom.

[Sidenote:1149a] Obviously the having any of these inclinations is
something foreign to what is denominated Vice, just as Brutishness is:
and when a man has them his mastering them is not properly Self-Control,
nor his being mastered by them Imperfection of Self-Control in the
proper sense, but only in the way of resemblance; just as we may say a
man of ungovernable wrath fails of Self-Control in respect of anger but
not simply fails of Self-Control. For all excessive folly, cowardice,
absence of Self-Control, or irritability, are either Brutish or morbid.
The man, for instance, who is naturally afraid of all things, even if
a mouse should stir, is cowardly after a Brutish sort; there was a man
again who, by reason of disease, was afraid of a cat: and of the fools,
they who are naturally destitute of Reason and live only by Sense are
Brutish, as are some tribes of the far-off barbarians, while others
who are so by reason of diseases, epileptic or frantic, are in morbid
states.

So then, of these inclinations, a man may sometimes merely have one
without yielding to it: I mean, suppose that Phalaris had restrained his
unnatural desire to eat a child: or he may both have and yield to it. As
then Vice when such as belongs to human nature is called Vice simply,
while the other is so called with the addition of "brutish" or "morbid,"
but not simply Vice, so manifestly there is Brutish and Morbid
Imperfection of Self-Control, but that alone is entitled to the name
without any qualification which is of the nature of utter absence of
Self-Control, as it is found in Man.

VI

It is plain then that the object-matter of Imperfect Self-Control and
Self-Control is restricted to the same as that of utter absence of
Self-Control and that of Perfected Self-Mastery, and that the rest is
the object-matter of a different species so named metaphorically and not
simply: we will now examine the position, "that Imperfect Self-Control
in respect of Anger is less disgraceful than that in respect of Lusts."

In the first place, it seems that Anger does in a way listen to Reason
but mishears it; as quick servants who run out before they have heard
the whole of what is said and then mistake the order; dogs, again, bark
at the slightest stir, before they have seen whether it be friend
or foe; just so Anger, by reason of its natural heat and quickness,
listening to Reason, but without having heard the command of Reason,
rushes to its revenge. That is to say, Reason or some impression on the
mind shows there is insolence or contempt in the offender, and then
Anger, reasoning as it were that one ought to fight against what is
such, fires up immediately: whereas Lust, if Reason or Sense, as the
case may be, merely says a thing is sweet, rushes to the enjoyment of
it: and so Anger follows Reason in a manner, but Lust does not and is
therefore more disgraceful: because he that cannot control his anger
yields in a manner to Reason, but the other to his Lust and not to
Reason at all. [Sidenote:1149b]

Again, a man is more excusable for following such desires as are
natural, just as he is for following such Lusts as are common to all and
to that degree in which they are common. Now Anger and irritability are
more natural than Lusts when in excess and for objects not necessary.
(This was the ground of the defence the man made who beat his father,
"My father," he said, "used to beat his, and his father his again, and
this little fellow here," pointing to his child, "will beat me when he
is grown a man: it runs in the family." And the father, as he was being
dragged along, bid his son leave off beating him at the door, because he
had himself been used to drag his father so far and no farther.)

Again, characters are less unjust in proportion as they involve less
insidiousness. Now the Angry man is not insidious, nor is Anger, but
quite open: but Lust is: as they say of Venus,

  "Cyprus-born Goddess, _weaver of deceits_"

Or Homer of the girdle called the Cestus,

  "Persuasiveness _cheating_ e'en the subtlest mind."

And so since this kind of Imperfect Self-Control is more unjust, it
is also more disgraceful than that in respect of Anger, and is simply
Imperfect Self-Control, and Vice in a certain sense. Again, no man feels
pain in being insolent, but every one who acts through Anger does act
with pain; and he who acts insolently does it with pleasure. If then
those things are most unjust with which we have most right to be angry,
then Imperfect Self-Control, arising from Lust, is more so than that
arising from Anger: because in Anger there is no insolence.

Well then, it is clear that Imperfect Self-Control in respect of
Lusts is more disgraceful than that in respect of Anger, and that the
object-matter of Self-Control, and the Imperfection of it, are bodily
Lusts and pleasures; but of these last we must take into account the
differences; for, as was said at the commencement, some are proper to
the human race and natural both in kind and degree, others Brutish, and
others caused by maimings and diseases.

Now the first of these only are the object-matter of Perfected
Self-Mastery and utter absence of Self-Control; and therefore we never
attribute either of these states to Brutes (except metaphorically,
and whenever any one kind of animal differs entirely from another in
insolence, mischievousness, or voracity), because they have not moral
choice or process of deliberation, but are quite different from that
kind of creature just as are madmen from other men.

[Sidenote: 1150a] Brutishness is not so low in the scale as Vice, yet
it is to be regarded with more fear: because it is not that the highest
principle has been corrupted, as in the human creature, but the subject
has it not at all.

It is much the same, therefore, as if one should compare an inanimate
with an animate being, which were the worse: for the badness of that
which has no principle of origination is always less harmful; now
Intellect is a principle of origination. A similar case would be the
comparing injustice and an unjust man together: for in different ways
each is the worst: a bad man would produce ten thousand times as much
harm as a bad brute.

VII

Now with respect to the pleasures and pains which come to a man through
Touch and Taste, and the desiring or avoiding such (which we determined
before to constitute the object-matter of the states of utter absence of
Self-Control and Perfected Self-Mastery), one may be so disposed as
to yield to temptations to which most men would be superior, or to
be superior to those to which most men would yield: in respect of
pleasures, these characters will be respectively the man of Imperfect
Self-Control, and the man of Self-Control; and, in respect of pains, the
man of Softness and the man of Endurance: but the moral state of most
men is something between the two, even though they lean somewhat to the
worse characters.

Again, since of the pleasures indicated some are necessary and some are
not, others are so to a certain degree but not the excess or defect of
them, and similarly also of Lusts and pains, the man who pursues the
excess of pleasant things, or such as are in themselves excess, or from
moral choice, for their own sake, and not for anything else which is to
result from them, is a man utterly void of Self-Control: for he must be
incapable of remorse, and so incurable, because he that has not remorse
is incurable. (He that has too little love of pleasure is the opposite
character, and the man of Perfected Self-Mastery the mean character.) He
is of a similar character who avoids the bodily pains, not because he
_cannot_, but because he _chooses not to_, withstand them.

But of the characters who go wrong without _choosing_ so to do, the one
is led on by reason of pleasure, the other because he avoids the pain it
would cost him to deny his lust; and so they are different the one from
the other. Now every one would pronounce a man worse for doing something
base without any impulse of desire, or with a very slight one, than for
doing the same from the impulse of a very strong desire; for striking
a man when not angry than if he did so in wrath: because one naturally
says, "What would he have done had he been under the influence of
passion?" (and on this ground, by the bye, the man utterly void of
Self-Control is worse than he who has it imperfectly). However, of the
two characters which have been mentioned [as included in that of utter
absence of Self-Control], the one is rather Softness, the other properly
the man of no Self-Control.

Furthermore, to the character of Imperfect Self-Control is opposed that
of Self-Control, and to that of Softness that of Endurance: because
Endurance consists in continued resistance but Self-Control in actual
mastery, and continued resistance and actual mastery are as different
as not being conquered is from conquering; and so Self-Control is more
choiceworthy than Endurance.

[Sidenote:1150b] Again, he who fails when exposed to those temptations
against which the common run of men hold out, and are well able to do
so, is Soft and Luxurious (Luxury being a kind of Softness): the kind of
man, I mean, to let his robe drag in the dirt to avoid the trouble
of lifting it, and who, aping the sick man, does not however suppose
himself wretched though he is like a wretched man. So it is too with
respect to Self-Control and the Imperfection of it: if a man yields to
pleasures or pains which are violent and excessive it is no matter for
wonder, but rather for allowance if he made what resistance he could
(instances are, Philoctetes in Theodectes' drama when wounded by the
viper; or Cercyon in the Alope of Carcinus, or men who in trying to
suppress laughter burst into a loud continuous fit of it, as happened,
you remember, to Xenophantus), but it is a matter for wonder when a man
yields to and cannot contend against those pleasures or pains which the
common herd are able to resist; always supposing his failure not to be
owing to natural constitution or disease, I mean, as the Scythian kings
are constitutionally Soft, or the natural difference between the sexes.

Again, the man who is a slave to amusement is commonly thought to be
destitute of Self-Control, but he really is Soft; because amusement
is an act of relaxing, being an act of resting, and the character in
question is one of those who exceed due bounds in respect of this.

Moreover of Imperfect Self-Control there are two forms, Precipitancy and
Weakness: those who have it in the latter form though they have made
resolutions do not abide by them by reason of passion; the others are
led by passion because they have never formed any resolutions at
all: while there are some who, like those who by tickling themselves
beforehand get rid of ticklishness, having felt and seen beforehand the
approach of temptation, and roused up themselves and their resolution,
yield not to passion; whether the temptation be somewhat pleasant or
somewhat painful. The Precipitate form of Imperfect Self-Control they
are most liable to who are constitutionally of a sharp or melancholy
temperament: because the one by reason of the swiftness, the other by
reason of the violence, of their passions, do not wait for Reason,
because they are disposed to follow whatever notion is impressed upon
their minds.

VIII

Again, the man utterly destitute of Self-Control, as was observed
before, is not given to remorse: for it is part of his character that
he abides by his moral choice: but the man of Imperfect Self-Control is
almost made up of remorse: and so the case is not as we determined it
before, but the former is incurable and the latter may be cured: for
depravity is like chronic diseases, dropsy and consumption for instance,
but Imperfect Self-Control is like acute disorders: the former being a
continuous evil, the latter not so. And, in fact, Imperfect Self-Control
and Confirmed Vice are different in kind: the latter being imperceptible
to its victim, the former not so.

[Sidenote: 1151a] But, of the different forms of Imperfect Self-Control,
those are better who are carried off their feet by a sudden access of
temptation than they who have Reason but do not abide by it; these
last being overcome by passion less in degree, and not wholly without
premeditation as are the others: for the man of Imperfect Self-Control
is like those who are soon intoxicated and by little wine and less than
the common run of men. Well then, that Imperfection of Self-Control is
not Confirmed Viciousness is plain: and yet perhaps it is such in a way,
because in one sense it is contrary to moral choice and in another the
result of it: at all events, in respect of the actions, the case is much
like what Demodocus said of the Miletians. "The people of Miletus are
not fools, but they do just the kind of things that fools do;" and so
they of Imperfect Self-Control are not unjust, but they do unjust acts.

But to resume. Since the man of Imperfect Self-Control is of such a
character as to follow bodily pleasures in excess and in defiance of
Right Reason, without acting on any deliberate conviction, whereas the
man utterly destitute of Self-Control does act upon a conviction which
rests on his natural inclination to follow after these pleasures; the
former may be easily persuaded to a different course, but the latter
not: for Virtue and Vice respectively preserve and corrupt the moral
principle; now the motive is the principle or starting point in moral
actions, just as axioms and postulates are in mathematics: and neither
in morals nor mathematics is it Reason which is apt to teach the
principle; but Excellence, either natural or acquired by custom, in
holding right notions with respect to the principle. He who does this in
morals is the man of Perfected Self-Mastery, and the contrary character
is the man utterly destitute of Self-Control.

Again, there is a character liable to be taken off his feet in defiance
of Right Reason because of passion; whom passion so far masters as to
prevent his acting in accordance with Right Reason, but not so far as to
make him be convinced that it is his proper line to follow after such
pleasures without limit: this character is the man of Imperfect Self-
Control, better than he who is utterly destitute of it, and not a bad
man simply and without qualification: because in him the highest and
best part, i.e. principle, is preserved: and there is another character
opposed to him who is apt to abide by his resolutions, and not to depart
from them; at all events, not at the instigation of passion. It is
evident then from all this, that Self-Control is a good state and the
Imperfection of it a bad one.

Next comes the question, whether a man is a man of Self-Control for
abiding by his conclusions and moral choice be they of what kind they
may, or only by the right one; or again, a man of Imperfect Self-Control
for not abiding by his conclusions and moral choice be they of whatever
kind; or, to put the case we did before, is he such for not abiding by
false conclusions and wrong moral choice?

Is not this the truth, that _incidentally_ it is by conclusions and
moral choice of any kind that the one character abides and the other
does not, but _per se_ true conclusions and right moral choice: to
explain what is meant by incidentally, and _per se_; suppose a man
chooses or pursues this thing for the sake of that, he is said to pursue
and choose that _per se_, but this only incidentally. For the term _per
se_ we use commonly the word "simply," and so, in a way, it is opinion
of any kind soever by which the two characters respectively abide or
not, but he is "simply" entitled to the designations who abides or not
by the true opinion.

There are also people, who have a trick of abiding by their, own
opinions, who are commonly called Positive, as they who are hard to
be persuaded, and whose convictions are not easily changed: now these
people bear some resemblance to the character of Self-Control, just as
the prodigal to the liberal or the rash man to the brave, but they are
different in many points. The man of Self-Control does not change by
reason of passion and lust, yet when occasion so requires he will be
easy of persuasion: but the Positive man changes not at the call of
Reason, though many of this class take up certain desires and are led by
their pleasures. Among the class of Positive are the Opinionated, the
Ignorant, and the Bearish: the first, from the motives of pleasure and
pain: I mean, they have the pleasurable feeling of a kind of victory in
not having their convictions changed, and they are pained when their
decrees, so to speak, are reversed: so that, in fact, they rather
resemble the man of Imperfect Self-Control than the man of Self-Control.

Again, there are some who depart from their resolutions not by reason of
any Imperfection of Self-Control; take, for instance, Neoptolemus in the
Philoctetes of Sophocles. Here certainly pleasure was the motive of his
departure from his resolution, but then it was one of a noble sort:
for to be truthful was noble in his eyes and he had been persuaded by
Ulysses to lie.

So it is not every one who acts from the motive of pleasure who is
utterly destitute of Self-Control or base or of Imperfect Self-Control,
only he who acts from the impulse of a base pleasure.

Moreover as there is a character who takes less pleasure than he ought
in bodily enjoyments, and he also fails to abide by the conclusion of
his Reason, the man of Self-Control is the mean between him and the man
of Imperfect Self-Control: that is to say, the latter fails to abide by
them because of somewhat too much, the former because of somewhat too
little; while the man of Self-Control abides by them, and never changes
by reason of anything else than such conclusions.

Now of course since Self-Control is good both the contrary States must
be bad, as indeed they plainly are: but because the one of them is seen
in few persons, and but rarely in them, Self-Control comes to be
viewed as if opposed only to the Imperfection of it, just as
Perfected Self-Mastery is thought to be opposed only to utter want of
Self-Control.

[Sidenote: 1152a] Again, as many terms are used in the way of
similitude, so people have come to talk of the Self-Control of the man
of Perfected Self-Mastery in the way of similitude: for the man of
Self-Control and the man of Perfected Self-Mastery have this in common,
that they do nothing against Right Reason on the impulse of bodily
pleasures, but then the former has bad desires, the latter not; and the
latter is so constituted as not even to feel pleasure contrary to his
Reason, the former feels but does not yield to it. Like again are the
man of Imperfect Self-Control and he who is utterly destitute of it,
though in reality distinct: both follow bodily pleasures, but the latter
under a notion that it is the proper line for him to take, his former
without any such notion.



X

And it is not possible for the same man to be at once a man of Practical
Wisdom and of Imperfect Self-Control: because the character of Practical
Wisdom includes, as we showed before, goodness of moral character.
And again, it is not knowledge merely, but aptitude for action, which
constitutes Practical Wisdom: and of this aptitude the man of Imperfect
Self-Control is destitute. But there is no reason why the Clever man
should not be of Imperfect Self-Control: and the reason why some men are
occasionally thought to be men of Practical Wisdom, and yet of Imperfect
Self-Control, is this, that Cleverness differs from Practical Wisdom in
the way I stated in a former book, and is very near it so far as the
intellectual element is concerned but differs in respect of the moral
choice.

Nor is the man of Imperfect Self-Control like the man who both has and
calls into exercise his knowledge, but like the man who, having it, is
overpowered by sleep or wine. Again, he acts voluntarily (because he
knows, in a certain sense, what he does and the result of it), but he is
not a confirmed bad man, for his moral choice is good, so he is at all
events only half bad. Nor is he unjust, because he does not act with
deliberate intent: for of the two chief forms of the character, the one
is not apt to abide by his deliberate resolutions, and the other, the
man of constitutional strength of passion, is not apt to deliberate at
all.

So in fact the man of Imperfect Self-Control is like a community which
makes all proper enactments, and has admirable laws, only does not act
on them, verifying the scoff of Anaxandrides,

  "That State did will it, which cares nought for laws;"
whereas the bad man is like one which acts upon its laws, but then
unfortunately they are bad ones. Imperfection of Self-Control and
Self-Control, after all, are above the average state of men; because he
of the latter character is more true to his Reason, and the former less
so, than is in the power of most men.

Again, of the two forms of Imperfect Self-Control that is more easily
cured which they have who are constitutionally of strong passions, than
that of those who form resolutions and break them; and they that are so
through habituation than they that are so naturally; since of course
custom is easier to change than nature, because the very resemblance of
custom to nature is what constitutes the difficulty of changing it; as
Evenus says,

  "Practice, I say, my friend, doth long endure,
  And at the last is even very nature."

We have now said then what Self-Control is, what Imperfection of
Self-Control, what Endurance, and what Softness, and how these states
are mutually related.

XI

[Sidenote: II52b]

To consider the subject of Pleasure and Pain falls within the province
of the Social-Science Philosopher, since he it is who has to fix the
Master-End which is to guide us in dominating any object absolutely evil
or good.

But we may say more: an inquiry into their nature is absolutely
necessary. First, because we maintained that Moral Virtue and Moral Vice
are both concerned with Pains and Pleasures: next, because the greater
part of mankind assert that Happiness must include Pleasure (which by
the way accounts for the word they use, makarioz; chaireiu being the
root of that word).

Now some hold that no one Pleasure is good, either in itself or as a
matter of result, because Good and Pleasure are not identical. Others
that some Pleasures are good but the greater number bad. There is yet a
third view; granting that every Pleasure is good, still the Chief Good
cannot possibly be Pleasure.

In support of the first opinion (that Pleasure is utterly not-good) it
is urged that:

I. Every Pleasure is a sensible process towards a complete state; but
no such process is akin to the end to be attained: _e.g._ no process of
building to the completed house.

2. The man of Perfected Self-Mastery avoids Pleasures.

3. The man of Practical Wisdom aims at avoiding Pain, not at attaining
Pleasure.

4. Pleasures are an impediment to thought, and the more so the more
keenly they are felt. An obvious instance will readily occur.

5. Pleasure cannot be referred to any Art: and yet every good is the
result of some Art.

6. Children and brutes pursue Pleasures.

In support of the second (that not all Pleasures are good), That there
are some base and matter of reproach, and some even hurtful: because
some things that are pleasant produce disease.

In support of the third (that Pleasure is not the Chief Good), That it
is not an End but a process towards creating an End.

This is, I think, a fair account of current views on the matter.

XII


But that the reasons alleged do not prove it either to be not-good or
the Chief Good is plain from the following considerations.

First. Good being either absolute or relative, of course the natures and
states embodying it will be so too; therefore also the movements and the
processes of creation. So, of those which are thought to be bad
some will be bad absolutely, but relatively not bad, perhaps even
choiceworthy; some not even choiceworthy relatively to any particular
person, only at certain times or for a short time but not in themselves
choiceworthy.

Others again are not even Pleasures at all though they produce that
impression on the mind: all such I mean as imply pain and whose purpose
is cure; those of sick people, for instance.

Next, since Good may be either an active working or a state, those
[Greek: _kinaeseis_ or _geneseis_] which tend to place us in our natural
state are pleasant incidentally because of that *[Sidenote: 1153a]
tendency: but the active working is really in the desires excited in the
remaining (sound) part of our state or nature: for there are Pleasures
which have no connection with pain or desire: the acts of contemplative
intellect, for instance, in which case there is no deficiency in the
nature or state of him who performs the acts.

A proof of this is that the same pleasant thing does not produce the
sensation of Pleasure when the natural state is being filled up or
completed as when it is already in its normal condition: in this latter
case what give the sensation are things pleasant _per se_, in the former
even those things which are contrary. I mean, you find people taking
pleasure in sharp or bitter things of which no one is naturally or in
itself pleasant; of course not therefore the Pleasures arising from
them, because it is obvious that as is the classification of pleasant
things such must be that of the Pleasures arising from them.

Next, it does not follow that there must be something else better than
any given pleasure because (as some say) the End must be better than
the process which creates it. For it is not true that all Pleasures
are processes or even attended by any process, but (some are) active
workings or even Ends: in fact they result not from our coming to be
something but from our using our powers. Again, it is not true that the
End is, in every case, distinct from the process: it is true only in
the case of such processes as conduce to the perfecting of the natural
state.

For which reason it is wrong to say that Pleasure is "a sensible process
of production." For "process etc." should be substituted "active working
of the natural state," for "sensible" "unimpeded." The reason of its
being thought to be a "process etc." is that it is good in the highest
sense: people confusing "active working" and "process," whereas they
really are distinct.

Next, as to the argument that there are bad Pleasures because some
things which are pleasant are also hurtful to health, it is the same as
saying that some healthful things are bad for "business." In this sense,
of course, both may be said to be bad, but then this does not make
them out to be bad _simpliciter_: the exercise of the pure Intellect
sometimes hurts a man's health: but what hinders Practical Wisdom or
any state whatever is, not the Pleasure peculiar to, but some Pleasure
foreign to it: the Pleasures arising from the exercise of the pure
Intellect or from learning only promote each.

Next. "No Pleasure is the work of any Art." What else would you expect?
No active working is the work of any Art, only the faculty of so
working. Still the perfumer's Art or the cook's are thought to belong to
Pleasure.

Next. "The man of Perfected Self-Mastery avoids Pleasures." "The man
of Practical Wisdom aims at escaping Pain rather than at attaining
Pleasure."

"Children and brutes pursue Pleasures."

One answer will do for all.

We have already said in what sense all Pleasures are good _per se_ and
in what sense not all are good: it is the latter class that brutes and
children pursue, such as are accompanied by desire and pain, that is the
bodily Pleasures (which answer to this description) and the excesses of
them: in short, those in respect of which the man utterly destitute of
Self-Control is thus utterly destitute. And it is the absence of the
pain arising from these Pleasures that the man of Practical Wisdom
aims at. It follows that these Pleasures are what the man of Perfected
Self-Mastery avoids: for obviously he has Pleasures peculiarly his own.

[Sidenote: XIII 1153_b_] Then again, it is allowed that Pain is an evil
and a thing to be avoided partly as bad _per se_, partly as being a
hindrance in some particular way. Now the contrary of that which is to
be avoided, _quâ_ it is to be avoided, _i.e._ evil, is good. Pleasure
then must be _a_ good.

The attempted answer of Speusippus, "that Pleasure may be opposed and
yet not contrary to Pain, just as the greater portion of any magnitude
is contrary to the less but only opposed to the exact half," will not
hold: for he cannot say that Pleasure is identical with evil of any
kind. Again. Granting that some Pleasures are low, there is no reason
why some particular Pleasure may not be very good, just as some
particular Science may be although there are some which are low.

Perhaps it even follows, since each state may have active working
unimpeded, whether the active workings of all be Happiness or that of
some one of them, that this active working, if it be unimpeded, must be
choiceworthy: now Pleasure is exactly this. So that the Chief Good may
be Pleasure of some kind, though most Pleasures be (let us assume) low
_per se_.

And for this reason all men think the happy life is pleasant, and
interweave Pleasure with Happiness. Reasonably enough: because Happiness
is perfect, but no impeded active working is perfect; and therefore
the happy man needs as an addition the goods of the body and the goods
external and fortune that in these points he may not be fettered. As for
those who say that he who is being tortured on the wheel, or falls into
great misfortunes is happy provided only he be good, they talk nonsense,
whether they mean to do so or not. On the other hand, because fortune
is needed as an addition, some hold good fortune to be identical with
Happiness: which it is not, for even this in excess is a hindrance, and
perhaps then has no right to be called good fortune since it is good
only in so far as it contributes to Happiness.

The fact that all animals, brute and human alike, pursue Pleasure, is
some presumption of its being in a sense the Chief Good;

("There must be something in what most folks say,") only as one and
the same nature or state neither is nor is thought to be the best, so
neither do all pursue the same Pleasure, Pleasure nevertheless all do.
Nay further, what they pursue is, perhaps, not what they think nor what
they would say they pursue, but really one and the same: for in all
there is some instinct above themselves. But the bodily Pleasures have
received the name exclusively, because theirs is the most frequent form
and that which is universally partaken of; and so, because to many these
alone are known they believe them to be the only ones which exist.

[Sidenote: II54a]

It is plain too that, unless Pleasure and its active working be good, it
will not be true that the happy man's life embodies Pleasure: for why
will he want it on the supposition that it is not good and that he can
live even with Pain? because, assuming that Pleasure is not good, then
Pain is neither evil nor good, and so why should he avoid it?

Besides, the life of the good man is not more pleasurable than any other
unless it be granted that his active workings are so too.

XIV

Some inquiry into the bodily Pleasures is also necessary for those who
say that some Pleasures, to be sure, are highly choiceworthy (the good
ones to wit), but not the bodily Pleasures; that is, those which are the
object-matter of the man utterly destitute of Self-Control.

If so, we ask, why are the contrary Pains bad? they cannot be (on their
assumption) because the contrary of bad is good.

May we not say that the necessary bodily Pleasures are good in the sense
in which that which is not-bad is good? or that they are good only up
to a certain point? because such states or movements as cannot have too
much of the better cannot have too much of Pleasure, but those which can
of the former can also of the latter. Now the bodily Pleasures do admit
of excess: in fact the low bad man is such because he pursues the excess
of them instead of those which are necessary (meat, drink, and the
objects of other animal appetites do give pleasure to all, but not in
right manner or degree to all). But his relation to Pain is exactly the
contrary: it is not excessive Pain, but Pain at all, that he avoids
[which makes him to be in this way too a bad low man], because only
in the case of him who pursues excessive Pleasure is Pain contrary to
excessive Pleasure.

It is not enough however merely to state the truth, we should also show
how the false view arises; because this strengthens conviction. I mean,
when we have given a probable reason why that impresses people as true
which really is not true, it gives them a stronger conviction of the
truth. And so we must now explain why the bodily Pleasures appear to
people to be more choiceworthy than any others.

The first obvious reason is, that bodily Pleasure drives out Pain; and
because Pain is felt in excess men pursue Pleasure in excess, _i.e._
generally bodily Pleasure, under the notion of its being a remedy for
that Pain. These remedies, moreover, come to be violent ones; which is
the very reason they are pursued, since the impression they produce
on the mind is owing to their being looked at side by side with their
contrary.

And, as has been said before, there are the two following reasons why
bodily Pleasure is thought to be not-good.

1. Some Pleasures of this class are actings of a low nature, whether
congenital as in brutes, or acquired by custom as in low bad men.

2. Others are in the nature of cures, cures that is of some deficiency;
now of course it is better to have [the healthy state] originally than
that it should accrue afterwards.

[Sidenote: 1154b] But some Pleasures result when natural states are
being perfected: these therefore are good as a matter of result.

Again, the very fact of their being violent causes them to be pursued by
such as can relish no others: such men in fact create violent thirsts
for themselves (if harmless ones then we find no fault, if harmful then
it is bad and low) because they have no other things to take
pleasure in, and the neutral state is distasteful to some people
constitutionally; for toil of some kind is inseparable from life, as
physiologists testify, telling us that the acts of seeing or hearing are
painful, only that we are used to the pain and do not find it out.

Similarly in youth the constant growth produces a state much like
that of vinous intoxication, and youth is pleasant. Again, men of the
melancholic temperament constantly need some remedial process (because
the body, from its temperament, is constantly being worried), and they
are in a chronic state of violent desire. But Pleasure drives out Pain;
not only such Pleasure as is directly contrary to Pain but even any
Pleasure provided it be strong: and this is how men come to be utterly
destitute of Self-Mastery, _i.e._ low and bad.

But those Pleasures which are unconnected with Pains do not admit of
excess: _i.e._ such as belong to objects which are naturally pleasant
and not merely as a matter of result: by the latter class I mean such
as are remedial, and the reason why these are thought to be pleasant is
that the cure results from the action in some way of that part of the
constitution which remains sound. By "pleasant naturally" I mean such as
put into action a nature which is pleasant.

The reason why no one and the same thing is invariably pleasant is that
our nature is, not simple, but complex, involving something different
from itself (so far as we are corruptible beings). Suppose then that one
part of this nature be doing something, this something is, to the other
part, unnatural: but, if there be an equilibrium of the two natures,
then whatever is being done is indifferent. It is obvious that if there
be any whose nature is simple and not complex, to such a being the same
course of acting will always be the most pleasurable.

For this reason it is that the Divinity feels Pleasure which is always
one, _i.e._ simple: not motion merely but also motionlessness acts, and
Pleasure resides rather in the absence than in the presence of motion.

The reason why the Poet's dictum "change is of all things most pleasant"
is true, is "a baseness in our blood;" for as the bad man is easily
changeable, bad must be also the nature that craves change, _i.e._ it is
neither simple nor good.

We have now said our say about Self-Control and its opposite; and about
Pleasure and Pain. What each is, and how the one set is good the other
bad. We have yet to speak of Friendship.




BOOK VIII

[Sidenote: I 1155_a_] Next would seem properly to follow a dissertation
on Friendship: because, in the first place, it is either itself a virtue
or connected with virtue; and next it is a thing most necessary for
life, since no one would choose to live without friends though he should
have all the other good things in the world: and, in fact, men who are
rich or possessed of authority and influence are thought to have special
need of friends: for where is the use of such prosperity if there be
taken away the doing of kindnesses of which friends are the most usual
and most commendable objects? Or how can it be kept or preserved without
friends? because the greater it is so much the more slippery and
hazardous: in poverty moreover and all other adversities men think
friends to be their only refuge.

Furthermore, Friendship helps the young to keep from error: the old, in
respect of attention and such deficiencies in action as their weakness
makes them liable to; and those who are in their prime, in respect of
noble deeds ("They _two_ together going," Homer says, you may remember),
because they are thus more able to devise plans and carry them out.

Again, it seems to be implanted in us by Nature: as, for instance, in
the parent towards the offspring and the offspring towards the parent
(not merely in the human species, but likewise in birds and most
animals), and in those of the same tribe towards one another, and
specially in men of the same nation; for which reason we commend those
men who love their fellows: and one may see in the course of travel how
close of kin and how friendly man is to man.

Furthermore, Friendship seems to be the bond of Social Communities, and
legislators seem to be more anxious to secure it than Justice even. I
mean, Unanimity is somewhat like to Friendship, and this they certainly
aim at and specially drive out faction as being inimical.

Again, where people are in Friendship Justice is not required; but, on
the other hand, though they are just they need Friendship in addition,
and that principle which is most truly just is thought to partake of the
nature of Friendship.

Lastly, not only is it a thing necessary but honourable likewise: since
we praise those who are fond of friends, and the having numerous friends
is thought a matter of credit to a man; some go so far as to hold, that
"good man" and "friend" are terms synonymous.

Yet the disputed points respecting it are not few: some men lay down
that it is a kind of resemblance, and that men who are like one another
are friends: whence come the common sayings, "Like will to like," "Birds
of a feather," and so on. Others, on the contrary, say, that all such
come under the maxim, "Two of a trade never agree."

[Sidenote: 1155b] Again, some men push their inquiries on these points
higher and reason physically: as Euripides, who says,

  "The earth by drought consumed doth love the rain,
  And the great heaven, overcharged with rain,
  Doth love to fall in showers upon the earth."

Heraclitus, again, maintains, that "contrariety is expedient, and that
the best agreement arises from things differing, and that all things
come into being in the way of the principle of antagonism."

Empedocles, among others, in direct opposition to these, affirms, that
"like aims at like."

These physical questions we will take leave to omit, inasmuch as they
are foreign to the present inquiry; and we will examine such as are
proper to man and concern moral characters and feelings: as, for
instance, "Does Friendship arise among all without distinction, or is it
impossible for bad men to be friends?" and, "Is there but one species of
Friendship, or several?" for they who ground the opinion that there is
but one on the fact that Friendship admits of degrees hold that upon
insufficient proof; because things which are different in species admit
likewise of degrees (on this point we have spoken before).


II

Our view will soon be cleared on these points when we have ascertained
what is properly the object-matter of Friendship: for it is thought that
not everything indiscriminately, but some peculiar matter alone, is the
object of this affection; that is to say, what is good, or pleasurable,
or useful. Now it would seem that that is useful through which accrues
any good or pleasure, and so the objects of Friendship, as absolute
Ends, are the good and the pleasurable.

A question here arises; whether it is good absolutely or that which is
good to the individuals, for which men feel Friendship (these two being
sometimes distinct): and similarly in respect of the pleasurable. It
seems then that each individual feels it towards that which is good to
himself, and that abstractedly it is the real good which is the object
of Friendship, and to each individual that which is good to each. It
comes then to this; that each individual feels Friendship not for what
_is_ but for that which _conveys to his mind the impression of being_
good to himself. But this will make no real difference, because that
which is truly the object of Friendship will also convey this impression
to the mind.

There are then three causes from which men feel Friendship: but the term
is not applied to the case of fondness for things inanimate because
there is no requital of the affection nor desire for the good of those
objects: it certainly savours of the ridiculous to say that a man fond
of wine wishes well to it: the only sense in which it is true being that
he wishes it to be kept safe and sound for his own use and benefit. But
to the friend they say one should wish all good for his sake. And when
men do thus wish good to another (he not *[Sidenote: 1156a]
reciprocating the feeling), people call them Kindly; because Friendship
they describe as being "Kindliness between persons who reciprocate it."
But must they not add that the feeling must be mutually known? for many
men are kindly disposed towards those whom they have never seen but whom
they conceive to be amiable or useful: and this notion amounts to the
same thing as a real feeling between them.

Well, these are plainly Kindly-disposed towards one another: but how can
one call them friends while their mutual feelings are unknown to one
another? to complete the idea of Friendship, then, it is requisite that
they have kindly feelings towards one another, and wish one another good
from one of the aforementioned causes, and that these kindly feelings
should be mutually known.

III


As the motives to Friendship differ in kind so do the respective
feelings and Friendships. The species then of Friendship are three, in
number equal to the objects of it, since in the line of each there may
be "mutual affection mutually known."

Now they who have Friendship for one another desire one another's good
according to the motive of their Friendship; accordingly they whose
motive is utility have no Friendship for one another really, but only in
so far as some good arises to them from one another.

And they whose motive is pleasure are in like case: I mean, they have
Friendship for men of easy pleasantry, not because they are of a given
character but because they are pleasant to themselves. So then they
whose motive to Friendship is utility love their friends for what is
good to themselves; they whose motive is pleasure do so for what is
pleasurable to themselves; that is to say, not in so far as the friend
beloved _is_ but in so far as he is useful or pleasurable. These
Friendships then are a matter of result: since the object is not beloved
in that he is the man he is but in that he furnishes advantage or
pleasure as the case may be. Such Friendships are of course very liable
to dissolution if the parties do not continue alike: I mean, that the
others cease to have any Friendship for them when they are no longer
pleasurable or useful. Now it is the nature of utility not to be
permanent but constantly varying: so, of course, when the motive which
made them friends is vanished, the Friendship likewise dissolves; since
it existed only relatively to those circumstances.

Friendship of this kind is thought to exist principally among the old
(because men at that time of life pursue not what is pleasurable but
what is profitable); and in such, of men in their prime and of the
young, as are given to the pursuit of profit. They that are such have no
intimate intercourse with one another; for sometimes they are not
even pleasurable to one another; nor, in fact, do they desire such
intercourse unless their friends are profitable to them, because they
are pleasurable only in so far as they have hopes of advantage. With
these Friendships is commonly ranked that of hospitality.

But the Friendship of the young is thought to be based on the motive
of pleasure: because they live at the beck and call of passion and
generally pursue what is pleasurable to themselves and the object of the
present moment: and as their age changes so likewise do their pleasures.

This is the reason why they form and dissolve Friendships rapidly: since
the Friendship changes with the pleasurable object and such pleasure
changes quickly.

[Sidenote: 1156b] The young are also much given up to Love; this passion
being, in great measure, a matter of impulse and based on pleasure: for
which cause they conceive Friendships and quickly drop them, changing
often in the same day: but these wish for society and intimate
intercourse with their friends, since they thus attain the object of
their Friendship.

That then is perfect Friendship which subsists between those who are
good and whose similarity consists in their goodness: for these men wish
one another's good in similar ways; in so far as they are good (and good
they are in themselves); and those are specially friends who wish good
to their friends for their sakes, because they feel thus towards them on
their own account and not as a mere matter of result; so the Friendship
between these men continues to subsist so long as they are good; and
goodness, we know, has in it a principle of permanence.

Moreover, each party is good abstractedly and also relatively to his
friend, for all good men are not only abstractedly good but also useful
to one another. Such friends are also mutually pleasurable because
all good men are so abstractedly, and also relatively to one another,
inasmuch as to each individual those actions are pleasurable which
correspond to his nature, and all such as are like them. Now when men
are good these will be always the same, or at least similar.

Friendship then under these circumstances is permanent, as we should
reasonably expect, since it combines in itself all the requisite
qualifications of friends. I mean, that Friendship of whatever kind is
based upon good or pleasure (either abstractedly or relatively to the
person entertaining the sentiment of Friendship), and results from a
similarity of some sort; and to this kind belong all the aforementioned
requisites in the parties themselves, because in this the parties are
similar, and so on: moreover, in it there is the abstractedly good and
the abstractedly pleasant, and as these are specially the object-matter
of Friendship so the feeling and the state of Friendship is found most
intense and most excellent in men thus qualified.

Rare it is probable Friendships of this kind will be, because men
of this kind are rare. Besides, all requisite qualifications being
presupposed, there is further required time and intimacy: for, as the
proverb says, men cannot know one another "till they have eaten the
requisite quantity of salt together;" nor can they in fact admit one
another to intimacy, much less be friends, till each has appeared to
the other and been proved to be a fit object of Friendship. They who
speedily commence an interchange of friendly actions may be said to wish
to be friends, but they are not so unless they are also proper objects
of Friendship and mutually known to be such: that is to say, a desire
for Friendship may arise quickly but not Friendship itself.


IV

Well, this Friendship is perfect both in respect of the time and in all
other points; and exactly the same and similar results accrue to each
party from the other; which ought to be the case between friends.

[Sidenote: II57a] The friendship based upon the pleasurable is, so to
say, a copy of this, since the good are sources of pleasure to one
another: and that based on utility likewise, the good being also
useful to one another. Between men thus connected Friendships are
most permanent when the same result accrues to both from one another,
pleasure, for instance; and not merely so but from the same source, as
in the case of two men of easy pleasantry; and not as it is in that of a
lover and the object of his affection, these not deriving their pleasure
from the same causes, but the former from seeing the latter and the
latter from receiving the attentions of the former: and when the bloom
of youth fades the Friendship sometimes ceases also, because then the
lover derives no pleasure from seeing and the object of his affection
ceases to receive the attentions which were paid before: in many cases,
however, people so connected continue friends, if being of similar
tempers they have come from custom to like one another's disposition.

Where people do not interchange pleasure but profit in matters of Love,
the Friendship is both less intense in degree and also less permanent:
in fact, they who are friends because of advantage commonly part when
the advantage ceases; for, in reality, they never were friends of one
another but of the advantage.

So then it appears that from motives of pleasure or profit bad men may
be friends to one another, or good men to bad men or men of neutral
character to one of any character whatever: but disinterestedly, for the
sake of one another, plainly the good alone can be friends; because
bad men have no pleasure even in themselves unless in so far as some
advantage arises.

And further, the Friendship of the good is alone superior to calumny;
it not being easy for men to believe a third person respecting one
whom they have long tried and proved: there is between good men mutual
confidence, and the feeling that one's friend would never have done one
wrong, and all other such things as are expected in Friendship really
worthy the name; but in the other kinds there is nothing to prevent all
such suspicions.

I call them Friendships, because since men commonly give the name of
friends to those who are connected from motives of profit (which is
justified by political language, for alliances between states are
thought to be contracted with a view to advantage), and to those who are
attached to one another by the motive of pleasure (as children are), we
may perhaps also be allowed to call such persons friends, and say there
are several species of Friendship; primarily and specially that of
the good, in that they are good, and the rest only in the way of
resemblance: I mean, people connected otherwise are friends in that way
in which there arises to them somewhat good and some mutual resemblance
(because, we must remember the pleasurable is good to those who are fond
of it).

These secondary Friendships, however, do not combine very well; that is
to say, the same persons do not become friends by reason of advantage
and by reason of the pleasurable, for these matters of result are not
often combined. And Friendship having been divided into these kinds, bad
[Sidenote: _1157b_] men will be friends by reason of pleasure or profit,
this being their point of resemblance; while the good are friends for
one another's sake, that is, in so far as they are good.

These last may be termed abstractedly and simply friends, the former as
a matter of result and termed friends from their resemblance to these
last.


V

Further; just as in respect of the different virtues some men are termed
good in respect of a certain inward state, others in respect of acts
of working, so is it in respect of Friendship: I mean, they who live
together take pleasure in, and impart good to, one another: but they who
are asleep or are locally separated do not perform acts, but only are in
such a state as to act in a friendly way if they acted at all: distance
has in itself no direct effect upon Friendship, but only prevents the
acting it out: yet, if the absence be protracted, it is thought to cause
a forgetfulness even of the Friendship: and hence it has been said,
"many and many a Friendship doth want of intercourse destroy."

Accordingly, neither the old nor the morose appear to be calculated for
Friendship, because the pleasurableness in them is small, and no one can
spend his days in company with that which is positively painful or even
not pleasurable; since to avoid the painful and aim at the pleasurable
is one of the most obvious tendencies of human nature. They who get on
with one another very fairly, but are not in habits of intimacy, are
rather like people having kindly feelings towards one another than
friends; nothing being so characteristic of friends as the living with
one another, because the necessitous desire assistance, and the happy
companionship, they being the last persons in the world for solitary
existence: but people cannot spend their time together unless they are
mutually pleasurable and take pleasure in the same objects, a quality
which is thought to appertain to the Friendship of companionship.

The connection then subsisting between the good is Friendship _par
excellence_, as has already been frequently said: since that which is
abstractedly good or pleasant is thought to be an object of Friendship
and choiceworthy, and to each individual whatever is such to him;
and the good man to the good man for both these reasons. (Now the
entertaining the sentiment is like a feeling, but Friendship itself
like a state: because the former may have for its object even things
inanimate, but requital of Friendship is attended with moral choice
which proceeds from a moral state: and again, men wish good to the
objects of their Friendship for their sakes, not in the way of a mere
feeling but of moral state.).

And the good, in loving their friend, love their own good (inasmuch as
the good man, when brought into that relation, becomes a good to him
with whom he is so connected), so that either party loves his own
good, and repays his friend equally both in wishing well and in the
pleasurable: for equality is said to be a tie of Friendship. Well, these
points belong most to the Friendship between good men.

But between morose or elderly men Friendship is less apt to arise,
because they are somewhat awkward-tempered, and take less pleasure in
intercourse and society; these being thought to be specially friendly
and productive of Friendship: and so young men become friends quickly,
old men not so (because people do not become friends with any, unless
they take pleasure in them); and in like manner neither do the morose.
Yet men of these classes entertain kindly feelings towards one another:
they wish good to one another and render mutual assistance in respect of
their needs, but they are not quite friends, because they neither
spend their time together nor take pleasure in one another, which
circumstances are thought specially to belong to Friendship.

To be a friend to many people, in the way of the perfect Friendship, is
not possible; just as you cannot be in love with many at once: it is,
so to speak, a state of excess which naturally has but one object; and
besides, it is not an easy thing for one man to be very much pleased
with many people at the same time, nor perhaps to find many really good.
Again, a man needs experience, and to be in habits of close intimacy,
which is very difficult.

But it _is_ possible to please many on the score of advantage and
pleasure: because there are many men of the kind, and the services may
be rendered in a very short time.

Of the two imperfect kinds that which most resembles the perfect is the
Friendship based upon pleasure, in which the same results accrue from
both and they take pleasure in one another or in the same objects; such
as are the Friendships of the young, because a generous spirit is most
found in these. The Friendship because of advantage is the connecting
link of shopkeepers.

Then again, the very happy have no need of persons who are profitable,
but of pleasant ones they have because they wish to have people to live
intimately with; and what is painful they bear for a short time indeed,
but continuously no one could support it, nay, not even the Chief Good
itself, if it were painful to him individually: and so they look out for
pleasant friends: perhaps they ought to require such to be good also;
and good moreover to themselves individually, because then they will
have all the proper requisites of Friendship.

Men in power are often seen to make use of several distinct friends:
for some are useful to them and others pleasurable, but the two are not
often united: because they do not, in fact, seek such as shall combine
pleasantness and goodness, nor such as shall be useful for honourable
purposes: but with a view to attain what is pleasant they look out for
men of easy-pleasantry; and again, for men who are clever at executing
any business put into their hands: and these qualifications are not
commonly found united in the same man.

It has been already stated that the good man unites the qualities of
pleasantness and usefulness: but then such a one will not be a friend to
a superior unless he be also his superior in goodness: for if this be
not the case, he cannot, being surpassed in one point, make things
equal by a proportionate degree of Friendship. And characters who unite
superiority of station and goodness are not common. Now all the kinds
of Friendship which have been already mentioned exist in a state of
equality, inasmuch as either the same results accrue to both and they
wish the same things to one another, or else they barter one thing
against another; pleasure, for instance, against profit: it has been
said already that Friendships of this latter kind are less intense in
degree and less permanent.

And it is their resemblance or dissimilarity to the same thing which
makes them to be thought to be and not to be Friendships: they show like
Friendships in right of their likeness to that which is based on virtue
(the one kind having the pleasurable, the other the profitable, both
of which belong also to the other); and again, they do not show like
Friendships by reason of their unlikeness to that true kind; which
unlikeness consists herein, that while that is above calumny and so
permanent these quickly change and differ in many other points.


VII

But there is another form of Friendship, that, namely, in which the one
party is superior to the other; as between father and son, elder and
younger, husband and wife, ruler and ruled. These also differ one from
another: I mean, the Friendship between parents and children is not the
same as between ruler and the ruled, nor has the father the same towards
the son as the son towards the father, nor the husband towards the wife
as she towards him; because the work, and therefore the excellence, of
each of these is different, and different therefore are the causes of
their feeling Friendship; distinct and different therefore are their
feelings and states of Friendship.

And the same results do not accrue to each from the other, nor in fact
ought they to be looked for: but, when children render to their parents
what they ought to the authors of their being, and parents to their sons
what they ought to their offspring, the Friendship between such parties
will be permanent and equitable.

Further; the feeling of Friendship should be in a due proportion in all
Friendships which are between superior and inferior; I mean, the better
man, or the more profitable, and so forth, should be the object of a
stronger feeling than he himself entertains, because when the feeling of
Friendship comes to be after a certain rate then equality in a certain
sense is produced, which is thought to be a requisite in Friendship.

(It must be remembered, however, that the equal is not in the same case
as regards Justice and Friendship: for in strict Justice the exactly
proportioned equal ranks first, and the actual numerically equal ranks
second, while in Friendship this is exactly reversed.)

[Sidenote: 1159a] And that equality is thus requisite is plainly shown
by the occurrence of a great difference of goodness or badness, or
prosperity, or something else: for in this case, people are not any
longer friends, nay they do not even feel that they ought to be. The
clearest illustration is perhaps the case of the gods, because they are
most superior in all good things. It is obvious too, in the case of
kings, for they who are greatly their inferiors do not feel entitled to
be friends to them; nor do people very insignificant to be friends to
those of very high excellence or wisdom. Of course, in such cases it
is out of the question to attempt to define up to what point they may
continue friends: for you may remove many points of agreement and the
Friendship last nevertheless; but when one of the parties is very far
separated (as a god from men), it cannot continue any longer.

This has given room for a doubt, whether friends do really wish to their
friends the very highest goods, as that they may be gods: because, in
case the wish were accomplished, they would no longer have them for
friends, nor in fact would they have the good things they had, because
friends are good things. If then it has been rightly said that a friend
wishes to his friend good things for that friend's sake, it must be
understood that he is to remain such as he now is: that is to say, he
will wish the greatest good to him of which as man he is capable: yet
perhaps not all, because each man desires good for himself most of all.

VIII

It is thought that desire for honour makes the mass of men wish rather
to be the objects of the feeling of Friendship than to entertain it
themselves (and for this reason they are fond of flatterers, a flatterer
being a friend inferior or at least pretending to be such and rather to
entertain towards another the feeling of Friendship than to be himself
the object of it), since the former is thought to be nearly the same as
being honoured, which the mass of men desire. And yet men seem to choose
honour, not for its own sake, but incidentally: I mean, the common run
of men delight to be honoured by those in power because of the hope it
raises; that is they think they shall get from them anything they may
happen to be in want of, so they delight in honour as an earnest of
future benefit. They again who grasp at honour at the hands of the good
and those who are really acquainted with their merits desire to confirm
their own opinion about themselves: so they take pleasure in the
conviction that they are good, which is based on the sentence of those
who assert it. But in being the objects of Friendship men delight for
its own sake, and so this may be judged to be higher than being honoured
and Friendship to be in itself choiceworthy. Friendship, moreover, is
thought to consist in feeling, rather than being the object of, the
sentiment of Friendship, which is proved by the delight mothers have in
the feeling: some there are who give their children to be adopted and
brought up by others, and knowing them bear this feeling towards them
never seeking to have it returned, if both are not possible; but seeming
to be content with seeing them well off and bearing this feeling
themselves towards them, even though they, by reason of ignorance, never
render to them any filial regard or love.

Since then Friendship stands rather in the entertaining, than in being
the object of, the sentiment, and they are praised who are fond of their
friends, it seems that entertaining--*[Sidenote: II59b]the sentiment is
the Excellence of friends; and so, in whomsoever this exists in due
proportion these are stable friends and their Friendship is permanent.
And in this way may they who are unequal best be friends, because they
may thus be made equal.

Equality, then, and similarity are a tie to Friendship, and specially
the similarity of goodness, because good men, being stable in
themselves, are also stable as regards others, and neither ask degrading
services nor render them, but, so to say, rather prevent them: for it is
the part of the good neither to do wrong themselves nor to allow their
friends in so doing.

The bad, on the contrary, have no principle of stability: in fact, they
do not even continue like themselves: only they come to be friends for
a short time from taking delight in one another's wickedness. Those
connected by motives of profit, or pleasure, hold together somewhat
longer: so long, that is to say, as they can give pleasure or profit
mutually.

The Friendship based on motives of profit is thought to be most of all
formed out of contrary elements: the poor man, for instance, is thus a
friend of the rich, and the ignorant of the man of information; that
is to say, a man desiring that of which he is, as it happens, in want,
gives something else in exchange for it. To this same class we may refer
the lover and beloved, the beautiful and the ill-favoured. For this
reason lovers sometimes show in a ridiculous light by claiming to be the
objects of as intense a feeling as they themselves entertain: of course
if they are equally fit objects of Friendship they are perhaps entitled
to claim this, but if they have nothing of the kind it is ridiculous.

Perhaps, moreover, the contrary does not aim at its contrary for its own
sake but incidentally: the mean is really what is grasped at; it being
good for the dry, for instance, not to become wet but to attain the
mean, and so of the hot, etc. However, let us drop these questions,
because they are in fact somewhat foreign to our purpose.

IX

It seems too, as was stated at the commencement, that Friendship and
Justice have the same object-matter, and subsist between the same
persons: I mean that in every Communion there is thought to be some
principle of Justice and also some Friendship: men address as friends,
for instance, those who are their comrades by sea, or in war, and in
like manner also those who are brought into Communion with them in other
ways: and the Friendship, because also the Justice, is co-extensive with
the Communion, This justifies the common proverb, "the goods of friends
are common," since Friendship rests upon Communion.

[1160a] Now brothers and intimate companions have all in common, but
other people have their property separate, and some have more in common
and others less, because the Friendships likewise differ in degree. So
too do the various principles of Justice involved, not being the same
between parents and children as between brothers, nor between companions
as between fellow-citizens merely, and so on of all the other
conceivable Friendships. Different also are the principles of Injustice
as regards these different grades, and the acts become intensified by
being done to friends; for instance, it is worse to rob your companion
than one who is merely a fellow-citizen; to refuse help to a brother
than to a stranger; and to strike your father than any one else. So then
the Justice naturally increases with the degree of Friendship, as being
between the same parties and of equal extent.

All cases of Communion are parts, so to say, of the great Social one,
since in them men associate with a view to some advantage and to procure
some of those things which are needful for life; and the great Social
Communion is thought originally to have been associated and to
continue for the sake of some advantage: this being the point at which
legislators aim, affirming that to be just which is generally expedient.
All the other cases of Communion aim at advantage in particular points;
the crew of a vessel at that which is to result from the voyage which is
undertaken with a view to making money, or some such object; comrades in
war at that which is to result from the war, grasping either at wealth
or victory, or it may be a political position; and those of the same
tribe, or Demus, in like manner.

Some of them are thought to be formed for pleasure's sake, those, for
instance, of bacchanals or club-fellows, which are with a view to
Sacrifice or merely company. But all these seem to be ranged under
the great Social one, inasmuch as the aim of this is, not merely the
expediency of the moment but, for life and at all times; with a view
to which the members of it institute sacrifices and their attendant
assemblies, to render honour to the gods and procure for themselves
respite from toil combined with pleasure. For it appears that
sacrifices and religious assemblies in old times were made as a kind of
first-fruits after the ingathering of the crops, because at such seasons
they had most leisure.

So then it appears that all the instances of Communion are parts of the
great Social one: and corresponding Friendships will follow upon such
Communions.


X

Of Political Constitutions there are three kinds; and equal in number
are the deflections from them, being, so to say, corruptions of them.

The former are Kingship, Aristocracy, and that which recognises the
principle of wealth, which it seems appropriate to call Timocracy (I
give to it the name of a political constitution because people commonly
do so). Of these the best is Monarchy, and Timocracy the worst.

[Sidenote: II6ob] From Monarchy the deflection is Despotism; both being
Monarchies but widely differing from each other; for the Despot looks to
his own advantage, but the King to that of his subjects: for he is in
fact no King who is not thoroughly independent and superior to the rest
in all good things, and he that is this has no further wants: he will
not then have to look to his own advantage but to that of his subjects,
for he that is not in such a position is a mere King elected by lot for
the nonce.

But Despotism is on a contrary footing to this Kingship, because the
Despot pursues his own good: and in the case of this its inferiority
is most evident, and what is worse is contrary to what is best. The
Transition to Despotism is made from Kingship, Despotism being a corrupt
form of Monarchy, that is to say, the bad King comes to be a Despot.

From Aristocracy to Oligarchy the transition is made by the fault of the
Rulers in distributing the public property contrary to right proportion;
and giving either all that is good, or the greatest share, to
themselves; and the offices to the same persons always, making wealth
their idol; thus a few bear rule and they bad men in the place of the
best.

From Timocracy the transition is to Democracy, they being contiguous:
for it is the nature of Timocracy to be in the hands of a multitude,
and all in the same grade of property are equal. Democracy is the least
vicious of all, since herein the form of the constitution undergoes
least change.

Well, these are generally the changes to which the various Constitutions
are liable, being the least in degree and the easiest to make.

Likenesses, and, as it were, models of them, one may find even in
Domestic life: for instance, the Communion between a Father and his Sons
presents the figure of Kingship, because the children are the Father's
care: and hence Homer names Jupiter Father because Kingship is intended
to be a paternal rule. Among the Persians, however, the Father's rule is
Despotic, for they treat their Sons as slaves. (The relation of Master
to Slaves is of the nature of Despotism because the point regarded
herein is the Master's interest): this now strikes me to be as it ought,
but the Persian custom to be mistaken; because for different persons
there should be different rules. [Sidenote: 1161a] Between Husband and
Wife the relation takes the form of Aristocracy, because he rules by
right and in such points only as the Husband should, and gives to
the Wife all that befits her to have. Where the Husband lords it in
everything he changes the relation into an Oligarchy; because he does
it contrary to right and not as being the better of the two. In some
instances the Wives take the reins of government, being heiresses: here
the rule is carried on not in right of goodness but by reason of wealth
and power, as it is in Oligarchies.

Timocracy finds its type in the relation of Brothers: they being equal
except as to such differences as age introduces: for which reason, if
they are very different in age, the Friendship comes to be no longer
a fraternal one: while Democracy is represented specially by families
which have no head (all being there equal), or in which the proper head
is weak and so every member does that which is right in his own eyes.


XI

Attendant then on each form of Political Constitution there plainly is
Friendship exactly co-extensive with the principle of Justice; that
between a King and his Subjects being in the relation of a superiority
of benefit, inasmuch as he benefits his subjects; it being assumed that
he is a good king and takes care of their welfare as a shepherd tends
his flock; whence Homer (to quote him again) calls Agamemnon, "shepherd
of the people." And of this same kind is the Paternal Friendship, only
that it exceeds the former in the greatness of the benefits done;
because the father is the author of being (which is esteemed the
greatest benefit) and of maintenance and education (these things are
also, by the way, ascribed to ancestors generally): and by the law of
nature the father has the right of rule over his sons, ancestors over
their descendants, and the king over his subjects.

These friendships are also between superiors and inferiors, for which
reason parents are not merely loved but also honoured. The principle of
Justice also between these parties is not exactly the same but according
to proportiton, because so also is the Friendship.

Now between Husband and Wife there is the same Friendship as in
Aristocracy: for the relation is determined by relative excellence, and
the better person has the greater good and each has what befits: so too
also is the principle of Justice between them.

The Fraternal Friendship is like that of Companions, because brothers
are equal and much of an age, and such persons have generally like
feelings and like dispositions. Like to this also is the Friendship of a
Timocracy, because the citizens are intended to be equal and equitable:
rule, therefore, passes from hand to hand, and is distributed on equal
terms: so too is the Friendship accordingly.

[Sidenote: 1161b] In the deflections from the constitutional forms, just
as the principle of Justice is but small so is the Friendship also: and
least of all in the most perverted form: in Despotism there is little
or no Friendship. For generally wherever the ruler and the ruled have
nothing in common there is no Friendship because there is no Justice;
but the case is as between an artisan and his tool, or between soul and
body, and master and slave; all these are benefited by those who use
them, but towards things inanimate there is neither Friendship nor
Justice: nor even towards a horse or an ox, or a slave _quâ_ slave,
because there is nothing in common: a slave as such is an animate tool,
a tool an inanimate slave. _Quâ_ slave, then, there is no Friendship
towards him, only _quâ_ man: for it is thought that there is some
principle of Justice between every man, and every other who can share in
law and be a party to an agreement; and so somewhat of Friendship, in so
far as he is man. So in Despotisms the Friendships and the principle of
Justice are inconsiderable in extent, but in Democracies they are most
considerable because they who are equal have much in common.

XII


Now of course all Friendship is based upon Communion, as has been
already stated: but one would be inclined to separate off from the rest
the Friendship of Kindred, and that of Companions: whereas those of men
of the same city, or tribe, or crew, and all such, are more peculiarly,
it would seem, based upon Communion, inasmuch as they plainly exist in
right of some agreement expressed or implied: among these one may rank
also the Friendship of Hospitality,

The Friendship of Kindred is likewise of many kinds, and appears in all
its varieties to depend on the Parental: parents, I mean, love their
children as being a part of themselves, children love their parents as
being themselves somewhat derived from them. But parents know their
offspring more than these know that they are from the parents, and the
source is more closely bound to that which is produced than that which
is produced is to that which formed it: of course, whatever is derived
from one's self is proper to that from which it is so derived (as, for
instance, a tooth or a hair, or any other thing whatever to him that
has it): but the source to it is in no degree proper, or in an inferior
degree at least.

Then again the greater length of time comes in: the parents love their
offspring from the first moment of their being, but their offspring
them only after a lapse of time when they have attained intelligence
or instinct. These considerations serve also to show why mothers have
greater strength of affection than fathers.

Now parents love their children as themselves (since what is derived
from themselves becomes a kind of other Self by the fact of separation),
but children their parents as being sprung from them. And brothers love
one another from being sprung from the same; that is, their sameness
with the common stock creates a sameness with one another; whence come
the phrases, "same blood," "root," and so on. In fact they are the same,
in a sense, even in the separate distinct individuals.

Then again the being brought up together, and the nearness of age, are
a great help towards Friendship, for a man likes one of his own age and
persons who are used to one another are companions, which accounts
for the resemblance between the Friendship of Brothers and that of
Companions.

[Sidenote:1162a] And cousins and all other relatives derive their bond
of union from these, that is to say, from their community of origin: and
the strength of this bond varies according to their respective distances
from the common ancestor.

Further: the Friendship felt by children towards parents, and by men
towards the gods, is as towards something good and above them; because
these have conferred the greatest possible benefits, in that they are
the causes of their being and being nourished, and of their having been
educated after they were brought into being.

And Friendship of this kind has also the pleasurable and the profitable
more than that between persons unconnected by blood, in proportion as
their life is also more shared in common. Then again in the Fraternal
Friendship there is all that there is in that of Companions, and more in
the good, and generally in those who are alike; in proportion as they
are more closely tied and from their very birth have a feeling of
affection for one another to begin with, and as they are more like in
disposition who spring from the same stock and have grown up together
and been educated alike: and besides this they have the greatest
opportunities in respect of time for proving one another, and can
therefore depend most securely upon the trial. The elements
of Friendship between other consanguinities will be of course
proportionably similar.

Between Husband and Wife there is thought to be Friendship by a law of
nature: man being by nature disposed to pair, more than to associate in
Communities: in proportion as the family is prior in order of time and
more absolutely necessary than the Community. And procreation is more
common to him with other animals; all the other animals have Communion
thus far, but human creatures cohabit not merely for the sake of
procreation but also with a view to life in general: because in this
connection the works are immediately divided, and some belong to the
man, others to the woman: thus they help one the other, putting what is
peculiar to each into the common stock.

And for these reasons this Friendship is thought to combine the
profitable and the pleasurable: it will be also based upon virtue if
they are good people; because each has goodness and they may take
delight in this quality in each other. Children too are thought to be a
tie: accordingly the childless sooner separate, for the children are a
good common to both and anything in common is a bond of union.

The question how a man is to live with his wife, or (more generally) one
friend with another, appears to be no other than this, how it is just
that they should: because plainly there is not the same principle
of Justice between a friend and friend, as between strangers, or
companions, or mere chance fellow-travellers.

XIII

[Sidenote:1162b] There are then, as was stated at the commencement of
this book, three kinds of Friendship, and in each there may be friends
on a footing of equality and friends in the relation of superior and
inferior; we find, I mean, that people who are alike in goodness, become
friends, and better with worse, and so also pleasant people; again,
because of advantage people are friends, either balancing exactly their
mutual profitableness or differing from one another herein. Well then,
those who are equal should in right of this equality be equalised also
by the degree of their Friendship and the other points, and those who
are on a footing of inequality by rendering Friendship in proportion to
the superiority of the other party.

Fault-finding and blame arises, either solely or most naturally, in
Friendship of which utility is the motive: for they who are friends by
reason of goodness, are eager to do kindnesses to one another because
this is a natural result of goodness and Friendship; and when men are
vying with each other for this End there can be no fault-finding nor
contention: since no one is annoyed at one who entertains for him the
sentiment of Friendship and does kindnesses to him, but if of a refined
mind he requites him with kind actions. And suppose that one of the two
exceeds the other, yet as he is attaining his object he will not find
fault with his friend, for good is the object of each party.

Neither can there well be quarrels between men who are friends for
pleasure's sake: because supposing them to delight in living together
then both attain their desire; or if not a man would be put in a
ridiculous light who should find fault with another for not pleasing
him, since it is in his power to forbear intercourse with him. But
the Friendship because of advantage is very liable to fault-finding;
because, as the parties use one another with a view to advantage, the
requirements are continually enlarging, and they think they have less
than of right belongs to them, and find fault because though justly
entitled they do not get as much as they want: while they who do the
kindnesses, can never come up to the requirements of those to whom they
are being done.

It seems also, that as the Just is of two kinds, the unwritten and the
legal, so Friendship because of advantage is of two kinds, what may
be called the Moral, and the Legal: and the most fruitful source of
complaints is that parties contract obligations and discharge them not
in the same line of Friendship. The Legal is upon specified conditions,
either purely tradesmanlike from hand to hand or somewhat more
gentlemanly as regards time but still by agreement a _quid pro quo_.

In this Legal kind the obligation is clear and admits of no dispute, the
friendly element is the delay in requiring its discharge: and for this
reason in some countries no actions can be maintained at Law for the
recovery of such debts, it being held that they who have dealt on the
footing of credit must be content to abide the issue.

That which may be termed the Moral kind is not upon specified
conditions, but a man gives as to his friend and so on: but still he
expects to receive an equivalent, or even more, as though he had not
given but lent: he also will find fault, because he does not get the
obligation discharged in the same way as it was contracted.

[Sidenote:1163a] Now this results from the fact, that all men, or the
generality at least, _wish_ what is honourable, but, when tested,
_choose_ what is profitable; and the doing kindnesses disinterestedly
is honourable while receiving benefits is profitable. In such cases one
should, if able, make a return proportionate to the good received, and
do so willingly, because one ought not to make a disinterested friend of
a man against his inclination: one should act, I say, as having made a
mistake originally in receiving kindness from one from whom one ought
not to have received it, he being not a friend nor doing the act
disinterestedly; one should therefore discharge one's self of the
obligation as having received a kindness on specified terms: and if able
a man would engage to repay the kindness, while if he were unable even
the doer of it would not expect it of him: so that if he is able he
ought to repay it. But one ought at the first to ascertain from whom
one is receiving kindness, and on what understanding, that on that same
understanding one may accept it or not.

A question admitting of dispute is whether one is to measure a kindness
by the good done to the receiver of it, and make this the standard by
which to requite, or by the kind intention of the doer?

For they who have received kindnesses frequently plead in depreciation
that they have received from their benefactors such things as were small
for them to give, or such as they themselves could have got from others:
while the doers of the kindnesses affirm that they gave the best they
had, and what could not have been got from others, and under danger, or
in such-like straits.

May we not say, that as utility is the motive of the Friendship the
advantage conferred on the receiver must be the standard? because he it
is who requests the kindness and the other serves him in his need on the
understanding that he is to get an equivalent: the assistance rendered
is then exactly proportionate to the advantage which the receiver has
obtained, and he should therefore repay as much as he gained by it, or
even more, this being more creditable.

In Friendships based on goodness, the question, of course, is never
raised, but herein the motive of the doer seems to be the proper
standard, since virtue and moral character depend principally on motive.


XIV

Quarrels arise also in those Friendships in which the parties are
unequal because each party thinks himself entitled to the greater share,
and of course, when this happens, the Friendship is broken up.

The man who is better than the other thinks that having the greater
share pertains to him of right, for that more is always awarded to the
good man: and similarly the man who is more profitable to another than
that other to him: "one who is useless," they say, "ought not to share
equally, for it comes to a tax, and not a Friendship, unless the fruits
of the Friendship are reaped in proportion to the works done:" their
notion being, that as in a money partnership they who contribute more
receive more so should it be in Friendship likewise.

On the other hand, the needy man and the less virtuous advance the
opposite claim: they urge that "it is the very business of a good friend
to help those who are in need, else what is the use of having a good or
powerful friend if one is not to reap the advantage at all?"

[Sidenote: 1163b] Now each seems to advance a right claim and to be
entitled to get more out of the connection than the other, only _not
more of the same thing_: but the superior man should receive more
respect, the needy man more profit: respect being the reward of goodness
and beneficence, profit being the aid of need.

This is plainly the principle acted upon in Political Communities:
he receives no honour who gives no good to the common stock: for the
property of the Public is given to him who does good to the Public, and
honour is the property of the Public; it is not possible both to make
money out of the Public and receive honour likewise; because no one will
put up with the less in every respect: so to him who suffers loss as
regards money they award honour, but money to him who can be paid by
gifts: since, as has been stated before, the observing due proportion
equalises and preserves Friendship.

Like rules then should be observed in the intercourse of friends who
are unequal; and to him who advantages another in respect of money, or
goodness, that other should repay honour, making requital according to
his power; because Friendship requires what is possible, not what is
strictly due, this being not possible in all cases, as in the honours
paid to the gods and to parents: no man could ever make the due return
in these cases, and so he is thought to be a good man who pays respect
according to his ability.

For this reason it may be judged never to be allowable for a son to
disown his father, whereas a father may his son: because he that owes
is bound to pay; now a son can never, by anything he has done, fully
requite the benefits first conferred on him by his father, and so is
always a debtor. But they to whom anything is owed may cast off their
debtors: therefore the father may his son. But at the same time it must
perhaps be admitted, that it seems no father ever _would_ sever himself
utterly from a son, except in a case of exceeding depravity: because,
independently of the natural Friendship, it is like human nature not to
put away from one's self the assistance which a son might render. But to
the son, if depraved, assisting his father is a thing to be avoided, or
at least one which he will not be very anxious to do; most men
being willing enough to receive kindness, but averse to doing it as
unprofitable.

Let thus much suffice on these points.




BOOK IX


I

[Sidenote: 1164a] Well, in all the Friendships the parties to which are
dissimilar it is the proportionate which equalises and preserves the
Friendship, as has been already stated: I mean, in the Social Friendship
the cobbler, for instance, gets an equivalent for his shoes after a
certain rate; and the weaver, and all others in like manner. Now in
this case a common measure has been provided in money, and to this
accordingly all things are referred and by this are measured: but in
the Friendship of Love the complaint is sometimes from the lover that,
though he loves exceedingly, his love is not requited; he having perhaps
all the time nothing that can be the object of Friendship: again,
oftentimes from the object of love that he who as a suitor promised any
and every thing now performs nothing. These cases occur because the
Friendship of the lover for the beloved object is based upon pleasure,
that of the other for him upon utility, and in one of the parties the
requisite quality is not found: for, as these are respectively the
grounds of the Friendship, the Friendship comes to be broken up because
the motives to it cease to exist: the parties loved not one another but
qualities in one another which are not permanent, and so neither are the
Friendships: whereas the Friendship based upon the moral character of
the parties, being independent and disinterested, is permanent, as we
have already stated.

Quarrels arise also when the parties realise different results and not
those which they desire; for the not attaining one's special object is
all one, in this case, with getting nothing at all: as in the well-known
case where a man made promises to a musician, rising in proportion to
the excellence of his music; but when, the next morning, the musician
claimed the performance of his promises, he said that he had given him
pleasure for pleasure: of course, if each party had intended this, it
would have been all right: but if the one desires amusement and the
other gain, and the one gets his object but the other not, the dealing
cannot be fair: because a man fixes his mind upon what he happens to
want, and will give so and so for that specific thing.

The question then arises, who is to fix the rate? the man who first
gives, or the man who first takes? because, _prima facie_, the man who
first gives seems to leave the rate to be fixed by the other party.
This, they say, was in fact the practice of Protagoras: when he taught
a man anything he would bid the learner estimate the worth of the
knowledge gained by his own private opinion; and then he used to take so
much from him. In such cases some people adopt the rule,

  "With specified reward a friend should be content."

They are certainly fairly found fault with who take the money in advance
and then do nothing of what they said they would do, their promises
having been so far beyond their ability; for such men do not perform
what they agreed, The Sophists, however, are perhaps obliged to take
this course, because no one would give a sixpence for their knowledge.
These then, I say, are fairly found fault with, because they do not what
they have already taken money for doing.

[Sidenote: 1164b] In cases where no stipulation as to the respective
services is made they who disinterestedly do the first service will not
raise the question (as we have said before), because it is the nature of
Friendship, based on mutual goodness to be reference to the intention of
the other, the intention being characteristic of the true friend and of
goodness.

And it would seem the same rule should be laid down for those who are
connected with one another as teachers and learners of philosophy; for
here the value of the commodity cannot be measured by money, and, in
fact, an exactly equivalent price cannot be set upon it, but perhaps it
is sufficient to do what one can, as in the case of the gods or one's
parents.

But where the original giving is not upon these terms but avowedly for
some return, the most proper course is perhaps for the requital to be
such as _both_ shall allow to be proportionate, and, where this cannot
be, then for the receiver to fix the value would seem to be not only
necessary but also fair: because when the first giver gets that which is
equivalent to the advantage received by the other, or to what he would
have given to secure the pleasure he has had, then he has the value from
him: for not only is this seen to be the course adopted in matters of
buying and selling but also in some places the law does not allow of
actions upon voluntary dealings; on the principle that when one man has
trusted another he must be content to have the obligation discharged in
the same spirit as he originally contracted it: that is to say, it is
thought fairer for the trusted, than for the trusting, party, to fix the
value. For, in general, those who have and those who wish to get things
do not set the same value on them: what is their own, and what they give
in each case, appears to them worth a great deal: but yet the return
is made according to the estimate of those who have received first, it
should perhaps be added that the receiver should estimate what he has
received, not by the value he sets upon it now that he has it, but by
that which he set upon it before he obtained it.


II

Questions also arise upon such points as the following: Whether one's
father has an unlimited claim on one's services and obedience, or
whether the sick man is to obey his physician? or, in an election of
a general, the warlike qualities of the candidates should be alone
regarded?

In like manner whether one should do a service rather to one's friend or
to a good man? whether one should rather requite a benefactor or give to
one's companion, supposing that both are not within one's power?

[Sidenote: 1165a] Is not the true answer that it is no easy task to
determine all such questions accurately, inasmuch as they involve
numerous differences of all kinds, in respect of amount and what is
honourable and what is necessary? It is obvious, of course, that no one
person can unite in himself all claims. Again, the requital of benefits
is, in general, a higher duty than doing unsolicited kindnesses to one's
companion; in other words, the discharging of a debt is more obligatory
upon one than the duty of giving to a companion. And yet this rule may
admit of exceptions; for instance, which is the higher duty? for one who
has been ransomed out of the hands of robbers to ransom in return his
ransomer, be he who he may, or to repay him on his demand though he has
not been taken by robbers, or to ransom his own father? for it would
seem that a man ought to ransom his father even in preference to
himself.

Well then, as has been said already, as a general rule the debt
should be discharged, but if in a particular case the giving greatly
preponderates as being either honourable or necessary, we must be swayed
by these considerations: I mean, in some cases the requital of the
obligation previously existing may not be equal; suppose, for instance,
that the original benefactor has conferred a kindness on a good man,
knowing him to be such, whereas this said good man has to repay it
believing him to be a scoundrel.

And again, in certain cases no obligation lies on a man to lend to one
who has lent to him; suppose, for instance, that a bad man lent to him,
as being a good man, under the notion that he should get repaid, whereas
the said good man has no hope of repayment from him being a bad man.
Either then the case is really as we have supposed it and then the claim
is not equal, or it is not so but supposed to be; and still in so acting
people are not to be thought to act wrongly. In short, as has been
oftentimes stated before, all statements regarding feelings and actions
can be definite only in proportion as their object-matter is so; it is
of course quite obvious that all people have not the same claim upon
one, nor are the claims of one's father unlimited; just as Jupiter does
not claim all kinds of sacrifice without distinction: and since the
claims of parents, brothers, companions, and benefactors, are all
different, we must give to each what belongs to and befits each.

And this is seen to be the course commonly pursued: to marriages men
commonly invite their relatives, because these are from a common stock
and therefore all the actions in any way pertaining thereto are common
also: and to funerals men think that relatives ought to assemble in
preference to other people, for the same reason.

And it would seem that in respect of maintenance it is our duty to
assist our parents in preference to all others, as being their debtors,
and because it is more honourable to succour in these respects the
authors of our existence than ourselves. Honour likewise we ought to pay
to our parents just as to the gods, but then, not all kinds of honour:
not the same, for instance, to a father as to a mother: nor again to a
father the honour due to a scientific man or to a general but that
which is a father's due, and in like manner to a mother that which is a
mother's.

To all our elders also the honour befitting their age, by rising up in
their presence, turning out of the way for them, and all similar marks
of respect: to our companions again, or brothers, frankness and free
participation in all we have. And to those of the same family, or tribe,
or city, with ourselves, and all similarly connected with us, we should
constantly try to render their due, and to discriminate what belongs to
each in respect of nearness of connection, or goodness, or intimacy:
of course in the case of those of the same class the discrimination is
easier; in that of those who are in different classes it is a matter of
more trouble. This, however, should not be a reason for giving up
the attempt, but we must observe the distinctions so far as it is
practicable to do so.

III

A question is also raised as to the propriety of dissolving or not
dissolving those Friendships the parties to which do not remain what
they were when the connection was formed.

[Sidenote: 1165b] Now surely in respect of those whose motive to
Friendship is utility or pleasure there can be nothing wrong in breaking
up the connection when they no longer have those qualities; because they
were friends [not of one another, but] of those qualities: and, these
having failed, it is only reasonable to expect that they should cease to
entertain the sentiment.

But a man has reason to find fault if the other party, being really
attached to him because of advantage or pleasure, pretended to be so
because of his moral character: in fact, as we said at the commencement,
the most common source of quarrels between friends is their not being
friends on the same grounds as they suppose themselves to be.

Now when a man has been deceived in having supposed himself to excite
the sentiment of Friendship by reason of his moral character, the other
party doing nothing to indicate he has but himself to blame: but when he
has been deceived by the pretence of the other he has a right to find
fault with the man who has so deceived him, aye even more than with
utterers of false coin, in proportion to the greater preciousness of
that which is the object-matter of the villany.

But suppose a man takes up another as being a good man, who turns out,
and is found by him, to be a scoundrel, is he bound still to entertain
Friendship for him? or may we not say at once it is impossible? since
it is not everything which is the object-matter of Friendship, but only
that which is good; and so there is no obligation to be a bad man's
friend, nor, in fact, ought one to be such: for one ought not to be a
lover of evil, nor to be assimilated to what is base; which would be
implied, because we have said before, like is friendly to like.

Are we then to break with him instantly? not in all cases; only where
our friends are incurably depraved; when there is a chance of amendment
we are bound to aid in repairing the moral character of our friends
even more than their substance, in proportion as it is better and
more closely related to Friendship. Still he who should break off the
connection is not to be judged to act wrongly, for he never was a friend
to such a character as the other now is, and therefore, since the man is
changed and he cannot reduce him to his original state, he backs out of
the connection.

To put another case: suppose that one party remains what he was when
the Friendship was formed, while the other becomes morally improved and
widely different from his friend in goodness; is the improved character
to treat the other as a friend?

May we not say it is impossible? The case of course is clearest where
there is a great difference, as in the Friendships of boys: for suppose
that of two boyish friends the one still continues a boy in mind and the
other becomes a man of the highest character, how can they be friends?
since they neither are pleased with the same objects nor like and
dislike the same things: for these points will not belong to them as
regards one another, and without them it was assumed they cannot be
friends because they cannot live in intimacy: and of the case of those
who cannot do so we have spoken before.

Well then, is the improved party to bear himself towards his former
friend in no way differently to what he would have done had the
connection never existed?

Surely he ought to bear in mind the intimacy of past times, and just as
we think ourselves bound to do favours for our friends in preference to
strangers, so to those who have been friends and are so no longer we
should allow somewhat on the score of previous Friendship, whenever the
cause of severance is not excessive depravity on their part.




IV

[Sidenote: II66a] Now the friendly feelings which are exhibited towards
our friends, and by which Friendships are characterised, seem to have
sprung out of those which we entertain toward ourselves. I mean, people
define a friend to be "one who intends and does what is good (or what
he believes to be good) to another for that other's sake," or "one who
wishes his friend to be and to live for that friend's own sake" (which
is the feeling of mothers towards their children, and of friends who
have come into collision). Others again, "one who lives with another and
chooses the same objects," or "one who sympathises with his friend in
his sorrows and in his joys" (this too is especially the case with
mothers).

Well, by some one of these marks people generally characterise
Friendship: and each of these the good man has towards himself, and all
others have them in so far as they suppose themselves to be good. (For,
as has been said before, goodness, that is the good man, seems to be a
measure to every one else.)

For he is at unity in himself, and with every part of his soul he
desires the same objects; and he wishes for himself both what is, and
what he believes to be, good; and he does it (it being characteristic
of the good man to work at what is good), and for the sake of himself,
inasmuch as he does it for the sake of his Intellectual Principle which
is generally thought to be a man's Self. Again, he wishes himself And
specially this Principle whereby he is an intelligent being, to live and
be preserved in life, because existence is a good to him that is a good
man.

But it is to himself that each individual wishes what is good, and no
man, conceiving the possibility of his becoming other than he now is,
chooses that that New Self should have all things indiscriminately: a
god, for instance, has at the present moment the Chief Good, but he has
it in right of being whatever he actually now is: and the Intelligent
Principle must be judged to be each man's Self, or at least eminently so
[though other Principles help, of course, to constitute him the man he
is]. Furthermore, the good man wishes to continue to live with himself;
for he can do it with pleasure, in that his memories of past actions are
full of delight and his anticipations of the future are good and such
are pleasurable. Then, again, he has good store of matter for his
Intellect to contemplate, and he most especially sympathises with his
Self in its griefs and joys, because the objects which give him pain and
pleasure are at all times the same, not one thing to-day and a different
one to-morrow: because he is not given to repentance, if one may so
speak. It is then because each of these feelings are entertained by the
good man towards his own Self and a friend feels towards a friend as
towards himself (a friend being in fact another Self), that Friendship
is thought to be some one of these things and they are accounted friends
in whom they are found. Whether or no there can really be Friendship
between a man and his Self is a question we will not at present
entertain: there may be thought to be Friendship, in so far as there are
two or more of the aforesaid requisites, and because the highest degree
of Friendship, in the usual acceptation of that term, resembles the
feeling entertained by a man towards himself.

[Sidenote: 1166b] But it may be urged that the aforesaid requisites are
to all appearance found in the common run of men, though they are men of
a low stamp.

May it not be answered, that they share in them only in so far as they
please themselves, and conceive themselves to be good? for certainly,
they are not either really, or even apparently, found in any one of
those who are very depraved and villainous; we may almost say not
even in those who are bad men at all: for they are at variance with
themselves and lust after different things from those which in cool
reason they wish for, just as men who fail of Self-Control: I mean, they
choose things which, though hurtful, are pleasurable, in preference to
those which in their own minds they believe to be good: others again,
from cowardice and indolence, decline to do what still they are
convinced is best for them: while they who from their depravity have
actually done many dreadful actions hate and avoid life, and accordingly
kill themselves: and the wicked seek others in whose company to spend
their time, but fly from themselves because they have many unpleasant
subjects of memory, and can only look forward to others like them when
in solitude but drown their remorse in the company of others: and as
they have nothing to raise the sentiment of Friendship so they never
feel it towards themselves.

Neither, in fact, can they who are of this character sympathise with
their Selves in their joys and sorrows, because their soul is, as it
were, rent by faction, and the one principle, by reason of the depravity
in them, is grieved at abstaining from certain things, while the other
and better principle is pleased thereat; and the one drags them this way
and the other that way, as though actually tearing them asunder. And
though it is impossible actually to have at the same time the sensations
of pain and pleasure; yet after a little time the man is sorry for
having been pleased, and he could wish that those objects had not given
him pleasure; for the wicked are full of remorse.

It is plain then that the wicked man cannot be in the position of a
friend even towards himself, because he has in himself nothing which can
excite the sentiment of Friendship. If then to be thus is exceedingly
wretched it is a man's duty to flee from wickedness with all his might
and to strive to be good, because thus may he be friends with himself
and may come to be a friend to another.

[Sidenote: V] Kindly Feeling, though resembling Friendship, is not
identical with it, because it may exist in reference to those whom we
do not know and without the object of it being aware of its existence,
which Friendship cannot. (This, by the way, has also been said before.)
And further, it is not even Affection because it does not imply
intensity nor yearning, which are both consequences of Affection. Again
Affection requires intimacy but Kindly Feeling may arise quite suddenly,
as happens sometimes in respect of men against whom people are matched
in any way, I mean they come to be kindly disposed to them and
sympathise in their wishes, but still they would not join them in any
action, because, as we said, they conceive this feeling of kindness
suddenly and so have but a superficial liking.

What it does seem to be is the starting point of a Friendship; just as
pleasure, received through the sight, is the commencement of Love: for
no one falls in love without being first pleased with the personal
appearance of the beloved object, and yet he who takes pleasure in it
does not therefore necessarily love, but when he wearies for the object
in its absence and desires its presence. Exactly in the same way men
cannot be friends without having passed through the stage of Kindly
Feeling, and yet they who are in that stage do not necessarily advance
to Friendship: they merely have an inert wish for the good of those
toward whom they entertain the feeling, but would not join them in
any action, nor put themselves out of the way for them. So that, in
a metaphorical way of speaking, one might say that it is dormant
Friendship, and when it has endured for a space and ripened into
intimacy comes to be real Friendship; but not that whose object is
advantage or pleasure, because such motives cannot produce even Kindly
Feeling.

I mean, he who has received a kindness requites it by Kindly Feeling
towards his benefactor, and is right in so doing: but he who wishes
another to be prosperous, because he has hope of advantage through his
instrumentality, does not seem to be kindly disposed to that person but
rather to himself; just as neither is he his friend if he pays court to
him for any interested purpose.

Kindly Feeling always arises by reason of goodness and a certain
amiability, when one man gives another the notion of being a fine
fellow, or brave man, etc., as we said was the case sometimes with those
matched against one another.

[Sidenote: VI] Unity of Sentiment is also plainly connected with
Friendship, and therefore is not the same as Unity of Opinion,
because this might exist even between people unacquainted with one
another.

Nor do men usually say people are united in sentiment merely because
they agree in opinion on _any_ point, as, for instance, on points
of astronomical science (Unity of Sentiment herein not having any
connection with Friendship), but they say that Communities have Unity of
Sentiment when they agree respecting points of expediency and take the
same line and carry out what has been determined in common consultation.

Thus we see that Unity of Sentiment has for its object matters of
action, and such of these as are of importance, and of mutual, or, in
the case of single States, common, interest: when, for instance, all
agree in the choice of magistrates, or forming alliance with the
Lacedæmonians, or appointing Pittacus ruler (that is to say, supposing
he himself was willing). [Sidenote: 1167_b_] But when each wishes
himself to be in power (as the brothers in the Phoenissæ), they quarrel
and form parties: for, plainly, Unity of Sentiment does not merely imply
that each entertains the same idea be it what it may, but that they do
so in respect of the same object, as when both the populace and the
sensible men of a State desire that the best men should be in office,
because then all attain their object.

Thus Unity of Sentiment is plainly a social Friendship, as it is also
said to be: since it has for its object-matter things expedient and
relating to life.

And this Unity exists among the good: for they have it towards
themselves and towards one another, being, if I may be allowed the
expression, in the same position: I mean, the wishes of such men are
steady and do not ebb and flow like the Euripus, and they wish what is
just and expedient and aim at these things in common.

The bad, on the contrary, can as little have Unity of Sentiment as they
can be real friends, except to a very slight extent, desiring as they
do unfair advantage in things profitable while they shirk labour and
service for the common good: and while each man wishes for these things
for himself he is jealous of and hinders his neighbour: and as they
do not watch over the common good it is lost. The result is that they
quarrel while they are for keeping one another to work but are not
willing to perform their just share.

[Sidenote: VII] Benefactors are commonly held to have more Friendship
for the objects of their kindness than these for them: and the fact
is made a subject of discussion and inquiry, as being contrary to
reasonable expectation.

The account of the matter which satisfies most persons is that the one
are debtors and the others creditors: and therefore that, as in the case
of actual loans the debtors wish their creditors out of the way while
the creditors are anxious for the preservation of their debtors, so
those who have done kindnesses desire the continued existence of the
people they have done them to, under the notion of getting a return
of their good offices, while these are not particularly anxious about
requital.

Epicharmus, I suspect, would very probably say that they who give this
solution judge from their own baseness; yet it certainly is like human
nature, for the generality of men have short memories on these points,
and aim rather at receiving than conferring benefits.

But the real cause, it would seem, rests upon nature, and the case is
not parallel to that of creditors; because in this there is no affection
to the persons, but merely a wish for their preservation with a view to
the return: whereas, in point of fact, they who have done kindnesses
feel friendship and love for those to whom they have done them, even
though they neither are, nor can by possibility hereafter be, in a
position to serve their benefactors.

[Sidenote: 1168_a_] And this is the case also with artisans; every one,
I mean, feels more affection for his own work than that work possibly
could for him if it were animate. It is perhaps specially the case with
poets: for these entertain very great affection for their poems, loving
them as their own children. It is to this kind of thing I should be
inclined to compare the case of benefactors: for the object of their
kindness is their own work, and so they love this more than this loves
its creator.

And the account of this is that existence is to all a thing choiceworthy
and an object of affection; now we exist by acts of working, that is, by
living and acting; he then that has created a given work exists, it may
be said, by his act of working: therefore he loves his work because he
loves existence. And this is natural, for the work produced displays in
act what existed before potentially.

Then again, the benefactor has a sense of honour in right of his action,
so that he may well take pleasure in him in whom this resides; but to
him who has received the benefit there is nothing honourable in respect
of his benefactor, only something advantageous which is both less
pleasant and less the object of Friendship.

Again, pleasure is derived from the actual working out of a present
action, from the anticipation of a future one, and from the recollection
of a past one: but the highest pleasure and special object of affection
is that which attends on the actual working. Now the benefactor's work
abides (for the honourable is enduring), but the advantage of him who
has received the kindness passes away.

Again, there is pleasure in recollecting honourable actions, but in
recollecting advantageous ones there is none at all or much less (by the
way though, the contrary is true of the expectation of advantage).

Further, the entertaining the feeling of Friendship is like acting on
another; but being the object of the feeling is like being acted upon.

So then, entertaining the sentiment of Friendship, and all feelings
connected with it, attend on those who, in the given case of a
benefaction, are the superior party.

Once more: all people value most what has cost them much labour in the
production; for instance, people who have themselves made their money
are fonder of it than those who have inherited it: and receiving
kindness is, it seems, unlaborious, but doing it is laborious. And this
is the reason why the female parents are most fond of their offspring;
for their part in producing them is attended with most labour, and they
know more certainly that they are theirs. This feeling would seem also
to belong to benefactors.

[Sidenote: VIII] A question is also raised as to whether it is right
to love one's Self best, or some one else: because men find fault with
those who love themselves best, and call them in a disparaging way
lovers of Self; and the bad man is thought to do everything he does
for his own sake merely, and the more so the more depraved he is;
accordingly men reproach him with never doing anything unselfish:
whereas the good man acts from a sense of honour (and the more so the
better man he is), and for his friend's sake, and is careless of his own
interest.

[Sidenote: 1168_b_] But with these theories facts are at variance, and
not unnaturally: for it is commonly said also that a man is to love most
him who is most his friend, and he is most a friend who wishes good to
him to whom he wishes it for that man's sake even though no one knows.
Now these conditions, and in fact all the rest by which a friend is
characterised, belong specially to each individual in respect of his
Self: for we have said before that all the friendly feelings are derived
to others from those which have Self primarily for their object. And all
the current proverbs support this view; for instance, "one soul," "the
goods of friends are common," "equality is a tie of Friendship," "the
knee is nearer than the shin." For all these things exist specially with
reference to a man's own Self: he is specially a friend to himself and
so he is bound to love himself the most.

It is with good reason questioned which of the two parties one should
follow, both having plausibility on their side. Perhaps then, in respect
of theories of this kind, the proper course is to distinguish and define
how far each is true, and in what way. If we could ascertain the sense
in which each uses the term "Self-loving," this point might be cleared
up.

Well now, they who use it disparagingly give the name to those who,
in respect of wealth, and honours, and pleasures of the body, give to
themselves the larger share: because the mass of mankind grasp after
these and are earnest about them as being the best things; which is the
reason why they are matters of contention. They who are covetous in
regard to these gratify their lusts and passions in general, that is to
say the irrational part of their soul: now the mass of mankind are so
disposed, for which reason the appellation has taken its rise from that
mass which is low and bad. Of course they are justly reproached who are
Self-loving in this sense.

And that the generality of men are accustomed to apply the term to
denominate those who do give such things to themselves is quite plain:
suppose, for instance, that a man were anxious to do, more than other
men, acts of justice, or self-mastery, or any other virtuous acts, and,
in general, were to secure to himself that which is abstractedly noble
and honourable, no one would call him Self-loving, nor blame him.

Yet might such an one be judged to be more truly Self-loving: certainly
he gives to himself the things which are most noble and most good,
and gratifies that Principle of his nature which is most rightfully
authoritative, and obeys it in everything: and just as that which
possesses the highest authority is thought to constitute a Community or
any other system, so also in the case of Man: and so he is most truly
Self-loving who loves and gratifies this Principle.

Again, men are said to have, or to fail of having, self-control,
according as the Intellect controls or not, it being plainly implied
thereby that this Principle constitutes each individual; and people are
thought to have done of themselves, and voluntarily, those things
specially which are done with Reason. [Sidenote: 1169_a_]

It is plain, therefore, that this Principle does, either entirely or
specially constitute the individual man, and that the good man specially
loves this. For this reason then he must be specially Self-loving, in a
kind other than that which is reproached, and as far superior to it as
living in accordance with Reason is to living at the beck and call of
passion, and aiming at the truly noble to aiming at apparent advantage.

Now all approve and commend those who are eminently earnest about
honourable actions, and if all would vie with one another in respect of
the [Greek: kalhon], and be intent upon doing what is most truly noble
and honourable, society at large would have all that is proper while
each individual in particular would have the greatest of goods, Virtue
being assumed to be such.

And so the good man ought to be Self-loving: because by doing what is
noble he will have advantage himself and will do good to others: but the
bad man ought not to be, because he will harm himself and his neighbours
by following low and evil passions. In the case of the bad man, what he
ought to do and what he does are at variance, but the good man does what
he ought to do, because all Intellect chooses what is best for itself
and the good man puts himself under the direction of Intellect.

Of the good man it is true likewise that he does many things for the
sake of his friends and his country, even to the extent of dying for
them, if need be: for money and honours, and, in short, all the good
things which others fight for, he will throw away while eager to secure
to himself the [Greek: kalhon]: he will prefer a brief and great joy
to a tame and enduring one, and to live nobly for one year rather than
ordinarily for many, and one great and noble action to many trifling
ones. And this is perhaps that which befals men who die for their
country and friends; they choose great glory for themselves: and they
will lavish their own money that their friends may receive more, for
hereby the friend gets the money but the man himself the [Greek:
kalhon]; so, in fact he gives to himself the greater good. It is the
same with honours and offices; all these things he will give up to his
friend, because this reflects honour and praise on himself: and so
with good reason is he esteemed a fine character since he chooses the
honourable before all things else. It is possible also to give up the
opportunities of action to a friend; and to have caused a friend's doing
a thing may be more noble than having done it one's self.

In short, in all praiseworthy things the good man does plainly give to
himself a larger share of the honourable. [Sidenote: 1169_b_] In this
sense it is right to be Self-loving, in the vulgar acceptation of the
term it is not.

[Sidenote: IX] A question is raised also respecting the Happy man,
whether he will want Friends, or no?

Some say that they who are blessed and independent have no need of
Friends, for they already have all that is good, and so, as being
independent, want nothing further: whereas the notion of a friend's
office is to be as it were a second Self and procure for a man what he
cannot get by himself: hence the saying,

  "When Fortune gives us good, what need we Friends?"

On the other hand, it looks absurd, while we are assigning to the Happy
man all other good things, not to give him Friends, which are, after
all, thought to be the greatest of external goods.

Again, if it is more characteristic of a friend to confer than to
receive kindnesses, and if to be beneficent belongs to the good man and
to the character of virtue, and if it is more noble to confer kindnesses
on friends than strangers, the good man will need objects for his
benefactions. And out of this last consideration springs a question
whether the need of Friends be greater in prosperity or adversity, since
the unfortunate man wants people to do him kindnesses and they who are
fortunate want objects for their kind acts.

Again, it is perhaps absurd to make our Happy man a solitary, because
no man would choose the possession of all goods in the world on the
condition of solitariness, man being a social animal and formed by
nature for living with others: of course the Happy man has this
qualification since he has all those things which are good by nature:
and it is obvious that the society of friends and good men must be
preferable to that of strangers and ordinary people, and we conclude,
therefore, that the Happy man does need Friends.

But then, what do they mean whom we quoted first, and how are they
right? Is it not that the mass of mankind mean by Friends those who are
useful? and of course the Happy man will not need such because he has
all good things already; neither will he need such as are Friends with
a view to the pleasurable, or at least only to a slight extent; because
his life, being already pleasurable, does not want pleasure imported
from without; and so, since the Happy man does not need Friends of these
kinds, he is thought not to need any at all.

But it may be, this is not true: for it was stated originally, that
Happiness is a kind of Working; now Working plainly is something
that must come into being, not be already there like a mere piece of
property.

[Sidenote: 1170_a_] If then the being happy consists in living and
working, and the good man's working is in itself excellent and
pleasurable (as we said at the commencement of the treatise), and if
what is our own reckons among things pleasurable, and if we can view our
neighbours better than ourselves and their actions better than we
can our own, then the actions of their Friends who are good men are
pleasurable to the good; inasmuch as they have both the requisites which
are naturally pleasant. So the man in the highest state of happiness
will need Friends of this kind, since he desires to contemplate good
actions, and actions of his own, which those of his friend, being a good
man, are. Again, common opinion requires that the Happy man live with
pleasure to himself: now life is burthensome to a man in solitude, for
it is not easy to work continuously by one's self, but in company with,
and in regard to others, it is easier, and therefore the working, being
pleasurable in itself will be more continuous (a thing which should be
in respect of the Happy man); for the good man, in that he is good takes
pleasure in the actions which accord with Virtue and is annoyed at those
which spring from Vice, just as a musical man is pleased with beautiful
music and annoyed by bad. And besides, as Theognis says, Virtue itself
may be improved by practice, from living with the good.

And, upon the following considerations more purely metaphysical, it will
probably appear that the good friend is naturally choiceworthy to the
good man. We have said before, that whatever is naturally good is also
in itself good and pleasant to the good man; now the fact of living, so
far as animals are concerned, is characterised generally by the power
of sentience, in man it is characterised by that of sentience, or
of rationality (the faculty of course being referred to the actual
operation of the faculty, certainly the main point is the actual
operation of it); so that living seems mainly to consist in the act of
sentience or exerting rationality: now the fact of living is in itself
one of the things that are good and pleasant (for it is a definite
totality, and whatever is such belongs to the nature of good), but what
is naturally good is good to the good man: for which reason it seems
to be pleasant to all. (Of course one must not suppose a life which is
depraved and corrupted, nor one spent in pain, for that which is such is
indefinite as are its inherent qualities: however, what is to be said of
pain will be clearer in what is to follow.)

If then the fact of living is in itself good and pleasant (and this
appears from the fact that all desire it, and specially those who are
good and in high happiness; their course of life being most choiceworthy
and their existence most choiceworthy likewise), then also he that sees
perceives that he sees; and he that hears perceives that he hears; and
he that walks perceives that he walks; and in all the other instances
in like manner there is a faculty which reflects upon and perceives the
fact that we are working, so that we can perceive that we perceive and
intellectually know that we intellectually know: but to perceive that we
perceive or that we intellectually know is to perceive that we exist,
since existence was defined to be perceiving or intellectually knowing.
[Sidenote: 1170_b_ Now to perceive that one lives is a thing pleasant
in itself, life being a thing naturally good, and the perceiving of the
presence in ourselves of things naturally good being pleasant.]

Therefore the fact of living is choiceworthy, and to the good specially
so since existence is good and pleasant to them: for they receive
pleasure from the internal consciousness of that which in itself is
good.

But the good man is to his friend as to himself, friend being but a name
for a second Self; therefore as his own existence is choiceworthy to
each so too, or similarly at least, is his friend's existence. But the
ground of one's own existence being choiceworthy is the perceiving of
one's self being good, any such perception being in itself pleasant.
Therefore one ought to be thoroughly conscious of one's friend's
existence, which will result from living with him, that is sharing in
his words and thoughts: for this is the meaning of the term as applied
to the human species, not mere feeding together as in the case of
brutes.

If then to the man in a high state of happiness existence is in itself
choiceworthy, being naturally good and pleasant, and so too a friend's
existence, then the friend also must be among things choiceworthy. But
whatever is choiceworthy to a man he should have or else he will be in
this point deficient. The man therefore who is to come up to our notion
"Happy" will need good Friends. Are we then to make our friends as
numerous as possible? or, as in respect of acquaintance it is thought
to have been well said "have not thou many acquaintances yet be not
without;" so too in respect of Friendship may we adopt the precept, and
say that a man should not be without friends, nor again have exceeding
many friends?

Now as for friends who are intended for use, the maxim I have quoted
will, it seems, fit in exceedingly well, because to requite the services
of many is a matter of labour, and a whole life would not be long enough
to do this for them. So that, if more numerous than what will suffice
for one's own life, they become officious, and are hindrances in respect
of living well: and so we do not want them. And again of those who are
to be for pleasure a few are quite enough, just like sweetening in our
food.




X


But of the good are we to make as many as ever we can, or is there
any measure of the number of friends, as there is of the number to
constitute a Political Community? I mean, you cannot make one out of ten
men, and if you increase the number to one hundred thousand it is not
any longer a Community. However, the number is not perhaps some one
definite number but any between certain extreme limits.

[Sidenote: 1171_a_] Well, of friends likewise there is a limited number,
which perhaps may be laid down to be the greatest number with whom it
would be possible to keep up intimacy; this being thought to be one of
the greatest marks of Friendship, and it being quite obvious that it is
not possible to be intimate with many, in other words, to part one's
self among many. And besides it must be remembered that they also are to
be friends to one another if they are all to live together: but it is a
matter of difficulty to find this in many men at once.

It comes likewise to be difficult to bring home to one's self the joys
and sorrows of many: because in all probability one would have to
sympathise at the same time with the joys of this one and the sorrows of
that other.

Perhaps then it is well not to endeavour to have very many friends but
so many as are enough for intimacy: because, in fact, it would seem not
to be possible to be very much a friend to many at the same time: and,
for the same reason, not to be in love with many objects at the same
time: love being a kind of excessive Friendship which implies but one
object: and all strong emotions must be limited in the number towards
whom they are felt.

And if we look to facts this seems to be so: for not many at a time
become friends in the way of companionship, all the famous Friendships
of the kind are between _two_ persons: whereas they who have many
friends, and meet everybody on the footing of intimacy, seem to be
friends really to no one except in the way of general society; I mean
the characters denominated as over-complaisant.

To be sure, in the way merely of society, a man may be a friend to many
without being necessarily over-complaisant, but being truly good: but
one cannot be a friend to many because of their virtue, and for the
persons' own sake; in fact, it is a matter for contentment to find even
a few such.


XI

Again: are friends most needed in prosperity or in adversity? they are
required, we know, in both states, because the unfortunate need help and
the prosperous want people to live with and to do kindnesses to: for
they have a desire to act kindly to some one.

To have friends is more necessary in adversity, and therefore in this
case useful ones are wanted; and to have them in prosperity is more
honourable, and this is why the prosperous want good men for friends, it
being preferable to confer benefits on, and to live with, these. For the
very presence of friends is pleasant even in adversity: since men when
grieved are comforted by the sympathy of their friends.

And from this, by the way, the question might be raised, whether it is
that they do in a manner take part of the weight of calamities, or only
that their presence, being pleasurable, and the consciousness of their
sympathy, make the pain of the sufferer less. However, we will not
further discuss whether these which have been suggested or some other
causes produce the relief, at least the effect we speak of is a matter
of plain fact.

[Sidenote: _1171b_] But their presence has probably a mixed effect: I
mean, not only is the very seeing friends pleasant, especially to one in
misfortune, and actual help towards lessening the grief is afforded
(the natural tendency of a friend, if he is gifted with tact, being
to comfort by look and word, because he is well acquainted with the
sufferer's temper and disposition and therefore knows what things give
him pleasure and pain), but also the perceiving a friend to be grieved
at his misfortunes causes the sufferer pain, because every one avoids
being cause of pain to his friends. And for this reason they who are
of a manly nature are cautious not to implicate their friends in their
pain; and unless a man is exceedingly callous to the pain of others he
cannot bear the pain which is thus caused to his friends: in short, he
does not admit men to wail with him, not being given to wail at all:
women, it is true, and men who resemble women, like to have others to
groan with them, and love such as friends and sympathisers. But it
is plain that it is our duty in all things to imitate the highest
character.

On the other hand, the advantages of friends in our prosperity are the
pleasurable intercourse and the consciousness that they are pleased at
our good fortune.

It would seem, therefore, that we ought to call in friends readily on
occasion of good fortune, because it is noble to be ready to do good to
others: but on occasion of bad fortune, we should do so with reluctance;
for we should as little as possible make others share in our ills; on
which principle goes the saying, "I am unfortunate, let that suffice."
The most proper occasion for calling them in is when with small trouble
or annoyance to themselves they can be of very great use to the person
who needs them.

But, on the contrary, it is fitting perhaps to go to one's friends in
their misfortunes unasked and with alacrity (because kindness is the
friend's office and specially towards those who are in need and who do
not demand it as a right, this being more creditable and more pleasant
to both); and on occasion of their good fortune to go readily, if we
can forward it in any way (because men need their friends for this
likewise), but to be backward in sharing it, any great eagerness to
receive advantage not being creditable.

One should perhaps be cautious not to present the appearance of
sullenness in declining the sympathy or help of friends, for this
happens occasionally.

It appears then that the presence of friends is, under all
circumstances, choiceworthy.

May we not say then that, as seeing the beloved object is most prized by
lovers and they choose this sense rather than any of the others because
Love

  "Is engendered in the eyes,
  With gazing fed,"

in like manner intimacy is to friends most choiceworthy, Friendship
being communion? Again, as a man is to himself so is he to his friend;
now with respect to himself the perception of his own existence is
choiceworthy, therefore is it also in respect of his friend.

And besides, their Friendship is acted out in intimacy, and so with good
reason they desire this. And whatever in each man's opinion constitutes
existence, or whatsoever it is for the sake of which they choose life,
herein they wish their friends to join with them; and so some men drink
together, others gamble, others join in gymnastic exercises or hunting,
others study philosophy together: in each case spending their days
together in that which they like best of all things in life, for since
they wish to be intimate with their friends they do and partake in those
things whereby they think to attain this object.

Therefore the Friendship of the wicked comes to be depraved; for, being
unstable, they share in what is bad and become depraved in being made
like to one another: but the Friendship of the good is good, growing
with their intercourse; they improve also, as it seems, by repeated
acts, and by mutual correction, for they receive impress from one
another in the points which give them pleasure; whence says the poet,

  "Thou from the good, good things shalt surely learn."

Here then we will terminate our discourse of Friendship. The next thing
is to go into the subject of Pleasure.




BOOK X


Next, it would seem, follows a discussion respecting Pleasure, for it is
thought to be most closely bound up with our kind: and so men train the
young, guiding them on their course by the rudders of Pleasure and Pain.
And to like and dislike what one ought is judged to be most important
for the formation of good moral character: because these feelings extend
all one's life through, giving a bias towards and exerting an influence
on the side of Virtue and Happiness, since men choose what is pleasant
and avoid what is painful.

Subjects such as these then, it would seem, we ought by no means to pass
by, and specially since they involve much difference of opinion. There
are those who call Pleasure the Chief Good; there are others who on the
contrary maintain that it is exceedingly bad; some perhaps from a real
conviction that such is the case, others from a notion that it is
better, in reference to our life and conduct, to show up Pleasure as
bad, even if it is not so really; arguing that, as the mass of men have
a bias towards it and are the slaves of their pleasures, it is right to
draw them to the contrary, for that so they may possibly arrive at the
mean.

I confess I suspect the soundness of this policy; in matters respecting
men's feelings and actions theories are less convincing than facts:
whenever, therefore, they are found conflicting with actual experience,
they not only are despised but involve the truth in their fall: he, for
instance, who deprecates Pleasure, if once seen to aim at it, gets the
credit of backsliding to it as being universally such as he said it was,
the mass of men being incapable of nice distinctions.

Real accounts, therefore, of such matters seem to be most expedient, not
with a view to knowledge merely but to life and conduct: for they are
believed as being in harm with facts, and so they prevail with the wise
to live in accordance with them.

But of such considerations enough: let us now proceed to the current
maxims respecting Pleasure.

II Now Eudoxus thought Pleasure to be the Chief Good because he saw all,
rational and irrational alike, aiming at it: and he argued that, since
in all what was the object of choice must be good and what most so the
best, the fact of all being drawn to the same thing proved this thing to
be the best for all: "For each," he said, "finds what is good for itself
just as it does its proper nourishment, and so that which is good for
all, and the object of the aim of all, is their Chief Good."

(And his theories were received, not so much for their own sake, as
because of his excellent moral character; for he was thought to be
eminently possessed of perfect self-mastery, and therefore it was not
thought that he said these things because he was a lover of Pleasure but
that he really was so convinced.)

And he thought his position was not less proved by the argument from the
contrary: that is, since Pain was in itself an object of avoidance to
all the contrary must be in like manner an object of choice.

Again he urged that that is most choiceworthy which we choose, not by
reason of, or with a view to, anything further; and that Pleasure is
confessedly of this kind because no one ever goes on to ask to what
purpose he is pleased, feeling that Pleasure is in itself choiceworthy.

Again, that when added to any other good it makes it more choiceworthy;
as, for instance, to actions of justice, or perfected self-mastery; and
good can only be increased by itself.

However, this argument at least seems to prove only that it belongs to
the class of goods, and not that it does so more than anything else: for
every good is more choicewortby in combination with some other than when
taken quite alone. In fact, it is by just such an argument that Plato
proves that Pleasure is not the Chief Good: "For," says he, "the life of
Pleasure is more choiceworthy in combination with Practical Wisdom than
apart from it; but, if the compound better then simple Pleasure cannot
be the Chief Good; because the very Chief Good cannot by any addition
become choiceworthy than it is already:" and it is obvious that nothing
else can be the Chief Good, which by combination with any of the things
in themselves good comes to be more choiceworthy.

What is there then of such a nature? (meaning, of course, whereof we can
partake; because that which we are in search of must be such).

As for those who object that "what all aim at is not necessarily good,"
I confess I cannot see much in what they say, because what all _think_
we say _is_. And he who would cut away this ground from under us will
not bring forward things more dependable: because if the argument had
rested on the desires of irrational creatures there might have been
something in what he says, but, since the rational also desire Pleasure,
how can his objection be allowed any weight? and it may be that, even in
the lower animals, there is some natural good principle above themselves
which aims at the good peculiar to them.

Nor does that seem to be sound which is urged respecting the argument
from the contrary: I mean, some people say "it does not follow that
Pleasure must be good because Pain is evil, since evil may be opposed to
evil, and both evil and good to what is indifferent:" now what they say
is right enough in itself but does not hold in the present instance.
If both Pleasure and Pain were bad both would have been objects of
avoidance; or if neither then neither would have been, at all events
they must have fared alike: but now men do plainly avoid the one as bad
and choose the other as good, and so there is a complete opposition. III
Nor again is Pleasure therefore excluded from being good because it
does not belong to the class of qualities: the acts of virtue are not
qualities, neither is Happiness [yet surely both are goods].

Again, they say the Chief Good is limited but Pleasure unlimited, in
that it admits of degrees.

Now if they judge this from the act of feeling Pleasure then the same
thing will apply to justice and all the other virtues, in respect of
which clearly it is said that men are more or less of such and such
characters (according to the different virtues), they are more just or
more brave, or one may practise justice and self-mastery more or less.

If, on the other hand, they judge in respect of the Pleasures themselves
then it may be they miss the true cause, namely that some are unmixed
and others mixed: for just as health being in itself limited, admits of
degrees, why should not Pleasure do so and yet be limited? in the former
case we account for it by the fact that there is not the same adjustment
of parts in all men, nor one and the same always in the same individual:
but health, though relaxed, remains up to a certain point, and differs
in degrees; and of course the same may be the case with Pleasure.

Again, assuming the Chief Good to be perfect and all Movements and
Generations imperfect, they try to shew that Pleasure is a Movement and
a Generation.

Yet they do not seem warranted in saying even that it is a Movement: for
to every Movement are thought to belong swiftness and slowness, and
if not in itself, as to that of the universe, yet relatively: but to
Pleasure neither of these belongs: for though one may have got quickly
into the state Pleasure, as into that of anger, one cannot be in the
state quickly, nor relatively to the state of any other person; but we
can walk or grow, and so on, quickly or slowly.

Of course it is possible to change into the state of Pleasure quickly or
slowly, but to act in the state (by which, I mean, have the perception
of Pleasure) quickly, is not possible. And how can it be a Generation?
because, according to notions generally held, not _any_thing is
generated from _any_thing, but a thing resolves itself into that out
of which it was generated: whereas of that of which Pleasure is a
Generation Pain is a Destruction.

Again, they say that Pain is a lack of something suitable to nature and
Pleasure a supply of it.

But these are affections of the body: now if Pleasure really is a
supplying of somewhat suitable to nature, that must feel the Pleasure in
which the supply takes place, therefore the body of course: yet this
is not thought to be so: neither then is Pleasure a supplying, only a
person of course will be pleased when a supply takes place just as he
will be pained when he is cut.

This notion would seem to have arisen out of the Pains and Pleasures
connected with natural nourishment; because, when people have felt a
lack and so have had Pain first, they, of course, are pleased with the
supply of their lack.

But this is not the case with all Pleasures: those attendant on
mathematical studies, for instance, are unconnected with any Pain; and
of such as attend on the senses those which arise through the sense of
Smell; and again, many sounds, and sights, and memories, and hopes: now
of what can these be Generations? because there has been here no lack of
anything to be afterwards supplied.

And to those who bring forward disgraceful Pleasures we may reply that
these are not really pleasant things; for it does not follow because
they are pleasant to the ill-disposed that we are to admit that they are
pleasant except to them; just as we should not say that those things
are really wholesome, or sweet, or bitter, which are so to the sick,
or those objects really white which give that impression to people
labouring under ophthalmia.

Or we might say thus, that the Pleasures are choiceworthy but not as
derived from these sources: just as wealth is, but not as the price of
treason; or health, but not on the terms of eating anything however
loathsome. Or again, may we not say that Pleasures differ in kind? those
derived from honourable objects, for instance are different from those
arising from disgraceful ones; and it is not possible to experience
the Pleasure of the just man without being just, or of the musical man
without being musical; and so on of others.

The distinction commonly drawn between the friend and the flatterer
would seem to show clearly either that Pleasure is not a good, or that
there are different kinds of Pleasure: for the former is thought to have
good as the object of his intercourse, the latter Pleasure only; and
this last is reproached, but the former men praise as having different
objects in his intercourse.

[Sidenote: 1174a]

Again, no one would choose to live with a child's intellect all his
life through, though receiving the highest possible Pleasure from such
objects as children receive it from; or to take Pleasure in doing any of
the most disgraceful things, though sure never to be pained.

There are many things also about which we should be diligent even though
they brought no Pleasure; as seeing, remembering, knowing, possessing
the various Excellences; and the fact that Pleasures do follow on these
naturally makes no difference, because we should certainly choose them
even though no Pleasure resulted from them.

It seems then to be plain that Pleasure is not the Chief Good, nor is
every kind of it choiceworthy: and that there are some choiceworthy in
themselves, differing in kind, _i.e._ in the sources from which they
are derived. Let this then suffice by way of an account of the current
maxims respecting Pleasure and Pain.

[Sidenote: IV]

Now what it is, and how characterised, will be more plain if we take up
the subject afresh.

An act of Sight is thought to be complete at any moment; that is to say,
it lacks nothing the accession of which subsequently will complete its
whole nature.

Well, Pleasure resembles this: because it is a whole, as one may say;
and one could not at any moment of time take a Pleasure whose whole
nature would be completed by its lasting for a longer time. And for this
reason it is not a Movement: for all Movement takes place in time of
certain duration and has a certain End to accomplish; for instance, the
Movement of house-building is then only complete when the builder has
produced what he intended, that is, either in the whole time [necessary
to complete the whole design], or in a given portion. But all the
subordinate Movements are incomplete in the parts of the time, and are
different in kind from the whole movement and from one another (I
mean, for instance, that the fitting the stones together is a Movement
different from that of fluting the column, and both again from the
construction of the Temple as a whole: but this last is complete as
lacking nothing to the result proposed; whereas that of the basement,
or of the triglyph, is incomplete, because each is a Movement of a part
merely).

As I said then, they differ in kind, and you cannot at any time you
choose find a Movement complete in its whole nature, but, if at all, in
the whole time requisite.

[Sidenote: 1174_b_]

And so it is with the Movement of walking and all others: for, if motion
be a Movement from one place to another place, then of it too there are
different kinds, flying, walking, leaping, and such-like. And not only
so, but there are different kinds even in walking: the where-from and
where-to are not the same in the whole Course as in a portion of it;
nor in one portion as in another; nor is crossing this line the same as
crossing that: because a man is not merely crossing a line but a line in
a given place, and this is in a different place from that.

Of Movement I have discoursed exactly in another treatise. I will now
therefore only say that it seems not to be complete at any given moment;
and that most movements are incomplete and specifically different, since
the whence and whither constitute different species.

But of Pleasure the whole nature is complete at any given moment: it
is plain then that Pleasure and Movement must be different from one
another, and that Pleasure belongs to the class of things whole and
complete. And this might appear also from the impossibility of moving
except in a definite time, whereas there is none with respect to the
sensation of Pleasure, for what exists at the very present moment is a
kind of "whole."

From these considerations then it is plain that people are not warranted
in saying that Pleasure is a Movement or a Generation: because these
terms are not applicable to all things, only to such as are divisible
and not "wholes:" I mean that of an act of Sight there is no Generation,
nor is there of a point, nor of a monad, nor is any one of these a
Movement or a Generation: neither then of Pleasure is there Movement or
Generation, because it is, as one may say, "a whole."

Now since every Percipient Faculty works upon the Object answering to
it, and perfectly the Faculty in a good state upon the most excellent of
the Objects within its range (for Perfect Working is thought to be much
what I have described; and we will not raise any question about saying
"the Faculty" works, instead of, "that subject wherein the Faculty
resides"), in each case the best Working is that of the Faculty in its
best state upon the best of the Objects answering to it. And this will
be, further, most perfect and most pleasant: for Pleasure is attendant
upon every Percipient Faculty, and in like manner on every intellectual
operation and speculation; and that is most pleasant which is most
perfect, and that most perfect which is the Working of the best Faculty
upon the most excellent of the Objects within its range.

And Pleasure perfects the Working. But Pleasure does not perfect it in
the same way as the Faculty and Object of Perception do, being good;
just as health and the physician are not in similar senses causes of a
healthy state.

And that Pleasure does arise upon the exercise of every Percipient
Faculty is evident, for we commonly say that sights and sounds are
pleasant; it is plain also that this is especially the case when the
Faculty is most excellent and works upon a similar Object: and when both
the Object and Faculty of Perception are such, Pleasure will always
exist, supposing of course an agent and a patient.

[Sidenote: 1175_a_]

Furthermore, Pleasure perfects the act of Working not in the way of an
inherent state but as a supervening finish, such as is bloom in people
at their prime. Therefore so long as the Object of intellectual or
sensitive Perception is such as it should be and also the Faculty which
discerns or realises the Object, there will be Pleasure in the Working:
because when that which has the capacity of being acted on and that
which is apt to act are alike and similarly related, the same result
follows naturally.

How is it then that no one feels Pleasure continuously? is it not that
he wearies, because all human faculties are incapable of unintermitting
exertion; and so, of course, Pleasure does not arise either, because
that follows upon the act of Working. But there are some things which
please when new, but afterwards not in the like way, for exactly the
same reason: that at first the mind is roused and works on these Objects
with its powers at full tension; just as they who are gazing stedfastly
at anything; but afterwards the act of Working is not of the kind it was
at first, but careless, and so the Pleasure too is dulled.

Again, a person may conclude that all men grasp at Pleasure, because all
aim likewise at Life and Life is an act of Working, and every man works
at and with those things which also he best likes; the musical man, for
instance, works with his hearing at music; the studious man with his
intellect at speculative questions, and so forth. And Pleasure perfects
the acts of Working, and so Life after which men grasp. No wonder then
that they aim also at Pleasure, because to each it perfects Life, which
is itself choiceworthy. (We will take leave to omit the question whether
we choose Life for Pleasure's sake of Pleasure for Life's sake; because
these two plainly are closely connected and admit not of separation;
since Pleasure comes not into being without Working, and again, every
Working Pleasure perfects.)

And this is one reason why Pleasures are thought to differ in kind,
because we suppose that things which differ in kind must be perfected by
things so differing: it plainly being the case with the productions of
Nature and Art; as animals, and trees, and pictures, and statues, and
houses, and furniture; and so we suppose that in like manner acts of
Working which are different in kind are perfected by things differing in
kind. Now Intellectual Workings differ specifically from those of the
Senses, and these last from one another; therefore so do the Pleasures
which perfect them.

This may be shown also from the intimate connection subsisting between
each Pleasure and the Working which it perfects: I mean, that the
Pleasure proper to any Working increases that Working; for they who
work with Pleasure sift all things more closely and carry them out to a
greater degree of nicety; for instance, those men become geometricians
who take Pleasure in geometry, and they apprehend particular points more
completely: in like manner men who are fond of music, or architecture,
or anything else, improve each on his own pursuit, because they feel
Pleasure in them. Thus the Pleasures aid in increasing the Workings, and
things which do so aid are proper and peculiar: but the things which are
proper and peculiar to others specifically different are themselves also
specifically different.

Yet even more clearly may this be shown from the fact that the Pleasures
arising from one kind of Workings hinder other Workings; for instance,
people who are fond of flute-music cannot keep their attention to
conversation or discourse when they catch the sound of a flute; because
they take more Pleasure in flute-playing than in the Working they are
at the time engaged on; in other words, the Pleasure attendant on
flute-playing destroys the Working of conversation or discourse. Much
the same kind of thing takes place in other cases, when a person is
engaged in two different Workings at the same time: that is, the
pleasanter of the two keeps pushing out the other, and, if the disparity
in pleasantness be great, then more and more till a man even ceases
altogether to work at the other.

This is the reason why, when we are very much pleased with anything
whatever, we do nothing else, and it is only when we are but moderately
pleased with one occupation that we vary it with another: people,
for instance, who eat sweetmeats in the theatre do so most when the
performance is indifferent.

Since then the proper and peculiar Pleasure gives accuracy to the
Workings and makes them more enduring and better of their kind, while
those Pleasures which are foreign to them mar them, it is plain there
is a wide difference between them: in fact, Pleasures foreign to any
Working have pretty much the same effect as the Pains proper to it,
which, in fact, destroy the Workings; I mean, if one man dislikes
writing, or another calculation, the one does not write, the other does
not calculate; because, in each case, the Working is attended with some
Pain: so then contrary effects are produced upon the Workings by the
Pleasures and Pains proper to them, by which I mean those which arise
upon the Working, in itself, independently of any other circumstances.
As for the Pleasures foreign to a Working, we have said already that
they produce a similar effect to the Pain proper to it; that is they
destroy the Working, only not in like way.

Well then, as Workings differ from one another in goodness and badness,
some being fit objects of choice, others of avoidance, and others in
their nature indifferent, Pleasures are similarly related; since its own
proper Pleasure attends or each Working: of course that proper to a good
Working is good, that proper to a bad, bad: for even the desires for
what is noble are praiseworthy, and for what is base blameworthy.

Furthermore, the Pleasures attendant on Workings are more closely
connected with them even than the desires after them: for these last
are separate both in time and nature, but the former are close to the
Workings, and so indivisible from them as to raise a question whether
the Working and the Pleasure are identical; but Pleasure does not seem
to be an Intellectual Operation nor a Faculty of Perception, because
that is absurd; but yet it gives some the impression of being the same
from not being separated from these.

As then the Workings are different so are their Pleasures; now Sight
differs from Touch in purity, and Hearing and Smelling from Taste;
therefore, in like manner, do their Pleasures; and again, Intellectual
Pleasures from these Sensual, and the different kinds both of
Intellectual and Sensual from one another.

It is thought, moreover, that each animal has a Pleasure proper to
itself, as it has a proper Work; that Pleasure of course which is
attendant on the Working. And the soundness of this will appear upon
particular inspection: for horse, dog, and man have different Pleasures;
as Heraclitus says, an ass would sooner have hay than gold; in other
words, provender is pleasanter to asses than gold. So then the Pleasures
of animals specifically different are also specifically different, but
those of the same, we may reasonably suppose, are without difference.

Yet in the case of human creatures they differ not a little: for the
very same things please some and pain others: and what are painful and
hateful to some are pleasant to and liked by others. The same is the
case with sweet things: the same will not seem so to the man in a fever
as to him who is in health: nor will the invalid and the person in
robust health have the same notion of warmth. The same is the case with
other things also.

Now in all such cases that is held to _be_ which impresses the good man
with the notion of being such and such; and if this is a second maxim
(as it is usually held to be), and Virtue, that is, the Good man, in
that he is such, is the measure of everything, then those must be real
Pleasures which gave him the impression of being so and those things
pleasant in which he takes Pleasure. Nor is it at all astonishing that
what are to him unpleasant should give another person the impression of
being pleasant, for men are liable to many corruptions and marrings; and
the things in question are not pleasant really, only to these particular
persons, and to them only as being thus disposed.

Well of course, you may say, it is obvious that we must assert those
which are confessedly disgraceful to be real Pleasures, except to
depraved tastes: but of those which are thought to be good what kind,
or which, must we say is _The Pleasure of Man?_ is not the answer plain
from considering the Workings, because the Pleasures follow upon these?

Whether then there be one or several Workings which belong to the
perfect and blessed man, the Pleasures which perfect these Workings must
be said to be specially and properly _The Pleasures of Man;_ and all
the rest in a secondary sense, and in various degrees according as the
Workings are related to those highest and best ones.


VI

Now that we have spoken about the Excellences of both kinds, and
Friendship in its varieties, and Pleasures, it remains to sketch out
Happiness, since we assume that to be the one End of all human things:
and we shall save time and trouble by recapitulating what was stated
before.

[Sidenote: 1176b] Well then, we said that it is not a State merely;
because, if it were, it might belong to one who slept all his life
through and merely vegetated, or to one who fell into very great
calamities: and so, if these possibilities displease us and we would
rather put it into the rank of some kind of Working (as was also said
before), and Workings are of different kinds (some being necessary
and choiceworthy with a view to other things, while others are so in
themselves), it is plain we must rank Happiness among those choiceworthy
for their own sakes and not among those which are so with a view to
something further: because Happiness has no lack of anything but is
self-sufficient.

By choiceworthy in themselves are meant those from which nothing is
sought beyond the act of Working: and of this kind are thought to be the
actions according to Virtue, because doing what is noble and excellent
is one of those things which are choiceworthy for their own sake alone.

And again, such amusements as are pleasant; because people do not choose
them with any further purpose: in fact they receive more harm than
profit from them, neglecting their persons and their property. Still the
common run of those who are judged happy take refuge in such pastimes,
which is the reason why they who have varied talent in such are highly
esteemed among despots; because they make themselves pleasant in those
things which these aim at, and these accordingly want such men.

Now these things are thought to be appurtenances of Happiness because
men in power spend their leisure herein: yet, it may be, we cannot
argue from the example of such men: because there is neither Virtue nor
Intellect necessarily involved in having power, and yet these are the
only sources of good Workings: nor does it follow that because these
men, never having tasted pure and generous Pleasure, take refuge in
bodily ones, we are therefore to believe them to be more choiceworthy:
for children too believe that those things are most excellent which are
precious in their eyes.

We may well believe that as children and men have different ideas as to
what is precious so too have the bad and the good: therefore, as we have
many times said, those things are really precious and pleasant which
seem so to the good man: and as to each individual that Working is most
choiceworthy which is in accordance with his own state to the good man
that is so which is in accordance with Virtue.

Happiness then stands not in amusement; in fact the very notion is
absurd of the End being amusement, and of one's toiling and enduring
hardness all one's life long with a view to amusement: for everything in
the world, so to speak, we choose with some further End in view, except
Happiness, for that is the End comprehending all others. Now to take
pains and to labour with a view to amusement is plainly foolish and
very childish: but to amuse one's self with a view to steady employment
afterwards, as Anacharsis says, is thought to be right: for amusement is
like rest, and men want rest because unable to labour continuously.

Rest, therefore, is not an End, because it is adopted with a view to
Working afterwards.

[Sidenote: 1177a] Again, it is held that the Happy Life must be one in
the way of Excellence, and this is accompanied by earnestness and stands
not in amusement. Moreover those things which are done in earnest, we
say, are better than things merely ludicrous and joined with amusement:
and we say that the Working of the better part, or the better man, is
more earnest; and the Working of the better is at once better and more
capable of Happiness.

Then, again, as for bodily Pleasures, any ordinary person, or even
a slave, might enjoy them, just as well as the best man living but
Happiness no one supposes a slave to share except so far as it is
implied in life: because Happiness stands not in such pastimes but in
the Workings in the way of Excellence, as has also been stated before.


VII

Now if Happiness is a Working in the way of Excellence of course that
Excellence must be the highest, that is to say, the Excellence of the
best Principle. Whether then this best Principle is Intellect or some
other which is thought naturally to rule and to lead and to conceive of
noble and divine things, whether being in its own nature divine or the
most divine of all our internal Principles, the Working of this in
accordance with its own proper Excellence must be the perfect Happiness.

That it is Contemplative has been already stated: and this would seem to
be consistent with what we said before and with truth: for, in the first
place, this Working is of the highest kind, since the Intellect is the
highest of our internal Principles and the subjects with which it
is conversant the highest of all which fall within the range of our
knowledge.

Next, it is also most Continuous: for we are better able to contemplate
than to do anything else whatever, continuously.

Again, we think Pleasure must be in some way an ingredient in Happiness,
and of all Workings in accordance with Excellence that in the way of
Science is confessedly most pleasant: at least the pursuit of Science is
thought to contain Pleasures admirable for purity and permanence; and it
is reasonable to suppose that the employment is more pleasant to those
who have mastered, than to those who are yet seeking for, it.

And the Self-Sufficiency which people speak of will attach chiefly to
the Contemplative Working: of course the actual necessaries of life are
needed alike by the man of science, and the just man, and all the other
characters; but, supposing all sufficiently supplied with these, the
just man needs people towards whom, and in concert with whom, to
practise his justice; and in like manner the man of perfected
self-mastery, and the brave man, and so on of the rest; whereas the man
of science can contemplate and speculate even when quite alone, and the
more entirely he deserves the appellation the more able is he to do so:
it may be he can do better for having fellow-workers but still he is
certainly most Self-Sufficient.

[Sidenote: 1177b] Again, this alone would seem to be rested in for
its own sake, since nothing results from it beyond the fact of having
contemplated; whereas from all things which are objects of moral action
we do mean to get something beside the doing them, be the same more or
less.

Also, Happiness is thought to stand in perfect rest; for we toil that we
may rest, and war that we may be at peace. Now all the Practical Virtues
require either society or war for their Working, and the actions
regarding these are thought to exclude rest; those of war entirely,
because no one chooses war, nor prepares for war, for war's sake: he
would indeed be thought a bloodthirsty villain who should make enemies
of his friends to secure the existence of fighting and bloodshed. The
Working also of the statesman excludes the idea of rest, and, beside the
actual work of government, seeks for power and dignities or at least
Happiness for the man himself and his fellow-citizens: a Happiness
distinct the national Happiness which we evidently seek as being
different and distinct.

If then of all the actions in accordance with the various virtues those
of policy and war are pre-eminent in honour and greatness, and these are
restless, and aim at some further End and are not choiceworthy for
their own sakes, but the Working of the Intellect, being apt for
contemplation, is thought to excel in earnestness, and to aim at no End
beyond itself and to have Pleasure of its own which helps to increase
the Working, and if the attributes of Self-Sufficiency, and capacity of
rest, and unweariedness (as far as is compatible with the infirmity
of human nature), and all other attributes of the highest Happiness,
plainly belong to this Working, this must be perfect Happiness, if
attaining a complete duration of life, which condition is added because
none of the points of Happiness is incomplete.

But such a life will be higher than mere human nature, because a man
will live thus, not in so far as he is man but in so far as there is in
him a divine Principle: and in proportion as this Principle excels
his composite nature so far does the Working thereof excel that in
accordance with any other kind of Excellence: and therefore, if pure
Intellect, as compared with human nature, is divine, so too will the
life in accordance with it be divine compared with man's ordinary life.
[Sidenote: 1178a] Yet must we not give ear to those who bid one as man
to mind only man's affairs, or as mortal only mortal things; but, so far
as we can, make ourselves like immortals and do all with a view to
living in accordance with the highest Principle in us, for small as it
may be in bulk yet in power and preciousness it far more excels all the
others.

In fact this Principle would seem to constitute each man's "Self," since
it is supreme and above all others in goodness it _would_ be absurd then
for a man not to choose his own life but that of some other.

And here will apply an observation made before, that whatever is proper
to each is naturally best and pleasantest to him: such then is to Man
the life in accordance with pure Intellect (since this Principle is most
truly Man), and if so, then it is also the happiest.


VIII

And second in degree of Happiness will be that Life which is in
accordance with the other kind of Excellence, for the Workings in
accordance with this are proper to Man: I mean, we do actions of
justice, courage, and the other virtues, towards one another, in
contracts, services of different kinds, and in all kinds of actions and
feelings too, by observing what is befitting for each: and all these
plainly are proper to man. Further, the Excellence of the Moral
character is thought to result in some points from physical
circumstances, and to be, in many, very closely connected with the
passions.

Again, Practical Wisdom and Excellence of the Moral character are
very closely united; since the Principles of Practical Wisdom are in
accordance with the Moral Virtues and these are right when they accord
with Practical Wisdom.

These moreover, as bound up with the passions, must belong to the
composite nature, and the Excellences or Virtues of the composite nature
are proper to man: therefore so too will be the life and Happiness which
is in accordance with them. But that of the Pure Intellect is separate
and distinct: and let this suffice upon the subject, since great
exactness is beyond our purpose,

It would seem, moreover, to require supply of external goods to a small
degree, or certainly less than the Moral Happiness: for, as far as
necessaries of life are concerned, we will suppose both characters to
need them equally (though, in point of fact, the man who lives in
society does take more pains about his person and all that kind of
thing; there will really be some little difference), but when we come to
consider their Workings there will be found a great difference.

I mean, the liberal man must have money to do his liberal actions with,
and the just man to meet his engagements (for mere intentions
are uncertain, and even those who are unjust make a pretence of
_wishing_ to do justly), and the brave man must have power, if
he is to perform any of the actions which appertain to his particular
Virtue, and the man of perfected self-mastery must have opportunity of
temptation, else how shall he or any of the others display his real
character?

[Sidenote: 1178b]

(By the way, a question is sometimes raised, whether the moral choice or
the actions have most to do with Virtue, since it consists in both: it
is plain that the perfection of virtuous action requires both: but for
the actions many things are required, and the greater and more numerous
they are the more.) But as for the man engaged in Contemplative
Speculation, not only are such things unnecessary for his Working, but,
so to speak, they are even hindrances: as regards the Contemplation at
least; because of course in so far as he is Man and lives in society he
chooses to do what Virtue requires, and so he will need such things
for maintaining his character as Man though not as a speculative
philosopher.

And that the perfect Happiness must be a kind of Contemplative Working
may appear also from the following consideration: our conception of the
gods is that they are above all blessed and happy: now what kind of
Moral actions are we to attribute to them? those of justice? nay,
will they not be set in a ridiculous light if represented as forming
contracts, and restoring deposits, and so on? well then, shall we
picture them performing brave actions, withstanding objects of fear and
meeting dangers, because it is noble to do so? or liberal ones? but to
whom shall they be giving? and further, it is absurd to think they have
money or anything of the kind. And as for actions of perfected
self-mastery, what can theirs be? would it not be a degrading praise
that they have no bad desires? In short, if one followed the subject
into all details all the circumstances connected with Moral actions
would appear trivial and unworthy of gods.

Still, every one believes that they live, and therefore that they
Work because it is not supposed that they sleep their time away like
Endymion: now if from a living being you take away Action, still more
if Creation, what remains but Contemplation? So then the Working of
the Gods, eminent in blessedness, will be one apt for Contemplative
Speculation; and of all human Workings that will have the greatest
capacity for Happiness which is nearest akin to this.

A corroboration of which position is the fact that the other animals
do not partake of Happiness, being completely shut out from any such
Working.

To the gods then all their life is blessed; and to men in so far as
there is in it some copy of such Working, but of the other animals none
is happy because it in no way shares in Contemplative Speculation.

Happiness then is co-extensive with this Contemplative Speculation, and
in proportion as people have the act of Contemplation so far have they
also the being happy, not incidentally, but in the way of Contemplative
Speculation because it is in itself precious.

So Happiness must be a kind of Contemplative Speculation; but since it
is Man we are speaking of he will need likewise External Prosperity,
because his Nature is not by itself sufficient for Speculation, but
there must be health of body, and nourishment, and tendance of all
kinds.

[Sidenote: 1179a] However, it must not be thought, because without
external goods a man cannot enjoy high Happiness, that therefore he
will require many and great goods in order to be happy: for neither
Self-sufficiency, nor Action, stand in Excess, and it is quite possible
to act nobly without being ruler of sea and land, since even with
moderate means a man may act in accordance with Virtue.

And this may be clearly seen in that men in private stations are thought
to act justly, not merely no less than men in power but even more: it
will be quite enough that just so much should belong to a man as is
necessary, for his life will be happy who works in accordance with
Virtue.

Solon perhaps drew a fair picture of the Happy, when he said that they
are men moderately supplied with external goods, and who have achieved
the most noble deeds, as he thought, and who have lived with perfect
self-mastery: for it is quite possible for men of moderate means to act
as they ought.

Anaxagoras also seems to have conceived of the Happy man not as either
rich or powerful, saying that he should not wonder if he were accounted
a strange man in the judgment of the multitude: for they judge by
outward circumstances of which alone they have any perception.

And thus the opinions of the Wise seem to be accordant with our account
of the matter: of course such things carry some weight, but truth, in
matters of moral action, is judged from facts and from actual life,
for herein rests the decision. So what we should do is to examine the
preceding statements by referring them to facts and to actual life, and
when they harmonise with facts we may accept them, when they are at
variance with them conceive of them as mere theories.

Now he that works in accordance with, and pays observance to, Pure
Intellect, and tends this, seems likely to be both in the best frame of
mind and dearest to the Gods: because if, as is thought, any care is
bestowed on human things by the Gods then it must be reasonable to think
that they take pleasure in what is best and most akin to themselves (and
this must be the Pure Intellect); and that they requite with kindness
those who love and honour this most, as paying observance to what is
dear to them, and as acting rightly and nobly. And it is quite obvious
that the man of Science chiefly combines all these: he is therefore
dearest to the Gods, and it is probable that he is at the same time most
Happy.

Thus then on this view also the man of Science will be most Happy.



IX

Now then that we have said enough in our sketchy kind of way
on these subjects; I mean, on the Virtues, and also on Friendship and
Pleasure; are we to suppose that our original purpose is completed? Must
we not rather acknowledge, what is commonly said, that in matters of
moral action mere Speculation and Knowledge is not the real End but
rather Practice: and if so, then neither in respect of Virtue is
Knowledge enough; we must further strive to have and exert it, and take
whatever other means there are of becoming good.

Now if talking and writing were of themselves sufficient to make men
good, they would justly, as Theognis observes have reaped numerous and
great rewards, and the thing to do would be to provide them: but in
point of fact, while they plainly have the power to guide and stimulate
the generous among the young and to base upon true virtuous principle
any noble and truly high-minded disposition, they as plainly are
powerless to guide the mass of men to Virtue and goodness; because it is
not their nature to be amenable to a sense of shame but only to fear;
nor to abstain from what is low and mean because it is disgraceful to do
it but because of the punishment attached to it: in fact, as they live
at the beck and call of passion, they pursue their own proper pleasures
and the means of securing them, and they avoid the contrary pains; but
as for what is noble and truly pleasurable they have not an idea of it,
inasmuch as they have never tasted of it.

Men such as these then what mere words can transform? No, indeed! it is
either actually impossible, or a task of no mean difficulty, to alter by
words what has been of old taken into men's very dispositions: and,
it may be, it is a ground for contentment if with all the means and
appliances for goodness in our hands we can attain to Virtue.

The formation of a virtuous character some ascribe to Nature, some to
Custom, and some to Teaching. Now Nature's part, be it what it may,
obviously does not rest with us, but belongs to those who in the truest
sense are fortunate, by reason of certain divine agency,

Then, as for Words and Precept, they, it is to be feared, will not avail
with all; but it may be necessary for the mind of the disciple to have
been previously prepared for liking and disliking as he ought; just as
the soil must, to nourish the seed sown. For he that lives in obedience
to passion cannot hear any advice that would dissuade him, nor, if he
heard, understand: now him that is thus how can one reform? in fact,
generally, passion is not thought to yield to Reason but to brute force.
So then there must be, to begin with, a kind of affinity to Virtue in
the disposition; which must cleave to what is honourable and loath
what is disgraceful. But to get right guidance towards Virtue from the
earliest youth is not easy unless one is brought up under laws of such
kind; because living with self-mastery and endurance is not pleasant to
the mass of men, and specially not to the young. For this reason the
food, and manner of living generally, ought to be the subject of
legal regulation, because things when become habitual will not be
disagreeable.

[Sidenote: 1180_a_] Yet perhaps it is not sufficient that men while
young should get right food and tendance, but, inasmuch as they will
have to practise and become accustomed to certain things even after they
have attained to man's estate, we shall want laws on these points as
well, and, in fine, respecting one's whole life, since the mass of men
are amenable to compulsion rather than Reason, and to punishment rather
than to a sense of honour.

And therefore some men hold that while lawgivers should employ the sense
of honour to exhort and guide men to Virtue, under the notion that they
will then obey who have been well trained in habits; they should
impose chastisement and penalties on those who disobey and are of less
promising nature; and the incurable expel entirely: because the good man
and he who lives under a sense of honour will be obedient to reason;
and the baser sort, who grasp at pleasure, will be kept in check, like
beasts of burthen by pain. Therefore also they say that the pains should
be such as are most contrary to the pleasures which are liked.

As has been said already, he who is to be good must have been brought up
and habituated well, and then live accordingly under good institutions,
and never do what is low and mean, either against or with his will. Now
these objects can be attained only by men living in accordance with some
guiding Intellect and right order, with power to back them.

As for the Paternal Rule, it possesses neither strength nor compulsory
power, nor in fact does the Rule of any one man, unless he is a king or
some one in like case: but the Law has power to compel, since it is a
declaration emanating from Practical Wisdom and Intellect. And people
feel enmity towards their fellow-men who oppose their impulses, however
rightly they may do so: the Law, on the contrary, is not the object of
hatred, though enforcing right rules.

The Lacedæmonian is nearly the only State in which the framer of the
Constitution has made any provision, it would seem, respecting the food
and manner of living of the people: in most States these points are
entirely neglected, and each man lives just as he likes, ruling his wife
and children Cyclops-Fashion.

Of course, the best thing would be that there should be a right Public
System and that we should be able to carry it out: but, since as a
public matter those points are neglected, the duty would seem to devolve
upon each individual to contribute to the cause of Virtue with his own
children and friends, or at least to make this his aim and purpose: and
this, it would seem, from what has been said, he will be best able to do
by making a Legislator of himself: since all public *[Sidenote: 1180_b_]
systems, it is plain, are formed by the instrumentality of laws and
those are good which are formed by that of good laws: whether they are
written or unwritten, whether they are applied to the training of one or
many, will not, it seems, make any difference, just as it does not in
music, gymnastics, or any other such accomplishments, which are gained
by practice.

For just as in Communities laws and customs prevail, so too in families
the express commands of the Head, and customs also: and even more in the
latter, because of blood-relationship and the benefits conferred:
for there you have, to begin with, people who have affection and are
naturally obedient to the authority which controls them.

Then, furthermore, Private training has advantages over Public, as in
the case of the healing art: for instance, as a general rule, a man who
is in a fever should keep quiet, and starve; but in a particular case,
perhaps, this may not hold good; or, to take a different illustration,
the boxer will not use the same way of fighting with all antagonists.

It would seem then that the individual will be most exactly attended to
under Private care, because so each will be more likely to obtain what
is expedient for him. Of course, whether in the art of healing, or
gymnastics, or any other, a man will treat individual cases the better
for being acquainted with general rules; as, "that so and so is good for
all, or for men in such and such cases:" because general maxims are not
only said to be but are the object-matter of sciences: still this is no
reason against the possibility of a man's taking excellent care of
some _one_ case, though he possesses no scientific knowledge but from
experience is exactly acquainted with what happens in each point; just
as some people are thought to doctor themselves best though they would
be wholly unable to administer relief to others. Yet it may seem to be
necessary nevertheless, for one who wishes to become a real artist and
well acquainted with the theory of his profession, to have recourse
to general principles and ascertain all their capacities: for we have
already stated that these are the object-matter of sciences.

If then it appears that we may become good through the instrumentality
of laws, of course whoso wishes to make men better by a system of care
and training must try to make a Legislator of himself; for to treat
skilfully just any one who may be put before you is not what any
ordinary person can do, but, if any one, he who has knowledge; as in the
healing art, and all others which involve careful practice and skill.

[Sidenote: 1181_a_] Will not then our next business be to inquire from
what sources, or how one may acquire this faculty of Legislation; or
shall we say, that, as in similar cases, Statesmen are the people to
learn from, since this faculty was thought to be a part of the Social
Science? Must we not admit that the Political Science plainly does not
stand on a similar footing to that of other sciences and faculties? I
mean, that while in all other cases those who impart the faculties
and themselves exert them are identical (physicians and painters for
instance) matters of Statesmanship the Sophists profess to teach, but
not one of them practises it, that being left to those actually engaged
in it: and these might really very well be thought to do it by some
singular knack and by mere practice rather than by any intellectual
process: for they neither write nor speak on these matters (though it
might be more to their credit than composing speeches for the courts or
the assembly), nor again have they made Statesmen of their own sons or
their friends.

One can hardly suppose but that they would have done so if they could,
seeing that they could have bequeathed no more precious legacy to their
communities, nor would they have preferred, for themselves or their
dearest friends, the possession of any faculty rather than this.

Practice, however, seems to contribute no little to its acquisition;
merely breathing the atmosphere of politics would never have made
Statesmen of them, and therefore we may conclude that they who would
acquire a knowledge of Statesmanship must have in addition practice.

But of the Sophists they who profess to teach it are plainly a long way
off from doing so: in fact, they have no knowledge at all of its nature
and objects; if they had, they would never have put it on the same
footing with Rhetoric or even on a lower: neither would they have
conceived it to be "an easy matter to legislate by simply collecting
such laws as are famous because of course one could select the best," as
though the selection were not a matter of skill, and the judging aright
a very great matter, as in Music: for they alone, who have practical
knowledge of a thing, can judge the performances rightly or understand
with what means and in what way they are accomplished, and what
harmonises with what: the unlearned must be content with being able to
discover whether the result is good or bad, as in painting.

[Sidenote: 1181_b_] Now laws may be called the performances or tangible
results of Political Science; how then can a man acquire from these
the faculty of Legislation, or choose the best? we do not see men made
physicians by compilations: and yet in these treatises men endeavour to
give not only the cases but also how they may be cured, and the proper
treatment in each case, dividing the various bodily habits. Well, these
are thought to be useful to professional men, but to the unprofessional
useless. In like manner it may be that collections of laws and
Constitutions would be exceedingly useful to such as are able to
speculate on them, and judge what is well, and what ill, and what
kind of things fit in with what others: but they who without this
qualification should go through such matters cannot have right judgment,
unless they have it by instinct, though they may become more intelligent
in such matters.

Since then those who have preceded us have left uninvestigated the
subject of Legislation, it will be better perhaps for us to investigate
it ourselves, and, in fact, the whole subject of Polity, that thus what
we may call Human Philosophy may be completed as far as in us lies.

First then, let us endeavour to get whatever fragments of good there may
be in the statements of our predecessors, next, from the Polities we
have collected, ascertain what kind of things preserve or destroy
Communities, and what, particular Constitutions; and the cause why some
are well and others ill managed, for after such inquiry, we shall be the
better able to take a concentrated view as to what kind of Constitution
is best, what kind of regulations are best for each, and what laws and
customs.

To this let us now proceed.

% chapter ethics (end)