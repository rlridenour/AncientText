\chapter{Parmenides} % (fold)
\label{cha:parmenides}


PARMENIDES

By Plato


Translated by Benjamin Jowett




INTRODUCTION AND ANALYSIS.

The awe with which Plato regarded the character of 'the great'
Parmenides has extended to the dialogue which he calls by his name. None
of the writings of Plato have been more copiously illustrated, both in
ancient and modern times, and in none of them have the interpreters
been more at variance with one another. Nor is this surprising. For the
Parmenides is more fragmentary and isolated than any other dialogue, and
the design of the writer is not expressly stated. The date is uncertain;
the relation to the other writings of Plato is also uncertain; the
connexion between the two parts is at first sight extremely obscure;
and in the latter of the two we are left in doubt as to whether Plato is
speaking his own sentiments by the lips of Parmenides, and overthrowing
him out of his own mouth, or whether he is propounding consequences
which would have been admitted by Zeno and Parmenides themselves. The
contradictions which follow from the hypotheses of the one and many have
been regarded by some as transcendental mysteries; by others as a mere
illustration, taken at random, of a new method. They seem to have been
inspired by a sort of dialectical frenzy, such as may be supposed to
have prevailed in the Megarian School (compare Cratylus, etc.). The
criticism on his own doctrine of Ideas has also been considered, not as
a real criticism, but as an exuberance of the metaphysical imagination
which enabled Plato to go beyond himself. To the latter part of the
dialogue we may certainly apply the words in which he himself describes
the earlier philosophers in the Sophist: 'They went on their way rather
regardless of whether we understood them or not.'

The Parmenides in point of style is one of the best of the Platonic
writings; the first portion of the dialogue is in no way defective in
ease and grace and dramatic interest; nor in the second part, where
there was no room for such qualities, is there any want of clearness or
precision. The latter half is an exquisite mosaic, of which the small
pieces are with the utmost fineness and regularity adapted to one
another. Like the Protagoras, Phaedo, and others, the whole is a
narrated dialogue, combining with the mere recital of the words spoken,
the observations of the reciter on the effect produced by them. Thus we
are informed by him that Zeno and Parmenides were not altogether pleased
at the request of Socrates that they would examine into the nature of
the one and many in the sphere of Ideas, although they received his
suggestion with approving smiles. And we are glad to be told that
Parmenides was 'aged but well-favoured,' and that Zeno was 'very
good-looking'; also that Parmenides affected to decline the great
argument, on which, as Zeno knew from experience, he was not unwilling
to enter. The character of Antiphon, the half-brother of Plato, who
had once been inclined to philosophy, but has now shown the hereditary
disposition for horses, is very naturally described. He is the sole
depositary of the famous dialogue; but, although he receives the
strangers like a courteous gentleman, he is impatient of the trouble of
reciting it. As they enter, he has been giving orders to a bridle-maker;
by this slight touch Plato verifies the previous description of him.
After a little persuasion he is induced to favour the Clazomenians, who
come from a distance, with a rehearsal. Respecting the visit of Zeno
and Parmenides to Athens, we may observe--first, that such a visit is
consistent with dates, and may possibly have occurred; secondly, that
Plato is very likely to have invented the meeting ('You, Socrates, can
easily invent Egyptian tales or anything else,' Phaedrus); thirdly, that
no reliance can be placed on the circumstance as determining the date
of Parmenides and Zeno; fourthly, that the same occasion appears to be
referred to by Plato in two other places (Theaet., Soph.).

Many interpreters have regarded the Parmenides as a 'reductio ad
absurdum' of the Eleatic philosophy. But would Plato have been likely to
place this in the mouth of the great Parmenides himself, who appeared
to him, in Homeric language, to be 'venerable and awful,' and to have
a 'glorious depth of mind'? (Theaet.). It may be admitted that he has
ascribed to an Eleatic stranger in the Sophist opinions which went
beyond the doctrines of the Eleatics. But the Eleatic stranger expressly
criticises the doctrines in which he had been brought up; he admits that
he is going to 'lay hands on his father Parmenides.' Nothing of this
kind is said of Zeno and Parmenides. How then, without a word of
explanation, could Plato assign to them the refutation of their own
tenets?

The conclusion at which we must arrive is that the Parmenides is not
a refutation of the Eleatic philosophy. Nor would such an explanation
afford any satisfactory connexion of the first and second parts of the
dialogue. And it is quite inconsistent with Plato's own relation to the
Eleatics. For of all the pre-Socratic philosophers, he speaks of them
with the greatest respect. But he could hardly have passed upon them a
more unmeaning slight than to ascribe to their great master tenets the
reverse of those which he actually held.

Two preliminary remarks may be made. First, that whatever latitude we
may allow to Plato in bringing together by a 'tour de force,' as in the
Phaedrus, dissimilar themes, yet he always in some way seeks to find
a connexion for them. Many threads join together in one the love and
dialectic of the Phaedrus. We cannot conceive that the great artist
would place in juxtaposition two absolutely divided and incoherent
subjects. And hence we are led to make a second remark: viz. that
no explanation of the Parmenides can be satisfactory which does not
indicate the connexion of the first and second parts. To suppose that
Plato would first go out of his way to make Parmenides attack the
Platonic Ideas, and then proceed to a similar but more fatal assault on
his own doctrine of Being, appears to be the height of absurdity.

Perhaps there is no passage in Plato showing greater metaphysical power
than that in which he assails his own theory of Ideas. The arguments are
nearly, if not quite, those of Aristotle; they are the objections which
naturally occur to a modern student of philosophy. Many persons will be
surprised to find Plato criticizing the very conceptions which have been
supposed in after ages to be peculiarly characteristic of him. How can
he have placed himself so completely without them? How can he have ever
persisted in them after seeing the fatal objections which might be urged
against them? The consideration of this difficulty has led a recent
critic (Ueberweg), who in general accepts the authorised canon of the
Platonic writings, to condemn the Parmenides as spurious. The accidental
want of external evidence, at first sight, seems to favour this opinion.

In answer, it might be sufficient to say, that no ancient writing of
equal length and excellence is known to be spurious. Nor is the silence
of Aristotle to be hastily assumed; there is at least a doubt whether
his use of the same arguments does not involve the inference that he
knew the work. And, if the Parmenides is spurious, like Ueberweg, we
are led on further than we originally intended, to pass a similar
condemnation on the Theaetetus and Sophist, and therefore on the
Politicus (compare Theaet., Soph.). But the objection is in reality
fanciful, and rests on the assumption that the doctrine of the Ideas was
held by Plato throughout his life in the same form. For the truth
is, that the Platonic Ideas were in constant process of growth and
transmutation; sometimes veiled in poetry and mythology, then again
emerging as fixed Ideas, in some passages regarded as absolute and
eternal, and in others as relative to the human mind, existing in
and derived from external objects as well as transcending them. The
anamnesis of the Ideas is chiefly insisted upon in the mythical portions
of the dialogues, and really occupies a very small space in the entire
works of Plato. Their transcendental existence is not asserted, and
is therefore implicitly denied in the Philebus; different forms
are ascribed to them in the Republic, and they are mentioned in the
Theaetetus, the Sophist, the Politicus, and the Laws, much as Universals
would be spoken of in modern books. Indeed, there are very faint traces
of the transcendental doctrine of Ideas, that is, of their existence
apart from the mind, in any of Plato's writings, with the exception of
the Meno, the Phaedrus, the Phaedo, and in portions of the Republic. The
stereotyped form which Aristotle has given to them is not found in Plato
(compare Essay on the Platonic Ideas in the Introduction to the Meno.)

The full discussion of this subject involves a comprehensive survey of
the philosophy of Plato, which would be out of place here. But, without
digressing further from the immediate subject of the Parmenides, we
may remark that Plato is quite serious in his objections to his own
doctrines: nor does Socrates attempt to offer any answer to them. The
perplexities which surround the one and many in the sphere of the Ideas
are also alluded to in the Philebus, and no answer is given to them. Nor
have they ever been answered, nor can they be answered by any one else
who separates the phenomenal from the real. To suppose that Plato, at a
later period of his life, reached a point of view from which he was able
to answer them, is a groundless assumption. The real progress of Plato's
own mind has been partly concealed from us by the dogmatic statements of
Aristotle, and also by the degeneracy of his own followers, with whom a
doctrine of numbers quickly superseded Ideas.

As a preparation for answering some of the difficulties which have
been suggested, we may begin by sketching the first portion of the
dialogue:--

Cephalus, of Clazomenae in Ionia, the birthplace of Anaxagoras, a
citizen of no mean city in the history of philosophy, who is the
narrator of the dialogue, describes himself as meeting Adeimantus and
Glaucon in the Agora at Athens. 'Welcome, Cephalus: can we do anything
for you in Athens?' 'Why, yes: I came to ask a favour of you. First,
tell me your half-brother's name, which I have forgotten--he was a mere
child when I was last here;--I know his father's, which is Pyrilampes.'
'Yes, and the name of our brother is Antiphon. But why do you ask?'
'Let me introduce to you some countrymen of mine, who are lovers of
philosophy; they have heard that Antiphon remembers a conversation of
Socrates with Parmenides and Zeno, of which the report came to him from
Pythodorus, Zeno's friend.' 'That is quite true.' 'And can they hear the
dialogue?' 'Nothing easier; in the days of his youth he made a careful
study of the piece; at present, his thoughts have another direction: he
takes after his grandfather, and has given up philosophy for horses.'

'We went to look for him, and found him giving instructions to a worker
in brass about a bridle. When he had done with him, and had learned
from his brothers the purpose of our visit, he saluted me as an old
acquaintance, and we asked him to repeat the dialogue. At first, he
complained of the trouble, but he soon consented. He told us that
Pythodorus had described to him the appearance of Parmenides and Zeno;
they had come to Athens at the great Panathenaea, the former being at
the time about sixty-five years old, aged but well-favoured--Zeno, who
was said to have been beloved of Parmenides in the days of his youth,
about forty, and very good-looking:--that they lodged with Pythodorus at
the Ceramicus outside the wall, whither Socrates, then a very young
man, came to see them: Zeno was reading one of his theses, which he
had nearly finished, when Pythodorus entered with Parmenides and
Aristoteles, who was afterwards one of the Thirty. When the recitation
was completed, Socrates requested that the first thesis of the treatise
might be read again.'

'You mean, Zeno,' said Socrates, 'to argue that being, if it is many,
must be both like and unlike, which is a contradiction; and each
division of your argument is intended to elicit a similar absurdity,
which may be supposed to follow from the assumption that being is many.'
'Such is my meaning.' 'I see,' said Socrates, turning to Parmenides,
'that Zeno is your second self in his writings too; you prove admirably
that the all is one: he gives proofs no less convincing that the many
are nought. To deceive the world by saying the same thing in entirely
different forms, is a strain of art beyond most of us.' 'Yes, Socrates,'
said Zeno; 'but though you are as keen as a Spartan hound, you do not
quite catch the motive of the piece, which was only intended to protect
Parmenides against ridicule by showing that the hypothesis of the
existence of the many involved greater absurdities than the hypothesis
of the one. The book was a youthful composition of mine, which was
stolen from me, and therefore I had no choice about the publication.' 'I
quite believe you,' said Socrates; 'but will you answer me a question? I
should like to know, whether you would assume an idea of likeness in the
abstract, which is the contradictory of unlikeness in the abstract, by
participation in either or both of which things are like or unlike
or partly both. For the same things may very well partake of like and
unlike in the concrete, though like and unlike in the abstract are
irreconcilable. Nor does there appear to me to be any absurdity in
maintaining that the same things may partake of the one and many, though
I should be indeed surprised to hear that the absolute one is also
many. For example, I, being many, that is to say, having many parts or
members, am yet also one, and partake of the one, being one of seven
who are here present (compare Philebus). This is not an absurdity, but
a truism. But I should be amazed if there were a similar entanglement in
the nature of the ideas themselves, nor can I believe that one and many,
like and unlike, rest and motion, in the abstract, are capable either of
admixture or of separation.'

Pythodorus said that in his opinion Parmenides and Zeno were not very
well pleased at the questions which were raised; nevertheless, they
looked at one another and smiled in seeming delight and admiration of
Socrates. 'Tell me,' said Parmenides, 'do you think that the abstract
ideas of likeness, unity, and the rest, exist apart from individuals
which partake of them? and is this your own distinction?' 'I think that
there are such ideas.' 'And would you make abstract ideas of the just,
the beautiful, the good?' 'Yes,' he said. 'And of human beings like
ourselves, of water, fire, and the like?' 'I am not certain.' 'And would
you be undecided also about ideas of which the mention will, perhaps,
appear laughable: of hair, mud, filth, and other things which are base
and vile?' 'No, Parmenides; visible things like these are, as I believe,
only what they appear to be: though I am sometimes disposed to imagine
that there is nothing without an idea; but I repress any such notion,
from a fear of falling into an abyss of nonsense.' 'You are young,
Socrates, and therefore naturally regard the opinions of men; the time
will come when philosophy will have a firmer hold of you, and you will
not despise even the meanest things. But tell me, is your meaning that
things become like by partaking of likeness, great by partaking of
greatness, just and beautiful by partaking of justice and beauty, and
so of other ideas?' 'Yes, that is my meaning.' 'And do you suppose the
individual to partake of the whole, or of the part?' 'Why not of the
whole?' said Socrates. 'Because,' said Parmenides, 'in that case the
whole, which is one, will become many.' 'Nay,' said Socrates, 'the whole
may be like the day, which is one and in many places: in this way
the ideas may be one and also many.' 'In the same sort of way,' said
Parmenides, 'as a sail, which is one, may be a cover to many--that is
your meaning?' 'Yes.' 'And would you say that each man is covered by the
whole sail, or by a part only?' 'By a part.' 'Then the ideas have parts,
and the objects partake of a part of them only?' 'That seems to follow.'
'And would you like to say that the ideas are really divisible and yet
remain one?' 'Certainly not.' 'Would you venture to affirm that great
objects have a portion only of greatness transferred to them; or that
small or equal objects are small or equal because they are only portions
of smallness or equality?' 'Impossible.' 'But how can individuals
participate in ideas, except in the ways which I have mentioned?' 'That
is not an easy question to answer.' 'I should imagine the conception of
ideas to arise as follows: you see great objects pervaded by a common
form or idea of greatness, which you abstract.' 'That is quite true.'
'And supposing you embrace in one view the idea of greatness thus gained
and the individuals which it comprises, a further idea of greatness
arises, which makes both great; and this may go on to infinity.'
Socrates replies that the ideas may be thoughts in the mind only; in
this case, the consequence would no longer follow. 'But must not the
thought be of something which is the same in all and is the idea? And
if the world partakes in the ideas, and the ideas are thoughts, must not
all things think? Or can thought be without thought?' 'I acknowledge the
unmeaningness of this,' says Socrates, 'and would rather have recourse
to the explanation that the ideas are types in nature, and that other
things partake of them by becoming like them.' 'But to become like them
is to be comprehended in the same idea; and the likeness of the idea and
the individuals implies another idea of likeness, and another without
end.' 'Quite true.' 'The theory, then, of participation by likeness
has to be given up. You have hardly yet, Socrates, found out the real
difficulty of maintaining abstract ideas.' 'What difficulty?' 'The
greatest of all perhaps is this: an opponent will argue that the ideas
are not within the range of human knowledge; and you cannot disprove the
assertion without a long and laborious demonstration, which he may be
unable or unwilling to follow. In the first place, neither you nor any
one who maintains the existence of absolute ideas will affirm that they
are subjective.' 'That would be a contradiction.' 'True; and therefore
any relation in these ideas is a relation which concerns themselves
only; and the objects which are named after them, are relative to one
another only, and have nothing to do with the ideas themselves.' 'How do
you mean?' said Socrates. 'I may illustrate my meaning in this way: one
of us has a slave; and the idea of a slave in the abstract is relative
to the idea of a master in the abstract; this correspondence of ideas,
however, has nothing to do with the particular relation of our slave to
us.--Do you see my meaning?' 'Perfectly.' 'And absolute knowledge in
the same way corresponds to absolute truth and being, and particular
knowledge to particular truth and being.' Clearly.' 'And there is a
subjective knowledge which is of subjective truth, having many kinds,
general and particular. But the ideas themselves are not subjective, and
therefore are not within our ken.' 'They are not.' 'Then the beautiful
and the good in their own nature are unknown to us?' 'It would seem so.'
'There is a worse consequence yet.' 'What is that?' 'I think we must
admit that absolute knowledge is the most exact knowledge, which we must
therefore attribute to God. But then see what follows: God, having
this exact knowledge, can have no knowledge of human things, as we
have divided the two spheres, and forbidden any passing from one to the
other:--the gods have knowledge and authority in their world only, as
we have in ours.' 'Yet, surely, to deprive God of knowledge is
monstrous.'--'These are some of the difficulties which are involved
in the assumption of absolute ideas; the learner will find them nearly
impossible to understand, and the teacher who has to impart them will
require superhuman ability; there will always be a suspicion, either
that they have no existence, or are beyond human knowledge.' 'There I
agree with you,' said Socrates. 'Yet if these difficulties induce you
to give up universal ideas, what becomes of the mind? and where are the
reasoning and reflecting powers? philosophy is at an end.' 'I certainly
do not see my way.' 'I think,' said Parmenides, 'that this arises out
of your attempting to define abstractions, such as the good and
the beautiful and the just, before you have had sufficient previous
training; I noticed your deficiency when you were talking with
Aristoteles, the day before yesterday. Your enthusiasm is a wonderful
gift; but I fear that unless you discipline yourself by dialectic
while you are young, truth will elude your grasp.' 'And what kind of
discipline would you recommend?' 'The training which you heard Zeno
practising; at the same time, I admire your saying to him that you did
not care to consider the difficulty in reference to visible objects,
but only in relation to ideas.' 'Yes; because I think that in visible
objects you may easily show any number of inconsistent consequences.'
'Yes; and you should consider, not only the consequences which follow
from a given hypothesis, but the consequences also which follow from the
denial of the hypothesis. For example, what follows from the assumption
of the existence of the many, and the counter-argument of what follows
from the denial of the existence of the many: and similarly of likeness
and unlikeness, motion, rest, generation, corruption, being and not
being. And the consequences must include consequences to the things
supposed and to other things, in themselves and in relation to one
another, to individuals whom you select, to the many, and to the all;
these must be drawn out both on the affirmative and on the negative
hypothesis,--that is, if you are to train yourself perfectly to the
intelligence of the truth.' 'What you are suggesting seems to be a
tremendous process, and one of which I do not quite understand the
nature,' said Socrates; 'will you give me an example?' 'You must not
impose such a task on a man of my years,' said Parmenides. 'Then will
you, Zeno?' 'Let us rather,' said Zeno, with a smile, 'ask Parmenides,
for the undertaking is a serious one, as he truly says; nor could I urge
him to make the attempt, except in a select audience of persons who will
understand him.' The whole party joined in the request.

Here we have, first of all, an unmistakable attack made by the youthful
Socrates on the paradoxes of Zeno. He perfectly understands their drift,
and Zeno himself is supposed to admit this. But they appear to him, as
he says in the Philebus also, to be rather truisms than paradoxes. For
every one must acknowledge the obvious fact, that the body being one
has many members, and that, in a thousand ways, the like partakes of
the unlike, the many of the one. The real difficulty begins with the
relations of ideas in themselves, whether of the one and many, or of
any other ideas, to one another and to the mind. But this was a problem
which the Eleatic philosophers had never considered; their thoughts had
not gone beyond the contradictions of matter, motion, space, and the
like.

It was no wonder that Parmenides and Zeno should hear the novel
speculations of Socrates with mixed feelings of admiration and
displeasure. He was going out of the received circle of disputation into
a region in which they could hardly follow him. From the crude idea of
Being in the abstract, he was about to proceed to universals or general
notions. There is no contradiction in material things partaking of the
ideas of one and many; neither is there any contradiction in the ideas
of one and many, like and unlike, in themselves. But the contradiction
arises when we attempt to conceive ideas in their connexion, or to
ascertain their relation to phenomena. Still he affirms the existence of
such ideas; and this is the position which is now in turn submitted to
the criticisms of Parmenides.

To appreciate truly the character of these criticisms, we must remember
the place held by Parmenides in the history of Greek philosophy. He
is the founder of idealism, and also of dialectic, or, in modern
phraseology, of metaphysics and logic (Theaet., Soph.). Like Plato,
he is struggling after something wider and deeper than satisfied the
contemporary Pythagoreans. And Plato with a true instinct recognizes
him as his spiritual father, whom he 'revered and honoured more than all
other philosophers together.' He may be supposed to have thought more
than he said, or was able to express. And, although he could not, as
a matter of fact, have criticized the ideas of Plato without an
anachronism, the criticism is appropriately placed in the mouth of the
founder of the ideal philosophy.

There was probably a time in the life of Plato when the ethical teaching
of Socrates came into conflict with the metaphysical theories of the
earlier philosophers, and he sought to supplement the one by the other.
The older philosophers were great and awful; and they had the charm of
antiquity. Something which found a response in his own mind seemed to
have been lost as well as gained in the Socratic dialectic. He felt no
incongruity in the veteran Parmenides correcting the youthful Socrates.
Two points in his criticism are especially deserving of notice. First
of all, Parmenides tries him by the test of consistency. Socrates is
willing to assume ideas or principles of the just, the beautiful, the
good, and to extend them to man (compare Phaedo); but he is reluctant to
admit that there are general ideas of hair, mud, filth, etc. There is an
ethical universal or idea, but is there also a universal of physics?--of
the meanest things in the world as well as of the greatest? Parmenides
rebukes this want of consistency in Socrates, which he attributes to his
youth. As he grows older, philosophy will take a firmer hold of him, and
then he will despise neither great things nor small, and he will think
less of the opinions of mankind (compare Soph.). Here is lightly touched
one of the most familiar principles of modern philosophy, that in the
meanest operations of nature, as well as in the noblest, in mud and
filth, as well as in the sun and stars, great truths are contained. At
the same time, we may note also the transition in the mind of Plato,
to which Aristotle alludes (Met.), when, as he says, he transferred the
Socratic universal of ethics to the whole of nature.

The other criticism of Parmenides on Socrates attributes to him a want
of practice in dialectic. He has observed this deficiency in him when
talking to Aristoteles on a previous occasion. Plato seems to imply
that there was something more in the dialectic of Zeno than in the mere
interrogation of Socrates. Here, again, he may perhaps be describing
the process which his own mind went through when he first became more
intimately acquainted, whether at Megara or elsewhere, with the Eleatic
and Megarian philosophers. Still, Parmenides does not deny to Socrates
the credit of having gone beyond them in seeking to apply the paradoxes
of Zeno to ideas; and this is the application which he himself makes of
them in the latter part of the dialogue. He then proceeds to explain
to him the sort of mental gymnastic which he should practise. He should
consider not only what would follow from a given hypothesis, but what
would follow from the denial of it, to that which is the subject of
the hypothesis, and to all other things. There is no trace in the
Memorabilia of Xenophon of any such method being attributed to
Socrates; nor is the dialectic here spoken of that 'favourite method' of
proceeding by regular divisions, which is described in the Phaedrus and
Philebus, and of which examples are given in the Politicus and in the
Sophist. It is expressly spoken of as the method which Socrates had
heard Zeno practise in the days of his youth (compare Soph.).

The discussion of Socrates with Parmenides is one of the most remarkable
passages in Plato. Few writers have ever been able to anticipate 'the
criticism of the morrow' on their favourite notions. But Plato may here
be said to anticipate the judgment not only of the morrow, but of
all after-ages on the Platonic Ideas. For in some points he touches
questions which have not yet received their solution in modern
philosophy.

The first difficulty which Parmenides raises respecting the Platonic
ideas relates to the manner in which individuals are connected with
them. Do they participate in the ideas, or do they merely resemble them?
Parmenides shows that objections may be urged against either of these
modes of conceiving the connection. Things are little by partaking of
littleness, great by partaking of greatness, and the like. But they
cannot partake of a part of greatness, for that will not make them
great, etc.; nor can each object monopolise the whole. The only answer
to this is, that 'partaking' is a figure of speech, really corresponding
to the processes which a later logic designates by the terms
'abstraction' and 'generalization.' When we have described accurately
the methods or forms which the mind employs, we cannot further criticize
them; at least we can only criticize them with reference to their
fitness as instruments of thought to express facts.

Socrates attempts to support his view of the ideas by the parallel of
the day, which is one and in many places; but he is easily driven from
his position by a counter illustration of Parmenides, who compares the
idea of greatness to a sail. He truly explains to Socrates that he has
attained the conception of ideas by a process of generalization. At
the same time, he points out a difficulty, which appears to be
involved--viz. that the process of generalization will go on to
infinity. Socrates meets the supposed difficulty by a flash of light,
which is indeed the true answer 'that the ideas are in our minds
only.' Neither realism is the truth, nor nominalism is the truth, but
conceptualism; and conceptualism or any other psychological theory falls
very far short of the infinite subtlety of language and thought.

But the realism of ancient philosophy will not admit of this answer,
which is repelled by Parmenides with another truth or half-truth of
later philosophy, 'Every subject or subjective must have an object.'
Here is the great though unconscious truth (shall we say?) or error,
which underlay the early Greek philosophy. 'Ideas must have a real
existence;' they are not mere forms or opinions, which may be changed
arbitrarily by individuals. But the early Greek philosopher never
clearly saw that true ideas were only universal facts, and that there
might be error in universals as well as in particulars.

Socrates makes one more attempt to defend the Platonic Ideas by
representing them as paradigms; this is again answered by the
'argumentum ad infinitum.' We may remark, in passing, that the process
which is thus described has no real existence. The mind, after having
obtained a general idea, does not really go on to form another which
includes that, and all the individuals contained under it, and another
and another without end. The difficulty belongs in fact to the Megarian
age of philosophy, and is due to their illogical logic, and to the
general ignorance of the ancients respecting the part played by language
in the process of thought. No such perplexity could ever trouble
a modern metaphysician, any more than the fallacy of 'calvus' or
'acervus,' or of 'Achilles and the tortoise.' These 'surds' of
metaphysics ought to occasion no more difficulty in speculation than a
perpetually recurring fraction in arithmetic.

It is otherwise with the objection which follows: How are we to bridge
the chasm between human truth and absolute truth, between gods and men?
This is the difficulty of philosophy in all ages: How can we get beyond
the circle of our own ideas, or how, remaining within them, can we have
any criterion of a truth beyond and independent of them? Parmenides
draws out this difficulty with great clearness. According to him, there
are not only one but two chasms: the first, between individuals and the
ideas which have a common name; the second, between the ideas in us and
the ideas absolute. The first of these two difficulties mankind, as
we may say, a little parodying the language of the Philebus, have long
agreed to treat as obsolete; the second remains a difficulty for us as
well as for the Greeks of the fourth century before Christ, and is the
stumbling-block of Kant's Kritik, and of the Hamiltonian adaptation
of Kant, as well as of the Platonic ideas. It has been said that 'you
cannot criticize Revelation.' 'Then how do you know what is Revelation,
or that there is one at all,' is the immediate rejoinder--'You know
nothing of things in themselves.' 'Then how do you know that there are
things in themselves?' In some respects, the difficulty pressed harder
upon the Greek than upon ourselves. For conceiving of God more under the
attribute of knowledge than we do, he was more under the necessity
of separating the divine from the human, as two spheres which had no
communication with one another.

It is remarkable that Plato, speaking by the mouth of Parmenides,
does not treat even this second class of difficulties as hopeless or
insoluble. He says only that they cannot be explained without a long and
laborious demonstration: 'The teacher will require superhuman ability,
and the learner will be hard of understanding.' But an attempt must be
made to find an answer to them; for, as Socrates and Parmenides both
admit, the denial of abstract ideas is the destruction of the mind. We
can easily imagine that among the Greek schools of philosophy in the
fourth century before Christ a panic might arise from the denial of
universals, similar to that which arose in the last century from Hume's
denial of our ideas of cause and effect. Men do not at first recognize
that thought, like digestion, will go on much the same, notwithstanding
any theories which may be entertained respecting the nature of the
process. Parmenides attributes the difficulties in which Socrates is
involved to a want of comprehensiveness in his mode of reasoning; he
should consider every question on the negative as well as the positive
hypothesis, with reference to the consequences which flow from the
denial as well as from the assertion of a given statement.

The argument which follows is the most singular in Plato. It appears
to be an imitation, or parody, of the Zenonian dialectic, just as the
speeches in the Phaedrus are an imitation of the style of Lysias, or as
the derivations in the Cratylus or the fallacies of the Euthydemus are
a parody of some contemporary Sophist. The interlocutor is not supposed,
as in most of the other Platonic dialogues, to take a living part in the
argument; he is only required to say 'Yes' and 'No' in the right places.
A hint has been already given that the paradoxes of Zeno admitted of a
higher application. This hint is the thread by which Plato connects the
two parts of the dialogue.

The paradoxes of Parmenides seem trivial to us, because the words to
which they relate have become trivial; their true nature as abstract
terms is perfectly understood by us, and we are inclined to regard the
treatment of them in Plato as a mere straw-splitting, or legerdemain of
words. Yet there was a power in them which fascinated the Neoplatonists
for centuries afterwards. Something that they found in them, or brought
to them--some echo or anticipation of a great truth or error, exercised
a wonderful influence over their minds. To do the Parmenides justice, we
should imagine similar aporiai raised on themes as sacred to us, as the
notions of One or Being were to an ancient Eleatic. 'If God is, what
follows? If God is not, what follows?' Or again: If God is or is not the
world; or if God is or is not many, or has or has not parts, or is or is
not in the world, or in time; or is or is not finite or infinite. Or if
the world is or is not; or has or has not a beginning or end; or is or
is not infinite, or infinitely divisible. Or again: if God is or is not
identical with his laws; or if man is or is not identical with the laws
of nature. We can easily see that here are many subjects for thought,
and that from these and similar hypotheses questions of great interest
might arise. And we also remark, that the conclusions derived from
either of the two alternative propositions might be equally impossible
and contradictory.

When we ask what is the object of these paradoxes, some have answered
that they are a mere logical puzzle, while others have seen in them an
Hegelian propaedeutic of the doctrine of Ideas. The first of these views
derives support from the manner in which Parmenides speaks of a similar
method being applied to all Ideas. Yet it is hard to suppose that Plato
would have furnished so elaborate an example, not of his own but of
the Eleatic dialectic, had he intended only to give an illustration of
method. The second view has been often overstated by those who, like
Hegel himself, have tended to confuse ancient with modern philosophy.
We need not deny that Plato, trained in the school of Cratylus and
Heracleitus, may have seen that a contradiction in terms is sometimes
the best expression of a truth higher than either (compare Soph.). But
his ideal theory is not based on antinomies. The correlation of Ideas
was the metaphysical difficulty of the age in which he lived; and the
Megarian and Cynic philosophy was a 'reductio ad absurdum' of their
isolation. To restore them to their natural connexion and to detect the
negative element in them is the aim of Plato in the Sophist. But his
view of their connexion falls very far short of the Hegelian identity
of Being and Not-being. The Being and Not-being of Plato never merge in
each other, though he is aware that 'determination is only negation.'

After criticizing the hypotheses of others, it may appear presumptuous
to add another guess to the many which have been already offered. May we
say, in Platonic language, that we still seem to see vestiges of a track
which has not yet been taken? It is quite possible that the obscurity
of the Parmenides would not have existed to a contemporary student of
philosophy, and, like the similar difficulty in the Philebus, is
really due to our ignorance of the mind of the age. There is an obscure
Megarian influence on Plato which cannot wholly be cleared up, and is
not much illustrated by the doubtful tradition of his retirement to
Megara after the death of Socrates. For Megara was within a walk of
Athens (Phaedr.), and Plato might have learned the Megarian doctrines
without settling there.

We may begin by remarking that the theses of Parmenides are expressly
said to follow the method of Zeno, and that the complex dilemma, though
declared to be capable of universal application, is applied in this
instance to Zeno's familiar question of the 'one and many.' Here, then,
is a double indication of the connexion of the Parmenides with the
Eristic school. The old Eleatics had asserted the existence of Being,
which they at first regarded as finite, then as infinite, then as
neither finite nor infinite, to which some of them had given what
Aristotle calls 'a form,' others had ascribed a material nature only.
The tendency of their philosophy was to deny to Being all predicates.
The Megarians, who succeeded them, like the Cynics, affirmed that no
predicate could be asserted of any subject; they also converted the
idea of Being into an abstraction of Good, perhaps with the view of
preserving a sort of neutrality or indifference between the mind and
things. As if they had said, in the language of modern philosophy:
'Being is not only neither finite nor infinite, neither at rest nor in
motion, but neither subjective nor objective.'

This is the track along which Plato is leading us. Zeno had attempted to
prove the existence of the one by disproving the existence of the many,
and Parmenides seems to aim at proving the existence of the subject
by showing the contradictions which follow from the assertion of any
predicates. Take the simplest of all notions, 'unity'; you cannot even
assert being or time of this without involving a contradiction. But is
the contradiction also the final conclusion? Probably no more than of
Zeno's denial of the many, or of Parmenides' assault upon the Ideas; no
more than of the earlier dialogues 'of search.' To us there seems to
be no residuum of this long piece of dialectics. But to the mind of
Parmenides and Plato, 'Gott-betrunkene Menschen,' there still remained
the idea of 'being' or 'good,' which could not be conceived, defined,
uttered, but could not be got rid of. Neither of them would have
imagined that their disputation ever touched the Divine Being (compare
Phil.). The same difficulties about Unity and Being are raised in the
Sophist; but there only as preliminary to their final solution.

If this view is correct, the real aim of the hypotheses of Parmenides
is to criticize the earlier Eleatic philosophy from the point of view of
Zeno or the Megarians. It is the same kind of criticism which Plato has
extended to his own doctrine of Ideas. Nor is there any want of poetical
consistency in attributing to the 'father Parmenides' the last review
of the Eleatic doctrines. The latest phases of all philosophies were
fathered upon the founder of the school.

Other critics have regarded the final conclusion of the Parmenides
either as sceptical or as Heracleitean. In the first case, they assume
that Plato means to show the impossibility of any truth. But this is not
the spirit of Plato, and could not with propriety be put into the mouth
of Parmenides, who, in this very dialogue, is urging Socrates, not to
doubt everything, but to discipline his mind with a view to the more
precise attainment of truth. The same remark applies to the second of
the two theories. Plato everywhere ridicules (perhaps unfairly) his
Heracleitean contemporaries: and if he had intended to support an
Heracleitean thesis, would hardly have chosen Parmenides, the condemner
of the 'undiscerning tribe who say that things both are and are not,'
to be the speaker. Nor, thirdly, can we easily persuade ourselves with
Zeller that by the 'one' he means the Idea; and that he is seeking to
prove indirectly the unity of the Idea in the multiplicity of phenomena.

We may now endeavour to thread the mazes of the labyrinth which
Parmenides knew so well, and trembled at the thought of them.

The argument has two divisions: There is the hypothesis that

     1.  One is.
     2.  One is not.
     If one is, it is nothing.
     If one is not, it is everything.

     But is and is not may be taken in two senses:
     Either one is one,
     Or, one has being,

     from which opposite consequences are deduced,
     1.a.  If one is one, it is nothing.
     1.b.  If one has being, it is all things.

     To which are appended two subordinate consequences:
     1.aa.  If one has being, all other things are.
     1.bb.  If one is one, all other things are not.

     The same distinction is then applied to the negative hypothesis:
     2.a.  If one is not one, it is all things.
     2.b.  If one has not being, it is nothing.

     Involving two parallel consequences respecting the other or remainder:
     2.aa.  If one is not one, other things are all.
     2.bb.  If one has not being, other things are not.


.....

'I cannot refuse,' said Parmenides, 'since, as Zeno remarks, we are
alone, though I may say with Ibycus, who in his old age fell in love, I,
like the old racehorse, tremble at the prospect of the course which I am
to run, and which I know so well. But as I must attempt this laborious
game, what shall be the subject? Suppose I take my own hypothesis of
the one.' 'By all means,' said Zeno. 'And who will answer me? Shall I
propose the youngest? he will be the most likely to say what he thinks,
and his answers will give me time to breathe.' 'I am the youngest,' said
Aristoteles, 'and at your service; proceed with your questions.'--The
result may be summed up as follows:--

1.a. One is not many, and therefore has no parts, and therefore is not
a whole, which is a sum of parts, and therefore has neither beginning,
middle, nor end, and is therefore unlimited, and therefore formless,
being neither round nor straight, for neither round nor straight can be
defined without assuming that they have parts; and therefore is not in
place, whether in another which would encircle and touch the one at
many points; or in itself, because that which is self-containing is also
contained, and therefore not one but two. This being premised, let us
consider whether one is capable either of motion or rest. For motion is
either change of substance, or motion on an axis, or from one place to
another. But the one is incapable of change of substance, which implies
that it ceases to be itself, or of motion on an axis, because there
would be parts around the axis; and any other motion involves change of
place. But existence in place has been already shown to be impossible;
and yet more impossible is coming into being in place, which implies
partial existence in two places at once, or entire existence neither
within nor without the same; and how can this be? And more impossible
still is the coming into being either as a whole or parts of that which
is neither a whole nor parts. The one, then, is incapable of motion.
But neither can the one be in anything, and therefore not in the same,
whether itself or some other, and is therefore incapable of rest.
Neither is one the same with itself or any other, or other than itself
or any other. For if other than itself, then other than one, and
therefore not one; and, if the same with other, it would be other, and
other than one. Neither can one while remaining one be other than other;
for other, and not one, is the other than other. But if not other by
virtue of being one, not by virtue of itself; and if not by virtue
of itself, not itself other, and if not itself other, not other than
anything. Neither will one be the same with itself. For the nature of
the same is not that of the one, but a thing which becomes the same with
anything does not become one; for example, that which becomes the same
with the many becomes many and not one. And therefore if the one is the
same with itself, the one is not one with itself; and therefore one and
not one. And therefore one is neither other than other, nor the same
with itself. Neither will the one be like or unlike itself or other;
for likeness is sameness of affections, and the one and the same are
different. And one having any affection which is other than being one
would be more than one. The one, then, cannot have the same affection
with and therefore cannot be like itself or other; nor can the one
have any other affection than its own, that is, be unlike itself or any
other, for this would imply that it was more than one. The one, then,
is neither like nor unlike itself or other. This being the case, neither
can the one be equal or unequal to itself or other. For equality implies
sameness of measure, as inequality implies a greater or less number
of measures. But the one, not having sameness, cannot have sameness of
measure; nor a greater or less number of measures, for that would imply
parts and multitude. Once more, can one be older or younger than itself
or other? or of the same age with itself or other? That would imply
likeness and unlikeness, equality and inequality. Therefore one cannot
be in time, because that which is in time is ever becoming older and
younger than itself, (for older and younger are relative terms, and he
who becomes older becomes younger,) and is also of the same age with
itself. None of which, or any other expressions of time, whether past,
future, or present, can be affirmed of one. One neither is, has been,
nor will be, nor becomes, nor has, nor will become. And, as these are
the only modes of being, one is not, and is not one. But to that which
is not, there is no attribute or relative, neither name nor word nor
idea nor science nor perception nor opinion appertaining. One, then, is
neither named, nor uttered, nor known, nor perceived, nor imagined. But
can all this be true? 'I think not.'

1.b. Let us, however, commence the inquiry again. We have to work out
all the consequences which follow on the assumption that the one is. If
one is, one partakes of being, which is not the same with one; the words
'being' and 'one' have different meanings. Observe the consequence: In
the one of being or the being of one are two parts, being and one, which
form one whole. And each of the two parts is also a whole, and involves
the other, and may be further subdivided into one and being, and is
therefore not one but two; and thus one is never one, and in this way
the one, if it is, becomes many and infinite. Again, let us conceive
of a one which by an effort of abstraction we separate from being: will
this abstract one be one or many? You say one only; let us see. In the
first place, the being of one is other than one; and one and being,
if different, are so because they both partake of the nature of other,
which is therefore neither one nor being; and whether we take being
and other, or being and one, or one and other, in any case we have two
things which separately are called either, and together both. And both
are two and either of two is severally one, and if one be added to any
of the pairs, the sum is three; and two is an even number, three an odd;
and two units exist twice, and therefore there are twice two; and three
units exist thrice, and therefore there are thrice three, and taken
together they give twice three and thrice two: we have even numbers
multiplied into even, and odd into even, and even into odd numbers. But
if one is, and both odd and even numbers are implied in one, must not
every number exist? And number is infinite, and therefore existence must
be infinite, for all and every number partakes of being; therefore
being has the greatest number of parts, and every part, however great or
however small, is equally one. But can one be in many places and yet be
a whole? If not a whole it must be divided into parts and represented
by a number corresponding to the number of the parts. And if so, we were
wrong in saying that being has the greatest number of parts; for being
is coequal and coextensive with one, and has no more parts than one; and
so the abstract one broken up into parts by being is many and infinite.
But the parts are parts of a whole, and the whole is their containing
limit, and the one is therefore limited as well as infinite in number;
and that which is a whole has beginning, middle, and end, and a middle
is equidistant from the extremes; and one is therefore of a certain
figure, round or straight, or a combination of the two, and being a
whole includes all the parts which are the whole, and is therefore
self-contained. But then, again, the whole is not in the parts, whether
all or some. Not in all, because, if in all, also in one; for, if
wanting in any one, how in all?--not in some, because the greater would
then be contained in the less. But if not in all, nor in any, nor in
some, either nowhere or in other. And if nowhere, nothing; therefore in
other. The one as a whole, then, is in another, but regarded as a sum
of parts is in itself; and is, therefore, both in itself and in another.
This being the case, the one is at once both at rest and in motion: at
rest, because resting in itself; in motion, because it is ever in other.
And if there is truth in what has preceded, one is the same and not the
same with itself and other. For everything in relation to every other
thing is either the same with it or other; or if neither the same nor
other, then in the relation of part to a whole or whole to a part. But
one cannot be a part or whole in relation to one, nor other than
one; and is therefore the same with one. Yet this sameness is again
contradicted by one being in another place from itself which is in the
same place; this follows from one being in itself and in another; one,
therefore, is other than itself. But if anything is other than anything,
will it not be other than other? And the not one is other than the one,
and the one than the not one; therefore one is other than all others.
But the same and the other exclude one another, and therefore the other
can never be in the same; nor can the other be in anything for ever so
short a time, as for that time the other will be in the same. And the
other, if never in the same, cannot be either in the one or in the not
one. And one is not other than not one, either by reason of other or
of itself; and therefore they are not other than one another at all.
Neither can the not one partake or be part of one, for in that case it
would be one; nor can the not one be number, for that also involves one.
And therefore, not being other than the one or related to the one as
a whole to parts or parts to a whole, not one is the same as one.
Wherefore the one is the same and also not the same with the others
and also with itself; and is therefore like and unlike itself and the
others, and just as different from the others as they are from the one,
neither more nor less. But if neither more nor less, equally different;
and therefore the one and the others have the same relations. This may
be illustrated by the case of names: when you repeat the same name twice
over, you mean the same thing; and when you say that the other is other
than the one, or the one other than the other, this very word other
(eteron), which is attributed to both, implies sameness. One, then, as
being other than others, and other as being other than one, are alike in
that they have the relation of otherness; and likeness is similarity
of relations. And everything as being other of everything is also like
everything. Again, same and other, like and unlike, are opposites: and
since in virtue of being other than the others the one is like them, in
virtue of being the same it must be unlike. Again, one, as having the
same relations, has no difference of relation, and is therefore not
unlike, and therefore like; or, as having different relations, is
different and unlike. Thus, one, as being the same and not the same with
itself and others--for both these reasons and for either of them--is
also like and unlike itself and the others. Again, how far can one touch
itself and the others? As existing in others, it touches the others; and
as existing in itself, touches only itself. But from another point of
view, that which touches another must be next in order of place; one,
therefore, must be next in order of place to itself, and would therefore
be two, and in two places. But one cannot be two, and therefore cannot
be in contact with itself. Nor again can one touch the other. Two
objects are required to make one contact; three objects make two
contacts; and all the objects in the world, if placed in a series, would
have as many contacts as there are objects, less one. But if one only
exists, and not two, there is no contact. And the others, being other
than one, have no part in one, and therefore none in number, and
therefore two has no existence, and therefore there is no contact.
For all which reasons, one has and has not contact with itself and the
others.

Once more, Is one equal and unequal to itself and the others? Suppose
one and the others to be greater or less than each other or equal to one
another, they will be greater or less or equal by reason of equality or
greatness or smallness inhering in them in addition to their own proper
nature. Let us begin by assuming smallness to be inherent in one: in
this case the inherence is either in the whole or in a part. If the
first, smallness is either coextensive with the whole one, or contains
the whole, and, if coextensive with the one, is equal to the one, or
if containing the one will be greater than the one. But smallness thus
performs the function of equality or of greatness, which is impossible.
Again, if the inherence be in a part, the same contradiction follows:
smallness will be equal to the part or greater than the part; therefore
smallness will not inhere in anything, and except the idea of smallness
there will be nothing small. Neither will greatness; for greatness will
have a greater;--and there will be no small in relation to which it is
great. And there will be no great or small in objects, but greatness
and smallness will be relative only to each other; therefore the others
cannot be greater or less than the one; also the one can neither exceed
nor be exceeded by the others, and they are therefore equal to one
another. And this will be true also of the one in relation to itself:
one will be equal to itself as well as to the others (talla). Yet one,
being in itself, must also be about itself, containing and contained,
and is therefore greater and less than itself. Further, there is nothing
beside the one and the others; and as these must be in something, they
must therefore be in one another; and as that in which a thing is is
greater than the thing, the inference is that they are both greater and
less than one another, because containing and contained in one another.
Therefore the one is equal to and greater and less than itself or other,
having also measures or parts or numbers equal to or greater or less
than itself or other.

But does one partake of time? This must be acknowledged, if the one
partakes of being. For 'to be' is the participation of being in present
time, 'to have been' in past, 'to be about to be' in future time. And
as time is ever moving forward, the one becomes older than itself; and
therefore younger than itself; and is older and also younger when in the
process of becoming it arrives at the present; and it is always older
and younger, for at any moment the one is, and therefore it becomes
and is not older and younger than itself but during an equal time with
itself, and is therefore contemporary with itself.

And what are the relations of the one to the others? Is it or does it
become older or younger than they? At any rate the others are more than
one, and one, being the least of all numbers, must be prior in time to
greater numbers. But on the other hand, one must come into being in a
manner accordant with its own nature. Now one has parts or others, and
has therefore a beginning, middle, and end, of which the beginning is
first and the end last. And the parts come into existence first; last of
all the whole, contemporaneously with the end, being therefore younger,
while the parts or others are older than the one. But, again, the one
comes into being in each of the parts as much as in the whole, and must
be of the same age with them. Therefore one is at once older and younger
than the parts or others, and also contemporaneous with them, for no
part can be a part which is not one. Is this true of becoming as well as
being? Thus much may be affirmed, that the same things which are older
or younger cannot become older or younger in a greater degree than they
were at first by the addition of equal times. But, on the other hand,
the one, if older than others, has come into being a longer time than
they have. And when equal time is added to a longer and shorter, the
relative difference between them is diminished. In this way that which
was older becomes younger, and that which was younger becomes older,
that is to say, younger and older than at first; and they ever become
and never have become, for then they would be. Thus the one and others
always are and are becoming and not becoming younger and also older than
one another. And one, partaking of time and also partaking of becoming
older and younger, admits of all time, present, past, and future--was,
is, shall be--was becoming, is becoming, will become. And there is
science of the one, and opinion and name and expression, as is already
implied in the fact of our inquiry.

Yet once more, if one be one and many, and neither one nor many, and
also participant of time, must there not be a time at which one as being
one partakes of being, and a time when one as not being one is deprived
of being? But these two contradictory states cannot be experienced
by the one both together: there must be a time of transition. And the
transition is a process of generation and destruction, into and from
being and not-being, the one and the others. For the generation of the
one is the destruction of the others, and the generation of the others
is the destruction of the one. There is also separation and aggregation,
assimilation and dissimilation, increase, diminution, equalization,
a passage from motion to rest, and from rest to motion in the one and
many. But when do all these changes take place? When does motion become
rest, or rest motion? The answer to this question will throw a light
upon all the others. Nothing can be in motion and at rest at the same
time; and therefore the change takes place 'in a moment'--which is a
strange expression, and seems to mean change in no time. Which is true
also of all the other changes, which likewise take place in no time.

1.aa. But if one is, what happens to the others, which in the first
place are not one, yet may partake of one in a certain way? The others
are other than the one because they have parts, for if they had no parts
they would be simply one, and parts imply a whole to which they belong;
otherwise each part would be a part of many, and being itself one of
them, of itself, and if a part of all, of each one of the other parts,
which is absurd. For a part, if not a part of one, must be a part of
all but this one, and if so not a part of each one; and if not a part
of each one, not a part of any one of many, and so not of one; and if of
none, how of all? Therefore a part is neither a part of many nor of
all, but of an absolute and perfect whole or one. And if the others have
parts, they must partake of the whole, and must be the whole of which
they are the parts. And each part, as the word 'each' implies, is also
an absolute one. And both the whole and the parts partake of one, for
the whole of which the parts are parts is one, and each part is one part
of the whole; and whole and parts as participating in one are other
than one, and as being other than one are many and infinite; and however
small a fraction you separate from them is many and not one. Yet the
fact of their being parts furnishes the others with a limit towards
other parts and towards the whole; they are finite and also infinite:
finite through participation in the one, infinite in their own nature.
And as being finite, they are alike; and as being infinite, they are
alike; but as being both finite and also infinite, they are in the
highest degree unlike. And all other opposites might without difficulty
be shown to unite in them.

1.bb. Once more, leaving all this: Is there not also an opposite series
of consequences which is equally true of the others, and may be deduced
from the existence of one? There is. One is distinct from the others,
and the others from one; for one and the others are all things, and
there is no third existence besides them. And the whole of one cannot
be in others nor parts of it, for it is separated from others and has
no parts, and therefore the others have no unity, nor plurality, nor
duality, nor any other number, nor any opposition or distinction, such
as likeness and unlikeness, some and other, generation and corruption,
odd and even. For if they had these they would partake either of one
opposite, and this would be a participation in one; or of two opposites,
and this would be a participation in two. Thus if one exists, one is all
things, and likewise nothing, in relation to one and to the others.

2.a. But, again, assume the opposite hypothesis, that the one is not,
and what is the consequence? In the first place, the proposition, that
one is not, is clearly opposed to the proposition, that not one is not.
The subject of any negative proposition implies at once knowledge and
difference. Thus 'one' in the proposition--'The one is not,' must be
something known, or the words would be unintelligible; and again this
'one which is not' is something different from other things. Moreover,
this and that, some and other, may be all attributed or related to
the one which is not, and which though non-existent may and must have
plurality, if the one only is non-existent and nothing else; but if all
is not-being there is nothing which can be spoken of. Also the one which
is not differs, and is different in kind from the others, and therefore
unlike them; and they being other than the one, are unlike the one,
which is therefore unlike them. But one, being unlike other, must be
like itself; for the unlikeness of one to itself is the destruction of
the hypothesis; and one cannot be equal to the others; for that would
suppose being in the one, and the others would be equal to one and like
one; both which are impossible, if one does not exist. The one which
is not, then, if not equal is unequal to the others, and in equality
implies great and small, and equality lies between great and small, and
therefore the one which is not partakes of equality. Further, the one
which is not has being; for that which is true is, and it is true that
the one is not. And so the one which is not, if remitting aught of the
being of non-existence, would become existent. For not being implies the
being of not-being, and being the not-being of not-being; or more truly
being partakes of the being of being and not of the being of not-being,
and not-being of the being of not-being and not of the not-being
of not-being. And therefore the one which is not has being and also
not-being. And the union of being and not-being involves change or
motion. But how can not-being, which is nowhere, move or change, either
from one place to another or in the same place? And whether it is or is
not, it would cease to be one if experiencing a change of substance. The
one which is not, then, is both in motion and at rest, is altered and
unaltered, and becomes and is destroyed, and does not become and is not
destroyed.

2.b. Once more, let us ask the question, If one is not, what happens in
regard to one? The expression 'is not' implies negation of being:--do we
mean by this to say that a thing, which is not, in a certain sense is?
or do we mean absolutely to deny being of it? The latter. Then the one
which is not can neither be nor become nor perish nor experience change
of substance or place. Neither can rest, or motion, or greatness, or
smallness, or equality, or unlikeness, or likeness either to itself or
other, or attribute or relation, or now or hereafter or formerly, or
knowledge or opinion or perception or name or anything else be asserted
of that which is not.

2.aa. Once more, if one is not, what becomes of the others? If we
speak of them they must be, and their very name implies difference, and
difference implies relation, not to the one, which is not, but to
one another. And they are others of each other not as units but
as infinities, the least of which is also infinity, and capable of
infinitesimal division. And they will have no unity or number, but only
a semblance of unity and number; and the least of them will appear large
and manifold in comparison with the infinitesimal fractions into which
it may be divided. Further, each particle will have the appearance of
being equal with the fractions. For in passing from the greater to the
less it must reach an intermediate point, which is equality. Moreover,
each particle although having a limit in relation to itself and to other
particles, yet it has neither beginning, middle, nor end; for there is
always a beginning before the beginning, and a middle within the middle,
and an end beyond the end, because the infinitesimal division is never
arrested by the one. Thus all being is one at a distance, and broken
up when near, and like at a distance and unlike when near; and also the
particles which compose being seem to be like and unlike, in rest and
motion, in generation and corruption, in contact and separation, if one
is not.

2.bb. Once more, let us inquire, If the one is not, and the others of
the one are, what follows? In the first place, the others will not be
the one, nor the many, for in that case the one would be contained in
them; neither will they appear to be one or many; because they have no
communion or participation in that which is not, nor semblance of that
which is not. If one is not, the others neither are, nor appear to be
one or many, like or unlike, in contact or separation. In short, if one
is not, nothing is.

The result of all which is, that whether one is or is not, one and the
others, in relation to themselves and to one another, are and are not,
and appear to be and appear not to be, in all manner of ways.

I. On the first hypothesis we may remark: first, That one is one is
an identical proposition, from which we might expect that no further
consequences could be deduced. The train of consequences which follows,
is inferred by altering the predicate into 'not many.' Yet, perhaps, if
a strict Eristic had been present, oios aner ei kai nun paren, he might
have affirmed that the not many presented a different aspect of the
conception from the one, and was therefore not identical with it. Such
a subtlety would be very much in character with the Zenonian dialectic.
Secondly, We may note, that the conclusion is really involved in the
premises. For one is conceived as one, in a sense which excludes all
predicates. When the meaning of one has been reduced to a point, there
is no use in saying that it has neither parts nor magnitude. Thirdly,
The conception of the same is, first of all, identified with the one;
and then by a further analysis distinguished from, and even opposed to
it. Fourthly, We may detect notions, which have reappeared in modern
philosophy, e.g. the bare abstraction of undefined unity, answering to
the Hegelian 'Seyn,' or the identity of contradictions 'that which is
older is also younger,' etc., or the Kantian conception of an a priori
synthetical proposition 'one is.'

II. In the first series of propositions the word 'is' is really the
copula; in the second, the verb of existence. As in the first series,
the negative consequence followed from one being affirmed to be
equivalent to the not many; so here the affirmative consequence is
deduced from one being equivalent to the many.

In the former case, nothing could be predicated of the one, but now
everything--multitude, relation, place, time, transition. One is
regarded in all the aspects of one, and with a reference to all the
consequences which flow, either from the combination or the separation
of them. The notion of transition involves the singular extra-temporal
conception of 'suddenness.' This idea of 'suddenness' is based upon the
contradiction which is involved in supposing that anything can be in two
places at once. It is a mere fiction; and we may observe that similar
antinomies have led modern philosophers to deny the reality of time and
space. It is not the infinitesimal of time, but the negative of time.
By the help of this invention the conception of change, which sorely
exercised the minds of early thinkers, seems to be, but is not really
at all explained. The difficulty arises out of the imperfection of
language, and should therefore be no longer regarded as a difficulty at
all. The only way of meeting it, if it exists, is to acknowledge that
this rather puzzling double conception is necessary to the expression
of the phenomena of motion or change, and that this and similar double
notions, instead of being anomalies, are among the higher and more
potent instruments of human thought.

The processes by which Parmenides obtains his remarkable results may be
summed up as follows: (1) Compound or correlative ideas which involve
each other, such as, being and not-being, one and many, are conceived
sometimes in a state of composition, and sometimes of division: (2) The
division or distinction is sometimes heightened into total opposition,
e.g. between one and same, one and other: or (3) The idea, which has
been already divided, is regarded, like a number, as capable of further
infinite subdivision: (4) The argument often proceeds 'a dicto secundum
quid ad dictum simpliciter' and conversely: (5) The analogy of opposites
is misused by him; he argues indiscriminately sometimes from what is
like, sometimes from what is unlike in them: (6) The idea of being or
not-being is identified with existence or non-existence in place
or time: (7) The same ideas are regarded sometimes as in process of
transition, sometimes as alternatives or opposites: (8) There are no
degrees or kinds of sameness, likeness, difference, nor any adequate
conception of motion or change: (9) One, being, time, like space in
Zeno's puzzle of Achilles and the tortoise, are regarded sometimes as
continuous and sometimes as discrete: (10) In some parts of the argument
the abstraction is so rarefied as to become not only fallacious, but
almost unintelligible, e.g. in the contradiction which is elicited out
of the relative terms older and younger: (11) The relation between two
terms is regarded under contradictory aspects, as for example when
the existence of the one and the non-existence of the one are equally
assumed to involve the existence of the many: (12) Words are used
through long chains of argument, sometimes loosely, sometimes with the
precision of numbers or of geometrical figures.

The argument is a very curious piece of work, unique in literature.
It seems to be an exposition or rather a 'reductio ad absurdum' of the
Megarian philosophy, but we are too imperfectly acquainted with this
last to speak with confidence about it. It would be safer to say that it
is an indication of the sceptical, hyperlogical fancies which prevailed
among the contemporaries of Socrates. It throws an indistinct light upon
Aristotle, and makes us aware of the debt which the world owes to him or
his school. It also bears a resemblance to some modern speculations, in
which an attempt is made to narrow language in such a manner that number
and figure may be made a calculus of thought. It exaggerates one side
of logic and forgets the rest. It has the appearance of a mathematical
process; the inventor of it delights, as mathematicians do, in eliciting
or discovering an unexpected result. It also helps to guard us against
some fallacies by showing the consequences which flow from them.

In the Parmenides we seem to breathe the spirit of the Megarian
philosophy, though we cannot compare the two in detail. But Plato also
goes beyond his Megarian contemporaries; he has split their straws over
again, and admitted more than they would have desired. He is indulging
the analytical tendencies of his age, which can divide but not combine.
And he does not stop to inquire whether the distinctions which he makes
are shadowy and fallacious, but 'whither the argument blows' he follows.

III. The negative series of propositions contains the first conception
of the negation of a negation. Two minus signs in arithmetic or algebra
make a plus. Two negatives destroy each other. This abstruse notion is
the foundation of the Hegelian logic. The mind must not only admit
that determination is negation, but must get through negation into
affirmation. Whether this process is real, or in any way an assistance
to thought, or, like some other logical forms, a mere figure of speech
transferred from the sphere of mathematics, may be doubted. That Plato
and the most subtle philosopher of the nineteenth century should have
lighted upon the same notion, is a singular coincidence of ancient and
modern thought.

IV. The one and the many or others are reduced to their strictest
arithmetical meaning. That one is three or three one, is a proposition
which has, perhaps, given rise to more controversy in the world than
any other. But no one has ever meant to say that three and one are to be
taken in the same sense. Whereas the one and many of the Parmenides have
precisely the same meaning; there is no notion of one personality or
substance having many attributes or qualities. The truth seems to
be rather the opposite of that which Socrates implies: There is no
contradiction in the concrete, but in the abstract; and the more
abstract the idea, the more palpable will be the contradiction. For just
as nothing can persuade us that the number one is the number three, so
neither can we be persuaded that any abstract idea is identical with
its opposite, although they may both inhere together in some external
object, or some more comprehensive conception. Ideas, persons, things
may be one in one sense and many in another, and may have various
degrees of unity and plurality. But in whatever sense and in whatever
degree they are one they cease to be many; and in whatever degree or
sense they are many they cease to be one.

Two points remain to be considered: 1st, the connexion between the first
and second parts of the dialogue; 2ndly, the relation of the Parmenides
to the other dialogues.

I. In both divisions of the dialogue the principal speaker is the same,
and the method pursued by him is also the same, being a criticism on
received opinions: first, on the doctrine of Ideas; secondly, of Being.
From the Platonic Ideas we naturally proceed to the Eleatic One or Being
which is the foundation of them. They are the same philosophy in two
forms, and the simpler form is the truer and deeper. For the Platonic
Ideas are mere numerical differences, and the moment we attempt to
distinguish between them, their transcendental character is lost; ideas
of justice, temperance, and good, are really distinguishable only with
reference to their application in the world. If we once ask how they
are related to individuals or to the ideas of the divine mind, they are
again merged in the aboriginal notion of Being. No one can answer the
questions which Parmenides asks of Socrates. And yet these questions are
asked with the express acknowledgment that the denial of ideas will be
the destruction of the human mind. The true answer to the difficulty
here thrown out is the establishment of a rational psychology; and
this is a work which is commenced in the Sophist. Plato, in urging the
difficulty of his own doctrine of Ideas, is far from denying that some
doctrine of Ideas is necessary, and for this he is paving the way.

In a similar spirit he criticizes the Eleatic doctrine of Being, not
intending to deny Ontology, but showing that the old Eleatic notion,
and the very name 'Being,' is unable to maintain itself against the
subtleties of the Megarians. He did not mean to say that Being or
Substance had no existence, but he is preparing for the development
of his later view, that ideas were capable of relation. The fact that
contradictory consequences follow from the existence or non-existence
of one or many, does not prove that they have or have not existence,
but rather that some different mode of conceiving them is required.
Parmenides may still have thought that 'Being was,' just as Kant would
have asserted the existence of 'things in themselves,' while denying the
transcendental use of the Categories.

Several lesser links also connect the first and second parts of the
dialogue: (1) The thesis is the same as that which Zeno has been already
discussing: (2) Parmenides has intimated in the first part, that the
method of Zeno should, as Socrates desired, be extended to Ideas: (3)
The difficulty of participating in greatness, smallness, equality is
urged against the Ideas as well as against the One.

II. The Parmenides is not only a criticism of the Eleatic notion of
Being, but also of the methods of reasoning then in existence, and
in this point of view, as well as in the other, may be regarded as an
introduction to the Sophist. Long ago, in the Euthydemus, the vulgar
application of the 'both and neither' Eristic had been subjected to a
similar criticism, which there takes the form of banter and irony, here
of illustration.

The attack upon the Ideas is resumed in the Philebus, and is followed
by a return to a more rational philosophy. The perplexity of the One and
Many is there confined to the region of Ideas, and replaced by a theory
of classification; the Good arranged in classes is also contrasted with
the barren abstraction of the Megarians. The war is carried on against
the Eristics in all the later dialogues, sometimes with a playful irony,
at other times with a sort of contempt. But there is no lengthened
refutation of them. The Parmenides belongs to that stage of the
dialogues of Plato in which he is partially under their influence, using
them as a sort of 'critics or diviners' of the truth of his own, and of
the Eleatic theories. In the Theaetetus a similar negative dialectic
is employed in the attempt to define science, which after every effort
remains undefined still. The same question is revived from the objective
side in the Sophist: Being and Not-being are no longer exhibited in
opposition, but are now reconciled; and the true nature of Not-being is
discovered and made the basis of the correlation of ideas. Some
links are probably missing which might have been supplied if we had
trustworthy accounts of Plato's oral teaching.

To sum up: the Parmenides of Plato is a critique, first, of the Platonic
Ideas, and secondly, of the Eleatic doctrine of Being. Neither are
absolutely denied. But certain difficulties and consequences are shown
in the assumption of either, which prove that the Platonic as well as
the Eleatic doctrine must be remodelled. The negation and contradiction
which are involved in the conception of the One and Many are preliminary
to their final adjustment. The Platonic Ideas are tested by the
interrogative method of Socrates; the Eleatic One or Being is tried by
the severer and perhaps impossible method of hypothetical consequences,
negative and affirmative. In the latter we have an example of the
Zenonian or Megarian dialectic, which proceeded, not 'by assailing
premises, but conclusions'; this is worked out and improved by Plato.
When primary abstractions are used in every conceivable sense, any or
every conclusion may be deduced from them. The words 'one,' 'other,'
'being,' 'like,' 'same,' 'whole,' and their opposites, have slightly
different meanings, as they are applied to objects of thought or
objects of sense--to number, time, place, and to the higher ideas of
the reason;--and out of their different meanings this 'feast' of
contradictions 'has been provided.'

...

The Parmenides of Plato belongs to a stage of philosophy which has
passed away. At first we read it with a purely antiquarian or historical
interest; and with difficulty throw ourselves back into a state of
the human mind in which Unity and Being occupied the attention of
philosophers. We admire the precision of the language, in which, as in
some curious puzzle, each word is exactly fitted into every other,
and long trains of argument are carried out with a sort of geometrical
accuracy. We doubt whether any abstract notion could stand the searching
cross-examination of Parmenides; and may at last perhaps arrive at the
conclusion that Plato has been using an imaginary method to work out an
unmeaning conclusion. But the truth is, that he is carrying on a process
which is not either useless or unnecessary in any age of philosophy.
We fail to understand him, because we do not realize that the questions
which he is discussing could have had any value or importance. We
suppose them to be like the speculations of some of the Schoolmen,
which end in nothing. But in truth he is trying to get rid of the
stumbling-blocks of thought which beset his contemporaries. Seeing that
the Megarians and Cynics were making knowledge impossible, he takes
their 'catch-words' and analyzes them from every conceivable point of
view. He is criticizing the simplest and most general of our ideas, in
which, as they are the most comprehensive, the danger of error is the
most serious; for, if they remain unexamined, as in a mathematical
demonstration, all that flows from them is affected, and the error
pervades knowledge far and wide. In the beginning of philosophy this
correction of human ideas was even more necessary than in our own
times, because they were more bound up with words; and words when once
presented to the mind exercised a greater power over thought. There is
a natural realism which says, 'Can there be a word devoid of meaning, or
an idea which is an idea of nothing?' In modern times mankind have often
given too great importance to a word or idea. The philosophy of the
ancients was still more in slavery to them, because they had not the
experience of error, which would have placed them above the illusion.

The method of the Parmenides may be compared with the process of
purgation, which Bacon sought to introduce into philosophy. Plato is
warning us against two sorts of 'Idols of the Den': first, his own
Ideas, which he himself having created is unable to connect in any way
with the external world; secondly, against two idols in particular,
'Unity' and 'Being,' which had grown up in the pre-Socratic philosophy,
and were still standing in the way of all progress and development of
thought. He does not say with Bacon, 'Let us make truth by experiment,'
or 'From these vague and inexact notions let us turn to facts.' The time
has not yet arrived for a purely inductive philosophy. The instruments
of thought must first be forged, that they may be used hereafter by
modern inquirers. How, while mankind were disputing about universals,
could they classify phenomena? How could they investigate causes, when
they had not as yet learned to distinguish between a cause and an end?
How could they make any progress in the sciences without first arranging
them? These are the deficiencies which Plato is seeking to supply in an
age when knowledge was a shadow of a name only. In the earlier dialogues
the Socratic conception of universals is illustrated by his genius; in
the Phaedrus the nature of division is explained; in the Republic the
law of contradiction and the unity of knowledge are asserted; in the
later dialogues he is constantly engaged both with the theory and
practice of classification. These were the 'new weapons,' as he terms
them in the Philebus, which he was preparing for the use of some who, in
after ages, would be found ready enough to disown their obligations
to the great master, or rather, perhaps, would be incapable of
understanding them.

Numberless fallacies, as we are often truly told, have originated in a
confusion of the 'copula,' and the 'verb of existence.' Would not the
distinction which Plato by the mouth of Parmenides makes between 'One
is one' and 'One has being' have saved us from this and many similar
confusions? We see again that a long period in the history of philosophy
was a barren tract, not uncultivated, but unfruitful, because there
was no inquiry into the relation of language and thought, and the
metaphysical imagination was incapable of supplying the missing link
between words and things. The famous dispute between Nominalists and
Realists would never have been heard of, if, instead of transferring the
Platonic Ideas into a crude Latin phraseology, the spirit of Plato had
been truly understood and appreciated. Upon the term substance at least
two celebrated theological controversies appear to hinge, which would
not have existed, or at least not in their present form, if we had
'interrogated' the word substance, as Plato has the notions of Unity and
Being. These weeds of philosophy have struck their roots deep into
the soil, and are always tending to reappear, sometimes in new-fangled
forms; while similar words, such as development, evolution, law, and
the like, are constantly put in the place of facts, even by writers who
profess to base truth entirely upon fact. In an unmetaphysical age there
is probably more metaphysics in the common sense (i.e. more a
priori assumption) than in any other, because there is more complete
unconsciousness that we are resting on our own ideas, while we please
ourselves with the conviction that we are resting on facts. We do
not consider how much metaphysics are required to place us above
metaphysics, or how difficult it is to prevent the forms of expression
which are ready made for our use from outrunning actual observation and
experiment.

In the last century the educated world were astonished to find that the
whole fabric of their ideas was falling to pieces, because Hume amused
himself by analyzing the word 'cause' into uniform sequence. Then arose
a philosophy which, equally regardless of the history of the mind,
sought to save mankind from scepticism by assigning to our notions
of 'cause and effect,' 'substance and accident,' 'whole and part,'
a necessary place in human thought. Without them we could have
no experience, and therefore they were supposed to be prior to
experience--to be incrusted on the 'I'; although in the phraseology of
Kant there could be no transcendental use of them, or, in other words,
they were only applicable within the range of our knowledge. But into
the origin of these ideas, which he obtains partly by an analysis of the
proposition, partly by development of the 'ego,' he never inquires--they
seem to him to have a necessary existence; nor does he attempt to
analyse the various senses in which the word 'cause' or 'substance' may
be employed.

The philosophy of Berkeley could never have had any meaning, even
to himself, if he had first analyzed from every point of view the
conception of 'matter.' This poor forgotten word (which was 'a very good
word' to describe the simplest generalization of external objects) is
now superseded in the vocabulary of physical philosophers by 'force,'
which seems to be accepted without any rigid examination of its meaning,
as if the general idea of 'force' in our minds furnished an explanation
of the infinite variety of forces which exist in the universe. A similar
ambiguity occurs in the use of the favourite word 'law,' which is
sometimes regarded as a mere abstraction, and then elevated into a real
power or entity, almost taking the place of God. Theology, again, is
full of undefined terms which have distracted the human mind for ages.
Mankind have reasoned from them, but not to them; they have drawn out
the conclusions without proving the premises; they have asserted the
premises without examining the terms. The passions of religious parties
have been roused to the utmost about words of which they could have
given no explanation, and which had really no distinct meaning. One sort
of them, faith, grace, justification, have been the symbols of one
class of disputes; as the words substance, nature, person, of another,
revelation, inspiration, and the like, of a third. All of them have been
the subject of endless reasonings and inferences; but a spell has hung
over the minds of theologians or philosophers which has prevented them
from examining the words themselves. Either the effort to rise above
and beyond their own first ideas was too great for them, or there might,
perhaps, have seemed to be an irreverence in doing so. About the Divine
Being Himself, in whom all true theological ideas live and move, men
have spoken and reasoned much, and have fancied that they instinctively
know Him. But they hardly suspect that under the name of God even
Christians have included two characters or natures as much opposed as
the good and evil principle of the Persians.

To have the true use of words we must compare them with things; in using
them we acknowledge that they seldom give a perfect representation of
our meaning. In like manner when we interrogate our ideas we find that
we are not using them always in the sense which we supposed. And Plato,
while he criticizes the inconsistency of his own doctrine of universals
and draws out the endless consequences which flow from the assertion
either that 'Being is' or that 'Being is not,' by no means intends
to deny the existence of universals or the unity under which they
are comprehended. There is nothing further from his thoughts than
scepticism. But before proceeding he must examine the foundations which
he and others have been laying; there is nothing true which is not from
some point of view untrue, nothing absolute which is not also relative
(compare Republic).

And so, in modern times, because we are called upon to analyze our ideas
and to come to a distinct understanding about the meaning of words;
because we know that the powers of language are very unequal to the
subtlety of nature or of mind, we do not therefore renounce the use of
them; but we replace them in their old connexion, having first tested
their meaning and quality, and having corrected the error which is
involved in them; or rather always remembering to make allowance for
the adulteration or alloy which they contain. We cannot call a new
metaphysical world into existence any more than we can frame a new
universal language; in thought as in speech, we are dependent on the
past. We know that the words 'cause' and 'effect' are very far from
representing to us the continuity or the complexity of nature or the
different modes or degrees in which phenomena are connected. Yet we
accept them as the best expression which we have of the correlation of
forces or objects. We see that the term 'law' is a mere abstraction,
under which laws of matter and of mind, the law of nature and the law of
the land are included, and some of these uses of the word are confusing,
because they introduce into one sphere of thought associations
which belong to another; for example, order or sequence is apt to be
confounded with external compulsion and the internal workings of the
mind with their material antecedents. Yet none of them can be dispensed
with; we can only be on our guard against the error or confusion which
arises out of them. Thus in the use of the word 'substance' we are far
from supposing that there is any mysterious substratum apart from the
objects which we see, and we acknowledge that the negative notion is
very likely to become a positive one. Still we retain the word as a
convenient generalization, though not without a double sense, substance,
and essence, derived from the two-fold translation of the Greek ousia.

So the human mind makes the reflection that God is not a person like
ourselves--is not a cause like the material causes in nature, nor even
an intelligent cause like a human agent--nor an individual, for He is
universal; and that every possible conception which we can form of Him
is limited by the human faculties. We cannot by any effort of thought
or exertion of faith be in and out of our own minds at the same instant.
How can we conceive Him under the forms of time and space, who is out of
time and space? How get rid of such forms and see Him as He is? How
can we imagine His relation to the world or to ourselves? Innumerable
contradictions follow from either of the two alternatives, that God is
or that He is not. Yet we are far from saying that we know nothing of
Him, because all that we know is subject to the conditions of human
thought. To the old belief in Him we return, but with corrections. He is
a person, but not like ourselves; a mind, but not a human mind; a cause,
but not a material cause, nor yet a maker or artificer. The words which
we use are imperfect expressions of His true nature; but we do not
therefore lose faith in what is best and highest in ourselves and in the
world.

'A little philosophy takes us away from God; a great deal brings us back
to Him.' When we begin to reflect, our first thoughts respecting Him and
ourselves are apt to be sceptical. For we can analyze our religious as
well as our other ideas; we can trace their history; we can criticize
their perversion; we see that they are relative to the human mind and
to one another. But when we have carried our criticism to the furthest
point, they still remain, a necessity of our moral nature, better known
and understood by us, and less liable to be shaken, because we are more
aware of their necessary imperfection. They come to us with 'better
opinion, better confirmation,' not merely as the inspirations either of
ourselves or of another, but deeply rooted in history and in the human
mind.




PARMENIDES


PERSONS OF THE DIALOGUE: Cephalus, Adeimantus, Glaucon, Antiphon,
Pythodorus, Socrates, Zeno, Parmenides, Aristoteles.

Cephalus rehearses a dialogue which is supposed to have been narrated in
his presence by Antiphon, the half-brother of Adeimantus and Glaucon, to
certain Clazomenians.


We had come from our home at Clazomenae to Athens, and met Adeimantus
and Glaucon in the Agora. Welcome, Cephalus, said Adeimantus, taking me
by the hand; is there anything which we can do for you in Athens?

Yes; that is why I am here; I wish to ask a favour of you.

What may that be? he said.

I want you to tell me the name of your half brother, which I have
forgotten; he was a mere child when I last came hither from Clazomenae,
but that was a long time ago; his father's name, if I remember rightly,
was Pyrilampes?

Yes, he said, and the name of our brother, Antiphon; but why do you ask?

Let me introduce some countrymen of mine, I said; they are lovers of
philosophy, and have heard that Antiphon was intimate with a certain
Pythodorus, a friend of Zeno, and remembers a conversation which took
place between Socrates, Zeno, and Parmenides many years ago, Pythodorus
having often recited it to him.

Quite true.

And could we hear it? I asked.

Nothing easier, he replied; when he was a youth he made a careful study
of the piece; at present his thoughts run in another direction; like his
grandfather Antiphon he is devoted to horses. But, if that is what you
want, let us go and look for him; he dwells at Melita, which is quite
near, and he has only just left us to go home.

Accordingly we went to look for him; he was at home, and in the act
of giving a bridle to a smith to be fitted. When he had done with the
smith, his brothers told him the purpose of our visit; and he saluted me
as an acquaintance whom he remembered from my former visit, and we
asked him to repeat the dialogue. At first he was not very willing, and
complained of the trouble, but at length he consented. He told us that
Pythodorus had described to him the appearance of Parmenides and Zeno;
they came to Athens, as he said, at the great Panathenaea; the former
was, at the time of his visit, about 65 years old, very white with age,
but well favoured. Zeno was nearly 40 years of age, tall and fair to
look upon; in the days of his youth he was reported to have been
beloved by Parmenides. He said that they lodged with Pythodorus in the
Ceramicus, outside the wall, whither Socrates, then a very young man,
came to see them, and many others with him; they wanted to hear the
writings of Zeno, which had been brought to Athens for the first time
on the occasion of their visit. These Zeno himself read to them in the
absence of Parmenides, and had very nearly finished when Pythodorus
entered, and with him Parmenides and Aristoteles who was afterwards
one of the Thirty, and heard the little that remained of the dialogue.
Pythodorus had heard Zeno repeat them before.

When the recitation was completed, Socrates requested that the first
thesis of the first argument might be read over again, and this having
been done, he said: What is your meaning, Zeno? Do you maintain that
if being is many, it must be both like and unlike, and that this is
impossible, for neither can the like be unlike, nor the unlike like--is
that your position?

Just so, said Zeno.

And if the unlike cannot be like, or the like unlike, then according to
you, being could not be many; for this would involve an impossibility.
In all that you say have you any other purpose except to disprove the
being of the many? and is not each division of your treatise intended to
furnish a separate proof of this, there being in all as many proofs of
the not-being of the many as you have composed arguments? Is that your
meaning, or have I misunderstood you?

No, said Zeno; you have correctly understood my general purpose.

I see, Parmenides, said Socrates, that Zeno would like to be not only
one with you in friendship but your second self in his writings too; he
puts what you say in another way, and would fain make believe that he is
telling us something which is new. For you, in your poems, say The All
is one, and of this you adduce excellent proofs; and he on the other
hand says There is no many; and on behalf of this he offers overwhelming
evidence. You affirm unity, he denies plurality. And so you deceive the
world into believing that you are saying different things when really
you are saying much the same. This is a strain of art beyond the reach
of most of us.

Yes, Socrates, said Zeno. But although you are as keen as a Spartan
hound in pursuing the track, you do not fully apprehend the true motive
of the composition, which is not really such an artificial work as you
imagine; for what you speak of was an accident; there was no pretence of
a great purpose; nor any serious intention of deceiving the world.
The truth is, that these writings of mine were meant to protect the
arguments of Parmenides against those who make fun of him and seek to
show the many ridiculous and contradictory results which they suppose
to follow from the affirmation of the one. My answer is addressed to the
partisans of the many, whose attack I return with interest by retorting
upon them that their hypothesis of the being of many, if carried out,
appears to be still more ridiculous than the hypothesis of the being
of one. Zeal for my master led me to write the book in the days of
my youth, but some one stole the copy; and therefore I had no choice
whether it should be published or not; the motive, however, of writing,
was not the ambition of an elder man, but the pugnacity of a young one.
This you do not seem to see, Socrates; though in other respects, as I
was saying, your notion is a very just one.

I understand, said Socrates, and quite accept your account. But tell
me, Zeno, do you not further think that there is an idea of likeness
in itself, and another idea of unlikeness, which is the opposite of
likeness, and that in these two, you and I and all other things to
which we apply the term many, participate--things which participate
in likeness become in that degree and manner like; and so far as they
participate in unlikeness become in that degree unlike, or both like and
unlike in the degree in which they participate in both? And may not all
things partake of both opposites, and be both like and unlike, by reason
of this participation?--Where is the wonder? Now if a person could prove
the absolute like to become unlike, or the absolute unlike to become
like, that, in my opinion, would indeed be a wonder; but there is
nothing extraordinary, Zeno, in showing that the things which only
partake of likeness and unlikeness experience both. Nor, again, if a
person were to show that all is one by partaking of one, and at the same
time many by partaking of many, would that be very astonishing. But if
he were to show me that the absolute one was many, or the absolute
many one, I should be truly amazed. And so of all the rest: I should
be surprised to hear that the natures or ideas themselves had these
opposite qualities; but not if a person wanted to prove of me that I was
many and also one. When he wanted to show that I was many he would say
that I have a right and a left side, and a front and a back, and an
upper and a lower half, for I cannot deny that I partake of multitude;
when, on the other hand, he wants to prove that I am one, he will say,
that we who are here assembled are seven, and that I am one and partake
of the one. In both instances he proves his case. So again, if a person
shows that such things as wood, stones, and the like, being many are
also one, we admit that he shows the coexistence of the one and many,
but he does not show that the many are one or the one many; he
is uttering not a paradox but a truism. If however, as I just now
suggested, some one were to abstract simple notions of like, unlike,
one, many, rest, motion, and similar ideas, and then to show that these
admit of admixture and separation in themselves, I should be very much
astonished. This part of the argument appears to be treated by you,
Zeno, in a very spirited manner; but, as I was saying, I should be
far more amazed if any one found in the ideas themselves which are
apprehended by reason, the same puzzle and entanglement which you have
shown to exist in visible objects.

While Socrates was speaking, Pythodorus thought that Parmenides and Zeno
were not altogether pleased at the successive steps of the argument; but
still they gave the closest attention, and often looked at one another,
and smiled as if in admiration of him. When he had finished, Parmenides
expressed their feelings in the following words:--

Socrates, he said, I admire the bent of your mind towards philosophy;
tell me now, was this your own distinction between ideas in themselves
and the things which partake of them? and do you think that there is an
idea of likeness apart from the likeness which we possess, and of the
one and many, and of the other things which Zeno mentioned?

I think that there are such ideas, said Socrates.

Parmenides proceeded: And would you also make absolute ideas of the just
and the beautiful and the good, and of all that class?

Yes, he said, I should.

And would you make an idea of man apart from us and from all other human
creatures, or of fire and water?

I am often undecided, Parmenides, as to whether I ought to include them
or not.

And would you feel equally undecided, Socrates, about things of which
the mention may provoke a smile?--I mean such things as hair, mud, dirt,
or anything else which is vile and paltry; would you suppose that each
of these has an idea distinct from the actual objects with which we come
into contact, or not?

Certainly not, said Socrates; visible things like these are such as
they appear to us, and I am afraid that there would be an absurdity in
assuming any idea of them, although I sometimes get disturbed, and begin
to think that there is nothing without an idea; but then again, when I
have taken up this position, I run away, because I am afraid that I may
fall into a bottomless pit of nonsense, and perish; and so I return to
the ideas of which I was just now speaking, and occupy myself with them.

Yes, Socrates, said Parmenides; that is because you are still young; the
time will come, if I am not mistaken, when philosophy will have a firmer
grasp of you, and then you will not despise even the meanest things; at
your age, you are too much disposed to regard the opinions of men. But
I should like to know whether you mean that there are certain ideas of
which all other things partake, and from which they derive their names;
that similars, for example, become similar, because they partake of
similarity; and great things become great, because they partake of
greatness; and that just and beautiful things become just and beautiful,
because they partake of justice and beauty?

Yes, certainly, said Socrates that is my meaning.

Then each individual partakes either of the whole of the idea or else of
a part of the idea? Can there be any other mode of participation?

There cannot be, he said.

Then do you think that the whole idea is one, and yet, being one, is in
each one of the many?

Why not, Parmenides? said Socrates.

Because one and the same thing will exist as a whole at the same time
in many separate individuals, and will therefore be in a state of
separation from itself.

Nay, but the idea may be like the day which is one and the same in many
places at once, and yet continuous with itself; in this way each idea
may be one and the same in all at the same time.

I like your way, Socrates, of making one in many places at once. You
mean to say, that if I were to spread out a sail and cover a number of
men, there would be one whole including many--is not that your meaning?

I think so.

And would you say that the whole sail includes each man, or a part of it
only, and different parts different men?

The latter.

Then, Socrates, the ideas themselves will be divisible, and things which
participate in them will have a part of them only and not the whole idea
existing in each of them?

That seems to follow.

Then would you like to say, Socrates, that the one idea is really
divisible and yet remains one?

Certainly not, he said.

Suppose that you divide absolute greatness, and that of the many great
things, each one is great in virtue of a portion of greatness less than
absolute greatness--is that conceivable?

No.

Or will each equal thing, if possessing some small portion of equality
less than absolute equality, be equal to some other thing by virtue of
that portion only?

Impossible.

Or suppose one of us to have a portion of smallness; this is but a part
of the small, and therefore the absolutely small is greater; if the
absolutely small be greater, that to which the part of the small is
added will be smaller and not greater than before.

How absurd!

Then in what way, Socrates, will all things participate in the ideas, if
they are unable to participate in them either as parts or wholes?

Indeed, he said, you have asked a question which is not easily answered.

Well, said Parmenides, and what do you say of another question?

What question?

I imagine that the way in which you are led to assume one idea of each
kind is as follows:--You see a number of great objects, and when you
look at them there seems to you to be one and the same idea (or nature)
in them all; hence you conceive of greatness as one.

Very true, said Socrates.

And if you go on and allow your mind in like manner to embrace in one
view the idea of greatness and of great things which are not the idea,
and to compare them, will not another greatness arise, which will appear
to be the source of all these?

It would seem so.

Then another idea of greatness now comes into view over and above
absolute greatness, and the individuals which partake of it; and then
another, over and above all these, by virtue of which they will all
be great, and so each idea instead of being one will be infinitely
multiplied.

But may not the ideas, asked Socrates, be thoughts only, and have no
proper existence except in our minds, Parmenides? For in that case each
idea may still be one, and not experience this infinite multiplication.

And can there be individual thoughts which are thoughts of nothing?

Impossible, he said.

The thought must be of something?

Yes.

Of something which is or which is not?

Of something which is.

Must it not be of a single something, which the thought recognizes as
attaching to all, being a single form or nature?

Yes.

And will not the something which is apprehended as one and the same in
all, be an idea?

From that, again, there is no escape.

Then, said Parmenides, if you say that everything else participates
in the ideas, must you not say either that everything is made up of
thoughts, and that all things think; or that they are thoughts but have
no thought?

The latter view, Parmenides, is no more rational than the previous one.
In my opinion, the ideas are, as it were, patterns fixed in nature, and
other things are like them, and resemblances of them--what is meant by
the participation of other things in the ideas, is really assimilation
to them.

But if, said he, the individual is like the idea, must not the idea also
be like the individual, in so far as the individual is a resemblance of
the idea? That which is like, cannot be conceived of as other than the
like of like.

Impossible.

And when two things are alike, must they not partake of the same idea?

They must.

And will not that of which the two partake, and which makes them alike,
be the idea itself?

Certainly.

Then the idea cannot be like the individual, or the individual like the
idea; for if they are alike, some further idea of likeness will always
be coming to light, and if that be like anything else, another; and new
ideas will be always arising, if the idea resembles that which partakes
of it?

Quite true.

The theory, then, that other things participate in the ideas by
resemblance, has to be given up, and some other mode of participation
devised?

It would seem so.

Do you see then, Socrates, how great is the difficulty of affirming the
ideas to be absolute?

Yes, indeed.

And, further, let me say that as yet you only understand a small part
of the difficulty which is involved if you make of each thing a single
idea, parting it off from other things.

What difficulty? he said.

There are many, but the greatest of all is this:--If an opponent argues
that these ideas, being such as we say they ought to be, must remain
unknown, no one can prove to him that he is wrong, unless he who denies
their existence be a man of great ability and knowledge, and is
willing to follow a long and laborious demonstration; he will remain
unconvinced, and still insist that they cannot be known.

What do you mean, Parmenides? said Socrates.

In the first place, I think, Socrates, that you, or any one who
maintains the existence of absolute essences, will admit that they
cannot exist in us.

No, said Socrates; for then they would be no longer absolute.

True, he said; and therefore when ideas are what they are in relation to
one another, their essence is determined by a relation among themselves,
and has nothing to do with the resemblances, or whatever they are to be
termed, which are in our sphere, and from which we receive this or that
name when we partake of them. And the things which are within our sphere
and have the same names with them, are likewise only relative to one
another, and not to the ideas which have the same names with them, but
belong to themselves and not to them.

What do you mean? said Socrates.

I may illustrate my meaning in this way, said Parmenides:--A master has
a slave; now there is nothing absolute in the relation between them,
which is simply a relation of one man to another. But there is also an
idea of mastership in the abstract, which is relative to the idea of
slavery in the abstract. These natures have nothing to do with us,
nor we with them; they are concerned with themselves only, and we with
ourselves. Do you see my meaning?

Yes, said Socrates, I quite see your meaning.

And will not knowledge--I mean absolute knowledge--answer to absolute
truth?

Certainly.

And each kind of absolute knowledge will answer to each kind of absolute
being?

Yes.

But the knowledge which we have, will answer to the truth which we have;
and again, each kind of knowledge which we have, will be a knowledge of
each kind of being which we have?

Certainly.

But the ideas themselves, as you admit, we have not, and cannot have?

No, we cannot.

And the absolute natures or kinds are known severally by the absolute
idea of knowledge?

Yes.

And we have not got the idea of knowledge?

No.

Then none of the ideas are known to us, because we have no share in
absolute knowledge?

I suppose not.

Then the nature of the beautiful in itself, and of the good in itself,
and all other ideas which we suppose to exist absolutely, are unknown to
us?

It would seem so.

I think that there is a stranger consequence still.

What is it?

Would you, or would you not say, that absolute knowledge, if there is
such a thing, must be a far more exact knowledge than our knowledge; and
the same of beauty and of the rest?

Yes.

And if there be such a thing as participation in absolute knowledge, no
one is more likely than God to have this most exact knowledge?

Certainly.

But then, will God, having absolute knowledge, have a knowledge of human
things?

Why not?

Because, Socrates, said Parmenides, we have admitted that the ideas are
not valid in relation to human things; nor human things in relation to
them; the relations of either are limited to their respective spheres.

Yes, that has been admitted.

And if God has this perfect authority, and perfect knowledge, his
authority cannot rule us, nor his knowledge know us, or any human thing;
just as our authority does not extend to the gods, nor our knowledge
know anything which is divine, so by parity of reason they, being gods,
are not our masters, neither do they know the things of men.

Yet, surely, said Socrates, to deprive God of knowledge is monstrous.

These, Socrates, said Parmenides, are a few, and only a few of the
difficulties in which we are involved if ideas really are and we
determine each one of them to be an absolute unity. He who hears what
may be said against them will deny the very existence of them--and even
if they do exist, he will say that they must of necessity be unknown
to man; and he will seem to have reason on his side, and as we were
remarking just now, will be very difficult to convince; a man must
be gifted with very considerable ability before he can learn that
everything has a class and an absolute essence; and still more
remarkable will he be who discovers all these things for himself, and
having thoroughly investigated them is able to teach them to others.

I agree with you, Parmenides, said Socrates; and what you say is very
much to my mind.

And yet, Socrates, said Parmenides, if a man, fixing his attention on
these and the like difficulties, does away with ideas of things and will
not admit that every individual thing has its own determinate idea which
is always one and the same, he will have nothing on which his mind can
rest; and so he will utterly destroy the power of reasoning, as you seem
to me to have particularly noted.

Very true, he said.

But, then, what is to become of philosophy? Whither shall we turn, if
the ideas are unknown?

I certainly do not see my way at present.

Yes, said Parmenides; and I think that this arises, Socrates, out of
your attempting to define the beautiful, the just, the good, and the
ideas generally, without sufficient previous training. I noticed your
deficiency, when I heard you talking here with your friend Aristoteles,
the day before yesterday. The impulse that carries you towards
philosophy is assuredly noble and divine; but there is an art which is
called by the vulgar idle talking, and which is often imagined to be
useless; in that you must train and exercise yourself, now that you are
young, or truth will elude your grasp.

And what is the nature of this exercise, Parmenides, which you would
recommend?

That which you heard Zeno practising; at the same time, I give you
credit for saying to him that you did not care to examine the perplexity
in reference to visible things, or to consider the question that way;
but only in reference to objects of thought, and to what may be called
ideas.

Why, yes, he said, there appears to me to be no difficulty in showing by
this method that visible things are like and unlike and may experience
anything.

Quite true, said Parmenides; but I think that you should go a step
further, and consider not only the consequences which flow from a
given hypothesis, but also the consequences which flow from denying the
hypothesis; and that will be still better training for you.

What do you mean? he said.

I mean, for example, that in the case of this very hypothesis of
Zeno's about the many, you should inquire not only what will be the
consequences to the many in relation to themselves and to the one, and
to the one in relation to itself and the many, on the hypothesis of the
being of the many, but also what will be the consequences to the one
and the many in their relation to themselves and to each other, on the
opposite hypothesis. Or, again, if likeness is or is not, what will
be the consequences in either of these cases to the subjects of the
hypothesis, and to other things, in relation both to themselves and to
one another, and so of unlikeness; and the same holds good of motion and
rest, of generation and destruction, and even of being and not-being.
In a word, when you suppose anything to be or not to be, or to be in any
way affected, you must look at the consequences in relation to the
thing itself, and to any other things which you choose,--to each of them
singly, to more than one, and to all; and so of other things, you must
look at them in relation to themselves and to anything else which you
suppose either to be or not to be, if you would train yourself perfectly
and see the real truth.

That, Parmenides, is a tremendous business of which you speak, and I do
not quite understand you; will you take some hypothesis and go through
the steps?--then I shall apprehend you better.

That, Socrates, is a serious task to impose on a man of my years.

Then will you, Zeno? said Socrates.

Zeno answered with a smile:--Let us make our petition to Parmenides
himself, who is quite right in saying that you are hardly aware of the
extent of the task which you are imposing on him; and if there were more
of us I should not ask him, for these are not subjects which any one,
especially at his age, can well speak of before a large audience; most
people are not aware that this roundabout progress through all things
is the only way in which the mind can attain truth and wisdom. And
therefore, Parmenides, I join in the request of Socrates, that I may
hear the process again which I have not heard for a long time.

When Zeno had thus spoken, Pythodorus, according to Antiphon's report
of him, said, that he himself and Aristoteles and the whole company
entreated Parmenides to give an example of the process. I cannot refuse,
said Parmenides; and yet I feel rather like Ibycus, who, when in his
old age, against his will, he fell in love, compared himself to an old
racehorse, who was about to run in a chariot race, shaking with fear at
the course he knew so well--this was his simile of himself. And I also
experience a trembling when I remember through what an ocean of words
I have to wade at my time of life. But I must indulge you, as Zeno says
that I ought, and we are alone. Where shall I begin? And what shall be
our first hypothesis, if I am to attempt this laborious pastime? Shall I
begin with myself, and take my own hypothesis the one? and consider the
consequences which follow on the supposition either of the being or of
the not-being of one?

By all means, said Zeno.

And who will answer me? he said. Shall I propose the youngest? He will
not make difficulties and will be the most likely to say what he thinks;
and his answers will give me time to breathe.

I am the one whom you mean, Parmenides, said Aristoteles; for I am the
youngest and at your service. Ask, and I will answer.

Parmenides proceeded: 1.a. If one is, he said, the one cannot be many?

Impossible.

Then the one cannot have parts, and cannot be a whole?

Why not?

Because every part is part of a whole; is it not?

Yes.

And what is a whole? would not that of which no part is wanting be a
whole?

Certainly.

Then, in either case, the one would be made up of parts; both as being a
whole, and also as having parts?

To be sure.

And in either case, the one would be many, and not one?

True.

But, surely, it ought to be one and not many?

It ought.

Then, if the one is to remain one, it will not be a whole, and will not
have parts?

No.

But if it has no parts, it will have neither beginning, middle, nor end;
for these would of course be parts of it.

Right.

But then, again, a beginning and an end are the limits of everything?

Certainly.

Then the one, having neither beginning nor end, is unlimited?

Yes, unlimited.

And therefore formless; for it cannot partake either of round or
straight.

But why?

Why, because the round is that of which all the extreme points are
equidistant from the centre?

Yes.

And the straight is that of which the centre intercepts the view of the
extremes?

True.

Then the one would have parts and would be many, if it partook either of
a straight or of a circular form?

Assuredly.

But having no parts, it will be neither straight nor round?

Right.

And, being of such a nature, it cannot be in any place, for it cannot be
either in another or in itself.

How so?

Because if it were in another, it would be encircled by that in which
it was, and would touch it at many places and with many parts; but that
which is one and indivisible, and does not partake of a circular nature,
cannot be touched all round in many places.

Certainly not.

But if, on the other hand, one were in itself, it would also be
contained by nothing else but itself; that is to say, if it were really
in itself; for nothing can be in anything which does not contain it.

Impossible.

But then, that which contains must be other than that which is
contained? for the same whole cannot do and suffer both at once; and if
so, one will be no longer one, but two?

True.

Then one cannot be anywhere, either in itself or in another?

No.

Further consider, whether that which is of such a nature can have either
rest or motion.

Why not?

Why, because the one, if it were moved, would be either moved in place
or changed in nature; for these are the only kinds of motion.

Yes.

And the one, when it changes and ceases to be itself, cannot be any
longer one.

It cannot.

It cannot therefore experience the sort of motion which is change of
nature?

Clearly not.

Then can the motion of the one be in place?

Perhaps.

But if the one moved in place, must it not either move round and round
in the same place, or from one place to another?

It must.

And that which moves in a circle must rest upon a centre; and that which
goes round upon a centre must have parts which are different from the
centre; but that which has no centre and no parts cannot possibly be
carried round upon a centre?

Impossible.

But perhaps the motion of the one consists in change of place?

Perhaps so, if it moves at all.

And have we not already shown that it cannot be in anything?

Yes.

Then its coming into being in anything is still more impossible; is it
not?

I do not see why.

Why, because anything which comes into being in anything, can neither
as yet be in that other thing while still coming into being, nor be
altogether out of it, if already coming into being in it.

Certainly not.

And therefore whatever comes into being in another must have parts, and
then one part may be in, and another part out of that other; but that
which has no parts can never be at one and the same time neither wholly
within nor wholly without anything.

True.

And is there not a still greater impossibility in that which has no
parts, and is not a whole, coming into being anywhere, since it cannot
come into being either as a part or as a whole?

Clearly.

Then it does not change place by revolving in the same spot, nor by
going somewhere and coming into being in something; nor again, by change
in itself?

Very true.

Then in respect of any kind of motion the one is immoveable?

Immoveable.

But neither can the one be in anything, as we affirm?

Yes, we said so.

Then it is never in the same?

Why not?

Because if it were in the same it would be in something.

Certainly.

And we said that it could not be in itself, and could not be in other?

True.

Then one is never in the same place?

It would seem not.

But that which is never in the same place is never quiet or at rest?

Never.

One then, as would seem, is neither at rest nor in motion?

It certainly appears so.

Neither will it be the same with itself or other; nor again, other than
itself or other.

How is that?

If other than itself it would be other than one, and would not be one.

True.

And if the same with other, it would be that other, and not itself; so
that upon this supposition too, it would not have the nature of one, but
would be other than one?

It would.

Then it will not be the same with other, or other than itself?

It will not.

Neither will it be other than other, while it remains one; for not one,
but only other, can be other than other, and nothing else.

True.

Then not by virtue of being one will it be other?

Certainly not.

But if not by virtue of being one, not by virtue of itself; and if not
by virtue of itself, not itself, and itself not being other at all, will
not be other than anything?

Right.

Neither will one be the same with itself.

How not?

Surely the nature of the one is not the nature of the same.

Why not?

It is not when anything becomes the same with anything that it becomes
one.

What of that?

Anything which becomes the same with the many, necessarily becomes many
and not one.

True.

But, if there were no difference between the one and the same, when a
thing became the same, it would always become one; and when it became
one, the same?

Certainly.

And, therefore, if one be the same with itself, it is not one with
itself, and will therefore be one and also not one.

Surely that is impossible.

And therefore the one can neither be other than other, nor the same with
itself.

Impossible.

And thus the one can neither be the same, nor other, either in relation
to itself or other?

No.

Neither will the one be like anything or unlike itself or other.

Why not?

Because likeness is sameness of affections.

Yes.

And sameness has been shown to be of a nature distinct from oneness?

That has been shown.

But if the one had any other affection than that of being one, it would
be affected in such a way as to be more than one; which is impossible.

True.

Then the one can never be so affected as to be the same either with
another or with itself?

Clearly not.

Then it cannot be like another, or like itself?

No.

Nor can it be affected so as to be other, for then it would be affected
in such a way as to be more than one.

It would.

That which is affected otherwise than itself or another, will be unlike
itself or another, for sameness of affections is likeness.

True.

But the one, as appears, never being affected otherwise, is never unlike
itself or other?

Never.

Then the one will never be either like or unlike itself or other?

Plainly not.

Again, being of this nature, it can neither be equal nor unequal either
to itself or to other.

How is that?

Why, because the one if equal must be of the same measures as that to
which it is equal.

True.

And if greater or less than things which are commensurable with it, the
one will have more measures than that which is less, and fewer than that
which is greater?

Yes.

And so of things which are not commensurate with it, the one will have
greater measures than that which is less and smaller than that which is
greater.

Certainly.

But how can that which does not partake of sameness, have either the
same measures or have anything else the same?

Impossible.

And not having the same measures, the one cannot be equal either with
itself or with another?

It appears so.

But again, whether it have fewer or more measures, it will have as many
parts as it has measures; and thus again the one will be no longer one
but will have as many parts as measures.

Right.

And if it were of one measure, it would be equal to that measure; yet it
has been shown to be incapable of equality.

It has.

Then it will neither partake of one measure, nor of many, nor of few,
nor of the same at all, nor be equal to itself or another; nor be
greater or less than itself, or other?

Certainly.

Well, and do we suppose that one can be older, or younger than anything,
or of the same age with it?

Why not?

Why, because that which is of the same age with itself or other, must
partake of equality or likeness of time; and we said that the one did
not partake either of equality or of likeness?

We did say so.

And we also said, that it did not partake of inequality or unlikeness.

Very true.

How then can one, being of this nature, be either older or younger than
anything, or have the same age with it?

In no way.

Then one cannot be older or younger, or of the same age, either with
itself or with another?

Clearly not.

Then the one, being of this nature, cannot be in time at all; for must
not that which is in time, be always growing older than itself?

Certainly.

And that which is older, must always be older than something which is
younger?

True.

Then, that which becomes older than itself, also becomes at the same
time younger than itself, if it is to have something to become older
than.

What do you mean?

I mean this:--A thing does not need to become different from another
thing which is already different; it IS different, and if its different
has become, it has become different; if its different will be, it will
be different; but of that which is becoming different, there cannot
have been, or be about to be, or yet be, a different--the only different
possible is one which is becoming.

That is inevitable.

But, surely, the elder is a difference relative to the younger, and to
nothing else.

True.

Then that which becomes older than itself must also, at the same time,
become younger than itself?

Yes.

But again, it is true that it cannot become for a longer or for a
shorter time than itself, but it must become, and be, and have become,
and be about to be, for the same time with itself?

That again is inevitable.

Then things which are in time, and partake of time, must in every case,
I suppose, be of the same age with themselves; and must also become at
once older and younger than themselves?

Yes.

But the one did not partake of those affections?

Not at all.

Then it does not partake of time, and is not in any time?

So the argument shows.

Well, but do not the expressions 'was,' and 'has become,' and 'was
becoming,' signify a participation of past time?

Certainly.

And do not 'will be,' 'will become,' 'will have become,' signify a
participation of future time?

Yes.

And 'is,' or 'becomes,' signifies a participation of present time?

Certainly.

And if the one is absolutely without participation in time, it never
had become, or was becoming, or was at any time, or is now become or
is becoming, or is, or will become, or will have become, or will be,
hereafter.

Most true.

But are there any modes of partaking of being other than these?

There are none.

Then the one cannot possibly partake of being?

That is the inference.

Then the one is not at all?

Clearly not.

Then the one does not exist in such way as to be one; for if it were
and partook of being, it would already be; but if the argument is to be
trusted, the one neither is nor is one?

True.

But that which is not admits of no attribute or relation?

Of course not.

Then there is no name, nor expression, nor perception, nor opinion, nor
knowledge of it?

Clearly not.

Then it is neither named, nor expressed, nor opined, nor known, nor does
anything that is perceive it.

So we must infer.

But can all this be true about the one?

I think not.

1.b. Suppose, now, that we return once more to the original hypothesis;
let us see whether, on a further review, any new aspect of the question
appears.

I shall be very happy to do so.

We say that we have to work out together all the consequences, whatever
they may be, which follow, if the one is?

Yes.

Then we will begin at the beginning:--If one is, can one be, and not
partake of being?

Impossible.

Then the one will have being, but its being will not be the same with
the one; for if the same, it would not be the being of the one; nor
would the one have participated in being, for the proposition that one
is would have been identical with the proposition that one is one;
but our hypothesis is not if one is one, what will follow, but if one
is:--am I not right?

Quite right.

We mean to say, that being has not the same significance as one?

Of course.

And when we put them together shortly, and say 'One is,' that is
equivalent to saying, 'partakes of being'?

Quite true.

Once more then let us ask, if one is what will follow. Does not this
hypothesis necessarily imply that one is of such a nature as to have
parts?

How so?

In this way:--If being is predicated of the one, if the one is, and one
of being, if being is one; and if being and one are not the same; and
since the one, which we have assumed, is, must not the whole, if it is
one, itself be, and have for its parts, one and being?

Certainly.

And is each of these parts--one and being--to be simply called a part,
or must the word 'part' be relative to the word 'whole'?

The latter.

Then that which is one is both a whole and has a part?

Certainly.

Again, of the parts of the one, if it is--I mean being and one--does
either fail to imply the other? is the one wanting to being, or being to
the one?

Impossible.

Thus, each of the parts also has in turn both one and being, and is at
the least made up of two parts; and the same principle goes on for ever,
and every part whatever has always these two parts; for being always
involves one, and one being; so that one is always disappearing, and
becoming two.

Certainly.

And so the one, if it is, must be infinite in multiplicity?

Clearly.

Let us take another direction.

What direction?

We say that the one partakes of being and therefore it is?

Yes.

And in this way, the one, if it has being, has turned out to be many?

True.

But now, let us abstract the one which, as we say, partakes of
being, and try to imagine it apart from that of which, as we say, it
partakes--will this abstract one be one only or many?

One, I think.

Let us see:--Must not the being of one be other than one? for the one is
not being, but, considered as one, only partook of being?

Certainly.

If being and the one be two different things, it is not because the one
is one that it is other than being; nor because being is being that it
is other than the one; but they differ from one another in virtue of
otherness and difference.

Certainly.

So that the other is not the same--either with the one or with being?

Certainly not.

And therefore whether we take being and the other, or being and the one,
or the one and the other, in every such case we take two things, which
may be rightly called both.

How so.

In this way--you may speak of being?

Yes.

And also of one?

Yes.

Then now we have spoken of either of them?

Yes.

Well, and when I speak of being and one, I speak of them both?

Certainly.

And if I speak of being and the other, or of the one and the other,--in
any such case do I not speak of both?

Yes.

And must not that which is correctly called both, be also two?

Undoubtedly.

And of two things how can either by any possibility not be one?

It cannot.

Then, if the individuals of the pair are together two, they must be
severally one?

Clearly.

And if each of them is one, then by the addition of any one to any pair,
the whole becomes three?

Yes.

And three are odd, and two are even?

Of course.

And if there are two there must also be twice, and if there are three
there must be thrice; that is, if twice one makes two, and thrice one
three?

Certainly.

There are two, and twice, and therefore there must be twice two; and
there are three, and there is thrice, and therefore there must be thrice
three?

Of course.

If there are three and twice, there is twice three; and if there are two
and thrice, there is thrice two?

Undoubtedly.

Here, then, we have even taken even times, and odd taken odd times, and
even taken odd times, and odd taken even times.

True.

And if this is so, does any number remain which has no necessity to be?

None whatever.

Then if one is, number must also be?

It must.

But if there is number, there must also be many, and infinite
multiplicity of being; for number is infinite in multiplicity, and
partakes also of being: am I not right?

Certainly.

And if all number participates in being, every part of number will also
participate?

Yes.

Then being is distributed over the whole multitude of things, and
nothing that is, however small or however great, is devoid of it? And,
indeed, the very supposition of this is absurd, for how can that which
is, be devoid of being?

In no way.

And it is divided into the greatest and into the smallest, and into
being of all sizes, and is broken up more than all things; the divisions
of it have no limit.

True.

Then it has the greatest number of parts?

Yes, the greatest number.

Is there any of these which is a part of being, and yet no part?

Impossible.

But if it is at all and so long as it is, it must be one, and cannot be
none?

Certainly.

Then the one attaches to every single part of being, and does not fail
in any part, whether great or small, or whatever may be the size of it?

True.

But reflect:--Can one, in its entirety, be in many places at the same
time?

No; I see the impossibility of that.

And if not in its entirety, then it is divided; for it cannot be present
with all the parts of being, unless divided.

True.

And that which has parts will be as many as the parts are?

Certainly.

Then we were wrong in saying just now, that being was distributed into
the greatest number of parts. For it is not distributed into parts more
than the one, into parts equal to the one; the one is never wanting
to being, or being to the one, but being two they are co-equal and
co-extensive.

Certainly that is true.

The one itself, then, having been broken up into parts by being, is many
and infinite?

True.

Then not only the one which has being is many, but the one itself
distributed by being, must also be many?

Certainly.

Further, inasmuch as the parts are parts of a whole, the one, as a
whole, will be limited; for are not the parts contained by the whole?

Certainly.

And that which contains, is a limit?

Of course.

Then the one if it has being is one and many, whole and parts, having
limits and yet unlimited in number?

Clearly.

And because having limits, also having extremes?

Certainly.

And if a whole, having beginning and middle and end. For can anything
be a whole without these three? And if any one of them is wanting to
anything, will that any longer be a whole?

No.

Then the one, as appears, will have beginning, middle, and end.

It will.

But, again, the middle will be equidistant from the extremes; or it
would not be in the middle?

Yes.

Then the one will partake of figure, either rectilinear or round, or a
union of the two?

True.

And if this is the case, it will be both in itself and in another too.

How?

Every part is in the whole, and none is outside the whole.

True.

And all the parts are contained by the whole?

Yes.

And the one is all its parts, and neither more nor less than all?

No.

And the one is the whole?

Of course.

But if all the parts are in the whole, and the one is all of them and
the whole, and they are all contained by the whole, the one will be
contained by the one; and thus the one will be in itself.

That is true.

But then, again, the whole is not in the parts--neither in all the
parts, nor in some one of them. For if it is in all, it must be in one;
for if there were any one in which it was not, it could not be in all
the parts; for the part in which it is wanting is one of all, and if the
whole is not in this, how can it be in them all?

It cannot.

Nor can the whole be in some of the parts; for if the whole were in some
of the parts, the greater would be in the less, which is impossible.

Yes, impossible.

But if the whole is neither in one, nor in more than one, nor in all of
the parts, it must be in something else, or cease to be anywhere at all?

Certainly.

If it were nowhere, it would be nothing; but being a whole, and not
being in itself, it must be in another.

Very true.

The one then, regarded as a whole, is in another, but regarded as being
all its parts, is in itself; and therefore the one must be itself in
itself and also in another.

Certainly.

The one then, being of this nature, is of necessity both at rest and in
motion?

How?

The one is at rest since it is in itself, for being in one, and not
passing out of this, it is in the same, which is itself.

True.

And that which is ever in the same, must be ever at rest?

Certainly.

Well, and must not that, on the contrary, which is ever in other, never
be in the same; and if never in the same, never at rest, and if not at
rest, in motion?

True.

Then the one being always itself in itself and other, must always be
both at rest and in motion?

Clearly.

And must be the same with itself, and other than itself; and also the
same with the others, and other than the others; this follows from its
previous affections.

How so?

Everything in relation to every other thing, is either the same or
other; or if neither the same nor other, then in the relation of a part
to a whole, or of a whole to a part.

Clearly.

And is the one a part of itself?

Certainly not.

Since it is not a part in relation to itself it cannot be related to
itself as whole to part?

It cannot.

But is the one other than one?

No.

And therefore not other than itself?

Certainly not.

If then it be neither other, nor a whole, nor a part in relation to
itself, must it not be the same with itself?

Certainly.

But then, again, a thing which is in another place from 'itself,' if
this 'itself' remains in the same place with itself, must be other than
'itself,' for it will be in another place?

True.

Then the one has been shown to be at once in itself and in another?

Yes.

Thus, then, as appears, the one will be other than itself?

True.

Well, then, if anything be other than anything, will it not be other
than that which is other?

Certainly.

And will not all things that are not one, be other than the one, and the
one other than the not-one?

Of course.

Then the one will be other than the others?

True.

But, consider:--Are not the absolute same, and the absolute other,
opposites to one another?

Of course.

Then will the same ever be in the other, or the other in the same?

They will not.

If then the other is never in the same, there is nothing in which
the other is during any space of time; for during that space of time,
however small, the other would be in the same. Is not that true?

Yes.

And since the other is never in the same, it can never be in anything
that is.

True.

Then the other will never be either in the not-one, or in the one?

Certainly not.

Then not by reason of otherness is the one other than the not-one, or
the not-one other than the one.

No.

Nor by reason of themselves will they be other than one another, if not
partaking of the other.

How can they be?

But if they are not other, either by reason of themselves or of the
other, will they not altogether escape being other than one another?

They will.

Again, the not-one cannot partake of the one; otherwise it would not
have been not-one, but would have been in some way one.

True.

Nor can the not-one be number; for having number, it would not have been
not-one at all.

It would not.

Again, is the not-one part of the one; or rather, would it not in that
case partake of the one?

It would.

If then, in every point of view, the one and the not-one are distinct,
then neither is the one part or whole of the not-one, nor is the not-one
part or whole of the one?

No.

But we said that things which are neither parts nor wholes of one
another, nor other than one another, will be the same with one
another:--so we said?

Yes.

Then shall we say that the one, being in this relation to the not-one,
is the same with it?

Let us say so.

Then it is the same with itself and the others, and also other than
itself and the others.

That appears to be the inference.

And it will also be like and unlike itself and the others?

Perhaps.

Since the one was shown to be other than the others, the others will
also be other than the one.

Yes.

And the one is other than the others in the same degree that the others
are other than it, and neither more nor less?

True.

And if neither more nor less, then in a like degree?

Yes.

In virtue of the affection by which the one is other than others and
others in like manner other than it, the one will be affected like the
others and the others like the one.

How do you mean?

I may take as an illustration the case of names: You give a name to a
thing?

Yes.

And you may say the name once or oftener?

Yes.

And when you say it once, you mention that of which it is the name? and
when more than once, is it something else which you mention? or must it
always be the same thing of which you speak, whether you utter the name
once or more than once?

Of course it is the same.

And is not 'other' a name given to a thing?

Certainly.

Whenever, then, you use the word 'other,' whether once or oftener, you
name that of which it is the name, and to no other do you give the name?

True.

Then when we say that the others are other than the one, and the one
other than the others, in repeating the word 'other' we speak of that
nature to which the name is applied, and of no other?

Quite true.

Then the one which is other than others, and the other which is other
than the one, in that the word 'other' is applied to both, will be in
the same condition; and that which is in the same condition is like?

Yes.

Then in virtue of the affection by which the one is other than the
others, every thing will be like every thing, for every thing is other
than every thing.

True.

Again, the like is opposed to the unlike?

Yes.

And the other to the same?

True again.

And the one was also shown to be the same with the others?

Yes.

And to be the same with the others is the opposite of being other than
the others?

Certainly.

And in that it was other it was shown to be like?

Yes.

But in that it was the same it will be unlike by virtue of the opposite
affection to that which made it like; and this was the affection of
otherness.

Yes.

The same then will make it unlike; otherwise it will not be the opposite
of the other.

True.

Then the one will be both like and unlike the others; like in so far as
it is other, and unlike in so far as it is the same.

Yes, that argument may be used.

And there is another argument.

What?

In so far as it is affected in the same way it is not affected
otherwise, and not being affected otherwise is not unlike, and not
being unlike, is like; but in so far as it is affected by other it is
otherwise, and being otherwise affected is unlike.

True.

Then because the one is the same with the others and other than the
others, on either of these two grounds, or on both of them, it will be
both like and unlike the others?

Certainly.

And in the same way as being other than itself and the same with itself,
on either of these two grounds and on both of them, it will be like and
unlike itself?

Of course.

Again, how far can the one touch or not touch itself and
others?--consider.

I am considering.

The one was shown to be in itself which was a whole?

True.

And also in other things?

Yes.

In so far as it is in other things it would touch other things, but in
so far as it is in itself it would be debarred from touching them, and
would touch itself only.

Clearly.

Then the inference is that it would touch both?

It would.

But what do you say to a new point of view? Must not that which is to
touch another be next to that which it is to touch, and occupy the place
nearest to that in which what it touches is situated?

True.

Then the one, if it is to touch itself, ought to be situated next to
itself, and occupy the place next to that in which itself is?

It ought.

And that would require that the one should be two, and be in two places
at once, and this, while it is one, will never happen.

No.

Then the one cannot touch itself any more than it can be two?

It cannot.

Neither can it touch others.

Why not?

The reason is, that whatever is to touch another must be in separation
from, and next to, that which it is to touch, and no third thing can be
between them.

True.

Two things, then, at the least are necessary to make contact possible?

They are.

And if to the two a third be added in due order, the number of terms
will be three, and the contacts two?

Yes.

And every additional term makes one additional contact, whence it
follows that the contacts are one less in number than the terms; the
first two terms exceeded the number of contacts by one, and the whole
number of terms exceeds the whole number of contacts by one in like
manner; and for every one which is afterwards added to the number of
terms, one contact is added to the contacts.

True.

Whatever is the whole number of things, the contacts will be always one
less.

True.

But if there be only one, and not two, there will be no contact?

How can there be?

And do we not say that the others being other than the one are not one
and have no part in the one?

True.

Then they have no number, if they have no one in them?

Of course not.

Then the others are neither one nor two, nor are they called by the name
of any number?

No.

One, then, alone is one, and two do not exist?

Clearly not.

And if there are not two, there is no contact?

There is not.

Then neither does the one touch the others, nor the others the one, if
there is no contact?

Certainly not.

For all which reasons the one touches and does not touch itself and the
others?

True.

Further--is the one equal and unequal to itself and others?

How do you mean?

If the one were greater or less than the others, or the others greater
or less than the one, they would not be greater or less than each other
in virtue of their being the one and the others; but, if in addition to
their being what they are they had equality, they would be equal to one
another, or if the one had smallness and the others greatness, or the
one had greatness and the others smallness--whichever kind had greatness
would be greater, and whichever had smallness would be smaller?

Certainly.

Then there are two such ideas as greatness and smallness; for if they
were not they could not be opposed to each other and be present in that
which is.

How could they?

If, then, smallness is present in the one it will be present either in
the whole or in a part of the whole?

Certainly.

Suppose the first; it will be either co-equal and co-extensive with the
whole one, or will contain the one?

Clearly.

If it be co-extensive with the one it will be co-equal with the one, or
if containing the one it will be greater than the one?

Of course.

But can smallness be equal to anything or greater than anything, and
have the functions of greatness and equality and not its own functions?

Impossible.

Then smallness cannot be in the whole of one, but, if at all, in a part
only?

Yes.

And surely not in all of a part, for then the difficulty of the whole
will recur; it will be equal to or greater than any part in which it is.

Certainly.

Then smallness will not be in anything, whether in a whole or in a part;
nor will there be anything small but actual smallness.

True.

Neither will greatness be in the one, for if greatness be in anything
there will be something greater other and besides greatness itself,
namely, that in which greatness is; and this too when the small itself
is not there, which the one, if it is great, must exceed; this, however,
is impossible, seeing that smallness is wholly absent.

True.

But absolute greatness is only greater than absolute smallness, and
smallness is only smaller than absolute greatness.

Very true.

Then other things not greater or less than the one, if they have neither
greatness nor smallness; nor have greatness or smallness any power of
exceeding or being exceeded in relation to the one, but only in relation
to one another; nor will the one be greater or less than them or others,
if it has neither greatness nor smallness.

Clearly not.

Then if the one is neither greater nor less than the others, it cannot
either exceed or be exceeded by them?

Certainly not.

And that which neither exceeds nor is exceeded, must be on an equality;
and being on an equality, must be equal.

Of course.

And this will be true also of the relation of the one to itself; having
neither greatness nor smallness in itself, it will neither exceed nor be
exceeded by itself, but will be on an equality with and equal to itself.

Certainly.

Then the one will be equal both to itself and the others?

Clearly so.

And yet the one, being itself in itself, will also surround and be
without itself; and, as containing itself, will be greater than itself;
and, as contained in itself, will be less; and will thus be greater and
less than itself.

It will.

Now there cannot possibly be anything which is not included in the one
and the others?

Of course not.

But, surely, that which is must always be somewhere?

Yes.

But that which is in anything will be less, and that in which it is will
be greater; in no other way can one thing be in another.

True.

And since there is nothing other or besides the one and the others, and
they must be in something, must they not be in one another, the one in
the others and the others in the one, if they are to be anywhere?

That is clear.

But inasmuch as the one is in the others, the others will be greater
than the one, because they contain the one, which will be less than the
others, because it is contained in them; and inasmuch as the others
are in the one, the one on the same principle will be greater than the
others, and the others less than the one.

True.

The one, then, will be equal to and greater and less than itself and the
others?

Clearly.

And if it be greater and less and equal, it will be of equal and more
and less measures or divisions than itself and the others, and if of
measures, also of parts?

Of course.

And if of equal and more and less measures or divisions, it will be in
number more or less than itself and the others, and likewise equal in
number to itself and to the others?

How is that?

It will be of more measures than those things which it exceeds, and of
as many parts as measures; and so with that to which it is equal, and
that than which it is less.

True.

And being greater and less than itself, and equal to itself, it will
be of equal measures with itself and of more and fewer measures than
itself; and if of measures then also of parts?

It will.

And being of equal parts with itself, it will be numerically equal to
itself; and being of more parts, more, and being of less, less than
itself?

Certainly.

And the same will hold of its relation to other things; inasmuch as it
is greater than them, it will be more in number than them; and inasmuch
as it is smaller, it will be less in number; and inasmuch as it is equal
in size to other things, it will be equal to them in number.

Certainly.

Once more, then, as would appear, the one will be in number both equal
to and more and less than both itself and all other things.

It will.

Does the one also partake of time? And is it and does it become older
and younger than itself and others, and again, neither younger nor older
than itself and others, by virtue of participation in time?

How do you mean?

If one is, being must be predicated of it?

Yes.

But to be (einai) is only participation of being in present time, and to
have been is the participation of being at a past time, and to be about
to be is the participation of being at a future time?

Very true.

Then the one, since it partakes of being, partakes of time?

Certainly.

And is not time always moving forward?

Yes.

Then the one is always becoming older than itself, since it moves
forward in time?

Certainly.

And do you remember that the older becomes older than that which becomes
younger?

I remember.

Then since the one becomes older than itself, it becomes younger at the
same time?

Certainly.

Thus, then, the one becomes older as well as younger than itself?

Yes.

And it is older (is it not?) when in becoming, it gets to the point of
time between 'was' and 'will be,' which is 'now': for surely in going
from the past to the future, it cannot skip the present?

No.

And when it arrives at the present it stops from becoming older, and
no longer becomes, but is older, for if it went on it would never be
reached by the present, for it is the nature of that which goes on,
to touch both the present and the future, letting go the present and
seizing the future, while in process of becoming between them.

True.

But that which is becoming cannot skip the present; when it reaches the
present it ceases to become, and is then whatever it may happen to be
becoming.

Clearly.

And so the one, when in becoming older it reaches the present, ceases to
become, and is then older.

Certainly.

And it is older than that than which it was becoming older, and it was
becoming older than itself.

Yes.

And that which is older is older than that which is younger?

True.

Then the one is younger than itself, when in becoming older it reaches
the present?

Certainly.

But the present is always present with the one during all its being; for
whenever it is it is always now.

Certainly.

Then the one always both is and becomes older and younger than itself?

Truly.

And is it or does it become a longer time than itself or an equal time
with itself?

An equal time.

But if it becomes or is for an equal time with itself, it is of the same
age with itself?

Of course.

And that which is of the same age, is neither older nor younger?

No.

The one, then, becoming and being the same time with itself, neither is
nor becomes older or younger than itself?

I should say not.

And what are its relations to other things? Is it or does it become
older or younger than they?

I cannot tell you.

You can at least tell me that others than the one are more than the
one--other would have been one, but the others have multitude, and are
more than one?

They will have multitude.

And a multitude implies a number larger than one?

Of course.

And shall we say that the lesser or the greater is the first to come or
to have come into existence?

The lesser.

Then the least is the first? And that is the one?

Yes.

Then the one of all things that have number is the first to come into
being; but all other things have also number, being plural and not
singular.

They have.

And since it came into being first it must be supposed to have come into
being prior to the others, and the others later; and the things which
came into being later, are younger than that which preceded them? And
so the other things will be younger than the one, and the one older than
other things?

True.

What would you say of another question? Can the one have come into being
contrary to its own nature, or is that impossible?

Impossible.

And yet, surely, the one was shown to have parts; and if parts, then a
beginning, middle and end?

Yes.

And a beginning, both of the one itself and of all other things, comes
into being first of all; and after the beginning, the others follow,
until you reach the end?

Certainly.

And all these others we shall affirm to be parts of the whole and of the
one, which, as soon as the end is reached, has become whole and one?

Yes; that is what we shall say.

But the end comes last, and the one is of such a nature as to come into
being with the last; and, since the one cannot come into being except in
accordance with its own nature, its nature will require that it should
come into being after the others, simultaneously with the end.

Clearly.

Then the one is younger than the others and the others older than the
one.

That also is clear in my judgment.

Well, and must not a beginning or any other part of the one or of
anything, if it be a part and not parts, being a part, be also of
necessity one?

Certainly.

And will not the one come into being together with each part--together
with the first part when that comes into being, and together with the
second part and with all the rest, and will not be wanting to any part,
which is added to any other part until it has reached the last and
become one whole; it will be wanting neither to the middle, nor to
the first, nor to the last, nor to any of them, while the process of
becoming is going on?

True.

Then the one is of the same age with all the others, so that if the one
itself does not contradict its own nature, it will be neither prior
nor posterior to the others, but simultaneous; and according to this
argument the one will be neither older nor younger than the others, nor
the others than the one, but according to the previous argument the one
will be older and younger than the others and the others than the one.

Certainly.

After this manner then the one is and has become. But as to its becoming
older and younger than the others, and the others than the one, and
neither older nor younger, what shall we say? Shall we say as of being
so also of becoming, or otherwise?

I cannot answer.

But I can venture to say, that even if one thing were older or younger
than another, it could not become older or younger in a greater degree
than it was at first; for equals added to unequals, whether to periods
of time or to anything else, leave the difference between them the same
as at first.

Of course.

Then that which is, cannot become older or younger than that which
is, since the difference of age is always the same; the one is and has
become older and the other younger; but they are no longer becoming so.

True.

And the one which is does not therefore become either older or younger
than the others which are.

No.

But consider whether they may not become older and younger in another
way.

In what way?

Just as the one was proven to be older than the others and the others
than the one.

And what of that?

If the one is older than the others, has come into being a longer time
than the others.

Yes.

But consider again; if we add equal time to a greater and a less time,
will the greater differ from the less time by an equal or by a smaller
portion than before?

By a smaller portion.

Then the difference between the age of the one and the age of the others
will not be afterwards so great as at first, but if an equal time be
added to both of them they will differ less and less in age?

Yes.

And that which differs in age from some other less than formerly, from
being older will become younger in relation to that other than which it
was older?

Yes, younger.

And if the one becomes younger the others aforesaid will become older
than they were before, in relation to the one.

Certainly.

Then that which had become younger becomes older relatively to that
which previously had become and was older; it never really is older, but
is always becoming, for the one is always growing on the side of youth
and the other on the side of age. And in like manner the older is always
in process of becoming younger than the younger; for as they are always
going in opposite directions they become in ways the opposite to one
another, the younger older than the older, and the older younger than
the younger. They cannot, however, have become; for if they had already
become they would be and not merely become. But that is impossible; for
they are always becoming both older and younger than one another: the
one becomes younger than the others because it was seen to be older and
prior, and the others become older than the one because they came into
being later; and in the same way the others are in the same relation to
the one, because they were seen to be older, and prior to the one.

That is clear.

Inasmuch then, one thing does not become older or younger than another,
in that they always differ from each other by an equal number, the one
cannot become older or younger than the others, nor the others than the
one; but inasmuch as that which came into being earlier and that which
came into being later must continually differ from each other by a
different portion--in this point of view the others must become older
and younger than the one, and the one than the others.

Certainly.

For all these reasons, then, the one is and becomes older and younger
than itself and the others, and neither is nor becomes older or younger
than itself or the others.

Certainly.

But since the one partakes of time, and partakes of becoming older and
younger, must it not also partake of the past, the present, and the
future?

Of course it must.

Then the one was and is and will be, and was becoming and is becoming
and will become?

Certainly.

And there is and was and will be something which is in relation to it
and belongs to it?

True.

And since we have at this moment opinion and knowledge and perception of
the one, there is opinion and knowledge and perception of it?

Quite right.

Then there is name and expression for it, and it is named and expressed,
and everything of this kind which appertains to other things appertains
to the one.

Certainly, that is true.

Yet once more and for the third time, let us consider: If the one is
both one and many, as we have described, and is neither one nor many,
and participates in time, must it not, in as far as it is one, at times
partake of being, and in as far as it is not one, at times not partake
of being?

Certainly.

But can it partake of being when not partaking of being, or not partake
of being when partaking of being?

Impossible.

Then the one partakes and does not partake of being at different times,
for that is the only way in which it can partake and not partake of the
same.

True.

And is there not also a time at which it assumes being and relinquishes
being--for how can it have and not have the same thing unless it
receives and also gives it up at some time?

Impossible.

And the assuming of being is what you would call becoming?

I should.

And the relinquishing of being you would call destruction?

I should.

The one then, as would appear, becomes and is destroyed by taking and
giving up being.

Certainly.

And being one and many and in process of becoming and being destroyed,
when it becomes one it ceases to be many, and when many, it ceases to be
one?

Certainly.

And as it becomes one and many, must it not inevitably experience
separation and aggregation?

Inevitably.

And whenever it becomes like and unlike it must be assimilated and
dissimilated?

Yes.

And when it becomes greater or less or equal it must grow or diminish or
be equalized?

True.

And when being in motion it rests, and when being at rest it changes to
motion, it can surely be in no time at all?

How can it?

But that a thing which is previously at rest should be afterwards
in motion, or previously in motion and afterwards at rest, without
experiencing change, is impossible.

Impossible.

And surely there cannot be a time in which a thing can be at once
neither in motion nor at rest?

There cannot.

But neither can it change without changing.

True.

When then does it change; for it cannot change either when at rest, or
when in motion, or when in time?

It cannot.

And does this strange thing in which it is at the time of changing
really exist?

What thing?

The moment. For the moment seems to imply a something out of which
change takes place into either of two states; for the change is not from
the state of rest as such, nor from the state of motion as such; but
there is this curious nature which we call the moment lying between rest
and motion, not being in any time; and into this and out of this what is
in motion changes into rest, and what is at rest into motion.

So it appears.

And the one then, since it is at rest and also in motion, will change
to either, for only in this way can it be in both. And in changing it
changes in a moment, and when it is changing it will be in no time, and
will not then be either in motion or at rest.

It will not.

And it will be in the same case in relation to the other changes, when
it passes from being into cessation of being, or from not-being into
becoming--then it passes between certain states of motion and rest, and
neither is nor is not, nor becomes nor is destroyed.

Very true.

And on the same principle, in the passage from one to many and from
many to one, the one is neither one nor many, neither separated nor
aggregated; and in the passage from like to unlike, and from unlike to
like, it is neither like nor unlike, neither in a state of assimilation
nor of dissimilation; and in the passage from small to great and equal
and back again, it will be neither small nor great, nor equal, nor in a
state of increase, or diminution, or equalization.

True.

All these, then, are the affections of the one, if the one has being.

Of course.

1.aa. But if one is, what will happen to the others--is not that also to
be considered?

Yes.

Let us show then, if one is, what will be the affections of the others
than the one.

Let us do so.

Inasmuch as there are things other than the one, the others are not the
one; for if they were they could not be other than the one.

Very true.

Nor are the others altogether without the one, but in a certain way they
participate in the one.

In what way?

Because the others are other than the one inasmuch as they have parts;
for if they had no parts they would be simply one.

Right.

And parts, as we affirm, have relation to a whole?

So we say.

And a whole must necessarily be one made up of many; and the parts will
be parts of the one, for each of the parts is not a part of many, but of
a whole.

How do you mean?

If anything were a part of many, being itself one of them, it will
surely be a part of itself, which is impossible, and it will be a part
of each one of the other parts, if of all; for if not a part of some
one, it will be a part of all the others but this one, and thus will not
be a part of each one; and if not a part of each, one it will not be a
part of any one of the many; and not being a part of any one, it cannot
be a part or anything else of all those things of none of which it is
anything.

Clearly not.

Then the part is not a part of the many, nor of all, but is of a certain
single form, which we call a whole, being one perfect unity framed out
of all--of this the part will be a part.

Certainly.

If, then, the others have parts, they will participate in the whole and
in the one.

True.

Then the others than the one must be one perfect whole, having parts.

Certainly.

And the same argument holds of each part, for the part must participate
in the one; for if each of the parts is a part, this means, I suppose,
that it is one separate from the rest and self-related; otherwise it is
not each.

True.

But when we speak of the part participating in the one, it must clearly
be other than one; for if not, it would not merely have participated,
but would have been one; whereas only the itself can be one.

Very true.

Both the whole and the part must participate in the one; for the whole
will be one whole, of which the parts will be parts; and each part will
be one part of the whole which is the whole of the part.

True.

And will not the things which participate in the one, be other than it?

Of course.

And the things which are other than the one will be many; for if the
things which are other than the one were neither one nor more than one,
they would be nothing.

True.

But, seeing that the things which participate in the one as a part, and
in the one as a whole, are more than one, must not those very things
which participate in the one be infinite in number?

How so?

Let us look at the matter thus:--Is it not a fact that in partaking of
the one they are not one, and do not partake of the one at the very time
when they are partaking of it?

Clearly.

They do so then as multitudes in which the one is not present?

Very true.

And if we were to abstract from them in idea the very smallest fraction,
must not that least fraction, if it does not partake of the one, be a
multitude and not one?

It must.

And if we continue to look at the other side of their nature, regarded
simply, and in itself, will not they, as far as we see them, be
unlimited in number?

Certainly.

And yet, when each several part becomes a part, then the parts have
a limit in relation to the whole and to each other, and the whole in
relation to the parts.

Just so.

The result to the others than the one is that the union of themselves
and the one appears to create a new element in them which gives to them
limitation in relation to one another; whereas in their own nature they
have no limit.

That is clear.

Then the others than the one, both as whole and parts, are infinite, and
also partake of limit.

Certainly.

Then they are both like and unlike one another and themselves.

How is that?

Inasmuch as they are unlimited in their own nature, they are all
affected in the same way.

True.

And inasmuch as they all partake of limit, they are all affected in the
same way.

Of course.

But inasmuch as their state is both limited and unlimited, they are
affected in opposite ways.

Yes.

And opposites are the most unlike of things.

Certainly.

Considered, then, in regard to either one of their affections, they will
be like themselves and one another; considered in reference to both of
them together, most opposed and most unlike.

That appears to be true.

Then the others are both like and unlike themselves and one another?

True.

And they are the same and also different from one another, and in motion
and at rest, and experience every sort of opposite affection, as may be
proved without difficulty of them, since they have been shown to have
experienced the affections aforesaid?

True.

1.bb. Suppose, now, that we leave the further discussion of these
matters as evident, and consider again upon the hypothesis that the
one is, whether opposite of all this is or is not equally true of the
others.

By all means.

Then let us begin again, and ask, If one is, what must be the affections
of the others?

Let us ask that question.

Must not the one be distinct from the others, and the others from the
one?

Why so?

Why, because there is nothing else beside them which is distinct from
both of them; for the expression 'one and the others' includes all
things.

Yes, all things.

Then we cannot suppose that there is anything different from them in
which both the one and the others might exist?

There is nothing.

Then the one and the others are never in the same?

True.

Then they are separated from each other?

Yes.

And we surely cannot say that what is truly one has parts?

Impossible.

Then the one will not be in the others as a whole, nor as part, if it be
separated from the others, and has no parts?

Impossible.

Then there is no way in which the others can partake of the one, if they
do not partake either in whole or in part?

It would seem not.

Then there is no way in which the others are one, or have in themselves
any unity?

There is not.

Nor are the others many; for if they were many, each part of them would
be a part of the whole; but now the others, not partaking in any way of
the one, are neither one nor many, nor whole, nor part.

True.

Then the others neither are nor contain two or three, if entirely
deprived of the one?

True.

Then the others are neither like nor unlike the one, nor is likeness
and unlikeness in them; for if they were like and unlike, or had in them
likeness and unlikeness, they would have two natures in them opposite to
one another.

That is clear.

But for that which partakes of nothing to partake of two things was held
by us to be impossible?

Impossible.

Then the others are neither like nor unlike nor both, for if they were
like or unlike they would partake of one of those two natures, which
would be one thing, and if they were both they would partake of
opposites which would be two things, and this has been shown to be
impossible.

True.

Therefore they are neither the same, nor other, nor in motion, nor at
rest, nor in a state of becoming, nor of being destroyed, nor greater,
nor less, nor equal, nor have they experienced anything else of the
sort; for, if they are capable of experiencing any such affection, they
will participate in one and two and three, and odd and even, and in
these, as has been proved, they do not participate, seeing that they are
altogether and in every way devoid of the one.

Very true.

Therefore if one is, the one is all things, and also nothing, both in
relation to itself and to other things.

Certainly.

2.a. Well, and ought we not to consider next what will be the
consequence if the one is not?

Yes; we ought.

What is the meaning of the hypothesis--If the one is not; is there any
difference between this and the hypothesis--If the not one is not?

There is a difference, certainly.

Is there a difference only, or rather are not the two expressions--if
the one is not, and if the not one is not, entirely opposed?

They are entirely opposed.

And suppose a person to say:--If greatness is not, if smallness is not,
or anything of that sort, does he not mean, whenever he uses such an
expression, that 'what is not' is other than other things?

To be sure.

And so when he says 'If one is not' he clearly means, that what 'is not'
is other than all others; we know what he means--do we not?

Yes, we do.

When he says 'one,' he says something which is known; and secondly
something which is other than all other things; it makes no difference
whether he predicate of one being or not-being, for that which is said
'not to be' is known to be something all the same, and is distinguished
from other things.

Certainly.

Then I will begin again, and ask: If one is not, what are the
consequences? In the first place, as would appear, there is a knowledge
of it, or the very meaning of the words, 'if one is not,' would not be
known.

True.

Secondly, the others differ from it, or it could not be described as
different from the others?

Certainly.

Difference, then, belongs to it as well as knowledge; for in speaking of
the one as different from the others, we do not speak of a difference in
the others, but in the one.

Clearly so.

Moreover, the one that is not is something and partakes of relation to
'that,' and 'this,' and 'these,' and the like, and is an attribute of
'this'; for the one, or the others than the one, could not have been
spoken of, nor could any attribute or relative of the one that is not
have been or been spoken of, nor could it have been said to be anything,
if it did not partake of 'some,' or of the other relations just now
mentioned.

True.

Being, then, cannot be ascribed to the one, since it is not; but the
one that is not may or rather must participate in many things, if it and
nothing else is not; if, however, neither the one nor the one that
is not is supposed not to be, and we are speaking of something of a
different nature, we can predicate nothing of it. But supposing that the
one that is not and nothing else is not, then it must participate in the
predicate 'that,' and in many others.

Certainly.

And it will have unlikeness in relation to the others, for the others
being different from the one will be of a different kind.

Certainly.

And are not things of a different kind also other in kind?

Of course.

And are not things other in kind unlike?

They are unlike.

And if they are unlike the one, that which they are unlike will clearly
be unlike them?

Clearly so.

Then the one will have unlikeness in respect of which the others are
unlike it?

That would seem to be true.

And if unlikeness to other things is attributed to it, it must have
likeness to itself.

How so?

If the one have unlikeness to one, something else must be meant; nor
will the hypothesis relate to one; but it will relate to something other
than one?

Quite so.

But that cannot be.

No.

Then the one must have likeness to itself?

It must.

Again, it is not equal to the others; for if it were equal, then it
would at once be and be like them in virtue of the equality; but if one
has no being, then it can neither be nor be like?

It cannot.

But since it is not equal to the others, neither can the others be equal
to it?

Certainly not.

And things that are not equal are unequal?

True.

And they are unequal to an unequal?

Of course.

Then the one partakes of inequality, and in respect of this the others
are unequal to it?

Very true.

And inequality implies greatness and smallness?

Yes.

Then the one, if of such a nature, has greatness and smallness?

That appears to be true.

And greatness and smallness always stand apart?

True.

Then there is always something between them?

There is.

And can you think of anything else which is between them other than
equality?

No, it is equality which lies between them.

Then that which has greatness and smallness also has equality, which
lies between them?

That is clear.

Then the one, which is not, partakes, as would appear, of greatness and
smallness and equality?

Clearly.

Further, it must surely in a sort partake of being?

How so?

It must be so, for if not, then we should not speak the truth in saying
that the one is not. But if we speak the truth, clearly we must say what
is. Am I not right?

Yes.

And since we affirm that we speak truly, we must also affirm that we say
what is?

Certainly.

Then, as would appear, the one, when it is not, is; for if it were
not to be when it is not, but (Or, 'to remit something of existence in
relation to not-being.') were to relinquish something of being, so as to
become not-being, it would at once be.

Quite true.

Then the one which is not, if it is to maintain itself, must have the
being of not-being as the bond of not-being, just as being must have as
a bond the not-being of not-being in order to perfect its own being;
for the truest assertion of the being of being and of the not-being of
not-being is when being partakes of the being of being, and not of the
being of not-being--that is, the perfection of being; and when not-being
does not partake of the not-being of not-being but of the being of
not-being--that is the perfection of not-being.

Most true.

Since then what is partakes of not-being, and what is not of being, must
not the one also partake of being in order not to be?

Certainly.

Then the one, if it is not, clearly has being?

Clearly.

And has not-being also, if it is not?

Of course.

But can anything which is in a certain state not be in that state
without changing?

Impossible.

Then everything which is and is not in a certain state, implies change?

Certainly.

And change is motion--we may say that?

Yes, motion.

And the one has been proved both to be and not to be?

Yes.

And therefore is and is not in the same state?

Yes.

Thus the one that is not has been shown to have motion also, because it
changes from being to not-being?

That appears to be true.

But surely if it is nowhere among what is, as is the fact, since it is
not, it cannot change from one place to another?

Impossible.

Then it cannot move by changing place?

No.

Nor can it turn on the same spot, for it nowhere touches the same, for
the same is, and that which is not cannot be reckoned among things that
are?

It cannot.

Then the one, if it is not, cannot turn in that in which it is not?

No.

Neither can the one, whether it is or is not, be altered into other
than itself, for if it altered and became different from itself, then we
could not be still speaking of the one, but of something else?

True.

But if the one neither suffers alteration, nor turns round in the same
place, nor changes place, can it still be capable of motion?

Impossible.

Now that which is unmoved must surely be at rest, and that which is at
rest must stand still?

Certainly.

Then the one that is not, stands still, and is also in motion?

That seems to be true.

But if it be in motion it must necessarily undergo alteration, for
anything which is moved, in so far as it is moved, is no longer in the
same state, but in another?

Yes.

Then the one, being moved, is altered?

Yes.

And, further, if not moved in any way, it will not be altered in any
way?

No.

Then, in so far as the one that is not is moved, it is altered, but in
so far as it is not moved, it is not altered?

Right.

Then the one that is not is altered and is not altered?

That is clear.

And must not that which is altered become other than it previously
was, and lose its former state and be destroyed; but that which is not
altered can neither come into being nor be destroyed?

Very true.

And the one that is not, being altered, becomes and is destroyed; and
not being altered, neither becomes nor is destroyed; and so the one that
is not becomes and is destroyed, and neither becomes nor is destroyed?

True.

2.b. And now, let us go back once more to the beginning, and see whether
these or some other consequences will follow.

Let us do as you say.

If one is not, we ask what will happen in respect of one? That is the
question.

Yes.

Do not the words 'is not' signify absence of being in that to which we
apply them?

Just so.

And when we say that a thing is not, do we mean that it is not in one
way but is in another? or do we mean, absolutely, that what is not has
in no sort or way or kind participation of being?

Quite absolutely.

Then, that which is not cannot be, or in any way participate in being?

It cannot.

And did we not mean by becoming, and being destroyed, the assumption of
being and the loss of being?

Nothing else.

And can that which has no participation in being, either assume or lose
being?

Impossible.

The one then, since it in no way is, cannot have or lose or assume being
in any way?

True.

Then the one that is not, since it in no way partakes of being, neither
perishes nor becomes?

No.

Then it is not altered at all; for if it were it would become and be
destroyed?

True.

But if it be not altered it cannot be moved?

Certainly not.

Nor can we say that it stands, if it is nowhere; for that which stands
must always be in one and the same spot?

Of course.

Then we must say that the one which is not never stands still and never
moves?

Neither.

Nor is there any existing thing which can be attributed to it; for if
there had been, it would partake of being?

That is clear.

And therefore neither smallness, nor greatness, nor equality, can be
attributed to it?

No.

Nor yet likeness nor difference, either in relation to itself or to
others?

Clearly not.

Well, and if nothing should be attributed to it, can other things be
attributed to it?

Certainly not.

And therefore other things can neither be like or unlike, the same, or
different in relation to it?

They cannot.

Nor can what is not, be anything, or be this thing, or be related to or
the attribute of this or that or other, or be past, present, or future.
Nor can knowledge, or opinion, or perception, or expression, or name, or
any other thing that is, have any concern with it?

No.

Then the one that is not has no condition of any kind?

Such appears to be the conclusion.

2.aa. Yet once more; if one is not, what becomes of the others? Let us
determine that.

Yes; let us determine that.

The others must surely be; for if they, like the one, were not, we could
not be now speaking of them.

True.

But to speak of the others implies difference--the terms 'other' and
'different' are synonymous?

True.

Other means other than other, and different, different from the
different?

Yes.

Then, if there are to be others, there is something than which they will
be other?

Certainly.

And what can that be?--for if the one is not, they will not be other
than the one.

They will not.

Then they will be other than each other; for the only remaining
alternative is that they are other than nothing.

True.

And they are each other than one another, as being plural and not
singular; for if one is not, they cannot be singular, but every particle
of them is infinite in number; and even if a person takes that which
appears to be the smallest fraction, this, which seemed one, in a moment
evanesces into many, as in a dream, and from being the smallest becomes
very great, in comparison with the fractions into which it is split up?

Very true.

And in such particles the others will be other than one another, if
others are, and the one is not?

Exactly.

And will there not be many particles, each appearing to be one, but not
being one, if one is not?

True.

And it would seem that number can be predicated of them if each of them
appears to be one, though it is really many?

It can.

And there will seem to be odd and even among them, which will also have
no reality, if one is not?

Yes.

And there will appear to be a least among them; and even this will seem
large and manifold in comparison with the many small fractions which are
contained in it?

Certainly.

And each particle will be imagined to be equal to the many and little;
for it could not have appeared to pass from the greater to the less
without having appeared to arrive at the middle; and thus would arise
the appearance of equality.

Yes.

And having neither beginning, middle, nor end, each separate particle
yet appears to have a limit in relation to itself and other.

How so?

Because, when a person conceives of any one of these as such, prior
to the beginning another beginning appears, and there is another end,
remaining after the end, and in the middle truer middles within but
smaller, because no unity can be conceived of any of them, since the one
is not.

Very true.

And so all being, whatever we think of, must be broken up into
fractions, for a particle will have to be conceived of without unity?

Certainly.

And such being when seen indistinctly and at a distance, appears to
be one; but when seen near and with keen intellect, every single thing
appears to be infinite, since it is deprived of the one, which is not?

Nothing more certain.

Then each of the others must appear to be infinite and finite, and one
and many, if others than the one exist and not the one.

They must.

Then will they not appear to be like and unlike?

In what way?

Just as in a picture things appear to be all one to a person standing at
a distance, and to be in the same state and alike?

True.

But when you approach them, they appear to be many and different; and
because of the appearance of the difference, different in kind from, and
unlike, themselves?

True.

And so must the particles appear to be like and unlike themselves and
each other.

Certainly.

And must they not be the same and yet different from one another, and in
contact with themselves, although they are separated, and having
every sort of motion, and every sort of rest, and becoming and being
destroyed, and in neither state, and the like, all which things may be
easily enumerated, if the one is not and the many are?

Most true.

2.bb. Once more, let us go back to the beginning, and ask if the one is
not, and the others of the one are, what will follow.

Let us ask that question.

In the first place, the others will not be one?

Impossible.

Nor will they be many; for if they were many one would be contained
in them. But if no one of them is one, all of them are nought, and
therefore they will not be many.

True.

If there be no one in the others, the others are neither many nor one.

They are not.

Nor do they appear either as one or many.

Why not?

Because the others have no sort or manner or way of communion with any
sort of not-being, nor can anything which is not, be connected with any
of the others; for that which is not has no parts.

True.

Nor is there an opinion or any appearance of not-being in connexion with
the others, nor is not-being ever in any way attributed to the others.

No.

Then if one is not, there is no conception of any of the others either
as one or many; for you cannot conceive the many without the one.

You cannot.

Then if one is not, the others neither are, nor can be conceived to be
either one or many?

It would seem not.

Nor as like or unlike?

No.

Nor as the same or different, nor in contact or separation, nor in any
of those states which we enumerated as appearing to be;--the others
neither are nor appear to be any of these, if one is not?

True.

Then may we not sum up the argument in a word and say truly: If one is
not, then nothing is?

Certainly.

Let thus much be said; and further let us affirm what seems to be the
truth, that, whether one is or is not, one and the others in relation to
themselves and one another, all of them, in every way, are and are not,
and appear to be and appear not to be.

Most true.




% chapter parmenides (end)