%!TEX root = ancient_text.tex


\chapter{Theaetetus} % (fold)
\label{cha:theaetetus}



% Theaetetus

% By Plato


% Translated by Benjamin Jowett







Euclid: Have you only just arrived from the country, Terpsion?

Terpsion: No, I came some time ago: and I have been in the Agora looking
for you, and wondering that I could not find you.

Euclid: But I was not in the city.

Terpsion: Where then?

Euclid: As I was going down to the harbour, I met Theaetetus--he was
being carried up to Athens from the army at Corinth.

Terpsion: Was he alive or dead?

Euclid: He was scarcely alive, for he has been badly wounded; but he was
suffering even more from the sickness which has broken out in the army.

Terpsion: The dysentery, you mean?

Euclid: Yes.

Terpsion: Alas! what a loss he will be!

Euclid: Yes, Terpsion, he is a noble fellow; only to-day I heard some
people highly praising his behaviour in this very battle.

Terpsion: No wonder; I should rather be surprised at hearing anything
else of him. But why did he go on, instead of stopping at Megara?

Euclid: He wanted to get home: although I entreated and advised him to
remain, he would not listen to me; so I set him on his way, and turned
back, and then I remembered what Socrates had said of him, and thought
how remarkably this, like all his predictions, had been fulfilled.
I believe that he had seen him a little before his own death, when
Theaetetus was a youth, and he had a memorable conversation with him,
which he repeated to me when I came to Athens; he was full of admiration
of his genius, and said that he would most certainly be a great man, if
he lived.

Terpsion: The prophecy has certainly been fulfilled; but what was the
conversation? can you tell me?

Euclid: No, indeed, not offhand; but I took notes of it as soon as I got
home; these I filled up from memory, writing them out at leisure; and
whenever I went to Athens, I asked Socrates about any point which I had
forgotten, and on my return I made corrections; thus I have nearly the
whole conversation written down.

Terpsion: I remember--you told me; and I have always been intending to
ask you to show me the writing, but have put off doing so; and now, why
should we not read it through?--having just come from the country, I
should greatly like to rest.

Euclid: I too shall be very glad of a rest, for I went with Theaetetus
as far as Erineum. Let us go in, then, and, while we are reposing, the
servant shall read to us.

Terpsion: Very good.

Euclid: Here is the roll, Terpsion; I may observe that I have introduced
Socrates, not as narrating to me, but as actually conversing with the
persons whom he mentioned--these were, Theodorus the geometrician (of
Cyrene), and Theaetetus. I have omitted, for the sake of convenience,
the interlocutory words `I said,' `I remarked,' which he used when he
spoke of himself, and again, `he agreed,' or `disagreed,' in the answer,
lest the repetition of them should be troublesome.

Terpsion: Quite right, Euclid.

Euclid: And now, boy, you may take the roll and read.

EUCLID'S SERVANT READS.

Socrates: If I cared enough about the Cyrenians, Theodorus, I would ask
you whether there are any rising geometricians or philosophers in that
part of the world. But I am more interested in our own Athenian youth,
and I would rather know who among them are likely to do well. I observe
them as far as I can myself, and I enquire of any one whom they follow,
and I see that a great many of them follow you, in which they are quite
right, considering your eminence in geometry and in other ways. Tell me
then, if you have met with any one who is good for anything.

Theodorus: Yes, Socrates, I have become acquainted with one very
remarkable Athenian youth, whom I commend to you as well worthy of your
attention. If he had been a beauty I should have been afraid to praise
him, lest you should suppose that I was in love with him; but he is no
beauty, and you must not be offended if I say that he is very like you;
for he has a snub nose and projecting eyes, although these features are
less marked in him than in you. Seeing, then, that he has no personal
attractions, I may freely say, that in all my acquaintance, which is
very large, I never knew any one who was his equal in natural gifts: for
he has a quickness of apprehension which is almost unrivalled, and he
is exceedingly gentle, and also the most courageous of men; there is a
union of qualities in him such as I have never seen in any other, and
should scarcely have thought possible; for those who, like him, have
quick and ready and retentive wits, have generally also quick tempers;
they are ships without ballast, and go darting about, and are mad rather
than courageous; and the steadier sort, when they have to face study,
prove stupid and cannot remember. Whereas he moves surely and smoothly
and successfully in the path of knowledge and enquiry; and he is full of
gentleness, flowing on silently like a river of oil; at his age, it is
wonderful.

Socrates: That is good news; whose son is he?

Theodorus: The name of his father I have forgotten, but the youth
himself is the middle one of those who are approaching us; he and his
companions have been anointing themselves in the outer court, and now
they seem to have finished, and are coming towards us. Look and see
whether you know him.

Socrates: I know the youth, but I do not know his name; he is the son of
Euphronius the Sunian, who was himself an eminent man, and such another
as his son is, according to your account of him; I believe that he left
a considerable fortune.

Theodorus: Theaetetus, Socrates, is his name; but I rather think that
the property disappeared in the hands of trustees; notwithstanding which
he is wonderfully liberal.

Socrates: He must be a fine fellow; tell him to come and sit by me.

Theodorus: I will. Come hither, Theaetetus, and sit by Socrates.

Socrates: By all means, Theaetetus, in order that I may see the
reflection of myself in your face, for Theodorus says that we are alike;
and yet if each of us held in his hands a lyre, and he said that they
were tuned alike, should we at once take his word, or should we ask
whether he who said so was or was not a musician?

Theaetetus: We should ask.

Socrates: And if we found that he was, we should take his word; and if
not, not?

Theaetetus: True.

Socrates: And if this supposed likeness of our faces is a matter of any
interest to us, we should enquire whether he who says that we are alike
is a painter or not?

Theaetetus: Certainly we should.

Socrates: And is Theodorus a painter?

Theaetetus: I never heard that he was.

Socrates: Is he a geometrician?

Theaetetus: Of course he is, Socrates.

Socrates: And is he an astronomer and calculator and musician, and in
general an educated man?

Theaetetus: I think so.

Socrates: If, then, he remarks on a similarity in our persons, either
by way of praise or blame, there is no particular reason why we should
attend to him.

Theaetetus: I should say not.

Socrates: But if he praises the virtue or wisdom which are the mental
endowments of either of us, then he who hears the praises will naturally
desire to examine him who is praised: and he again should be willing to
exhibit himself.

Theaetetus: Very true, Socrates.

Socrates: Then now is the time, my dear Theaetetus, for me to examine,
and for you to exhibit; since although Theodorus has praised many a
citizen and stranger in my hearing, never did I hear him praise any one
as he has been praising you.

Theaetetus: I am glad to hear it, Socrates; but what if he was only in
jest?

Socrates: Nay, Theodorus is not given to jesting; and I cannot allow you
to retract your consent on any such pretence as that. If you do, he will
have to swear to his words; and we are perfectly sure that no one will
be found to impugn him. Do not be shy then, but stand to your word.

Theaetetus: I suppose I must, if you wish it.

Socrates: In the first place, I should like to ask what you learn of
Theodorus: something of geometry, perhaps?

Theaetetus: Yes.

Socrates: And astronomy and harmony and calculation?

Theaetetus: I do my best.

Socrates: Yes, my boy, and so do I; and my desire is to learn of him,
or of anybody who seems to understand these things. And I get on pretty
well in general; but there is a little difficulty which I want you and
the company to aid me in investigating. Will you answer me a question:
`Is not learning growing wiser about that which you learn?'

Theaetetus: Of course.

Socrates: And by wisdom the wise are wise?

Theaetetus: Yes.

Socrates: And is that different in any way from knowledge?

Theaetetus: What?

Socrates: Wisdom; are not men wise in that which they know?

Theaetetus: Certainly they are.

Socrates: Then wisdom and knowledge are the same?

Theaetetus: Yes.

Socrates: Herein lies the difficulty which I can never solve to my
satisfaction--What is knowledge? Can we answer that question? What say
you? which of us will speak first? whoever misses shall sit down, as at
a game of ball, and shall be donkey, as the boys say; he who lasts out
his competitors in the game without missing, shall be our king,
and shall have the right of putting to us any questions which he
pleases...Why is there no reply? I hope, Theodorus, that I am not
betrayed into rudeness by my love of conversation? I only want to make
us talk and be friendly and sociable.

Theodorus: The reverse of rudeness, Socrates: but I would rather that
you would ask one of the young fellows; for the truth is, that I am
unused to your game of question and answer, and I am too old to learn;
the young will be more suitable, and they will improve more than
I shall, for youth is always able to improve. And so having made a
beginning with Theaetetus, I would advise you to go on with him and not
let him off.

Socrates: Do you hear, Theaetetus, what Theodorus says? The philosopher,
whom you would not like to disobey, and whose word ought to be a command
to a young man, bids me interrogate you. Take courage, then, and nobly
say what you think that knowledge is.

Theaetetus: Well, Socrates, I will answer as you and he bid me; and if I
make a mistake, you will doubtless correct me.

Socrates: We will, if we can.

Theaetetus: Then, I think that the sciences which I learn from
Theodorus--geometry, and those which you just now mentioned--are
knowledge; and I would include the art of the cobbler and other
craftsmen; these, each and all of, them, are knowledge.

Socrates: Too much, Theaetetus, too much; the nobility and liberality of
your nature make you give many and diverse things, when I am asking for
one simple thing.

Theaetetus: What do you mean, Socrates?

Socrates: Perhaps nothing. I will endeavour, however, to explain what I
believe to be my meaning: When you speak of cobbling, you mean the art
or science of making shoes?

Theaetetus: Just so.

Socrates: And when you speak of carpentering, you mean the art of making
wooden implements?

Theaetetus: I do.

Socrates: In both cases you define the subject matter of each of the two
arts?

Theaetetus: True.

Socrates: But that, Theaetetus, was not the point of my question: we
wanted to know not the subjects, nor yet the number of the arts or
sciences, for we were not going to count them, but we wanted to know the
nature of knowledge in the abstract. Am I not right?

Theaetetus: Perfectly right.

Socrates: Let me offer an illustration: Suppose that a person were to
ask about some very trivial and obvious thing--for example, What is
clay? and we were to reply, that there is a clay of potters, there is
a clay of oven-makers, there is a clay of brick-makers; would not the
answer be ridiculous?

Theaetetus: Truly.

Socrates: In the first place, there would be an absurdity in assuming
that he who asked the question would understand from our answer the
nature of `clay,' merely because we added `of the image-makers,' or of
any other workers. How can a man understand the name of anything, when
he does not know the nature of it?

Theaetetus: He cannot.

Socrates: Then he who does not know what science or knowledge is, has no
knowledge of the art or science of making shoes?

Theaetetus: None.

Socrates: Nor of any other science?

Theaetetus: No.

Socrates: And when a man is asked what science or knowledge is, to
give in answer the name of some art or science is ridiculous; for the
question is, `What is knowledge?' and he replies, `A knowledge of this
or that.'

Theaetetus: True.

Socrates: Moreover, he might answer shortly and simply, but he makes an
enormous circuit. For example, when asked about the clay, he might have
said simply, that clay is moistened earth--what sort of clay is not to
the point.

Theaetetus: Yes, Socrates, there is no difficulty as you put the
question. You mean, if I am not mistaken, something like what occurred
to me and to my friend here, your namesake Socrates, in a recent
discussion.

Socrates: What was that, Theaetetus?

Theaetetus: Theodorus was writing out for us something about roots, such
as the roots of three or five, showing that they are incommensurable by
the unit: he selected other examples up to seventeen--there he stopped.
Now as there are innumerable roots, the notion occurred to us of
attempting to include them all under one name or class.

Socrates: And did you find such a class?

Theaetetus: I think that we did; but I should like to have your opinion.

Socrates: Let me hear.

Theaetetus: We divided all numbers into two classes: those which are
made up of equal factors multiplying into one another, which we compared
to square figures and called square or equilateral numbers;--that was
one class.

Socrates: Very good.

Theaetetus: The intermediate numbers, such as three and five, and every
other number which is made up of unequal factors, either of a greater
multiplied by a less, or of a less multiplied by a greater, and when
regarded as a figure, is contained in unequal sides;--all these we
compared to oblong figures, and called them oblong numbers.

Socrates: Capital; and what followed?

Theaetetus: The lines, or sides, which have for their squares the
equilateral plane numbers, were called by us lengths or magnitudes; and
the lines which are the roots of (or whose squares are equal to) the
oblong numbers, were called powers or roots; the reason of this latter
name being, that they are commensurable with the former [i.e., with the
so-called lengths or magnitudes] not in linear measurement, but in the
value of the superficial content of their squares; and the same about
solids.

Socrates: Excellent, my boys; I think that you fully justify the praises
of Theodorus, and that he will not be found guilty of false witness.

Theaetetus: But I am unable, Socrates, to give you a similar answer
about knowledge, which is what you appear to want; and therefore
Theodorus is a deceiver after all.

Socrates: Well, but if some one were to praise you for running, and to
say that he never met your equal among boys, and afterwards you were
beaten in a race by a grown-up man, who was a great runner--would the
praise be any the less true?

Theaetetus: Certainly not.

Socrates: And is the discovery of the nature of knowledge so small a
matter, as just now said? Is it not one which would task the powers of
men perfect in every way?

Theaetetus: By heaven, they should be the top of all perfection!

Socrates: Well, then, be of good cheer; do not say that Theodorus was
mistaken about you, but do your best to ascertain the true nature of
knowledge, as well as of other things.

Theaetetus: I am eager enough, Socrates, if that would bring to light
the truth.

Socrates: Come, you made a good beginning just now; let your own answer
about roots be your model, and as you comprehended them all in one
class, try and bring the many sorts of knowledge under one definition.

Theaetetus: I can assure you, Socrates, that I have tried very often,
when the report of questions asked by you was brought to me; but I can
neither persuade myself that I have a satisfactory answer to give, nor
hear of any one who answers as you would have him; and I cannot shake
off a feeling of anxiety.

Socrates: These are the pangs of labour, my dear Theaetetus; you have
something within you which you are bringing to the birth.

Theaetetus: I do not know, Socrates; I only say what I feel.

Socrates: And have you never heard, simpleton, that I am the son of a
midwife, brave and burly, whose name was Phaenarete?

Theaetetus: Yes, I have.

Socrates: And that I myself practise midwifery?

Theaetetus: No, never.

Socrates: Let me tell you that I do though, my friend: but you must not
reveal the secret, as the world in general have not found me out; and
therefore they only say of me, that I am the strangest of mortals and
drive men to their wits' end. Did you ever hear that too?

Theaetetus: Yes.

Socrates: Shall I tell you the reason?

Theaetetus: By all means.

Socrates: Bear in mind the whole business of the midwives, and then you
will see my meaning better:--No woman, as you are probably aware, who is
still able to conceive and bear, attends other women, but only those who
are past bearing.

Theaetetus: Yes, I know.

Socrates: The reason of this is said to be that Artemis--the goddess of
childbirth--is not a mother, and she honours those who are like herself;
but she could not allow the barren to be midwives, because human nature
cannot know the mystery of an art without experience; and therefore she
assigned this office to those who are too old to bear.

Theaetetus: I dare say.

Socrates: And I dare say too, or rather I am absolutely certain, that
the midwives know better than others who is pregnant and who is not?

Theaetetus: Very true.

Socrates: And by the use of potions and incantations they are able to
arouse the pangs and to soothe them at will; they can make those bear
who have a difficulty in bearing, and if they think fit they can smother
the embryo in the womb.

Theaetetus: They can.

Socrates: Did you ever remark that they are also most cunning
matchmakers, and have a thorough knowledge of what unions are likely to
produce a brave brood?

Theaetetus: No, never.

Socrates: Then let me tell you that this is their greatest pride, more
than cutting the umbilical cord. And if you reflect, you will see that
the same art which cultivates and gathers in the fruits of the earth,
will be most likely to know in what soils the several plants or seeds
should be deposited.

Theaetetus: Yes, the same art.

Socrates: And do you suppose that with women the case is otherwise?

Theaetetus: I should think not.

Socrates: Certainly not; but midwives are respectable women who have a
character to lose, and they avoid this department of their profession,
because they are afraid of being called procuresses, which is a name
given to those who join together man and woman in an unlawful and
unscientific way; and yet the true midwife is also the true and only
matchmaker.

Theaetetus: Clearly.

Socrates: Such are the midwives, whose task is a very important one, but
not so important as mine; for women do not bring into the world at one
time real children, and at another time counterfeits which are with
difficulty distinguished from them; if they did, then the discernment of
the true and false birth would be the crowning achievement of the art of
midwifery--you would think so?

Theaetetus: Indeed I should.

Socrates: Well, my art of midwifery is in most respects like theirs; but
differs, in that I attend men and not women; and look after their souls
when they are in labour, and not after their bodies: and the triumph of
my art is in thoroughly examining whether the thought which the mind of
the young man brings forth is a false idol or a noble and true birth.
And like the midwives, I am barren, and the reproach which is often
made against me, that I ask questions of others and have not the wit to
answer them myself, is very just--the reason is, that the god compels me
to be a midwife, but does not allow me to bring forth. And therefore
I am not myself at all wise, nor have I anything to show which is
the invention or birth of my own soul, but those who converse with me
profit. Some of them appear dull enough at first, but afterwards, as
our acquaintance ripens, if the god is gracious to them, they all make
astonishing progress; and this in the opinion of others as well as in
their own. It is quite dear that they never learned anything from me;
the many fine discoveries to which they cling are of their own making.
But to me and the god they owe their delivery. And the proof of my words
is, that many of them in their ignorance, either in their self-conceit
despising me, or falling under the influence of others, have gone away
too soon; and have not only lost the children of whom I had previously
delivered them by an ill bringing up, but have stifled whatever else
they had in them by evil communications, being fonder of lies and shams
than of the truth; and they have at last ended by seeing themselves, as
others see them, to be great fools. Aristeides, the son of Lysimachus,
is one of them, and there are many others. The truants often return to
me, and beg that I would consort with them again--they are ready to
go to me on their knees--and then, if my familiar allows, which is not
always the case, I receive them, and they begin to grow again. Dire
are the pangs which my art is able to arouse and to allay in those who
consort with me, just like the pangs of women in childbirth; night and
day they are full of perplexity and travail which is even worse than
that of the women. So much for them. And there are others, Theaetetus,
who come to me apparently having nothing in them; and as I know that
they have no need of my art, I coax them into marrying some one, and
by the grace of God I can generally tell who is likely to do them good.
Many of them I have given away to Prodicus, and many to other inspired
sages. I tell you this long story, friend Theaetetus, because I suspect,
as indeed you seem to think yourself, that you are in labour--great with
some conception. Come then to me, who am a midwife's son and myself a
midwife, and do your best to answer the questions which I will ask you.
And if I abstract and expose your first-born, because I discover upon
inspection that the conception which you have formed is a vain shadow,
do not quarrel with me on that account, as the manner of women is when
their first children are taken from them. For I have actually known some
who were ready to bite me when I deprived them of a darling folly; they
did not perceive that I acted from goodwill, not knowing that no god is
the enemy of man--that was not within the range of their ideas; neither
am I their enemy in all this, but it would be wrong for me to admit
falsehood, or to stifle the truth. Once more, then, Theaetetus, I repeat
my old question, `What is knowledge?'--and do not say that you cannot
tell; but quit yourself like a man, and by the help of God you will be
able to tell.

Theaetetus: At any rate, Socrates, after such an exhortation I should be
ashamed of not trying to do my best. Now he who knows perceives what he
knows, and, as far as I can see at present, knowledge is perception.

Socrates: Bravely said, boy; that is the way in which you should express
your opinion. And now, let us examine together this conception of yours,
and see whether it is a true birth or a mere wind-egg:--You say that
knowledge is perception?

Theaetetus: Yes.

Socrates: Well, you have delivered yourself of a very important doctrine
about knowledge; it is indeed the opinion of Protagoras, who has another
way of expressing it. Man, he says, is the measure of all things, of the
existence of things that are, and of the non-existence of things that
are not:--You have read him?

Theaetetus: O yes, again and again.

Socrates: Does he not say that things are to you such as they appear to
you, and to me such as they appear to me, and that you and I are men?

Theaetetus: Yes, he says so.

Socrates: A wise man is not likely to talk nonsense. Let us try to
understand him: the same wind is blowing, and yet one of us may be cold
and the other not, or one may be slightly and the other very cold?

Theaetetus: Quite true.

Socrates: Now is the wind, regarded not in relation to us but
absolutely, cold or not; or are we to say, with Protagoras, that the
wind is cold to him who is cold, and not to him who is not?

Theaetetus: I suppose the last.

Socrates: Then it must appear so to each of them?

Theaetetus: Yes.

Socrates: And `appears to him' means the same as `he perceives.'

Theaetetus: True.

Socrates: Then appearing and perceiving coincide in the case of hot and
cold, and in similar instances; for things appear, or may be supposed to
be, to each one such as he perceives them?

Theaetetus: Yes.

Socrates: Then perception is always of existence, and being the same as
knowledge is unerring?

Theaetetus: Clearly.

Socrates: In the name of the Graces, what an almighty wise man
Protagoras must have been! He spoke these things in a parable to the
common herd, like you and me, but told the truth, `his Truth,' (In
allusion to a book of Protagoras' which bore this title.) in secret to
his own disciples.

Theaetetus: What do you mean, Socrates?

Socrates: I am about to speak of a high argument, in which all things
are said to be relative; you cannot rightly call anything by any name,
such as great or small, heavy or light, for the great will be small and
the heavy light--there is no single thing or quality, but out of motion
and change and admixture all things are becoming relatively to one
another, which `becoming' is by us incorrectly called being, but is
really becoming, for nothing ever is, but all things are becoming.
Summon all philosophers--Protagoras, Heracleitus, Empedocles, and the
rest of them, one after another, and with the exception of Parmenides
they will agree with you in this. Summon the great masters of either
kind of poetry--Epicharmus, the prince of Comedy, and Homer of Tragedy;
when the latter sings of

`Ocean whence sprang the gods, and mother Tethys,'

does he not mean that all things are the offspring, of flux and motion?

Theaetetus: I think so.

Socrates: And who could take up arms against such a great army having
Homer for its general, and not appear ridiculous? (Compare Cratylus.)

Theaetetus: Who indeed, Socrates?

Socrates: Yes, Theaetetus; and there are plenty of other proofs
which will show that motion is the source of what is called being and
becoming, and inactivity of not-being and destruction; for fire and
warmth, which are supposed to be the parent and guardian of all other
things, are born of movement and of friction, which is a kind of
motion;--is not this the origin of fire?

Theaetetus: It is.

Socrates: And the race of animals is generated in the same way?

Theaetetus: Certainly.

Socrates: And is not the bodily habit spoiled by rest and idleness, but
preserved for a long time by motion and exercise?

Theaetetus: True.

Socrates: And what of the mental habit? Is not the soul informed, and
improved, and preserved by study and attention, which are motions; but
when at rest, which in the soul only means want of attention and study,
is uninformed, and speedily forgets whatever she has learned?

Theaetetus: True.

Socrates: Then motion is a good, and rest an evil, to the soul as well
as to the body?

Theaetetus: Clearly.

Socrates: I may add, that breathless calm, stillness and the like waste
and impair, while wind and storm preserve; and the palmary argument of
all, which I strongly urge, is the golden chain in Homer, by which
he means the sun, thereby indicating that so long as the sun and the
heavens go round in their orbits, all things human and divine are and
are preserved, but if they were chained up and their motions ceased,
then all things would be destroyed, and, as the saying is, turned upside
down.

Theaetetus: I believe, Socrates, that you have truly explained his
meaning.

Socrates: Then now apply his doctrine to perception, my good friend, and
first of all to vision; that which you call white colour is not in your
eyes, and is not a distinct thing which exists out of them. And you must
not assign any place to it: for if it had position it would be, and be
at rest, and there would be no process of becoming.

Theaetetus: Then what is colour?

Socrates: Let us carry the principle which has just been affirmed, that
nothing is self-existent, and then we shall see that white, black,
and every other colour, arises out of the eye meeting the appropriate
motion, and that what we call a colour is in each case neither the
active nor the passive element, but something which passes between
them, and is peculiar to each percipient; are you quite certain that the
several colours appear to a dog or to any animal whatever as they appear
to you?

Theaetetus: Far from it.

Socrates: Or that anything appears the same to you as to another man?
Are you so profoundly convinced of this? Rather would it not be true
that it never appears exactly the same to you, because you are never
exactly the same?

Theaetetus: The latter.

Socrates: And if that with which I compare myself in size, or which
I apprehend by touch, were great or white or hot, it could not become
different by mere contact with another unless it actually changed; nor
again, if the comparing or apprehending subject were great or white
or hot, could this, when unchanged from within, become changed by any
approximation or affection of any other thing. The fact is that in
our ordinary way of speaking we allow ourselves to be driven into most
ridiculous and wonderful contradictions, as Protagoras and all who take
his line of argument would remark.

Theaetetus: How? and of what sort do you mean?

Socrates: A little instance will sufficiently explain my meaning: Here
are six dice, which are more by a half when compared with four, and
fewer by a half than twelve--they are more and also fewer. How can you
or any one maintain the contrary?

Theaetetus: Very true.

Socrates: Well, then, suppose that Protagoras or some one asks whether
anything can become greater or more if not by increasing, how would you
answer him, Theaetetus?

Theaetetus: I should say `No,' Socrates, if I were to speak my mind
in reference to this last question, and if I were not afraid of
contradicting my former answer.

Socrates: Capital! excellent! spoken like an oracle, my boy! And if you
reply `Yes,' there will be a case for Euripides; for our tongue will be
unconvinced, but not our mind. (In allusion to the well-known line of
Euripides, Hippol.: e gloss omomoch e de thren anomotos.)

Theaetetus: Very true.

Socrates: The thoroughbred Sophists, who know all that can be known
about the mind, and argue only out of the superfluity of their wits,
would have had a regular sparring-match over this, and would have
knocked their arguments together finely. But you and I, who have no
professional aims, only desire to see what is the mutual relation of
these principles,--whether they are consistent with each or not.

Theaetetus: Yes, that would be my desire.

Socrates: And mine too. But since this is our feeling, and there is
plenty of time, why should we not calmly and patiently review our own
thoughts, and thoroughly examine and see what these appearances in
us really are? If I am not mistaken, they will be described by us as
follows:--first, that nothing can become greater or less, either in
number or magnitude, while remaining equal to itself--you would agree?

Theaetetus: Yes.

Socrates: Secondly, that without addition or subtraction there is no
increase or diminution of anything, but only equality.

Theaetetus: Quite true.

Socrates: Thirdly, that what was not before cannot be afterwards,
without becoming and having become.

Theaetetus: Yes, truly.

Socrates: These three axioms, if I am not mistaken, are fighting with
one another in our minds in the case of the dice, or, again, in such a
case as this--if I were to say that I, who am of a certain height and
taller than you, may within a year, without gaining or losing in height,
be not so tall--not that I should have lost, but that you would have
increased. In such a case, I am afterwards what I once was not, and yet
I have not become; for I could not have become without becoming, neither
could I have become less without losing somewhat of my height; and I
could give you ten thousand examples of similar contradictions, if
we admit them at all. I believe that you follow me, Theaetetus; for I
suspect that you have thought of these questions before now.

Theaetetus: Yes, Socrates, and I am amazed when I think of them; by the
Gods I am! and I want to know what on earth they mean; and there are
times when my head quite swims with the contemplation of them.

Socrates: I see, my dear Theaetetus, that Theodorus had a true insight
into your nature when he said that you were a philosopher, for wonder
is the feeling of a philosopher, and philosophy begins in wonder. He was
not a bad genealogist who said that Iris (the messenger of heaven)
is the child of Thaumas (wonder). But do you begin to see what is the
explanation of this perplexity on the hypothesis which we attribute to
Protagoras?

Theaetetus: Not as yet.

Socrates: Then you will be obliged to me if I help you to unearth the
hidden `truth' of a famous man or school.

Theaetetus: To be sure, I shall be very much obliged.

Socrates: Take a look round, then, and see that none of the uninitiated
are listening. Now by the uninitiated I mean the people who believe in
nothing but what they can grasp in their hands, and who will not allow
that action or generation or anything invisible can have real existence.

Theaetetus: Yes, indeed, Socrates, they are very hard and impenetrable
mortals.

Socrates: Yes, my boy, outer barbarians. Far more ingenious are the
brethren whose mysteries I am about to reveal to you. Their first
principle is, that all is motion, and upon this all the affections of
which we were just now speaking are supposed to depend: there is nothing
but motion, which has two forms, one active and the other passive, both
in endless number; and out of the union and friction of them there is
generated a progeny endless in number, having two forms, sense and the
object of sense, which are ever breaking forth and coming to the birth
at the same moment. The senses are variously named hearing, seeing,
smelling; there is the sense of heat, cold, pleasure, pain, desire,
fear, and many more which have names, as well as innumerable others
which are without them; each has its kindred object,--each variety
of colour has a corresponding variety of sight, and so with sound and
hearing, and with the rest of the senses and the objects akin to them.
Do you see, Theaetetus, the bearings of this tale on the preceding
argument?

Theaetetus: Indeed I do not.

Socrates: Then attend, and I will try to finish the story. The purport
is that all these things are in motion, as I was saying, and that this
motion is of two kinds, a slower and a quicker; and the slower elements
have their motions in the same place and with reference to things near
them, and so they beget; but what is begotten is swifter, for it
is carried to fro, and moves from place to place. Apply this to
sense:--When the eye and the appropriate object meet together and give
birth to whiteness and the sensation connatural with it, which could not
have been given by either of them going elsewhere, then, while the
sight is flowing from the eye, whiteness proceeds from the object which
combines in producing the colour; and so the eye is fulfilled with
sight, and really sees, and becomes, not sight, but a seeing eye;
and the object which combined to form the colour is fulfilled with
whiteness, and becomes not whiteness but a white thing, whether wood or
stone or whatever the object may be which happens to be coloured white.
And this is true of all sensible objects, hard, warm, and the like,
which are similarly to be regarded, as I was saying before, not as
having any absolute existence, but as being all of them of whatever kind
generated by motion in their intercourse with one another; for of the
agent and patient, as existing in separation, no trustworthy conception,
as they say, can be formed, for the agent has no existence until united
with the patient, and the patient has no existence until united with
the agent; and that which by uniting with something becomes an agent, by
meeting with some other thing is converted into a patient. And from
all these considerations, as I said at first, there arises a general
reflection, that there is no one self-existent thing, but everything
is becoming and in relation; and being must be altogether abolished,
although from habit and ignorance we are compelled even in this
discussion to retain the use of the term. But great philosophers tell us
that we are not to allow either the word `something,' or `belonging to
something,' or `to me,' or `this,' or `that,' or any other detaining
name to be used, in the language of nature all things are being created
and destroyed, coming into being and passing into new forms; nor can any
name fix or detain them; he who attempts to fix them is easily refuted.
And this should be the way of speaking, not only of particulars but
of aggregates; such aggregates as are expressed in the word `man,' or
`stone,' or any name of an animal or of a class. O Theaetetus, are not
these speculations sweet as honey? And do you not like the taste of them
in the mouth?

Theaetetus: I do not know what to say, Socrates; for, indeed, I cannot
make out whether you are giving your own opinion or only wanting to draw
me out.

Socrates: You forget, my friend, that I neither know, nor profess to
know, anything of these matters; you are the person who is in labour, I
am the barren midwife; and this is why I soothe you, and offer you one
good thing after another, that you may taste them. And I hope that I may
at last help to bring your own opinion into the light of day: when this
has been accomplished, then we will determine whether what you have
brought forth is only a wind-egg or a real and genuine birth. Therefore,
keep up your spirits, and answer like a man what you think.

Theaetetus: Ask me.

Socrates: Then once more: Is it your opinion that nothing is but what
becomes?--the good and the noble, as well as all the other things which
we were just now mentioning?

Theaetetus: When I hear you discoursing in this style, I think that
there is a great deal in what you say, and I am very ready to assent.

Socrates: Let us not leave the argument unfinished, then; for there
still remains to be considered an objection which may be raised about
dreams and diseases, in particular about madness, and the various
illusions of hearing and sight, or of other senses. For you know that
in all these cases the esse-percipi theory appears to be unmistakably
refuted, since in dreams and illusions we certainly have false
perceptions; and far from saying that everything is which appears, we
should rather say that nothing is which appears.

Theaetetus: Very true, Socrates.

Socrates: But then, my boy, how can any one contend that knowledge is
perception, or that to every man what appears is?

Theaetetus: I am afraid to say, Socrates, that I have nothing to answer,
because you rebuked me just now for making this excuse; but I certainly
cannot undertake to argue that madmen or dreamers think truly, when they
imagine, some of them that they are gods, and others that they can fly,
and are flying in their sleep.

Socrates: Do you see another question which can be raised about these
phenomena, notably about dreaming and waking?

Theaetetus: What question?

Socrates: A question which I think that you must often have heard
persons ask:--How can you determine whether at this moment we are
sleeping, and all our thoughts are a dream; or whether we are awake, and
talking to one another in the waking state?

Theaetetus: Indeed, Socrates, I do not know how to prove the one
any more than the other, for in both cases the facts precisely
correspond;--and there is no difficulty in supposing that during all
this discussion we have been talking to one another in a dream; and when
in a dream we seem to be narrating dreams, the resemblance of the two
states is quite astonishing.

Socrates: You see, then, that a doubt about the reality of sense is
easily raised, since there may even be a doubt whether we are awake
or in a dream. And as our time is equally divided between sleeping
and waking, in either sphere of existence the soul contends that the
thoughts which are present to our minds at the time are true; and during
one half of our lives we affirm the truth of the one, and, during the
other half, of the other; and are equally confident of both.

Theaetetus: Most true.

Socrates: And may not the same be said of madness and other disorders?
the difference is only that the times are not equal.

Theaetetus: Certainly.

Socrates: And is truth or falsehood to be determined by duration of
time?

Theaetetus: That would be in many ways ridiculous.

Socrates: But can you certainly determine by any other means which of
these opinions is true?

Theaetetus: I do not think that I can.

Socrates: Listen, then, to a statement of the other side of the
argument, which is made by the champions of appearance. They would say,
as I imagine--Can that which is wholly other than something, have the
same quality as that from which it differs? and observe, Theaetetus,
that the word `other' means not `partially,' but `wholly other.'

Theaetetus: Certainly, putting the question as you do, that which is
wholly other cannot either potentially or in any other way be the same.

Socrates: And must therefore be admitted to be unlike?

Theaetetus: True.

Socrates: If, then, anything happens to become like or unlike itself or
another, when it becomes like we call it the same--when unlike, other?

Theaetetus: Certainly.

Socrates: Were we not saying that there are agents many and infinite,
and patients many and infinite?

Theaetetus: Yes.

Socrates: And also that different combinations will produce results
which are not the same, but different?

Theaetetus: Certainly.

Socrates: Let us take you and me, or anything as an example:--There is
Socrates in health, and Socrates sick--Are they like or unlike?

Theaetetus: You mean to compare Socrates in health as a whole, and
Socrates in sickness as a whole?

Socrates: Exactly; that is my meaning.

Theaetetus: I answer, they are unlike.

Socrates: And if unlike, they are other?

Theaetetus: Certainly.

Socrates: And would you not say the same of Socrates sleeping and
waking, or in any of the states which we were mentioning?

Theaetetus: I should.

Socrates: All agents have a different patient in Socrates, accordingly
as he is well or ill.

Theaetetus: Of course.

Socrates: And I who am the patient, and that which is the agent, will
produce something different in each of the two cases?

Theaetetus: Certainly.

Socrates: The wine which I drink when I am in health, appears sweet and
pleasant to me?

Theaetetus: True.

Socrates: For, as has been already acknowledged, the patient and agent
meet together and produce sweetness and a perception of sweetness, which
are in simultaneous motion, and the perception which comes from the
patient makes the tongue percipient, and the quality of sweetness which
arises out of and is moving about the wine, makes the wine both to be
and to appear sweet to the healthy tongue.

Theaetetus: Certainly; that has been already acknowledged.

Socrates: But when I am sick, the wine really acts upon another and a
different person?

Theaetetus: Yes.

Socrates: The combination of the draught of wine, and the Socrates
who is sick, produces quite another result; which is the sensation of
bitterness in the tongue, and the motion and creation of bitterness in
and about the wine, which becomes not bitterness but something bitter;
as I myself become not perception but percipient?

Theaetetus: True.

Socrates: There is no other object of which I shall ever have the same
perception, for another object would give another perception, and would
make the percipient other and different; nor can that object which
affects me, meeting another subject, produce the same, or become
similar, for that too would produce another result from another subject,
and become different.

Theaetetus: True.

Socrates: Neither can I by myself, have this sensation, nor the object
by itself, this quality.

Theaetetus: Certainly not.

Socrates: When I perceive I must become percipient of something--there
can be no such thing as perceiving and perceiving nothing; the object,
whether it become sweet, bitter, or of any other quality, must have
relation to a percipient; nothing can become sweet which is sweet to no
one.

Theaetetus: Certainly not.

Socrates: Then the inference is, that we (the agent and patient) are or
become in relation to one another; there is a law which binds us one to
the other, but not to any other existence, nor each of us to himself;
and therefore we can only be bound to one another; so that whether
a person says that a thing is or becomes, he must say that it is or
becomes to or of or in relation to something else; but he must not
say or allow any one else to say that anything is or becomes
absolutely:--such is our conclusion.

Theaetetus: Very true, Socrates.

Socrates: Then, if that which acts upon me has relation to me and to no
other, I and no other am the percipient of it?

Theaetetus: Of course.

Socrates: Then my perception is true to me, being inseparable from my
own being; and, as Protagoras says, to myself I am judge of what is and
what is not to me.

Theaetetus: I suppose so.

Socrates: How then, if I never err, and if my mind never trips in the
conception of being or becoming, can I fail of knowing that which I
perceive?

Theaetetus: You cannot.

Socrates: Then you were quite right in affirming that knowledge is only
perception; and the meaning turns out to be the same, whether with Homer
and Heracleitus, and all that company, you say that all is motion and
flux, or with the great sage Protagoras, that man is the measure of all
things; or with Theaetetus, that, given these premises, perception is
knowledge. Am I not right, Theaetetus, and is not this your new-born
child, of which I have delivered you? What say you?

Theaetetus: I cannot but agree, Socrates.

Socrates: Then this is the child, however he may turn out, which you and
I have with difficulty brought into the world. And now that he is born,
we must run round the hearth with him, and see whether he is worth
rearing, or is only a wind-egg and a sham. Is he to be reared in any
case, and not exposed? or will you bear to see him rejected, and not get
into a passion if I take away your first-born?

Theodorus: Theaetetus will not be angry, for he is very good-natured.
But tell me, Socrates, in heaven's name, is this, after all, not the
truth?

Socrates: You, Theodorus, are a lover of theories, and now you
innocently fancy that I am a bag full of them, and can easily pull one
out which will overthrow its predecessor. But you do not see that in
reality none of these theories come from me; they all come from him who
talks with me. I only know just enough to extract them from the wisdom
of another, and to receive them in a spirit of fairness. And now I shall
say nothing myself, but shall endeavour to elicit something from our
young friend.

Theodorus: Do as you say, Socrates; you are quite right.

Socrates: Shall I tell you, Theodorus, what amazes me in your
acquaintance Protagoras?

Theodorus: What is it?

Socrates: I am charmed with his doctrine, that what appears is to
each one, but I wonder that he did not begin his book on Truth with a
declaration that a pig or a dog-faced baboon, or some other yet stranger
monster which has sensation, is the measure of all things; then he might
have shown a magnificent contempt for our opinion of him by informing
us at the outset that while we were reverencing him like a God for
his wisdom he was no better than a tadpole, not to speak of his
fellow-men--would not this have produced an overpowering effect? For
if truth is only sensation, and no man can discern another's feelings
better than he, or has any superior right to determine whether his
opinion is true or false, but each, as we have several times repeated,
is to himself the sole judge, and everything that he judges is true and
right, why, my friend, should Protagoras be preferred to the place
of wisdom and instruction, and deserve to be well paid, and we poor
ignoramuses have to go to him, if each one is the measure of his own
wisdom? Must he not be talking `ad captandum' in all this? I say nothing
of the ridiculous predicament in which my own midwifery and the whole
art of dialectic is placed; for the attempt to supervise or refute the
notions or opinions of others would be a tedious and enormous piece of
folly, if to each man his own are right; and this must be the case if
Protagoras' Truth is the real truth, and the philosopher is not merely
amusing himself by giving oracles out of the shrine of his book.

Theodorus: He was a friend of mine, Socrates, as you were saying, and
therefore I cannot have him refuted by my lips, nor can I oppose you
when I agree with you; please, then, to take Theaetetus again; he seemed
to answer very nicely.

Socrates: If you were to go into a Lacedaemonian palestra, Theodorus,
would you have a right to look on at the naked wrestlers, some of them
making a poor figure, if you did not strip and give them an opportunity
of judging of your own person?

Theodorus: Why not, Socrates, if they would allow me, as I think you
will, in consideration of my age and stiffness; let some more supple
youth try a fall with you, and do not drag me into the gymnasium.

Socrates: Your will is my will, Theodorus, as the proverbial
philosophers say, and therefore I will return to the sage Theaetetus:
Tell me, Theaetetus, in reference to what I was saying, are you not
lost in wonder, like myself, when you find that all of a sudden you are
raised to the level of the wisest of men, or indeed of the gods?--for
you would assume the measure of Protagoras to apply to the gods as well
as men?

Theaetetus: Certainly I should, and I confess to you that I am lost in
wonder. At first hearing, I was quite satisfied with the doctrine, that
whatever appears is to each one, but now the face of things has changed.

Socrates: Why, my dear boy, you are young, and therefore your ear
is quickly caught and your mind influenced by popular arguments.
Protagoras, or some one speaking on his behalf, will doubtless say in
reply,--Good people, young and old, you meet and harangue, and bring
in the gods, whose existence or non-existence I banish from writing and
speech, or you talk about the reason of man being degraded to the level
of the brutes, which is a telling argument with the multitude, but not
one word of proof or demonstration do you offer. All is probability with
you, and yet surely you and Theodorus had better reflect whether you
are disposed to admit of probability and figures of speech in matters
of such importance. He or any other mathematician who argued from
probabilities and likelihoods in geometry, would not be worth an ace.

Theaetetus: But neither you nor we, Socrates, would be satisfied with
such arguments.

Socrates: Then you and Theodorus mean to say that we must look at the
matter in some other way?

Theaetetus: Yes, in quite another way.

Socrates: And the way will be to ask whether perception is or is not the
same as knowledge; for this was the real point of our argument, and with
a view to this we raised (did we not?) those many strange questions.

Theaetetus: Certainly.

Socrates: Shall we say that we know every thing which we see and hear?
for example, shall we say that not having learned, we do not hear the
language of foreigners when they speak to us? or shall we say that
we not only hear, but know what they are saying? Or again, if we see
letters which we do not understand, shall we say that we do not see
them? or shall we aver that, seeing them, we must know them?

Theaetetus: We shall say, Socrates, that we know what we actually see
and hear of them--that is to say, we see and know the figure and colour
of the letters, and we hear and know the elevation or depression of the
sound of them; but we do not perceive by sight and hearing, or know,
that which grammarians and interpreters teach about them.

Socrates: Capital, Theaetetus; and about this there shall be no dispute,
because I want you to grow; but there is another difficulty coming,
which you will also have to repulse.

Theaetetus: What is it?

Socrates: Some one will say, Can a man who has ever known anything, and
still has and preserves a memory of that which he knows, not know that
which he remembers at the time when he remembers? I have, I fear, a
tedious way of putting a simple question, which is only, whether a man
who has learned, and remembers, can fail to know?

Theaetetus: Impossible, Socrates; the supposition is monstrous.

Socrates: Am I talking nonsense, then? Think: is not seeing perceiving,
and is not sight perception?

Theaetetus: True.

Socrates: And if our recent definition holds, every man knows that which
he has seen?

Theaetetus: Yes.

Socrates: And you would admit that there is such a thing as memory?

Theaetetus: Yes.

Socrates: And is memory of something or of nothing?

Theaetetus: Of something, surely.

Socrates: Of things learned and perceived, that is?

Theaetetus: Certainly.

Socrates: Often a man remembers that which he has seen?

Theaetetus: True.

Socrates: And if he closed his eyes, would he forget?

Theaetetus: Who, Socrates, would dare to say so?

Socrates: But we must say so, if the previous argument is to be
maintained.

Theaetetus: What do you mean? I am not quite sure that I understand you,
though I have a strong suspicion that you are right.

Socrates: As thus: he who sees knows, as we say, that which he sees; for
perception and sight and knowledge are admitted to be the same.

Theaetetus: Certainly.

Socrates: But he who saw, and has knowledge of that which he saw,
remembers, when he closes his eyes, that which he no longer sees.

Theaetetus: True.

Socrates: And seeing is knowing, and therefore not-seeing is
not-knowing?

Theaetetus: Very true.

Socrates: Then the inference is, that a man may have attained the
knowledge of something, which he may remember and yet not know, because
he does not see; and this has been affirmed by us to be a monstrous
supposition.

Theaetetus: Most true.

Socrates: Thus, then, the assertion that knowledge and perception are
one, involves a manifest impossibility?

Theaetetus: Yes.

Socrates: Then they must be distinguished?

Theaetetus: I suppose that they must.

Socrates: Once more we shall have to begin, and ask `What is knowledge?'
and yet, Theaetetus, what are we going to do?

Theaetetus: About what?

Socrates: Like a good-for-nothing cock, without having won the victory,
we walk away from the argument and crow.

Theaetetus: How do you mean?

Socrates: After the manner of disputers (Lys.; Phaedo; Republic), we
were satisfied with mere verbal consistency, and were well pleased if in
this way we could gain an advantage. Although professing not to be mere
Eristics, but philosophers, I suspect that we have unconsciously fallen
into the error of that ingenious class of persons.

Theaetetus: I do not as yet understand you.

Socrates: Then I will try to explain myself: just now we asked the
question, whether a man who had learned and remembered could fail to
know, and we showed that a person who had seen might remember when he
had his eyes shut and could not see, and then he would at the same
time remember and not know. But this was an impossibility. And so the
Protagorean fable came to nought, and yours also, who maintained that
knowledge is the same as perception.

Theaetetus: True.

Socrates: And yet, my friend, I rather suspect that the result would
have been different if Protagoras, who was the father of the first of
the two brats, had been alive; he would have had a great deal to say on
their behalf. But he is dead, and we insult over his orphan child; and
even the guardians whom he left, and of whom our friend Theodorus is
one, are unwilling to give any help, and therefore I suppose that I must
take up his cause myself, and see justice done?

Theodorus: Not I, Socrates, but rather Callias, the son of Hipponicus,
is guardian of his orphans. I was too soon diverted from the
abstractions of dialectic to geometry. Nevertheless, I shall be grateful
to you if you assist him.

Socrates: Very good, Theodorus; you shall see how I will come to the
rescue. If a person does not attend to the meaning of terms as they are
commonly used in argument, he may be involved even in greater paradoxes
than these. Shall I explain this matter to you or to Theaetetus?

Theodorus: To both of us, and let the younger answer; he will incur less
disgrace if he is discomfited.

Socrates: Then now let me ask the awful question, which is this:--Can a
man know and also not know that which he knows?

Theodorus: How shall we answer, Theaetetus?

Theaetetus: He cannot, I should say.

Socrates: He can, if you maintain that seeing is knowing. When you are
imprisoned in a well, as the saying is, and the self-assured adversary
closes one of your eyes with his hand, and asks whether you can see
his cloak with the eye which he has closed, how will you answer the
inevitable man?

Theaetetus: I should answer, `Not with that eye but with the other.'

Socrates: Then you see and do not see the same thing at the same time.

Theaetetus: Yes, in a certain sense.

Socrates: None of that, he will reply; I do not ask or bid you answer
in what sense you know, but only whether you know that which you do not
know. You have been proved to see that which you do not see; and you
have already admitted that seeing is knowing, and that not-seeing is
not-knowing: I leave you to draw the inference.

Theaetetus: Yes; the inference is the contradictory of my assertion.

Socrates: Yes, my marvel, and there might have been yet worse things in
store for you, if an opponent had gone on to ask whether you can have a
sharp and also a dull knowledge, and whether you can know near, but not
at a distance, or know the same thing with more or less intensity,
and so on without end. Such questions might have been put to you by a
light-armed mercenary, who argued for pay. He would have lain in wait
for you, and when you took up the position, that sense is knowledge,
he would have made an assault upon hearing, smelling, and the other
senses;--he would have shown you no mercy; and while you were lost in
envy and admiration of his wisdom, he would have got you into his
net, out of which you would not have escaped until you had come to an
understanding about the sum to be paid for your release. Well, you ask,
and how will Protagoras reinforce his position? Shall I answer for him?

Theaetetus: By all means.

Socrates: He will repeat all those things which we have been urging on
his behalf, and then he will close with us in disdain, and say:--The
worthy Socrates asked a little boy, whether the same man could remember
and not know the same thing, and the boy said No, because he was
frightened, and could not see what was coming, and then Socrates made
fun of poor me. The truth is, O slatternly Socrates, that when you ask
questions about any assertion of mine, and the person asked is found
tripping, if he has answered as I should have answered, then I am
refuted, but if he answers something else, then he is refuted and not
I. For do you really suppose that any one would admit the memory which a
man has of an impression which has passed away to be the same with that
which he experienced at the time? Assuredly not. Or would he hesitate to
acknowledge that the same man may know and not know the same thing? Or,
if he is afraid of making this admission, would he ever grant that one
who has become unlike is the same as before he became unlike? Or would
he admit that a man is one at all, and not rather many and infinite as
the changes which take place in him? I speak by the card in order to
avoid entanglements of words. But, O my good sir, he will say, come to
the argument in a more generous spirit; and either show, if you can,
that our sensations are not relative and individual, or, if you admit
them to be so, prove that this does not involve the consequence that the
appearance becomes, or, if you will have the word, is, to the individual
only. As to your talk about pigs and baboons, you are yourself behaving
like a pig, and you teach your hearers to make sport of my writings in
the same ignorant manner; but this is not to your credit. For I declare
that the truth is as I have written, and that each of us is a measure
of existence and of non-existence. Yet one man may be a thousand times
better than another in proportion as different things are and appear
to him. And I am far from saying that wisdom and the wise man have no
existence; but I say that the wise man is he who makes the evils which
appear and are to a man, into goods which are and appear to him. And
I would beg you not to press my words in the letter, but to take the
meaning of them as I will explain them. Remember what has been already
said,--that to the sick man his food appears to be and is bitter, and to
the man in health the opposite of bitter. Now I cannot conceive that one
of these men can be or ought to be made wiser than the other: nor can
you assert that the sick man because he has one impression is foolish,
and the healthy man because he has another is wise; but the one state
requires to be changed into the other, the worse into the better. As
in education, a change of state has to be effected, and the sophist
accomplishes by words the change which the physician works by the aid
of drugs. Not that any one ever made another think truly, who previously
thought falsely. For no one can think what is not, or, think anything
different from that which he feels; and this is always true. But as the
inferior habit of mind has thoughts of kindred nature, so I conceive
that a good mind causes men to have good thoughts; and these which the
inexperienced call true, I maintain to be only better, and not truer
than others. And, O my dear Socrates, I do not call wise men tadpoles:
far from it; I say that they are the physicians of the human body, and
the husbandmen of plants--for the husbandmen also take away the evil and
disordered sensations of plants, and infuse into them good and healthy
sensations--aye and true ones; and the wise and good rhetoricians
make the good instead of the evil to seem just to states; for whatever
appears to a state to be just and fair, so long as it is regarded as
such, is just and fair to it; but the teacher of wisdom causes the good
to take the place of the evil, both in appearance and in reality. And in
like manner the Sophist who is able to train his pupils in this spirit
is a wise man, and deserves to be well paid by them. And so one man is
wiser than another; and no one thinks falsely, and you, whether you will
or not, must endure to be a measure. On these foundations the argument
stands firm, which you, Socrates, may, if you please, overthrow by an
opposite argument, or if you like you may put questions to me--a method
to which no intelligent person will object, quite the reverse. But I
must beg you to put fair questions: for there is great inconsistency
in saying that you have a zeal for virtue, and then always behaving
unfairly in argument. The unfairness of which I complain is that you do
not distinguish between mere disputation and dialectic: the disputer
may trip up his opponent as often as he likes, and make fun; but the
dialectician will be in earnest, and only correct his adversary when
necessary, telling him the errors into which he has fallen through his
own fault, or that of the company which he has previously kept. If
you do so, your adversary will lay the blame of his own confusion and
perplexity on himself, and not on you. He will follow and love you, and
will hate himself, and escape from himself into philosophy, in order
that he may become different from what he was. But the other mode of
arguing, which is practised by the many, will have just the opposite
effect upon him; and as he grows older, instead of turning philosopher,
he will come to hate philosophy. I would recommend you, therefore, as
I said before, not to encourage yourself in this polemical and
controversial temper, but to find out, in a friendly and congenial
spirit, what we really mean when we say that all things are in motion,
and that to every individual and state what appears, is. In this manner
you will consider whether knowledge and sensation are the same or
different, but you will not argue, as you were just now doing, from the
customary use of names and words, which the vulgar pervert in all sorts
of ways, causing infinite perplexity to one another. Such, Theodorus, is
the very slight help which I am able to offer to your old friend; had he
been living, he would have helped himself in a far more gloriose style.

Theodorus: You are jesting, Socrates; indeed, your defence of him has
been most valorous.

Socrates: Thank you, friend; and I hope that you observed Protagoras
bidding us be serious, as the text, `Man is the measure of all things,'
was a solemn one; and he reproached us with making a boy the medium of
discourse, and said that the boy's timidity was made to tell against his
argument; he also declared that we made a joke of him.

Theodorus: How could I fail to observe all that, Socrates?

Socrates: Well, and shall we do as he says?

Theodorus: By all means.

Socrates: But if his wishes are to be regarded, you and I must take up
the argument, and in all seriousness, and ask and answer one another,
for you see that the rest of us are nothing but boys. In no other way
can we escape the imputation, that in our fresh analysis of his thesis
we are making fun with boys.

Theodorus: Well, but is not Theaetetus better able to follow a
philosophical enquiry than a great many men who have long beards?

Socrates: Yes, Theodorus, but not better than you; and therefore please
not to imagine that I am to defend by every means in my power your
departed friend; and that you are to defend nothing and nobody. At any
rate, my good man, do not sheer off until we know whether you are a
true measure of diagrams, or whether all men are equally measures and
sufficient for themselves in astronomy and geometry, and the other
branches of knowledge in which you are supposed to excel them.

Theodorus: He who is sitting by you, Socrates, will not easily avoid
being drawn into an argument; and when I said just now that you would
excuse me, and not, like the Lacedaemonians, compel me to strip and
fight, I was talking nonsense--I should rather compare you to Scirrhon,
who threw travellers from the rocks; for the Lacedaemonian rule is
`strip or depart,' but you seem to go about your work more after the
fashion of Antaeus: you will not allow any one who approaches you to
depart until you have stripped him, and he has been compelled to try a
fall with you in argument.

Socrates: There, Theodorus, you have hit off precisely the nature of my
complaint; but I am even more pugnacious than the giants of old, for I
have met with no end of heroes; many a Heracles, many a Theseus, mighty
in words, has broken my head; nevertheless I am always at this rough
exercise, which inspires me like a passion. Please, then, to try a fall
with me, whereby you will do yourself good as well as me.

Theodorus: I consent; lead me whither you will, for I know that you are
like destiny; no man can escape from any argument which you may weave
for him. But I am not disposed to go further than you suggest.

Socrates: Once will be enough; and now take particular care that we
do not again unwittingly expose ourselves to the reproach of talking
childishly.

Theodorus: I will do my best to avoid that error.

Socrates: In the first place, let us return to our old objection, and
see whether we were right in blaming and taking offence at Protagoras
on the ground that he assumed all to be equal and sufficient in wisdom;
although he admitted that there was a better and worse, and that in
respect of this, some who as he said were the wise excelled others.

Theodorus: Very true.

Socrates: Had Protagoras been living and answered for himself, instead
of our answering for him, there would have been no need of our reviewing
or reinforcing the argument. But as he is not here, and some one may
accuse us of speaking without authority on his behalf, had we not better
come to a clearer agreement about his meaning, for a great deal may be
at stake?

Theodorus: True.

Socrates: Then let us obtain, not through any third person, but from his
own statement and in the fewest words possible, the basis of agreement.

Theodorus: In what way?

Socrates: In this way:--His words are, `What seems to a man, is to him.'

Theodorus: Yes, so he says.

Socrates: And are not we, Protagoras, uttering the opinion of man, or
rather of all mankind, when we say that every one thinks himself wiser
than other men in some things, and their inferior in others? In the
hour of danger, when they are in perils of war, or of the sea, or of
sickness, do they not look up to their commanders as if they were
gods, and expect salvation from them, only because they excel them in
knowledge? Is not the world full of men in their several employments,
who are looking for teachers and rulers of themselves and of the
animals? and there are plenty who think that they are able to teach
and able to rule. Now, in all this is implied that ignorance and wisdom
exist among them, at least in their own opinion.

Theodorus: Certainly.

Socrates: And wisdom is assumed by them to be true thought, and
ignorance to be false opinion.

Theodorus: Exactly.

Socrates: How then, Protagoras, would you have us treat the argument?
Shall we say that the opinions of men are always true, or sometimes true
and sometimes false? In either case, the result is the same, and their
opinions are not always true, but sometimes true and sometimes false.
For tell me, Theodorus, do you suppose that you yourself, or any other
follower of Protagoras, would contend that no one deems another ignorant
or mistaken in his opinion?

Theodorus: The thing is incredible, Socrates.

Socrates: And yet that absurdity is necessarily involved in the thesis
which declares man to be the measure of all things.

Theodorus: How so?

Socrates: Why, suppose that you determine in your own mind something to
be true, and declare your opinion to me; let us assume, as he argues,
that this is true to you. Now, if so, you must either say that the rest
of us are not the judges of this opinion or judgment of yours, or that
we judge you always to have a true opinion? But are there not thousands
upon thousands who, whenever you form a judgment, take up arms against
you and are of an opposite judgment and opinion, deeming that you judge
falsely?

Theodorus: Yes, indeed, Socrates, thousands and tens of thousands, as
Homer says, who give me a world of trouble.

Socrates: Well, but are we to assert that what you think is true to you
and false to the ten thousand others?

Theodorus: No other inference seems to be possible.

Socrates: And how about Protagoras himself? If neither he nor the
multitude thought, as indeed they do not think, that man is the measure
of all things, must it not follow that the truth of which Protagoras
wrote would be true to no one? But if you suppose that he himself
thought this, and that the multitude does not agree with him, you must
begin by allowing that in whatever proportion the many are more than
one, in that proportion his truth is more untrue than true.

Theodorus: That would follow if the truth is supposed to vary with
individual opinion.

Socrates: And the best of the joke is, that he acknowledges the truth
of their opinion who believe his own opinion to be false; for he admits
that the opinions of all men are true.

Theodorus: Certainly.

Socrates: And does he not allow that his own opinion is false, if he
admits that the opinion of those who think him false is true?

Theodorus: Of course.

Socrates: Whereas the other side do not admit that they speak falsely?

Theodorus: They do not.

Socrates: And he, as may be inferred from his writings, agrees that this
opinion is also true.

Theodorus: Clearly.

Socrates: Then all mankind, beginning with Protagoras, will contend,
or rather, I should say that he will allow, when he concedes that his
adversary has a true opinion--Protagoras, I say, will himself allow that
neither a dog nor any ordinary man is the measure of anything which he
has not learned--am I not right?

Theodorus: Yes.

Socrates: And the truth of Protagoras being doubted by all, will be true
neither to himself to any one else?

Theodorus: I think, Socrates, that we are running my old friend too
hard.

Socrates: But I do not know that we are going beyond the truth.
Doubtless, as he is older, he may be expected to be wiser than we are.
And if he could only just get his head out of the world below, he would
have overthrown both of us again and again, me for talking nonsense and
you for assenting to me, and have been off and underground in a trice.
But as he is not within call, we must make the best use of our own
faculties, such as they are, and speak out what appears to us to be
true. And one thing which no one will deny is, that there are great
differences in the understandings of men.

Theodorus: In that opinion I quite agree.

Socrates: And is there not most likely to be firm ground in the
distinction which we were indicating on behalf of Protagoras, viz. that
most things, and all immediate sensations, such as hot, dry, sweet,
are only such as they appear; if however difference of opinion is to be
allowed at all, surely we must allow it in respect of health or disease?
for every woman, child, or living creature has not such a knowledge of
what conduces to health as to enable them to cure themselves.

Theodorus: I quite agree.

Socrates: Or again, in politics, while affirming that just and unjust,
honourable and disgraceful, holy and unholy, are in reality to each
state such as the state thinks and makes lawful, and that in determining
these matters no individual or state is wiser than another, still the
followers of Protagoras will not deny that in determining what is or is
not expedient for the community one state is wiser and one counsellor
better than another--they will scarcely venture to maintain, that what
a city enacts in the belief that it is expedient will always be really
expedient. But in the other case, I mean when they speak of justice and
injustice, piety and impiety, they are confident that in nature these
have no existence or essence of their own--the truth is that which is
agreed on at the time of the agreement, and as long as the agreement
lasts; and this is the philosophy of many who do not altogether go along
with Protagoras. Here arises a new question, Theodorus, which threatens
to be more serious than the last.

Theodorus: Well, Socrates, we have plenty of leisure.

Socrates: That is true, and your remark recalls to my mind an
observation which I have often made, that those who have passed their
days in the pursuit of philosophy are ridiculously at fault when they
have to appear and speak in court. How natural is this!

Theodorus: What do you mean?

Socrates: I mean to say, that those who have been trained in philosophy
and liberal pursuits are as unlike those who from their youth upwards
have been knocking about in the courts and such places, as a freeman is
in breeding unlike a slave.

Theodorus: In what is the difference seen?

Socrates: In the leisure spoken of by you, which a freeman can always
command: he has his talk out in peace, and, like ourselves, he wanders
at will from one subject to another, and from a second to a third,--if
the fancy takes him, he begins again, as we are doing now, caring not
whether his words are many or few; his only aim is to attain the truth.
But the lawyer is always in a hurry; there is the water of the clepsydra
driving him on, and not allowing him to expatiate at will: and there is
his adversary standing over him, enforcing his rights; the indictment,
which in their phraseology is termed the affidavit, is recited at
the time: and from this he must not deviate. He is a servant, and is
continually disputing about a fellow-servant before his master, who is
seated, and has the cause in his hands; the trial is never about some
indifferent matter, but always concerns himself; and often the race
is for his life. The consequence has been, that he has become keen and
shrewd; he has learned how to flatter his master in word and indulge him
in deed; but his soul is small and unrighteous. His condition, which has
been that of a slave from his youth upwards, has deprived him of growth
and uprightness and independence; dangers and fears, which were too
much for his truth and honesty, came upon him in early years, when the
tenderness of youth was unequal to them, and he has been driven into
crooked ways; from the first he has practised deception and retaliation,
and has become stunted and warped. And so he has passed out of youth
into manhood, having no soundness in him; and is now, as he thinks,
a master in wisdom. Such is the lawyer, Theodorus. Will you have the
companion picture of the philosopher, who is of our brotherhood; or
shall we return to the argument? Do not let us abuse the freedom of
digression which we claim.

Theodorus: Nay, Socrates, not until we have finished what we are about;
for you truly said that we belong to a brotherhood which is free, and
are not the servants of the argument; but the argument is our servant,
and must wait our leisure. Who is our judge? Or where is the spectator
having any right to censure or control us, as he might the poets?

Socrates: Then, as this is your wish, I will describe the leaders; for
there is no use in talking about the inferior sort. In the first place,
the lords of philosophy have never, from their youth upwards, known
their way to the Agora, or the dicastery, or the council, or any other
political assembly; they neither see nor hear the laws or decrees,
as they are called, of the state written or recited; the eagerness of
political societies in the attainment of offices--clubs, and banquets,
and revels, and singing-maidens,--do not enter even into their dreams.
Whether any event has turned out well or ill in the city, what disgrace
may have descended to any one from his ancestors, male or female, are
matters of which the philosopher no more knows than he can tell, as they
say, how many pints are contained in the ocean. Neither is he conscious
of his ignorance. For he does not hold aloof in order that he may gain a
reputation; but the truth is, that the outer form of him only is in the
city: his mind, disdaining the littlenesses and nothingnesses of human
things, is `flying all abroad' as Pindar says, measuring earth and
heaven and the things which are under and on the earth and above
the heaven, interrogating the whole nature of each and all in their
entirety, but not condescending to anything which is within reach.

Theodorus: What do you mean, Socrates?

Socrates: I will illustrate my meaning, Theodorus, by the jest which the
clever witty Thracian handmaid is said to have made about Thales, when
he fell into a well as he was looking up at the stars. She said, that he
was so eager to know what was going on in heaven, that he could not see
what was before his feet. This is a jest which is equally applicable to
all philosophers. For the philosopher is wholly unacquainted with his
next-door neighbour; he is ignorant, not only of what he is doing, but
he hardly knows whether he is a man or an animal; he is searching into
the essence of man, and busy in enquiring what belongs to such a nature
to do or suffer different from any other;--I think that you understand
me, Theodorus?

Theodorus: I do, and what you say is true.

Socrates: And thus, my friend, on every occasion, private as well as
public, as I said at first, when he appears in a law-court, or in any
place in which he has to speak of things which are at his feet and
before his eyes, he is the jest, not only of Thracian handmaids but of
the general herd, tumbling into wells and every sort of disaster through
his inexperience. His awkwardness is fearful, and gives the impression
of imbecility. When he is reviled, he has nothing personal to say in
answer to the civilities of his adversaries, for he knows no scandals
of any one, and they do not interest him; and therefore he is laughed at
for his sheepishness; and when others are being praised and glorified,
in the simplicity of his heart he cannot help going into fits of
laughter, so that he seems to be a downright idiot. When he hears a
tyrant or king eulogized, he fancies that he is listening to the
praises of some keeper of cattle--a swineherd, or shepherd, or perhaps a
cowherd, who is congratulated on the quantity of milk which he squeezes
from them; and he remarks that the creature whom they tend, and out of
whom they squeeze the wealth, is of a less tractable and more insidious
nature. Then, again, he observes that the great man is of necessity as
ill-mannered and uneducated as any shepherd--for he has no leisure,
and he is surrounded by a wall, which is his mountain-pen. Hearing
of enormous landed proprietors of ten thousand acres and more, our
philosopher deems this to be a trifle, because he has been accustomed to
think of the whole earth; and when they sing the praises of family, and
say that some one is a gentleman because he can show seven generations
of wealthy ancestors, he thinks that their sentiments only betray a
dull and narrow vision in those who utter them, and who are not educated
enough to look at the whole, nor to consider that every man has had
thousands and ten thousands of progenitors, and among them have been
rich and poor, kings and slaves, Hellenes and barbarians, innumerable.
And when people pride themselves on having a pedigree of twenty-five
ancestors, which goes back to Heracles, the son of Amphitryon, he cannot
understand their poverty of ideas. Why are they unable to calculate that
Amphitryon had a twenty-fifth ancestor, who might have been anybody,
and was such as fortune made him, and he had a fiftieth, and so on? He
amuses himself with the notion that they cannot count, and thinks that a
little arithmetic would have got rid of their senseless vanity. Now, in
all these cases our philosopher is derided by the vulgar, partly because
he is thought to despise them, and also because he is ignorant of what
is before him, and always at a loss.

Theodorus: That is very true, Socrates.

Socrates: But, O my friend, when he draws the other into upper air,
and gets him out of his pleas and rejoinders into the contemplation of
justice and injustice in their own nature and in their difference from
one another and from all other things; or from the commonplaces about
the happiness of a king or of a rich man to the consideration of
government, and of human happiness and misery in general--what they
are, and how a man is to attain the one and avoid the other--when that
narrow, keen, little legal mind is called to account about all this, he
gives the philosopher his revenge; for dizzied by the height at which
he is hanging, whence he looks down into space, which is a strange
experience to him, he being dismayed, and lost, and stammering
broken words, is laughed at, not by Thracian handmaidens or any other
uneducated persons, for they have no eye for the situation, but by every
man who has not been brought up a slave. Such are the two characters,
Theodorus: the one of the freeman, who has been trained in liberty and
leisure, whom you call the philosopher,--him we cannot blame because
he appears simple and of no account when he has to perform some menial
task, such as packing up bed-clothes, or flavouring a sauce or fawning
speech; the other character is that of the man who is able to do all
this kind of service smartly and neatly, but knows not how to wear his
cloak like a gentleman; still less with the music of discourse can he
hymn the true life aright which is lived by immortals or men blessed of
heaven.

Theodorus: If you could only persuade everybody, Socrates, as you do me,
of the truth of your words, there would be more peace and fewer evils
among men.

Socrates: Evils, Theodorus, can never pass away; for there must always
remain something which is antagonistic to good. Having no place among
the gods in heaven, of necessity they hover around the mortal nature,
and this earthly sphere. Wherefore we ought to fly away from earth to
heaven as quickly as we can; and to fly away is to become like God,
as far as this is possible; and to become like him, is to become holy,
just, and wise. But, O my friend, you cannot easily convince mankind
that they should pursue virtue or avoid vice, not merely in order that a
man may seem to be good, which is the reason given by the world, and in
my judgment is only a repetition of an old wives' fable. Whereas,
the truth is that God is never in any way unrighteous--he is perfect
righteousness; and he of us who is the most righteous is most like him.
Herein is seen the true cleverness of a man, and also his nothingness
and want of manhood. For to know this is true wisdom and virtue, and
ignorance of this is manifest folly and vice. All other kinds of wisdom
or cleverness, which seem only, such as the wisdom of politicians, or
the wisdom of the arts, are coarse and vulgar. The unrighteous man, or
the sayer and doer of unholy things, had far better not be encouraged
in the illusion that his roguery is clever; for men glory in their
shame--they fancy that they hear others saying of them, `These are not
mere good-for-nothing persons, mere burdens of the earth, but such as
men should be who mean to dwell safely in a state.' Let us tell them
that they are all the more truly what they do not think they are because
they do not know it; for they do not know the penalty of injustice,
which above all things they ought to know--not stripes and death, as
they suppose, which evil-doers often escape, but a penalty which cannot
be escaped.

Theodorus: What is that?

Socrates: There are two patterns eternally set before them; the one
blessed and divine, the other godless and wretched: but they do not see
them, or perceive that in their utter folly and infatuation they are
growing like the one and unlike the other, by reason of their evil
deeds; and the penalty is, that they lead a life answering to the
pattern which they are growing like. And if we tell them, that unless
they depart from their cunning, the place of innocence will not receive
them after death; and that here on earth, they will live ever in the
likeness of their own evil selves, and with evil friends--when they hear
this they in their superior cunning will seem to be listening to the
talk of idiots.

Theodorus: Very true, Socrates.

Socrates: Too true, my friend, as I well know; there is, however, one
peculiarity in their case: when they begin to reason in private about
their dislike of philosophy, if they have the courage to hear the
argument out, and do not run away, they grow at last strangely
discontented with themselves; their rhetoric fades away, and they become
helpless as children. These however are digressions from which we must
now desist, or they will overflow, and drown the original argument; to
which, if you please, we will now return.

Theodorus: For my part, Socrates, I would rather have the digressions,
for at my age I find them easier to follow; but if you wish, let us go
back to the argument.

Socrates: Had we not reached the point at which the partisans of the
perpetual flux, who say that things are as they seem to each one, were
confidently maintaining that the ordinances which the state commanded
and thought just, were just to the state which imposed them, while they
were in force; this was especially asserted of justice; but as to the
good, no one had any longer the hardihood to contend of any ordinances
which the state thought and enacted to be good that these, while they
were in force, were really good;--he who said so would be playing with
the name `good,' and would not touch the real question--it would be a
mockery, would it not?

Theodorus: Certainly it would.

Socrates: He ought not to speak of the name, but of the thing which is
contemplated under the name.

Theodorus: Right.

Socrates: Whatever be the term used, the good or expedient is the aim
of legislation, and as far as she has an opinion, the state imposes all
laws with a view to the greatest expediency; can legislation have any
other aim?

Theodorus: Certainly not.

Socrates: But is the aim attained always? do not mistakes often happen?

Theodorus: Yes, I think that there are mistakes.

Socrates: The possibility of error will be more distinctly recognised,
if we put the question in reference to the whole class under which the
good or expedient falls. That whole class has to do with the future, and
laws are passed under the idea that they will be useful in after-time;
which, in other words, is the future.

Theodorus: Very true.

Socrates: Suppose now, that we ask Protagoras, or one of his disciples,
a question:--O, Protagoras, we will say to him, Man is, as you declare,
the measure of all things--white, heavy, light: of all such things he
is the judge; for he has the criterion of them in himself, and when he
thinks that things are such as he experiences them to be, he thinks what
is and is true to himself. Is it not so?

Theodorus: Yes.

Socrates: And do you extend your doctrine, Protagoras (as we shall
further say), to the future as well as to the present; and has he the
criterion not only of what in his opinion is but of what will be, and do
things always happen to him as he expected? For example, take the case
of heat:--When an ordinary man thinks that he is going to have a fever,
and that this kind of heat is coming on, and another person, who is a
physician, thinks the contrary, whose opinion is likely to prove right?
Or are they both right?--he will have a heat and fever in his own
judgment, and not have a fever in the physician's judgment?

Theodorus: How ludicrous!

Socrates: And the vinegrower, if I am not mistaken, is a better judge of
the sweetness or dryness of the vintage which is not yet gathered than
the harp-player?

Theodorus: Certainly.

Socrates: And in musical composition the musician will know better than
the training master what the training master himself will hereafter
think harmonious or the reverse?

Theodorus: Of course.

Socrates: And the cook will be a better judge than the guest, who is
not a cook, of the pleasure to be derived from the dinner which is in
preparation; for of present or past pleasure we are not as yet arguing;
but can we say that every one will be to himself the best judge of the
pleasure which will seem to be and will be to him in the future?--nay,
would not you, Protagoras, better guess which arguments in a court would
convince any one of us than the ordinary man?

Theodorus: Certainly, Socrates, he used to profess in the strongest
manner that he was the superior of all men in this respect.

Socrates: To be sure, friend: who would have paid a large sum for the
privilege of talking to him, if he had really persuaded his visitors
that neither a prophet nor any other man was better able to judge what
will be and seem to be in the future than every one could for himself?

Theodorus: Who indeed?

Socrates: And legislation and expediency are all concerned with the
future; and every one will admit that states, in passing laws, must
often fail of their highest interests?

Theodorus: Quite true.

Socrates: Then we may fairly argue against your master, that he must
admit one man to be wiser than another, and that the wiser is a measure:
but I, who know nothing, am not at all obliged to accept the honour
which the advocate of Protagoras was just now forcing upon me, whether I
would or not, of being a measure of anything.

Theodorus: That is the best refutation of him, Socrates; although he is
also caught when he ascribes truth to the opinions of others, who give
the lie direct to his own opinion.

Socrates: There are many ways, Theodorus, in which the doctrine that
every opinion of every man is true may be refuted; but there is more
difficulty in proving that states of feeling, which are present to a
man, and out of which arise sensations and opinions in accordance with
them, are also untrue. And very likely I have been talking nonsense
about them; for they may be unassailable, and those who say that there
is clear evidence of them, and that they are matters of knowledge, may
probably be right; in which case our friend Theaetetus was not so far
from the mark when he identified perception and knowledge. And therefore
let us draw nearer, as the advocate of Protagoras desires; and give the
truth of the universal flux a ring: is the theory sound or not? at any
rate, no small war is raging about it, and there are combination not a
few.

Theodorus: No small, war, indeed, for in Ionia the sect makes rapid
strides; the disciples of Heracleitus are most energetic upholders of
the doctrine.

Socrates: Then we are the more bound, my dear Theodorus, to examine the
question from the foundation as it is set forth by themselves.

Theodorus: Certainly we are. About these speculations of Heracleitus,
which, as you say, are as old as Homer, or even older still, the
Ephesians themselves, who profess to know them, are downright mad, and
you cannot talk with them on the subject. For, in accordance with their
text-books, they are always in motion; but as for dwelling upon an
argument or a question, and quietly asking and answering in turn, they
can no more do so than they can fly; or rather, the determination of
these fellows not to have a particle of rest in them is more than
the utmost powers of negation can express. If you ask any of them a
question, he will produce, as from a quiver, sayings brief and dark, and
shoot them at you; and if you inquire the reason of what he has said,
you will be hit by some other new-fangled word, and will make no way
with any of them, nor they with one another; their great care is, not
to allow of any settled principle either in their arguments or in
their minds, conceiving, as I imagine, that any such principle would be
stationary; for they are at war with the stationary, and do what they
can to drive it out everywhere.

Socrates: I suppose, Theodorus, that you have only seen them when they
were fighting, and have never stayed with them in time of peace,
for they are no friends of yours; and their peace doctrines are only
communicated by them at leisure, as I imagine, to those disciples of
theirs whom they want to make like themselves.

Theodorus: Disciples! my good sir, they have none; men of their sort are
not one another's disciples, but they grow up at their own sweet will,
and get their inspiration anywhere, each of them saying of his neighbour
that he knows nothing. From these men, then, as I was going to remark,
you will never get a reason, whether with their will or without their
will; we must take the question out of their hands, and make the
analysis ourselves, as if we were doing geometrical problem.

Socrates: Quite right too; but as touching the aforesaid problem, have
we not heard from the ancients, who concealed their wisdom from the many
in poetical figures, that Oceanus and Tethys, the origin of all things,
are streams, and that nothing is at rest? And now the moderns, in their
superior wisdom, have declared the same openly, that the cobbler too may
hear and learn of them, and no longer foolishly imagine that some things
are at rest and others in motion--having learned that all is motion,
he will duly honour his teachers. I had almost forgotten the opposite
doctrine, Theodorus,

     `Alone Being remains unmoved, which is the name for the all.'

This is the language of Parmenides, Melissus, and their followers, who
stoutly maintain that all being is one and self-contained, and has no
place in which to move. What shall we do, friend, with all these people;
for, advancing step by step, we have imperceptibly got between the
combatants, and, unless we can protect our retreat, we shall pay the
penalty of our rashness--like the players in the palaestra who are
caught upon the line, and are dragged different ways by the two parties.
Therefore I think that we had better begin by considering those whom we
first accosted, `the river-gods,' and, if we find any truth in them, we
will help them to pull us over, and try to get away from the others. But
if the partisans of `the whole' appear to speak more truly, we will fly
off from the party which would move the immovable, to them. And if I
find that neither of them have anything reasonable to say, we shall
be in a ridiculous position, having so great a conceit of our own poor
opinion and rejecting that of ancient and famous men. O Theodorus, do
you think that there is any use in proceeding when the danger is so
great?

Theodorus: Nay, Socrates, not to examine thoroughly what the two parties
have to say would be quite intolerable.

Socrates: Then examine we must, since you, who were so reluctant to
begin, are so eager to proceed. The nature of motion appears to be the
question with which we begin. What do they mean when they say that all
things are in motion? Is there only one kind of motion, or, as I rather
incline to think, two? I should like to have your opinion upon this
point in addition to my own, that I may err, if I must err, in your
company; tell me, then, when a thing changes from one place to another,
or goes round in the same place, is not that what is called motion?

Theodorus: Yes.

Socrates: Here then we have one kind of motion. But when a thing,
remaining on the same spot, grows old, or becomes black from being
white, or hard from being soft, or undergoes any other change, may not
this be properly called motion of another kind?

Theodorus: I think so.

Socrates: Say rather that it must be so. Of motion then there are these
two kinds, `change,' and `motion in place.'

Theodorus: You are right.

Socrates: And now, having made this distinction, let us address
ourselves to those who say that all is motion, and ask them whether all
things according to them have the two kinds of motion, and are changed
as well as move in place, or is one thing moved in both ways, and
another in one only?

Theodorus: Indeed, I do not know what to answer; but I think they would
say that all things are moved in both ways.

Socrates: Yes, comrade; for, if not, they would have to say that the
same things are in motion and at rest, and there would be no more truth
in saying that all things are in motion, than that all things are at
rest.

Theodorus: To be sure.

Socrates: And if they are to be in motion, and nothing is to be devoid
of motion, all things must always have every sort of motion?

Theodorus: Most true.

Socrates: Consider a further point: did we not understand them to
explain the generation of heat, whiteness, or anything else, in some
such manner as the following:--were they not saying that each of them
is moving between the agent and the patient, together with a perception,
and that the patient ceases to be a perceiving power and becomes a
percipient, and the agent a quale instead of a quality? I suspect that
quality may appear a strange and uncouth term to you, and that you
do not understand the abstract expression. Then I will take concrete
instances: I mean to say that the producing power or agent becomes
neither heat nor whiteness but hot and white, and the like of other
things. For I must repeat what I said before, that neither the agent
nor patient have any absolute existence, but when they come together
and generate sensations and their objects, the one becomes a thing of a
certain quality, and the other a percipient. You remember?

Theodorus: Of course.

Socrates: We may leave the details of their theory unexamined, but
we must not forget to ask them the only question with which we are
concerned: Are all things in motion and flux?

Theodorus: Yes, they will reply.

Socrates: And they are moved in both those ways which we distinguished,
that is to say, they move in place and are also changed?

Theodorus: Of course, if the motion is to be perfect.

Socrates: If they only moved in place and were not changed, we should
be able to say what is the nature of the things which are in motion and
flux?

Theodorus: Exactly.

Socrates: But now, since not even white continues to flow white, and
whiteness itself is a flux or change which is passing into another
colour, and is never to be caught standing still, can the name of any
colour be rightly used at all?

Theodorus: How is that possible, Socrates, either in the case of this
or of any other quality--if while we are using the word the object is
escaping in the flux?

Socrates: And what would you say of perceptions, such as sight and
hearing, or any other kind of perception? Is there any stopping in the
act of seeing and hearing?

Theodorus: Certainly not, if all things are in motion.

Socrates: Then we must not speak of seeing any more than of not-seeing,
nor of any other perception more than of any non-perception, if all
things partake of every kind of motion?

Theodorus: Certainly not.

Socrates: Yet perception is knowledge: so at least Theaetetus and I were
saying.

Theodorus: Very true.

Socrates: Then when we were asked what is knowledge, we no more answered
what is knowledge than what is not knowledge?

Theodorus: I suppose not.

Socrates: Here, then, is a fine result: we corrected our first answer
in our eagerness to prove that nothing is at rest. But if nothing is at
rest, every answer upon whatever subject is equally right: you may say
that a thing is or is not thus; or, if you prefer, `becomes' thus; and
if we say `becomes,' we shall not then hamper them with words expressive
of rest.

Theodorus: Quite true.

Socrates: Yes, Theodorus, except in saying `thus' and `not thus.' But
you ought not to use the word `thus,' for there is no motion in `thus'
or in `not thus.' The maintainers of the doctrine have as yet no words
in which to express themselves, and must get a new language. I know of
no word that will suit them, except perhaps `no how,' which is perfectly
indefinite.

Theodorus: Yes, that is a manner of speaking in which they will be quite
at home.

Socrates: And so, Theodorus, we have got rid of your friend without
assenting to his doctrine, that every man is the measure of all
things--a wise man only is a measure; neither can we allow that
knowledge is perception, certainly not on the hypothesis of a perpetual
flux, unless perchance our friend Theaetetus is able to convince us that
it is.

Theodorus: Very good, Socrates; and now that the argument about the
doctrine of Protagoras has been completed, I am absolved from answering;
for this was the agreement.

Theaetetus: Not, Theodorus, until you and Socrates have discussed the
doctrine of those who say that all things are at rest, as you were
proposing.

Theodorus: You, Theaetetus, who are a young rogue, must not instigate
your elders to a breach of faith, but should prepare to answer Socrates
in the remainder of the argument.

Theaetetus: Yes, if he wishes; but I would rather have heard about the
doctrine of rest.

Theodorus: Invite Socrates to an argument--invite horsemen to the open
plain; do but ask him, and he will answer.

Socrates: Nevertheless, Theodorus, I am afraid that I shall not be able
to comply with the request of Theaetetus.

Theodorus: Not comply! for what reason?

Socrates: My reason is that I have a kind of reverence; not so much for
Melissus and the others, who say that `All is one and at rest,' as for
the great leader himself, Parmenides, venerable and awful, as in Homeric
language he may be called;--him I should be ashamed to approach in a
spirit unworthy of him. I met him when he was an old man, and I was a
mere youth, and he appeared to me to have a glorious depth of mind.
And I am afraid that we may not understand his words, and may be still
further from understanding his meaning; above all I fear that the nature
of knowledge, which is the main subject of our discussion, may be thrust
out of sight by the unbidden guests who will come pouring in upon our
feast of discourse, if we let them in--besides, the question which is
now stirring is of immense extent, and will be treated unfairly if only
considered by the way; or if treated adequately and at length, will put
into the shade the other question of knowledge. Neither the one nor the
other can be allowed; but I must try by my art of midwifery to deliver
Theaetetus of his conceptions about knowledge.

Theaetetus: Very well; do so if you will.

Socrates: Then now, Theaetetus, take another view of the subject: you
answered that knowledge is perception?

Theaetetus: I did.

Socrates: And if any one were to ask you: With what does a man see black
and white colours? and with what does he hear high and low sounds?--you
would say, if I am not mistaken, `With the eyes and with the ears.'

Theaetetus: I should.

Socrates: The free use of words and phrases, rather than minute
precision, is generally characteristic of a liberal education, and
the opposite is pedantic; but sometimes precision is necessary, and I
believe that the answer which you have just given is open to the charge
of incorrectness; for which is more correct, to say that we see or hear
with the eyes and with the ears, or through the eyes and through the
ears.

Theaetetus: I should say `through,' Socrates, rather than `with.'

Socrates: Yes, my boy, for no one can suppose that in each of us, as
in a sort of Trojan horse, there are perched a number of unconnected
senses, which do not all meet in some one nature, the mind, or whatever
we please to call it, of which they are the instruments, and with which
through them we perceive objects of sense.

Theaetetus: I agree with you in that opinion.

Socrates: The reason why I am thus precise is, because I want to know
whether, when we perceive black and white through the eyes, and again,
other qualities through other organs, we do not perceive them with one
and the same part of ourselves, and, if you were asked, you might refer
all such perceptions to the body. Perhaps, however, I had better allow
you to answer for yourself and not interfere. Tell me, then, are not
the organs through which you perceive warm and hard and light and sweet,
organs of the body?

Theaetetus: Of the body, certainly.

Socrates: And you would admit that what you perceive through one
faculty you cannot perceive through another; the objects of hearing,
for example, cannot be perceived through sight, or the objects of sight
through hearing?

Theaetetus: Of course not.

Socrates: If you have any thought about both of them, this common
perception cannot come to you, either through the one or the other
organ?

Theaetetus: It cannot.

Socrates: How about sounds and colours: in the first place you would
admit that they both exist?

Theaetetus: Yes.

Socrates: And that either of them is different from the other, and the
same with itself?

Theaetetus: Certainly.

Socrates: And that both are two and each of them one?

Theaetetus: Yes.

Socrates: You can further observe whether they are like or unlike one
another?

Theaetetus: I dare say.

Socrates: But through what do you perceive all this about them? for
neither through hearing nor yet through seeing can you apprehend that
which they have in common. Let me give you an illustration of the
point at issue:--If there were any meaning in asking whether sounds and
colours are saline or not, you would be able to tell me what faculty
would consider the question. It would not be sight or hearing, but some
other.

Theaetetus: Certainly; the faculty of taste.

Socrates: Very good; and now tell me what is the power which discerns,
not only in sensible objects, but in all things, universal notions, such
as those which are called being and not-being, and those others
about which we were just asking--what organs will you assign for the
perception of these notions?

Theaetetus: You are thinking of being and not being, likeness and
unlikeness, sameness and difference, and also of unity and other numbers
which are applied to objects of sense; and you mean to ask, through
what bodily organ the soul perceives odd and even numbers and other
arithmetical conceptions.

Socrates: You follow me excellently, Theaetetus; that is precisely what
I am asking.

Theaetetus: Indeed, Socrates, I cannot answer; my only notion is, that
these, unlike objects of sense, have no separate organ, but that the
mind, by a power of her own, contemplates the universals in all things.

Socrates: You are a beauty, Theaetetus, and not ugly, as Theodorus was
saying; for he who utters the beautiful is himself beautiful and good.
And besides being beautiful, you have done me a kindness in releasing me
from a very long discussion, if you are clear that the soul views some
things by herself and others through the bodily organs. For that was my
own opinion, and I wanted you to agree with me.

Theaetetus: I am quite clear.

Socrates: And to which class would you refer being or essence; for this,
of all our notions, is the most universal?

Theaetetus: I should say, to that class which the soul aspires to know
of herself.

Socrates: And would you say this also of like and unlike, same and
other?

Theaetetus: Yes.

Socrates: And would you say the same of the noble and base, and of good
and evil?

Theaetetus: These I conceive to be notions which are essentially
relative, and which the soul also perceives by comparing in herself
things past and present with the future.

Socrates: And does she not perceive the hardness of that which is hard
by the touch, and the softness of that which is soft equally by the
touch?

Theaetetus: Yes.

Socrates: But their essence and what they are, and their opposition
to one another, and the essential nature of this opposition, the soul
herself endeavours to decide for us by the review and comparison of
them?

Theaetetus: Certainly.

Socrates: The simple sensations which reach the soul through the body
are given at birth to men and animals by nature, but their reflections
on the being and use of them are slowly and hardly gained, if they are
ever gained, by education and long experience.

Theaetetus: Assuredly.

Socrates: And can a man attain truth who fails of attaining being?

Theaetetus: Impossible.

Socrates: And can he who misses the truth of anything, have a knowledge
of that thing?

Theaetetus: He cannot.

Socrates: Then knowledge does not consist in impressions of sense, but
in reasoning about them; in that only, and not in the mere impression,
truth and being can be attained?

Theaetetus: Clearly.

Socrates: And would you call the two processes by the same name, when
there is so great a difference between them?

Theaetetus: That would certainly not be right.

Socrates: And what name would you give to seeing, hearing, smelling,
being cold and being hot?

Theaetetus: I should call all of them perceiving--what other name could
be given to them?

Socrates: Perception would be the collective name of them?

Theaetetus: Certainly.

Socrates: Which, as we say, has no part in the attainment of truth any
more than of being?

Theaetetus: Certainly not.

Socrates: And therefore not in science or knowledge?

Theaetetus: No.

Socrates: Then perception, Theaetetus, can never be the same as
knowledge or science?

Theaetetus: Clearly not, Socrates; and knowledge has now been most
distinctly proved to be different from perception.

Socrates: But the original aim of our discussion was to find out rather
what knowledge is than what it is not; at the same time we have made
some progress, for we no longer seek for knowledge in perception at all,
but in that other process, however called, in which the mind is alone
and engaged with being.

Theaetetus: You mean, Socrates, if I am not mistaken, what is called
thinking or opining.

Socrates: You conceive truly. And now, my friend, please to begin
again at this point; and having wiped out of your memory all that has
preceded, see if you have arrived at any clearer view, and once more say
what is knowledge.

Theaetetus: I cannot say, Socrates, that all opinion is knowledge,
because there may be a false opinion; but I will venture to assert, that
knowledge is true opinion: let this then be my reply; and if this is
hereafter disproved, I must try to find another.

Socrates: That is the way in which you ought to answer, Theaetetus, and
not in your former hesitating strain, for if we are bold we shall gain
one of two advantages; either we shall find what we seek, or we shall be
less likely to think that we know what we do not know--in either case we
shall be richly rewarded. And now, what are you saying?--Are there
two sorts of opinion, one true and the other false; and do you define
knowledge to be the true?

Theaetetus: Yes, according to my present view.

Socrates: Is it still worth our while to resume the discussion touching
opinion?

Theaetetus: To what are you alluding?

Socrates: There is a point which often troubles me, and is a great
perplexity to me, both in regard to myself and others. I cannot make out
the nature or origin of the mental experience to which I refer.

Theaetetus: Pray what is it?

Socrates: How there can be false opinion--that difficulty still troubles
the eye of my mind; and I am uncertain whether I shall leave the
question, or begin over again in a new way.

Theaetetus: Begin again, Socrates,--at least if you think that there is
the slightest necessity for doing so. Were not you and Theodorus just
now remarking very truly, that in discussions of this kind we may take
our own time?

Socrates: You are quite right, and perhaps there will be no harm in
retracing our steps and beginning again. Better a little which is well
done, than a great deal imperfectly.

Theaetetus: Certainly.

Socrates: Well, and what is the difficulty? Do we not speak of false
opinion, and say that one man holds a false and another a true opinion,
as though there were some natural distinction between them?

Theaetetus: We certainly say so.

Socrates: All things and everything are either known or not known.
I leave out of view the intermediate conceptions of learning and
forgetting, because they have nothing to do with our present question.

Theaetetus: There can be no doubt, Socrates, if you exclude these, that
there is no other alternative but knowing or not knowing a thing.

Socrates: That point being now determined, must we not say that he who
has an opinion, must have an opinion about something which he knows or
does not know?

Theaetetus: He must.

Socrates: He who knows, cannot but know; and he who does not know,
cannot know?

Theaetetus: Of course.

Socrates: What shall we say then? When a man has a false opinion does
he think that which he knows to be some other thing which he knows, and
knowing both, is he at the same time ignorant of both?

Theaetetus: That, Socrates, is impossible.

Socrates: But perhaps he thinks of something which he does not know as
some other thing which he does not know; for example, he knows neither
Theaetetus nor Socrates, and yet he fancies that Theaetetus is Socrates,
or Socrates Theaetetus?

Theaetetus: How can he?

Socrates: But surely he cannot suppose what he knows to be what he does
not know, or what he does not know to be what he knows?

Theaetetus: That would be monstrous.

Socrates: Where, then, is false opinion? For if all things are either
known or unknown, there can be no opinion which is not comprehended
under this alternative, and so false opinion is excluded.

Theaetetus: Most true.

Socrates: Suppose that we remove the question out of the sphere of
knowing or not knowing, into that of being and not-being.

Theaetetus: What do you mean?

Socrates: May we not suspect the simple truth to be that he who thinks
about anything, that which is not, will necessarily think what is false,
whatever in other respects may be the state of his mind?

Theaetetus: That, again, is not unlikely, Socrates.

Socrates: Then suppose some one to say to us, Theaetetus:--Is
it possible for any man to think that which is not, either as a
self-existent substance or as a predicate of something else? And suppose
that we answer, `Yes, he can, when he thinks what is not true.'--That
will be our answer?

Theaetetus: Yes.

Socrates: But is there any parallel to this?

Theaetetus: What do you mean?

Socrates: Can a man see something and yet see nothing?

Theaetetus: Impossible.

Socrates: But if he sees any one thing, he sees something that exists.
Do you suppose that what is one is ever to be found among non-existing
things?

Theaetetus: I do not.

Socrates: He then who sees some one thing, sees something which is?

Theaetetus: Clearly.

Socrates: And he who hears anything, hears some one thing, and hears
that which is?

Theaetetus: Yes.

Socrates: And he who touches anything, touches something which is one
and therefore is?

Theaetetus: That again is true.

Socrates: And does not he who thinks, think some one thing?

Theaetetus: Certainly.

Socrates: And does not he who thinks some one thing, think something
which is?

Theaetetus: I agree.

Socrates: Then he who thinks of that which is not, thinks of nothing?

Theaetetus: Clearly.

Socrates: And he who thinks of nothing, does not think at all?

Theaetetus: Obviously.

Socrates: Then no one can think that which is not, either as a
self-existent substance or as a predicate of something else?

Theaetetus: Clearly not.

Socrates: Then to think falsely is different from thinking that which is
not?

Theaetetus: It would seem so.

Socrates: Then false opinion has no existence in us, either in the
sphere of being or of knowledge?

Theaetetus: Certainly not.

Socrates: But may not the following be the description of what we
express by this name?

Theaetetus: What?

Socrates: May we not suppose that false opinion or thought is a sort of
heterodoxy; a person may make an exchange in his mind, and say that one
real object is another real object. For thus he always thinks that which
is, but he puts one thing in place of another; and missing the aim of
his thoughts, he may be truly said to have false opinion.

Theaetetus: Now you appear to me to have spoken the exact truth: when a
man puts the base in the place of the noble, or the noble in the place
of the base, then he has truly false opinion.

Socrates: I see, Theaetetus, that your fear has disappeared, and that
you are beginning to despise me.

Theaetetus: What makes you say so?

Socrates: You think, if I am not mistaken, that your `truly false' is
safe from censure, and that I shall never ask whether there can be
a swift which is slow, or a heavy which is light, or any other
self-contradictory thing, which works, not according to its own nature,
but according to that of its opposite. But I will not insist upon this,
for I do not wish needlessly to discourage you. And so you are satisfied
that false opinion is heterodoxy, or the thought of something else?

Theaetetus: I am.

Socrates: It is possible then upon your view for the mind to conceive of
one thing as another?

Theaetetus: True.

Socrates: But must not the mind, or thinking power, which misplaces
them, have a conception either of both objects or of one of them?

Theaetetus: Certainly.

Socrates: Either together or in succession?

Theaetetus: Very good.

Socrates: And do you mean by conceiving, the same which I mean?

Theaetetus: What is that?

Socrates: I mean the conversation which the soul holds with herself in
considering of anything. I speak of what I scarcely understand; but the
soul when thinking appears to me to be just talking--asking questions
of herself and answering them, affirming and denying. And when she has
arrived at a decision, either gradually or by a sudden impulse, and has
at last agreed, and does not doubt, this is called her opinion. I
say, then, that to form an opinion is to speak, and opinion is a word
spoken,--I mean, to oneself and in silence, not aloud or to another:
What think you?

Theaetetus: I agree.

Socrates: Then when any one thinks of one thing as another, he is saying
to himself that one thing is another?

Theaetetus: Yes.

Socrates: But do you ever remember saying to yourself that the noble
is certainly base, or the unjust just; or, best of all--have you ever
attempted to convince yourself that one thing is another? Nay, not even
in sleep, did you ever venture to say to yourself that odd is even, or
anything of the kind?

Theaetetus: Never.

Socrates: And do you suppose that any other man, either in his senses
or out of them, ever seriously tried to persuade himself that an ox is a
horse, or that two are one?

Theaetetus: Certainly not.

Socrates: But if thinking is talking to oneself, no one speaking and
thinking of two objects, and apprehending them both in his soul, will
say and think that the one is the other of them, and I must add, that
even you, lover of dispute as you are, had better let the word `other'
alone (i.e. not insist that `one' and `other' are the same (Both words
in Greek are called eteron: compare Parmen.; Euthyd.)). I mean to say,
that no one thinks the noble to be base, or anything of the kind.

Theaetetus: I will give up the word `other,' Socrates; and I agree to
what you say.

Socrates: If a man has both of them in his thoughts, he cannot think
that the one of them is the other?

Theaetetus: True.

Socrates: Neither, if he has one of them only in his mind and not the
other, can he think that one is the other?

Theaetetus: True; for we should have to suppose that he apprehends that
which is not in his thoughts at all.

Socrates: Then no one who has either both or only one of the two objects
in his mind can think that the one is the other. And therefore, he who
maintains that false opinion is heterodoxy is talking nonsense; for
neither in this, any more than in the previous way, can false opinion
exist in us.

Theaetetus: No.

Socrates: But if, Theaetetus, this is not admitted, we shall be driven
into many absurdities.

Theaetetus: What are they?

Socrates: I will not tell you until I have endeavoured to consider the
matter from every point of view. For I should be ashamed of us if we
were driven in our perplexity to admit the absurd consequences of which
I speak. But if we find the solution, and get away from them, we may
regard them only as the difficulties of others, and the ridicule will
not attach to us. On the other hand, if we utterly fail, I suppose that
we must be humble, and allow the argument to trample us under foot,
as the sea-sick passenger is trampled upon by the sailor, and to do
anything to us. Listen, then, while I tell you how I hope to find a way
out of our difficulty.

Theaetetus: Let me hear.

Socrates: I think that we were wrong in denying that a man could think
what he knew to be what he did not know; and that there is a way in
which such a deception is possible.

Theaetetus: You mean to say, as I suspected at the time, that I may know
Socrates, and at a distance see some one who is unknown to me, and whom
I mistake for him--then the deception will occur?

Socrates: But has not that position been relinquished by us, because
involving the absurdity that we should know and not know the things
which we know?

Theaetetus: True.

Socrates: Let us make the assertion in another form, which may or may
not have a favourable issue; but as we are in a great strait, every
argument should be turned over and tested. Tell me, then, whether I am
right in saying that you may learn a thing which at one time you did not
know?

Theaetetus: Certainly you may.

Socrates: And another and another?

Theaetetus: Yes.

Socrates: I would have you imagine, then, that there exists in the mind
of man a block of wax, which is of different sizes in different men;
harder, moister, and having more or less of purity in one than another,
and in some of an intermediate quality.

Theaetetus: I see.

Socrates: Let us say that this tablet is a gift of Memory, the mother
of the Muses; and that when we wish to remember anything which we have
seen, or heard, or thought in our own minds, we hold the wax to the
perceptions and thoughts, and in that material receive the impression of
them as from the seal of a ring; and that we remember and know what is
imprinted as long as the image lasts; but when the image is effaced, or
cannot be taken, then we forget and do not know.

Theaetetus: Very good.

Socrates: Now, when a person has this knowledge, and is considering
something which he sees or hears, may not false opinion arise in the
following manner?

Theaetetus: In what manner?

Socrates: When he thinks what he knows, sometimes to be what he knows,
and sometimes to be what he does not know. We were wrong before in
denying the possibility of this.

Theaetetus: And how would you amend the former statement?

Socrates: I should begin by making a list of the impossible cases which
must be excluded. (1) No one can think one thing to be another when he
does not perceive either of them, but has the memorial or seal of both
of them in his mind; nor can any mistaking of one thing for another
occur, when he only knows one, and does not know, and has no impression
of the other; nor can he think that one thing which he does not know is
another thing which he does not know, or that what he does not know
is what he knows; nor (2) that one thing which he perceives is another
thing which he perceives, or that something which he perceives is
something which he does not perceive; or that something which he does
not perceive is something else which he does not perceive; or that
something which he does not perceive is something which he perceives;
nor again (3) can he think that something which he knows and perceives,
and of which he has the impression coinciding with sense, is something
else which he knows and perceives, and of which he has the impression
coinciding with sense;--this last case, if possible, is still more
inconceivable than the others; nor (4) can he think that something which
he knows and perceives, and of which he has the memorial coinciding with
sense, is something else which he knows; nor so long as these agree,
can he think that a thing which he knows and perceives is another thing
which he perceives; or that a thing which he does not know and does not
perceive, is the same as another thing which he does not know and does
not perceive;--nor again, can he suppose that a thing which he does not
know and does not perceive is the same as another thing which he does
not know; or that a thing which he does not know and does not perceive
is another thing which he does not perceive:--All these utterly and
absolutely exclude the possibility of false opinion. The only cases, if
any, which remain, are the following.

Theaetetus: What are they? If you tell me, I may perhaps understand you
better; but at present I am unable to follow you.

Socrates: A person may think that some things which he knows, or which
he perceives and does not know, are some other things which he knows and
perceives; or that some things which he knows and perceives, are other
things which he knows and perceives.

Theaetetus: I understand you less than ever now.

Socrates: Hear me once more, then:--I, knowing Theodorus, and
remembering in my own mind what sort of person he is, and also what sort
of person Theaetetus is, at one time see them, and at another time do
not see them, and sometimes I touch them, and at another time not, or
at one time I may hear them or perceive them in some other way, and at
another time not perceive them, but still I remember them, and know them
in my own mind.

Theaetetus: Very true.

Socrates: Then, first of all, I want you to understand that a man may or
may not perceive sensibly that which he knows.

Theaetetus: True.

Socrates: And that which he does not know will sometimes not be
perceived by him and sometimes will be perceived and only perceived?

Theaetetus: That is also true.

Socrates: See whether you can follow me better now: Socrates can
recognize Theodorus and Theaetetus, but he sees neither of them,
nor does he perceive them in any other way; he cannot then by any
possibility imagine in his own mind that Theaetetus is Theodorus. Am I
not right?

Theaetetus: You are quite right.

Socrates: Then that was the first case of which I spoke.

Theaetetus: Yes.

Socrates: The second case was, that I, knowing one of you and not
knowing the other, and perceiving neither, can never think him whom I
know to be him whom I do not know.

Theaetetus: True.

Socrates: In the third case, not knowing and not perceiving either of
you, I cannot think that one of you whom I do not know is the other whom
I do not know. I need not again go over the catalogue of excluded cases,
in which I cannot form a false opinion about you and Theodorus, either
when I know both or when I am in ignorance of both, or when I know one
and not the other. And the same of perceiving: do you understand me?

Theaetetus: I do.

Socrates: The only possibility of erroneous opinion is, when knowing you
and Theodorus, and having on the waxen block the impression of both of
you given as by a seal, but seeing you imperfectly and at a distance,
I try to assign the right impression of memory to the right visual
impression, and to fit this into its own print: if I succeed,
recognition will take place; but if I fail and transpose them, putting
the foot into the wrong shoe--that is to say, putting the vision of
either of you on to the wrong impression, or if my mind, like the sight
in a mirror, which is transferred from right to left, err by reason of
some similar affection, then `heterodoxy' and false opinion ensues.

Theaetetus: Yes, Socrates, you have described the nature of opinion with
wonderful exactness.

Socrates: Or again, when I know both of you, and perceive as well as
know one of you, but not the other, and my knowledge of him does not
accord with perception--that was the case put by me just now which you
did not understand.

Theaetetus: No, I did not.

Socrates: I meant to say, that when a person knows and perceives one of
you, his knowledge coincides with his perception, he will never think
him to be some other person, whom he knows and perceives, and the
knowledge of whom coincides with his perception--for that also was a
case supposed.

Theaetetus: True.

Socrates: But there was an omission of the further case, in which, as
we now say, false opinion may arise, when knowing both, and seeing, or
having some other sensible perception of both, I fail in holding the
seal over against the corresponding sensation; like a bad archer, I miss
and fall wide of the mark--and this is called falsehood.

Theaetetus: Yes; it is rightly so called.

Socrates: When, therefore, perception is present to one of the seals
or impressions but not to the other, and the mind fits the seal of the
absent perception on the one which is present, in any case of this sort
the mind is deceived; in a word, if our view is sound, there can be no
error or deception about things which a man does not know and has never
perceived, but only in things which are known and perceived; in these
alone opinion turns and twists about, and becomes alternately true and
false;--true when the seals and impressions of sense meet straight and
opposite--false when they go awry and crooked.

Theaetetus: And is not that, Socrates, nobly said?

Socrates: Nobly! yes; but wait a little and hear the explanation, and
then you will say so with more reason; for to think truly is noble and
to be deceived is base.

Theaetetus: Undoubtedly.

Socrates: And the origin of truth and error is as follows:--When the wax
in the soul of any one is deep and abundant, and smooth and perfectly
tempered, then the impressions which pass through the senses and sink
into the heart of the soul, as Homer says in a parable, meaning to
indicate the likeness of the soul to wax (Kerh Kerhos); these, I say,
being pure and clear, and having a sufficient depth of wax, are also
lasting, and minds, such as these, easily learn and easily retain,
and are not liable to confusion, but have true thoughts, for they have
plenty of room, and having clear impressions of things, as we term them,
quickly distribute them into their proper places on the block. And such
men are called wise. Do you agree?

Theaetetus: Entirely.

Socrates: But when the heart of any one is shaggy--a quality which the
all-wise poet commends, or muddy and of impure wax, or very soft, or
very hard, then there is a corresponding defect in the mind--the soft
are good at learning, but apt to forget; and the hard are the reverse;
the shaggy and rugged and gritty, or those who have an admixture of
earth or dung in their composition, have the impressions indistinct,
as also the hard, for there is no depth in them; and the soft too are
indistinct, for their impressions are easily confused and effaced. Yet
greater is the indistinctness when they are all jostled together in a
little soul, which has no room. These are the natures which have false
opinion; for when they see or hear or think of anything, they are
slow in assigning the right objects to the right impressions--in their
stupidity they confuse them, and are apt to see and hear and think
amiss--and such men are said to be deceived in their knowledge of
objects, and ignorant.

Theaetetus: No man, Socrates, can say anything truer than that.

Socrates: Then now we may admit the existence of false opinion in us?

Theaetetus: Certainly.

Socrates: And of true opinion also?

Theaetetus: Yes.

Socrates: We have at length satisfactorily proven beyond a doubt there
are these two sorts of opinion?

Theaetetus: Undoubtedly.

Socrates: Alas, Theaetetus, what a tiresome creature is a man who is
fond of talking!

Theaetetus: What makes you say so?

Socrates: Because I am disheartened at my own stupidity and tiresome
garrulity; for what other term will describe the habit of a man who
is always arguing on all sides of a question; whose dulness cannot be
convinced, and who will never leave off?

Theaetetus: But what puts you out of heart?

Socrates: I am not only out of heart, but in positive despair; for I do
not know what to answer if any one were to ask me:--O Socrates, have you
indeed discovered that false opinion arises neither in the comparison of
perceptions with one another nor yet in thought, but in union of thought
and perception? Yes, I shall say, with the complacence of one who thinks
that he has made a noble discovery.

Theaetetus: I see no reason why we should be ashamed of our
demonstration, Socrates.

Socrates: He will say: You mean to argue that the man whom we only think
of and do not see, cannot be confused with the horse which we do not see
or touch, but only think of and do not perceive? That I believe to be my
meaning, I shall reply.

Theaetetus: Quite right.

Socrates: Well, then, he will say, according to that argument, the
number eleven, which is only thought, can never be mistaken for twelve,
which is only thought: How would you answer him?

Theaetetus: I should say that a mistake may very likely arise between
the eleven or twelve which are seen or handled, but that no similar
mistake can arise between the eleven and twelve which are in the mind.

Socrates: Well, but do you think that no one ever put before his own
mind five and seven,--I do not mean five or seven men or horses, but
five or seven in the abstract, which, as we say, are recorded on the
waxen block, and in which false opinion is held to be impossible; did
no man ever ask himself how many these numbers make when added together,
and answer that they are eleven, while another thinks that they are
twelve, or would all agree in thinking and saying that they are twelve?

Theaetetus: Certainly not; many would think that they are eleven, and
in the higher numbers the chance of error is greater still; for I assume
you to be speaking of numbers in general.

Socrates: Exactly; and I want you to consider whether this does not
imply that the twelve in the waxen block are supposed to be eleven?

Theaetetus: Yes, that seems to be the case.

Socrates: Then do we not come back to the old difficulty? For he who
makes such a mistake does think one thing which he knows to be another
thing which he knows; but this, as we said, was impossible, and afforded
an irresistible proof of the non-existence of false opinion, because
otherwise the same person would inevitably know and not know the same
thing at the same time.

Theaetetus: Most true.

Socrates: Then false opinion cannot be explained as a confusion of
thought and sense, for in that case we could not have been mistaken
about pure conceptions of thought; and thus we are obliged to say,
either that false opinion does not exist, or that a man may not know
that which he knows;--which alternative do you prefer?

Theaetetus: It is hard to determine, Socrates.

Socrates: And yet the argument will scarcely admit of both. But, as we
are at our wits' end, suppose that we do a shameless thing?

Theaetetus: What is it?

Socrates: Let us attempt to explain the verb `to know.'

Theaetetus: And why should that be shameless?

Socrates: You seem not to be aware that the whole of our discussion from
the very beginning has been a search after knowledge, of which we are
assumed not to know the nature.

Theaetetus: Nay, but I am well aware.

Socrates: And is it not shameless when we do not know what knowledge is,
to be explaining the verb `to know'? The truth is, Theaetetus, that we
have long been infected with logical impurity. Thousands of times have
we repeated the words `we know,' and `do not know,' and `we have or have
not science or knowledge,' as if we could understand what we are saying
to one another, so long as we remain ignorant about knowledge; and at
this moment we are using the words `we understand,' `we are ignorant,'
as though we could still employ them when deprived of knowledge or
science.

Theaetetus: But if you avoid these expressions, Socrates, how will you
ever argue at all?

Socrates: I could not, being the man I am. The case would be different
if I were a true hero of dialectic: and O that such an one were present!
for he would have told us to avoid the use of these terms; at the same
time he would not have spared in you and me the faults which I have
noted. But, seeing that we are no great wits, shall I venture to say
what knowing is? for I think that the attempt may be worth making.

Theaetetus: Then by all means venture, and no one shall find fault with
you for using the forbidden terms.

Socrates: You have heard the common explanation of the verb `to know'?

Theaetetus: I think so, but I do not remember it at the moment.

Socrates: They explain the word `to know' as meaning `to have
knowledge.'

Theaetetus: True.

Socrates: I should like to make a slight change, and say `to possess'
knowledge.

Theaetetus: How do the two expressions differ?

Socrates: Perhaps there may be no difference; but still I should like
you to hear my view, that you may help me to test it.

Theaetetus: I will, if I can.

Socrates: I should distinguish `having' from `possessing': for example,
a man may buy and keep under his control a garment which he does not
wear; and then we should say, not that he has, but that he possesses the
garment.

Theaetetus: It would be the correct expression.

Socrates: Well, may not a man `possess' and yet not `have' knowledge
in the sense of which I am speaking? As you may suppose a man to have
caught wild birds--doves or any other birds--and to be keeping them in
an aviary which he has constructed at home; we might say of him in one
sense, that he always has them because he possesses them, might we not?

Theaetetus: Yes.

Socrates: And yet, in another sense, he has none of them; but they are
in his power, and he has got them under his hand in an enclosure of his
own, and can take and have them whenever he likes;--he can catch any
which he likes, and let the bird go again, and he may do so as often as
he pleases.

Theaetetus: True.

Socrates: Once more, then, as in what preceded we made a sort of waxen
figment in the mind, so let us now suppose that in the mind of each man
there is an aviary of all sorts of birds--some flocking together apart
from the rest, others in small groups, others solitary, flying anywhere
and everywhere.

Theaetetus: Let us imagine such an aviary--and what is to follow?

Socrates: We may suppose that the birds are kinds of knowledge, and that
when we were children, this receptacle was empty; whenever a man has
gotten and detained in the enclosure a kind of knowledge, he may be
said to have learned or discovered the thing which is the subject of the
knowledge: and this is to know.

Theaetetus: Granted.

Socrates: And further, when any one wishes to catch any of these
knowledges or sciences, and having taken, to hold it, and again to let
them go, how will he express himself?--will he describe the `catching'
of them and the original `possession' in the same words? I will make
my meaning clearer by an example:--You admit that there is an art of
arithmetic?

Theaetetus: To be sure.

Socrates: Conceive this under the form of a hunt after the science of
odd and even in general.

Theaetetus: I follow.

Socrates: Having the use of the art, the arithmetician, if I am not
mistaken, has the conceptions of number under his hand, and can transmit
them to another.

Theaetetus: Yes.

Socrates: And when transmitting them he may be said to teach them, and
when receiving to learn them, and when receiving to learn them, and when
having them in possession in the aforesaid aviary he may be said to know
them.

Theaetetus: Exactly.

Socrates: Attend to what follows: must not the perfect arithmetician
know all numbers, for he has the science of all numbers in his mind?

Theaetetus: True.

Socrates: And he can reckon abstract numbers in his head, or things
about him which are numerable?

Theaetetus: Of course he can.

Socrates: And to reckon is simply to consider how much such and such a
number amounts to?

Theaetetus: Very true.

Socrates: And so he appears to be searching into something which he
knows, as if he did not know it, for we have already admitted that he
knows all numbers;--you have heard these perplexing questions raised?

Theaetetus: I have.

Socrates: May we not pursue the image of the doves, and say that the
chase after knowledge is of two kinds? one kind is prior to possession
and for the sake of possession, and the other for the sake of taking and
holding in the hands that which is possessed already. And thus, when a
man has learned and known something long ago, he may resume and get hold
of the knowledge which he has long possessed, but has not at hand in his
mind.

Theaetetus: True.

Socrates: That was my reason for asking how we ought to speak when an
arithmetician sets about numbering, or a grammarian about reading? Shall
we say, that although he knows, he comes back to himself to learn what
he already knows?

Theaetetus: It would be too absurd, Socrates.

Socrates: Shall we say then that he is going to read or number what he
does not know, although we have admitted that he knows all letters and
all numbers?

Theaetetus: That, again, would be an absurdity.

Socrates: Then shall we say that about names we care nothing?--any one
may twist and turn the words `knowing' and `learning' in any way which
he likes, but since we have determined that the possession of knowledge
is not the having or using it, we do assert that a man cannot not
possess that which he possesses; and, therefore, in no case can a man
not know that which he knows, but he may get a false opinion about it;
for he may have the knowledge, not of this particular thing, but of some
other;--when the various numbers and forms of knowledge are flying about
in the aviary, and wishing to capture a certain sort of knowledge out
of the general store, he takes the wrong one by mistake, that is to say,
when he thought eleven to be twelve, he got hold of the ring-dove which
he had in his mind, when he wanted the pigeon.

Theaetetus: A very rational explanation.

Socrates: But when he catches the one which he wants, then he is not
deceived, and has an opinion of what is, and thus false and true opinion
may exist, and the difficulties which were previously raised disappear.
I dare say that you agree with me, do you not?

Theaetetus: Yes.

Socrates: And so we are rid of the difficulty of a man's not knowing
what he knows, for we are not driven to the inference that he does not
possess what he possesses, whether he be or be not deceived. And yet I
fear that a greater difficulty is looking in at the window.

Theaetetus: What is it?

Socrates: How can the exchange of one knowledge for another ever become
false opinion?

Theaetetus: What do you mean?

Socrates: In the first place, how can a man who has the knowledge of
anything be ignorant of that which he knows, not by reason of ignorance,
but by reason of his own knowledge? And, again, is it not an extreme
absurdity that he should suppose another thing to be this, and this to
be another thing;--that, having knowledge present with him in his mind,
he should still know nothing and be ignorant of all things?--you might
as well argue that ignorance may make a man know, and blindness make him
see, as that knowledge can make him ignorant.

Theaetetus: Perhaps, Socrates, we may have been wrong in making only
forms of knowledge our birds: whereas there ought to have been forms of
ignorance as well, flying about together in the mind, and then he who
sought to take one of them might sometimes catch a form of knowledge,
and sometimes a form of ignorance; and thus he would have a false
opinion from ignorance, but a true one from knowledge, about the same
thing.

Socrates: I cannot help praising you, Theaetetus, and yet I must beg you
to reconsider your words. Let us grant what you say--then, according to
you, he who takes ignorance will have a false opinion--am I right?

Theaetetus: Yes.

Socrates: He will certainly not think that he has a false opinion?

Theaetetus: Of course not.

Socrates: He will think that his opinion is true, and he will fancy that
he knows the things about which he has been deceived?

Theaetetus: Certainly.

Socrates: Then he will think that he has captured knowledge and not
ignorance?

Theaetetus: Clearly.

Socrates: And thus, after going a long way round, we are once more face
to face with our original difficulty. The hero of dialectic will retort
upon us:--`O my excellent friends, he will say, laughing, if a man knows
the form of ignorance and the form of knowledge, can he think that one
of them which he knows is the other which he knows? or, if he knows
neither of them, can he think that the one which he knows not is another
which he knows not? or, if he knows one and not the other, can he think
the one which he knows to be the one which he does not know? or the one
which he does not know to be the one which he knows? or will you tell me
that there are other forms of knowledge which distinguish the right and
wrong birds, and which the owner keeps in some other aviaries or graven
on waxen blocks according to your foolish images, and which he may be
said to know while he possesses them, even though he have them not at
hand in his mind? And thus, in a perpetual circle, you will be compelled
to go round and round, and you will make no progress.' What are we to
say in reply, Theaetetus?

Theaetetus: Indeed, Socrates, I do not know what we are to say.

Socrates: Are not his reproaches just, and does not the argument truly
show that we are wrong in seeking for false opinion until we know what
knowledge is; that must be first ascertained; then, the nature of false
opinion?

Theaetetus: I cannot but agree with you, Socrates, so far as we have yet
gone.

Socrates: Then, once more, what shall we say that knowledge is?--for we
are not going to lose heart as yet.

Theaetetus: Certainly, I shall not lose heart, if you do not.

Socrates: What definition will be most consistent with our former views?

Theaetetus: I cannot think of any but our old one, Socrates.

Socrates: What was it?

Theaetetus: Knowledge was said by us to be true opinion; and true
opinion is surely unerring, and the results which follow from it are all
noble and good.

Socrates: He who led the way into the river, Theaetetus, said `The
experiment will show;' and perhaps if we go forward in the search, we
may stumble upon the thing which we are looking for; but if we stay
where we are, nothing will come to light.

Theaetetus: Very true; let us go forward and try.

Socrates: The trail soon comes to an end, for a whole profession is
against us.

Theaetetus: How is that, and what profession do you mean?

Socrates: The profession of the great wise ones who are called orators
and lawyers; for these persuade men by their art and make them think
whatever they like, but they do not teach them. Do you imagine that
there are any teachers in the world so clever as to be able to convince
others of the truth about acts of robbery or violence, of which
they were not eye-witnesses, while a little water is flowing in the
clepsydra?

Theaetetus: Certainly not, they can only persuade them.

Socrates: And would you not say that persuading them is making them have
an opinion?

Theaetetus: To be sure.

Socrates: When, therefore, judges are justly persuaded about matters
which you can know only by seeing them, and not in any other way, and
when thus judging of them from report they attain a true opinion about
them, they judge without knowledge, and yet are rightly persuaded, if
they have judged well.

Theaetetus: Certainly.

Socrates: And yet, O my friend, if true opinion in law courts and
knowledge are the same, the perfect judge could not have judged rightly
without knowledge; and therefore I must infer that they are not the
same.

Theaetetus: That is a distinction, Socrates, which I have heard made
by some one else, but I had forgotten it. He said that true opinion,
combined with reason, was knowledge, but that the opinion which had
no reason was out of the sphere of knowledge; and that things of which
there is no rational account are not knowable--such was the singular
expression which he used--and that things which have a reason or
explanation are knowable.

Socrates: Excellent; but then, how did he distinguish between things
which are and are not `knowable'? I wish that you would repeat to me
what he said, and then I shall know whether you and I have heard the
same tale.

Theaetetus: I do not know whether I can recall it; but if another person
would tell me, I think that I could follow him.

Socrates: Let me give you, then, a dream in return for a
dream:--Methought that I too had a dream, and I heard in my dream that
the primeval letters or elements out of which you and I and all other
things are compounded, have no reason or explanation; you can only name
them, but no predicate can be either affirmed or denied of them, for in
the one case existence, in the other non-existence is already implied,
neither of which must be added, if you mean to speak of this or that
thing by itself alone. It should not be called itself, or that, or each,
or alone, or this, or the like; for these go about everywhere and are
applied to all things, but are distinct from them; whereas, if the first
elements could be described, and had a definition of their own, they
would be spoken of apart from all else. But none of these primeval
elements can be defined; they can only be named, for they have nothing
but a name, and the things which are compounded of them, as they are
complex, are expressed by a combination of names, for the combination
of names is the essence of a definition. Thus, then, the elements or
letters are only objects of perception, and cannot be defined or known;
but the syllables or combinations of them are known and expressed, and
are apprehended by true opinion. When, therefore, any one forms the true
opinion of anything without rational explanation, you may say that his
mind is truly exercised, but has no knowledge; for he who cannot give
and receive a reason for a thing, has no knowledge of that thing; but
when he adds rational explanation, then, he is perfected in knowledge
and may be all that I have been denying of him. Was that the form in
which the dream appeared to you?

Theaetetus: Precisely.

Socrates: And you allow and maintain that true opinion, combined with
definition or rational explanation, is knowledge?

Theaetetus: Exactly.

Socrates: Then may we assume, Theaetetus, that to-day, and in this
casual manner, we have found a truth which in former times many wise men
have grown old and have not found?

Theaetetus: At any rate, Socrates, I am satisfied with the present
statement.

Socrates: Which is probably correct--for how can there be knowledge
apart from definition and true opinion? And yet there is one point in
what has been said which does not quite satisfy me.

Theaetetus: What was it?

Socrates: What might seem to be the most ingenious notion of all:--That
the elements or letters are unknown, but the combination or syllables
known.

Theaetetus: And was that wrong?

Socrates: We shall soon know; for we have as hostages the instances
which the author of the argument himself used.

Theaetetus: What hostages?

Socrates: The letters, which are the clements; and the syllables, which
are the combinations;--he reasoned, did he not, from the letters of the
alphabet?

Theaetetus: Yes; he did.

Socrates: Let us take them and put them to the test, or rather, test
ourselves:--What was the way in which we learned letters? and, first of
all, are we right in saying that syllables have a definition, but that
letters have no definition?

Theaetetus: I think so.

Socrates: I think so too; for, suppose that some one asks you to spell
the first syllable of my name:--Theaetetus, he says, what is SO?

Theaetetus: I should reply S and O.

Socrates: That is the definition which you would give of the syllable?

Theaetetus: I should.

Socrates: I wish that you would give me a similar definition of the S.

Theaetetus: But how can any one, Socrates, tell the elements of an
element? I can only reply, that S is a consonant, a mere noise, as
of the tongue hissing; B, and most other letters, again, are neither
vowel-sounds nor noises. Thus letters may be most truly said to be
undefined; for even the most distinct of them, which are the seven
vowels, have a sound only, but no definition at all.

Socrates: Then, I suppose, my friend, that we have been so far right in
our idea about knowledge?

Theaetetus: Yes; I think that we have.

Socrates: Well, but have we been right in maintaining that the syllables
can be known, but not the letters?

Theaetetus: I think so.

Socrates: And do we mean by a syllable two letters, or if there are
more, all of them, or a single idea which arises out of the combination
of them?

Theaetetus: I should say that we mean all the letters.

Socrates: Take the case of the two letters S and O, which form the first
syllable of my own name; must not he who knows the syllable, know both
of them?

Theaetetus: Certainly.

Socrates: He knows, that is, the S and O?

Theaetetus: Yes.

Socrates: But can he be ignorant of either singly and yet know both
together?

Theaetetus: Such a supposition, Socrates, is monstrous and unmeaning.

Socrates: But if he cannot know both without knowing each, then if he is
ever to know the syllable, he must know the letters first; and thus the
fine theory has again taken wings and departed.

Theaetetus: Yes, with wonderful celerity.

Socrates: Yes, we did not keep watch properly. Perhaps we ought to have
maintained that a syllable is not the letters, but rather one single
idea framed out of them, having a separate form distinct from them.

Theaetetus: Very true; and a more likely notion than the other.

Socrates: Take care; let us not be cowards and betray a great and
imposing theory.

Theaetetus: No, indeed.

Socrates: Let us assume then, as we now say, that the syllable is
a simple form arising out of the several combinations of harmonious
elements--of letters or of any other elements.

Theaetetus: Very good.

Socrates: And it must have no parts.

Theaetetus: Why?

Socrates: Because that which has parts must be a whole of all the parts.
Or would you say that a whole, although formed out of the parts, is a
single notion different from all the parts?

Theaetetus: I should.

Socrates: And would you say that all and the whole are the same, or
different?

Theaetetus: I am not certain; but, as you like me to answer at once, I
shall hazard the reply, that they are different.

Socrates: I approve of your readiness, Theaetetus, but I must take time
to think whether I equally approve of your answer.

Theaetetus: Yes; the answer is the point.

Socrates: According to this new view, the whole is supposed to differ
from all?

Theaetetus: Yes.

Socrates: Well, but is there any difference between all (in the plural)
and the all (in the singular)? Take the case of number:--When we say
one, two, three, four, five, six; or when we say twice three, or three
times two, or four and two, or three and two and one, are we speaking of
the same or of different numbers?

Theaetetus: Of the same.

Socrates: That is of six?

Theaetetus: Yes.

Socrates: And in each form of expression we spoke of all the six?

Theaetetus: True.

Socrates: Again, in speaking of all (in the plural) is there not one
thing which we express?

Theaetetus: Of course there is.

Socrates: And that is six?

Theaetetus: Yes.

Socrates: Then in predicating the word `all' of things measured by
number, we predicate at the same time a singular and a plural?

Theaetetus: Clearly we do.

Socrates: Again, the number of the acre and the acre are the same; are
they not?

Theaetetus: Yes.

Socrates: And the number of the stadium in like manner is the stadium?

Theaetetus: Yes.

Socrates: And the army is the number of the army; and in all similar
cases, the entire number of anything is the entire thing?

Theaetetus: True.

Socrates: And the number of each is the parts of each?

Theaetetus: Exactly.

Socrates: Then as many things as have parts are made up of parts?

Theaetetus: Clearly.

Socrates: But all the parts are admitted to be the all, if the entire
number is the all?

Theaetetus: True.

Socrates: Then the whole is not made up of parts, for it would be the
all, if consisting of all the parts?

Theaetetus: That is the inference.

Socrates: But is a part a part of anything but the whole?

Theaetetus: Yes, of the all.

Socrates: You make a valiant defence, Theaetetus. And yet is not the all
that of which nothing is wanting?

Theaetetus: Certainly.

Socrates: And is not a whole likewise that from which nothing is absent?
but that from which anything is absent is neither a whole nor all;--if
wanting in anything, both equally lose their entirety of nature.

Theaetetus: I now think that there is no difference between a whole and
all.

Socrates: But were we not saying that when a thing has parts, all the
parts will be a whole and all?

Theaetetus: Certainly.

Socrates: Then, as I was saying before, must not the alternative be that
either the syllable is not the letters, and then the letters are not
parts of the syllable, or that the syllable will be the same with the
letters, and will therefore be equally known with them?

Theaetetus: You are right.

Socrates: And, in order to avoid this, we suppose it to be different
from them?

Theaetetus: Yes.

Socrates: But if letters are not parts of syllables, can you tell me of
any other parts of syllables, which are not letters?

Theaetetus: No, indeed, Socrates; for if I admit the existence of parts
in a syllable, it would be ridiculous in me to give up letters and seek
for other parts.

Socrates: Quite true, Theaetetus, and therefore, according to our
present view, a syllable must surely be some indivisible form?

Theaetetus: True.

Socrates: But do you remember, my friend, that only a little while ago
we admitted and approved the statement, that of the first elements out
of which all other things are compounded there could be no definition,
because each of them when taken by itself is uncompounded; nor can one
rightly attribute to them the words `being' or `this,' because they
are alien and inappropriate words, and for this reason the letters or
elements were indefinable and unknown?

Theaetetus: I remember.

Socrates: And is not this also the reason why they are simple and
indivisible? I can see no other.

Theaetetus: No other reason can be given.

Socrates: Then is not the syllable in the same case as the elements or
letters, if it has no parts and is one form?

Theaetetus: To be sure.

Socrates: If, then, a syllable is a whole, and has many parts or
letters, the letters as well as the syllable must be intelligible and
expressible, since all the parts are acknowledged to be the same as the
whole?

Theaetetus: True.

Socrates: But if it be one and indivisible, then the syllables and the
letters are alike undefined and unknown, and for the same reason?

Theaetetus: I cannot deny that.

Socrates: We cannot, therefore, agree in the opinion of him who says
that the syllable can be known and expressed, but not the letters.

Theaetetus: Certainly not; if we may trust the argument.

Socrates: Well, but will you not be equally inclined to disagree with
him, when you remember your own experience in learning to read?

Theaetetus: What experience?

Socrates: Why, that in learning you were kept trying to distinguish the
separate letters both by the eye and by the ear, in order that, when
you heard them spoken or saw them written, you might not be confused by
their position.

Theaetetus: Very true.

Socrates: And is the education of the harp-player complete unless he can
tell what string answers to a particular note; the notes, as every one
would allow, are the elements or letters of music?

Theaetetus: Exactly.

Socrates: Then, if we argue from the letters and syllables which we know
to other simples and compounds, we shall say that the letters or simple
elements as a class are much more certainly known than the syllables,
and much more indispensable to a perfect knowledge of any subject; and
if some one says that the syllable is known and the letter unknown,
we shall consider that either intentionally or unintentionally he is
talking nonsense?

Theaetetus: Exactly.

Socrates: And there might be given other proofs of this belief, if I am
not mistaken. But do not let us in looking for them lose sight of the
question before us, which is the meaning of the statement, that right
opinion with rational definition or explanation is the most perfect form
of knowledge.

Theaetetus: We must not.

Socrates: Well, and what is the meaning of the term `explanation'? I
think that we have a choice of three meanings.

Theaetetus: What are they?

Socrates: In the first place, the meaning may be, manifesting one's
thought by the voice with verbs and nouns, imaging an opinion in the
stream which flows from the lips, as in a mirror or water. Does not
explanation appear to be of this nature?

Theaetetus: Certainly; he who so manifests his thought, is said to
explain himself.

Socrates: And every one who is not born deaf or dumb is able sooner or
later to manifest what he thinks of anything; and if so, all those who
have a right opinion about anything will also have right explanation;
nor will right opinion be anywhere found to exist apart from knowledge.

Theaetetus: True.

Socrates: Let us not, therefore, hastily charge him who gave this
account of knowledge with uttering an unmeaning word; for perhaps he
only intended to say, that when a person was asked what was the nature
of anything, he should be able to answer his questioner by giving the
elements of the thing.

Theaetetus: As for example, Socrates...?

Socrates: As, for example, when Hesiod says that a waggon is made up
of a hundred planks. Now, neither you nor I could describe all of
them individually; but if any one asked what is a waggon, we should be
content to answer, that a waggon consists of wheels, axle, body, rims,
yoke.

Theaetetus: Certainly.

Socrates: And our opponent will probably laugh at us, just as he would
if we professed to be grammarians and to give a grammatical account of
the name of Theaetetus, and yet could only tell the syllables and not
the letters of your name--that would be true opinion, and not knowledge;
for knowledge, as has been already remarked, is not attained until,
combined with true opinion, there is an enumeration of the elements out
of which anything is composed.

Theaetetus: Yes.

Socrates: In the same general way, we might also have true opinion about
a waggon; but he who can describe its essence by an enumeration of the
hundred planks, adds rational explanation to true opinion, and instead
of opinion has art and knowledge of the nature of a waggon, in that he
attains to the whole through the elements.

Theaetetus: And do you not agree in that view, Socrates?

Socrates: If you do, my friend; but I want to know first, whether you
admit the resolution of all things into their elements to be a rational
explanation of them, and the consideration of them in syllables or
larger combinations of them to be irrational--is this your view?

Theaetetus: Precisely.

Socrates: Well, and do you conceive that a man has knowledge of any
element who at one time affirms and at another time denies that element
of something, or thinks that the same thing is composed of different
elements at different times?

Theaetetus: Assuredly not.

Socrates: And do you not remember that in your case and in that of
others this often occurred in the process of learning to read?

Theaetetus: You mean that I mistook the letters and misspelt the
syllables?

Socrates: Yes.

Theaetetus: To be sure; I perfectly remember, and I am very far from
supposing that they who are in this condition have knowledge.

Socrates: When a person at the time of learning writes the name of
Theaetetus, and thinks that he ought to write and does write Th and e;
but, again, meaning to write the name of Theododorus, thinks that he
ought to write and does write T and e--can we suppose that he knows the
first syllables of your two names?

Theaetetus: We have already admitted that such a one has not yet
attained knowledge.

Socrates: And in like manner be may enumerate without knowing them the
second and third and fourth syllables of your name?

Theaetetus: He may.

Socrates: And in that case, when he knows the order of the letters and
can write them out correctly, he has right opinion?

Theaetetus: Clearly.

Socrates: But although we admit that he has right opinion, he will still
be without knowledge?

Theaetetus: Yes.

Socrates: And yet he will have explanation, as well as right opinion,
for he knew the order of the letters when he wrote; and this we admit to
be explanation.

Theaetetus: True.

Socrates: Then, my friend, there is such a thing as right opinion united
with definition or explanation, which does not as yet attain to the
exactness of knowledge.

Theaetetus: It would seem so.

Socrates: And what we fancied to be a perfect definition of knowledge
is a dream only. But perhaps we had better not say so as yet, for were
there not three explanations of knowledge, one of which must, as we
said, be adopted by him who maintains knowledge to be true opinion
combined with rational explanation? And very likely there may be found
some one who will not prefer this but the third.

Theaetetus: You are quite right; there is still one remaining. The first
was the image or expression of the mind in speech; the second, which has
just been mentioned, is a way of reaching the whole by an enumeration of
the elements. But what is the third definition?

Socrates: There is, further, the popular notion of telling the mark or
sign of difference which distinguishes the thing in question from all
others.

Theaetetus: Can you give me any example of such a definition?

Socrates: As, for example, in the case of the sun, I think that you
would be contented with the statement that the sun is the brightest of
the heavenly bodies which revolve about the earth.

Theaetetus: Certainly.

Socrates: Understand why:--the reason is, as I was just now saying, that
if you get at the difference and distinguishing characteristic of each
thing, then, as many persons affirm, you will get at the definition or
explanation of it; but while you lay hold only of the common and not of
the characteristic notion, you will only have the definition of those
things to which this common quality belongs.

Theaetetus: I understand you, and your account of definition is in my
judgment correct.

Socrates: But he, who having right opinion about anything, can find out
the difference which distinguishes it from other things will know that
of which before he had only an opinion.

Theaetetus: Yes; that is what we are maintaining.

Socrates: Nevertheless, Theaetetus, on a nearer view, I find myself
quite disappointed; the picture, which at a distance was not so bad, has
now become altogether unintelligible.

Theaetetus: What do you mean?

Socrates: I will endeavour to explain: I will suppose myself to have
true opinion of you, and if to this I add your definition, then I have
knowledge, but if not, opinion only.

Theaetetus: Yes.

Socrates: The definition was assumed to be the interpretation of your
difference.

Theaetetus: True.

Socrates: But when I had only opinion, I had no conception of your
distinguishing characteristics.

Theaetetus: I suppose not.

Socrates: Then I must have conceived of some general or common nature
which no more belonged to you than to another.

Theaetetus: True.

Socrates: Tell me, now--How in that case could I have formed a judgment
of you any more than of any one else? Suppose that I imagine Theaetetus
to be a man who has nose, eyes, and mouth, and every other member
complete; how would that enable me to distinguish Theaetetus from
Theodorus, or from some outer barbarian?

Theaetetus: How could it?

Socrates: Or if I had further conceived of you, not only as having nose
and eyes, but as having a snub nose and prominent eyes, should I have
any more notion of you than of myself and others who resemble me?

Theaetetus: Certainly not.

Socrates: Surely I can have no conception of Theaetetus until your
snub-nosedness has left an impression on my mind different from the
snub-nosedness of all others whom I have ever seen, and until your other
peculiarities have a like distinctness; and so when I meet you to-morrow
the right opinion will be re-called?

Theaetetus: Most true.

Socrates: Then right opinion implies the perception of differences?

Theaetetus: Clearly.

Socrates: What, then, shall we say of adding reason or explanation to
right opinion? If the meaning is, that we should form an opinion of
the way in which something differs from another thing, the proposal is
ridiculous.

Theaetetus: How so?

Socrates: We are supposed to acquire a right opinion of the differences
which distinguish one thing from another when we have already a right
opinion of them, and so we go round and round:--the revolution of the
scytal, or pestle, or any other rotatory machine, in the same circles,
is as nothing compared with such a requirement; and we may be truly
described as the blind directing the blind; for to add those things
which we already have, in order that we may learn what we already think,
is like a soul utterly benighted.

Theaetetus: Tell me; what were you going to say just now, when you asked
the question?

Socrates: If, my boy, the argument, in speaking of adding the
definition, had used the word to `know,' and not merely `have an
opinion' of the difference, this which is the most promising of all the
definitions of knowledge would have come to a pretty end, for to know is
surely to acquire knowledge.

Theaetetus: True.

Socrates: And so, when the question is asked, What is knowledge? this
fair argument will answer `Right opinion with knowledge,'--knowledge,
that is, of difference, for this, as the said argument maintains, is
adding the definition.

Theaetetus: That seems to be true.

Socrates: But how utterly foolish, when we are asking what is knowledge,
that the reply should only be, right opinion with knowledge of
difference or of anything! And so, Theaetetus, knowledge is neither
sensation nor true opinion, nor yet definition and explanation
accompanying and added to true opinion?

Theaetetus: I suppose not.

Socrates: And are you still in labour and travail, my dear friend, or
have you brought all that you have to say about knowledge to the birth?

Theaetetus: I am sure, Socrates, that you have elicited from me a good
deal more than ever was in me.

Socrates: And does not my art show that you have brought forth wind, and
that the offspring of your brain are not worth bringing up?

Theaetetus: Very true.

Socrates: But if, Theaetetus, you should ever conceive afresh, you will
be all the better for the present investigation, and if not, you will be
soberer and humbler and gentler to other men, and will be too modest
to fancy that you know what you do not know. These are the limits of my
art; I can no further go, nor do I know aught of the things which great
and famous men know or have known in this or former ages. The office
of a midwife I, like my mother, have received from God; she delivered
women, I deliver men; but they must be young and noble and fair.

And now I have to go to the porch of the King Archon, where I am to meet
Meletus and his indictment. To-morrow morning, Theodorus, I shall hope
to see you again at this place.





% chapter theaetetus (end)