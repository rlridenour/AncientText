\documentclass{tufte-book}


\hypersetup{colorlinks}% uncomment this line if you prefer colored hyperlinks (e.g., for onscreen viewing)

%%
% Book metadata
\title{Readings in Ancient Philosophy}
\author[Randall L. Ridenour]{Randall L. Ridenour}
% \publisher{Publisher of This Book}

%%
% If they're installed, use Bergamo and Chantilly from www.fontsite.com.
% They're clones of Bembo and Gill Sans, respectively.
%\IfFileExists{bergamo.sty}{\usepackage[osf]{bergamo}}{}% Bembo
%\IfFileExists{chantill.sty}{\usepackage{chantill}}{}% Gill Sans

%\usepackage{microtype}

%%
% Just some sample text
\usepackage{lipsum}
\usepackage{amssymb}

%%
% For nicely typeset tabular material
\usepackage{booktabs}

%%
% For graphics / images
\usepackage{graphicx}
\setkeys{Gin}{width=\linewidth,totalheight=\textheight,keepaspectratio}
\graphicspath{{graphics/}}

% The fancyvrb package lets us customize the formatting of verbatim
% environments.  We use a slightly smaller font.
\usepackage{fancyvrb}
\fvset{fontsize=\normalsize}

%%
% Prints argument within hanging parentheses (i.e., parentheses that take
% up no horizontal space).  Useful in tabular environments.
\newcommand{\hangp}[1]{\makebox[0pt][r]{(}#1\makebox[0pt][l]{)}}

%%
% Prints an asterisk that takes up no horizontal space.
% Useful in tabular environments.
\newcommand{\hangstar}{\makebox[0pt][l]{*}}

%%
% Prints a trailing space in a smart way.
\usepackage{xspace}

%%
% Some shortcuts for Tufte's book titles.  The lowercase commands will
% produce the initials of the book title in italics.  The all-caps commands
% will print out the full title of the book in italics.
\newcommand{\vdqi}{\textit{VDQI}\xspace}
\newcommand{\ei}{\textit{EI}\xspace}
\newcommand{\ve}{\textit{VE}\xspace}
\newcommand{\be}{\textit{BE}\xspace}
\newcommand{\VDQI}{\textit{The Visual Display of Quantitative Information}\xspace}
\newcommand{\EI}{\textit{Envisioning Information}\xspace}
\newcommand{\VE}{\textit{Visual Explanations}\xspace}
\newcommand{\BE}{\textit{Beautiful Evidence}\xspace}

\newcommand{\TL}{Tufte-\LaTeX\xspace}

% Prints the month name (e.g., January) and the year (e.g., 2008)
\newcommand{\monthyear}{%
  \ifcase\month\or January\or February\or March\or April\or May\or June\or
  July\or August\or September\or October\or November\or
  December\fi\space\number\year
}


% Prints an epigraph and speaker in sans serif, all-caps type.
\newcommand{\openepigraph}[2]{%
  %\sffamily\fontsize{14}{16}\selectfont
  \begin{fullwidth}
  \sffamily\large
  \begin{doublespace}
  \noindent\allcaps{#1}\\% epigraph
  \noindent\allcaps{#2}% author
  \end{doublespace}
  \end{fullwidth}
}

% Inserts a blank page
\newcommand{\blankpage}{\newpage\hbox{}\thispagestyle{empty}\newpage}

\usepackage{units}

% Typesets the font size, leading, and measure in the form of 10/12x26 pc.
\newcommand{\measure}[3]{#1/#2$\times$\unit[#3]{pc}}

% Macros for typesetting the documentation
\newcommand{\hlred}[1]{\textcolor{Maroon}{#1}}% prints in red
\newcommand{\hangleft}[1]{\makebox[0pt][r]{#1}}
\newcommand{\hairsp}{\hspace{1pt}}% hair space
\newcommand{\hquad}{\hskip0.5em\relax}% half quad space
\newcommand{\TODO}{\textcolor{red}{\bf TODO!}\xspace}
\newcommand{\ie}{\textit{i.\hairsp{}e.}\xspace}
\newcommand{\eg}{\textit{e.\hairsp{}g.}\xspace}
\newcommand{\na}{\quad--}% used in tables for N/A cells
\providecommand{\XeLaTeX}{X\lower.5ex\hbox{\kern-0.15em\reflectbox{E}}\kern-0.1em\LaTeX}
\newcommand{\tXeLaTeX}{\XeLaTeX\index{XeLaTeX@\protect\XeLaTeX}}
% \index{\texttt{\textbackslash xyz}@\hangleft{\texttt{\textbackslash}}\texttt{xyz}}
\newcommand{\tuftebs}{\symbol{'134}}% a backslash in tt type in OT1/T1
\newcommand{\doccmdnoindex}[2][]{\texttt{\tuftebs#2}}% command name -- adds backslash automatically (and doesn't add cmd to the index)
\newcommand{\doccmddef}[2][]{%
  \hlred{\texttt{\tuftebs#2}}\label{cmd:#2}%
  \ifthenelse{\isempty{#1}}%
    {% add the command to the index
      \index{#2 command@\protect\hangleft{\texttt{\tuftebs}}\texttt{#2}}% command name
    }%
    {% add the command and package to the index
      \index{#2 command@\protect\hangleft{\texttt{\tuftebs}}\texttt{#2} (\texttt{#1} package)}% command name
      \index{#1 package@\texttt{#1} package}\index{packages!#1@\texttt{#1}}% package name
    }%
}% command name -- adds backslash automatically
\newcommand{\doccmd}[2][]{%
  \texttt{\tuftebs#2}%
  \ifthenelse{\isempty{#1}}%
    {% add the command to the index
      \index{#2 command@\protect\hangleft{\texttt{\tuftebs}}\texttt{#2}}% command name
    }%
    {% add the command and package to the index
      \index{#2 command@\protect\hangleft{\texttt{\tuftebs}}\texttt{#2} (\texttt{#1} package)}% command name
      \index{#1 package@\texttt{#1} package}\index{packages!#1@\texttt{#1}}% package name
    }%
}% command name -- adds backslash automatically
\newcommand{\docopt}[1]{\ensuremath{\langle}\textrm{\textit{#1}}\ensuremath{\rangle}}% optional command argument
\newcommand{\docarg}[1]{\textrm{\textit{#1}}}% (required) command argument
\newenvironment{docspec}{\begin{quotation}\ttfamily\parskip0pt\parindent0pt\ignorespaces}{\end{quotation}}% command specification environment
\newcommand{\docenv}[1]{\texttt{#1}\index{#1 environment@\texttt{#1} environment}\index{environments!#1@\texttt{#1}}}% environment name
\newcommand{\docenvdef}[1]{\hlred{\texttt{#1}}\label{env:#1}\index{#1 environment@\texttt{#1} environment}\index{environments!#1@\texttt{#1}}}% environment name
\newcommand{\docpkg}[1]{\texttt{#1}\index{#1 package@\texttt{#1} package}\index{packages!#1@\texttt{#1}}}% package name
\newcommand{\doccls}[1]{\texttt{#1}}% document class name
\newcommand{\docclsopt}[1]{\texttt{#1}\index{#1 class option@\texttt{#1} class option}\index{class options!#1@\texttt{#1}}}% document class option name
\newcommand{\docclsoptdef}[1]{\hlred{\texttt{#1}}\label{clsopt:#1}\index{#1 class option@\texttt{#1} class option}\index{class options!#1@\texttt{#1}}}% document class option name defined
\newcommand{\docmsg}[2]{\bigskip\begin{fullwidth}\noindent\ttfamily#1\end{fullwidth}\medskip\par\noindent#2}
\newcommand{\docfilehook}[2]{\texttt{#1}\index{file hooks!#2}\index{#1@\texttt{#1}}}
\newcommand{\doccounter}[1]{\texttt{#1}\index{#1 counter@\texttt{#1} counter}}

\setcounter{secnumdepth}{0}

% Generates the index
\usepackage{makeidx}
\makeindex

\begin{document}

% Front matter
\frontmatter

% r.1 blank page
% \blankpage




% r.3 full title page
% \maketitle

\begin{titlepage}
	
	{\LARGE \emph{Randall L. Ridenour}}
	
	\bigskip
	
	\bigskip
	\bigskip
	\bigskip
	\bigskip
	\bigskip
	\bigskip
	\bigskip
	\bigskip
	\bigskip
	\bigskip
	\bigskip
	\bigskip
	
	{\Huge \emph{Ancient Philosophy}}  \\ {\LARGE An Anthology}
	
	\vfill
	
	\textsc{Oklahoma Baptist University}

	
	

\end{titlepage}

% v.4 copyright page
\newpage
\begin{fullwidth}
~\vfill
\thispagestyle{empty}
\setlength{\parindent}{0pt}
\setlength{\parskip}{\baselineskip}
Copyright \copyright\ \the\year\ \thanklessauthor

% \par\smallcaps{Published by \thanklesspublisher}
% 
% \par\smallcaps{tufte-latex.googlecode.com}
% 
% \par Licensed under the Apache License, Version 2.0 (the ``License''); you may not
% use this file except in compliance with the License. You may obtain a copy
% of the License at \url{http://www.apache.org/licenses/LICENSE-2.0}. Unless
% required by applicable law or agreed to in writing, software distributed
% under the License is distributed on an \smallcaps{``AS IS'' BASIS, WITHOUT
% WARRANTIES OR CONDITIONS OF ANY KIND}, either express or implied. See the
% License for the specific language governing permissions and limitations
% under the License.\index{license}
% 
% \par\textit{First printing, \monthyear}
\end{fullwidth}

% v.2 epigraphs
\newpage\thispagestyle{empty}

\openepigraph{%
Philosophy begins in wonder.
}{Plato}

% r.5 contents
\tableofcontents

% \listoffigures
% 
% \listoftables

% r.7 dedication
% \cleardoublepage
% ~\vfill
% \begin{doublespace}
% \noindent\fontsize{18}{22}\selectfont\itshape
% \nohyphenation
% Understanding is the reward of faith. Therefore seek not to understand that you may believe, but believe, that you may understand. 
% 
% \mbox{Augustine} 
% \end{doublespace}
% \vfill
% \vfill


% r.9 introduction
\cleardoublepage
%%
% Start the main matter (normal chapters)
\mainmatter


% tex files to include


% 


\chapter{Euthyphro} % (fold)
\label{cha:euthyphro}


Euthyphro. Why have you left the Lyceum, Socrates? and what are you doing in the Porch of the King Archon? Surely you cannot be concerned in a suit before the King, like myself?

Socrates. Not in a suit, Euthyphro; impeachment is the word which the Athenians use.

Euth. What ! I suppose that some one has been prosecuting you, for I cannot believe that you are the prosecutor of another.

Soc. Certainly not.

Euth. Then some one else has been prosecuting you?

Soc. Yes.

Euth. And who is he?

Soc. A young man who is little known, Euthyphro; and I hardly know him: his name is Meletus, and he is of the deme of Pitthis. Perhaps you may remember his appearance; he has a beak, and long straight hair, and a beard which is ill grown.

Euth. No, I do not remember him, Socrates. But what is the charge which he brings against you?

Soc. What is the charge? Well, a very serious charge, which shows a good deal of character in the young man, and for which he is certainly not to be despised. He says he knows how the youth are corrupted and who are their corruptors. I fancy that he must be a wise man, and seeing that I am the reverse of a wise man, he has found me out, and is going to accuse me of corrupting his young friends. And of this our mother the state is to be the judge. Of all our political men he is the only one who seems to me to begin in the right way, with the cultivation of virtue in youth; like a good husbandman, he makes the young shoots his first care, and clears away us who are the destroyers of them. This is only the first step; he will afterwards attend to the elder branches; and if he goes on as he has begun, he will be a very great public benefactor.

Euth. I hope that he may; but I rather fear, Socrates, that the opposite will turn out to be the truth. My opinion is that in attacking you he is simply aiming a blow at the foundation of the state. But in what way does he say that you corrupt the young?

Soc. He brings a wonderful accusation against me, which at first hearing excites surprise: he says that I am a poet or maker of gods, and that I invent new gods and deny the existence of old ones; this is the ground of his indictment.

Euth. I understand, Socrates; he means to attack you about the familiar sign which occasionally, as you say, comes to you. He thinks that you are a neologian, and he is going to have you up before the court for this. He knows that such a charge is readily received by the world, as I myself know too well; for when I speak in the assembly about divine things, and foretell the future to them, they laugh at me and think me a madman. Yet every word that I say is true. But they are jealous of us all; and we must be brave and go at them.

Soc. Their laughter, friend Euthyphro, is not a matter of much consequence. For a man may be thought wise; but the Athenians, I suspect, do not much trouble themselves about him until he begins to impart his wisdom to others, and then for some reason or other, perhaps, as you say, from jealousy, they are angry.

Euth. I am never likely to try their temper in this way.

Soc. I dare say not, for you are reserved in your behaviour, and seldom impart your wisdom. But I have a benevolent habit of pouring out myself to everybody, and would even pay for a listener, and I am afraid that the Athenians may think me too talkative. Now if, as I was saying, they would only laugh at me, as you say that they laugh at you, the time might pass gaily enough in the court; but perhaps they may be in earnest, and then what the end will be you soothsayers only can predict.

Euth. I dare say that the affair will end in nothing, Socrates, and that you will win your cause; and I think that I shall win my own.

Soc. And what is your suit, Euthyphro? are you the pursuer or the defendant?

Euth. I am the pursuer.

Soc. Of whom?

Euth. You will think me mad when I tell you.

Soc. Why, has the fugitive wings?

Euth. Nay, he is not very volatile at his time of life.

Soc. Who is he?

Euth. My father.

Soc. Your father ! my good man?

Euth. Yes.

Soc. And of what is he accused?

Euth. Of murder, Socrates.

Soc. By the powers, Euthyphro ! how little does the common herd know of the nature of right and truth. A man must be an extraordinary man, and have made great strides in wisdom, before he could have seen his way to bring such an action.

Euth. Indeed, Socrates, he must.

Soc. I suppose that the man whom your father murdered was one of your relatives, clearly he was; for if he had been a stranger you would never have thought of prosecuting him.

Euth. I am amused, Socrates, at your making a distinction between one who is a relation and one who is not a relation; for surely the pollution is the same in either case, if you knowingly associate with the murderer when you ought to clear yourself and him by proceeding against him. The real question is whether the murdered man has been justly slain. If justly, then your duty is to let the matter alone; but if unjustly, then even if the murderer lives under the same roof with you and eats at the same table, proceed against him. Now the man who is dead was a poor dependent of mine who worked for us as a field labourer on our farm in Naxos, and one day in a fit of drunken passion he got into a quarrel with one of our domestic servants and slew him. My father bound him hand and foot and threw him into a ditch, and then sent to Athens to ask of a diviner what he should do with him. Meanwhile he never attended to him and took no care about him, for he regarded him as a murderer; and thought that no great harm would be done even if he did die. Now this was just what happened. For such was the effect of cold and hunger and chains upon him, that before the messenger returned from the diviner, he was dead. And my father and family are angry with me for taking the part of the murderer and prosecuting my father. They say that he did not kill him, and that if he did, dead man was but a murderer, and I ought not to take any notice, for that a son is impious who prosecutes a father. Which shows, Socrates, how little they know what the gods think about piety and impiety.

Soc. Good heavens, Euthyphro ! and is your knowledge of religion and of things pious and impious so very exact, that, supposing the circumstances to be as you state them, you are not afraid lest you too may be doing an impious thing in bringing an action against your father?

Euth. The best of Euthyphro, and that which distinguishes him, Socrates, from other men, is his exact knowledge of all such matters. What should I be good for without it?

Soc. Rare friend ! I think that I cannot do better than be your disciple. Then before the trial with Meletus comes on I shall challenge him, and say that I have always had a great interest in religious questions, and now, as he charges me with rash imaginations and innovations in religion, I have become your disciple. You, Meletus, as I shall say to him, acknowledge Euthyphro to be a great theologian, and sound in his opinions; and if you approve of him you ought to approve of me, and not have me into court; but if you disapprove, you should begin by indicting him who is my teacher, and who will be the ruin, not of the young, but of the old; that is to say, of myself whom he instructs, and of his old father whom he admonishes and chastises. And if Meletus refuses to listen to me, but will go on, and will not shift the indictment from me to you, I cannot do better than repeat this challenge in the court.

Euth. Yes, indeed, Socrates; and if he attempts to indict me I am mistaken if I do not find a flaw in him; the court shall have a great deal more to say to him than to me.

Soc. And I, my dear friend, knowing this, am desirous of becoming your disciple. For I observe that no one appears to notice you,  not even this Meletus; but his sharp eyes have found me out at once, and he has indicted me for impiety. And therefore, I adjure you to tell me the nature of piety and impiety, which you said that you knew so well, and of murder, and of other offences against the gods. What are they? Is not piety in every action always the same? and impiety, again,  is it not always the opposite of piety, and also the same with itself, having, as impiety, one notion which includes whatever is impious?

Euth. To be sure, Socrates.

Soc. And what is piety, and what is impiety?

Euth. Piety is doing as I am doing; that is to say, prosecuting any one who is guilty of murder, sacrilege, or of any similar crime, whether he be your father or mother, or whoever he may be, that makes no difference; and not to prosecute them is impiety. And please to consider, Socrates, what a notable proof I will give you of the truth of my words, a proof which I have already given to others:, of the principle, I mean, that the impious, whoever he may be, ought not to go unpunished. For do not men regard Zeus as the best and most righteous of the gods?, and yet they admit that he bound his father (Cronos) because he wickedly devoured his sons, and that he too had punished his own father (Uranus) for a similar reason, in a nameless manner. And yet when I proceed against my father, they are angry with me. So inconsistent are they in their way of talking when the gods are concerned, and when I am concerned.

Soc. May not this be the reason, Euthyphro, why I am charged with impiety, that I cannot away with these stories about the gods? and therefore I suppose that people think me wrong. But, as you who are well informed about them approve of them, I cannot do better than assent to your superior wisdom. What else can I say, confessing as I do, that I know nothing about them? Tell me, for the love of Zeus, whether you really believe that they are true.

Euth. Yes, Socrates; and things more wonderful still, of which the world is in ignorance.

Soc. And do you really believe that the gods, fought with one another, and had dire quarrels, battles, and the like, as the poets say, and as you may see represented in the works of great artists? The temples are full of them; and notably the robe of Athene, which is carried up to the Acropolis at the great Panathenaea, is embroidered with them. Are all these tales of the gods true, Euthyphro?

Euth. Yes, Socrates; and, as I was saying, I can tell you, if you would like to hear them, many other things about the gods which would quite amaze you.

Soc. I dare say; and you shall tell me them at some other time when I have leisure. But just at present I would rather hear from you a more precise answer, which you have not as yet given, my friend, to the question, What is `piety'? When asked, you only replied, Doing as you do, charging your father with murder.

Euth. And what I said was true, Socrates.

Soc. No doubt, Euthyphro; but you would admit that there are many other pious acts?

Euth. There are.

Soc. Remember that I did not ask you to give me two or three examples of piety, but to explain the general idea which makes all pious things to be pious. Do you not recollect that there was one idea which made the impious impious, and the pious pious?

Euth. I remember.

Soc. Tell me what is the nature of this idea, and then I shall have a standard to which I may look, and by which I may measure actions, whether yours or those of any one else, and then I shall be able to say that such and such an action is pious, such another impious.

Euth. I will tell you, if you like.

Soc. I should very much like.

Euth. Piety, then, is that which is dear to the gods, and impiety is that which is not dear to them.

Soc. Very good, Euthyphro; you have now given me the sort of answer which I wanted. But whether what you say is true or not I cannot as yet tell, although I make no doubt that you will prove the truth of your words.

Euth. Of course.

Soc. Come, then, and let us examine what we are saying. That thing or person which is dear to the gods is pious, and that thing or person which is hateful to the gods is impious, these two being the extreme opposites of one another. Was not that said?

Euth. It was.

Soc. And well said?

Euth. Yes, Socrates, I thought so; it was certainly said.

Soc. And further, Euthyphro, the gods were admitted to have enmities and hatreds and differences?

Euth. Yes, that was also said.

Soc. And what sort of difference creates enmity and anger? Suppose for example that you and I, my good friend, differ about a number; do differences of this sort make us enemies and set us at variance with one another? Do we not go at once to arithmetic, and put an end to them by a sum?

Euth. True.

Soc. Or suppose that we differ about magnitudes, do we not quickly end the differences by measuring?

Euth. Very true.

Soc. And we end a controversy about heavy and light by resorting to a weighing machine?

Euth. To be sure.

Soc. But what differences are there which cannot be thus decided, and which therefore make us angry and set us at enmity with one another? I dare say the answer does not occur to you at the moment, and therefore I will suggest that these enmities arise when the matters of difference are the just and unjust, good and evil, honourable and dishonourable. Are not these the points about which men differ, and about which when we are unable satisfactorily to decide our differences, you and I and all of us quarrel, when we do quarrel?

Euth. Yes, Socrates, the nature of the differences about which we quarrel is such as you describe.

Soc. And the quarrels of the gods, noble Euthyphro, when they occur, are of a like nature?

Euth. Certainly they are.

Soc. They have differences of opinion, as you say, about good and evil, just and unjust, honourable and dishonourable: there would have been no quarrels among them, if there had been no such differences, would there now?

Euth. You are quite right.

Soc. Does not every man love that which he deems noble and just and good, and hate the opposite of them?

Euth. Very true.

Soc. But, as you say, people regard the same things, some as just and others as unjust,, about these they dispute; and so there arise wars and fightings among them.

Euth. Very true.

Soc. Then the same things are hated by the gods and loved by the gods, and are both hateful and dear to them?

Euth. True.

Soc. And upon this view the same things, Euthyphro, will be pious and also impious?

Euth. So I should suppose.

Soc. Then, my friend, I remark with surprise that you have not answered the question which I asked. For I certainly did not ask you to tell me what action is both pious and impious: but now it would seem that what is loved by the gods is also hated by them. And therefore, Euthyphro, in thus chastising your father you may very likely be doing what is agreeable to Zeus but disagreeable to Cronos or Uranus, and what is acceptable to Hephaestus but unacceptable to Here, and there may be other gods who have similar differences of opinion.

Euth. But I believe, Socrates, that all the gods would be agreed as to the propriety of punishing a murderer: there would be no difference of opinion about that.

Soc. Well, but speaking of men, Euthyphro, did you ever hear any one arguing that a murderer or any sort of evil-doer ought to be let off?

Euth. I should rather say that these are the questions which they are always arguing, especially in courts of law: they commit all sorts of crimes, and there is nothing which they will not do or say in their own defence.

Soc. But do they admit their guilt, Euthyphro, and yet say that they ought not to be punished?

Euth. No; they do not.

Soc. Then there are some things which they do not venture to say and do: for they do not venture to argue that the guilty are to be unpunished, but they deny their guilt, do they not?

Euth. Yes.

Soc. Then they do not argue that the evil-doer should not be punished, but they argue about the fact of who the evil-doer is, and what he did and when?

Euth. True.

Soc. And the gods are in the same case, if as you assert they quarrel about just and unjust, and some of them say while others deny that injustice is done among them. For surely neither God nor man will ever venture to say that the doer of injustice is not to be punished?

Euth. That is true, Socrates, in the main.

Soc. But they join issue about the particulars, gods and men alike; and, if they dispute at all, they dispute about some act which is called in question, and which by some is affirmed to be just, by others to be unjust. Is not that true?

Euth. Quite true.

Soc. Well then, my dear friend Euthyphro, do tell me, for my better instruction and information, what proof have you that in the opinion of all the gods a servant who is guilty of murder, and is put in chains by the master of the dead man, and dies because he is put in chains before he who bound him can learn from the interpreters of the gods what he ought to do with him, dies unjustly; and that on behalf of such an one a son ought to proceed against his father and accuse him of murder. How would you show that all the gods absolutely agree in approving of his act? Prove to me that they do, and I will applaud your wisdom as long as I live.

Euth. It will be a difficult task; but I could make the matter very dear indeed to you.

Soc. I understand; you mean to say that I am not so quick of apprehension as the judges: for to them you will be sure to prove that the act is unjust, and hateful to the gods.

Euth. Yes indeed, Socrates; at least if they will listen to me.

Soc. But they will be sure to listen if they find that you are a good speaker. There was a notion that came into my mind while you were speaking; I said to myself: `Well, and what if Euthyphro does prove to me that all the gods regarded the death of the serf as unjust, how do I know anything more of the nature of piety and impiety? for granting that this action may be hateful to the gods, still piety and impiety are not adequately defined by these distinctions, for that which is hateful to the gods has been shown to be also pleasing and dear to them.' And therefore, Euthyphro, I do not ask you to prove this; I will suppose, if you like, that all the gods condemn and abominate such an action. But I will amend the definition so far as to say that what all the gods hate is impious, and what they love pious or holy; and what some of them love and others hate is both or neither. Shall this be our definition of piety and impiety?

Euth. Why not, Socrates?

Soc. Why not ! certainly, as far as I am concerned, Euthyphro, there is no reason why not. But whether this admission will greatly assist you in the task of instructing me as you promised, is a matter for you to consider.

Euth. Yes, I should say that what all the gods love is pious and holy, and the opposite which they all hate, impious.

Soc. Ought we to enquire into the truth of this, Euthyphro, or simply to accept the mere statement on our own authority and that of others? What do you say?

Euth. We should enquire; and I believe that the statement will stand the test of enquiry.

Soc. We shall know better, my good friend, in a little while. The point which I should first wish to understand is whether the pious or holy is beloved by the gods because it is holy, or holy because it is beloved of the gods.

Euth. I do not understand your meaning, Socrates.

Soc. I will endeavour to explain: we, speak of carrying and we speak of being carried, of leading and being led, seeing and being seen. You know that in all such cases there is a difference, and you know also in what the difference lies?

Euth. I think that I understand.

Soc. And is not that which is beloved distinct from that which loves?

Euth. Certainly.

Soc. Well; and now tell me, is that which is carried in this state of carrying because it is carried, or for some other reason?

Euth. No; that is the reason.

Soc. And the same is true of what is led and of what is seen?

Euth. True.

Soc. And a thing is not seen because it is visible, but conversely, visible because it is seen; nor is a thing led because it is in the state of being led, or carried because it is in the state of being carried, but the converse of this. And now I think, Euthyphro, that my meaning will be intelligible; and my meaning is, that any state of action or passion implies previous action or passion. It does not become because it is becoming, but it is in a state of becoming because it becomes; neither does it suffer because it is in a state of suffering, but it is in a state of suffering because it suffers. Do you not agree?

Euth. Yes.

Soc. Is not that which is loved in some state either of becoming or suffering?

Euth. Yes.

Soc. And the same holds as in the previous instances; the state of being loved follows the act of being loved, and not the act the state.

Euth. Certainly.

Soc. And what do you say of piety, Euthyphro: is not piety, according to your definition, loved by all the gods?

Euth. Yes.

Soc. Because it is pious or holy, or for some other reason?

Euth. No, that is the reason.

Soc. It is loved because it is holy, not holy because it is loved?

Euth. Yes.

Soc. And that which is dear to the gods is loved by them, and is in a state to be loved of them because it is loved of them?

Euth. Certainly.

Soc. Then that which is dear to the gods, Euthyphro, is not holy, nor is that which is holy loved of God, as you affirm; but they are two different things.

Euth. How do you mean, Socrates?

Soc. I mean to say that the holy has been acknowledge by us to be loved of God because it is holy, not to be holy because it is loved.

Euth. Yes.

Soc. But that which is dear to the gods is dear to them because it is loved by them, not loved by them because it is dear to them.

Euth. True.

Soc. But, friend Euthyphro, if that which is holy is the same with that which is dear to God, and is loved because it is holy, then that which is dear to God would have been loved as being dear to God; but if that which dear to God is dear to him because loved by him, then that which is holy would have been holy because loved by him. But now you see that the reverse is the case, and that they are quite different from one another. For one (theophiles) is of a kind to be loved cause it is loved, and the other (osion) is loved because it is of a kind to be loved. Thus you appear to me, Euthyphro, when I ask you what is the essence of holiness, to offer an attribute only, and not the essence, the attribute of being loved by all the gods. But you still refuse to explain to me the nature of holiness. And therefore, if you please, I will ask you not to hide your treasure, but to tell me once more what holiness or piety really is, whether dear to the gods or not (for that is a matter about which we will not quarrel) and what is impiety?

Euth. I really do not know, Socrates, how to express what I mean. For somehow or other our arguments, on whatever ground we rest them, seem to turn round and walk away from us.

Soc. Your words, Euthyphro, are like the handiwork of my ancestor Daedalus; and if I were the sayer or propounder of them, you might say that my arguments walk away and will not remain fixed where they are placed because I am a descendant of his. But now, since these notions are your own, you must find some other gibe, for they certainly, as you yourself allow, show an inclination to be on the move.

Euth. Nay, Socrates, I shall still say that you are the Daedalus who sets arguments in motion; not I, certainly, but you make them move or go round, for they would never have stirred, as far as I am concerned.

Soc. Then I must be a greater than Daedalus: for whereas he only made his own inventions to move, I move those of other people as well. And the beauty of it is, that I would rather not. For I would give the wisdom of Daedalus, and the wealth of Tantalus, to be able to detain them and keep them fixed. But enough of this. As I perceive that you are lazy, I will myself endeavor to show you how you might instruct me in the nature of piety; and I hope that you will not grudge your labour. Tell me, then, Is not that which is pious necessarily just?

Euth. Yes.

Soc. And is, then, all which is just pious? or, is that which is pious all just, but that which is just, only in part and not all, pious?

Euth. I do not understand you, Socrates.

Soc. And yet I know that you are as much wiser than I am, as you are younger. But, as I was saying, revered friend, the abundance of your wisdom makes you lazy. Please to exert yourself, for there is no real difficulty in understanding me. What I mean I may explain by an illustration of what I do not mean. The poet (Stasinus) sings,

Of Zeus, the author and creator of all these things,

You will not tell: for where there is fear there is also reverence.

Now I disagree with this poet. Shall I tell you in what respect?

Euth. By all means.

Soc. I should not say that where there is fear there is also reverence; for I am sure that many persons fear poverty and disease, and the like evils, but I do not perceive that they reverence the objects of their fear.

Euth. Very true.

Soc. But where reverence is, there is fear; for he who has a feeling of reverence and shame about the commission of any action, fears and is afraid of an ill reputation.

Euth. No doubt.

Soc. Then we are wrong in saying that where there is fear there is also reverence; and we should say, where there is reverence there is also fear. But there is not always reverence where there is fear; for fear is a more extended notion, and reverence is a part of fear, just as the odd is a part of number, and number is a more extended notion than the odd. I suppose that you follow me now?

Euth. Quite well.

Soc. That was the sort of question which I meant to raise when I asked whether the just is always the pious, or the pious always the just; and whether there may not be justice where there is not piety; for justice is the more extended notion of which piety is only a part. Do you dissent?

Euth. No, I think that you are quite right.

Soc. Then, if piety is a part of justice, I suppose that we should enquire what part? If you had pursued the enquiry in the previous cases; for instance, if you had asked me what is an even number, and what part of number the even is, I should have had no difficulty in replying, a number which represents a figure having two equal sides. Do you not agree?

Euth. Yes, I quite agree.

Soc. In like manner, I want you to tell me what part of justice is piety or holiness, that I may be able to tell Meletus not to do me injustice, or indict me for impiety, as I am now adequately instructed by you in the nature of piety or holiness, and their opposites.

Euth. Piety or holiness, Socrates, appears to me to be that part of justice which attends to the gods, as there is the other part of justice which attends to men.

Soc. That is good, Euthyphro; yet still there is a little point about which I should like to have further information, What is the meaning of `attention'? For attention can hardly be used in the same sense when applied to the gods as when applied to other things. For instance, horses are said to require attention, and not every person is able to attend to them, but only a person skilled in horsemanship. Is it not so?

Euth. Certainly.

Soc. I should suppose that the art of horsemanship is the art of attending to horses?

Euth. Yes.

Soc. Nor is every one qualified to attend to dogs, but only the huntsman?

Euth. True.

Soc. And I should also conceive that the art of the huntsman is the art of attending to dogs?

Euth. Yes.

Soc. As the art of the ox herd is the art of attending to oxen?

Euth. Very true.

Soc. In like manner holiness or piety is the art of attending to the gods?, that would be your meaning, Euthyphro?

Euth. Yes.

Soc. And is not attention always designed for the good or benefit of that to which the attention is given? As in the case of horses, you may observe that when attended to by the horseman's art they are benefited and improved, are they not?

Euth. True.

Soc. As the dogs are benefited by the huntsman's art, and the oxen by the art of the ox herd, and all other things are tended or attended for their good and not for their hurt?

Euth. Certainly, not for their hurt.

Soc. But for their good?

Euth. Of course.

Soc. And does piety or holiness, which has been defined to be the art of attending to the gods, benefit or improve them? Would you say that when you do a holy act you make any of the gods better?

Euth. No, no; that was certainly not what I meant.

Soc. And I, Euthyphro, never supposed that you did. I asked you the question about the nature of the attention, because I thought that you did not.

Euth. You do me justice, Socrates; that is not the sort of attention which I mean.

Soc. Good: but I must still ask what is this attention to the gods which is called piety?

Euth. It is such, Socrates, as servants show to their masters.

Soc. I understand, a sort of ministration to the gods.

Euth. Exactly.

Soc. Medicine is also a sort of ministration or service, having in view the attainment of some object, would you not say of health?

Euth. I should.

Soc. Again, there is an art which ministers to the ship-builder with a view to the attainment of some result?

Euth. Yes, Socrates, with a view to the building of a ship.

Soc. As there is an art which ministers to the housebuilder with a view to the building of a house?

Euth. Yes.

Soc. And now tell me, my good friend, about the art which ministers to the gods: what work does that help to accomplish? For you must surely know if, as you say, you are of all men living the one who is best instructed in religion.

Euth. And I speak the truth, Socrates.

Soc. Tell me then, oh tell me, what is that fair work which the gods do by the help of our ministrations?

Euth. Many and fair, Socrates, are the works which they do. Soc. Why, my friend, and so are those of a general. But the chief of them is easily told. Would you not say that victory in war is the chief of them?

Euth. Certainly.

Soc. Many and fair, too, are the works of the husbandman, if I am not mistaken; but his chief work is the production of food from the earth?

Euth. Exactly.

Soc. And of the many and fair things done by the gods, which is the chief or principal one?

Euth. I have told you already, Socrates, that to learn all these things accurately will be very tiresome. Let me simply say that piety or holiness is learning, how to please the gods in word and deed, by prayers and sacrifices. Such piety, is the salvation of families and states, just as the impious, which is unpleasing to the gods, is their ruin and destruction.

Soc. I think that you could have answered in much fewer words the chief question which I asked, Euthyphro, if you had chosen. But I see plainly that you are not disposed to instruct me, dearly not: else why, when we reached the point, did you turn, aside? Had you only answered me I should have truly learned of you by this time the nature of piety. Now, as the asker of a question is necessarily dependent on the answerer, whither he leads, I must follow; and can only ask again, what is the pious, and what is piety? Do you mean that they are a, sort of science of praying and sacrificing?

Euth. Yes, I do.

Soc. And sacrificing is giving to the gods, and prayer is asking of the gods?

Euth. Yes, Socrates.

Soc. Upon this view, then piety is a science of asking and giving?

Euth. You understand me capitally, Socrates.

Soc. Yes, my friend; the. reason is that I am a votary of your science, and give my mind to it, and therefore nothing which you say will be thrown away upon me. Please then to tell me, what is the nature of this service to the gods? Do you mean that we prefer requests and give gifts to them?

Euth. Yes, I do.

Soc. Is not the right way of asking to ask of them what we want?

Euth. Certainly.

Soc. And the right way of giving is to give to them in return what they want of us. There would be no, in an art which gives to any one that which he does not want.

Euth. Very true, Socrates.

Soc. Then piety, Euthyphro, is an art which gods and men have of doing business with one another?

Euth. That is an expression which you may use, if you like.

Soc. But I have no particular liking for anything but the truth. I wish, however, that you would tell me what benefit accrues to the gods from our gifts. There is no doubt about what they give to us; for there is no good thing which they do not give; but how we can give any good thing to them in return is far from being equally clear. If they give everything and we give nothing, that must be an affair of business in which we have very greatly the advantage of them.

Euth. And do you imagine, Socrates, that any benefit accrues to the gods from our gifts?

Soc. But if not, Euthyphro, what is the meaning of gifts which are conferred by us upon the gods?

Euth. What else, but tributes of honour; and, as I was just now saying, what pleases them?

Soc. Piety, then, is pleasing to the gods, but not beneficial or dear to them?

Euth. I should say that nothing could be dearer.

Soc. Then once more the assertion is repeated that piety is dear to the gods?

Euth. Certainly.

Soc. And when you say this, can you wonder at your words not standing firm, but walking away? Will you accuse me of being the Daedalus who makes them walk away, not perceiving that there is another and far greater artist than Daedalus who makes them go round in a circle, and he is yourself; for the argument, as you will perceive, comes round to the same point. Were we not saying that the holy or pious was not the same with that which is loved of the gods? Have you forgotten?

Euth. I quite remember.

Soc. And are you not saying that what is loved of the gods is holy; and is not this the same as what is dear to them, do you see?

Euth. True.

Soc. Then either we were wrong in former assertion; or, if we were right then, we are wrong now.

Euth. One of the two must be true.

Soc. Then we must begin again and ask, What is piety? That is an enquiry which I shall never be weary of pursuing as far as in me lies; and I entreat you not to scorn me, but to apply your mind to the utmost, and tell me the truth. For, if any man knows, you are he; and therefore I must detain you, like Proteus, until you tell. If you had not certainly known the nature of piety and impiety, I am confident that you would never, on behalf of a serf, have charged your aged father with murder. You would not have run such a risk of doing wrong in the sight of the gods, and you would have had too much respect for the opinions of men. I am sure, therefore, that you know the nature of piety and impiety. Speak out then, my dear Euthyphro, and do not hide your knowledge.

Euth. Another time, Socrates; for I am in a hurry, and must go now.

Soc. Alas ! my companion, and will you leave me in despair? I was hoping that you would instruct me in the nature of piety and impiety; and then I might have cleared myself of Meletus and his indictment. I would have told him that I had been enlightened by Euthyphro, and had given up rash innovations and speculations, in which I indulged only through ignorance, and that now I am about to lead a better life.


% chapter euthyphro (end)


% \chapter{Crito} % (fold)
\label{cha:crito}



Translated by Benjamin Jowett

Persons of the Dialogue
SOCRATES
CRITO

Scene
The Prison of Socrates.
----------------------------------------------------------------------

Socrates. WHY have you come at this hour, Crito? it must be quite
early. 

Crito. Yes, certainly. 

Soc. What is the exact time? 

Cr. The dawn is breaking. 

Soc. I wonder the keeper of the prison would let you in.

Cr. He knows me because I often come, Socrates; moreover. I have done
him a kindness. 

Soc. And are you only just come? 

Cr. No, I came some time ago. 

Soc. Then why did you sit and say nothing, instead of awakening me
at once? 

Cr. Why, indeed, Socrates, I myself would rather not have all this
sleeplessness and sorrow. But I have been wondering at your peaceful
slumbers, and that was the reason why I did not awaken you, because
I wanted you to be out of pain. I have always thought you happy in
the calmness of your temperament; but never did I see the like of
the easy, cheerful way in which you bear this calamity. 

Soc. Why, Crito, when a man has reached my age he ought not to be
repining at the prospect of death. 

Cr. And yet other old men find themselves in similar misfortunes,
and age does not prevent them from repining. 

Soc. That may be. But you have not told me why you come at this early
hour. 

Cr. I come to bring you a message which is sad and painful; not, as
I believe, to yourself but to all of us who are your friends, and
saddest of all to me. 

Soc. What! I suppose that the ship has come from Delos, on the arrival
of which I am to die? 

Cr. No, the ship has not actually arrived, but she will probably be
here to-day, as persons who have come from Sunium tell me that they
have left her there; and therefore to-morrow, Socrates, will be the
last day of your life. 

Soc. Very well, Crito; if such is the will of God, I am willing; but
my belief is that there will be a delay of a day. 

Cr. Why do you say this? 

Soc. I will tell you. I am to die on the day after the arrival of
the ship? 

Cr. Yes; that is what the authorities say. 

Soc. But I do not think that the ship will be here until to-morrow;
this I gather from a vision which I had last night, or rather only
just now, when you fortunately allowed me to sleep. 

Cr. And what was the nature of the vision? 

Soc. There came to me the likeness of a woman, fair and comely, clothed
in white raiment, who called to me and said: O Socrates-

``The third day hence, to Phthia shalt thou go." 

Cr. What a singular dream, Socrates! 

Soc. There can be no doubt about the meaning Crito, I think.

Cr. Yes: the meaning is only too clear. But, O! my beloved Socrates,
let me entreat you once more to take my advice and escape. For if
you die I shall not only lose a friend who can never be replaced,
but there is another evil: people who do not know you and me will
believe that I might have saved you if I had been willing to give
money, but that I did not care. Now, can there be a worse disgrace
than this- that I should be thought to value money more than the life
of a friend? For the many will not be persuaded that I wanted you
to escape, and that you refused. 

Soc. But why, my dear Crito, should we care about the opinion of the
many? Good men, and they are the only persons who are worth considering,
will think of these things truly as they happened. 

Cr. But do you see. Socrates, that the opinion of the many must be
regarded, as is evident in your own case, because they can do the
very greatest evil to anyone who has lost their good opinion?

Soc. I only wish, Crito, that they could; for then they could also
do the greatest good, and that would be well. But the truth is, that
they can do neither good nor evil: they cannot make a man wise or
make him foolish; and whatever they do is the result of chance.

Cr. Well, I will not dispute about that; but please to tell me, Socrates,
whether you are not acting out of regard to me and your other friends:
are you not afraid that if you escape hence we may get into trouble
with the informers for having stolen you away, and lose either the
whole or a great part of our property; or that even a worse evil may
happen to us? Now, if this is your fear, be at ease; for in order
to save you, we ought surely to run this or even a greater risk; be
persuaded, then, and do as I say. 

Soc. Yes, Crito, that is one fear which you mention, but by no means
the only one. 

Cr. Fear not. There are persons who at no great cost are willing to
save you and bring you out of prison; and as for the informers, you
may observe that they are far from being exorbitant in their demands;
a little money will satisfy them. My means, which, as I am sure, are
ample, are at your service, and if you have a scruple about spending
all mine, here are strangers who will give you the use of theirs;
and one of them, Simmias the Theban, has brought a sum of money for
this very purpose; and Cebes and many others are willing to spend
their money too. I say, therefore, do not on that account hesitate
about making your escape, and do not say, as you did in the court,
that you will have a difficulty in knowing what to do with yourself
if you escape. For men will love you in other places to which you
may go, and not in Athens only; there are friends of mine in Thessaly,
if you like to go to them, who will value and protect you, and no
Thessalian will give you any trouble. Nor can I think that you are
justified, Socrates, in betraying your own life when you might be
saved; this is playing into the hands of your enemies and destroyers;
and moreover I should say that you were betraying your children; for
you might bring them up and educate them; instead of which you go
away and leave them, and they will have to take their chance; and
if they do not meet with the usual fate of orphans, there will be
small thanks to you. No man should bring children into the world who
is unwilling to persevere to the end in their nurture and education.
But you are choosing the easier part, as I think, not the better and
manlier, which would rather have become one who professes virtue in
all his actions, like yourself. And, indeed, I am ashamed not only
of you, but of us who are your friends, when I reflect that this entire
business of yours will be attributed to our want of courage. The trial
need never have come on, or might have been brought to another issue;
and the end of all, which is the crowning absurdity, will seem to
have been permitted by us, through cowardice and baseness, who might
have saved you, as you might have saved yourself, if we had been good
for anything (for there was no difficulty in escaping); and we did
not see how disgraceful, Socrates, and also miserable all this will
be to us as well as to you. Make your mind up then, or rather have
your mind already made up, for the time of deliberation is over, and
there is only one thing to be done, which must be done, if at all,
this very night, and which any delay will render all but impossible;
I beseech you therefore, Socrates, to be persuaded by me, and to do
as I say. 

Soc. Dear Crito, your zeal is invaluable, if a right one; but if wrong,
the greater the zeal the greater the evil; and therefore we ought
to consider whether these things shall be done or not. For I am and
always have been one of those natures who must be guided by reason,
whatever the reason may be which upon reflection appears to me to
be the best; and now that this fortune has come upon me, I cannot
put away the reasons which I have before given: the principles which
I have hitherto honored and revered I still honor, and unless we can
find other and better principles on the instant, I am certain not
to agree with you; no, not even if the power of the multitude could
inflict many more imprisonments, confiscations, deaths, frightening
us like children with hobgoblin terrors. But what will be the fairest
way of considering the question? Shall I return to your old argument
about the opinions of men, some of which are to be regarded, and others,
as we were saying, are not to be regarded? Now were we right in maintaining
this before I was condemned? And has the argument which was once good
now proved to be talk for the sake of talking; in fact an amusement
only, and altogether vanity? That is what I want to consider with
your help, Crito: whether, under my present circumstances, the argument
appears to be in any way different or not; and is to be allowed by
me or disallowed. That argument, which, as I believe, is maintained
by many who assume to be authorities, was to the effect, as I was
saying, that the opinions of some men are to be regarded, and of other
men not to be regarded. Now you, Crito, are a disinterested person
who are not going to die to-morrow- at least, there is no human probability
of this, and you are therefore not liable to be deceived by the circumstances
in which you are placed. Tell me, then, whether I am right in saying
that some opinions, and the opinions of some men only, are to be valued,
and other opinions, and the opinions of other men, are not to be valued.
I ask you whether I was right in maintaining this? 

Cr. Certainly. 

Soc. The good are to be regarded, and not the bad? 

Cr. Yes. 

Soc. And the opinions of the wise are good, and the opinions of the
unwise are evil? 

Cr. Certainly. 

Soc. And what was said about another matter? Was the disciple in gymnastics
supposed to attend to the praise and blame and opinion of every man,
or of one man only- his physician or trainer, whoever that was?

Cr. Of one man only. 

Soc. And he ought to fear the censure and welcome the praise of that
one only, and not of the many? 

Cr. That is clear. 

Soc. And he ought to live and train, and eat and drink in the way
which seems good to his single master who has understanding, rather
than according to the opinion of all other men put together?

Cr. True. 

Soc. And if he disobeys and disregards the opinion and approval of
the one, and regards the opinion of the many who have no understanding,
will he not suffer evil? 

Cr. Certainly he will. 

Soc. And what will the evil be, whither tending and what affcting,
in the disobedient person? 

Cr. Clearly, affecting the body; that is what is destroyed by the
evil. 

Soc. Very good; and is not this true, Crito, of other things which
we need not separately enumerate? In the matter of just and unjust,
fair and foul, good and evil, which are the subjects of our present
consultation, ought we to follow the opinion of the many and to fear
them; or the opinion of the one man who has understanding, and whom
we ought to fear and reverence more than all the rest of the world:
and whom deserting we shall destroy and injure that principle in us
which may be assumed to be improved by justice and deteriorated by
injustice; is there not such a principle? 

Cr. Certainly there is, Socrates. 

Soc. Take a parallel instance; if, acting under the advice of men
who have no understanding, we destroy that which is improvable by
health and deteriorated by disease- when that has been destroyed,
I say, would life be worth having? And that is- the body?

Cr. Yes. 

Soc. Could we live, having an evil and corrupted body? 

Cr. Certainly not. 

Soc. And will life be worth having, if that higher part of man be
depraved, which is improved by justice and deteriorated by injustice?
Do we suppose that principle, whatever it may be in man, which has
to do with justice and injustice, to be inferior to the body?

Cr. Certainly not. 

Soc. More honored, then? 

Cr. Far more honored. 

Soc. Then, my friend, we must not regard what the many say of us:
but what he, the one man who has understanding of just and unjust,
will say, and what the truth will say. And therefore you begin in
error when you suggest that we should regard the opinion of the many
about just and unjust, good and evil, honorable and dishonorable.
Well, someone will say, ``But the many can kill us." 

Cr. Yes, Socrates; that will clearly be the answer. 

Soc. That is true; but still I find with surprise that the old argument
is, as I conceive, unshaken as ever. And I should like to know Whether
I may say the same of another proposition- that not life, but a good
life, is to be chiefly valued? 

Cr. Yes, that also remains. 

Soc. And a good life is equivalent to a just and honorable one- that
holds also? 

Cr. Yes, that holds. 

Soc. From these premises I proceed to argue the question whether I
ought or ought not to try to escape without the consent of the Athenians:
and if I am clearly right in escaping, then I will make the attempt;
but if not, I will abstain. The other considerations which you mention,
of money and loss of character, and the duty of educating children,
are, I fear, only the doctrines of the multitude, who would be as
ready to call people to life, if they were able, as they are to put
them to death- and with as little reason. But now, since the argument
has thus far prevailed, the only question which remains to be considered
is, whether we shall do rightly either in escaping or in suffering
others to aid in our escape and paying them in money and thanks, or
whether we shan not do rightly; and if the latter, then death or any
other calamity which may ensue on my remaining here must not be allowed
to enter into the calculation. 

Cr. I think that you are right, Socrates; how then shall we proceed?

Soc. Let us consider the matter together, and do you either refute
me if you can, and I will be convinced; or else cease, my dear friend,
from repeating to me that I ought to escape against the wishes of
the Athenians: for I am extremely desirous to be persuaded by you,
but not against my own better judgment. And now please to consider
my first position, and do your best to answer me. 

Cr. I will do my best. 

Soc. Are we to say that we are never intentionally to do wrong, or
that in one way we ought and in another way we ought not to do wrong,
or is doing wrong always evil and dishonorable, as I was just now
saying, and as has been already acknowledged by us? Are all our former
admissions which were made within a few days to be thrown away? And
have we, at our age, been earnestly discoursing with one another all
our life long only to discover that we are no better than children?
Or are we to rest assured, in spite of the opinion of the many, and
in spite of consequences whether better or worse, of the truth of
what was then said, that injustice is always an evil and dishonor
to him who acts unjustly? Shall we affirm that? 

Cr. Yes. 

Soc. Then we must do no wrong? 

Cr. Certainly not. 

Soc. Nor when injured injure in return, as the many imagine; for we
must injure no one at all? 

Cr. Clearly not. 

Soc. Again, Crito, may we do evil? 

Cr. Surely not, Socrates. 

Soc. And what of doing evil in return for evil, which is the morality
of the many-is that just or not? 

Cr. Not just. 

Soc. For doing evil to another is the same as injuring him?

Cr. Very true. 

Soc. Then we ought not to retaliate or render evil for evil to anyone,
whatever evil we may have suffered from him. But I would have you
consider, Crito, whether you really mean what you are saying. For
this opinion has never been held, and never will be held, by any considerable
number of persons; and those who are agreed and those who are not
agreed upon this point have no common ground, and can only despise
one another, when they see how widely they differ. Tell me, then,
whether you agree with and assent to my first principle, that neither
injury nor retaliation nor warding off evil by evil is ever right.
And shall that be the premise of our agreement? Or do you decline
and dissent from this? For this has been of old and is still my opinion;
but, if you are of another opinion, let me hear what you have to say.
If, however, you remain of the same mind as formerly, I will proceed
to the next step. 

Cr. You may proceed, for I have not changed my mind. 

Soc. Then I will proceed to the next step, which may be put in the
form of a question: Ought a man to do what he admits to be right,
or ought he to betray the right? 

Cr. He ought to do what he thinks right. 

Soc. But if this is true, what is the application? In leaving the
prison against the will of the Athenians, do I wrong any? or rather
do I not wrong those whom I ought least to wrong? Do I not desert
the principles which were acknowledged by us to be just? What do you
say? 

Cr. I cannot tell, Socrates, for I do not know. 

Soc. Then consider the matter in this way: Imagine that I am about
to play truant (you may call the proceeding by any name which you
like), and the laws and the government come and interrogate me: ``Tell
us, Socrates," they say; ``what are you about? are you going by an
act of yours to overturn us- the laws and the whole State, as far
as in you lies? Do you imagine that a State can subsist and not be
overthrown, in which the decisions of law have no power, but are set
aside and overthrown by individuals?" What will be our answer, Crito,
to these and the like words? Anyone, and especially a clever rhetorician,
will have a good deal to urge about the evil of setting aside the
law which requires a sentence to be carried out; and we might reply,
"Yes; but the State has injured us and given an unjust sentence."
Suppose I say that? 

Cr. Very good, Socrates. 
Soc. ``And was that our agreement with you?" the law would sar, ``or
were you to abide by the sentence of the State?" And if I were to
express astonishment at their saying this, the law would probably
add: ``Answer, Socrates, instead of opening your eyes: you are in the
habit of asking and answering questions. Tell us what complaint you
have to make against us which justifies you in attempting to destroy
us and the State? In the first place did we not bring you into existence?
Your father married your mother by our aid and begat you. Say whether
you have any objection to urge against those of us who regulate marriage?"
None, I should reply. ``Or against those of us who regulate the system
of nurture and education of children in which you were trained? Were
not the laws, who have the charge of this, right in commanding your
father to train you in music and gymnastic?" Right, I should reply.
``Well, then, since you were brought into the world and nurtured and
educated by us, can you deny in the first place that you are our child
and slave, as your fathers were before you? And if this is true you
are not on equal terms with us; nor can you think that you have a
right to do to us what we are doing to you. Would you have any right
to strike or revile or do any other evil to a father or to your master,
if you had one, when you have been struck or reviled by him, or received
some other evil at his hands?- you would not say this? And because
we think right to destroy you, do you think that you have any right
to destroy us in return, and your country as far as in you lies? And
will you, O professor of true virtue, say that you are justified in
this? Has a philosopher like you failed to discover that our country
is more to be valued and higher and holier far than mother or father
or any ancestor, and more to be regarded in the eyes of the gods and
of men of understanding? also to be soothed, and gently and reverently
entreated when angry, even more than a father, and if not persuaded,
obeyed? And when we are punished by her, whether with imprisonment
or stripes, the punishment is to be endured in silence; and if she
leads us to wounds or death in battle, thither we follow as is right;
neither may anyone yield or retreat or leave his rank, but whether
in battle or in a court of law, or in any other place, he must do
what his city and his country order him; or he must change their view
of what is just: and if he may do no violence to his father or mother,
much less may he do violence to his country." What answer shall we
make to this, Crito? Do the laws speak truly, or do they not?

Cr. I think that they do. 

Soc. Then the laws will say: ``Consider, Socrates, if this is true,
that in your present attempt you are going to do us wrong. For, after
having brought you into the world, and nurtured and educated you,
and given you and every other citizen a share in every good that we
had to give, we further proclaim and give the right to every Athenian,
that if he does not like us when he has come of age and has seen the
ways of the city, and made our acquaintance, he may go where he pleases
and take his goods with him; and none of us laws will forbid him or
interfere with him. Any of you who does not like us and the city,
and who wants to go to a colony or to any other city, may go where
he likes, and take his goods with him. But he who has experience of
the manner in which we order justice and administer the State, and
still remains, has entered into an implied contract that he will do
as we command him. And he who disobeys us is, as we maintain, thrice
wrong: first, because in disobeying us he is disobeying his parents;
secondly, because we are the authors of his education; thirdly, because
he has made an agreement with us that he will duly obey our commands;
and he neither obeys them nor convinces us that our commands are wrong;
and we do not rudely impose them, but give him the alternative of
obeying or convincing us; that is what we offer and he does neither.
These are the sort of accusations to which, as we were saying, you,
Socrates, will be exposed if you accomplish your intentions; you,
above all other Athenians." Suppose I ask, why is this? they will
justly retort upon me that I above all other men have acknowledged
the agreement. ``There is clear proof," they will say, ``Socrates, that
we and the city were not displeasing to you. Of all Athenians you
have been the most constant resident in the city, which, as you never
leave, you may be supposed to love. For you never went out of the
city either to see the games, except once when you went to the Isthmus,
or to any other place unless when you were on military service; nor
did you travel as other men do. Nor had you any curiosity to know
other States or their laws: your affections did not go beyond us and
our State; we were your especial favorites, and you acquiesced in
our government of you; and this is the State in which you begat your
children, which is a proof of your satisfaction. Moreover, you might,
if you had liked, have fixed the penalty at banishment in the course
of the trial-the State which refuses to let you go now would have
let you go then. But you pretended that you preferred death to exile,
and that you were not grieved at death. And now you have forgotten
these fine sentiments, and pay no respect to us, the laws, of whom
you are the destroyer; and are doing what only a miserable slave would
do, running away and turning your back upon the compacts and agreements
which you made as a citizen. And first of all answer this very question:
Are we right in saying that you agreed to be governed according to
us in deed, and not in word only? Is that true or not?" How shall
we answer that, Crito? Must we not agree? 

Cr. There is no help, Socrates. 

Soc. Then will they not say: ``You, Socrates, are breaking the covenants
and agreements which you made with us at your leisure, not in any
haste or under any compulsion or deception, but having had seventy
years to think of them, during which time you were at liberty to leave
the city, if we were not to your mind, or if our covenants appeared
to you to be unfair. You had your choice, and might have gone either
to Lacedaemon or Crete, which you often praise for their good government,
or to some other Hellenic or foreign State. Whereas you, above all
other Athenians, seemed to be so fond of the State, or, in other words,
of us her laws (for who would like a State that has no laws?), that
you never stirred out of her: the halt, the blind, the maimed, were
not more stationary in her than you were. And now you run away and
forsake your agreements. Not so, Socrates, if you will take our advice;
do not make yourself ridiculous by escaping out of the city.

"For just consider, if you transgress and err in this sort of way,
what good will you do, either to yourself or to your friends? That
your friends will be driven into exile and deprived of citizenship,
or will lose their property, is tolerably certain; and you yourself,
if you fly to one of the neighboring cities, as, for example, Thebes
or Megara, both of which are well-governed cities, will come to them
as an enemy, Socrates, and their government will be against you, and
all patriotic citizens will cast an evil eye upon you as a subverter
of the laws, and you will confirm in the minds of the judges the justice
of their own condemnation of you. For he who is a corrupter of the
laws is more than likely to be corrupter of the young and foolish
portion of mankind. Will you then flee from well-ordered cities and
virtuous men? and is existence worth having on these terms? Or will
you go to them without shame, and talk to them, Socrates? And what
will you say to them? What you say here about virtue and justice and
institutions and laws being the best things among men? Would that
be decent of you? Surely not. But if you go away from well-governed
States to Crito's friends in Thessaly, where there is great disorder
and license, they will be charmed to have the tale of your escape
from prison, set off with ludicrous particulars of the manner in which
you were wrapped in a goatskin or some other disguise, and metamorphosed
as the fashion of runaways is- that is very likely; but will there
be no one to remind you that in your old age you violated the most
sacred laws from a miserable desire of a little more life? Perhaps
not, if you keep them in a good temper; but if they are out of temper
you will hear many degrading things; you will live, but how?- as the
flatterer of all men, and the servant of all men; and doing what?-
eating and drinking in Thessaly, having gone abroad in order that
you may get a dinner. And where will be your fine sentiments about
justice and virtue then? Say that you wish to live for the sake of
your children, that you may bring them up and educate them- will you
take them into Thessaly and deprive them of Athenian citizenship?
Is that the benefit which you would confer upon them? Or are you under
the impression that they will be better cared for and educated here
if you are still alive, although absent from them; for that your friends
will take care of them? Do you fancy that if you are an inhabitant
of Thessaly they will take care of them, and if you are an inhabitant
of the other world they will not take care of them? Nay; but if they
who call themselves friends are truly friends, they surely will.

``Listen, then, Socrates, to us who have brought you up. Think not
of life and children first, and of justice afterwards, but of justice
first, that you may be justified before the princes of the world below.
For neither will you nor any that belong to you be happier or holier
or juster in this life, or happier in another, if you do as Crito
bids. Now you depart in innocence, a sufferer and not a doer of evil;
a victim, not of the laws, but of men. But if you go forth, returning
evil for evil, and injury for injury, breaking the covenants and agreements
which you have made with us, and wronging those whom you ought least
to wrong, that is to say, yourself, your friends, your country, and
us, we shall be angry with you while you live, and our brethren, the
laws in the world below, will receive you as an enemy; for they will
know that you have done your best to destroy us. Listen, then, to
us and not to Crito." 

This is the voice which I seem to hear murmuring in my ears, like
the sound of the flute in the ears of the mystic; that voice, I say,
is humming in my ears, and prevents me from hearing any other. And
I know that anything more which you will say will be in vain. Yet
speak, if you have anything to say. 

Cr. I have nothing to say, Socrates. 

Soc. Then let me follow the intimations of the will of God.

THE END

% chapter crito (end)
% \chapter{Apology} % (fold)
\label{cha:apology}

Apology
By Plato


Translated by Benjamin Jowett

Socrates' Defense

How you have felt, O men of Athens, at hearing the speeches of my
accusers, I cannot tell; but I know that their persuasive words almost
made me forget who I was - such was the effect of them; and yet they
have hardly spoken a word of truth. But many as their falsehoods were,
there was one of them which quite amazed me; - I mean when they told
you to be upon your guard, and not to let yourselves be deceived by
the force of my eloquence. They ought to have been ashamed of saying
this, because they were sure to be detected as soon as I opened my
lips and displayed my deficiency; they certainly did appear to be
most shameless in saying this, unless by the force of eloquence they
mean the force of truth; for then I do indeed admit that I am eloquent.
But in how different a way from theirs! Well, as I was saying, they
have hardly uttered a word, or not more than a word, of truth; but
you shall hear from me the whole truth: not, however, delivered after
their manner, in a set oration duly ornamented with words and phrases.
No indeed! but I shall use the words and arguments which occur to
me at the moment; for I am certain that this is right, and that at
my time of life I ought not to be appearing before you, O men of Athens,
in the character of a juvenile orator - let no one expect this of
me. And I must beg of you to grant me one favor, which is this - If
you hear me using the same words in my defence which I have been in
the habit of using, and which most of you may have heard in the agora,
and at the tables of the money-changers, or anywhere else, I would
ask you not to be surprised at this, and not to interrupt me. For
I am more than seventy years of age, and this is the first time that
I have ever appeared in a court of law, and I am quite a stranger
to the ways of the place; and therefore I would have you regard me
as if I were really a stranger, whom you would excuse if he spoke
in his native tongue, and after the fashion of his country; - that
I think is not an unfair request. Never mind the manner, which may
or may not be good; but think only of the justice of my cause, and
give heed to that: let the judge decide justly and the speaker speak
truly. 

And first, I have to reply to the older charges and to my first accusers,
and then I will go to the later ones. For I have had many accusers,
who accused me of old, and their false charges have continued during
many years; and I am more afraid of them than of Anytus and his associates,
who are dangerous, too, in their own way. But far more dangerous are
these, who began when you were children, and took possession of your
minds with their falsehoods, telling of one Socrates, a wise man,
who speculated about the heaven above, and searched into the earth
beneath, and made the worse appear the better cause. These are the
accusers whom I dread; for they are the circulators of this rumor,
and their hearers are too apt to fancy that speculators of this sort
do not believe in the gods. And they are many, and their charges against
me are of ancient date, and they made them in days when you were impressible
- in childhood, or perhaps in youth - and the cause when heard went
by default, for there was none to answer. And, hardest of all, their
names I do not know and cannot tell; unless in the chance of a comic
poet. But the main body of these slanderers who from envy and malice
have wrought upon you - and there are some of them who are convinced
themselves, and impart their convictions to others - all these, I
say, are most difficult to deal with; for I cannot have them up here,
and examine them, and therefore I must simply fight with shadows in
my own defence, and examine when there is no one who answers. I will
ask you then to assume with me, as I was saying, that my opponents
are of two kinds - one recent, the other ancient; and I hope that
you will see the propriety of my answering the latter first, for these
accusations you heard long before the others, and much oftener.

Well, then, I will make my defence, and I will endeavor in the short
time which is allowed to do away with this evil opinion of me which
you have held for such a long time; and I hope I may succeed, if this
be well for you and me, and that my words may find favor with you.
But I know that to accomplish this is not easy - I quite see the nature
of the task. Let the event be as God wills: in obedience to the law
I make my defence. 

I will begin at the beginning, and ask what the accusation is which
has given rise to this slander of me, and which has encouraged Meletus
to proceed against me. What do the slanderers say? They shall be my
prosecutors, and I will sum up their words in an affidavit. "Socrates
is an evil-doer, and a curious person, who searches into things under
the earth and in heaven, and he makes the worse appear the better
cause; and he teaches the aforesaid doctrines to others." That is
the nature of the accusation, and that is what you have seen yourselves
in the comedy of Aristophanes; who has introduced a man whom he calls
Socrates, going about and saying that he can walk in the air, and
talking a deal of nonsense concerning matters of which I do not pretend
to know either much or little - not that I mean to say anything disparaging
of anyone who is a student of natural philosophy. I should be very
sorry if Meletus could lay that to my charge. But the simple truth
is, O Athenians, that I have nothing to do with these studies. Very
many of those here present are witnesses to the truth of this, and
to them I appeal. Speak then, you who have heard me, and tell your
neighbors whether any of you have ever known me hold forth in few
words or in many upon matters of this sort. ... You hear their answer.
And from what they say of this you will be able to judge of the truth
of the rest. 

As little foundation is there for the report that I am a teacher,
and take money; that is no more true than the other. Although, if
a man is able to teach, I honor him for being paid. There is Gorgias
of Leontium, and Prodicus of Ceos, and Hippias of Elis, who go the
round of the cities, and are able to persuade the young men to leave
their own citizens, by whom they might be taught for nothing, and
come to them, whom they not only pay, but are thankful if they may
be allowed to pay them. There is actually a Parian philosopher residing
in Athens, of whom I have heard; and I came to hear of him in this
way: - I met a man who has spent a world of money on the Sophists,
Callias the son of Hipponicus, and knowing that he had sons, I asked
him: "Callias," I said, "if your two sons were foals or calves, there
would be no difficulty in finding someone to put over them; we should
hire a trainer of horses or a farmer probably who would improve and
perfect them in their own proper virtue and excellence; but as they
are human beings, whom are you thinking of placing over them? Is there
anyone who understands human and political virtue? You must have thought
about this as you have sons; is there anyone?" "There is," he said.
"Who is he?" said I, "and of what country? and what does he charge?"
"Evenus the Parian," he replied; "he is the man, and his charge is
five minae." Happy is Evenus, I said to myself, if he really has this
wisdom, and teaches at such a modest charge. Had I the same, I should
have been very proud and conceited; but the truth is that I have no
knowledge of the kind. 

I dare say, Athenians, that someone among you will reply, "Why is
this, Socrates, and what is the origin of these accusations of you:
for there must have been something strange which you have been doing?
All this great fame and talk about you would never have arisen if
you had been like other men: tell us, then, why this is, as we should
be sorry to judge hastily of you." Now I regard this as a fair challenge,
and I will endeavor to explain to you the origin of this name of "wise,"
and of this evil fame. Please to attend then. And although some of
you may think I am joking, I declare that I will tell you the entire
truth. Men of Athens, this reputation of mine has come of a certain
sort of wisdom which I possess. If you ask me what kind of wisdom,
I reply, such wisdom as is attainable by man, for to that extent I
am inclined to believe that I am wise; whereas the persons of whom
I was speaking have a superhuman wisdom, which I may fail to describe,
because I have it not myself; and he who says that I have, speaks
falsely, and is taking away my character. And here, O men of Athens,
I must beg you not to interrupt me, even if I seem to say something
extravagant. For the word which I will speak is not mine. I will refer
you to a witness who is worthy of credit, and will tell you about
my wisdom - whether I have any, and of what sort - and that witness
shall be the god of Delphi. You must have known Chaerephon; he was
early a friend of mine, and also a friend of yours, for he shared
in the exile of the people, and returned with you. Well, Chaerephon,
as you know, was very impetuous in all his doings, and he went to
Delphi and boldly asked the oracle to tell him whether - as I was
saying, I must beg you not to interrupt - he asked the oracle to tell
him whether there was anyone wiser than I was, and the Pythian prophetess
answered that there was no man wiser. Chaerephon is dead himself,
but his brother, who is in court, will confirm the truth of this story.

Why do I mention this? Because I am going to explain to you why I
have such an evil name. When I heard the answer, I said to myself,
What can the god mean? and what is the interpretation of this riddle?
for I know that I have no wisdom, small or great. What can he mean
when he says that I am the wisest of men? And yet he is a god and
cannot lie; that would be against his nature. After a long consideration,
I at last thought of a method of trying the question. I reflected
that if I could only find a man wiser than myself, then I might go
to the god with a refutation in my hand. I should say to him, "Here
is a man who is wiser than I am; but you said that I was the wisest."
Accordingly I went to one who had the reputation of wisdom, and observed
to him - his name I need not mention; he was a politician whom I selected
for examination - and the result was as follows: When I began to talk
with him, I could not help thinking that he was not really wise, although
he was thought wise by many, and wiser still by himself; and I went
and tried to explain to him that he thought himself wise, but was
not really wise; and the consequence was that he hated me, and his
enmity was shared by several who were present and heard me. So I left
him, saying to myself, as I went away: Well, although I do not suppose
that either of us knows anything really beautiful and good, I am better
off than he is - for he knows nothing, and thinks that he knows. I
neither know nor think that I know. In this latter particular, then,
I seem to have slightly the advantage of him. Then I went to another,
who had still higher philosophical pretensions, and my conclusion
was exactly the same. I made another enemy of him, and of many others
besides him. 

After this I went to one man after another, being not unconscious
of the enmity which I provoked, and I lamented and feared this: but
necessity was laid upon me - the word of God, I thought, ought to
be considered first. And I said to myself, Go I must to all who appear
to know, and find out the meaning of the oracle. And I swear to you,
Athenians, by the dog I swear! - for I must tell you the truth - the
result of my mission was just this: I found that the men most in repute
were all but the most foolish; and that some inferior men were really
wiser and better. I will tell you the tale of my wanderings and of
the "Herculean" labors, as I may call them, which I endured only to
find at last the oracle irrefutable. When I left the politicians,
I went to the poets; tragic, dithyrambic, and all sorts. And there,
I said to myself, you will be detected; now you will find out that
you are more ignorant than they are. Accordingly, I took them some
of the most elaborate passages in their own writings, and asked what
was the meaning of them - thinking that they would teach me something.
Will you believe me? I am almost ashamed to speak of this, but still
I must say that there is hardly a person present who would not have
talked better about their poetry than they did themselves. That showed
me in an instant that not by wisdom do poets write poetry, but by
a sort of genius and inspiration; they are like diviners or soothsayers
who also say many fine things, but do not understand the meaning of
them. And the poets appeared to me to be much in the same case; and
I further observed that upon the strength of their poetry they believed
themselves to be the wisest of men in other things in which they were
not wise. So I departed, conceiving myself to be superior to them
for the same reason that I was superior to the politicians.

At last I went to the artisans, for I was conscious that I knew nothing
at all, as I may say, and I was sure that they knew many fine things;
and in this I was not mistaken, for they did know many things of which
I was ignorant, and in this they certainly were wiser than I was.
But I observed that even the good artisans fell into the same error
as the poets; because they were good workmen they thought that they
also knew all sorts of high matters, and this defect in them overshadowed
their wisdom - therefore I asked myself on behalf of the oracle, whether
I would like to be as I was, neither having their knowledge nor their
ignorance, or like them in both; and I made answer to myself and the
oracle that I was better off as I was. 

This investigation has led to my having many enemies of the worst
and most dangerous kind, and has given occasion also to many calumnies,
and I am called wise, for my hearers always imagine that I myself
possess the wisdom which I find wanting in others: but the truth is,
O men of Athens, that God only is wise; and in this oracle he means
to say that the wisdom of men is little or nothing; he is not speaking
of Socrates, he is only using my name as an illustration, as if he
said, He, O men, is the wisest, who, like Socrates, knows that his
wisdom is in truth worth nothing. And so I go my way, obedient to
the god, and make inquisition into the wisdom of anyone, whether citizen
or stranger, who appears to be wise; and if he is not wise, then in
vindication of the oracle I show him that he is not wise; and this
occupation quite absorbs me, and I have no time to give either to
any public matter of interest or to any concern of my own, but I am
in utter poverty by reason of my devotion to the god. 

There is another thing: - young men of the richer classes, who have
not much to do, come about me of their own accord; they like to hear
the pretenders examined, and they often imitate me, and examine others
themselves; there are plenty of persons, as they soon enough discover,
who think that they know something, but really know little or nothing:
and then those who are examined by them instead of being angry with
themselves are angry with me: This confounded Socrates, they say;
this villainous misleader of youth! - and then if somebody asks them,
Why, what evil does he practise or teach? they do not know, and cannot
tell; but in order that they may not appear to be at a loss, they
repeat the ready-made charges which are used against all philosophers
about teaching things up in the clouds and under the earth, and having
no gods, and making the worse appear the better cause; for they do
not like to confess that their pretence of knowledge has been detected
- which is the truth: and as they are numerous and ambitious and energetic,
and are all in battle array and have persuasive tongues, they have
filled your ears with their loud and inveterate calumnies. And this
is the reason why my three accusers, Meletus and Anytus and Lycon,
have set upon me; Meletus, who has a quarrel with me on behalf of
the poets; Anytus, on behalf of the craftsmen; Lycon, on behalf of
the rhetoricians: and as I said at the beginning, I cannot expect
to get rid of this mass of calumny all in a moment. And this, O men
of Athens, is the truth and the whole truth; I have concealed nothing,
I have dissembled nothing. And yet I know that this plainness of speech
makes them hate me, and what is their hatred but a proof that I am
speaking the truth? - this is the occasion and reason of their slander
of me, as you will find out either in this or in any future inquiry.

I have said enough in my defence against the first class of my accusers;
I turn to the second class, who are headed by Meletus, that good and
patriotic man, as he calls himself. And now I will try to defend myself
against them: these new accusers must also have their affidavit read.
What do they say? Something of this sort: - That Socrates is a doer
of evil, and corrupter of the youth, and he does not believe in the
gods of the state, and has other new divinities of his own. That is
the sort of charge; and now let us examine the particular counts.
He says that I am a doer of evil, who corrupt the youth; but I say,
O men of Athens, that Meletus is a doer of evil, and the evil is that
he makes a joke of a serious matter, and is too ready at bringing
other men to trial from a pretended zeal and interest about matters
in which he really never had the smallest interest. And the truth
of this I will endeavor to prove. 

Come hither, Meletus, and let me ask a question of you. You think
a great deal about the improvement of youth? 

Yes, I do. 

Tell the judges, then, who is their improver; for you must know, as
you have taken the pains to discover their corrupter, and are citing
and accusing me before them. Speak, then, and tell the judges who
their improver is. Observe, Meletus, that you are silent, and have
nothing to say. But is not this rather disgraceful, and a very considerable
proof of what I was saying, that you have no interest in the matter?
Speak up, friend, and tell us who their improver is. 

The laws. 

But that, my good sir, is not my meaning. I want to know who the person
is, who, in the first place, knows the laws. 

The judges, Socrates, who are present in court. 

What do you mean to say, Meletus, that they are able to instruct and
improve youth? 

Certainly they are. 

What, all of them, or some only and not others? 

All of them. 

By the goddess Here, that is good news! There are plenty of improvers,
then. And what do you say of the audience, - do they improve them?

Yes, they do. 

And the senators? 

Yes, the senators improve them. 

But perhaps the members of the citizen assembly corrupt them? - or
do they too improve them? 

They improve them. 

Then every Athenian improves and elevates them; all with the exception
of myself; and I alone am their corrupter? Is that what you affirm?

That is what I stoutly affirm. 

I am very unfortunate if that is true. But suppose I ask you a question:
Would you say that this also holds true in the case of horses? Does
one man do them harm and all the world good? Is not the exact opposite
of this true? One man is able to do them good, or at least not many;
- the trainer of horses, that is to say, does them good, and others
who have to do with them rather injure them? Is not that true, Meletus,
of horses, or any other animals? Yes, certainly. Whether you and Anytus
say yes or no, that is no matter. Happy indeed would be the condition
of youth if they had one corrupter only, and all the rest of the world
were their improvers. And you, Meletus, have sufficiently shown that
you never had a thought about the young: your carelessness is seen
in your not caring about matters spoken of in this very indictment.

And now, Meletus, I must ask you another question: Which is better,
to live among bad citizens, or among good ones? Answer, friend, I
say; for that is a question which may be easily answered. Do not the
good do their neighbors good, and the bad do them evil? 

Certainly. 

And is there anyone who would rather be injured than benefited by
those who live with him? Answer, my good friend; the law requires
you to answer - does anyone like to be injured? 

Certainly not. 

And when you accuse me of corrupting and deteriorating the youth,
do you allege that I corrupt them intentionally or unintentionally?

Intentionally, I say. 

But you have just admitted that the good do their neighbors good,
and the evil do them evil. Now is that a truth which your superior
wisdom has recognized thus early in life, and am I, at my age, in
such darkness and ignorance as not to know that if a man with whom
I have to live is corrupted by me, I am very likely to be harmed by
him, and yet I corrupt him, and intentionally, too; - that is what
you are saying, and of that you will never persuade me or any other
human being. But either I do not corrupt them, or I corrupt them unintentionally,
so that on either view of the case you lie. If my offence is unintentional,
the law has no cognizance of unintentional offences: you ought to
have taken me privately, and warned and admonished me; for if I had
been better advised, I should have left off doing what I only did
unintentionally - no doubt I should; whereas you hated to converse
with me or teach me, but you indicted me in this court, which is a
place not of instruction, but of punishment. 

I have shown, Athenians, as I was saying, that Meletus has no care
at all, great or small, about the matter. But still I should like
to know, Meletus, in what I am affirmed to corrupt the young. I suppose
you mean, as I infer from your indictment, that I teach them not to
acknowledge the gods which the state acknowledges, but some other
new divinities or spiritual agencies in their stead. These are the
lessons which corrupt the youth, as you say. 

Yes, that I say emphatically. 

Then, by the gods, Meletus, of whom we are speaking, tell me and the
court, in somewhat plainer terms, what you mean! for I do not as yet
understand whether you affirm that I teach others to acknowledge some
gods, and therefore do believe in gods and am not an entire atheist
- this you do not lay to my charge; but only that they are not the
same gods which the city recognizes - the charge is that they are
different gods. Or, do you mean to say that I am an atheist simply,
and a teacher of atheism? 

I mean the latter - that you are a complete atheist. 

That is an extraordinary statement, Meletus. Why do you say that?
Do you mean that I do not believe in the godhead of the sun or moon,
which is the common creed of all men? 

I assure you, judges, that he does not believe in them; for he says
that the sun is stone, and the moon earth. 

Friend Meletus, you think that you are accusing Anaxagoras; and you
have but a bad opinion of the judges, if you fancy them ignorant to
such a degree as not to know that those doctrines are found in the
books of Anaxagoras the Clazomenian, who is full of them. And these
are the doctrines which the youth are said to learn of Socrates, when
there are not unfrequently exhibitions of them at the theatre (price
of admission one drachma at the most); and they might cheaply purchase
them, and laugh at Socrates if he pretends to father such eccentricities.
And so, Meletus, you really think that I do not believe in any god?

I swear by Zeus that you believe absolutely in none at all.

You are a liar, Meletus, not believed even by yourself. For I cannot
help thinking, O men of Athens, that Meletus is reckless and impudent,
and that he has written this indictment in a spirit of mere wantonness
and youthful bravado. Has he not compounded a riddle, thinking to
try me? He said to himself: - I shall see whether this wise Socrates
will discover my ingenious contradiction, or whether I shall be able
to deceive him and the rest of them. For he certainly does appear
to me to contradict himself in the indictment as much as if he said
that Socrates is guilty of not believing in the gods, and yet of believing
in them - but this surely is a piece of fun. 

I should like you, O men of Athens, to join me in examining what I
conceive to be his inconsistency; and do you, Meletus, answer. And
I must remind you that you are not to interrupt me if I speak in my
accustomed manner. 

Did ever man, Meletus, believe in the existence of human things, and
not of human beings? ... I wish, men of Athens, that he would answer,
and not be always trying to get up an interruption. Did ever any man
believe in horsemanship, and not in horses? or in flute-playing, and
not in flute-players? No, my friend; I will answer to you and to the
court, as you refuse to answer for yourself. There is no man who ever
did. But now please to answer the next question: Can a man believe
in spiritual and divine agencies, and not in spirits or demigods?

He cannot. 

I am glad that I have extracted that answer, by the assistance of
the court; nevertheless you swear in the indictment that I teach and
believe in divine or spiritual agencies (new or old, no matter for
that); at any rate, I believe in spiritual agencies, as you say and
swear in the affidavit; but if I believe in divine beings, I must
believe in spirits or demigods; - is not that true? Yes, that is true,
for I may assume that your silence gives assent to that. Now what
are spirits or demigods? are they not either gods or the sons of gods?
Is that true? 

Yes, that is true. 

But this is just the ingenious riddle of which I was speaking: the
demigods or spirits are gods, and you say first that I don't believe
in gods, and then again that I do believe in gods; that is, if I believe
in demigods. For if the demigods are the illegitimate sons of gods,
whether by the Nymphs or by any other mothers, as is thought, that,
as all men will allow, necessarily implies the existence of their
parents. You might as well affirm the existence of mules, and deny
that of horses and asses. Such nonsense, Meletus, could only have
been intended by you as a trial of me. You have put this into the
indictment because you had nothing real of which to accuse me. But
no one who has a particle of understanding will ever be convinced
by you that the same man can believe in divine and superhuman things,
and yet not believe that there are gods and demigods and heroes.

I have said enough in answer to the charge of Meletus: any elaborate
defence is unnecessary; but as I was saying before, I certainly have
many enemies, and this is what will be my destruction if I am destroyed;
of that I am certain; - not Meletus, nor yet Anytus, but the envy
and detraction of the world, which has been the death of many good
men, and will probably be the death of many more; there is no danger
of my being the last of them. 

Someone will say: And are you not ashamed, Socrates, of a course of
life which is likely to bring you to an untimely end? To him I may
fairly answer: There you are mistaken: a man who is good for anything
ought not to calculate the chance of living or dying; he ought only
to consider whether in doing anything he is doing right or wrong -
acting the part of a good man or of a bad. Whereas, according to your
view, the heroes who fell at Troy were not good for much, and the
son of Thetis above all, who altogether despised danger in comparison
with disgrace; and when his goddess mother said to him, in his eagerness
to slay Hector, that if he avenged his companion Patroclus, and slew
Hector, he would die himself - "Fate," as she said, "waits upon you
next after Hector"; he, hearing this, utterly despised danger and
death, and instead of fearing them, feared rather to live in dishonor,
and not to avenge his friend. "Let me die next," he replies, "and
be avenged of my enemy, rather than abide here by the beaked ships,
a scorn and a burden of the earth." Had Achilles any thought of death
and danger? For wherever a man's place is, whether the place which
he has chosen or that in which he has been placed by a commander,
there he ought to remain in the hour of danger; he should not think
of death or of anything, but of disgrace. And this, O men of Athens,
is a true saying. 

Strange, indeed, would be my conduct, O men of Athens, if I who, when
I was ordered by the generals whom you chose to command me at Potidaea
and Amphipolis and Delium, remained where they placed me, like any
other man, facing death; if, I say, now, when, as I conceive and imagine,
God orders me to fulfil the philosopher's mission of searching into
myself and other men, I were to desert my post through fear of death,
or any other fear; that would indeed be strange, and I might justly
be arraigned in court for denying the existence of the gods, if I
disobeyed the oracle because I was afraid of death: then I should
be fancying that I was wise when I was not wise. For this fear of
death is indeed the pretence of wisdom, and not real wisdom, being
the appearance of knowing the unknown; since no one knows whether
death, which they in their fear apprehend to be the greatest evil,
may not be the greatest good. Is there not here conceit of knowledge,
which is a disgraceful sort of ignorance? And this is the point in
which, as I think, I am superior to men in general, and in which I
might perhaps fancy myself wiser than other men, - that whereas I
know but little of the world below, I do not suppose that I know:
but I do know that injustice and disobedience to a better, whether
God or man, is evil and dishonorable, and I will never fear or avoid
a possible good rather than a certain evil. And therefore if you let
me go now, and reject the counsels of Anytus, who said that if I were
not put to death I ought not to have been prosecuted, and that if
I escape now, your sons will all be utterly ruined by listening to
my words - if you say to me, Socrates, this time we will not mind
Anytus, and will let you off, but upon one condition, that are to
inquire and speculate in this way any more, and that if you are caught
doing this again you shall die; - if this was the condition on which
you let me go, I should reply: Men of Athens, I honor and love you;
but I shall obey God rather than you, and while I have life and strength
I shall never cease from the practice and teaching of philosophy,
exhorting anyone whom I meet after my manner, and convincing him,
saying: O my friend, why do you who are a citizen of the great and
mighty and wise city of Athens, care so much about laying up the greatest
amount of money and honor and reputation, and so little about wisdom
and truth and the greatest improvement of the soul, which you never
regard or heed at all? Are you not ashamed of this? And if the person
with whom I am arguing says: Yes, but I do care; I do not depart or
let him go at once; I interrogate and examine and cross-examine him,
and if I think that he has no virtue, but only says that he has, I
reproach him with undervaluing the greater, and overvaluing the less.
And this I should say to everyone whom I meet, young and old, citizen
and alien, but especially to the citizens, inasmuch as they are my
brethren. For this is the command of God, as I would have you know;
and I believe that to this day no greater good has ever happened in
the state than my service to the God. For I do nothing but go about
persuading you all, old and young alike, not to take thought for your
persons and your properties, but first and chiefly to care about the
greatest improvement of the soul. I tell you that virtue is not given
by money, but that from virtue come money and every other good of
man, public as well as private. This is my teaching, and if this is
the doctrine which corrupts the youth, my influence is ruinous indeed.
But if anyone says that this is not my teaching, he is speaking an
untruth. Wherefore, O men of Athens, I say to you, do as Anytus bids
or not as Anytus bids, and either acquit me or not; but whatever you
do, know that I shall never alter my ways, not even if I have to die
many times. 

Men of Athens, do not interrupt, but hear me; there was an agreement
between us that you should hear me out. And I think that what I am
going to say will do you good: for I have something more to say, at
which you may be inclined to cry out; but I beg that you will not
do this. I would have you know that, if you kill such a one as I am,
you will injure yourselves more than you will injure me. Meletus and
Anytus will not injure me: they cannot; for it is not in the nature
of things that a bad man should injure a better than himself. I do
not deny that he may, perhaps, kill him, or drive him into exile,
or deprive him of civil rights; and he may imagine, and others may
imagine, that he is doing him a great injury: but in that I do not
agree with him; for the evil of doing as Anytus is doing - of unjustly
taking away another man's life - is greater far. And now, Athenians,
I am not going to argue for my own sake, as you may think, but for
yours, that you may not sin against the God, or lightly reject his
boon by condemning me. For if you kill me you will not easily find
another like me, who, if I may use such a ludicrous figure of speech,
am a sort of gadfly, given to the state by the God; and the state
is like a great and noble steed who is tardy in his motions owing
to his very size, and requires to be stirred into life. I am that
gadfly which God has given the state and all day long and in all places
am always fastening upon you, arousing and persuading and reproaching
you. And as you will not easily find another like me, I would advise
you to spare me. I dare say that you may feel irritated at being suddenly
awakened when you are caught napping; and you may think that if you
were to strike me dead, as Anytus advises, which you easily might,
then you would sleep on for the remainder of your lives, unless God
in his care of you gives you another gadfly. And that I am given to
you by God is proved by this: - that if I had been like other men,
I should not have neglected all my own concerns, or patiently seen
the neglect of them during all these years, and have been doing yours,
coming to you individually, like a father or elder brother, exhorting
you to regard virtue; this I say, would not be like human nature.
And had I gained anything, or if my exhortations had been paid, there
would have been some sense in that: but now, as you will perceive,
not even the impudence of my accusers dares to say that I have ever
exacted or sought pay of anyone; they have no witness of that. And
I have a witness of the truth of what I say; my poverty is a sufficient
witness. 

Someone may wonder why I go about in private, giving advice and busying
myself with the concerns of others, but do not venture to come forward
in public and advise the state. I will tell you the reason of this.
You have often heard me speak of an oracle or sign which comes to
me, and is the divinity which Meletus ridicules in the indictment.
This sign I have had ever since I was a child. The sign is a voice
which comes to me and always forbids me to do something which I am
going to do, but never commands me to do anything, and this is what
stands in the way of my being a politician. And rightly, as I think.
For I am certain, O men of Athens, that if I had engaged in politics,
I should have perished long ago and done no good either to you or
to myself. And don't be offended at my telling you the truth: for
the truth is that no man who goes to war with you or any other multitude,
honestly struggling against the commission of unrighteousness and
wrong in the state, will save his life; he who will really fight for
the right, if he would live even for a little while, must have a private
station and not a public one. 

I can give you as proofs of this, not words only, but deeds, which
you value more than words. Let me tell you a passage of my own life,
which will prove to you that I should never have yielded to injustice
from any fear of death, and that if I had not yielded I should have
died at once. I will tell you a story - tasteless, perhaps, and commonplace,
but nevertheless true. The only office of state which I ever held,
O men of Athens, was that of senator; the tribe Antiochis, which is
my tribe, had the presidency at the trial of the generals who had
not taken up the bodies of the slain after the battle of Arginusae;
and you proposed to try them all together, which was illegal, as you
all thought afterwards; but at the time I was the only one of the
Prytanes who was opposed to the illegality, and I gave my vote against
you; and when the orators threatened to impeach and arrest me, and
have me taken away, and you called and shouted, I made up my mind
that I would run the risk, having law and justice with me, rather
than take part in your injustice because I feared imprisonment and
death. This happened in the days of the democracy. But when the oligarchy
of the Thirty was in power, they sent for me and four others into
the rotunda, and bade us bring Leon the Salaminian from Salamis, as
they wanted to execute him. This was a specimen of the sort of commands
which they were always giving with the view of implicating as many
as possible in their crimes; and then I showed, not in words only,
but in deed, that, if I may be allowed to use such an expression,
I cared not a straw for death, and that my only fear was the fear
of doing an unrighteous or unholy thing. For the strong arm of that
oppressive power did not frighten me into doing wrong; and when we
came out of the rotunda the other four went to Salamis and fetched
Leon, but I went quietly home. For which I might have lost my life,
had not the power of the Thirty shortly afterwards come to an end.
And to this many will witness. 

Now do you really imagine that I could have survived all these years,
if I had led a public life, supposing that like a good man I had always
supported the right and had made justice, as I ought, the first thing?
No, indeed, men of Athens, neither I nor any other. But I have been
always the same in all my actions, public as well as private, and
never have I yielded any base compliance to those who are slanderously
termed my disciples or to any other. For the truth is that I have
no regular disciples: but if anyone likes to come and hear me while
I am pursuing my mission, whether he be young or old, he may freely
come. Nor do I converse with those who pay only, and not with those
who do not pay; but anyone, whether he be rich or poor, may ask and
answer me and listen to my words; and whether he turns out to be a
bad man or a good one, that cannot be justly laid to my charge, as
I never taught him anything. And if anyone says that he has ever learned
or heard anything from me in private which all the world has not heard,
I should like you to know that he is speaking an untruth.

But I shall be asked, Why do people delight in continually conversing
with you? I have told you already, Athenians, the whole truth about
this: they like to hear the cross-examination of the pretenders to
wisdom; there is amusement in this. And this is a duty which the God
has imposed upon me, as I am assured by oracles, visions, and in every
sort of way in which the will of divine power was ever signified to
anyone. This is true, O Athenians; or, if not true, would be soon
refuted. For if I am really corrupting the youth, and have corrupted
some of them already, those of them who have grown up and have become
sensible that I gave them bad advice in the days of their youth should
come forward as accusers and take their revenge; and if they do not
like to come themselves, some of their relatives, fathers, brothers,
or other kinsmen, should say what evil their families suffered at
my hands. Now is their time. Many of them I see in the court. There
is Crito, who is of the same age and of the same deme with myself;
and there is Critobulus his son, whom I also see. Then again there
is Lysanias of Sphettus, who is the father of Aeschines - he is present;
and also there is Antiphon of Cephisus, who is the father of Epignes;
and there are the brothers of several who have associated with me.
There is Nicostratus the son of Theosdotides, and the brother of Theodotus
(now Theodotus himself is dead, and therefore he, at any rate, will
not seek to stop him); and there is Paralus the son of Demodocus,
who had a brother Theages; and Adeimantus the son of Ariston, whose
brother Plato is present; and Aeantodorus, who is the brother of Apollodorus,
whom I also see. I might mention a great many others, any of whom
Meletus should have produced as witnesses in the course of his speech;
and let him still produce them, if he has forgotten - I will make
way for him. And let him say, if he has any testimony of the sort
which he can produce. Nay, Athenians, the very opposite is the truth.
For all these are ready to witness on behalf of the corrupter, of
the destroyer of their kindred, as Meletus and Anytus call me; not
the corrupted youth only - there might have been a motive for that
- but their uncorrupted elder relatives. Why should they too support
me with their testimony? Why, indeed, except for the sake of truth
and justice, and because they know that I am speaking the truth, and
that Meletus is lying. 

Well, Athenians, this and the like of this is nearly all the defence
which I have to offer. Yet a word more. Perhaps there may be someone
who is offended at me, when he calls to mind how he himself, on a
similar or even a less serious occasion, had recourse to prayers and
supplications with many tears, and how he produced his children in
court, which was a moving spectacle, together with a posse of his
relations and friends; whereas I, who am probably in danger of my
life, will do none of these things. Perhaps this may come into his
mind, and he may be set against me, and vote in anger because he is
displeased at this. Now if there be such a person among you, which
I am far from affirming, I may fairly reply to him: My friend, I am
a man, and like other men, a creature of flesh and blood, and not
of wood or stone, as Homer says; and I have a family, yes, and sons.
O Athenians, three in number, one of whom is growing up, and the two
others are still young; and yet I will not bring any of them hither
in order to petition you for an acquittal. And why not? Not from any
self-will or disregard of you. Whether I am or am not afraid of death
is another question, of which I will not now speak. But my reason
simply is that I feel such conduct to be discreditable to myself,
and you, and the whole state. One who has reached my years, and who
has a name for wisdom, whether deserved or not, ought not to debase
himself. At any rate, the world has decided that Socrates is in some
way superior to other men. And if those among you who are said to
be superior in wisdom and courage, and any other virtue, demean themselves
in this way, how shameful is their conduct! I have seen men of reputation,
when they have been condemned, behaving in the strangest manner: they
seemed to fancy that they were going to suffer something dreadful
if they died, and that they could be immortal if you only allowed
them to live; and I think that they were a dishonor to the state,
and that any stranger coming in would say of them that the most eminent
men of Athens, to whom the Athenians themselves give honor and command,
are no better than women. And I say that these things ought not to
be done by those of us who are of reputation; and if they are done,
you ought not to permit them; you ought rather to show that you are
more inclined to condemn, not the man who is quiet, but the man who
gets up a doleful scene, and makes the city ridiculous. 

But, setting aside the question of dishonor, there seems to be something
wrong in petitioning a judge, and thus procuring an acquittal instead
of informing and convincing him. For his duty is, not to make a present
of justice, but to give judgment; and he has sworn that he will judge
according to the laws, and not according to his own good pleasure;
and neither he nor we should get into the habit of perjuring ourselves
- there can be no piety in that. Do not then require me to do what
I consider dishonorable and impious and wrong, especially now, when
I am being tried for impiety on the indictment of Meletus. For if,
O men of Athens, by force of persuasion and entreaty, I could overpower
your oaths, then I should be teaching you to believe that there are
no gods, and convict myself, in my own defence, of not believing in
them. But that is not the case; for I do believe that there are gods,
and in a far higher sense than that in which any of my accusers believe
in them. And to you and to God I commit my cause, to be determined
by you as is best for you and me. 

(The jury finds Socrates guilty.)

Socrates' Proposal for his Sentence

There are many reasons why I am not grieved, O men of Athens, at the
vote of condemnation. I expected it, and am only surprised that the
votes are so nearly equal; for I had thought that the majority against
me would have been far larger; but now, had thirty votes gone over
to the other side, I should have been acquitted. And I may say that
I have escaped Meletus. And I may say more; for without the assistance
of Anytus and Lycon, he would not have had a fifth part of the votes,
as the law requires, in which case he would have incurred a fine of
a thousand drachmae, as is evident. 

And so he proposes death as the penalty. And what shall I propose
on my part, O men of Athens? Clearly that which is my due. And what
is that which I ought to pay or to receive? What shall be done to
the man who has never had the wit to be idle during his whole life;
but has been careless of what the many care about - wealth, and family
interests, and military offices, and speaking in the assembly, and
magistracies, and plots, and parties. Reflecting that I was really
too honest a man to follow in this way and live, I did not go where
I could do no good to you or to myself; but where I could do the greatest
good privately to everyone of you, thither I went, and sought to persuade
every man among you that he must look to himself, and seek virtue
and wisdom before he looks to his private interests, and look to the
state before he looks to the interests of the state; and that this
should be the order which he observes in all his actions. What shall
be done to such a one? Doubtless some good thing, O men of Athens,
if he has his reward; and the good should be of a kind suitable to
him. What would be a reward suitable to a poor man who is your benefactor,
who desires leisure that he may instruct you? There can be no more
fitting reward than maintenance in the Prytaneum, O men of Athens,
a reward which he deserves far more than the citizen who has won the
prize at Olympia in the horse or chariot race, whether the chariots
were drawn by two horses or by many. For I am in want, and he has
enough; and he only gives you the appearance of happiness, and I give
you the reality. And if I am to estimate the penalty justly, I say
that maintenance in the Prytaneum is the just return. 

Perhaps you may think that I am braving you in saying this, as in
what I said before about the tears and prayers. But that is not the
case. I speak rather because I am convinced that I never intentionally
wronged anyone, although I cannot convince you of that - for we have
had a short conversation only; but if there were a law at Athens,
such as there is in other cities, that a capital cause should not
be decided in one day, then I believe that I should have convinced
you; but now the time is too short. I cannot in a moment refute great
slanders; and, as I am convinced that I never wronged another, I will
assuredly not wrong myself. I will not say of myself that I deserve
any evil, or propose any penalty. Why should I? Because I am afraid
of the penalty of death which Meletus proposes? When I do not know
whether death is a good or an evil, why should I propose a penalty
which would certainly be an evil? Shall I say imprisonment? And why
should I live in prison, and be the slave of the magistrates of the
year - of the Eleven? Or shall the penalty be a fine, and imprisonment
until the fine is paid? There is the same objection. I should have
to lie in prison, for money I have none, and I cannot pay. And if
I say exile (and this may possibly be the penalty which you will affix),
I must indeed be blinded by the love of life if I were to consider
that when you, who are my own citizens, cannot endure my discourses
and words, and have found them so grievous and odious that you would
fain have done with them, others are likely to endure me. No, indeed,
men of Athens, that is not very likely. And what a life should I lead,
at my age, wandering from city to city, living in ever-changing exile,
and always being driven out! For I am quite sure that into whatever
place I go, as here so also there, the young men will come to me;
and if I drive them away, their elders will drive me out at their
desire: and if I let them come, their fathers and friends will drive
me out for their sakes. 

Someone will say: Yes, Socrates, but cannot you hold your tongue,
and then you may go into a foreign city, and no one will interfere
with you? Now I have great difficulty in making you understand my
answer to this. For if I tell you that this would be a disobedience
to a divine command, and therefore that I cannot hold my tongue, you
will not believe that I am serious; and if I say again that the greatest
good of man is daily to converse about virtue, and all that concerning
which you hear me examining myself and others, and that the life which
is unexamined is not worth living - that you are still less likely
to believe. And yet what I say is true, although a thing of which
it is hard for me to persuade you. Moreover, I am not accustomed to
think that I deserve any punishment. Had I money I might have proposed
to give you what I had, and have been none the worse. But you see
that I have none, and can only ask you to proportion the fine to my
means. However, I think that I could afford a minae, and therefore
I propose that penalty; Plato, Crito, Critobulus, and Apollodorus,
my friends here, bid me say thirty minae, and they will be the sureties.
Well then, say thirty minae, let that be the penalty; for that they
will be ample security to you. 

(The jury condemns Socrates to death.)

Socrates' Comments on his Sentence

Not much time will be gained, O Athenians, in return for the evil
name which you will get from the detractors of the city, who will
say that you killed Socrates, a wise man; for they will call me wise
even although I am not wise when they want to reproach you. If you
had waited a little while, your desire would have been fulfilled in
the course of nature. For I am far advanced in years, as you may perceive,
and not far from death. I am speaking now only to those of you who
have condemned me to death. And I have another thing to say to them:
You think that I was convicted through deficiency of words - I mean,
that if I had thought fit to leave nothing undone, nothing unsaid,
I might have gained an acquittal. Not so; the deficiency which led
to my conviction was not of words - certainly not. But I had not the
boldness or impudence or inclination to address you as you would have
liked me to address you, weeping and wailing and lamenting, and saying
and doing many things which you have been accustomed to hear from
others, and which, as I say, are unworthy of me. But I thought that
I ought not to do anything common or mean in the hour of danger: nor
do I now repent of the manner of my defence, and I would rather die
having spoken after my manner, than speak in your manner and live.
For neither in war nor yet at law ought any man to use every way of
escaping death. For often in battle there is no doubt that if a man
will throw away his arms, and fall on his knees before his pursuers,
he may escape death; and in other dangers there are other ways of
escaping death, if a man is willing to say and do anything. The difficulty,
my friends, is not in avoiding death, but in avoiding unrighteousness;
for that runs faster than death. I am old and move slowly, and the
slower runner has overtaken me, and my accusers are keen and quick,
and the faster runner, who is unrighteousness, has overtaken them.
And now I depart hence condemned by you to suffer the penalty of death,
and they, too, go their ways condemned by the truth to suffer the
penalty of villainy and wrong; and I must abide by my award - let
them abide by theirs. I suppose that these things may be regarded
as fated, - and I think that they are well. 

And now, O men who have condemned me, I would fain prophesy to you;
for I am about to die, and that is the hour in which men are gifted
with prophetic power. And I prophesy to you who are my murderers,
that immediately after my death punishment far heavier than you have
inflicted on me will surely await you. Me you have killed because
you wanted to escape the accuser, and not to give an account of your
lives. But that will not be as you suppose: far otherwise. For I say
that there will be more accusers of you than there are now; accusers
whom hitherto I have restrained: and as they are younger they will
be more severe with you, and you will be more offended at them. For
if you think that by killing men you can avoid the accuser censuring
your lives, you are mistaken; that is not a way of escape which is
either possible or honorable; the easiest and noblest way is not to
be crushing others, but to be improving yourselves. This is the prophecy
which I utter before my departure, to the judges who have condemned
me. 

Friends, who would have acquitted me, I would like also to talk with
you about this thing which has happened, while the magistrates are
busy, and before I go to the place at which I must die. Stay then
awhile, for we may as well talk with one another while there is time.
You are my friends, and I should like to show you the meaning of this
event which has happened to me. O my judges - for you I may truly
call judges - I should like to tell you of a wonderful circumstance.
Hitherto the familiar oracle within me has constantly been in the
habit of opposing me even about trifles, if I was going to make a
slip or error about anything; and now as you see there has come upon
me that which may be thought, and is generally believed to be, the
last and worst evil. But the oracle made no sign of opposition, either
as I was leaving my house and going out in the morning, or when I
was going up into this court, or while I was speaking, at anything
which I was going to say; and yet I have often been stopped in the
middle of a speech; but now in nothing I either said or did touching
this matter has the oracle opposed me. What do I take to be the explanation
of this? I will tell you. I regard this as a proof that what has happened
to me is a good, and that those of us who think that death is an evil
are in error. This is a great proof to me of what I am saying, for
the customary sign would surely have opposed me had I been going to
evil and not to good. 

Let us reflect in another way, and we shall see that there is great
reason to hope that death is a good, for one of two things: - either
death is a state of nothingness and utter unconsciousness, or, as
men say, there is a change and migration of the soul from this world
to another. Now if you suppose that there is no consciousness, but
a sleep like the sleep of him who is undisturbed even by the sight
of dreams, death will be an unspeakable gain. For if a person were
to select the night in which his sleep was undisturbed even by dreams,
and were to compare with this the other days and nights of his life,
and then were to tell us how many days and nights he had passed in
the course of his life better and more pleasantly than this one, I
think that any man, I will not say a private man, but even the great
king, will not find many such days or nights, when compared with the
others. Now if death is like this, I say that to die is gain; for
eternity is then only a single night. But if death is the journey
to another place, and there, as men say, all the dead are, what good,
O my friends and judges, can be greater than this? If indeed when
the pilgrim arrives in the world below, he is delivered from the professors
of justice in this world, and finds the true judges who are said to
give judgment there, Minos and Rhadamanthus and Aeacus and Triptolemus,
and other sons of God who were righteous in their own life, that pilgrimage
will be worth making. What would not a man give if he might converse
with Orpheus and Musaeus and Hesiod and Homer? Nay, if this be true,
let me die again and again. I, too, shall have a wonderful interest
in a place where I can converse with Palamedes, and Ajax the son of
Telamon, and other heroes of old, who have suffered death through
an unjust judgment; and there will be no small pleasure, as I think,
in comparing my own sufferings with theirs. Above all, I shall be
able to continue my search into true and false knowledge; as in this
world, so also in that; I shall find out who is wise, and who pretends
to be wise, and is not. What would not a man give, O judges, to be
able to examine the leader of the great Trojan expedition; or Odysseus
or Sisyphus, or numberless others, men and women too! What infinite
delight would there be in conversing with them and asking them questions!
For in that world they do not put a man to death for this; certainly
not. For besides being happier in that world than in this, they will
be immortal, if what is said is true. 

Wherefore, O judges, be of good cheer about death, and know this of
a truth - that no evil can happen to a good man, either in life or
after death. He and his are not neglected by the gods; nor has my
own approaching end happened by mere chance. But I see clearly that
to die and be released was better for me; and therefore the oracle
gave no sign. For which reason also, I am not angry with my accusers,
or my condemners; they have done me no harm, although neither of them
meant to do me any good; and for this I may gently blame them.

Still I have a favor to ask of them. When my sons are grown up, I
would ask you, O my friends, to punish them; and I would have you
trouble them, as I have troubled you, if they seem to care about riches,
or anything, more than about virtue; or if they pretend to be something
when they are really nothing, - then reprove them, as I have reproved
you, for not caring about that for which they ought to care, and thinking
that they are something when they are really nothing. And if you do
this, I and my sons will have received justice at your hands.

The hour of departure has arrived, and we go our ways - I to die,
and you to live. Which is better God only knows. 

THE END

% chapter apology (end)
% \chapter{Meno} % (fold)
\label{cha:meno}


Meno
By Plato


Translated by Benjamin Jowett

Persons of the Dialogue
MENO
SOCRATES
A SLAVE OF MENO
ANYTUS
----------------------------------------------------------------------

Meno. Can you tell me, Socrates, whether virtue is acquired by teaching
or by practice; or if neither by teaching nor practice, then whether
it comes to man by nature, or in what other way? 

Socrates. O Meno, there was a time when the Thessalians were famous
among the other Hellenes only for their riches and their riding; but
now, if I am not mistaken, they are equally famous for their wisdom,
especially at Larisa, which is the native city of your friend Aristippus.
And this is Gorgias' doing; for when he came there, the flower of
the Aleuadae, among them your admirer Aristippus, and the other chiefs
of the Thessalians, fell in love with his wisdom. And he has taught
you the habit of answering questions in a grand and bold style, which
becomes those who know, and is the style in which he himself answers
all comers; and any Hellene who likes may ask him anything. How different
is our lot! my dear Meno. Here at Athens there is a dearth of the
commodity, and all wisdom seems to have emigrated from us to you.
I am certain that if you were to ask any Athenian whether virtue was
natural or acquired, he would laugh in your face, and say: "Stranger,
you have far too good an opinion of me, if you think that I can answer
your question. For I literally do not know what virtue is, and much
less whether it is acquired by teaching or not." And I myself, Meno,
living as I do in this region of poverty, am as poor as the rest of
the world; and I confess with shame that I know literally nothing
about virtue; and when I do not know the "quid" of anything how can
I know the "quale"? How, if I knew nothing at all of Meno, could I
tell if he was fair, or the opposite of fair; rich and noble, or the
reverse of rich and noble? Do you think that I could? 

Men. No, Indeed. But are you in earnest, Socrates, in saying that
you do not know what virtue is? And am I to carry back this report
of you to Thessaly? 

Soc. Not only that, my dear boy, but you may say further that I have
never known of any one else who did, in my judgment. 

Men. Then you have never met Gorgias when he was at Athens?

Soc. Yes, I have. 

Men. And did you not think that he knew? 

Soc. I have not a good memory, Meno, and therefore I cannot now tell
what I thought of him at the time. And I dare say that he did know,
and that you know what he said: please, therefore, to remind me of
what he said; or, if you would rather, tell me your own view; for
I suspect that you and he think much alike. 

Men. Very true. 

Soc. Then as he is not here, never mind him, and do you tell me: By
the gods, Meno, be generous, and tell me what you say that virtue
is; for I shall be truly delighted to find that I have been mistaken,
and that you and Gorgias do really have this knowledge; although I
have been just saying that I have never found anybody who had.

Men. There will be no difficulty, Socrates, in answering your question.
Let us take first the virtue of a man-he should know how to administer
the state, and in the administration of it to benefit his friends
and harm his enemies; and he must also be careful not to suffer harm
himself. A woman's virtue, if you wish to know about that, may also
be easily described: her duty is to order her house, and keep what
is indoors, and obey her husband. Every age, every condition of life,
young or old, male or female, bond or free, has a different virtue:
there are virtues numberless, and no lack of definitions of them;
for virtue is relative to the actions and ages of each of us in all
that we do. And the same may be said of vice, Socrates. 

Soc. How fortunate I am, Meno! When I ask you for one virtue, you
present me with a swarm of them, which are in your keeping. Suppose
that I carry on the figure of the swarm, and ask of you, What is the
nature of the bee? and you answer that there are many kinds of bees,
and I reply: But do bees differ as bees, because there are many and
different kinds of them; or are they not rather to be distinguished
by some other quality, as for example beauty, size, or shape? How
would you answer me? 

Men. I should answer that bees do not differ from one another, as
bees. 

Soc. And if I went on to say: That is what I desire to know, Meno;
tell me what is the quality in which they do not differ, but are all
alike;-would you be able to answer? 

Men. I should. 

Soc. And so of the virtues, however many and different they may be,
they have all a common nature which makes them virtues; and on this
he who would answer the question, "What is virtue?" would do well
to have his eye fixed: Do you understand? 

Men. I am beginning to understand; but I do not as yet take hold of
the question as I could wish. 

Soc. When you say, Meno, that there is one virtue of a man, another
of a woman, another of a child, and so on, does this apply only to
virtue, or would you say the same of health, and size, and strength?
Or is the nature of health always the same, whether in man or woman?

Men. I should say that health is the same, both in man and woman.

Soc. And is not this true of size and strength? If a woman is strong,
she will be strong by reason of the same form and of the same strength
subsisting in her which there is in the man. I mean to say that strength,
as strength, whether of man or woman, is the same. Is there any difference?

Men. I think not. 

Soc. And will not virtue, as virtue, be the same, whether in a child
or in a grown-up person, in a woman or in a man? 

Men. I cannot help feeling, Socrates, that this case is different
from the others. 

Soc. But why? Were you not saying that the virtue of a man was to
order a state, and the virtue of a woman was to order a house?

Men. I did say so. 

Soc. And can either house or state or anything be well ordered without
temperance and without justice? 

Men. Certainly not. 

Soc. Then they who order a state or a house temperately or justly
order them with temperance and justice? 

Men. Certainly. 

Soc. Then both men and women, if they are to be good men and women,
must have the same virtues of temperance and justice? 

Men. True. 

Soc. And can either a young man or an elder one be good, if they are
intemperate and unjust? 

Men. They cannot. 

Soc. They must be temperate and just? 

Men. Yes. 

Soc. Then all men are good in the same way, and by participation in
the same virtues? 

Men. Such is the inference. 

Soc. And they surely would not have been good in the same way, unless
their virtue had been the same? 

Men. They would not. 

Soc. Then now that the sameness of all virtue has been proven, try
and remember what you and Gorgias say that virtue is. 

Men. Will you have one definition of them all? 

Soc. That is what I am seeking. 

Men. If you want to have one definition of them all, I know not what
to say, but that virtue is the power of governing mankind.

Soc. And does this definition of virtue include all virtue? Is virtue
the same in a child and in a slave, Meno? Can the child govern his
father, or the slave his master; and would he who governed be any
longer a slave? 

Men. I think not, Socrates. 

Soc. No, indeed; there would be small reason in that. Yet once more,
fair friend; according to you, virtue is "the power of governing";
but do you not add "justly and not unjustly"? 

Men. Yes, Socrates; I agree there; for justice is virtue.

Soc. Would you say "virtue," Meno, or "a virtue"? 

Men. What do you mean? 

Soc. I mean as I might say about anything; that a round, for example,
is "a figure" and not simply "figure," and I should adopt this mode
of speaking, because there are other figures. 

Men. Quite right; and that is just what I am saying about virtue-that
there are other virtues as well as justice. 

Soc. What are they? tell me the names of them, as I would tell you
the names of the other figures if you asked me. 

Men. Courage and temperance and wisdom and magnanimity are virtues;
and there are many others. 

Soc. Yes, Meno; and again we are in the same case: in searching after
one virtue we have found many, though not in the same way as before;
but we have been unable to find the common virtue which runs through
them all. 

Men. Why, Socrates, even now I am not able to follow you in the attempt
to get at one common notion of virtue as of other things.

Soc. No wonder; but I will try to get nearer if I can, for you know
that all things have a common notion. Suppose now that some one asked
you the question which I asked before: Meno, he would say, what is
figure? And if you answered "roundness," he would reply to you, in
my way of speaking, by asking whether you would say that roundness
is "figure" or "a figure"; and you would answer "a figure."

Men. Certainly. 

Soc. And for this reason-that there are other figures? 

Men. Yes. 

Soc. And if he proceeded to ask, What other figures are there? you
would have told him. 

Men. I should. 

Soc. And if he similarly asked what colour is, and you answered whiteness,
and the questioner rejoined, Would you say that whiteness is colour
or a colour? you would reply, A colour, because there are other colours
as well. 

Men. I should. 

Soc. And if he had said, Tell me what they are?-you would have told
him of other colours which are colours just as much as whiteness.

Men. Yes. 

Soc. And suppose that he were to pursue the matter in my way, he would
say: Ever and anon we are landed in particulars, but this is not what
I want; tell me then, since you call them by a common name, and say
that they are all figures, even when opposed to one another, what
is that common nature which you designate as figure-which contains
straight as well as round, and is no more one than the other-that
would be your mode of speaking? 

Men. Yes. 

Soc. And in speaking thus, you do not mean to say that the round is
round any more than straight, or the straight any more straight than
round? 

Men. Certainly not. 

Soc. You only assert that the round figure is not more a figure than
the straight, or the straight than the round? 

Men. Very true. 

Soc. To what then do we give the name of figure? Try and answer. Suppose
that when a person asked you this question either about figure or
colour, you were to reply, Man, I do not understand what you want,
or know what you are saying; he would look rather astonished and say:
Do you not understand that I am looking for the "simile in multis"?
And then he might put the question in another form: Mono, he might
say, what is that "simile in multis" which you call figure, and which
includes not only round and straight figures, but all? Could you not
answer that question, Meno? I wish that you would try; the attempt
will be good practice with a view to the answer about virtue.

Men. I would rather that you should answer, Socrates. 

Soc. Shall I indulge you? 

Men. By all means. 

Soc. And then you will tell me about virtue? 

Men. I will. 

Soc. Then I must do my best, for there is a prize to be won.

Men. Certainly. 

Soc. Well, I will try and explain to you what figure is. What do you
say to this answer?-Figure is the only thing which always follows
colour. Will you be satisfied with it, as I am sure that I should
be, if you would let me have a similar definition of virtue?

Men. But, Socrates, it is such a simple answer. 

Soc. Why simple? 

Men. Because, according to you, figure is that which always follows
colour. 

(Soc. Granted.) 

Men. But if a person were to say that he does not know what colour
is, any more than what figure is-what sort of answer would you have
given him? 

Soc. I should have told him the truth. And if he were a philosopher
of the eristic and antagonistic sort, I should say to him: You have
my answer, and if I am wrong, your business is to take up the argument
and refute me. But if we were friends, and were talking as you and
I are now, I should reply in a milder strain and more in the dialectician's
vein; that is to say, I should not only speak the truth, but I should
make use of premisses which the person interrogated would be willing
to admit. And this is the way in which I shall endeavour to approach
you. You will acknowledge, will you not, that there is such a thing
as an end, or termination, or extremity?-all which words use in the
same sense, although I am aware that Prodicus might draw distinctions
about them: but still you, I am sure, would speak of a thing as ended
or terminated-that is all which I am saying-not anything very difficult.

Men. Yes, I should; and I believe that I understand your meaning.

Soc. And you would speak of a surface and also of a solid, as for
example in geometry. 

Men. Yes. 

Soc. Well then, you are now in a condition to understand my definition
of figure. I define figure to be that in which the solid ends; or,
more concisely, the limit of solid. 

Men. And now, Socrates, what is colour? 

Soc. You are outrageous, Meno, in thus plaguing a poor old man to
give you an answer, when you will not take the trouble of remembering
what is Gorgias' definition of virtue. 

Men. When you have told me what I ask, I will tell you, Socrates.

Soc. A man who was blindfolded has only to hear you talking, and he
would know that you are a fair creature and have still many lovers.

Men. Why do you think so? 

Soc. Why, because you always speak in imperatives: like all beauties
when they are in their prime, you are tyrannical; and also, as I suspect,
you have found out that I have weakness for the fair, and therefore
to humour you I must answer. 

Men. Please do. 

Soc. Would you like me to answer you after the manner of Gorgias,
which is familiar to you? 

Men. I should like nothing better. 

Soc. Do not he and you and Empedocles say that there are certain effluences
of existence? 

Men. Certainly. 

Soc. And passages into which and through which the effluences pass?

Men. Exactly. 

Soc. And some of the effluences fit into the passages, and some of
them are too small or too large? 

Men. True. 

Soc. And there is such a thing as sight? 

Men. Yes. 

Soc. And now, as Pindar says, "read my meaning" colour is an effluence
of form, commensurate with sight, and palpable to sense.

Men. That, Socrates, appears to me to be an admirable answer.

Soc. Why, yes, because it happens to be one which you have been in
the habit of hearing: and your wit will have discovered, I suspect,
that you may explain in the same way the nature of sound and smell,
and of many other similar phenomena. 

Men. Quite true. 

Soc. The answer, Meno, was in the orthodox solemn vein, and therefore
was more acceptable to you than the other answer about figure.

Men. Yes. 

Soc. And yet, O son of Alexidemus, I cannot help thinking that the
other was the better; and I am sure that you would be of the same
opinion, if you would only stay and be initiated, and were not compelled,
as you said yesterday, to go away before the mysteries. 

Men. But I will stay, Socrates, if you will give me many such answers.

Soc. Well then, for my own sake as well as for yours, I will do my
very best; but I am afraid that I shall not be able to give you very
many as good: and now, in your turn, you are to fulfil your promise,
and tell me what virtue is in the universal; and do not make a singular
into a plural, as the facetious say of those who break a thing, but
deliver virtue to me whole and sound, and not broken into a number
of pieces: I have given you the pattern. 

Men. Well then, Socrates, virtue, as I take it, is when he, who desires
the honourable, is able to provide it for himself; so the poet says,
and I say too- 

Virtue is the desire of things honourable and the power of attaining
them. 

Soc. And does he who desires the honourable also desire the good?

Men. Certainly. 

Soc. Then are there some who desire the evil and others who desire
the good? Do not all men, my dear sir, desire good? 

Men. I think not. 

Soc. There are some who desire evil? 

Men. Yes. 

Soc. Do you mean that they think the evils which they desire, to be
good; or do they know that they are evil and yet desire them?

Men. Both, I think. 

Soc. And do you really imagine, Meno, that a man knows evils to be
evils and desires them notwithstanding? 

Men. Certainly I do. 

Soc. And desire is of possession? 

Men. Yes, of possession. 

Soc. And does he think that the evils will do good to him who possesses
them, or does he know that they will do him harm? 

Men. There are some who think that the evils will do them good, and
others who know that they will do them harm. 

Soc. And, in your opinion, do those who think that they will do them
good know that they are evils? 

Men. Certainly not. 

Soc. Is it not obvious that those who are ignorant of their nature
do not desire them; but they desire what they suppose to be goods
although they are really evils; and if they are mistaken and suppose
the evils to be good they really desire goods? 

Men. Yes, in that case. 

Soc. Well, and do those who, as you say, desire evils, and think that
evils are hurtful to the possessor of them, know that they will be
hurt by them? 

Men. They must know it. 

Soc. And must they not suppose that those who are hurt are miserable
in proportion to the hurt which is inflicted upon them? 

Men. How can it be otherwise? 

Soc. But are not the miserable ill-fated? 

Men. Yes, indeed. 

Soc. And does any one desire to be miserable and ill-fated?

Men. I should say not, Socrates. 

Soc. But if there is no one who desires to be miserable, there is
no one, Meno, who desires evil; for what is misery but the desire
and possession of evil? 

Men. That appears to be the truth, Socrates, and I admit that nobody
desires evil. 

Soc. And yet, were you not saying just now that virtue is the desire
and power of attaining good? 

Men. Yes, I did say so. 

Soc. But if this be affirmed, then the desire of good is common to
all, and one man is no better than another in that respect?

Men. True. 

Soc. And if one man is not better than another in desiring good, he
must be better in the power of attaining it? 

Men. Exactly. 

Soc. Then, according to your definition, virtue would appear to be
the power of attaining good? 

Men. I entirely approve, Socrates, of the manner in which you now
view this matter. 

Soc. Then let us see whether what you say is true from another point
of view; for very likely you may be right:-You affirm virtue to be
the power of attaining goods? 

Men. Yes. 

Soc. And the goods which mean are such as health and wealth and the
possession of gold and silver, and having office and honour in the
state-those are what you would call goods? 

Men. Yes, I should include all those. 

Soc. Then, according to Meno, who is the hereditary friend of the
great king, virtue is the power of getting silver and gold; and would
you add that they must be gained piously, justly, or do you deem this
to be of no consequence? And is any mode of acquisition, even if unjust
and dishonest, equally to be deemed virtue? 

Men. Not virtue, Socrates, but vice. 

Soc. Then justice or temperance or holiness, or some other part of
virtue, as would appear, must accompany the acquisition, and without
them the mere acquisition of good will not be virtue. 

Men. Why, how can there be virtue without these? 

Soc. And the non-acquisition of gold and silver in a dishonest manner
for oneself or another, or in other words the want of them, may be
equally virtue? 

Men. True. 

Soc. Then the acquisition of such goods is no more virtue than the
non-acquisition and want of them, but whatever is accompanied by justice
or honesty is virtue, and whatever is devoid of justice is vice.

Men. It cannot be otherwise, in my judgment. 

Soc. And were we not saying just now that justice, temperance, and
the like, were each of them a part of virtue? 

Men. Yes. 

Soc. And so, Meno, this is the way in which you mock me.

Men. Why do you say that, Socrates? 

Soc. Why, because I asked you to deliver virtue into my hands whole
and unbroken, and I gave you a pattern according to which you were
to frame your answer; and you have forgotten already, and tell me
that virtue is the power of attaining good justly, or with justice;
and justice you acknowledge to be a part of virtue. 

Men. Yes. 

Soc. Then it follows from your own admissions, that virtue is doing
what you do with a part of virtue; for justice and the like are said
by you to be parts of virtue. 

Men. What of that? 

Soc. What of that! Why, did not I ask you to tell me the nature of
virtue as a whole? And you are very far from telling me this; but
declare every action to be virtue which is done with a part of virtue;
as though you had told me and I must already know the whole of virtue,
and this too when frittered away into little pieces. And, therefore,
my dear I fear that I must begin again and repeat the same question:
What is virtue? for otherwise, I can only say, that every action done
with a part of virtue is virtue; what else is the meaning of saying
that every action done with justice is virtue? Ought I not to ask
the question over again; for can any one who does not know virtue
know a part of virtue? 

Men. No; I do not say that he can. 

Soc. Do you remember how, in the example of figure, we rejected any
answer given in terms which were as yet unexplained or unadmitted?

Men. Yes, Socrates; and we were quite right in doing so.

Soc. But then, my friend, do not suppose that we can explain to any
one the nature of virtue as a whole through some unexplained portion
of virtue, or anything at all in that fashion; we should only have
to ask over again the old question, What is virtue? Am I not right?

Men. I believe that you are. 

Soc. Then begin again, and answer me, What, according to you and your
friend Gorgias, is the definition of virtue? 

Men. O Socrates, I used to be told, before I knew you, that you were
always doubting yourself and making others doubt; and now you are
casting your spells over me, and I am simply getting bewitched and
enchanted, and am at my wits' end. And if I may venture to make a
jest upon you, you seem to me both in your appearance and in your
power over others to be very like the flat torpedo fish, who torpifies
those who come near him and touch him, as you have now torpified me,
I think. For my soul and my tongue are really torpid, and I do not
know how to answer you; and though I have been delivered of an infinite
variety of speeches about virtue before now, and to many persons-and
very good ones they were, as I thought-at this moment I cannot even
say what virtue is. And I think that. you are very wise in not voyaging
and going away from home, for if you did in other places as do in
Athens, you would be cast into prison as a magician. 

Soc. You are a rogue, Meno, and had all but caught me. 

Men. What do you mean, Socrates? 

Soc. I can tell why you made a simile about me. 

Men. Why? 

Soc. In order that I might make another simile about you. For I know
that all pretty young gentlemen like to have pretty similes made about
them-as well they may-but I shall not return the compliment. As to
my being a torpedo, if the torpedo is torpid as well as the cause
of torpidity in others, then indeed I am a torpedo, but not otherwise;
for I perplex others, not because I am clear, but because I am utterly
perplexed myself. And now I know not what virtue is, and you seem
to be in the same case, although you did once perhaps know before
you touched me. However, I have no objection to join with you in the
enquiry. 

Men. And how will you enquire, Socrates, into that which you do not
know? What will you put forth as the subject of enquiry? And if you
find what you want, how will you ever know that this is the thing
which you did not know? 

Soc. I know, Meno, what you mean; but just see what a tiresome dispute
you are introducing. You argue that man cannot enquire either about
that which he knows, or about that which he does not know; for if
he knows, he has no need to enquire; and if not, he cannot; for he
does not know the, very subject about which he is to enquire.

Men. Well, Socrates, and is not the argument sound? 

Soc. I think not. 

Men. Why not? 

Soc. I will tell you why: I have heard from certain wise men and women
who spoke of things divine that- 

Men. What did they say? 

Soc. They spoke of a glorious truth, as I conceive. 

Men. What was it? and who were they? 

Soc. Some of them were priests and priestesses, who had studied how
they might be able to give a reason of their profession: there, have
been poets also, who spoke of these things by inspiration, like Pindar,
and many others who were inspired. And they say-mark, now, and see
whether their words are true-they say that the soul of man is immortal,
and at one time has an end, which is termed dying, and at another
time is born again, but is never destroyed. And the moral is, that
a man ought to live always in perfect holiness. "For in the ninth
year Persephone sends the souls of those from whom she has received
the penalty of ancient crime back again from beneath into the light
of the sun above, and these are they who become noble kings and mighty
men and great in wisdom and are called saintly heroes in after ages."
The soul, then, as being immortal, and having been born again many
times, rand having seen all things that exist, whether in this world
or in the world below, has knowledge of them all; and it is no wonder
that she should be able to call to remembrance all that she ever knew
about virtue, and about everything; for as all nature is akin, and
the soul has learned all things; there is no difficulty in her eliciting
or as men say learning, out of a single recollection -all the rest,
if a man is strenuous and does not faint; for all enquiry and all
learning is but recollection. And therefore we ought not to listen
to this sophistical argument about the impossibility of enquiry: for
it will make us idle; and is sweet only to the sluggard; but the other
saying will make us active and inquisitive. In that confiding, I will
gladly enquire with you into the nature of virtue. 

Men. Yes, Socrates; but what do you mean by saying that we do not
learn, and that what we call learning is only a process of recollection?
Can you teach me how this is? 

Soc. I told you, Meno, just now that you were a rogue, and now you
ask whether I can teach you, when I am saying that there is no teaching,
but only recollection; and thus you imagine that you will involve
me in a contradiction. 

Men. Indeed, Socrates, I protest that I had no such intention. I only
asked the question from habit; but if you can prove to me that what
you say is true, I wish that you would. 

Soc. It will be no easy matter, but I will try to please you to the
utmost of my power. Suppose that you call one of your numerous attendants,
that I may demonstrate on him. 

Men. Certainly. Come hither, boy. 

Soc. He is Greek, and speaks Greek, does he not? 

Men. Yes, indeed; he was born in the house. 

Soc. Attend now to the questions which I ask him, and observe whether
he learns of me or only remembers. 

Men. I will. 

Soc. Tell me, boy, do you know that a figure like this is a square?

Boy. I do. 

Soc. And you know that a square figure has these four lines equal?

Boy. Certainly. 

Soc. And these lines which I have drawn through the middle of the
square are also equal? 

Boy. Yes. 

Soc. A square may be of any size? 

Boy. Certainly. 

Soc. And if one side of the figure be of two feet, and the other side
be of two feet, how much will the whole be? Let me explain: if in
one direction the space was of two feet, and in other direction of
one foot, the whole would be of two feet taken once? 

Boy. Yes. 

Soc. But since this side is also of two feet, there are twice two
feet? 

Boy. There are. 

Soc. Then the square is of twice two feet? 

Boy. Yes. 

Soc. And how many are twice two feet? count and tell me.

Boy. Four, Socrates. 

Soc. And might there not be another square twice as large as this,
and having like this the lines equal? 

Boy. Yes. 

Soc. And of how many feet will that be? 

Boy. Of eight feet. 

Soc. And now try and tell me the length of the line which forms the
side of that double square: this is two feet-what will that be?

Boy. Clearly, Socrates, it will be double. 

Soc. Do you observe, Meno, that I am not teaching the boy anything,
but only asking him questions; and now he fancies that he knows how
long a line is necessary in order to produce a figure of eight square
feet; does he not? 

Men. Yes. 

Soc. And does he really know? 

Men. Certainly not. 

Soc. He only guesses that because the square is double, the line is
double. 

Men. True. 

Soc. Observe him while he recalls the steps in regular order. (To
the Boy.) Tell me, boy, do you assert that a double space comes from
a double line? Remember that I am not speaking of an oblong, but of
a figure equal every way, and twice the size of this-that is to say
of eight feet; and I want to know whether you still say that a double
square comes from double line? 

Boy. Yes. 

Soc. But does not this line become doubled if we add another such
line here? 

Boy. Certainly. 

Soc. And four such lines will make a space containing eight feet?

Boy. Yes. 

Soc. Let us describe such a figure: Would you not say that this is
the figure of eight feet? 

Boy. Yes. 

Soc. And are there not these four divisions in the figure, each of
which is equal to the figure of four feet? 

Boy. True. 

Soc. And is not that four times four? 

Boy. Certainly. 

Soc. And four times is not double? 

Boy. No, indeed. 

Soc. But how much? 

Boy. Four times as much. 

Soc. Therefore the double line, boy, has given a space, not twice,
but four times as much. 

Boy. True. 

Soc. Four times four are sixteen-are they not? 

Boy. Yes. 

Soc. What line would give you a space of right feet, as this gives
one of sixteen feet;-do you see? 

Boy. Yes. 

Soc. And the space of four feet is made from this half line?

Boy. Yes. 

Soc. Good; and is not a space of eight feet twice the size of this,
and half the size of the other? 

Boy. Certainly. 

Soc. Such a space, then, will be made out of a line greater than this
one, and less than that one? 

Boy. Yes; I think so. 

Soc. Very good; I like to hear you say what you think. And now tell
me, is not this a line of two feet and that of four? 

Boy. Yes. 

Soc. Then the line which forms the side of eight feet ought to be
more than this line of two feet, and less than the other of four feet?

Boy. It ought. 

Soc. Try and see if you can tell me how much it will be.

Boy. Three feet. 

Soc. Then if we add a half to this line of two, that will be the line
of three. Here are two and there is one; and on the other side, here
are two also and there is one: and that makes the figure of which
you speak? 

Boy. Yes. 

Soc. But if there are three feet this way and three feet that way,
the whole space will be three times three feet? 

Boy. That is evident. 

Soc. And how much are three times three feet? 

Boy. Nine. 

Soc. And how much is the double of four? 

Boy. Eight. 

Soc. Then the figure of eight is not made out of a of three?

Boy. No. 

Soc. But from what line?-tell me exactly; and if you would rather
not reckon, try and show me the line. 

Boy. Indeed, Socrates, I do not know. 

Soc. Do you see, Meno, what advances he has made in his power of recollection?
He did not know at first, and he does not know now, what is the side
of a figure of eight feet: but then he thought that he knew, and answered
confidently as if he knew, and had no difficulty; now he has a difficulty,
and neither knows nor fancies that he knows. 

Men. True. 

Soc. Is he not better off in knowing his ignorance? 

Men. I think that he is. 

Soc. If we have made him doubt, and given him the "torpedo's shock,"
have we done him any harm? 

Men. I think not. 

Soc. We have certainly, as would seem, assisted him in some degree
to the discovery of the truth; and now he will wish to remedy his
ignorance, but then he would have been ready to tell all the world
again and again that the double space should have a double side.

Men. True. 

Soc. But do you suppose that he would ever have enquired into or learned
what he fancied that he knew, though he was really ignorant of it,
until he had fallen into perplexity under the idea that he did not
know, and had desired to know? 

Men. I think not, Socrates. 

Soc. Then he was the better for the torpedo's touch? 

Men. I think so. 

Soc. Mark now the farther development. I shall only ask him, and not
teach him, and he shall share the enquiry with me: and do you watch
and see if you find me telling or explaining anything to him, instead
of eliciting his opinion. Tell me, boy, is not this a square of four
feet which I have drawn? 

Boy. Yes. 

Soc. And now I add another square equal to the former one?

Boy. Yes. 

Soc. And a third, which is equal to either of them? 

Boy. Yes. 

Soc. Suppose that we fill up the vacant corner? 

Boy. Very good. 

Soc. Here, then, there are four equal spaces? 

Boy. Yes. 

Soc. And how many times larger is this space than this other?

Boy. Four times. 

Soc. But it ought to have been twice only, as you will remember.

Boy. True. 

Soc. And does not this line, reaching from corner to corner, bisect
each of these spaces? 

Boy. Yes. 

Soc. And are there not here four equal lines which contain this space?

Boy. There are. 

Soc. Look and see how much this space is. 

Boy. I do not understand. 

Soc. Has not each interior line cut off half of the four spaces?

Boy. Yes. 

Soc. And how many spaces are there in this section? 

Boy. Four. 

Soc. And how many in this? 

Boy. Two. 

Soc. And four is how many times two? 

Boy. Twice. 

Soc. And this space is of how many feet? 

Boy. Of eight feet. 

Soc. And from what line do you get this figure? 

Boy. From this. 

Soc. That is, from the line which extends from corner to corner of
the figure of four feet? 

Boy. Yes. 

Soc. And that is the line which the learned call the diagonal. And
if this is the proper name, then you, Meno's slave, are prepared to
affirm that the double space is the square of the diagonal?

Boy. Certainly, Socrates. 

Soc. What do you say of him, Meno? Were not all these answers given
out of his own head? 

Men. Yes, they were all his own. 

Soc. And yet, as we were just now saying, he did not know?

Men. True. 

Soc. But still he had in him those notions of his-had he not?

Men. Yes. 

Soc. Then he who does not know may still have true notions of that
which he does not know? 

Men. He has. 

Soc. And at present these notions have just been stirred up in him,
as in a dream; but if he were frequently asked the same questions,
in different forms, he would know as well as any one at last?

Men. I dare say. 

Soc. Without any one teaching him he will recover his knowledge for
himself, if he is only asked questions? 

Men. Yes. 

Soc. And this spontaneous recovery of knowledge in him is recollection?

Men. True. 

Soc. And this knowledge which he now has must he not either have acquired
or always possessed? 

Men. Yes. 

Soc. But if he always possessed this knowledge he would always have
known; or if he has acquired the knowledge he could not have acquired
it in this life, unless he has been taught geometry; for he may be
made to do the same with all geometry and every other branch of knowledge.
Now, has any one ever taught him all this? You must know about him,
if, as you say, he was born and bred in your house. 

Men. And I am certain that no one ever did teach him. 

Soc. And yet he has the knowledge? 

Men. The fact, Socrates, is undeniable. 

Soc. But if he did not acquire the knowledge in this life, then he
must have had and learned it at some other time? 

Men. Clearly he must. 

Soc. Which must have been the time when he was not a man?

Men. Yes. 

Soc. And if there have been always true thoughts in him, both at the
time when he was and was not a man, which only need to be awakened
into knowledge by putting questions to him, his soul must have always
possessed this knowledge, for he always either was or was not a man?

Men. Obviously. 

Soc. And if the truth of all things always existed in the soul, then
the soul is immortal. Wherefore be of good cheer, and try to recollect
what you do not know, or rather what you do not remember.

Men. I feel, somehow, that I like what you are saying. 

Soc. And I, Meno, like what I am saying. Some things I have said of
which I am not altogether confident. But that we shall be better and
braver and less helpless if we think that we ought to enquire, than
we should have been if we indulged in the idle fancy that there was
no knowing and no use in seeking to know what we do not know;-that
is a theme upon which I am ready to fight, in word and deed, to the
utmost of my power. 

Men. There again, Socrates, your words seem to me excellent.

Soc. Then, as we are agreed that a man should enquire about that which
he does not know, shall you and I make an effort to enquire together
into the nature of virtue? 

Men. By all means, Socrates. And yet I would much rather return to
my original question, Whether in seeking to acquire virtue we should
regard it as a thing to be taught, or as a gift of nature, or as coming
to men in some other way? 

Soc. Had I the command of you as well as of myself, Meno, I would
not have enquired whether virtue is given by instruction or not, until
we had first ascertained "what it is." But as you think only of controlling
me who am your slave, and never of controlling yourself,-such being
your notion of freedom, I must yield to you, for you are irresistible.
And therefore I have now to enquire into the qualities of a thing
of which I do not as yet know the nature. At any rate, will you condescend
a little, and allow the question "Whether virtue is given by instruction,
or in any other way," to be argued upon hypothesis? As the geometrician,
when he is asked whether a certain triangle is capable being inscribed
in a certain circle, will reply: "I cannot tell you as yet; but I
will offer a hypothesis which may assist us in forming a conclusion:
If the figure be such that when you have produced a given side of
it, the given area of the triangle falls short by an area corresponding
to the part produced, then one consequence follows, and if this is
impossible then some other; and therefore I wish to assume a hypothesis
before I tell you whether this triangle is capable of being inscribed
in the circle":-that is a geometrical hypothesis. And we too, as we
know not the nature and -qualities of virtue, must ask, whether virtue
is or not taught, under a hypothesis: as thus, if virtue is of such
a class of mental goods, will it be taught or not? Let the first hypothesis
be-that virtue is or is not knowledge,-in that case will it be taught
or not? or, as we were just now saying, remembered"? For there is
no use in disputing about the name. But is virtue taught or not? or
rather, does not everyone see that knowledge alone is taught?

Men. I agree. 

Soc. Then if virtue is knowledge, virtue will be taught?

Men. Certainly. 

Soc. Then now we have made a quick end of this question: if virtue
is of such a nature, it will be taught; and if not, not?

Men. Certainly. 

Soc. The next question is, whether virtue is knowledge or of another
species? 

Men. Yes, that appears to be the -question which comes next in order.

Soc. Do we not say that virtue is a good?-This is a hypothesis which
is not set aside. 

Men. Certainly. 

Soc. Now, if there be any sort-of good which is distinct from knowledge,
virtue may be that good; but if knowledge embraces all good, then
we shall be right in think in that virtue is knowledge? 

Men. True. 

Soc. And virtue makes us good? 

Men. Yes. 

Soc. And if we are good, then we are profitable; for all good things
are profitable? 

Men. Yes. 

Soc. Then virtue is profitable? 

Men. That is the only inference. 

Soc. Then now let us see what are the things which severally profit
us. Health and strength, and beauty and wealth-these, and the like
of these, we call profitable? 

Men. True. 

Soc. And yet these things may also sometimes do us harm: would you
not think so? 

Men. Yes. 

Soc. And what is the guiding principle which makes them profitable
or the reverse? Are they not profitable when they are rightly used,
and hurtful when they are not rightly used? 

Men. Certainly. 

Soc. Next, let us consider the goods of the soul: they are temperance,
justice, courage, quickness of apprehension, memory, magnanimity,
and the like? 

Men. Surely. 

Soc. And such of these as are not knowledge, but of another sort,
are sometimes profitable and sometimes hurtful; as, for example, courage
wanting prudence, which is only a sort of confidence? When a man has
no sense he is harmed by courage, but when he has sense he is profited?

Men. True. 

Soc. And the same may be said of temperance and quickness of apprehension;
whatever things are learned or done with sense are profitable, but
when done without sense they are hurtful? 

Men. Very true. 

Soc. And in general, all that the attempts or endures, when under
the guidance of wisdom, ends in happiness; but when she is under the
guidance of folly, in the opposite? 

Men. That appears to be true. 

Soc. If then virtue is a quality of the soul, and is admitted to be
profitable, it must be wisdom or prudence, since none of the things
of the soul are either profitable or hurtful in themselves, but they
are all made profitable or hurtful by the addition of wisdom or of
folly; and therefore and therefore if virtue is profitable, virtue
must be a sort of wisdom or prudence? 

Men. I quite agree. 

Soc. And the other goods, such as wealth and the like, of which we
were just now saying that they are sometimes good and sometimes evil,
do not they also become profitable or hurtful, accordingly as the
soul guides and uses them rightly or wrongly; just as the things of
the soul herself are benefited when under the guidance of wisdom and
harmed by folly? 

Men. True. 

Soc. And the wise soul guides them rightly, and the foolish soul wrongly.

Men. Yes. 

Soc. And is not this universally true of human nature? All other things
hang upon the soul, and the things of the soul herself hang upon wisdom,
if they are to be good; and so wisdom is inferred to be that which
profits-and virtue, as we say, is profitable? 

Men. Certainly. 

Soc. And thus we arrive at the conclusion that virtue is either wholly
or partly wisdom? 

Men. I think that what you are saying, Socrates, is very true.

Soc. But if this is true, then the good are not by nature good?

Men. I think not. 

Soc. If they had been, there would assuredly have been discerners
of characters among us who would have known our future great men;
and on their showing we should have adopted them, and when we had
got them, we should have kept them in the citadel out of the way of
harm, and set a stamp upon them far rather than upon a piece of gold,
in order that no one might tamper with them; and when they grew up
they would have been useful to the state? 

Men. Yes, Socrates, that would have been the right way. 

Soc. But if the good are not by nature good, are they made good by
instruction? 

Men. There appears to be no other alternative, Socrates. On the supposition
that virtue is knowledge, there can be no doubt that virtue is taught.

Soc. Yes, indeed; but what if the supposition is erroneous?

Men. I certainly thought just now that we were right. 

Soc. Yes, Meno; but a principle which has any soundness should stand
firm not only just now, but always. 

Men. Well; and why are you so slow of heart to believe that knowledge
is virtue? 

Soc. I will try and tell you why, Meno. I do not retract the assertion
that if virtue is knowledge it may be taught; but I fear that I have
some reason in doubting whether virtue is knowledge: for consider
now. and say whether virtue, and not only virtue but anything that
is taught, must not have teachers and disciples? 

Men. Surely. 

Soc. And conversely, may not the art of which neither teachers nor
disciples exist be assumed to be incapable of being taught?

Men. True; but do you think that there are no teachers of virtue?

Soc. I have certainly often enquired whether there were any, and taken
great pains to find them, and have never succeeded; and many have
assisted me in the search, and they were the persons whom I thought
the most likely to know. Here at the moment when he is wanted we fortunately
have sitting by us Anytus, the very person of whom we should make
enquiry; to him then let us repair. In the first Place, he is the
son of a wealthy and wise father, Anthemion, who acquired his wealth,
not by accident or gift, like Ismenias the Theban (who has recently
made himself as rich as Polycrates), but by his own skill and industry,
and who is a well-conditioned, modest man, not insolent, or over-bearing,
or annoying; moreover, this son of his has received a good education,
as the Athenian people certainly appear to think, for they choose
him to fill the highest offices. And these are the sort of men from
whom you are likely to learn whether there are any teachers of virtue,
and who they are. Please, Anytus, to help me and your friend Meno
in answering our question, Who are the teachers? Consider the matter
thus: If we wanted Meno to be a good physician, to whom should we
send him? Should we not send him to the physicians? 

Any. Certainly. 

Soc. Or if we wanted him to be a good cobbler, should we not send
him to the cobblers? 

Any. Yes. 

Soc. And so forth? 

Any. Yes. 

Soc. Let me trouble you with one more question. When we say that we
should be right in sending him to the physicians if we wanted him
to be a physician, do we mean that we should be right in sending him
to those who profess the art, rather than to those who do not, and
to those who demand payment for teaching the art, and profess to teach
it to any one who will come and learn? And if these were our reasons,
should we not be right in sending him? 

Any. Yes. 

Soc. And might not the same be said of flute-playing, and of the other
arts? Would a man who wanted to make another a flute-player refuse
to send him to those who profess to teach the art for money, and be
plaguing other persons to give him instruction, who are not professed
teachers and who never had a single disciple in that branch of knowledge
which he wishes him to acquire-would not such conduct be the height
of folly? 

Any. Yes, by Zeus, and of ignorance too. 

Soc. Very good. And now you are in a position to advise with me about
my friend Meno. He has been telling me, Anytus, that he desires to
attain that kind of wisdom and-virtue by which men order the state
or the house, and honour their parents, and know when to receive and
when to send away citizens and strangers, as a good man should. Now,
to whom should he go in order that he may learn this virtue? Does
not the previous argument imply clearly that we should send him to
those who profess and avouch that they are the common teachers of
all Hellas, and are ready to impart instruction to any one who likes,
at a fixed price? 

Any. Whom do you mean, Socrates? 

Soc. You surely know, do you not, Anytus, that these are the people
whom mankind call Sophists? 

Any. By Heracles, Socrates, forbear! I only hope that no friend or
kinsman or acquaintance of mine, whether citizen or stranger, will
ever be so mad as to allow himself to be corrupted by them; for they
are a manifest pest and corrupting influences to those who have to
do with them. 

Soc. What, Anytus? Of all the people who profess that they know how
to do men good, do you mean to say that these are the only ones who
not only do them no good, but positively corrupt those who are entrusted
to them, and in return for this disservice have the face to demand
money? Indeed, I cannot believe you; for I know of a single man, Protagoras,
who made more out of his craft than the illustrious Pheidias, who
created such noble works, or any ten other statuaries. How could that
A mender of old shoes, or patcher up of clothes, who made the shoes
or clothes worse than he received them, could not have remained thirty
days undetected, and would very soon have starved; whereas during
more than forty years, Protagoras was corrupting all Hellas, and sending
his disciples from him worse than he received them, and he was never
found out. For, if I am not mistaken,-he was about seventy years old
at his death, forty of which were spent in the practice of his profession;
and during all that time he had a good reputation, which to this day
he retains: and not only Protagoras, but many others are well spoken
of; some who lived before him, and others who are still living. Now,
when you say that they deceived and corrupted the youth, are they
to be supposed to have corrupted them consciously or unconsciously?
Can those who were deemed by many to be the wisest men of Hellas have
been out of their minds? 

Any. Out of their minds! No, Socrates; the young men who gave their
money to them, were out of their minds, and their relations and guardians
who entrusted their youth to the care of these men were still more
out of their minds, and most of all, the cities who allowed them to
come in, and did not drive them out, citizen and stranger alike.

Soc. Has any of the Sophists wronged you, Anytus? What makes you so
angry with them? 

Any. No, indeed, neither I nor any of my belongings has ever had,
nor would I suffer them to have, anything to do with them.

Soc. Then you are entirely unacquainted with them? 

Any. And I have no wish to be acquainted. 

Soc. Then, my dear friend, how can you know whether a thing is good
or bad of which you are wholly ignorant? 

Any. Quite well; I am sure that I know what manner of men these are,
whether I am acquainted with them or not. 

Soc. You must be a diviner, Anytus, for I really cannot make out,
judging from your own words, how, if you are not acquainted with them,
you know about them. But I am not enquiring of you who are the teachers
who will corrupt Meno (let them be, if you please, the Sophists);
I only ask you to tell him who there is in this great city who will
teach him how to become eminent in the virtues which I was just, now
describing. He is the friend of your family, and you will oblige him.

Any. Why do you not tell him yourself? 

Soc. I have told him whom I supposed to be the teachers of these things;
but I learn from you that I am utterly at fault, and I dare say that
you are right. And now I wish that you, on your part, would tell me
to whom among the Athenians he should go. Whom would you name? Any.
Why single out individuals? Any Athenian gentleman, taken at random,
if he will mind him, will do far more, good to him than the Sophists.

Soc. And did those gentlemen grow of themselves; and without having
been taught by any one, were they nevertheless able to teach others
that which they had never learned themselves? 

Any. I imagine that they learned of the previous generation of gentlemen.
Have there not been many good men in this city? 

Soc. Yes, certainly, Anytus; and many good statesmen also there always
have been and there are still, in the city of Athens. But the question
is whether they were also good teachers of their own virtue;-not whether
there are, or have been, good men in this part of the world, but whether
virtue can be taught, is the question which we have been discussing.
Now, do we mean to say that the good men our own and of other times
knew how to impart to others that virtue which they had themselves;
or is virtue a thing incapable of being communicated or imparted by
one man to another? That is the question which I and Meno have been
arguing. Look at the matter in your own way: Would you not admit that
Themistocles was a good man? 

Any. Certainly; no man better. 

Soc. And must not he then have been a good teacher, if any man ever
was a good teacher, of his own virtue? 

Any. Yes certainly,-if he wanted to be so. 

Soc. But would he not have wanted? He would, at any rate, have desired
to make his own son a good man and a gentleman; he could not have
been jealous of him, or have intentionally abstained from imparting
to him his own virtue. Did you never hear that he made his son Cleophantus
a famous horseman; and had him taught to stand upright on horseback
and hurl a javelin, and to do many other marvellous things; and in
anything which could be learned from a master he was well trained?
Have you not heard from our elders of him? 

Any. I have. 

Soc. Then no one could say that his son showed any want of capacity?

Any. Very likely not. 

Soc. But did any one, old or young, ever say in your hearing that
Cleophantus, son of Themistocles, was a wise or good man, as his father
was? 

Any. I have certainly never heard any one say so. 

Soc. And if virtue could have been taught, would his father Themistocles
have sought to train him in these minor accomplishments, and allowed
him who, as you must remember, was his own son, to be no better than
his neighbours in those qualities in which he himself excelled?

Any. Indeed, indeed, I think not. 

Soc. Here was a teacher of virtue whom you admit to be among the best
men of the past. Let us take another,-Aristides, the son of Lysimachus:
would you not acknowledge that he was a good man? 

Any. To be sure I should. 

Soc. And did not he train his son Lysimachus better than any other
Athenian in all that could be done for him by the help of masters?
But what has been the result? Is he a bit better than any other mortal?
He is an acquaintance of yours, and you see what he is like. There
is Pericles, again, magnificent in his wisdom; and he, as you are
aware, had two sons, Paralus and Xanthippus. 

Any. I know. 

Soc. And you know, also, that he taught them to be unrivalled horsemen,
and had them trained in music and gymnastics and all sorts of arts-in
these respects they were on a level with the best-and had he no wish
to make good men of them? Nay, he must have wished it. But virtue,
as I suspect, could not be taught. And that you may not suppose the
incompetent teachers to be only the meaner sort of Athenians and few
in number, remember again that Thucydides had two sons, Melesias and
Stephanus, whom, besides giving them a good education in other things,
he trained in wrestling, and they were the best wrestlers in Athens:
one of them he committed to the care of Xanthias, and the other of
Eudorus, who had the reputation of being the most celebrated wrestlers
of that day. Do you remember them? 

Any. I have heard of them. 

Soc. Now, can there be a doubt that Thucydides, whose children were
taught things for which he had to spend money, would have taught them
to be good men, which would have cost him nothing, if virtue could
have been taught? Will you reply that he was a mean man, and had not
many friends among the Athenians and allies? Nay, but he was of a
great family, and a man of influence at Athens and in all Hellas,
and, if virtue could have been taught, he would have found out some
Athenian or foreigner who would have made good men of his sons, if
he could not himself spare the time from cares of state. Once more,
I suspect, friend Anytus, that virtue is not a thing which can be
taught? 

Any. Socrates, I think that you are too ready to speak evil of men:
and, if you will take my advice, I would recommend you to be careful.
Perhaps there is no city in which it is not easier to do men harm
than to do them good, and this is certainly the case at Athens, as
I believe that you know. 

Soc. O Meno, think that Anytus is in a rage. And he may well be in
a rage, for he thinks, in the first place, that I am defaming these
gentlemen; and in the second place, he is of opinion that he is one
of them himself. But some day he will know what is the meaning of
defamation, and if he ever does, he will forgive me. Meanwhile I will
return to you, Meno; for I suppose that there are gentlemen in your
region too? 

Men. Certainly there are. 

Soc. And are they willing to teach the young? and do they profess
to be teachers? and do they agree that virtue is taught?

Men. No indeed, Socrates, they are anything but agreed; you may hear
them saying at one time that virtue can be taught, and then again
the reverse. 

Soc. Can we call those teachers who do not acknowledge the possibility
of their own vocation? 

Men. I think not, Socrates. 

Soc. And what do you think of these Sophists, who are the only professors?
Do they seem to you to be teachers of virtue? 

Men. I often wonder, Socrates, that Gorgias is never heard promising
to teach virtue: and when he hears others promising he only laughs
at them; but he thinks that men should be taught to speak.

Soc. Then do you not think that the Sophists are teachers?

Men. I cannot tell you, Socrates; like the rest of the world, I am
in doubt, and sometimes I think that they are teachers and sometimes
not. 

Soc. And are you aware that not you only and other politicians have
doubts whether virtue can be taught or not, but that Theognis the
poet says the very same thing? 

Men. Where does he say so? 

Soc. In these elegiac verses: 

Eat and drink and sit with the mighty, and make yourself agreeable
to them; for from the good you will learn what is good, but if you
mix with the bad you will lose the intelligence which you already
have. Do you observe that here he seems to imply that virtue can be
taught? 

Men. Clearly. 

Soc. But in some other verses he shifts about and says: 

If understanding could be created and put into a man, then they [who
were able to perform this feat] would have obtained great rewards.
And again:- 

Never would a bad son have sprung from a good sire, for he would have
heard the voice of instruction; but not by teaching will you ever
make a bad man into a good one. And this, as you may remark, is a
contradiction of the other. 

Men. Clearly. 

Soc. And is there anything else of which the professors are affirmed
not only not to be teachers of others, but to be ignorant themselves,
and bad at the knowledge of that which they are professing to teach?
or is there anything about which even the acknowledged "gentlemen"
are sometimes saying that "this thing can be taught," and sometimes
the opposite? Can you say that they are teachers in any true sense
whose ideas are in such confusion? 

Men. I should say, certainly not. 

Soc. But if neither the Sophists nor the gentlemen are teachers, clearly
there can be no other teachers? 

Men. No. 

Soc. And if there are no teachers, neither are there disciples?

Men. Agreed. 

Soc. And we have admitted that a thing cannot be taught of which there
are neither teachers nor disciples? 

Men. We have. 

Soc. And there are no teachers of virtue to be found anywhere?

Men. There are not. 

Soc. And if there are no teachers, neither are there scholars?

Men. That, I think, is true. 

Soc. Then virtue cannot be taught? 

Men. Not if we are right in our view. But I cannot believe, Socrates,
that there are no good men: And if there are, how did they come into
existence? 

Soc. I am afraid, Meno, that you and I are not good for much, and
that Gorgias has been as poor an educator of you as Prodicus has been
of me. Certainly we shall have to look to ourselves, and try to find
some one who will help in some way or other to improve us. This I
say, because I observe that in the previous discussion none of us
remarked that right and good action is possible to man under other
guidance than that of knowledge (episteme);-and indeed if this be
denied, there is no seeing how there can be any good men at all.

Men. How do you mean, Socrates? 

Soc. I mean that good men are necessarily useful or profitable. Were
we not right in admitting this? It must be so. 

Men. Yes. 

Soc. And in supposing that they will be useful only if they are true
guides to us of action-there we were also right? 

Men. Yes. 

Soc. But when we said that a man cannot be a good guide unless he
have knowledge (phrhonesis), this we were wrong. 

Men. What do you mean by the word "right"? 

Soc. I will explain. If a man knew the way to Larisa, or anywhere
else, and went to the place and led others thither, would he not be
a right and good guide? 

Men. Certainly. 

Soc. And a person who had a right opinion about the way, but had never
been and did not know, might be a good guide also, might he not?

Men. Certainly. 

Soc. And while he has true opinion about that which the other knows,
he will be just as good a guide if he thinks the truth, as he who
knows the truth? 

Men. Exactly. 

Soc. Then true opinion is as good a guide to correct action as knowledge;
and that was the point which we omitted in our speculation about the
nature of virtue, when we said that knowledge only is the guide of
right action; whereas there is also right opinion. 

Men. True. 

Soc. Then right opinion is not less useful than knowledge?

Men. The difference, Socrates, is only that he who has knowledge will
always be right; but he who has right opinion will sometimes be right,
and sometimes not. 

Soc. What do you mean? Can he be wrong who has right opinion, so long
as he has right opinion? 

Men. I admit the cogency of your argument, and therefore, Socrates,
I wonder that knowledge should be preferred to right opinion-or why
they should ever differ. 

Soc. And shall I explain this wonder to you? 

Men. Do tell me. 

Soc. You would not wonder if you had ever observed the images of Daedalus;
but perhaps you have not got them in your country? 

Men. What have they to do with the question? 

Soc. Because they require to be fastened in order to keep them, and
if they are not fastened they will play truant and run away.

Men. Well. what of that? 

Soc. I mean to say that they are not very valuable possessions if
they are at liberty, for they will walk off like runaway slaves; but
when fastened, they are of great value, for they are really beautiful
works of art. Now this is an illustration of the nature of true opinions:
while they abide with us they are beautiful and fruitful, but they
run away out of the human soul, and do not remain long, and therefore
they are not of much value until they are fastened by the tie of the
cause; and this fastening of them, friend Meno, is recollection, as
you and I have agreed to call it. But when they are bound, in the
first place, they have the nature of knowledge; and, in the second
place, they are abiding. And this is why knowledge is more honourable
and excellent than true opinion, because fastened by a chain.

Men. What you are saying, Socrates, seems to be very like the truth.

Soc. I too speak rather in ignorance; I only conjecture. And yet that
knowledge differs from true opinion is no matter of conjecture with
me. There are not many things which I profess to know, but this is
most certainly one of them. 

Men. Yes, Socrates; and you are quite right in saying so.

Soc. And am I not also right in saying that true opinion leading the
way perfects action quite as well as knowledge? 

Men. There again, Socrates, I think you are right. 

Soc. Then right opinion is not a whit inferior to knowledge, or less
useful in action; nor is the man who has right opinion inferior to
him who has knowledge? 

Men. True. 

Soc. And surely the good man has been acknowledged by us to be useful?

Men. Yes. 

Soc. Seeing then that men become good and useful to states, not only
because they have knowledge, but because they have right opinion,
and that neither knowledge nor right opinion is given to man by nature
or acquired by him-(do you imagine either of them to be given by nature?

Men. Not I.) 

Soc. Then if they are not given by nature, neither are the good by
nature good? 

Men. Certainly not. 

Soc. And nature being excluded, then came the question whether virtue
is acquired by teaching? 

Men. Yes. 

Soc. If virtue was wisdom [or knowledge], then, as we thought, it
was taught? 

Men. Yes. 

Soc. And if it was taught it was wisdom? 

Men. Certainly. 

Soc. And if there were teachers, it might be taught; and if there
were no teachers, not? 

Men. True. 

Soc. But surely we acknowledged that there were no teachers of virtue?

Men. Yes. 

Soc. Then we acknowledged that it was not taught, and was not wisdom?

Men. Certainly. 

Soc. And yet we admitted that it was a good? 

Men. Yes. 

Soc. And the right guide is useful and good? 

Men. Certainly. 

Soc. And the only right guides are knowledge and true opinion-these
are the guides of man; for things which happen by chance are not under
the guidance of man: but the guides of man are true opinion and knowledge.

Men. I think so too. 

Soc. But if virtue is not taught, neither is virtue knowledge.

Men. Clearly not. 

Soc. Then of two good and useful things, one, which is knowledge,
has been set aside, and cannot be supposed to be our guide in political
life. 

Men. I think not. 

Soc. And therefore not by any wisdom, and not because they were wise,
did Themistocles and those others of whom Anytus spoke govern states.
This was the reason why they were unable to make others like themselves-because
their virtue was not grounded on knowledge. 

Men. That is probably true, Socrates. 

Soc. But if not by knowledge, the only alternative which remains is
that statesmen must have guided states by right opinion, which is
in politics what divination is in religion; for diviners and also
prophets say many things truly, but they know not what they say.

Men. So I believe. 

Soc. And may we not, Meno, truly call those men "divine" who, having
no understanding, yet succeed in many a grand deed and word?

Men. Certainly. 

Soc. Then we shall also be right in calling divine those whom we were
just now speaking of as diviners and prophets, including the whole
tribe of poets. Yes, and statesmen above all may be said to be divine
and illumined, being inspired and possessed of God, in which condition
they say many grand things, not knowing what they say. 

Men. Yes. 

Soc. And the women too, Meno, call good men divine-do they not? and
the Spartans, when they praise a good man, say "that he is a divine
man." 

Men. And I think, Socrates, that they are right; although very likely
our friend Anytus may take offence at the word. 

Soc. I da not care; as for Anytus, there will be another opportunity
of talking with him. To sum up our enquiry-the result seems to be,
if we are at all right in our view, that virtue is neither natural
nor acquired, but an instinct given by God to the virtuous. Nor is
the instinct accompanied by reason, unless there may be supposed to
be among statesmen some one who is capable of educating statesmen.
And if there be such an one, he may be said to be among the living
what Homer says that Tiresias was among the dead, "he alone has understanding;
but the rest are flitting shades"; and he and his virtue in like manner
will be a reality among shadows. 

Men. That is excellent, Socrates. 

Soc. Then, Meno, the conclusion is that virtue comes to the virtuous
by the gift of God. But we shall never know the certain truth until,
before asking how virtue is given, we enquire into the actual nature
of virtue. I fear that I must go away, but do you, now that you are
persuaded yourself, persuade our friend Anytus. And do not let him
be so exasperated; if you can conciliate him, you will have done good
service to the Athenian people. 

THE END

% chapter meno (end)
% \chapter{Theaetetus} % (fold)
\label{cha:theaetetus}



THEAETETUS

By Plato


Translated by Benjamin Jowett




INTRODUCTION AND ANALYSIS.

Some dialogues of Plato are of so various a character that their
relation to the other dialogues cannot be determined with any degree of
certainty. The Theaetetus, like the Parmenides, has points of similarity
both with his earlier and his later writings. The perfection of style,
the humour, the dramatic interest, the complexity of structure, the
fertility of illustration, the shifting of the points of view, are
characteristic of his best period of authorship. The vain search, the
negative conclusion, the figure of the midwives, the constant profession
of ignorance on the part of Socrates, also bear the stamp of the early
dialogues, in which the original Socrates is not yet Platonized. Had we
no other indications, we should be disposed to range the Theaetetus with
the Apology and the Phaedrus, and perhaps even with the Protagoras and
the Laches.

But when we pass from the style to an examination of the subject,
we trace a connection with the later rather than with the earlier
dialogues. In the first place there is the connexion, indicated by Plato
himself at the end of the dialogue, with the Sophist, to which in
many respects the Theaetetus is so little akin. (1) The same persons
reappear, including the younger Socrates, whose name is just mentioned
in the Theaetetus; (2) the theory of rest, which Socrates has declined
to consider, is resumed by the Eleatic Stranger; (3) there is a similar
allusion in both dialogues to the meeting of Parmenides and Socrates
(Theaet., Soph.); and (4) the inquiry into not-being in the Sophist
supplements the question of false opinion which is raised in the
Theaetetus. (Compare also Theaet. and Soph. for parallel turns of
thought.) Secondly, the later date of the dialogue is confirmed by the
absence of the doctrine of recollection and of any doctrine of ideas
except that which derives them from generalization and from reflection
of the mind upon itself. The general character of the Theaetetus is
dialectical, and there are traces of the same Megarian influences which
appear in the Parmenides, and which later writers, in their matter of
fact way, have explained by the residence of Plato at Megara. Socrates
disclaims the character of a professional eristic, and also, with a sort
of ironical admiration, expresses his inability to attain the Megarian
precision in the use of terms. Yet he too employs a similar sophistical
skill in overturning every conceivable theory of knowledge.

The direct indications of a date amount to no more than this: the
conversation is said to have taken place when Theaetetus was a youth,
and shortly before the death of Socrates. At the time of his own death
he is supposed to be a full-grown man. Allowing nine or ten years for
the interval between youth and manhood, the dialogue could not have been
written earlier than 390, when Plato was about thirty-nine years of age.
No more definite date is indicated by the engagement in which Theaetetus
is said to have fallen or to have been wounded, and which may have taken
place any time during the Corinthian war, between the years 390-387.
The later date which has been suggested, 369, when the Athenians and
Lacedaemonians disputed the Isthmus with Epaminondas, would make the
age of Theaetetus at his death forty-five or forty-six. This a little
impairs the beauty of Socrates' remark, that 'he would be a great man if
he lived.'

In this uncertainty about the place of the Theaetetus, it seemed better,
as in the case of the Republic, Timaeus, Critias, to retain the order in
which Plato himself has arranged this and the two companion dialogues.
We cannot exclude the possibility which has been already noticed in
reference to other works of Plato, that the Theaetetus may not have
been all written continuously; or the probability that the Sophist and
Politicus, which differ greatly in style, were only appended after a
long interval of time. The allusion to Parmenides compared with the
Sophist, would probably imply that the dialogue which is called by his
name was already in existence; unless, indeed, we suppose the passage in
which the allusion occurs to have been inserted afterwards. Again,
the Theaetetus may be connected with the Gorgias, either dialogue from
different points of view containing an analysis of the real and apparent
(Schleiermacher); and both may be brought into relation with the Apology
as illustrating the personal life of Socrates. The Philebus, too, may
with equal reason be placed either after or before what, in the language
of Thrasyllus, may be called the Second Platonic Trilogy. Both the
Parmenides and the Sophist, and still more the Theaetetus, have points
of affinity with the Cratylus, in which the principles of rest and
motion are again contrasted, and the Sophistical or Protagorean theory
of language is opposed to that which is attributed to the disciple
of Heracleitus, not to speak of lesser resemblances in thought and
language. The Parmenides, again, has been thought by some to hold an
intermediate position between the Theaetetus and the Sophist; upon this
view, the Sophist may be regarded as the answer to the problems about
One and Being which have been raised in the Parmenides. Any of these
arrangements may suggest new views to the student of Plato; none of them
can lay claim to an exclusive probability in its favour.

The Theaetetus is one of the narrated dialogues of Plato, and is
the only one which is supposed to have been written down. In a short
introductory scene, Euclides and Terpsion are described as meeting
before the door of Euclides' house in Megara. This may have been a
spot familiar to Plato (for Megara was within a walk of Athens), but no
importance can be attached to the accidental introduction of the founder
of the Megarian philosophy. The real intention of the preface is to
create an interest about the person of Theaetetus, who has just been
carried up from the army at Corinth in a dying state. The expectation
of his death recalls the promise of his youth, and especially the famous
conversation which Socrates had with him when he was quite young, a few
days before his own trial and death, as we are once more reminded at the
end of the dialogue. Yet we may observe that Plato has himself forgotten
this, when he represents Euclides as from time to time coming to Athens
and correcting the copy from Socrates' own mouth. The narrative, having
introduced Theaetetus, and having guaranteed the authenticity of the
dialogue (compare Symposium, Phaedo, Parmenides), is then dropped. No
further use is made of the device. As Plato himself remarks, who in this
as in some other minute points is imitated by Cicero (De Amicitia), the
interlocutory words are omitted.

Theaetetus, the hero of the battle of Corinth and of the dialogue, is
a disciple of Theodorus, the great geometrician, whose science is thus
indicated to be the propaedeutic to philosophy. An interest has been
already excited about him by his approaching death, and now he is
introduced to us anew by the praises of his master Theodorus. He is a
youthful Socrates, and exhibits the same contrast of the fair soul and
the ungainly face and frame, the Silenus mask and the god within, which
are described in the Symposium. The picture which Theodorus gives of
his courage and patience and intelligence and modesty is verified in
the course of the dialogue. His courage is shown by his behaviour in the
battle, and his other qualities shine forth as the argument proceeds.
Socrates takes an evident delight in 'the wise Theaetetus,' who has more
in him than 'many bearded men'; he is quite inspired by his answers. At
first the youth is lost in wonder, and is almost too modest to speak,
but, encouraged by Socrates, he rises to the occasion, and grows full of
interest and enthusiasm about the great question. Like a youth, he has
not finally made up his mind, and is very ready to follow the lead of
Socrates, and to enter into each successive phase of the discussion
which turns up. His great dialectical talent is shown in his power of
drawing distinctions, and of foreseeing the consequences of his own
answers. The enquiry about the nature of knowledge is not new to him;
long ago he has felt the 'pang of philosophy,' and has experienced the
youthful intoxication which is depicted in the Philebus. But he
has hitherto been unable to make the transition from mathematics to
metaphysics. He can form a general conception of square and oblong
numbers, but he is unable to attain a similar expression of knowledge
in the abstract. Yet at length he begins to recognize that there are
universal conceptions of being, likeness, sameness, number, which the
mind contemplates in herself, and with the help of Socrates is conducted
from a theory of sense to a theory of ideas.

There is no reason to doubt that Theaetetus was a real person, whose
name survived in the next generation. But neither can any importance be
attached to the notices of him in Suidas and Proclus, which are probably
based on the mention of him in Plato. According to a confused statement
in Suidas, who mentions him twice over, first, as a pupil of Socrates,
and then of Plato, he is said to have written the first work on the Five
Solids. But no early authority cites the work, the invention of which
may have been easily suggested by the division of roots, which Plato
attributes to him, and the allusion to the backward state of solid
geometry in the Republic. At any rate, there is no occasion to recall
him to life again after the battle of Corinth, in order that we may
allow time for the completion of such a work (Muller). We may also
remark that such a supposition entirely destroys the pathetic interest
of the introduction.

Theodorus, the geometrician, had once been the friend and disciple of
Protagoras, but he is very reluctant to leave his retirement and defend
his old master. He is too old to learn Socrates' game of question and
answer, and prefers the digressions to the main argument, because he
finds them easier to follow. The mathematician, as Socrates says in the
Republic, is not capable of giving a reason in the same manner as the
dialectician, and Theodorus could not therefore have been appropriately
introduced as the chief respondent. But he may be fairly appealed to,
when the honour of his master is at stake. He is the 'guardian of his
orphans,' although this is a responsibility which he wishes to throw
upon Callias, the friend and patron of all Sophists, declaring that
he himself had early 'run away' from philosophy, and was absorbed in
mathematics. His extreme dislike to the Heraclitean fanatics, which may
be compared with the dislike of Theaetetus to the materialists, and his
ready acceptance of the noble words of Socrates, are noticeable traits
of character.

The Socrates of the Theaetetus is the same as the Socrates of the
earlier dialogues. He is the invincible disputant, now advanced in
years, of the Protagoras and Symposium; he is still pursuing his divine
mission, his 'Herculean labours,' of which he has described the origin
in the Apology; and he still hears the voice of his oracle, bidding him
receive or not receive the truant souls. There he is supposed to have
a mission to convict men of self-conceit; in the Theaetetus he has
assigned to him by God the functions of a man-midwife, who delivers men
of their thoughts, and under this character he is present throughout the
dialogue. He is the true prophet who has an insight into the natures
of men, and can divine their future; and he knows that sympathy is the
secret power which unlocks their thoughts. The hit at Aristides, the son
of Lysimachus, who was specially committed to his charge in the Laches,
may be remarked by the way. The attempt to discover the definition
of knowledge is in accordance with the character of Socrates as he
is described in the Memorabilia, asking What is justice? what is
temperance? and the like. But there is no reason to suppose that he
would have analyzed the nature of perception, or traced the connexion
of Protagoras and Heracleitus, or have raised the difficulty respecting
false opinion. The humorous illustrations, as well as the serious
thoughts, run through the dialogue. The snubnosedness of Theaetetus, a
characteristic which he shares with Socrates, and the man-midwifery
of Socrates, are not forgotten in the closing words. At the end of the
dialogue, as in the Euthyphro, he is expecting to meet Meletus at the
porch of the king Archon; but with the same indifference to the result
which is everywhere displayed by him, he proposes that they shall
reassemble on the following day at the same spot. The day comes, and
in the Sophist the three friends again meet, but no further allusion is
made to the trial, and the principal share in the argument is assigned,
not to Socrates, but to an Eleatic stranger; the youthful Theaetetus
also plays a different and less independent part. And there is no
allusion in the Introduction to the second and third dialogues, which
are afterwards appended. There seems, therefore, reason to think that
there is a real change, both in the characters and in the design.

The dialogue is an enquiry into the nature of knowledge, which is
interrupted by two digressions. The first is the digression about the
midwives, which is also a leading thought or continuous image, like the
wave in the Republic, appearing and reappearing at intervals. Again and
again we are reminded that the successive conceptions of knowledge are
extracted from Theaetetus, who in his turn truly declares that Socrates
has got a great deal more out of him than ever was in him. Socrates is
never weary of working out the image in humorous details,--discerning
the symptoms of labour, carrying the child round the hearth, fearing
that Theaetetus will bite him, comparing his conceptions to wind-eggs,
asserting an hereditary right to the occupation. There is also a serious
side to the image, which is an apt similitude of the Socratic theory
of education (compare Republic, Sophist), and accords with the ironical
spirit in which the wisest of men delights to speak of himself.

The other digression is the famous contrast of the lawyer and
philosopher. This is a sort of landing-place or break in the middle of
the dialogue. At the commencement of a great discussion, the reflection
naturally arises, How happy are they who, like the philosopher, have
time for such discussions (compare Republic)! There is no reason for the
introduction of such a digression; nor is a reason always needed, any
more than for the introduction of an episode in a poem, or of a topic in
conversation. That which is given by Socrates is quite sufficient, viz.
that the philosopher may talk and write as he pleases. But though not
very closely connected, neither is the digression out of keeping with
the rest of the dialogue. The philosopher naturally desires to pour
forth the thoughts which are always present to him, and to discourse of
the higher life. The idea of knowledge, although hard to be defined, is
realised in the life of philosophy. And the contrast is the favourite
antithesis between the world, in the various characters of sophist,
lawyer, statesman, speaker, and the philosopher,--between opinion and
knowledge,--between the conventional and the true.

The greater part of the dialogue is devoted to setting up and throwing
down definitions of science and knowledge. Proceeding from the lower to
the higher by three stages, in which perception, opinion, reasoning are
successively examined, we first get rid of the confusion of the idea of
knowledge and specific kinds of knowledge,--a confusion which has been
already noticed in the Lysis, Laches, Meno, and other dialogues. In
the infancy of logic, a form of thought has to be invented before the
content can be filled up. We cannot define knowledge until the nature of
definition has been ascertained. Having succeeded in making his meaning
plain, Socrates proceeds to analyze (1) the first definition which
Theaetetus proposes: 'Knowledge is sensible perception.' This is
speedily identified with the Protagorean saying, 'Man is the measure
of all things;' and of this again the foundation is discovered in the
perpetual flux of Heracleitus. The relativeness of sensation is then
developed at length, and for a moment the definition appears to be
accepted. But soon the Protagorean thesis is pronounced to be suicidal;
for the adversaries of Protagoras are as good a measure as he is, and
they deny his doctrine. He is then supposed to reply that the perception
may be true at any given instant. But the reply is in the end shown to
be inconsistent with the Heraclitean foundation, on which the doctrine
has been affirmed to rest. For if the Heraclitean flux is extended to
every sort of change in every instant of time, how can any thought
or word be detained even for an instant? Sensible perception, like
everything else, is tumbling to pieces. Nor can Protagoras himself
maintain that one man is as good as another in his knowledge of the
future; and 'the expedient,' if not 'the just and true,' belongs to the
sphere of the future.

And so we must ask again, What is knowledge? The comparison of
sensations with one another implies a principle which is above
sensation, and which resides in the mind itself. We are thus led to look
for knowledge in a higher sphere, and accordingly Theaetetus, when again
interrogated, replies (2) that 'knowledge is true opinion.' But how is
false opinion possible? The Megarian or Eristic spirit within us revives
the question, which has been already asked and indirectly answered in
the Meno: 'How can a man be ignorant of that which he knows?' No answer
is given to this not unanswerable question. The comparison of the mind
to a block of wax, or to a decoy of birds, is found wanting.

But are we not inverting the natural order in looking for opinion before
we have found knowledge? And knowledge is not true opinion; for the
Athenian dicasts have true opinion but not knowledge. What then
is knowledge? We answer (3), 'True opinion, with definition or
explanation.' But all the different ways in which this statement may be
understood are set aside, like the definitions of courage in the Laches,
or of friendship in the Lysis, or of temperance in the Charmides. At
length we arrive at the conclusion, in which nothing is concluded.

There are two special difficulties which beset the student of the
Theaetetus: (1) he is uncertain how far he can trust Plato's account of
the theory of Protagoras; and he is also uncertain (2) how far, and in
what parts of the dialogue, Plato is expressing his own opinion.
The dramatic character of the work renders the answer to both these
questions difficult.

1. In reply to the first, we have only probabilities to offer. Three
main points have to be decided: (a) Would Protagoras have identified
his own thesis, 'Man is the measure of all things,' with the other,
'All knowledge is sensible perception'? (b) Would he have based the
relativity of knowledge on the Heraclitean flux? (c) Would he have
asserted the absoluteness of sensation at each instant? Of the work of
Protagoras on 'Truth' we know nothing, with the exception of the two
famous fragments, which are cited in this dialogue, 'Man is the measure
of all things,' and, 'Whether there are gods or not, I cannot tell.' Nor
have we any other trustworthy evidence of the tenets of Protagoras, or
of the sense in which his words are used. For later writers, including
Aristotle in his Metaphysics, have mixed up the Protagoras of Plato, as
they have the Socrates of Plato, with the real person.

Returning then to the Theaetetus, as the only possible source from which
an answer to these questions can be obtained, we may remark, that Plato
had 'The Truth' of Protagoras before him, and frequently refers to the
book. He seems to say expressly, that in this work the doctrine of the
Heraclitean flux was not to be found; 'he told the real truth' (not
in the book, which is so entitled, but) 'privately to his
disciples,'--words which imply that the connexion between the doctrines
of Protagoras and Heracleitus was not generally recognized in Greece,
but was really discovered or invented by Plato. On the other hand,
the doctrine that 'Man is the measure of all things,' is expressly
identified by Socrates with the other statement, that 'What appears to
each man is to him;' and a reference is made to the books in which the
statement occurs;--this Theaetetus, who has 'often read the books,' is
supposed to acknowledge (so Cratylus). And Protagoras, in the speech
attributed to him, never says that he has been misunderstood: he rather
seems to imply that the absoluteness of sensation at each instant was
to be found in his words. He is only indignant at the 'reductio ad
absurdum' devised by Socrates for his 'homo mensura,' which Theodorus
also considers to be 'really too bad.'

The question may be raised, how far Plato in the Theaetetus could
have misrepresented Protagoras without violating the laws of dramatic
probability. Could he have pretended to cite from a well-known writing
what was not to be found there? But such a shadowy enquiry is not worth
pursuing further. We need only remember that in the criticism which
follows of the thesis of Protagoras, we are criticizing the Protagoras
of Plato, and not attempting to draw a precise line between his real
sentiments and those which Plato has attributed to him.

2. The other difficulty is a more subtle, and also a more important one,
because bearing on the general character of the Platonic dialogues. On
a first reading of them, we are apt to imagine that the truth is only
spoken by Socrates, who is never guilty of a fallacy himself, and is the
great detector of the errors and fallacies of others. But this natural
presumption is disturbed by the discovery that the Sophists are
sometimes in the right and Socrates in the wrong. Like the hero of a
novel, he is not to be supposed always to represent the sentiments
of the author. There are few modern readers who do not side with
Protagoras, rather than with Socrates, in the dialogue which is
called by his name. The Cratylus presents a similar difficulty: in
his etymologies, as in the number of the State, we cannot tell how
far Socrates is serious; for the Socratic irony will not allow him
to distinguish between his real and his assumed wisdom. No one is the
superior of the invincible Socrates in argument (except in the first
part of the Parmenides, where he is introduced as a youth); but he is by
no means supposed to be in possession of the whole truth. Arguments are
often put into his mouth (compare Introduction to the Gorgias) which
must have seemed quite as untenable to Plato as to a modern writer.
In this dialogue a great part of the answer of Protagoras is just
and sound; remarks are made by him on verbal criticism, and on the
importance of understanding an opponent's meaning, which are conceived
in the true spirit of philosophy. And the distinction which he is
supposed to draw between Eristic and Dialectic, is really a criticism of
Plato on himself and his own criticism of Protagoras.

The difficulty seems to arise from not attending to the dramatic
character of the writings of Plato. There are two, or more, sides to
questions; and these are parted among the different speakers. Sometimes
one view or aspect of a question is made to predominate over the rest,
as in the Gorgias or Sophist; but in other dialogues truth is divided,
as in the Laches and Protagoras, and the interest of the piece consists
in the contrast of opinions. The confusion caused by the irony of
Socrates, who, if he is true to his character, cannot say anything
of his own knowledge, is increased by the circumstance that in the
Theaetetus and some other dialogues he is occasionally playing both
parts himself, and even charging his own arguments with unfairness. In
the Theaetetus he is designedly held back from arriving at a conclusion.
For we cannot suppose that Plato conceived a definition of knowledge to
be impossible. But this is his manner of approaching and surrounding a
question. The lights which he throws on his subject are indirect, but
they are not the less real for that. He has no intention of proving a
thesis by a cut-and-dried argument; nor does he imagine that a great
philosophical problem can be tied up within the limits of a definition.
If he has analyzed a proposition or notion, even with the severity of
an impossible logic, if half-truths have been compared by him with
other half-truths, if he has cleared up or advanced popular ideas, or
illustrated a new method, his aim has been sufficiently accomplished.

The writings of Plato belong to an age in which the power of analysis
had outrun the means of knowledge; and through a spurious use of
dialectic, the distinctions which had been already 'won from the void
and formless infinite,' seemed to be rapidly returning to their original
chaos. The two great speculative philosophies, which a century earlier
had so deeply impressed the mind of Hellas, were now degenerating into
Eristic. The contemporaries of Plato and Socrates were vainly trying to
find new combinations of them, or to transfer them from the object to
the subject. The Megarians, in their first attempts to attain a severer
logic, were making knowledge impossible (compare Theaet.). They were
asserting 'the one good under many names,' and, like the Cynics, seem
to have denied predication, while the Cynics themselves were depriving
virtue of all which made virtue desirable in the eyes of Socrates and
Plato. And besides these, we find mention in the later writings of
Plato, especially in the Theaetetus, Sophist, and Laws, of certain
impenetrable godless persons, who will not believe what they 'cannot
hold in their hands'; and cannot be approached in argument, because they
cannot argue (Theat; Soph.). No school of Greek philosophers exactly
answers to these persons, in whom Plato may perhaps have blended some
features of the Atomists with the vulgar materialistic tendencies of
mankind in general (compare Introduction to the Sophist).

And not only was there a conflict of opinions, but the stage which the
mind had reached presented other difficulties hardly intelligible to
us, who live in a different cycle of human thought. All times of mental
progress are times of confusion; we only see, or rather seem to see
things clearly, when they have been long fixed and defined. In the
age of Plato, the limits of the world of imagination and of pure
abstraction, of the old world and the new, were not yet fixed. The
Greeks, in the fourth century before Christ, had no words for 'subject'
and 'object,' and no distinct conception of them; yet they were always
hovering about the question involved in them. The analysis of sense, and
the analysis of thought, were equally difficult to them; and hopelessly
confused by the attempt to solve them, not through an appeal to facts,
but by the help of general theories respecting the nature of the
universe.

Plato, in his Theaetetus, gathers up the sceptical tendencies of his
age, and compares them. But he does not seek to reconstruct out of them
a theory of knowledge. The time at which such a theory could be framed
had not yet arrived. For there was no measure of experience with which
the ideas swarming in men's minds could be compared; the meaning of
the word 'science' could scarcely be explained to them, except from the
mathematical sciences, which alone offered the type of universality and
certainty. Philosophy was becoming more and more vacant and abstract,
and not only the Platonic Ideas and the Eleatic Being, but all
abstractions seemed to be at variance with sense and at war with one
another.

The want of the Greek mind in the fourth century before Christ was
not another theory of rest or motion, or Being or atoms, but rather a
philosophy which could free the mind from the power of abstractions
and alternatives, and show how far rest and how far motion, how far the
universal principle of Being and the multitudinous principle of atoms,
entered into the composition of the world; which could distinguish
between the true and false analogy, and allow the negative as well as
the positive a place in human thought. To such a philosophy Plato, in
the Theaetetus, offers many contributions. He has followed philosophy
into the region of mythology, and pointed out the similarities of
opposing phases of thought. He has also shown that extreme abstractions
are self-destructive, and, indeed, hardly distinguishable from one
another. But his intention is not to unravel the whole subject of
knowledge, if this had been possible; and several times in the course
of the dialogue he rejects explanations of knowledge which have germs of
truth in them; as, for example, 'the resolution of the compound into the
simple;' or 'right opinion with a mark of difference.'

...

Terpsion, who has come to Megara from the country, is described as
having looked in vain for Euclides in the Agora; the latter explains
that he has been down to the harbour, and on his way thither had met
Theaetetus, who was being carried up from the army to Athens. He was
scarcely alive, for he had been badly wounded at the battle of Corinth,
and had taken the dysentery which prevailed in the camp. The mention of
his condition suggests the reflection, 'What a loss he will be!' 'Yes,
indeed,' replies Euclid; 'only just now I was hearing of his noble
conduct in the battle.' 'That I should expect; but why did he not remain
at Megara?' 'I wanted him to remain, but he would not; so I went with
him as far as Erineum; and as I parted from him, I remembered that
Socrates had seen him when he was a youth, and had a remarkable
conversation with him, not long before his own death; and he then
prophesied of him that he would be a great man if he lived.' 'How true
that has been; how like all that Socrates said! And could you repeat the
conversation?' 'Not from memory; but I took notes when I returned home,
which I afterwards filled up at leisure, and got Socrates to correct
them from time to time, when I came to Athens'...Terpsion had long
intended to ask for a sight of this writing, of which he had already
heard. They are both tired, and agree to rest and have the conversation
read to them by a servant...'Here is the roll, Terpsion; I need
only observe that I have omitted, for the sake of convenience, the
interlocutory words, "said I," "said he"; and that Theaetetus, and
Theodorus, the geometrician of Cyrene, are the persons with whom
Socrates is conversing.'

Socrates begins by asking Theodorus whether, in his visit to Athens, he
has found any Athenian youth likely to attain distinction in science.
'Yes, Socrates, there is one very remarkable youth, with whom I have
become acquainted. He is no beauty, and therefore you need not imagine
that I am in love with him; and, to say the truth, he is very like you,
for he has a snub nose, and projecting eyes, although these features are
not so marked in him as in you. He combines the most various qualities,
quickness, patience, courage; and he is gentle as well as wise, always
silently flowing on, like a river of oil. Look! he is the middle one of
those who are entering the palaestra.'

Socrates, who does not know his name, recognizes him as the son of
Euphronius, who was himself a good man and a rich. He is informed by
Theodorus that the youth is named Theaetetus, but the property of his
father has disappeared in the hands of trustees; this does not, however,
prevent him from adding liberality to his other virtues. At the desire
of Socrates he invites Theaetetus to sit by them.

'Yes,' says Socrates, 'that I may see in you, Theaetetus, the image
of my ugly self, as Theodorus declares. Not that his remark is of any
importance; for though he is a philosopher, he is not a painter, and
therefore he is no judge of our faces; but, as he is a man of science,
he may be a judge of our intellects. And if he were to praise the mental
endowments of either of us, in that case the hearer of the eulogy ought
to examine into what he says, and the subject should not refuse to be
examined.' Theaetetus consents, and is caught in a trap (compare the
similar trap which is laid for Theodorus). 'Then, Theaetetus, you will
have to be examined, for Theodorus has been praising you in a style of
which I never heard the like.' 'He was only jesting.' 'Nay, that is not
his way; and I cannot allow you, on that pretence, to retract the assent
which you have already given, or I shall make Theodorus repeat your
praises, and swear to them.' Theaetetus, in reply, professes that he is
willing to be examined, and Socrates begins by asking him what he learns
of Theodorus. He is himself anxious to learn anything of anybody; and
now he has a little question to which he wants Theaetetus or Theodorus
(or whichever of the company would not be 'donkey' to the rest) to find
an answer. Without further preface, but at the same time apologizing
for his eagerness, he asks, 'What is knowledge?' Theodorus is too old
to answer questions, and begs him to interrogate Theaetetus, who has the
advantage of youth.

Theaetetus replies, that knowledge is what he learns of Theodorus,
i.e. geometry and arithmetic; and that there are other kinds of
knowledge--shoemaking, carpentering, and the like. But Socrates rejoins,
that this answer contains too much and also too little. For although
Theaetetus has enumerated several kinds of knowledge, he has not
explained the common nature of them; as if he had been asked, 'What is
clay?' and instead of saying 'Clay is moistened earth,' he had answered,
'There is one clay of image-makers, another of potters, another of
oven-makers.' Theaetetus at once divines that Socrates means him to
extend to all kinds of knowledge the same process of generalization
which he has already learned to apply to arithmetic. For he has
discovered a division of numbers into square numbers, 4, 9, 16, etc.,
which are composed of equal factors, and represent figures which have
equal sides, and oblong numbers, 3, 5, 6, 7, etc., which are composed of
unequal factors, and represent figures which have unequal sides. But
he has never succeeded in attaining a similar conception of knowledge,
though he has often tried; and, when this and similar questions were
brought to him from Socrates, has been sorely distressed by them.
Socrates explains to him that he is in labour. For men as well as
women have pangs of labour; and both at times require the assistance of
midwives. And he, Socrates, is a midwife, although this is a secret; he
has inherited the art from his mother bold and bluff, and he ushers into
light, not children, but the thoughts of men. Like the midwives, who are
'past bearing children,' he too can have no offspring--the God will not
allow him to bring anything into the world of his own. He also reminds
Theaetetus that the midwives are or ought to be the only matchmakers
(this is the preparation for a biting jest); for those who reap the
fruit are most likely to know on what soil the plants will grow. But
respectable midwives avoid this department of practice--they do not want
to be called procuresses. There are some other differences between the
two sorts of pregnancy. For women do not bring into the world at one
time real children and at another time idols which are with difficulty
distinguished from them. 'At first,' says Socrates in his character of
the man-midwife, 'my patients are barren and stolid, but after a while
they "round apace," if the gods are propitious to them; and this is due
not to me but to themselves; I and the god only assist in bringing their
ideas to the birth. Many of them have left me too soon, and the result
has been that they have produced abortions; or when I have delivered
them of children they have lost them by an ill bringing up, and have
ended by seeing themselves, as others see them, to be great fools.
Aristides, the son of Lysimachus, is one of these, and there have been
others. The truants often return to me and beg to be taken back; and
then, if my familiar allows me, which is not always the case, I receive
them, and they begin to grow again. There come to me also those who have
nothing in them, and have no need of my art; and I am their matchmaker
(see above), and marry them to Prodicus or some other inspired sage who
is likely to suit them. I tell you this long story because I suspect
that you are in labour. Come then to me, who am a midwife, and the son
of a midwife, and I will deliver you. And do not bite me, as the women
do, if I abstract your first-born; for I am acting out of good-will
towards you; the God who is within me is the friend of man, though he
will not allow me to dissemble the truth. Once more then, Theaetetus,
I repeat my old question--"What is knowledge?" Take courage, and by the
help of God you will discover an answer.' 'My answer is, that knowledge
is perception.' 'That is the theory of Protagoras, who has another way
of expressing the same thing when he says, "Man is the measure of all
things." He was a very wise man, and we should try to understand him.
In order to illustrate his meaning let me suppose that there is the same
wind blowing in our faces, and one of us may be hot and the other cold.
How is this? Protagoras will reply that the wind is hot to him who is
cold, cold to him who is hot. And "is" means "appears," and when you
say "appears to him," that means "he feels." Thus feeling, appearance,
perception, coincide with being. I suspect, however, that this was only
a "facon de parler," by which he imposed on the common herd like you and
me; he told "the truth" (in allusion to the title of his book, which
was called "The Truth") in secret to his disciples. For he was really
a votary of that famous philosophy in which all things are said to be
relative; nothing is great or small, or heavy or light, or one, but all
is in motion and mixture and transition and flux and generation, not
"being," as we ignorantly affirm, but "becoming." This has been the
doctrine, not of Protagoras only, but of all philosophers, with the
single exception of Parmenides; Empedocles, Heracleitus, and others, and
all the poets, with Epicharmus, the king of Comedy, and Homer, the king
of Tragedy, at their head, have said the same; the latter has these
words--

"Ocean, whence the gods sprang, and mother Tethys."

And many arguments are used to show, that motion is the source of life,
and rest of death: fire and warmth are produced by friction, and living
creatures owe their origin to a similar cause; the bodily frame is
preserved by exercise and destroyed by indolence; and if the sun ceased
to move, "chaos would come again." Now apply this doctrine of "All
is motion" to the senses, and first of all to the sense of sight. The
colour of white, or any other colour, is neither in the eyes nor out of
them, but ever in motion between the object and the eye, and varying in
the case of every percipient. All is relative, and, as the followers of
Protagoras remark, endless contradictions arise when we deny this; e.g.
here are six dice; they are more than four and less than twelve; "more
and also less," would you not say?' 'Yes.' 'But Protagoras will retort:
"Can anything be more or less without addition or subtraction?"'

'I should say "No" if I were not afraid of contradicting my former
answer.'

'And if you say "Yes," the tongue will escape conviction but not the
mind, as Euripides would say?' 'True.' 'The thoroughbred Sophists, who
know all that can be known, would have a sparring match over this, but
you and I, who have no professional pride, want only to discover whether
our ideas are clear and consistent. And we cannot be wrong in saying,
first, that nothing can be greater or less while remaining equal;
secondly, that there can be no becoming greater or less without addition
or subtraction; thirdly, that what is and was not, cannot be without
having become. But then how is this reconcilable with the case of the
dice, and with similar examples?--that is the question.' 'I am often
perplexed and amazed, Socrates, by these difficulties.' 'That is because
you are a philosopher, for philosophy begins in wonder, and Iris is
the child of Thaumas. Do you know the original principle on which the
doctrine of Protagoras is based?' 'No.' 'Then I will tell you; but we
must not let the uninitiated hear, and by the uninitiated I mean the
obstinate people who believe in nothing which they cannot hold in their
hands. The brethren whose mysteries I am about to unfold to you are far
more ingenious. They maintain that all is motion; and that motion
has two forms, action and passion, out of which endless phenomena are
created, also in two forms--sense and the object of sense--which come to
the birth together. There are two kinds of motions, a slow and a fast;
the motions of the agent and the patient are slower, because they move
and create in and about themselves, but the things which are born of
them have a swifter motion, and pass rapidly from place to place.
The eye and the appropriate object come together, and give birth to
whiteness and the sensation of whiteness; the eye is filled with seeing,
and becomes not sight but a seeing eye, and the object is filled with
whiteness, and becomes not whiteness but white; and no other compound of
either with another would have produced the same effect. All sensation
is to be resolved into a similar combination of an agent and patient.
Of either, taken separately, no idea can be formed; and the agent may
become a patient, and the patient an agent. Hence there arises a general
reflection that nothing is, but all things become; no name can detain or
fix them. Are not these speculations charming, Theaetetus, and very good
for a person in your interesting situation? I am offering you specimens
of other men's wisdom, because I have no wisdom of my own, and I want
to deliver you of something; and presently we will see whether you
have brought forth wind or not. Tell me, then, what do you think of the
notion that "All things are becoming"?'

'When I hear your arguments, I am marvellously ready to assent.'

'But I ought not to conceal from you that there is a serious objection
which may be urged against this doctrine of Protagoras. For there are
states, such as madness and dreaming, in which perception is false; and
half our life is spent in dreaming; and who can say that at this instant
we are not dreaming? Even the fancies of madmen are real at the time.
But if knowledge is perception, how can we distinguish between the true
and the false in such cases? Having stated the objection, I will now
state the answer. Protagoras would deny the continuity of phenomena;
he would say that what is different is entirely different, and whether
active or passive has a different power. There are infinite agents and
patients in the world, and these produce in every combination of them a
different perception. Take myself as an instance:--Socrates may be ill
or he may be well,--and remember that Socrates, with all his accidents,
is spoken of. The wine which I drink when I am well is pleasant to
me, but the same wine is unpleasant to me when I am ill. And there
is nothing else from which I can receive the same impression, nor can
another receive the same impression from the wine. Neither can I and the
object of sense become separately what we become together. For the one
in becoming is relative to the other, but they have no other relation;
and the combination of them is absolute at each moment. (In modern
language, the act of sensation is really indivisible, though capable of
a mental analysis into subject and object.) My sensation alone is true,
and true to me only. And therefore, as Protagoras says, "To myself I
am the judge of what is and what is not." Thus the flux of Homer and
Heracleitus, the great Protagorean saying that "Man is the measure of
all things," the doctrine of Theaetetus that "Knowledge is perception,"
have all the same meaning. And this is thy new-born child, which by my
art I have brought to light; and you must not be angry if instead of
rearing your infant we expose him.'

'Theaetetus will not be angry,' says Theodorus; 'he is very
good-natured. But I should like to know, Socrates, whether you mean to
say that all this is untrue?'

'First reminding you that I am not the bag which contains the arguments,
but that I extract them from Theaetetus, shall I tell you what amazes me
in your friend Protagoras?'

'What may that be?'

'I like his doctrine that what appears is; but I wonder that he did
not begin his great work on Truth with a declaration that a pig, or a
dog-faced baboon, or any other monster which has sensation, is a measure
of all things; then, while we were reverencing him as a god, he might
have produced a magnificent effect by expounding to us that he was no
wiser than a tadpole. For if sensations are always true, and one man's
discernment is as good as another's, and every man is his own judge,
and everything that he judges is right and true, then what need of
Protagoras to be our instructor at a high figure; and why should we
be less knowing than he is, or have to go to him, if every man is the
measure of all things? My own art of midwifery, and all dialectic, is
an enormous folly, if Protagoras' "Truth" be indeed truth, and the
philosopher is not merely amusing himself by giving oracles out of his
book.'

Theodorus thinks that Socrates is unjust to his master, Protagoras; but
he is too old and stiff to try a fall with him, and therefore refers him
to Theaetetus, who is already driven out of his former opinion by the
arguments of Socrates.

Socrates then takes up the defence of Protagoras, who is supposed to
reply in his own person--'Good people, you sit and declaim about the
gods, of whose existence or non-existence I have nothing to say, or you
discourse about man being reduced to the level of the brutes; but what
proof have you of your statements? And yet surely you and Theodorus had
better reflect whether probability is a safe guide. Theodorus would be
a bad geometrician if he had nothing better to offer.'...Theaetetus is
affected by the appeal to geometry, and Socrates is induced by him to
put the question in a new form. He proceeds as follows:--'Should we say
that we know what we see and hear,--e.g. the sound of words or the sight
of letters in a foreign tongue?'

'We should say that the figures of the letters, and the pitch of the
voice in uttering them, were known to us, but not the meaning of them.'

'Excellent; I want you to grow, and therefore I will leave that answer
and ask another question: Is not seeing perceiving?' 'Very true.' 'And
he who sees knows?' 'Yes.' 'And he who remembers, remembers that which
he sees and knows?' 'Very true.' 'But if he closes his eyes, does he not
remember?' 'He does.' 'Then he may remember and not see; and if seeing
is knowing, he may remember and not know. Is not this a "reductio ad
absurdum" of the hypothesis that knowledge is sensible perception? Yet
perhaps we are crowing too soon; and if Protagoras, "the father of the
myth," had been alive, the result might have been very different. But he
is dead, and Theodorus, whom he left guardian of his "orphan," has not
been very zealous in defending him.'

Theodorus objects that Callias is the true guardian, but he hopes that
Socrates will come to the rescue. Socrates prefaces his defence by
resuming the attack. He asks whether a man can know and not know at the
same time? 'Impossible.' Quite possible, if you maintain that seeing is
knowing. The confident adversary, suiting the action to the word, shuts
one of your eyes; and now, says he, you see and do not see, but do
you know and not know? And a fresh opponent darts from his ambush, and
transfers to knowledge the terms which are commonly applied to sight.
He asks whether you can know near and not at a distance; whether you can
have a sharp and also a dull knowledge. While you are wondering at his
incomparable wisdom, he gets you into his power, and you will not escape
until you have come to an understanding with him about the money which
is to be paid for your release.

But Protagoras has not yet made his defence; and already he may be heard
contemptuously replying that he is not responsible for the admissions
which were made by a boy, who could not foresee the coming move, and
therefore had answered in a manner which enabled Socrates to raise a
laugh against himself. 'But I cannot be fairly charged,' he will say,
'with an answer which I should not have given; for I never maintained
that the memory of a feeling is the same as a feeling, or denied that a
man might know and not know the same thing at the same time. Or, if you
will have extreme precision, I say that man in different relations is
many or rather infinite in number. And I challenge you, either to show
that his perceptions are not individual, or that if they are, what
appears to him is not what is. As to your pigs and baboons, you are
yourself a pig, and you make my writings a sport of other swine. But
I still affirm that man is the measure of all things, although I admit
that one man may be a thousand times better than another, in proportion
as he has better impressions. Neither do I deny the existence of wisdom
or of the wise man. But I maintain that wisdom is a practical remedial
power of turning evil into good, the bitterness of disease into the
sweetness of health, and does not consist in any greater truth or
superior knowledge. For the impressions of the sick are as true as the
impressions of the healthy; and the sick are as wise as the healthy. Nor
can any man be cured of a false opinion, for there is no such thing;
but he may be cured of the evil habit which generates in him an evil
opinion. This is effected in the body by the drugs of the physician, and
in the soul by the words of the Sophist; and the new state or opinion
is not truer, but only better than the old. And philosophers are not
tadpoles, but physicians and husbandmen, who till the soil and infuse
health into animals and plants, and make the good take the place of the
evil, both in individuals and states. Wise and good rhetoricians make
the good to appear just in states (for that is just which appears just
to a state), and in return, they deserve to be well paid. And you,
Socrates, whether you please or not, must continue to be a measure.
This is my defence, and I must request you to meet me fairly. We are
professing to reason, and not merely to dispute; and there is a great
difference between reasoning and disputation. For the disputer is always
seeking to trip up his opponent; and this is a mode of argument which
disgusts men with philosophy as they grow older. But the reasoner is
trying to understand him and to point out his errors to him, whether
arising from his own or from his companion's fault; he does not argue
from the customary use of names, which the vulgar pervert in all manner
of ways. If you are gentle to an adversary he will follow and love you;
and if defeated he will lay the blame on himself, and seek to escape
from his own prejudices into philosophy. I would recommend you,
Socrates, to adopt this humaner method, and to avoid captious and verbal
criticisms.'

Such, Theodorus, is the very slight help which I am able to afford to
your friend; had he been alive, he would have helped himself in far
better style.

'You have made a most valorous defence.'

Yes; but did you observe that Protagoras bade me be serious, and
complained of our getting up a laugh against him with the aid of a boy?
He meant to intimate that you must take the place of Theaetetus, who may
be wiser than many bearded men, but not wiser than you, Theodorus.

'The rule of the Spartan Palaestra is, Strip or depart; but you are like
the giant Antaeus, and will not let me depart unless I try a fall with
you.'

Yes, that is the nature of my complaint. And many a Hercules, many a
Theseus mighty in deeds and words has broken my head; but I am always at
this rough game. Please, then, to favour me.

'On the condition of not exceeding a single fall, I consent.'

Socrates now resumes the argument. As he is very desirous of doing
justice to Protagoras, he insists on citing his own words,--'What
appears to each man is to him.' And how, asks Socrates, are these words
reconcileable with the fact that all mankind are agreed in thinking
themselves wiser than others in some respects, and inferior to them in
others? In the hour of danger they are ready to fall down and worship
any one who is their superior in wisdom as if he were a god. And the
world is full of men who are asking to be taught and willing to be
ruled, and of other men who are willing to rule and teach them. All
which implies that men do judge of one another's impressions, and think
some wise and others foolish. How will Protagoras answer this argument?
For he cannot say that no one deems another ignorant or mistaken. If you
form a judgment, thousands and tens of thousands are ready to maintain
the opposite. The multitude may not and do not agree in Protagoras'
own thesis that 'Man is the measure of all things;' and then who is to
decide? Upon his own showing must not his 'truth' depend on the number
of suffrages, and be more or less true in proportion as he has more or
fewer of them? And he must acknowledge further, that they speak truly
who deny him to speak truly, which is a famous jest. And if he admits
that they speak truly who deny him to speak truly, he must admit that
he himself does not speak truly. But his opponents will refuse to admit
this of themselves, and he must allow that they are right in their
refusal. The conclusion is, that all mankind, including Protagoras
himself, will deny that he speaks truly; and his truth will be true
neither to himself nor to anybody else.

Theodorus is inclined to think that this is going too far. Socrates
ironically replies, that he is not going beyond the truth. But if the
old Protagoras could only pop his head out of the world below, he would
doubtless give them both a sound castigation and be off to the shades
in an instant. Seeing that he is not within call, we must examine the
question for ourselves. It is clear that there are great differences in
the understandings of men. Admitting, with Protagoras, that immediate
sensations of hot, cold, and the like, are to each one such as they
appear, yet this hypothesis cannot be extended to judgments or opinions.
And even if we were to admit further,--and this is the view of some who
are not thorough-going followers of Protagoras,--that right and wrong,
holy and unholy, are to each state or individual such as they appear,
still Protagoras will not venture to maintain that every man is equally
the measure of expediency, or that the thing which seems is expedient
to every one. But this begins a new question. 'Well, Socrates, we have
plenty of leisure. Yes, we have, and, after the manner of philosophers,
we are digressing; I have often observed how ridiculous this habit of
theirs makes them when they appear in court. 'What do you mean?' I mean
to say that a philosopher is a gentleman, but a lawyer is a servant.
The one can have his talk out, and wander at will from one subject
to another, as the fancy takes him; like ourselves, he may be long or
short, as he pleases. But the lawyer is always in a hurry; there is the
clepsydra limiting his time, and the brief limiting his topics, and his
adversary is standing over him and exacting his rights. He is a servant
disputing about a fellow-servant before his master, who holds the cause
in his hands; the path never diverges, and often the race is for his
life. Such experiences render him keen and shrewd; he learns the arts of
flattery, and is perfect in the practice of crooked ways; dangers have
come upon him too soon, when the tenderness of youth was unable to meet
them with truth and honesty, and he has resorted to counter-acts of
dishonesty and falsehood, and become warped and distorted; without any
health or freedom or sincerity in him he has grown up to manhood, and is
or esteems himself to be a master of cunning. Such are the lawyers; will
you have the companion picture of philosophers? or will this be too much
of a digression?

'Nay, Socrates, the argument is our servant, and not our master. Who is
the judge or where is the spectator, having a right to control us?'

I will describe the leaders, then: for the inferior sort are not worth
the trouble. The lords of philosophy have not learned the way to the
dicastery or ecclesia; they neither see nor hear the laws and votes of
the state, written or recited; societies, whether political or festive,
clubs, and singing maidens do not enter even into their dreams. And the
scandals of persons or their ancestors, male and female, they know no
more than they can tell the number of pints in the ocean. Neither
are they conscious of their own ignorance; for they do not practise
singularity in order to gain reputation, but the truth is, that the
outer form of them only is residing in the city; the inner man, as
Pindar says, is going on a voyage of discovery, measuring as with line
and rule the things which are under and in the earth, interrogating the
whole of nature, only not condescending to notice what is near them.

'What do you mean, Socrates?'

I will illustrate my meaning by the jest of the witty maid-servant, who
saw Thales tumbling into a well, and said of him, that he was so eager
to know what was going on in heaven, that he could not see what was
before his feet. This is applicable to all philosophers. The philosopher
is unacquainted with the world; he hardly knows whether his neighbour is
a man or an animal. For he is always searching into the essence of man,
and enquiring what such a nature ought to do or suffer different from
any other. Hence, on every occasion in private life and public, as I was
saying, when he appears in a law-court or anywhere, he is the joke, not
only of maid-servants, but of the general herd, falling into wells
and every sort of disaster; he looks such an awkward, inexperienced
creature, unable to say anything personal, when he is abused, in answer
to his adversaries (for he knows no evil of any one); and when he hears
the praises of others, he cannot help laughing from the bottom of
his soul at their pretensions; and this also gives him a ridiculous
appearance. A king or tyrant appears to him to be a kind of swine-herd
or cow-herd, milking away at an animal who is much more troublesome and
dangerous than cows or sheep; like the cow-herd, he has no time to be
educated, and the pen in which he keeps his flock in the mountains is
surrounded by a wall. When he hears of large landed properties of ten
thousand acres or more, he thinks of the whole earth; or if he is
told of the antiquity of a family, he remembers that every one has had
myriads of progenitors, rich and poor, Greeks and barbarians, kings
and slaves. And he who boasts of his descent from Amphitryon in the
twenty-fifth generation, may, if he pleases, add as many more, and
double that again, and our philosopher only laughs at his inability to
do a larger sum. Such is the man at whom the vulgar scoff; he seems to
them as if he could not mind his feet. 'That is very true, Socrates.'
But when he tries to draw the quick-witted lawyer out of his pleas and
rejoinders to the contemplation of absolute justice or injustice in
their own nature, or from the popular praises of wealthy kings to the
view of happiness and misery in themselves, or to the reasons why a man
should seek after the one and avoid the other, then the situation is
reversed; the little wretch turns giddy, and is ready to fall over the
precipice; his utterance becomes thick, and he makes himself ridiculous,
not to servant-maids, but to every man of liberal education. Such are
the two pictures: the one of the philosopher and gentleman, who may be
excused for not having learned how to make a bed, or cook up flatteries;
the other, a serviceable knave, who hardly knows how to wear his
cloak,--still less can he awaken harmonious thoughts or hymn virtue's
praises.

'If the world, Socrates, were as ready to receive your words as I am,
there would be greater peace and less evil among mankind.'

Evil, Theodorus, must ever remain in this world to be the antagonist of
good, out of the way of the gods in heaven. Wherefore also we should fly
away from ourselves to them; and to fly to them is to become like them;
and to become like them is to become holy, just and true. But many
live in the old wives' fable of appearances; they think that you should
follow virtue in order that you may seem to be good. And yet the truth
is, that God is righteous; and of men, he is most like him who is most
righteous. To know this is wisdom; and in comparison of this the wisdom
of the arts or the seeming wisdom of politicians is mean and common. The
unrighteous man is apt to pride himself on his cunning; when others
call him rogue, he says to himself: 'They only mean that I am one who
deserves to live, and not a mere burden of the earth.' But he should
reflect that his ignorance makes his condition worse than if he knew.
For the penalty of injustice is not death or stripes, but the fatal
necessity of becoming more and more unjust. Two patterns of life are set
before him; the one blessed and divine, the other godless and wretched;
and he is growing more and more like the one and unlike the other. He
does not see that if he continues in his cunning, the place of innocence
will not receive him after death. And yet if such a man has the courage
to hear the argument out, he often becomes dissatisfied with himself,
and has no more strength in him than a child.--But we have digressed
enough.

'For my part, Socrates, I like the digressions better than the argument,
because I understand them better.'

To return. When we left off, the Protagoreans and Heracliteans were
maintaining that the ordinances of the State were just, while they
lasted. But no one would maintain that the laws of the State were always
good or expedient, although this may be the intention of them. For
the expedient has to do with the future, about which we are liable to
mistake. Now, would Protagoras maintain that man is the measure not
only of the present and past, but of the future; and that there is no
difference in the judgments of men about the future? Would an untrained
man, for example, be as likely to know when he is going to have a fever,
as the physician who attended him? And if they differ in opinion,
which of them is likely to be right; or are they both right? Is not a
vine-grower a better judge of a vintage which is not yet gathered, or a
cook of a dinner which is in preparation, or Protagoras of the probable
effect of a speech than an ordinary person? The last example speaks 'ad
hominen.' For Protagoras would never have amassed a fortune if every man
could judge of the future for himself. He is, therefore, compelled to
admit that he is a measure; but I, who know nothing, am not equally
convinced that I am. This is one way of refuting him; and he is refuted
also by the authority which he attributes to the opinions of others, who
deny his opinions. I am not equally sure that we can disprove the truth
of immediate states of feeling. But this leads us to the doctrine of
the universal flux, about which a battle-royal is always going on in the
cities of Ionia. 'Yes; the Ephesians are downright mad about the
flux; they cannot stop to argue with you, but are in perpetual motion,
obedient to their text-books. Their restlessness is beyond expression,
and if you ask any of them a question, they will not answer, but dart at
you some unintelligible saying, and another and another, making no way
either with themselves or with others; for nothing is fixed in them
or their ideas,--they are at war with fixed principles.' I suppose,
Theodorus, that you have never seen them in time of peace, when they
discourse at leisure to their disciples? 'Disciples! they have none;
they are a set of uneducated fanatics, and each of them says of the
other that they have no knowledge. We must trust to ourselves, and not
to them for the solution of the problem.' Well, the doctrine is old,
being derived from the poets, who speak in a figure of Oceanus and
Tethys; the truth was once concealed, but is now revealed by the
superior wisdom of a later generation, and made intelligible to the
cobbler, who, on hearing that all is in motion, and not some things
only, as he ignorantly fancied, may be expected to fall down and worship
his teachers. And the opposite doctrine must not be forgotten:--

     'Alone being remains unmoved which is the name for all,'

as Parmenides affirms. Thus we are in the midst of the fray; both
parties are dragging us to their side; and we are not certain which of
them are in the right; and if neither, then we shall be in a ridiculous
position, having to set up our own opinion against ancient and famous
men.

Let us first approach the river-gods, or patrons of the flux.

When they speak of motion, must they not include two kinds of motion,
change of place and change of nature?--And all things must be supposed
to have both kinds of motion; for if not, the same things would be at
rest and in motion, which is contrary to their theory. And did we not
say, that all sensations arise thus: they move about between the agent
and patient together with a perception, and the patient ceases to be a
perceiving power and becomes a percipient, and the agent a quale instead
of a quality; but neither has any absolute existence? But now we make
the further discovery, that neither white or whiteness, nor any sense
or sensation, can be predicated of anything, for they are in a perpetual
flux. And therefore we must modify the doctrine of Theaetetus and
Protagoras, by asserting further that knowledge is and is not sensation;
and of everything we must say equally, that this is and is not, or
becomes or becomes not. And still the word 'this' is not quite correct,
for language fails in the attempt to express their meaning.

At the close of the discussion, Theodorus claims to be released from the
argument, according to his agreement. But Theaetetus insists that they
shall proceed to consider the doctrine of rest. This is declined by
Socrates, who has too much reverence for the great Parmenides lightly
to attack him. (We shall find that he returns to the doctrine of rest
in the Sophist; but at present he does not wish to be diverted from
his main purpose, which is, to deliver Theaetetus of his conception of
knowledge.) He proceeds to interrogate him further. When he says that
'knowledge is in perception,' with what does he perceive? The first
answer is, that he perceives sights with the eye, and sounds with the
ear. This leads Socrates to make the reflection that nice distinctions
of words are sometimes pedantic, but sometimes necessary; and he
proposes in this case to substitute the word 'through' for 'with.' For
the senses are not like the Trojan warriors in the horse, but have
a common centre of perception, in which they all meet. This common
principle is able to compare them with one another, and must therefore
be distinct from them (compare Republic). And as there are facts of
sense which are perceived through the organs of the body, there are also
mathematical and other abstractions, such as sameness and difference,
likeness and unlikeness, which the soul perceives by herself. Being is
the most universal of these abstractions. The good and the beautiful are
abstractions of another kind, which exist in relation and which above
all others the mind perceives in herself, comparing within her past,
present, and future. For example; we know a thing to be hard or soft by
the touch, of which the perception is given at birth to men and animals.
But the essence of hardness or softness, or the fact that this hardness
is, and is the opposite of softness, is slowly learned by reflection and
experience. Mere perception does not reach being, and therefore fails of
truth; and therefore has no share in knowledge. But if so, knowledge
is not perception. What then is knowledge? The mind, when occupied
by herself with being, is said to have opinion--shall we say that
'Knowledge is true opinion'? But still an old difficulty recurs; we
ask ourselves, 'How is false opinion possible?' This difficulty may be
stated as follows:--

Either we know or do not know a thing (for the intermediate processes
of learning and forgetting need not at present be considered); and in
thinking or having an opinion, we must either know or not know that
which we think, and we cannot know and be ignorant at the same time; we
cannot confuse one thing which we do not know, with another thing which
we do not know; nor can we think that which we do not know to be that
which we know, or that which we know to be that which we do not know.
And what other case is conceivable, upon the supposition that we either
know or do not know all things? Let us try another answer in the sphere
of being: 'When a man thinks, and thinks that which is not.' But would
this hold in any parallel case? Can a man see and see nothing? or hear
and hear nothing? or touch and touch nothing? Must he not see, hear, or
touch some one existing thing? For if he thinks about nothing he does
not think, and not thinking he cannot think falsely. And so the path of
being is closed against us, as well as the path of knowledge. But may
there not be 'heterodoxy,' or transference of opinion;--I mean, may not
one thing be supposed to be another? Theaetetus is confident that this
must be 'the true falsehood,' when a man puts good for evil or evil
for good. Socrates will not discourage him by attacking the paradoxical
expression 'true falsehood,' but passes on. The new notion involves a
process of thinking about two things, either together or alternately.
And thinking is the conversing of the mind with herself, which is
carried on in question and answer, until she no longer doubts, but
determines and forms an opinion. And false opinion consists in saying to
yourself, that one thing is another. But did you ever say to yourself,
that good is evil, or evil good? Even in sleep, did you ever imagine
that odd was even? Or did any man in his senses ever fancy that an ox
was a horse, or that two are one? So that we can never think one thing
to be another; for you must not meet me with the verbal quibble that
one--eteron--is other--eteron (both 'one' and 'other' in Greek are
called 'other'--eteron). He who has both the two things in his mind,
cannot misplace them; and he who has only one of them in his mind,
cannot misplace them--on either supposition transplacement is
inconceivable.

But perhaps there may still be a sense in which we can think that
which we do not know to be that which we know: e.g. Theaetetus may know
Socrates, but at a distance he may mistake another person for him. This
process may be conceived by the help of an image. Let us suppose that
every man has in his mind a block of wax of various qualities, the gift
of Memory, the mother of the Muses; and on this he receives the seal or
stamp of those sensations and perceptions which he wishes to remember.
That which he succeeds in stamping is remembered and known by him as
long as the impression lasts; but that, of which the impression is
rubbed out or imperfectly made, is forgotten, and not known. No one can
think one thing to be another, when he has the memorial or seal of both
of these in his soul, and a sensible impression of neither; or when he
knows one and does not know the other, and has no memorial or seal of
the other; or when he knows neither; or when he perceives both, or one
and not the other, or neither; or when he perceives and knows both,
and identifies what he perceives with what he knows (this is still more
impossible); or when he does not know one, and does not know and does
not perceive the other; or does not perceive one, and does not know
and does not perceive the other; or has no perception or knowledge
of either--all these cases must be excluded. But he may err when he
confuses what he knows or perceives, or what he perceives and does not
know, with what he knows, or what he knows and perceives with what he
knows and perceives.

Theaetetus is unable to follow these distinctions; which Socrates
proceeds to illustrate by examples, first of all remarking, that
knowledge may exist without perception, and perception without
knowledge. I may know Theodorus and Theaetetus and not see them; I
may see them, and not know them. 'That I understand.' But I could not
mistake one for the other if I knew you both, and had no perception of
either; or if I knew one only, and perceived neither; or if I knew
and perceived neither, or in any other of the excluded cases. The only
possibility of error is: 1st, when knowing you and Theodorus, and having
the impression of both of you on the waxen block, I, seeing you both
imperfectly and at a distance, put the foot in the wrong shoe--that
is to say, put the seal or stamp on the wrong object: or 2ndly, when
knowing both of you I only see one; or when, seeing and knowing you
both, I fail to identify the impression and the object. But there could
be no error when perception and knowledge correspond.

The waxen block in the heart of a man's soul, as I may say in the words
of Homer, who played upon the words ker and keros, may be smooth and
deep, and large enough, and then the signs are clearly marked and
lasting, and do not get confused. But in the 'hairy heart,' as the
all-wise poet sings, when the wax is muddy or hard or moist, there is
a corresponding confusion and want of retentiveness; in the muddy and
impure there is indistinctness, and still more in the hard, for there
the impressions have no depth of wax, and in the moist they are too
soon effaced. Yet greater is the indistinctness when they are all jolted
together in a little soul, which is narrow and has no room. These are
the sort of natures which have false opinion; from stupidity they see
and hear and think amiss; and this is falsehood and ignorance. Error,
then, is a confusion of thought and sense.

Theaetetus is delighted with this explanation. But Socrates has no
sooner found the new solution than he sinks into a fit of despondency.
For an objection occurs to him:--May there not be errors where there is
no confusion of mind and sense? e.g. in numbers. No one can confuse
the man whom he has in his thoughts with the horse which he has in his
thoughts, but he may err in the addition of five and seven. And observe
that these are purely mental conceptions. Thus we are involved once more
in the dilemma of saying, either that there is no such thing as false
opinion, or that a man knows what he does not know.

We are at our wit's end, and may therefore be excused for making a
bold diversion. All this time we have been repeating the words 'know,'
'understand,' yet we do not know what knowledge is. 'Why, Socrates,
how can you argue at all without using them?' Nay, but the true hero
of dialectic would have forbidden me to use them until I had explained
them. And I must explain them now. The verb 'to know' has two senses,
to have and to possess knowledge, and I distinguish 'having' from
'possessing.' A man may possess a garment which he does not wear; or he
may have wild birds in an aviary; these in one sense he possesses, and
in another he has none of them. Let this aviary be an image of the mind,
as the waxen block was; when we are young, the aviary is empty; after
a time the birds are put in; for under this figure we may describe
different forms of knowledge;--there are some of them in groups, and
some single, which are flying about everywhere; and let us suppose
a hunt after the science of odd and even, or some other science. The
possession of the birds is clearly not the same as the having them in
the hand. And the original chase of them is not the same as taking them
in the hand when they are already caged.

This distinction between use and possession saves us from the absurdity
of supposing that we do not know what we know, because we may know in
one sense, i.e. possess, what we do not know in another, i.e. use. But
have we not escaped one difficulty only to encounter a greater? For how
can the exchange of two kinds of knowledge ever become false opinion?
As well might we suppose that ignorance could make a man know, or that
blindness could make him see. Theaetetus suggests that in the aviary
there may be flying about mock birds, or forms of ignorance, and we
put forth our hands and grasp ignorance, when we are intending to grasp
knowledge. But how can he who knows the forms of knowledge and the forms
of ignorance imagine one to be the other? Is there some other form of
knowledge which distinguishes them? and another, and another? Thus we go
round and round in a circle and make no progress.

All this confusion arises out of our attempt to explain false opinion
without having explained knowledge. What then is knowledge? Theaetetus
repeats that knowledge is true opinion. But this seems to be refuted by
the instance of orators and judges. For surely the orator cannot convey
a true knowledge of crimes at which the judges were not present; he
can only persuade them, and the judge may form a true opinion and truly
judge. But if true opinion were knowledge they could not have judged
without knowledge.

Once more. Theaetetus offers a definition which he has heard: Knowledge
is true opinion accompanied by definition or explanation. Socrates has
had a similar dream, and has further heard that the first elements are
names only, and that definition or explanation begins when they are
combined; the letters are unknown, the syllables or combinations
are known. But this new hypothesis when tested by the letters of the
alphabet is found to break down. The first syllable of Socrates' name
is SO. But what is SO? Two letters, S and O, a sibilant and a vowel,
of which no further explanation can be given. And how can any one
be ignorant of either of them, and yet know both of them? There is,
however, another alternative:--We may suppose that the syllable has a
separate form or idea distinct from the letters or parts. The all of the
parts may not be the whole. Theaetetus is very much inclined to adopt
this suggestion, but when interrogated by Socrates he is unable to
draw any distinction between the whole and all the parts. And if the
syllables have no parts, then they are those original elements of which
there is no explanation. But how can the syllable be known if the letter
remains unknown? In learning to read as children, we are first taught
the letters and then the syllables. And in music, the notes, which
are the letters, have a much more distinct meaning to us than the
combination of them.

Once more, then, we must ask the meaning of the statement, that
'Knowledge is right opinion, accompanied by explanation or definition.'
Explanation may mean, (1) the reflection or expression of a man's
thoughts--but every man who is not deaf and dumb is able to express his
thoughts--or (2) the enumeration of the elements of which anything is
composed. A man may have a true opinion about a waggon, but then, and
then only, has he knowledge of a waggon when he is able to enumerate
the hundred planks of Hesiod. Or he may know the syllables of the name
Theaetetus, but not the letters; yet not until he knows both can he be
said to have knowledge as well as opinion. But on the other hand he may
know the syllable 'The' in the name Theaetetus, yet he may be mistaken
about the same syllable in the name Theodorus, and in learning to read
we often make such mistakes. And even if he could write out all the
letters and syllables of your name in order, still he would only have
right opinion. Yet there may be a third meaning of the definition,
besides the image or expression of the mind, and the enumeration of the
elements, viz. (3) perception of difference.

For example, I may see a man who has eyes, nose, and mouth;--that will
not distinguish him from any other man. Or he may have a snub-nose and
prominent eyes;--that will not distinguish him from myself and you and
others who are like me. But when I see a certain kind of snub-nosedness,
then I recognize Theaetetus. And having this sign of difference, I have
knowledge. But have I knowledge or opinion of this difference; if I
have only opinion I have not knowledge; if I have knowledge we assume
a disputed term; for knowledge will have to be defined as right opinion
with knowledge of difference.

And so, Theaetetus, knowledge is neither perception nor true opinion,
nor yet definition accompanying true opinion. And I have shown that the
children of your brain are not worth rearing. Are you still in labour,
or have you brought all you have to say about knowledge to the birth? If
you have any more thoughts, you will be the better for having got rid of
these; or if you have none, you will be the better for not fancying that
you know what you do not know. Observe the limits of my art, which, like
my mother's, is an art of midwifery; I do not pretend to compare with
the good and wise of this and other ages.

And now I go to meet Meletus at the porch of the King Archon; but
to-morrow I shall hope to see you again, Theodorus, at this place.

...

I. The saying of Theaetetus, that 'Knowledge is sensible perception,'
may be assumed to be a current philosophical opinion of the age. 'The
ancients,' as Aristotle (De Anim.) says, citing a verse of Empedocles,
'affirmed knowledge to be the same as perception.' We may now examine
these words, first, with reference to their place in the history of
philosophy, and secondly, in relation to modern speculations.

(a) In the age of Socrates the mind was passing from the object to the
subject. The same impulse which a century before had led men to form
conceptions of the world, now led them to frame general notions of the
human faculties and feelings, such as memory, opinion, and the like. The
simplest of these is sensation, or sensible perception, by which Plato
seems to mean the generalized notion of feelings and impressions of
sense, without determining whether they are conscious or not.

The theory that 'Knowledge is sensible perception' is the antithesis of
that which derives knowledge from the mind (Theaet.), or which assumes
the existence of ideas independent of the mind (Parm.). Yet from their
extreme abstraction these theories do not represent the opposite poles
of thought in the same way that the corresponding differences would
in modern philosophy. The most ideal and the most sensational have a
tendency to pass into one another; Heracleitus, like his great successor
Hegel, has both aspects. The Eleatic isolation of Being and the Megarian
or Cynic isolation of individuals are placed in the same class by Plato
(Soph.); and the same principle which is the symbol of motion to one
mind is the symbol of rest to another. The Atomists, who are sometimes
regarded as the Materialists of Plato, denied the reality of sensation.
And in the ancient as well as the modern world there were reactions from
theory to experience, from ideas to sense. This is a point of view from
which the philosophy of sensation presented great attraction to the
ancient thinker. Amid the conflict of ideas and the variety of opinions,
the impression of sense remained certain and uniform. Hardness,
softness, cold, heat, etc. are not absolutely the same to different
persons, but the art of measuring could at any rate reduce them all
to definite natures (Republic). Thus the doctrine that knowledge is
perception supplies or seems to supply a firm standing ground. Like the
other notions of the earlier Greek philosophy, it was held in a very
simple way, without much basis of reasoning, and without suggesting the
questions which naturally arise in our own minds on the same subject.

(b) The fixedness of impressions of sense furnishes a link of connexion
between ancient and modern philosophy. The modern thinker often repeats
the parallel axiom, 'All knowledge is experience.' He means to say that
the outward and not the inward is both the original source and the final
criterion of truth, because the outward can be observed and analyzed;
the inward is only known by external results, and is dimly perceived
by each man for himself. In what does this differ from the saying of
Theaetetus? Chiefly in this--that the modern term 'experience,' while
implying a point of departure in sense and a return to sense, also
includes all the processes of reasoning and imagination which have
intervened. The necessary connexion between them by no means affords a
measure of the relative degree of importance which is to be ascribed to
either element. For the inductive portion of any science may be small,
as in mathematics or ethics, compared with that which the mind has
attained by reasoning and reflection on a very few facts.

II. The saying that 'All knowledge is sensation' is identified by Plato
with the Protagorean thesis that 'Man is the measure of all things.'
The interpretation which Protagoras himself is supposed to give of these
latter words is: 'Things are to me as they appear to me, and to you as
they appear to you.' But there remains still an ambiguity both in the
text and in the explanation, which has to be cleared up. Did Protagoras
merely mean to assert the relativity of knowledge to the human mind? Or
did he mean to deny that there is an objective standard of truth?

These two questions have not been always clearly distinguished; the
relativity of knowledge has been sometimes confounded with uncertainty.
The untutored mind is apt to suppose that objects exist independently
of the human faculties, because they really exist independently of the
faculties of any individual. In the same way, knowledge appears to be
a body of truths stored up in books, which when once ascertained are
independent of the discoverer. Further consideration shows us that these
truths are not really independent of the mind; there is an adaptation of
one to the other, of the eye to the object of sense, of the mind to the
conception. There would be no world, if there neither were nor ever
had been any one to perceive the world. A slight effort of reflection
enables us to understand this; but no effort of reflection will enable
us to pass beyond the limits of our own faculties, or to imagine the
relation or adaptation of objects to the mind to be different from that
of which we have experience. There are certain laws of language and
logic to which we are compelled to conform, and to which our ideas
naturally adapt themselves; and we can no more get rid of them than we
can cease to be ourselves. The absolute and infinite, whether explained
as self-existence, or as the totality of human thought, or as the
Divine nature, if known to us at all, cannot escape from the category of
relation.

But because knowledge is subjective or relative to the mind, we are
not to suppose that we are therefore deprived of any of the tests or
criteria of truth. One man still remains wiser than another, a
more accurate observer and relater of facts, a truer measure of the
proportions of knowledge. The nature of testimony is not altered, nor
the verification of causes by prescribed methods less certain. Again,
the truth must often come to a man through others, according to the
measure of his capacity and education. But neither does this affect the
testimony, whether written or oral, which he knows by experience to
be trustworthy. He cannot escape from the laws of his own mind; and
he cannot escape from the further accident of being dependent for his
knowledge on others. But still this is no reason why he should always be
in doubt; of many personal, of many historical and scientific facts he
may be absolutely assured. And having such a mass of acknowledged truth
in the mathematical and physical, not to speak of the moral sciences,
the moderns have certainly no reason to acquiesce in the statement
that truth is appearance only, or that there is no difference between
appearance and truth.

The relativity of knowledge is a truism to us, but was a great
psychological discovery in the fifth century before Christ. Of this
discovery, the first distinct assertion is contained in the thesis of
Protagoras. Probably he had no intention either of denying or affirming
an objective standard of truth. He did not consider whether man in the
higher or man in the lower sense was a 'measure of all things.' Like
other great thinkers, he was absorbed with one idea, and that idea was
the absoluteness of perception. Like Socrates, he seemed to see that
philosophy must be brought back from 'nature' to 'truth,' from the world
to man. But he did not stop to analyze whether he meant 'man' in the
concrete or man in the abstract, any man or some men, 'quod semper quod
ubique' or individual private judgment. Such an analysis lay beyond
his sphere of thought; the age before Socrates had not arrived at these
distinctions. Like the Cynics, again, he discarded knowledge in any
higher sense than perception. For 'truer' or 'wiser' he substituted
the word 'better,' and is not unwilling to admit that both states and
individuals are capable of practical improvement. But this improvement
does not arise from intellectual enlightenment, nor yet from
the exertion of the will, but from a change of circumstances and
impressions; and he who can effect this change in himself or others may
be deemed a philosopher. In the mode of effecting it, while agreeing
with Socrates and the Cynics in the importance which he attaches to
practical life, he is at variance with both of them. To suppose that
practice can be divorced from speculation, or that we may do good
without caring about truth, is by no means singular, either in
philosophy or life. The singularity of this, as of some other
(so-called) sophistical doctrines, is the frankness with which they are
avowed, instead of being veiled, as in modern times, under ambiguous and
convenient phrases.

Plato appears to treat Protagoras much as he himself is treated by
Aristotle; that is to say, he does not attempt to understand him from
his own point of view. But he entangles him in the meshes of a more
advanced logic. To which Protagoras is supposed to reply by Megarian
quibbles, which destroy logic, 'Not only man, but each man, and each
man at each moment.' In the arguments about sight and memory there is a
palpable unfairness which is worthy of the great 'brainless brothers,'
Euthydemus and Dionysodorus, and may be compared with the egkekalummenos
('obvelatus') of Eubulides. For he who sees with one eye only cannot be
truly said both to see and not to see; nor is memory, which is liable
to forget, the immediate knowledge to which Protagoras applies the
term. Theodorus justly charges Socrates with going beyond the truth;
and Protagoras has equally right on his side when he protests against
Socrates arguing from the common use of words, which 'the vulgar pervert
in all manner of ways.'

III. The theory of Protagoras is connected by Aristotle as well as Plato
with the flux of Heracleitus. But Aristotle is only following Plato,
and Plato, as we have already seen, did not mean to imply that such a
connexion was admitted by Protagoras himself. His metaphysical genius
saw or seemed to see a common tendency in them, just as the modern
historian of ancient philosophy might perceive a parallelism between
two thinkers of which they were probably unconscious themselves. We must
remember throughout that Plato is not speaking of Heracleitus, but of
the Heracliteans, who succeeded him; nor of the great original ideas of
the master, but of the Eristic into which they had degenerated a hundred
years later. There is nothing in the fragments of Heracleitus which at
all justifies Plato's account of him. His philosophy may be resolved
into two elements--first, change, secondly, law or measure pervading
the change: these he saw everywhere, and often expressed in strange
mythological symbols. But he has no analysis of sensible perception such
as Plato attributes to him; nor is there any reason to suppose that
he pushed his philosophy into that absolute negation in which
Heracliteanism was sunk in the age of Plato. He never said that 'change
means every sort of change;' and he expressly distinguished between
'the general and particular understanding.' Like a poet, he surveyed
the elements of mythology, nature, thought, which lay before him, and
sometimes by the light of genius he saw or seemed to see a mysterious
principle working behind them. But as has been the case with other great
philosophers, and with Plato and Aristotle themselves, what was really
permanent and original could not be understood by the next generation,
while a perverted logic carried out his chance expressions with an
illogical consistency. His simple and noble thoughts, like those of the
great Eleatic, soon degenerated into a mere strife of words. And when
thus reduced to mere words, they seem to have exercised a far wider
influence in the cities of Ionia (where the people 'were mad about
them') than in the life-time of Heracleitus--a phenomenon which, though
at first sight singular, is not without a parallel in the history of
philosophy and theology.

It is this perverted form of the Heraclitean philosophy which is
supposed to effect the final overthrow of Protagorean sensationalism.
For if all things are changing at every moment, in all sorts of ways,
then there is nothing fixed or defined at all, and therefore no sensible
perception, nor any true word by which that or anything else can be
described. Of course Protagoras would not have admitted the justice
of this argument any more than Heracleitus would have acknowledged the
'uneducated fanatics' who appealed to his writings. He might have said,
'The excellent Socrates has first confused me with Heracleitus, and
Heracleitus with his Ephesian successors, and has then disproved the
existence both of knowledge and sensation. But I am not responsible
for what I never said, nor will I admit that my common-sense account of
knowledge can be overthrown by unintelligible Heraclitean paradoxes.'

IV. Still at the bottom of the arguments there remains a truth, that
knowledge is something more than sensible perception;--this alone would
not distinguish man from a tadpole. The absoluteness of sensations
at each moment destroys the very consciousness of sensations (compare
Phileb.), or the power of comparing them. The senses are not mere holes
in a 'Trojan horse,' but the organs of a presiding nature, in which they
meet. A great advance has been made in psychology when the senses
are recognized as organs of sense, and we are admitted to see or feel
'through them' and not 'by them,' a distinction of words which, as
Socrates observes, is by no means pedantic. A still further step has
been made when the most abstract notions, such as Being and Not-being,
sameness and difference, unity and plurality, are acknowledged to be the
creations of the mind herself, working upon the feelings or impressions
of sense. In this manner Plato describes the process of acquiring them,
in the words 'Knowledge consists not in the feelings or affections
(pathemasi), but in the process of reasoning about them (sullogismo).'
Here, is in the Parmenides, he means something not really different
from generalization. As in the Sophist, he is laying the foundation of a
rational psychology, which is to supersede the Platonic reminiscence of
Ideas as well as the Eleatic Being and the individualism of Megarians
and Cynics.

V. Having rejected the doctrine that 'Knowledge is perception,' we now
proceed to look for a definition of knowledge in the sphere of opinion.
But here we are met by a singular difficulty: How is false opinion
possible? For we must either know or not know that which is presented
to the mind or to sense. We of course should answer at once: 'No; the
alternative is not necessary, for there may be degrees of knowledge; and
we may know and have forgotten, or we may be learning, or we may have a
general but not a particular knowledge, or we may know but not be able
to explain;' and many other ways may be imagined in which we know and do
not know at the same time. But these answers belong to a later stage of
metaphysical discussion; whereas the difficulty in question naturally
arises owing to the childhood of the human mind, like the parallel
difficulty respecting Not-being. Men had only recently arrived at the
notion of opinion; they could not at once define the true and pass
beyond into the false. The very word doxa was full of ambiguity, being
sometimes, as in the Eleatic philosophy, applied to the sensible world,
and again used in the more ordinary sense of opinion. There is no
connexion between sensible appearance and probability, and yet both
of them met in the word doxa, and could hardly be disengaged from one
another in the mind of the Greek living in the fifth or fourth century
B.C. To this was often added, as at the end of the fifth book of the
Republic, the idea of relation, which is equally distinct from either of
them; also a fourth notion, the conclusion of the dialectical process,
the making up of the mind after she has been 'talking to herself'
(Theat.).

We are not then surprised that the sphere of opinion and of Not-being
should be a dusky, half-lighted place (Republic), belonging neither
to the old world of sense and imagination, nor to the new world of
reflection and reason. Plato attempts to clear up this darkness. In
his accustomed manner he passes from the lower to the higher, without
omitting the intermediate stages. This appears to be the reason why he
seeks for the definition of knowledge first in the sphere of opinion.
Hereafter we shall find that something more than opinion is required.

False opinion is explained by Plato at first as a confusion of mind and
sense, which arises when the impression on the mind does not correspond
to the impression made on the senses. It is obvious that this
explanation (supposing the distinction between impressions on the mind
and impressions on the senses to be admitted) does not account for all
forms of error; and Plato has excluded himself from the consideration of
the greater number, by designedly omitting the intermediate processes
of learning and forgetting; nor does he include fallacies in the use of
language or erroneous inferences. But he is struck by one possibility
of error, which is not covered by his theory, viz. errors in arithmetic.
For in numbers and calculation there is no combination of thought and
sense, and yet errors may often happen. Hence he is led to discard the
explanation which might nevertheless have been supposed to hold good
(for anything which he says to the contrary) as a rationale of error, in
the case of facts derived from sense.

Another attempt is made to explain false opinion by assigning to error
a sort of positive existence. But error or ignorance is essentially
negative--a not-knowing; if we knew an error, we should be no longer in
error. We may veil our difficulty under figures of speech, but these,
although telling arguments with the multitude, can never be the real
foundation of a system of psychology. Only they lead us to dwell upon
mental phenomena which if expressed in an abstract form would not be
realized by us at all. The figure of the mind receiving impressions is
one of those images which have rooted themselves for ever in language.
It may or may not be a 'gracious aid' to thought; but it cannot be
got rid of. The other figure of the enclosure is also remarkable as
affording the first hint of universal all-pervading ideas,--a notion
further carried out in the Sophist. This is implied in the birds, some
in flocks, some solitary, which fly about anywhere and everywhere. Plato
discards both figures, as not really solving the question which to us
appears so simple: 'How do we make mistakes?' The failure of the enquiry
seems to show that we should return to knowledge, and begin with that;
and we may afterwards proceed, with a better hope of success, to the
examination of opinion.

But is true opinion really distinct from knowledge? The difference
between these he seeks to establish by an argument, which to us appears
singular and unsatisfactory. The existence of true opinion is proved
by the rhetoric of the law courts, which cannot give knowledge, but
may give true opinion. The rhetorician cannot put the judge or juror in
possession of all the facts which prove an act of violence, but he may
truly persuade them of the commission of such an act. Here the idea of
true opinion seems to be a right conclusion from imperfect knowledge.
But the correctness of such an opinion will be purely accidental; and is
really the effect of one man, who has the means of knowing, persuading
another who has not. Plato would have done better if he had said that
true opinion was a contradiction in terms.

Assuming the distinction between knowledge and opinion, Theaetetus, in
answer to Socrates, proceeds to define knowledge as true opinion, with
definite or rational explanation. This Socrates identifies with another
and different theory, of those who assert that knowledge first begins
with a proposition.

The elements may be perceived by sense, but they are names, and cannot
be defined. When we assign to them some predicate, they first begin to
have a meaning (onomaton sumploke logou ousia). This seems equivalent
to saying, that the individuals of sense become the subject of knowledge
when they are regarded as they are in nature in relation to other
individuals.

Yet we feel a difficulty in following this new hypothesis. For must not
opinion be equally expressed in a proposition? The difference between
true and false opinion is not the difference between the particular and
the universal, but between the true universal and the false. Thought may
be as much at fault as sight. When we place individuals under a
class, or assign to them attributes, this is not knowledge, but a very
rudimentary process of thought; the first generalization of all, without
which language would be impossible. And has Plato kept altogether clear
of a confusion, which the analogous word logos tends to create, of a
proposition and a definition? And is not the confusion increased by the
use of the analogous term 'elements,' or 'letters'? For there is no real
resemblance between the relation of letters to a syllable, and of the
terms to a proposition.

Plato, in the spirit of the Megarian philosophy, soon discovers a flaw
in the explanation. For how can we know a compound of which the simple
elements are unknown to us? Can two unknowns make a known? Can a whole
be something different from the parts? The answer of experience is that
they can; for we may know a compound, which we are unable to analyze
into its elements; and all the parts, when united, may be more than all
the parts separated: e.g. the number four, or any other number, is more
than the units which are contained in it; any chemical compound is more
than and different from the simple elements. But ancient philosophy
in this, as in many other instances, proceeding by the path of mental
analysis, was perplexed by doubts which warred against the plainest
facts.

Three attempts to explain the new definition of knowledge still remain
to be considered. They all of them turn on the explanation of logos. The
first account of the meaning of the word is the reflection of thought in
speech--a sort of nominalism 'La science est une langue bien faite.' But
anybody who is not dumb can say what he thinks; therefore mere speech
cannot be knowledge. And yet we may observe, that there is in this
explanation an element of truth which is not recognized by Plato; viz.
that truth and thought are inseparable from language, although mere
expression in words is not truth. The second explanation of logos is the
enumeration of the elementary parts of the complex whole. But this is
only definition accompanied with right opinion, and does not yet attain
to the certainty of knowledge. Plato does not mention the greater
objection, which is, that the enumeration of particulars is endless;
such a definition would be based on no principle, and would not help us
at all in gaining a common idea. The third is the best explanation,--the
possession of a characteristic mark, which seems to answer to the
logical definition by genus and difference. But this, again, is equally
necessary for right opinion; and we have already determined, although
not on very satisfactory grounds, that knowledge must be distinguished
from opinion. A better distinction is drawn between them in the Timaeus.
They might be opposed as philosophy and rhetoric, and as conversant
respectively with necessary and contingent matter. But no true idea of
the nature of either of them, or of their relation to one another, could
be framed until science obtained a content. The ancient philosophers
in the age of Plato thought of science only as pure abstraction, and to
this opinion stood in no relation.

Like Theaetetus, we have attained to no definite result. But an
interesting phase of ancient philosophy has passed before us. And the
negative result is not to be despised. For on certain subjects, and in
certain states of knowledge, the work of negation or clearing the ground
must go on, perhaps for a generation, before the new structure can begin
to rise. Plato saw the necessity of combating the illogical logic of
the Megarians and Eristics. For the completion of the edifice, he makes
preparation in the Theaetetus, and crowns the work in the Sophist.

Many (1) fine expressions, and (2) remarks full of wisdom, (3) also
germs of a metaphysic of the future, are scattered up and down in
the dialogue. Such, for example, as (1) the comparison of Theaetetus'
progress in learning to the 'noiseless flow of a river of oil';
the satirical touch, 'flavouring a sauce or fawning speech'; or the
remarkable expression, 'full of impure dialectic'; or the lively images
under which the argument is described,--'the flood of arguments pouring
in,' the fresh discussions 'bursting in like a band of revellers.'
(2) As illustrations of the second head, may be cited the remark of
Socrates, that 'distinctions of words, although sometimes pedantic, are
also necessary'; or the fine touch in the character of the lawyer,
that 'dangers came upon him when the tenderness of youth was unequal to
them'; or the description of the manner in which the spirit is broken
in a wicked man who listens to reproof until he becomes like a child; or
the punishment of the wicked, which is not physical suffering, but the
perpetual companionship of evil (compare Gorgias); or the saying, often
repeated by Aristotle and others, that 'philosophy begins in wonder,
for Iris is the child of Thaumas'; or the superb contempt with which
the philosopher takes down the pride of wealthy landed proprietors by
comparison of the whole earth. (3) Important metaphysical ideas are: a.
the conception of thought, as the mind talking to herself; b. the notion
of a common sense, developed further by Aristotle, and the explicit
declaration, that the mind gains her conceptions of Being, sameness,
number, and the like, from reflection on herself; c. the excellent
distinction of Theaetetus (which Socrates, speaking with emphasis,
'leaves to grow') between seeing the forms or hearing the sounds of
words in a foreign language, and understanding the meaning of them; and
d. the distinction of Socrates himself between 'having' and 'possessing'
knowledge, in which the answer to the whole discussion appears to be
contained.

...

There is a difference between ancient and modern psychology, and we
have a difficulty in explaining one in the terms of the other. To us the
inward and outward sense and the inward and outward worlds of which they
are the organs are parted by a wall, and appear as if they could never
be confounded. The mind is endued with faculties, habits, instincts, and
a personality or consciousness in which they are bound together. Over
against these are placed forms, colours, external bodies coming into
contact with our own body. We speak of a subject which is ourselves,
of an object which is all the rest. These are separable in thought, but
united in any act of sensation, reflection, or volition. As there are
various degrees in which the mind may enter into or be abstracted from
the operations of sense, so there are various points at which this
separation or union may be supposed to occur. And within the sphere
of mind the analogy of sense reappears; and we distinguish not only
external objects, but objects of will and of knowledge which we contrast
with them. These again are comprehended in a higher object, which
reunites with the subject. A multitude of abstractions are created by
the efforts of successive thinkers which become logical determinations;
and they have to be arranged in order, before the scheme of thought is
complete. The framework of the human intellect is not the peculium of
an individual, but the joint work of many who are of all ages and
countries. What we are in mind is due, not merely to our physical, but
to our mental antecedents which we trace in history, and more especially
in the history of philosophy. Nor can mental phenomena be truly
explained either by physiology or by the observation of consciousness
apart from their history. They have a growth of their own, like the
growth of a flower, a tree, a human being. They may be conceived as of
themselves constituting a common mind, and having a sort of personal
identity in which they coexist.

So comprehensive is modern psychology, seeming to aim at constructing
anew the entire world of thought. And prior to or simultaneously with
this construction a negative process has to be carried on, a clearing
away of useless abstractions which we have inherited from the past. Many
erroneous conceptions of the mind derived from former philosophies
have found their way into language, and we with difficulty disengage
ourselves from them. Mere figures of speech have unconsciously
influenced the minds of great thinkers. Also there are some
distinctions, as, for example, that of the will and of the reason, and
of the moral and intellectual faculties, which are carried further than
is justified by experience. Any separation of things which we cannot see
or exactly define, though it may be necessary, is a fertile source of
error. The division of the mind into faculties or powers or virtues is
too deeply rooted in language to be got rid of, but it gives a false
impression. For if we reflect on ourselves we see that all our faculties
easily pass into one another, and are bound together in a single mind or
consciousness; but this mental unity is apt to be concealed from us by
the distinctions of language.

A profusion of words and ideas has obscured rather than enlightened
mental science. It is hard to say how many fallacies have arisen from
the representation of the mind as a box, as a 'tabula rasa,' a book,
a mirror, and the like. It is remarkable how Plato in the Theaetetus,
after having indulged in the figure of the waxen tablet and the decoy,
afterwards discards them. The mind is also represented by another class
of images, as the spring of a watch, a motive power, a breath, a stream,
a succession of points or moments. As Plato remarks in the Cratylus,
words expressive of motion as well as of rest are employed to describe
the faculties and operations of the mind; and in these there is
contained another store of fallacies. Some shadow or reflection of the
body seems always to adhere to our thoughts about ourselves, and mental
processes are hardly distinguished in language from bodily ones. To see
or perceive are used indifferently of both; the words intuition, moral
sense, common sense, the mind's eye, are figures of speech transferred
from one to the other. And many other words used in early poetry or in
sacred writings to express the works of mind have a materialistic sound;
for old mythology was allied to sense, and the distinction of matter and
mind had not as yet arisen. Thus materialism receives an illusive aid
from language; and both in philosophy and religion the imaginary figure
or association easily takes the place of real knowledge.

Again, there is the illusion of looking into our own minds as if our
thoughts or feelings were written down in a book. This is another figure
of speech, which might be appropriately termed 'the fallacy of the
looking-glass.' We cannot look at the mind unless we have the eye
which sees, and we can only look, not into, but out of the mind at
the thoughts, words, actions of ourselves and others. What we dimly
recognize within us is not experience, but rather the suggestion of an
experience, which we may gather, if we will, from the observation of the
world. The memory has but a feeble recollection of what we were saying
or doing a few weeks or a few months ago, and still less of what we
were thinking or feeling. This is one among many reasons why there is
so little self-knowledge among mankind; they do not carry with them
the thought of what they are or have been. The so-called 'facts of
consciousness' are equally evanescent; they are facts which nobody ever
saw, and which can neither be defined nor described. Of the three laws
of thought the first (All A = A) is an identical proposition--that is to
say, a mere word or symbol claiming to be a proposition: the two others
(Nothing can be A and not A, and Everything is either A or not A) are
untrue, because they exclude degrees and also the mixed modes and double
aspects under which truth is so often presented to us. To assert that
man is man is unmeaning; to say that he is free or necessary and cannot
be both is a half truth only. These are a few of the entanglements which
impede the natural course of human thought. Lastly, there is the fallacy
which lies still deeper, of regarding the individual mind apart from
the universal, or either, as a self-existent entity apart from the ideas
which are contained in them.

In ancient philosophies the analysis of the mind is still rudimentary
and imperfect. It naturally began with an effort to disengage the
universal from sense--this was the first lifting up of the mist. It
wavered between object and subject, passing imperceptibly from one or
Being to mind and thought. Appearance in the outward object was for a
time indistinguishable from opinion in the subject. At length mankind
spoke of knowing as well as of opining or perceiving. But when the word
'knowledge' was found how was it to be explained or defined? It was not
an error, it was a step in the right direction, when Protagoras said
that 'Man is the measure of all things,' and that 'All knowledge is
perception.' This was the subjective which corresponded to the objective
'All is flux.' But the thoughts of men deepened, and soon they began
to be aware that knowledge was neither sense, nor yet opinion--with or
without explanation; nor the expression of thought, nor the enumeration
of parts, nor the addition of characteristic marks. Motion and rest were
equally ill adapted to express its nature, although both must in some
sense be attributed to it; it might be described more truly as the mind
conversing with herself; the discourse of reason; the hymn of dialectic,
the science of relations, of ideas, of the so-called arts and sciences,
of the one, of the good, of the all:--this is the way along which Plato
is leading us in his later dialogues. In its higher signification it was
the knowledge, not of men, but of gods, perfect and all sufficing:--like
other ideals always passing out of sight, and nevertheless present to
the mind of Aristotle as well as Plato, and the reality to which they
were both tending. For Aristotle as well as Plato would in modern
phraseology have been termed a mystic; and like him would have defined
the higher philosophy to be 'Knowledge of being or essence,'--words to
which in our own day we have a difficulty in attaching a meaning.

Yet, in spite of Plato and his followers, mankind have again and again
returned to a sensational philosophy. As to some of the early thinkers,
amid the fleetings of sensible objects, ideas alone seemed to be fixed,
so to a later generation amid the fluctuation of philosophical opinions
the only fixed points appeared to be outward objects. Any pretence
of knowledge which went beyond them implied logical processes, of the
correctness of which they had no assurance and which at best were only
probable. The mind, tired of wandering, sought to rest on firm ground;
when the idols of philosophy and language were stripped off, the
perception of outward objects alone remained. The ancient Epicureans
never asked whether the comparison of these with one another did not
involve principles of another kind which were above and beyond them. In
like manner the modern inductive philosophy forgot to enquire into
the meaning of experience, and did not attempt to form a conception of
outward objects apart from the mind, or of the mind apart from them.
Soon objects of sense were merged in sensations and feelings, but
feelings and sensations were still unanalyzed. At last we return to
the doctrine attributed by Plato to Protagoras, that the mind is only a
succession of momentary perceptions. At this point the modern philosophy
of experience forms an alliance with ancient scepticism.

The higher truths of philosophy and religion are very far removed from
sense. Admitting that, like all other knowledge, they are derived from
experience, and that experience is ultimately resolvable into facts
which come to us through the eye and ear, still their origin is a
mere accident which has nothing to do with their true nature. They
are universal and unseen; they belong to all times--past, present, and
future. Any worthy notion of mind or reason includes them. The proof of
them is, 1st, their comprehensiveness and consistency with one another;
2ndly, their agreement with history and experience. But sensation is of
the present only, is isolated, is and is not in successive moments. It
takes the passing hour as it comes, following the lead of the eye or
ear instead of the command of reason. It is a faculty which man has in
common with the animals, and in which he is inferior to many of them.
The importance of the senses in us is that they are the apertures of the
mind, doors and windows through which we take in and make our own the
materials of knowledge. Regarded in any other point of view sensation
is of all mental acts the most trivial and superficial. Hence the term
'sensational' is rightly used to express what is shallow in thought and
feeling.

We propose in what follows, first of all, like Plato in the Theaetetus,
to analyse sensation, and secondly to trace the connexion between
theories of sensation and a sensational or Epicurean philosophy.

Paragraph I. We, as well as the ancients, speak of the five senses, and
of a sense, or common sense, which is the abstraction of them. The
term 'sense' is also used metaphorically, both in ancient and modern
philosophy, to express the operations of the mind which are immediate or
intuitive. Of the five senses, two--the sight and the hearing--are of
a more subtle and complex nature, while two others--the smell and the
taste--seem to be only more refined varieties of touch. All of them
are passive, and by this are distinguished from the active faculty of
speech: they receive impressions, but do not produce them, except in so
far as they are objects of sense themselves.

Physiology speaks to us of the wonderful apparatus of nerves, muscles,
tissues, by which the senses are enabled to fulfil their functions. It
traces the connexion, though imperfectly, of the bodily organs with the
operations of the mind. Of these latter, it seems rather to know the
conditions than the causes. It can prove to us that without the brain we
cannot think, and that without the eye we cannot see: and yet there is
far more in thinking and seeing than is given by the brain and the eye.
It observes the 'concomitant variations' of body and mind. Psychology,
on the other hand, treats of the same subject regarded from another
point of view. It speaks of the relation of the senses to one another;
it shows how they meet the mind; it analyzes the transition from sense
to thought. The one describes their nature as apparent to the outward
eye; by the other they are regarded only as the instruments of the mind.
It is in this latter point of view that we propose to consider them.

The simplest sensation involves an unconscious or nascent operation
of the mind; it implies objects of sense, and objects of sense have
differences of form, number, colour. But the conception of an object
without us, or the power of discriminating numbers, forms, colours,
is not given by the sense, but by the mind. A mere sensation does not
attain to distinctness: it is a confused impression, sugkechumenon ti,
as Plato says (Republic), until number introduces light and order
into the confusion. At what point confusion becomes distinctness is a
question of degree which cannot be precisely determined. The distant
object, the undefined notion, come out into relief as we approach
them or attend to them. Or we may assist the analysis by attempting
to imagine the world first dawning upon the eye of the infant or of a
person newly restored to sight. Yet even with them the mind as well
as the eye opens or enlarges. For all three are inseparably bound
together--the object would be nowhere and nothing, if not perceived by
the sense, and the sense would have no power of distinguishing without
the mind.

But prior to objects of sense there is a third nature in which they
are contained--that is to say, space, which may be explained in various
ways. It is the element which surrounds them; it is the vacuum or void
which they leave or occupy when passing from one portion of space to
another. It might be described in the language of ancient philosophy, as
'the Not-being' of objects. It is a negative idea which in the course
of ages has become positive. It is originally derived from the
contemplation of the world without us--the boundless earth or sea, the
vacant heaven, and is therefore acquired chiefly through the sense of
sight: to the blind the conception of space is feeble and inadequate,
derived for the most part from touch or from the descriptions of others.
At first it appears to be continuous; afterwards we perceive it to be
capable of division by lines or points, real or imaginary. By the
help of mathematics we form another idea of space, which is altogether
independent of experience. Geometry teaches us that the innumerable
lines and figures by which space is or may be intersected are absolutely
true in all their combinations and consequences. New and unchangeable
properties of space are thus developed, which are proved to us in a
thousand ways by mathematical reasoning as well as by common experience.
Through quantity and measure we are conducted to our simplest and purest
notion of matter, which is to the cube or solid what space is to
the square or surface. And all our applications of mathematics are
applications of our ideas of space to matter. No wonder then that they
seem to have a necessary existence to us. Being the simplest of our
ideas, space is also the one of which we have the most difficulty in
ridding ourselves. Neither can we set a limit to it, for wherever we
fix a limit, space is springing up beyond. Neither can we conceive a
smallest or indivisible portion of it; for within the smallest there
is a smaller still; and even these inconceivable qualities of space,
whether the infinite or the infinitesimal, may be made the subject of
reasoning and have a certain truth to us.

Whether space exists in the mind or out of it, is a question which has
no meaning. We should rather say that without it the mind is incapable
of conceiving the body, and therefore of conceiving itself. The mind may
be indeed imagined to contain the body, in the same way that Aristotle
(partly following Plato) supposes God to be the outer heaven or circle
of the universe. But how can the individual mind carry about the
universe of space packed up within, or how can separate minds have
either a universe of their own or a common universe? In such conceptions
there seems to be a confusion of the individual and the universal. To
say that we can only have a true idea of ourselves when we deny the
reality of that by which we have any idea of ourselves is an absurdity.
The earth which is our habitation and 'the starry heaven above' and
we ourselves are equally an illusion, if space is only a quality or
condition of our minds.

Again, we may compare the truths of space with other truths derived from
experience, which seem to have a necessity to us in proportion to the
frequency of their recurrence or the truth of the consequences which may
be inferred from them. We are thus led to remark that the necessity
in our ideas of space on which much stress has been laid, differs in a
slight degree only from the necessity which appears to belong to other
of our ideas, e.g. weight, motion, and the like. And there is another
way in which this necessity may be explained. We have been taught it,
and the truth which we were taught or which we inherited has never been
contradicted in all our experience and is therefore confirmed by it. Who
can resist an idea which is presented to him in a general form in every
moment of his life and of which he finds no instance to the contrary?
The greater part of what is sometimes regarded as the a priori intuition
of space is really the conception of the various geometrical figures of
which the properties have been revealed by mathematical analysis. And
the certainty of these properties is immeasurably increased to us by our
finding that they hold good not only in every instance, but in all the
consequences which are supposed to flow from them.

Neither must we forget that our idea of space, like our other ideas,
has a history. The Homeric poems contain no word for it; even the later
Greek philosophy has not the Kantian notion of space, but only the
definite 'place' or 'the infinite.' To Plato, in the Timaeus, it is
known only as the 'nurse of generation.' When therefore we speak of
the necessity of our ideas of space we must remember that this is a
necessity which has grown up with the growth of the human mind, and
has been made by ourselves. We can free ourselves from the perplexities
which are involved in it by ascending to a time in which they did not
as yet exist. And when space or time are described as 'a priori forms or
intuitions added to the matter given in sensation,' we should consider
that such expressions belong really to the 'pre-historic study' of
philosophy, i.e. to the eighteenth century, when men sought to explain
the human mind without regard to history or language or the social
nature of man.

In every act of sense there is a latent perception of space, of which
we only become conscious when objects are withdrawn from it. There are
various ways in which we may trace the connexion between them. We may
think of space as unresisting matter, and of matter as divided into
objects; or of objects again as formed by abstraction into a collective
notion of matter, and of matter as rarefied into space. And motion may
be conceived as the union of there and not there in space, and force
as the materializing or solidification of motion. Space again is the
individual and universal in one; or, in other words, a perception and
also a conception. So easily do what are sometimes called our simple
ideas pass into one another, and differences of kind resolve themselves
into differences of degree.

Within or behind space there is another abstraction in many respects
similar to it--time, the form of the inward, as space is the form of the
outward. As we cannot think of outward objects of sense or of outward
sensations without space, so neither can we think of a succession of
sensations without time. It is the vacancy of thoughts or sensations,
as space is the void of outward objects, and we can no more imagine
the mind without the one than the world without the other. It is to
arithmetic what space is to geometry; or, more strictly, arithmetic may
be said to be equally applicable to both. It is defined in our minds,
partly by the analogy of space and partly by the recollection of events
which have happened to us, or the consciousness of feelings which we are
experiencing. Like space, it is without limit, for whatever beginning or
end of time we fix, there is a beginning and end before them, and so
on without end. We speak of a past, present, and future, and again the
analogy of space assists us in conceiving of them as coexistent. When
the limit of time is removed there arises in our minds the idea of
eternity, which at first, like time itself, is only negative, but
gradually, when connected with the world and the divine nature, like
the other negative infinity of space, becomes positive. Whether time is
prior to the mind and to experience, or coeval with them, is (like
the parallel question about space) unmeaning. Like space it has been
realized gradually: in the Homeric poems, or even in the Hesiodic
cosmogony, there is no more notion of time than of space. The conception
of being is more general than either, and might therefore with greater
plausibility be affirmed to be a condition or quality of the mind. The a
priori intuitions of Kant would have been as unintelligible to Plato as
his a priori synthetical propositions to Aristotle. The philosopher of
Konigsberg supposed himself to be analyzing a necessary mode of thought:
he was not aware that he was dealing with a mere abstraction. But now
that we are able to trace the gradual developement of ideas through
religion, through language, through abstractions, why should we
interpose the fiction of time between ourselves and realities? Why
should we single out one of these abstractions to be the a priori
condition of all the others? It comes last and not first in the order of
our thoughts, and is not the condition precedent of them, but the last
generalization of them. Nor can any principle be imagined more suicidal
to philosophy than to assume that all the truth which we are capable of
attaining is seen only through an unreal medium. If all that exists
in time is illusion, we may well ask with Plato, 'What becomes of the
mind?'

Leaving the a priori conditions of sensation we may proceed to consider
acts of sense. These admit of various degrees of duration or intensity;
they admit also of a greater or less extension from one object, which is
perceived directly, to many which are perceived indirectly or in a less
degree, and to the various associations of the object which are latent
in the mind. In general the greater the intension the less the extension
of them. The simplest sensation implies some relation of objects to
one another, some position in space, some relation to a previous or
subsequent sensation. The acts of seeing and hearing may be almost
unconscious and may pass away unnoted; they may also leave an impression
behind them or power of recalling them. If, after seeing an object we
shut our eyes, the object remains dimly seen in the same or about the
same place, but with form and lineaments half filled up. This is the
simplest act of memory. And as we cannot see one thing without at
the same time seeing another, different objects hang together in
recollection, and when we call for one the other quickly follows. To
think of the place in which we have last seen a thing is often the
best way of recalling it to the mind. Hence memory is dependent on
association. The act of recollection may be compared to the sight of an
object at a great distance which we have previously seen near and seek
to bring near to us in thought. Memory is to sense as dreaming is to
waking; and like dreaming has a wayward and uncertain power of recalling
impressions from the past.

Thus begins the passage from the outward to the inward sense. But as
yet there is no conception of a universal--the mind only remembers
the individual object or objects, and is always attaching to them some
colour or association of sense. The power of recollection seems to
depend on the intensity or largeness of the perception, or on the
strength of some emotion with which it is inseparably connected. This is
the natural memory which is allied to sense, such as children appear to
have and barbarians and animals. It is necessarily limited in range, and
its limitation is its strength. In later life, when the mind has become
crowded with names, acts, feelings, images innumerable, we acquire
by education another memory of system and arrangement which is both
stronger and weaker than the first--weaker in the recollection
of sensible impressions as they are represented to us by eye or
ear--stronger by the natural connexion of ideas with objects or with one
another. And many of the notions which form a part of the train of our
thoughts are hardly realized by us at the time, but, like numbers or
algebraical symbols, are used as signs only, thus lightening the labour
of recollection.

And now we may suppose that numerous images present themselves to the
mind, which begins to act upon them and to arrange them in various
ways. Besides the impression of external objects present with us or just
absent from us, we have a dimmer conception of other objects which have
disappeared from our immediate recollection and yet continue to exist in
us. The mind is full of fancies which are passing to and fro before it.
Some feeling or association calls them up, and they are uttered by the
lips. This is the first rudimentary imagination, which may be truly
described in the language of Hobbes, as 'decaying sense,' an expression
which may be applied with equal truth to memory as well. For memory and
imagination, though we sometimes oppose them, are nearly allied; the
difference between them seems chiefly to lie in the activity of the one
compared with the passivity of the other. The sense decaying in memory
receives a flash of light or life from imagination. Dreaming is a link
of connexion between them; for in dreaming we feebly recollect and also
feebly imagine at one and the same time. When reason is asleep the
lower part of the mind wanders at will amid the images which have been
received from without, the intelligent element retires, and the sensual
or sensuous takes its place. And so in the first efforts of imagination
reason is latent or set aside; and images, in part disorderly, but also
having a unity (however imperfect) of their own, pour like a flood over
the mind. And if we could penetrate into the heads of animals we should
probably find that their intelligence, or the state of what in them is
analogous to our intelligence, is of this nature.

Thus far we have been speaking of men, rather in the points in which
they resemble animals than in the points in which they differ from
them. The animal too has memory in various degrees, and the elements
of imagination, if, as appears to be the case, he dreams. How far their
powers or instincts are educated by the circumstances of their lives
or by intercourse with one another or with mankind, we cannot precisely
tell. They, like ourselves, have the physical inheritance of form,
scent, hearing, sight, and other qualities or instincts. But they
have not the mental inheritance of thoughts and ideas handed down by
tradition, 'the slow additions that build up the mind' of the human
race. And language, which is the great educator of mankind, is wanting
in them; whereas in us language is ever present--even in the infant the
latent power of naming is almost immediately observable. And therefore
the description which has been already given of the nascent power of
the faculties is in reality an anticipation. For simultaneous with their
growth in man a growth of language must be supposed. The child of two
years old sees the fire once and again, and the feeble observation of
the same recurring object is associated with the feeble utterance of the
name by which he is taught to call it. Soon he learns to utter the name
when the object is no longer there, but the desire or imagination of it
is present to him. At first in every use of the word there is a colour
of sense, an indistinct picture of the object which accompanies it. But
in later years he sees in the name only the universal or class word, and
the more abstract the notion becomes, the more vacant is the image
which is presented to him. Henceforward all the operations of his mind,
including the perceptions of sense, are a synthesis of sensations,
words, conceptions. In seeing or hearing or looking or listening the
sensible impression prevails over the conception and the word. In
reflection the process is reversed--the outward object fades away into
nothingness, the name or the conception or both together are everything.
Language, like number, is intermediate between the two, partaking of the
definiteness of the outer and of the universality of the inner world.
For logic teaches us that every word is really a universal, and only
condescends by the help of position or circumlocution to become the
expression of individuals or particulars. And sometimes by using words
as symbols we are able to give a 'local habitation and a name' to the
infinite and inconceivable.

Thus we see that no line can be drawn between the powers of sense and
of reflection--they pass imperceptibly into one another. We may indeed
distinguish between the seeing and the closed eye--between the sensation
and the recollection of it. But this distinction carries us a very
little way, for recollection is present in sight as well as sight
in recollection. There is no impression of sense which does not
simultaneously recall differences of form, number, colour, and the like.
Neither is such a distinction applicable at all to our internal bodily
sensations, which give no sign of themselves when unaccompanied with
pain, and even when we are most conscious of them, have often no
assignable place in the human frame. Who can divide the nerves or great
nervous centres from the mind which uses them? Who can separate the
pains and pleasures of the mind from the pains and pleasures of the
body? The words 'inward and outward,' 'active and passive,' 'mind and
body,' are best conceived by us as differences of degree passing into
differences of kind, and at one time and under one aspect acting in
harmony and then again opposed. They introduce a system and order into
the knowledge of our being; and yet, like many other general terms, are
often in advance of our actual analysis or observation.

According to some writers the inward sense is only the fading away or
imperfect realization of the outward. But this leaves out of sight one
half of the phenomenon. For the mind is not only withdrawn from
the world of sense but introduced to a higher world of thought and
reflection, in which, like the outward sense, she is trained and
educated. By use the outward sense becomes keener and more intense,
especially when confined within narrow limits. The savage with little
or no thought has a quicker discernment of the track than the civilised
man; in like manner the dog, having the help of scent as well as of
sight, is superior to the savage. By use again the inward thought
becomes more defined and distinct; what was at first an effort is made
easy by the natural instrumentality of language, and the mind learns to
grasp universals with no more exertion than is required for the sight of
an outward object. There is a natural connexion and arrangement of them,
like the association of objects in a landscape. Just as a note or two of
music suffices to recall a whole piece to the musician's or composer's
mind, so a great principle or leading thought suggests and arranges a
world of particulars. The power of reflection is not feebler than the
faculty of sense, but of a higher and more comprehensive nature. It not
only receives the universals of sense, but gives them a new content by
comparing and combining them with one another. It withdraws from the
seen that it may dwell in the unseen. The sense only presents us with
a flat and impenetrable surface: the mind takes the world to pieces and
puts it together on a new pattern. The universals which are detached
from sense are reconstructed in science. They and not the mere
impressions of sense are the truth of the world in which we live; and
(as an argument to those who will only believe 'what they can hold in
their hands') we may further observe that they are the source of our
power over it. To say that the outward sense is stronger than the
inward is like saying that the arm of the workman is stronger than the
constructing or directing mind.

Returning to the senses we may briefly consider two questions--first
their relation to the mind, secondly, their relation to outward
objects:--

1. The senses are not merely 'holes set in a wooden horse' (Theaet.),
but instruments of the mind with which they are organically connected.
There is no use of them without some use of words--some natural or
latent logic--some previous experience or observation. Sensation, like
all other mental processes, is complex and relative, though apparently
simple. The senses mutually confirm and support one another; it is hard
to say how much our impressions of hearing may be affected by those
of sight, or how far our impressions of sight may be corrected by the
touch, especially in infancy. The confirmation of them by one another
cannot of course be given by any one of them. Many intuitions which are
inseparable from the act of sense are really the result of complicated
reasonings. The most cursory glance at objects enables the experienced
eye to judge approximately of their relations and distance, although
nothing is impressed upon the retina except colour, including gradations
of light and shade. From these delicate and almost imperceptible
differences we seem chiefly to derive our ideas of distance and
position. By comparison of what is near with what is distant we learn
that the tree, house, river, etc. which are a long way off are objects
of a like nature with those which are seen by us in our immediate
neighbourhood, although the actual impression made on the eye is very
different in one case and in the other. This is a language of 'large and
small letters' (Republic), slightly differing in form and exquisitely
graduated by distance, which we are learning all our life long, and
which we attain in various degrees according to our powers of sight or
observation. There is nor the consideration. The greater or less strain
upon the nerves of the eye or ear is communicated to the mind and
silently informs the judgment. We have also the use not of one eye only,
but of two, which give us a wider range, and help us to discern, by the
greater or less acuteness of the angle which the rays of sight form,
the distance of an object and its relation to other objects. But we are
already passing beyond the limits of our actual knowledge on a subject
which has given rise to many conjectures. More important than the
addition of another conjecture is the observation, whether in the case
of sight or of any other sense, of the great complexity of the causes
and the great simplicity of the effect.

The sympathy of the mind and the ear is no less striking than
the sympathy of the mind and the eye. Do we not seem to perceive
instinctively and as an act of sense the differences of articulate
speech and of musical notes? Yet how small a part of speech or of music
is produced by the impression of the ear compared with that which is
furnished by the mind!

Again: the more refined faculty of sense, as in animals so also in man,
seems often to be transmitted by inheritance. Neither must we forget
that in the use of the senses, as in his whole nature, man is a social
being, who is always being educated by language, habit, and the teaching
of other men as well as by his own observation. He knows distance
because he is taught it by a more experienced judgment than his own; he
distinguishes sounds because he is told to remark them by a person of
a more discerning ear. And as we inherit from our parents or other
ancestors peculiar powers of sense or feeling, so we improve and
strengthen them, not only by regular teaching, but also by sympathy and
communion with other persons.

2. The second question, namely, that concerning the relation of the mind
to external objects, is really a trifling one, though it has been made
the subject of a famous philosophy. We may if we like, with Berkeley,
resolve objects of sense into sensations; but the change is one of name
only, and nothing is gained and something is lost by such a resolution
or confusion of them. For we have not really made a single step towards
idealism, and any arbitrary inversion of our ordinary modes of speech is
disturbing to the mind. The youthful metaphysician is delighted at his
marvellous discovery that nothing is, and that what we see or feel is
our sensation only: for a day or two the world has a new interest to
him; he alone knows the secret which has been communicated to him by the
philosopher, that mind is all--when in fact he is going out of his mind
in the first intoxication of a great thought. But he soon finds that
all things remain as they were--the laws of motion, the properties of
matter, the qualities of substances. After having inflicted his theories
on any one who is willing to receive them 'first on his father and
mother, secondly on some other patient listener, thirdly on his dog,'
he finds that he only differs from the rest of mankind in the use of a
word. He had once hoped that by getting rid of the solidity of matter he
might open a passage to worlds beyond. He liked to think of the world as
the representation of the divine nature, and delighted to imagine angels
and spirits wandering through space, present in the room in which he is
sitting without coming through the door, nowhere and everywhere at the
same instant. At length he finds that he has been the victim of his own
fancies; he has neither more nor less evidence of the supernatural than
he had before. He himself has become unsettled, but the laws of the
world remain fixed as at the beginning. He has discovered that his
appeal to the fallibility of sense was really an illusion. For whatever
uncertainty there may be in the appearances of nature, arises only out
of the imperfection or variation of the human senses, or possibly from
the deficiency of certain branches of knowledge; when science is able to
apply her tests, the uncertainty is at an end. We are apt sometimes
to think that moral and metaphysical philosophy are lowered by the
influence which is exercised over them by physical science. But any
interpretation of nature by physical science is far in advance of such
idealism. The philosophy of Berkeley, while giving unbounded license to
the imagination, is still grovelling on the level of sense.

We may, if we please, carry this scepticism a step further, and
deny, not only objects of sense, but the continuity of our sensations
themselves. We may say with Protagoras and Hume that what is appears,
and that what appears appears only to individuals, and to the same
individual only at one instant. But then, as Plato asks,--and we must
repeat the question,--What becomes of the mind? Experience tells us by a
thousand proofs that our sensations of colour, taste, and the like,
are the same as they were an instant ago--that the act which we are
performing one minute is continued by us in the next--and also
supplies abundant proof that the perceptions of other men are, speaking
generally, the same or nearly the same with our own. After having slowly
and laboriously in the course of ages gained a conception of a whole and
parts, of the constitution of the mind, of the relation of man to God
and nature, imperfect indeed, but the best we can, we are asked to
return again to the 'beggarly elements' of ancient scepticism, and
acknowledge only atoms and sensations devoid of life or unity. Why
should we not go a step further still and doubt the existence of the
senses of all things? We are but 'such stuff as dreams are made of;'
for we have left ourselves no instruments of thought by which we can
distinguish man from the animals, or conceive of the existence even of a
mollusc. And observe, this extreme scepticism has been allowed to spring
up among us, not, like the ancient scepticism, in an age when nature and
language really seemed to be full of illusions, but in the eighteenth
and nineteenth centuries, when men walk in the daylight of inductive
science.

The attractiveness of such speculations arises out of their true nature
not being perceived. They are veiled in graceful language; they are not
pushed to extremes; they stop where the human mind is disposed also to
stop--short of a manifest absurdity. Their inconsistency is not observed
by their authors or by mankind in general, who are equally inconsistent
themselves. They leave on the mind a pleasing sense of wonder and
novelty: in youth they seem to have a natural affinity to one class of
persons as poetry has to another; but in later life either we drift
back into common sense, or we make them the starting-points of a higher
philosophy.

We are often told that we should enquire into all things before we
accept them;--with what limitations is this true? For we cannot use
our senses without admitting that we have them, or think without
presupposing that there is in us a power of thought, or affirm that all
knowledge is derived from experience without implying that this first
principle of knowledge is prior to experience. The truth seems to be
that we begin with the natural use of the mind as of the body, and
we seek to describe this as well as we can. We eat before we know the
nature of digestion; we think before we know the nature of reflection.
As our knowledge increases, our perception of the mind enlarges also. We
cannot indeed get beyond facts, but neither can we draw any line which
separates facts from ideas. And the mind is not something separate
from them but included in them, and they in the mind, both having a
distinctness and individuality of their own. To reduce our conception of
mind to a succession of feelings and sensations is like the attempt to
view a wide prospect by inches through a microscope, or to calculate a
period of chronology by minutes. The mind ceases to exist when it loses
its continuity, which though far from being its highest determination,
is yet necessary to any conception of it. Even an inanimate nature
cannot be adequately represented as an endless succession of states or
conditions.

Paragraph II. Another division of the subject has yet to be considered:
Why should the doctrine that knowledge is sensation, in ancient times,
or of sensationalism or materialism in modern times, be allied to the
lower rather than to the higher view of ethical philosophy? At
first sight the nature and origin of knowledge appear to be wholly
disconnected from ethics and religion, nor can we deny that the ancient
Stoics were materialists, or that the materialist doctrines prevalent
in modern times have been associated with great virtues, or that both
religious and philosophical idealism have not unfrequently parted
company with practice. Still upon the whole it must be admitted that the
higher standard of duty has gone hand in hand with the higher conception
of knowledge. It is Protagoras who is seeking to adapt himself to
the opinions of the world; it is Plato who rises above them: the one
maintaining that all knowledge is sensation; the other basing the
virtues on the idea of good. The reason of this phenomenon has now to be
examined.

By those who rest knowledge immediately upon sense, that explanation of
human action is deemed to be the truest which is nearest to sense. As
knowledge is reduced to sensation, so virtue is reduced to feeling,
happiness or good to pleasure. The different virtues--the various
characters which exist in the world--are the disguises of self-interest.
Human nature is dried up; there is no place left for imagination, or in
any higher sense for religion. Ideals of a whole, or of a state, or of a
law of duty, or of a divine perfection, are out of place in an Epicurean
philosophy. The very terms in which they are expressed are suspected of
having no meaning. Man is to bring himself back as far as he is able to
the condition of a rational beast. He is to limit himself to the pursuit
of pleasure, but of this he is to make a far-sighted calculation;--he is
to be rationalized, secularized, animalized: or he is to be an amiable
sceptic, better than his own philosophy, and not falling below the
opinions of the world.

Imagination has been called that 'busy faculty' which is always
intruding upon us in the search after truth. But imagination is also
that higher power by which we rise above ourselves and the commonplaces
of thought and life. The philosophical imagination is another name for
reason finding an expression of herself in the outward world. To deprive
life of ideals is to deprive it of all higher and comprehensive aims and
of the power of imparting and communicating them to others. For men are
taught, not by those who are on a level with them, but by those who rise
above them, who see the distant hills, who soar into the empyrean. Like
a bird in a cage, the mind confined to sense is always being brought
back from the higher to the lower, from the wider to the narrower view
of human knowledge. It seeks to fly but cannot: instead of aspiring
towards perfection, 'it hovers about this lower world and the earthly
nature.' It loses the religious sense which more than any other seems to
take a man out of himself. Weary of asking 'What is truth?' it accepts
the 'blind witness of eyes and ears;' it draws around itself the curtain
of the physical world and is satisfied. The strength of a sensational
philosophy lies in the ready accommodation of it to the minds of men;
many who have been metaphysicians in their youth, as they advance in
years are prone to acquiesce in things as they are, or rather appear to
be. They are spectators, not thinkers, and the best philosophy is that
which requires of them the least amount of mental effort.

As a lower philosophy is easier to apprehend than a higher, so a lower
way of life is easier to follow; and therefore such a philosophy seems
to derive a support from the general practice of mankind. It appeals to
principles which they all know and recognize: it gives back to them in a
generalized form the results of their own experience. To the man of the
world they are the quintessence of his own reflections upon life. To
follow custom, to have no new ideas or opinions, not to be straining
after impossibilities, to enjoy to-day with just so much forethought as
is necessary to provide for the morrow, this is regarded by the greater
part of the world as the natural way of passing through existence. And
many who have lived thus have attained to a lower kind of happiness
or equanimity. They have possessed their souls in peace without ever
allowing them to wander into the region of religious or political
controversy, and without any care for the higher interests of man. But
nearly all the good (as well as some of the evil) which has ever been
done in this world has been the work of another spirit, the work of
enthusiasts and idealists, of apostles and martyrs. The leaders
of mankind have not been of the gentle Epicurean type; they have
personified ideas; they have sometimes also been the victims of them.
But they have always been seeking after a truth or ideal of which they
fell short; and have died in a manner disappointed of their hopes that
they might lift the human race out of the slough in which they found
them. They have done little compared with their own visions and
aspirations; but they have done that little, only because they sought to
do, and once perhaps thought that they were doing, a great deal more.

The philosophies of Epicurus or Hume give no adequate or dignified
conception of the mind. There is no organic unity in a succession of
feeling or sensations; no comprehensiveness in an infinity of separate
actions. The individual never reflects upon himself as a whole; he can
hardly regard one act or part of his life as the cause or effect of any
other act or part. Whether in practice or speculation, he is to himself
only in successive instants. To such thinkers, whether in ancient or in
modern times, the mind is only the poor recipient of impressions--not
the heir of all the ages, or connected with all other minds. It
begins again with its own modicum of experience having only such vague
conceptions of the wisdom of the past as are inseparable from language
and popular opinion. It seeks to explain from the experience of the
individual what can only be learned from the history of the world. It
has no conception of obligation, duty, conscience--these are to the
Epicurean or Utilitarian philosopher only names which interfere with our
natural perceptions of pleasure and pain.

There seem then to be several answers to the question, Why the theory
that all knowledge is sensation is allied to the lower rather than to
the higher view of ethical philosophy:--1st, Because it is easier to
understand and practise; 2ndly, Because it is fatal to the pursuit of
ideals, moral, political, or religious; 3rdly, Because it deprives us of
the means and instruments of higher thought, of any adequate conception
of the mind, of knowledge, of conscience, of moral obligation.

...

ON THE NATURE AND LIMITS Of PSYCHOLOGY.

     O gar arche men o me oide, teleute de kai ta metaxu ex ou me
     oide sumpeplektai, tis mechane ten toiauten omologian pote
     epistemen genesthai; Plato Republic.

     Monon gar auto legeiv, osper gumnon kai aperemomenon apo ton
     onton apanton, adunaton.  Soph.

Since the above essay first appeared, many books on Psychology have been
given to the world, partly based upon the views of Herbart and other
German philosophers, partly independent of them. The subject has gained
in bulk and extent; whether it has had any true growth is more doubtful.
It begins to assume the language and claim the authority of a science;
but it is only an hypothesis or outline, which may be filled up in many
ways according to the fancy of individual thinkers. The basis of it is
a precarious one,--consciousness of ourselves and a somewhat uncertain
observation of the rest of mankind. Its relations to other sciences
are not yet determined: they seem to be almost too complicated to be
ascertained. It may be compared to an irregular building, run up hastily
and not likely to last, because its foundations are weak, and in many
places rest only on the surface of the ground. It has sought rather to
put together scattered observations and to make them into a system than
to describe or prove them. It has never severely drawn the line between
facts and opinions. It has substituted a technical phraseology for the
common use of language, being neither able to win acceptance for the one
nor to get rid of the other.

The system which has thus arisen appears to be a kind of metaphysic
narrowed to the point of view of the individual mind, through which, as
through some new optical instrument limiting the sphere of vision, the
interior of thought and sensation is examined. But the individual mind
in the abstract, as distinct from the mind of a particular individual
and separated from the environment of circumstances, is a fiction only.
Yet facts which are partly true gather around this fiction and are
naturally described by the help of it. There is also a common type of
the mind which is derived from the comparison of many minds with one
another and with our own. The phenomena of which Psychology treats are
familiar to us, but they are for the most part indefinite; they relate
to a something inside the body, which seems also to overleap the limits
of space. The operations of this something, when isolated, cannot be
analyzed by us or subjected to observation and experiment. And there is
another point to be considered. The mind, when thinking, cannot
survey that part of itself which is used in thought. It can only
be contemplated in the past, that is to say, in the history of the
individual or of the world. This is the scientific method of studying
the mind. But Psychology has also some other supports, specious rather
than real. It is partly sustained by the false analogy of Physical
Science and has great expectations from its near relationship to
Physiology. We truly remark that there is an infinite complexity of
the body corresponding to the infinite subtlety of the mind; we are
conscious that they are very nearly connected. But in endeavouring to
trace the nature of the connexion we are baffled and disappointed. In
our knowledge of them the gulf remains the same: no microscope has ever
seen into thought; no reflection on ourselves has supplied the missing
link between mind and matter...These are the conditions of this very
inexact science, and we shall only know less of it by pretending to
know more, or by assigning to it a form or style to which it has not yet
attained and is not really entitled.

Experience shows that any system, however baseless and ineffectual, in
our own or in any other age, may be accepted and continue to be studied,
if it seeks to satisfy some unanswered question or is based upon some
ancient tradition, especially if it takes the form and uses the language
of inductive philosophy. The fact therefore that such a science exists
and is popular, affords no evidence of its truth or value. Many who have
pursued it far into detail have never examined the foundations on which
it rests. The have been many imaginary subjects of knowledge of which
enthusiastic persons have made a lifelong study, without ever asking
themselves what is the evidence for them, what is the use of them, how
long they will last? They may pass away, like the authors of them, and
'leave not a wrack behind;' or they may survive in fragments. Nor is it
only in the Middle Ages, or in the literary desert of China or of India,
that such systems have arisen; in our own enlightened age, growing up
by the side of Physics, Ethics, and other really progressive sciences,
there is a weary waste of knowledge, falsely so-called. There are sham
sciences which no logic has ever put to the test, in which the desire
for knowledge invents the materials of it.

And therefore it is expedient once more to review the bases of
Psychology, lest we should be imposed upon by its pretensions. The
study of it may have done good service by awakening us to the sense of
inveterate errors familiarized by language, yet it may have fallen into
still greater ones; under the pretence of new investigations it may be
wasting the lives of those who are engaged in it. It may also be found
that the discussion of it will throw light upon some points in the
Theaetetus of Plato,--the oldest work on Psychology which has come down
to us. The imaginary science may be called, in the language of ancient
philosophy, 'a shadow of a part of Dialectic or Metaphysic' (Gorg.).

In this postscript or appendix we propose to treat, first, of the true
bases of Psychology; secondly, of the errors into which the students of
it are most likely to fall; thirdly, of the principal subjects which
are usually comprehended under it; fourthly, of the form which facts
relating to the mind most naturally assume.

We may preface the enquiry by two or three remarks:--

(1) We do not claim for the popular Psychology the position of a science
at all; it cannot, like the Physical Sciences, proceed by the Inductive
Method: it has not the necessity of Mathematics: it does not, like
Metaphysic, argue from abstract notions or from internal coherence. It
is made up of scattered observations. A few of these, though they may
sometimes appear to be truisms, are of the greatest value, and free
from all doubt. We are conscious of them in ourselves; we observe them
working in others; we are assured of them at all times. For example, we
are absolutely certain, (a) of the influence exerted by the mind over
the body or by the body over the mind: (b) of the power of association,
by which the appearance of some person or the occurrence of some event
recalls to mind, not always but often, other persons and events: (c)
of the effect of habit, which is strongest when least disturbed by
reflection, and is to the mind what the bones are to the body: (d) of
the real, though not unlimited, freedom of the human will: (e) of the
reference, more or less distinct, of our sensations, feelings, thoughts,
actions, to ourselves, which is called consciousness, or, when in
excess, self-consciousness: (f) of the distinction of the 'I' and 'Not
I,' of ourselves and outward objects. But when we attempt to gather up
these elements in a single system, we discover that the links by which
we combine them are apt to be mere words. We are in a country which
has never been cleared or surveyed; here and there only does a gleam of
light come through the darkness of the forest.

(2) These fragments, although they can never become science in the
ordinary sense of the word, are a real part of knowledge and may be of
great value in education. We may be able to add a good deal to them from
our own experience, and we may verify them by it. Self-examination
is one of those studies which a man can pursue alone, by attention to
himself and the processes of his individual mind. He may learn much
about his own character and about the character of others, if he will
'make his mind sit down' and look at itself in the glass. The great, if
not the only use of such a study is a practical one,--to know, first,
human nature, and, secondly, our own nature, as it truly is.

(3) Hence it is important that we should conceive of the mind in the
noblest and simplest manner. While acknowledging that language has been
the greatest factor in the formation of human thought, we must endeavour
to get rid of the disguises, oppositions, contradictions, which
arise out of it. We must disengage ourselves from the ideas which the
customary use of words has implanted in us. To avoid error as much as
possible when we are speaking of things unseen, the principal terms
which we use should be few, and we should not allow ourselves to be
enslaved by them. Instead of seeking to frame a technical language,
we should vary our forms of speech, lest they should degenerate into
formulas. A difficult philosophical problem is better understood when
translated into the vernacular.

I.a. Psychology is inseparable from language, and early language
contains the first impressions or the oldest experience of man
respecting himself. These impressions are not accurate representations
of the truth; they are the reflections of a rudimentary age of
philosophy. The first and simplest forms of thought are rooted so deep
in human nature that they can never be got rid of; but they have been
perpetually enlarged and elevated, and the use of many words has been
transferred from the body to the mind. The spiritual and intellectual
have thus become separated from the material--there is a cleft between
them; and the heart and the conscience of man rise above the dominion
of the appetites and create a new language in which they too find
expression. As the differences of actions begin to be perceived, more
and more names are needed. This is the first analysis of the human mind;
having a general foundation in popular experience, it is moulded to
a certain extent by hierophants and philosophers. (See Introd. to
Cratylus.)

b. This primitive psychology is continually receiving additions from the
first thinkers, who in return take a colour from the popular language
of the time. The mind is regarded from new points of view, and becomes
adapted to new conditions of knowledge. It seeks to isolate itself
from matter and sense, and to assert its independence in thought. It
recognizes that it is independent of the external world. It has five
or six natural states or stages:--(1) sensation, in which it is almost
latent or quiescent: (2) feeling, or inner sense, when the mind is just
awakening: (3) memory, which is decaying sense, and from time to time,
as with a spark or flash, has the power of recollecting or reanimating
the buried past: (4) thought, in which images pass into abstract notions
or are intermingled with them: (5) action, in which the mind moves
forward, of itself, or under the impulse of want or desire or pain,
to attain or avoid some end or consequence: and (6) there is the
composition of these or the admixture or assimilation of them in various
degrees. We never see these processes of the mind, nor can we tell the
causes of them. But we know them by their results, and learn from other
men that so far as we can describe to them or they to us the workings of
the mind, their experience is the same or nearly the same with our own.

c. But the knowledge of the mind is not to any great extent derived
from the observation of the individual by himself. It is the growing
consciousness of the human race, embodied in language, acknowledged
by experience, and corrected from time to time by the influence of
literature and philosophy. A great, perhaps the most important, part of
it is to be found in early Greek thought. In the Theaetetus of Plato it
has not yet become fixed: we are still stumbling on the threshold.
In Aristotle the process is more nearly completed, and has gained
innumerable abstractions, of which many have had to be thrown away
because relative only to the controversies of the time. In the interval
between Thales and Aristotle were realized the distinctions of mind and
body, of universal and particular, of infinite and infinitesimal, of
idea and phenomenon; the class conceptions of faculties and virtues, the
antagonism of the appetites and the reason; and connected with this, at
a higher stage of development, the opposition of moral and intellectual
virtue; also the primitive conceptions of unity, being, rest, motion,
and the like. These divisions were not really scientific, but rather
based on popular experience. They were not held with the precision of
modern thinkers, but taken all together they gave a new existence to the
mind in thought, and greatly enlarged and more accurately defined man's
knowledge of himself and of the world. The majority of them have been
accepted by Christian and Western nations. Yet in modern times we have
also drifted so far away from Aristotle, that if we were to frame a
system on his lines we should be at war with ordinary language and
untrue to our own consciousness. And there have been a few both in
mediaeval times and since the Reformation who have rebelled against the
Aristotelian point of view. Of these eccentric thinkers there have been
various types, but they have all a family likeness. According to them,
there has been too much analysis and too little synthesis, too much
division of the mind into parts and too little conception of it as a
whole or in its relation to God and the laws of the universe. They have
thought that the elements of plurality and unity have not been duly
adjusted. The tendency of such writers has been to allow the personality
of man to be absorbed in the universal, or in the divine nature, and to
deny the distinction between matter and mind, or to substitute one for
the other. They have broken some of the idols of Psychology: they have
challenged the received meaning of words: they have regarded the mind
under many points of view. But though they may have shaken the old, they
have not established the new; their views of philosophy, which seem like
the echo of some voice from the East, have been alien to the mind of
Europe.

d. The Psychology which is found in common language is in some degree
verified by experience, but not in such a manner as to give it
the character of an exact science. We cannot say that words always
correspond to facts. Common language represents the mind from different
and even opposite points of view, which cannot be all of them equally
true (compare Cratylus). Yet from diversity of statements and opinions
may be obtained a nearer approach to the truth than is to be gained
from any one of them. It also tends to correct itself, because it is
gradually brought nearer to the common sense of mankind. There are
some leading categories or classifications of thought, which, though
unverified, must always remain the elements from which the science or
study of the mind proceeds. For example, we must assume ideas before we
can analyze them, and also a continuing mind to which they belong;
the resolution of it into successive moments, which would say, with
Protagoras, that the man is not the same person which he was a minute
ago, is, as Plato implies in the Theaetetus, an absurdity.

e. The growth of the mind, which may be traced in the histories of
religions and philosophies and in the thoughts of nations, is one of the
deepest and noblest modes of studying it. Here we are dealing with the
reality, with the greater and, as it may be termed, the most sacred part
of history. We study the mind of man as it begins to be inspired by a
human or divine reason, as it is modified by circumstances, as it is
distributed in nations, as it is renovated by great movements, which go
beyond the limits of nations and affect human society on a scale still
greater, as it is created or renewed by great minds, who, looking down
from above, have a wider and more comprehensive vision. This is an
ambitious study, of which most of us rather 'entertain conjecture'
than arrive at any detailed or accurate knowledge. Later arises the
reflection how these great ideas or movements of the world have
been appropriated by the multitude and found a way to the minds of
individuals. The real Psychology is that which shows how the increasing
knowledge of nature and the increasing experience of life have always
been slowly transforming the mind, how religions too have been modified
in the course of ages 'that God may be all and in all.' E pollaplasion,
eoe, to ergon e os nun zeteitai prostatteis.

f. Lastly, though we speak of the study of mind in a special sense, it
may also be said that there is no science which does not contribute to
our knowledge of it. The methods of science and their analogies are new
faculties, discovered by the few and imparted to the many. They are
to the mind, what the senses are to the body; or better, they may be
compared to instruments such as the telescope or microscope by which the
discriminating power of the senses, or to other mechanical inventions,
by which the strength and skill of the human body is so immeasurably
increased.

II. The new Psychology, whatever may be its claim to the authority of a
science, has called attention to many facts and corrected many errors,
which without it would have been unexamined. Yet it is also itself
very liable to illusion. The evidence on which it rests is vague and
indefinite. The field of consciousness is never seen by us as a whole,
but only at particular points, which are always changing. The veil
of language intercepts facts. Hence it is desirable that in making an
approach to the study we should consider at the outset what are the
kinds of error which most easily affect it, and note the differences
which separate it from other branches of knowledge.

a. First, we observe the mind by the mind. It would seem therefore that
we are always in danger of leaving out the half of that which is the
subject of our enquiry. We come at once upon the difficulty of what is
the meaning of the word. Does it differ as subject and object in the
same manner? Can we suppose one set of feelings or one part of the mind
to interpret another? Is the introspecting thought the same with the
thought which is introspected? Has the mind the power of surveying its
whole domain at one and the same time?--No more than the eye can take in
the whole human body at a glance. Yet there may be a glimpse round the
corner, or a thought transferred in a moment from one point of view to
another, which enables us to see nearly the whole, if not at once,
at any rate in succession. Such glimpses will hardly enable us to
contemplate from within the mind in its true proportions. Hence the
firmer ground of Psychology is not the consciousness of inward feelings
but the observation of external actions, being the actions not only of
ourselves, but of the innumerable persons whom we come across in life.

b. The error of supposing partial or occasional explanation of
mental phenomena to be the only or complete ones. For example, we are
disinclined to admit of the spontaneity or discontinuity of the mind--it
seems to us like an effect without a cause, and therefore we suppose the
train of our thoughts to be always called up by association. Yet it is
probable, or indeed certain, that of many mental phenomena there are no
mental antecedents, but only bodily ones.

c. The false influence of language. We are apt to suppose that when
there are two or more words describing faculties or processes of the
mind, there are real differences corresponding to them. But this is not
the case. Nor can we determine how far they do or do not exist, or
by what degree or kind of difference they are distinguished. The same
remark may be made about figures of speech. They fill up the vacancy of
knowledge; they are to the mind what too much colour is to the eye; but
the truth is rather concealed than revealed by them.

d. The uncertain meaning of terms, such as Consciousness, Conscience,
Will, Law, Knowledge, Internal and External Sense; these, in the
language of Plato, 'we shamelessly use, without ever having taken the
pains to analyze them.'

e. A science such as Psychology is not merely an hypothesis, but
an hypothesis which, unlike the hypotheses of Physics, can never be
verified. It rests only on the general impressions of mankind, and there
is little or no hope of adding in any considerable degree to our stock
of mental facts.

f. The parallelism of the Physical Sciences, which leads us to analyze
the mind on the analogy of the body, and so to reduce mental operations
to the level of bodily ones, or to confound one with the other.

g. That the progress of Physiology may throw a new light on Psychology
is a dream in which scientific men are always tempted to indulge. But
however certain we may be of the connexion between mind and body, the
explanation of the one by the other is a hidden place of nature which
has hitherto been investigated with little or no success.

h. The impossibility of distinguishing between mind and body. Neither
in thought nor in experience can we separate them. They seem to act
together; yet we feel that we are sometimes under the dominion of the
one, sometimes of the other, and sometimes, both in the common use of
language and in fact, they transform themselves, the one into the good
principle, the other into the evil principle; and then again the 'I'
comes in and mediates between them. It is also difficult to distinguish
outward facts from the ideas of them in the mind, or to separate the
external stimulus to a sensation from the activity of the organ, or this
from the invisible agencies by which it reaches the mind, or any process
of sense from its mental antecedent, or any mental energy from its
nervous expression.

i. The fact that mental divisions tend to run into one another, and that
in speaking of the mind we cannot always distinguish differences of kind
from differences of degree; nor have we any measure of the strength and
intensity of our ideas or feelings.

j. Although heredity has been always known to the ancients as well as
ourselves to exercise a considerable influence on human character, yet
we are unable to calculate what proportion this birth-influence bears to
nurture and education. But this is the real question. We cannot pursue
the mind into embryology: we can only trace how, after birth, it begins
to grow. But how much is due to the soil, how much to the original
latent seed, it is impossible to distinguish. And because we are certain
that heredity exercises a considerable, but undefined influence, we must
not increase the wonder by exaggerating it.

k. The love of system is always tending to prevail over the historical
investigation of the mind, which is our chief means of knowing it. It
equally tends to hinder the other great source of our knowledge of the
mind, the observation of its workings and processes which we can make
for ourselves.

l. The mind, when studied through the individual, is apt to be
isolated--this is due to the very form of the enquiry; whereas, in
truth, it is indistinguishable from circumstances, the very language
which it uses being the result of the instincts of long-forgotten
generations, and every word which a man utters being the answer to some
other word spoken or suggested by somebody else.

III. The tendency of the preceding remarks has been to show that
Psychology is necessarily a fragment, and is not and cannot be a
connected system. We cannot define or limit the mind, but we can
describe it. We can collect information about it; we can enumerate the
principal subjects which are included in the study of it. Thus we are
able to rehabilitate Psychology to some extent, not as a branch of
science, but as a collection of facts bearing on human life, as a
part of the history of philosophy, as an aspect of Metaphysic. It is a
fragment of a science only, which in all probability can never make any
great progress or attain to much clearness or exactness. It is however
a kind of knowledge which has a great interest for us and is always
present to us, and of which we carry about the materials in our own
bosoms. We can observe our minds and we can experiment upon them, and
the knowledge thus acquired is not easily forgotten, and is a help to us
in study as well as in conduct.

The principal subjects of Psychology may be summed up as follows:--

a. The relation of man to the world around him,--in what sense and
within what limits can he withdraw from its laws or assert himself
against them (Freedom and Necessity), and what is that which we suppose
to be thus independent and which we call ourselves? How does the inward
differ from the outward and what is the relation between them, and where
do we draw the line by which we separate mind from matter, the soul
from the body? Is the mind active or passive, or partly both? Are its
movements identical with those of the body, or only preconcerted and
coincident with them, or is one simply an aspect of the other?

b. What are we to think of time and space? Time seems to have a nearer
connexion with the mind, space with the body; yet time, as well as
space, is necessary to our idea of either. We see also that they have
an analogy with one another, and that in Mathematics they often
interpenetrate. Space or place has been said by Kant to be the form of
the outward, time of the inward sense. He regards them as parts or
forms of the mind. But this is an unfortunate and inexpressive way of
describing their relation to us. For of all the phenomena present to the
human mind they seem to have most the character of objective existence.
There is no use in asking what is beyond or behind them; we cannot get
rid of them. And to throw the laws of external nature which to us are
the type of the immutable into the subjective side of the antithesis
seems to be equally inappropriate.

c. When in imagination we enter into the closet of the mind and withdraw
ourselves from the external world, we seem to find there more or less
distinct processes which may be described by the words, 'I perceive,' 'I
feel,' 'I think,' 'I want,' 'I wish,' 'I like,' 'I dislike,' 'I fear,'
'I know,' 'I remember,' 'I imagine,' 'I dream,' 'I act,' 'I endeavour,'
'I hope.' These processes would seem to have the same notions attached
to them in the minds of all educated persons. They are distinguished
from one another in thought, but they intermingle. It is possible to
reflect upon them or to become conscious of them in a greater or less
degree, or with a greater or less continuity or attention, and thus
arise the intermittent phenomena of consciousness or self-consciousness.
The use of all of them is possible to us at all times; and therefore
in any operation of the mind the whole are latent. But we are able to
characterise them sufficiently by that part of the complex action which
is the most prominent. We have no difficulty in distinguishing an act
of sight or an act of will from an act of thought, although thought is
present in both of them. Hence the conception of different faculties or
different virtues is precarious, because each of them is passing into
the other, and they are all one in the mind itself; they appear and
reappear, and may all be regarded as the ever-varying phases or aspects
or differences of the same mind or person.

d. Nearest the sense in the scale of the intellectual faculties
is memory, which is a mode rather than a faculty of the mind, and
accompanies all mental operations. There are two principal kinds of it,
recollection and recognition,--recollection in which forgotten things
are recalled or return to the mind, recognition in which the mind finds
itself again among things once familiar. The simplest way in which we
can represent the former to ourselves is by shutting our eyes and trying
to recall in what we term the mind's eye the picture of the
surrounding scene, or by laying down the book which we are reading and
recapitulating what we can remember of it. But many times more powerful
than recollection is recognition, perhaps because it is more assisted by
association. We have known and forgotten, and after a long interval the
thing which we have seen once is seen again by us, but with a different
feeling, and comes back to us, not as new knowledge, but as a thing to
which we ourselves impart a notion already present to us; in Plato's
words, we set the stamp upon the wax. Every one is aware of the
difference between the first and second sight of a place, between a
scene clothed with associations or bare and divested of them. We say to
ourselves on revisiting a spot after a long interval: How many things
have happened since I last saw this! There is probably no impression
ever received by us of which we can venture to say that the vestiges are
altogether lost, or that we might not, under some circumstances, recover
it. A long-forgotten knowledge may be easily renewed and therefore is
very different from ignorance. Of the language learnt in childhood not
a word may be remembered, and yet, when a new beginning is made, the
old habit soon returns, the neglected organs come back into use, and the
river of speech finds out the dried-up channel.

e. 'Consciousness' is the most treacherous word which is employed in
the study of the mind, for it is used in many senses, and has rarely,
if ever, been minutely analyzed. Like memory, it accompanies all mental
operations, but not always continuously, and it exists in various
degrees. It may be imperceptible or hardly perceptible: it may be the
living sense that our thoughts, actions, sufferings, are our own. It
is a kind of attention which we pay to ourselves, and is intermittent
rather than continuous. Its sphere has been exaggerated. It is sometimes
said to assure us of our freedom; but this is an illusion: as there
may be a real freedom without consciousness of it, so there may be a
consciousness of freedom without the reality. It may be regarded as a
higher degree of knowledge when we not only know but know that we know.
Consciousness is opposed to habit, inattention, sleep, death. It may be
illustrated by its derivative conscience, which speaks to men, not only
of right and wrong in the abstract, but of right and wrong actions in
reference to themselves and their circumstances.

f. Association is another of the ever-present phenomena of the human
mind. We speak of the laws of association, but this is an expression
which is confusing, for the phenomenon itself is of the most capricious
and uncertain sort. It may be briefly described as follows. The simplest
case of association is that of sense. When we see or hear separately
one of two things, which we have previously seen or heard together, the
occurrence of the one has a tendency to suggest the other. So the sight
or name of a house may recall to our minds the memory of those who once
lived there. Like may recall like and everything its opposite. The parts
of a whole, the terms of a series, objects lying near, words having
a customary order stick together in the mind. A word may bring back a
passage of poetry or a whole system of philosophy; from one end of the
world or from one pole of knowledge we may travel to the other in an
indivisible instant. The long train of association by which we pass from
one point to the other, involving every sort of complex relation, so
sudden, so accidental, is one of the greatest wonders of mind...This
process however is not always continuous, but often intermittent: we can
think of things in isolation as well as in association; we do not mean
that they must all hang from one another. We can begin again after an
interval of rest or vacancy, as a new train of thought suddenly arises,
as, for example, when we wake of a morning or after violent exercise.
Time, place, the same colour or sound or smell or taste, will often
call up some thought or recollection either accidentally or naturally
associated with them. But it is equally noticeable that the new thought
may occur to us, we cannot tell how or why, by the spontaneous action of
the mind itself or by the latent influence of the body. Both science and
poetry are made up of associations or recollections, but we must observe
also that the mind is not wholly dependent on them, having also the
power of origination.

There are other processes of the mind which it is good for us to study
when we are at home and by ourselves,--the manner in which thought
passes into act, the conflict of passion and reason in many stages, the
transition from sensuality to love or sentiment and from earthly love to
heavenly, the slow and silent influence of habit, which little by little
changes the nature of men, the sudden change of the old nature of man
into a new one, wrought by shame or by some other overwhelming impulse.
These are the greater phenomena of mind, and he who has thought of
them for himself will live and move in a better-ordered world, and will
himself be a better-ordered man.

At the other end of the 'globus intellectualis,' nearest, not to earth
and sense, but to heaven and God, is the personality of man, by which
he holds communion with the unseen world. Somehow, he knows not how,
somewhere, he knows not where, under this higher aspect of his being he
grasps the ideas of God, freedom and immortality; he sees the forms of
truth, holiness and love, and is satisfied with them. No account of the
mind can be complete which does not admit the reality or the possibility
of another life. Whether regarded as an ideal or as a fact, the highest
part of man's nature and that in which it seems most nearly to approach
the divine, is a phenomenon which exists, and must therefore be included
within the domain of Psychology.

IV. We admit that there is no perfect or ideal Psychology. It is not a
whole in the same sense in which Chemistry, Physiology, or Mathematics
are wholes: that is to say, it is not a connected unity of knowledge.
Compared with the wealth of other sciences, it rests upon a small number
of facts; and when we go beyond these, we fall into conjectures and
verbal discussions. The facts themselves are disjointed; the causes of
them run up into other sciences, and we have no means of tracing
them from one to the other. Yet it may be true of this, as of other
beginnings of knowledge, that the attempt to put them together has
tested the truth of them, and given a stimulus to the enquiry into them.

Psychology should be natural, not technical. It should take the form
which is the most intelligible to the common understanding, because it
has to do with common things, which are familiar to us all. It should
aim at no more than every reflecting man knows or can easily verify for
himself. When simple and unpretentious, it is least obscured by words,
least liable to fall under the influence of Physiology or Metaphysic.
It should argue, not from exceptional, but from ordinary phenomena. It
should be careful to distinguish the higher and the lower elements of
human nature, and not allow one to be veiled in the disguise of the
other, lest through the slippery nature of language we should pass
imperceptibly from good to evil, from nature in the higher to nature in
the neutral or lower sense. It should assert consistently the unity of
the human faculties, the unity of knowledge, the unity of God and law.
The difference between the will and the affections and between the
reason and the passions should also be recognized by it.

Its sphere is supposed to be narrowed to the individual soul; but it
cannot be thus separated in fact. It goes back to the beginnings of
things, to the first growth of language and philosophy, and to the whole
science of man. There can be no truth or completeness in any study of
the mind which is confined to the individual. The nature of language,
though not the whole, is perhaps at present the most important element
in our knowledge of it. It is not impossible that some numerical laws
may be found to have a place in the relations of mind and matter, as in
the rest of nature. The old Pythagorean fancy that the soul 'is or has
in it harmony' may in some degree be realized. But the indications
of such numerical harmonies are faint; either the secret of them lies
deeper than we can discover, or nature may have rebelled against the use
of them in the composition of men and animals. It is with qualitative
rather than with quantitative differences that we are concerned
in Psychology. The facts relating to the mind which we obtain from
Physiology are negative rather than positive. They show us, not the
processes of mental action, but the conditions of which when deprived
the mind ceases to act. It would seem as if the time had not yet arrived
when we can hope to add anything of much importance to our knowledge
of the mind from the investigations of the microscope. The elements of
Psychology can still only be learnt from reflections on ourselves, which
interpret and are also interpreted by our experience of others. The
history of language, of philosophy, and religion, the great thoughts or
inventions or discoveries which move mankind, furnish the larger moulds
or outlines in which the human mind has been cast. From these the
individual derives so much as he is able to comprehend or has the
opportunity of learning.




THEAETETUS


PERSONS OF THE DIALOGUE: Socrates, Theodorus, Theaetetus.

Euclid and Terpsion meet in front of Euclid's house in Megara; they
enter the house, and the dialogue is read to them by a servant.


EUCLID: Have you only just arrived from the country, Terpsion?

TERPSION: No, I came some time ago: and I have been in the Agora looking
for you, and wondering that I could not find you.

EUCLID: But I was not in the city.

TERPSION: Where then?

EUCLID: As I was going down to the harbour, I met Theaetetus--he was
being carried up to Athens from the army at Corinth.

TERPSION: Was he alive or dead?

EUCLID: He was scarcely alive, for he has been badly wounded; but he was
suffering even more from the sickness which has broken out in the army.

TERPSION: The dysentery, you mean?

EUCLID: Yes.

TERPSION: Alas! what a loss he will be!

EUCLID: Yes, Terpsion, he is a noble fellow; only to-day I heard some
people highly praising his behaviour in this very battle.

TERPSION: No wonder; I should rather be surprised at hearing anything
else of him. But why did he go on, instead of stopping at Megara?

EUCLID: He wanted to get home: although I entreated and advised him to
remain, he would not listen to me; so I set him on his way, and turned
back, and then I remembered what Socrates had said of him, and thought
how remarkably this, like all his predictions, had been fulfilled.
I believe that he had seen him a little before his own death, when
Theaetetus was a youth, and he had a memorable conversation with him,
which he repeated to me when I came to Athens; he was full of admiration
of his genius, and said that he would most certainly be a great man, if
he lived.

TERPSION: The prophecy has certainly been fulfilled; but what was the
conversation? can you tell me?

EUCLID: No, indeed, not offhand; but I took notes of it as soon as I got
home; these I filled up from memory, writing them out at leisure; and
whenever I went to Athens, I asked Socrates about any point which I had
forgotten, and on my return I made corrections; thus I have nearly the
whole conversation written down.

TERPSION: I remember--you told me; and I have always been intending to
ask you to show me the writing, but have put off doing so; and now, why
should we not read it through?--having just come from the country, I
should greatly like to rest.

EUCLID: I too shall be very glad of a rest, for I went with Theaetetus
as far as Erineum. Let us go in, then, and, while we are reposing, the
servant shall read to us.

TERPSION: Very good.

EUCLID: Here is the roll, Terpsion; I may observe that I have introduced
Socrates, not as narrating to me, but as actually conversing with the
persons whom he mentioned--these were, Theodorus the geometrician (of
Cyrene), and Theaetetus. I have omitted, for the sake of convenience,
the interlocutory words 'I said,' 'I remarked,' which he used when he
spoke of himself, and again, 'he agreed,' or 'disagreed,' in the answer,
lest the repetition of them should be troublesome.

TERPSION: Quite right, Euclid.

EUCLID: And now, boy, you may take the roll and read.

EUCLID'S SERVANT READS.

SOCRATES: If I cared enough about the Cyrenians, Theodorus, I would ask
you whether there are any rising geometricians or philosophers in that
part of the world. But I am more interested in our own Athenian youth,
and I would rather know who among them are likely to do well. I observe
them as far as I can myself, and I enquire of any one whom they follow,
and I see that a great many of them follow you, in which they are quite
right, considering your eminence in geometry and in other ways. Tell me
then, if you have met with any one who is good for anything.

THEODORUS: Yes, Socrates, I have become acquainted with one very
remarkable Athenian youth, whom I commend to you as well worthy of your
attention. If he had been a beauty I should have been afraid to praise
him, lest you should suppose that I was in love with him; but he is no
beauty, and you must not be offended if I say that he is very like you;
for he has a snub nose and projecting eyes, although these features are
less marked in him than in you. Seeing, then, that he has no personal
attractions, I may freely say, that in all my acquaintance, which is
very large, I never knew any one who was his equal in natural gifts: for
he has a quickness of apprehension which is almost unrivalled, and he
is exceedingly gentle, and also the most courageous of men; there is a
union of qualities in him such as I have never seen in any other, and
should scarcely have thought possible; for those who, like him, have
quick and ready and retentive wits, have generally also quick tempers;
they are ships without ballast, and go darting about, and are mad rather
than courageous; and the steadier sort, when they have to face study,
prove stupid and cannot remember. Whereas he moves surely and smoothly
and successfully in the path of knowledge and enquiry; and he is full of
gentleness, flowing on silently like a river of oil; at his age, it is
wonderful.

SOCRATES: That is good news; whose son is he?

THEODORUS: The name of his father I have forgotten, but the youth
himself is the middle one of those who are approaching us; he and his
companions have been anointing themselves in the outer court, and now
they seem to have finished, and are coming towards us. Look and see
whether you know him.

SOCRATES: I know the youth, but I do not know his name; he is the son of
Euphronius the Sunian, who was himself an eminent man, and such another
as his son is, according to your account of him; I believe that he left
a considerable fortune.

THEODORUS: Theaetetus, Socrates, is his name; but I rather think that
the property disappeared in the hands of trustees; notwithstanding which
he is wonderfully liberal.

SOCRATES: He must be a fine fellow; tell him to come and sit by me.

THEODORUS: I will. Come hither, Theaetetus, and sit by Socrates.

SOCRATES: By all means, Theaetetus, in order that I may see the
reflection of myself in your face, for Theodorus says that we are alike;
and yet if each of us held in his hands a lyre, and he said that they
were tuned alike, should we at once take his word, or should we ask
whether he who said so was or was not a musician?

THEAETETUS: We should ask.

SOCRATES: And if we found that he was, we should take his word; and if
not, not?

THEAETETUS: True.

SOCRATES: And if this supposed likeness of our faces is a matter of any
interest to us, we should enquire whether he who says that we are alike
is a painter or not?

THEAETETUS: Certainly we should.

SOCRATES: And is Theodorus a painter?

THEAETETUS: I never heard that he was.

SOCRATES: Is he a geometrician?

THEAETETUS: Of course he is, Socrates.

SOCRATES: And is he an astronomer and calculator and musician, and in
general an educated man?

THEAETETUS: I think so.

SOCRATES: If, then, he remarks on a similarity in our persons, either
by way of praise or blame, there is no particular reason why we should
attend to him.

THEAETETUS: I should say not.

SOCRATES: But if he praises the virtue or wisdom which are the mental
endowments of either of us, then he who hears the praises will naturally
desire to examine him who is praised: and he again should be willing to
exhibit himself.

THEAETETUS: Very true, Socrates.

SOCRATES: Then now is the time, my dear Theaetetus, for me to examine,
and for you to exhibit; since although Theodorus has praised many a
citizen and stranger in my hearing, never did I hear him praise any one
as he has been praising you.

THEAETETUS: I am glad to hear it, Socrates; but what if he was only in
jest?

SOCRATES: Nay, Theodorus is not given to jesting; and I cannot allow you
to retract your consent on any such pretence as that. If you do, he will
have to swear to his words; and we are perfectly sure that no one will
be found to impugn him. Do not be shy then, but stand to your word.

THEAETETUS: I suppose I must, if you wish it.

SOCRATES: In the first place, I should like to ask what you learn of
Theodorus: something of geometry, perhaps?

THEAETETUS: Yes.

SOCRATES: And astronomy and harmony and calculation?

THEAETETUS: I do my best.

SOCRATES: Yes, my boy, and so do I; and my desire is to learn of him,
or of anybody who seems to understand these things. And I get on pretty
well in general; but there is a little difficulty which I want you and
the company to aid me in investigating. Will you answer me a question:
'Is not learning growing wiser about that which you learn?'

THEAETETUS: Of course.

SOCRATES: And by wisdom the wise are wise?

THEAETETUS: Yes.

SOCRATES: And is that different in any way from knowledge?

THEAETETUS: What?

SOCRATES: Wisdom; are not men wise in that which they know?

THEAETETUS: Certainly they are.

SOCRATES: Then wisdom and knowledge are the same?

THEAETETUS: Yes.

SOCRATES: Herein lies the difficulty which I can never solve to my
satisfaction--What is knowledge? Can we answer that question? What say
you? which of us will speak first? whoever misses shall sit down, as at
a game of ball, and shall be donkey, as the boys say; he who lasts out
his competitors in the game without missing, shall be our king,
and shall have the right of putting to us any questions which he
pleases...Why is there no reply? I hope, Theodorus, that I am not
betrayed into rudeness by my love of conversation? I only want to make
us talk and be friendly and sociable.

THEODORUS: The reverse of rudeness, Socrates: but I would rather that
you would ask one of the young fellows; for the truth is, that I am
unused to your game of question and answer, and I am too old to learn;
the young will be more suitable, and they will improve more than
I shall, for youth is always able to improve. And so having made a
beginning with Theaetetus, I would advise you to go on with him and not
let him off.

SOCRATES: Do you hear, Theaetetus, what Theodorus says? The philosopher,
whom you would not like to disobey, and whose word ought to be a command
to a young man, bids me interrogate you. Take courage, then, and nobly
say what you think that knowledge is.

THEAETETUS: Well, Socrates, I will answer as you and he bid me; and if I
make a mistake, you will doubtless correct me.

SOCRATES: We will, if we can.

THEAETETUS: Then, I think that the sciences which I learn from
Theodorus--geometry, and those which you just now mentioned--are
knowledge; and I would include the art of the cobbler and other
craftsmen; these, each and all of, them, are knowledge.

SOCRATES: Too much, Theaetetus, too much; the nobility and liberality of
your nature make you give many and diverse things, when I am asking for
one simple thing.

THEAETETUS: What do you mean, Socrates?

SOCRATES: Perhaps nothing. I will endeavour, however, to explain what I
believe to be my meaning: When you speak of cobbling, you mean the art
or science of making shoes?

THEAETETUS: Just so.

SOCRATES: And when you speak of carpentering, you mean the art of making
wooden implements?

THEAETETUS: I do.

SOCRATES: In both cases you define the subject matter of each of the two
arts?

THEAETETUS: True.

SOCRATES: But that, Theaetetus, was not the point of my question: we
wanted to know not the subjects, nor yet the number of the arts or
sciences, for we were not going to count them, but we wanted to know the
nature of knowledge in the abstract. Am I not right?

THEAETETUS: Perfectly right.

SOCRATES: Let me offer an illustration: Suppose that a person were to
ask about some very trivial and obvious thing--for example, What is
clay? and we were to reply, that there is a clay of potters, there is
a clay of oven-makers, there is a clay of brick-makers; would not the
answer be ridiculous?

THEAETETUS: Truly.

SOCRATES: In the first place, there would be an absurdity in assuming
that he who asked the question would understand from our answer the
nature of 'clay,' merely because we added 'of the image-makers,' or of
any other workers. How can a man understand the name of anything, when
he does not know the nature of it?

THEAETETUS: He cannot.

SOCRATES: Then he who does not know what science or knowledge is, has no
knowledge of the art or science of making shoes?

THEAETETUS: None.

SOCRATES: Nor of any other science?

THEAETETUS: No.

SOCRATES: And when a man is asked what science or knowledge is, to
give in answer the name of some art or science is ridiculous; for the
question is, 'What is knowledge?' and he replies, 'A knowledge of this
or that.'

THEAETETUS: True.

SOCRATES: Moreover, he might answer shortly and simply, but he makes an
enormous circuit. For example, when asked about the clay, he might have
said simply, that clay is moistened earth--what sort of clay is not to
the point.

THEAETETUS: Yes, Socrates, there is no difficulty as you put the
question. You mean, if I am not mistaken, something like what occurred
to me and to my friend here, your namesake Socrates, in a recent
discussion.

SOCRATES: What was that, Theaetetus?

THEAETETUS: Theodorus was writing out for us something about roots, such
as the roots of three or five, showing that they are incommensurable by
the unit: he selected other examples up to seventeen--there he stopped.
Now as there are innumerable roots, the notion occurred to us of
attempting to include them all under one name or class.

SOCRATES: And did you find such a class?

THEAETETUS: I think that we did; but I should like to have your opinion.

SOCRATES: Let me hear.

THEAETETUS: We divided all numbers into two classes: those which are
made up of equal factors multiplying into one another, which we compared
to square figures and called square or equilateral numbers;--that was
one class.

SOCRATES: Very good.

THEAETETUS: The intermediate numbers, such as three and five, and every
other number which is made up of unequal factors, either of a greater
multiplied by a less, or of a less multiplied by a greater, and when
regarded as a figure, is contained in unequal sides;--all these we
compared to oblong figures, and called them oblong numbers.

SOCRATES: Capital; and what followed?

THEAETETUS: The lines, or sides, which have for their squares the
equilateral plane numbers, were called by us lengths or magnitudes; and
the lines which are the roots of (or whose squares are equal to) the
oblong numbers, were called powers or roots; the reason of this latter
name being, that they are commensurable with the former [i.e., with the
so-called lengths or magnitudes] not in linear measurement, but in the
value of the superficial content of their squares; and the same about
solids.

SOCRATES: Excellent, my boys; I think that you fully justify the praises
of Theodorus, and that he will not be found guilty of false witness.

THEAETETUS: But I am unable, Socrates, to give you a similar answer
about knowledge, which is what you appear to want; and therefore
Theodorus is a deceiver after all.

SOCRATES: Well, but if some one were to praise you for running, and to
say that he never met your equal among boys, and afterwards you were
beaten in a race by a grown-up man, who was a great runner--would the
praise be any the less true?

THEAETETUS: Certainly not.

SOCRATES: And is the discovery of the nature of knowledge so small a
matter, as just now said? Is it not one which would task the powers of
men perfect in every way?

THEAETETUS: By heaven, they should be the top of all perfection!

SOCRATES: Well, then, be of good cheer; do not say that Theodorus was
mistaken about you, but do your best to ascertain the true nature of
knowledge, as well as of other things.

THEAETETUS: I am eager enough, Socrates, if that would bring to light
the truth.

SOCRATES: Come, you made a good beginning just now; let your own answer
about roots be your model, and as you comprehended them all in one
class, try and bring the many sorts of knowledge under one definition.

THEAETETUS: I can assure you, Socrates, that I have tried very often,
when the report of questions asked by you was brought to me; but I can
neither persuade myself that I have a satisfactory answer to give, nor
hear of any one who answers as you would have him; and I cannot shake
off a feeling of anxiety.

SOCRATES: These are the pangs of labour, my dear Theaetetus; you have
something within you which you are bringing to the birth.

THEAETETUS: I do not know, Socrates; I only say what I feel.

SOCRATES: And have you never heard, simpleton, that I am the son of a
midwife, brave and burly, whose name was Phaenarete?

THEAETETUS: Yes, I have.

SOCRATES: And that I myself practise midwifery?

THEAETETUS: No, never.

SOCRATES: Let me tell you that I do though, my friend: but you must not
reveal the secret, as the world in general have not found me out; and
therefore they only say of me, that I am the strangest of mortals and
drive men to their wits' end. Did you ever hear that too?

THEAETETUS: Yes.

SOCRATES: Shall I tell you the reason?

THEAETETUS: By all means.

SOCRATES: Bear in mind the whole business of the midwives, and then you
will see my meaning better:--No woman, as you are probably aware, who is
still able to conceive and bear, attends other women, but only those who
are past bearing.

THEAETETUS: Yes, I know.

SOCRATES: The reason of this is said to be that Artemis--the goddess of
childbirth--is not a mother, and she honours those who are like herself;
but she could not allow the barren to be midwives, because human nature
cannot know the mystery of an art without experience; and therefore she
assigned this office to those who are too old to bear.

THEAETETUS: I dare say.

SOCRATES: And I dare say too, or rather I am absolutely certain, that
the midwives know better than others who is pregnant and who is not?

THEAETETUS: Very true.

SOCRATES: And by the use of potions and incantations they are able to
arouse the pangs and to soothe them at will; they can make those bear
who have a difficulty in bearing, and if they think fit they can smother
the embryo in the womb.

THEAETETUS: They can.

SOCRATES: Did you ever remark that they are also most cunning
matchmakers, and have a thorough knowledge of what unions are likely to
produce a brave brood?

THEAETETUS: No, never.

SOCRATES: Then let me tell you that this is their greatest pride, more
than cutting the umbilical cord. And if you reflect, you will see that
the same art which cultivates and gathers in the fruits of the earth,
will be most likely to know in what soils the several plants or seeds
should be deposited.

THEAETETUS: Yes, the same art.

SOCRATES: And do you suppose that with women the case is otherwise?

THEAETETUS: I should think not.

SOCRATES: Certainly not; but midwives are respectable women who have a
character to lose, and they avoid this department of their profession,
because they are afraid of being called procuresses, which is a name
given to those who join together man and woman in an unlawful and
unscientific way; and yet the true midwife is also the true and only
matchmaker.

THEAETETUS: Clearly.

SOCRATES: Such are the midwives, whose task is a very important one, but
not so important as mine; for women do not bring into the world at one
time real children, and at another time counterfeits which are with
difficulty distinguished from them; if they did, then the discernment of
the true and false birth would be the crowning achievement of the art of
midwifery--you would think so?

THEAETETUS: Indeed I should.

SOCRATES: Well, my art of midwifery is in most respects like theirs; but
differs, in that I attend men and not women; and look after their souls
when they are in labour, and not after their bodies: and the triumph of
my art is in thoroughly examining whether the thought which the mind of
the young man brings forth is a false idol or a noble and true birth.
And like the midwives, I am barren, and the reproach which is often
made against me, that I ask questions of others and have not the wit to
answer them myself, is very just--the reason is, that the god compels me
to be a midwife, but does not allow me to bring forth. And therefore
I am not myself at all wise, nor have I anything to show which is
the invention or birth of my own soul, but those who converse with me
profit. Some of them appear dull enough at first, but afterwards, as
our acquaintance ripens, if the god is gracious to them, they all make
astonishing progress; and this in the opinion of others as well as in
their own. It is quite dear that they never learned anything from me;
the many fine discoveries to which they cling are of their own making.
But to me and the god they owe their delivery. And the proof of my words
is, that many of them in their ignorance, either in their self-conceit
despising me, or falling under the influence of others, have gone away
too soon; and have not only lost the children of whom I had previously
delivered them by an ill bringing up, but have stifled whatever else
they had in them by evil communications, being fonder of lies and shams
than of the truth; and they have at last ended by seeing themselves, as
others see them, to be great fools. Aristeides, the son of Lysimachus,
is one of them, and there are many others. The truants often return to
me, and beg that I would consort with them again--they are ready to
go to me on their knees--and then, if my familiar allows, which is not
always the case, I receive them, and they begin to grow again. Dire
are the pangs which my art is able to arouse and to allay in those who
consort with me, just like the pangs of women in childbirth; night and
day they are full of perplexity and travail which is even worse than
that of the women. So much for them. And there are others, Theaetetus,
who come to me apparently having nothing in them; and as I know that
they have no need of my art, I coax them into marrying some one, and
by the grace of God I can generally tell who is likely to do them good.
Many of them I have given away to Prodicus, and many to other inspired
sages. I tell you this long story, friend Theaetetus, because I suspect,
as indeed you seem to think yourself, that you are in labour--great with
some conception. Come then to me, who am a midwife's son and myself a
midwife, and do your best to answer the questions which I will ask you.
And if I abstract and expose your first-born, because I discover upon
inspection that the conception which you have formed is a vain shadow,
do not quarrel with me on that account, as the manner of women is when
their first children are taken from them. For I have actually known some
who were ready to bite me when I deprived them of a darling folly; they
did not perceive that I acted from goodwill, not knowing that no god is
the enemy of man--that was not within the range of their ideas; neither
am I their enemy in all this, but it would be wrong for me to admit
falsehood, or to stifle the truth. Once more, then, Theaetetus, I repeat
my old question, 'What is knowledge?'--and do not say that you cannot
tell; but quit yourself like a man, and by the help of God you will be
able to tell.

THEAETETUS: At any rate, Socrates, after such an exhortation I should be
ashamed of not trying to do my best. Now he who knows perceives what he
knows, and, as far as I can see at present, knowledge is perception.

SOCRATES: Bravely said, boy; that is the way in which you should express
your opinion. And now, let us examine together this conception of yours,
and see whether it is a true birth or a mere wind-egg:--You say that
knowledge is perception?

THEAETETUS: Yes.

SOCRATES: Well, you have delivered yourself of a very important doctrine
about knowledge; it is indeed the opinion of Protagoras, who has another
way of expressing it. Man, he says, is the measure of all things, of the
existence of things that are, and of the non-existence of things that
are not:--You have read him?

THEAETETUS: O yes, again and again.

SOCRATES: Does he not say that things are to you such as they appear to
you, and to me such as they appear to me, and that you and I are men?

THEAETETUS: Yes, he says so.

SOCRATES: A wise man is not likely to talk nonsense. Let us try to
understand him: the same wind is blowing, and yet one of us may be cold
and the other not, or one may be slightly and the other very cold?

THEAETETUS: Quite true.

SOCRATES: Now is the wind, regarded not in relation to us but
absolutely, cold or not; or are we to say, with Protagoras, that the
wind is cold to him who is cold, and not to him who is not?

THEAETETUS: I suppose the last.

SOCRATES: Then it must appear so to each of them?

THEAETETUS: Yes.

SOCRATES: And 'appears to him' means the same as 'he perceives.'

THEAETETUS: True.

SOCRATES: Then appearing and perceiving coincide in the case of hot and
cold, and in similar instances; for things appear, or may be supposed to
be, to each one such as he perceives them?

THEAETETUS: Yes.

SOCRATES: Then perception is always of existence, and being the same as
knowledge is unerring?

THEAETETUS: Clearly.

SOCRATES: In the name of the Graces, what an almighty wise man
Protagoras must have been! He spoke these things in a parable to the
common herd, like you and me, but told the truth, 'his Truth,' (In
allusion to a book of Protagoras' which bore this title.) in secret to
his own disciples.

THEAETETUS: What do you mean, Socrates?

SOCRATES: I am about to speak of a high argument, in which all things
are said to be relative; you cannot rightly call anything by any name,
such as great or small, heavy or light, for the great will be small and
the heavy light--there is no single thing or quality, but out of motion
and change and admixture all things are becoming relatively to one
another, which 'becoming' is by us incorrectly called being, but is
really becoming, for nothing ever is, but all things are becoming.
Summon all philosophers--Protagoras, Heracleitus, Empedocles, and the
rest of them, one after another, and with the exception of Parmenides
they will agree with you in this. Summon the great masters of either
kind of poetry--Epicharmus, the prince of Comedy, and Homer of Tragedy;
when the latter sings of

'Ocean whence sprang the gods, and mother Tethys,'

does he not mean that all things are the offspring, of flux and motion?

THEAETETUS: I think so.

SOCRATES: And who could take up arms against such a great army having
Homer for its general, and not appear ridiculous? (Compare Cratylus.)

THEAETETUS: Who indeed, Socrates?

SOCRATES: Yes, Theaetetus; and there are plenty of other proofs
which will show that motion is the source of what is called being and
becoming, and inactivity of not-being and destruction; for fire and
warmth, which are supposed to be the parent and guardian of all other
things, are born of movement and of friction, which is a kind of
motion;--is not this the origin of fire?

THEAETETUS: It is.

SOCRATES: And the race of animals is generated in the same way?

THEAETETUS: Certainly.

SOCRATES: And is not the bodily habit spoiled by rest and idleness, but
preserved for a long time by motion and exercise?

THEAETETUS: True.

SOCRATES: And what of the mental habit? Is not the soul informed, and
improved, and preserved by study and attention, which are motions; but
when at rest, which in the soul only means want of attention and study,
is uninformed, and speedily forgets whatever she has learned?

THEAETETUS: True.

SOCRATES: Then motion is a good, and rest an evil, to the soul as well
as to the body?

THEAETETUS: Clearly.

SOCRATES: I may add, that breathless calm, stillness and the like waste
and impair, while wind and storm preserve; and the palmary argument of
all, which I strongly urge, is the golden chain in Homer, by which
he means the sun, thereby indicating that so long as the sun and the
heavens go round in their orbits, all things human and divine are and
are preserved, but if they were chained up and their motions ceased,
then all things would be destroyed, and, as the saying is, turned upside
down.

THEAETETUS: I believe, Socrates, that you have truly explained his
meaning.

SOCRATES: Then now apply his doctrine to perception, my good friend, and
first of all to vision; that which you call white colour is not in your
eyes, and is not a distinct thing which exists out of them. And you must
not assign any place to it: for if it had position it would be, and be
at rest, and there would be no process of becoming.

THEAETETUS: Then what is colour?

SOCRATES: Let us carry the principle which has just been affirmed, that
nothing is self-existent, and then we shall see that white, black,
and every other colour, arises out of the eye meeting the appropriate
motion, and that what we call a colour is in each case neither the
active nor the passive element, but something which passes between
them, and is peculiar to each percipient; are you quite certain that the
several colours appear to a dog or to any animal whatever as they appear
to you?

THEAETETUS: Far from it.

SOCRATES: Or that anything appears the same to you as to another man?
Are you so profoundly convinced of this? Rather would it not be true
that it never appears exactly the same to you, because you are never
exactly the same?

THEAETETUS: The latter.

SOCRATES: And if that with which I compare myself in size, or which
I apprehend by touch, were great or white or hot, it could not become
different by mere contact with another unless it actually changed; nor
again, if the comparing or apprehending subject were great or white
or hot, could this, when unchanged from within, become changed by any
approximation or affection of any other thing. The fact is that in
our ordinary way of speaking we allow ourselves to be driven into most
ridiculous and wonderful contradictions, as Protagoras and all who take
his line of argument would remark.

THEAETETUS: How? and of what sort do you mean?

SOCRATES: A little instance will sufficiently explain my meaning: Here
are six dice, which are more by a half when compared with four, and
fewer by a half than twelve--they are more and also fewer. How can you
or any one maintain the contrary?

THEAETETUS: Very true.

SOCRATES: Well, then, suppose that Protagoras or some one asks whether
anything can become greater or more if not by increasing, how would you
answer him, Theaetetus?

THEAETETUS: I should say 'No,' Socrates, if I were to speak my mind
in reference to this last question, and if I were not afraid of
contradicting my former answer.

SOCRATES: Capital! excellent! spoken like an oracle, my boy! And if you
reply 'Yes,' there will be a case for Euripides; for our tongue will be
unconvinced, but not our mind. (In allusion to the well-known line of
Euripides, Hippol.: e gloss omomoch e de thren anomotos.)

THEAETETUS: Very true.

SOCRATES: The thoroughbred Sophists, who know all that can be known
about the mind, and argue only out of the superfluity of their wits,
would have had a regular sparring-match over this, and would have
knocked their arguments together finely. But you and I, who have no
professional aims, only desire to see what is the mutual relation of
these principles,--whether they are consistent with each or not.

THEAETETUS: Yes, that would be my desire.

SOCRATES: And mine too. But since this is our feeling, and there is
plenty of time, why should we not calmly and patiently review our own
thoughts, and thoroughly examine and see what these appearances in
us really are? If I am not mistaken, they will be described by us as
follows:--first, that nothing can become greater or less, either in
number or magnitude, while remaining equal to itself--you would agree?

THEAETETUS: Yes.

SOCRATES: Secondly, that without addition or subtraction there is no
increase or diminution of anything, but only equality.

THEAETETUS: Quite true.

SOCRATES: Thirdly, that what was not before cannot be afterwards,
without becoming and having become.

THEAETETUS: Yes, truly.

SOCRATES: These three axioms, if I am not mistaken, are fighting with
one another in our minds in the case of the dice, or, again, in such a
case as this--if I were to say that I, who am of a certain height and
taller than you, may within a year, without gaining or losing in height,
be not so tall--not that I should have lost, but that you would have
increased. In such a case, I am afterwards what I once was not, and yet
I have not become; for I could not have become without becoming, neither
could I have become less without losing somewhat of my height; and I
could give you ten thousand examples of similar contradictions, if
we admit them at all. I believe that you follow me, Theaetetus; for I
suspect that you have thought of these questions before now.

THEAETETUS: Yes, Socrates, and I am amazed when I think of them; by the
Gods I am! and I want to know what on earth they mean; and there are
times when my head quite swims with the contemplation of them.

SOCRATES: I see, my dear Theaetetus, that Theodorus had a true insight
into your nature when he said that you were a philosopher, for wonder
is the feeling of a philosopher, and philosophy begins in wonder. He was
not a bad genealogist who said that Iris (the messenger of heaven)
is the child of Thaumas (wonder). But do you begin to see what is the
explanation of this perplexity on the hypothesis which we attribute to
Protagoras?

THEAETETUS: Not as yet.

SOCRATES: Then you will be obliged to me if I help you to unearth the
hidden 'truth' of a famous man or school.

THEAETETUS: To be sure, I shall be very much obliged.

SOCRATES: Take a look round, then, and see that none of the uninitiated
are listening. Now by the uninitiated I mean the people who believe in
nothing but what they can grasp in their hands, and who will not allow
that action or generation or anything invisible can have real existence.

THEAETETUS: Yes, indeed, Socrates, they are very hard and impenetrable
mortals.

SOCRATES: Yes, my boy, outer barbarians. Far more ingenious are the
brethren whose mysteries I am about to reveal to you. Their first
principle is, that all is motion, and upon this all the affections of
which we were just now speaking are supposed to depend: there is nothing
but motion, which has two forms, one active and the other passive, both
in endless number; and out of the union and friction of them there is
generated a progeny endless in number, having two forms, sense and the
object of sense, which are ever breaking forth and coming to the birth
at the same moment. The senses are variously named hearing, seeing,
smelling; there is the sense of heat, cold, pleasure, pain, desire,
fear, and many more which have names, as well as innumerable others
which are without them; each has its kindred object,--each variety
of colour has a corresponding variety of sight, and so with sound and
hearing, and with the rest of the senses and the objects akin to them.
Do you see, Theaetetus, the bearings of this tale on the preceding
argument?

THEAETETUS: Indeed I do not.

SOCRATES: Then attend, and I will try to finish the story. The purport
is that all these things are in motion, as I was saying, and that this
motion is of two kinds, a slower and a quicker; and the slower elements
have their motions in the same place and with reference to things near
them, and so they beget; but what is begotten is swifter, for it
is carried to fro, and moves from place to place. Apply this to
sense:--When the eye and the appropriate object meet together and give
birth to whiteness and the sensation connatural with it, which could not
have been given by either of them going elsewhere, then, while the
sight is flowing from the eye, whiteness proceeds from the object which
combines in producing the colour; and so the eye is fulfilled with
sight, and really sees, and becomes, not sight, but a seeing eye;
and the object which combined to form the colour is fulfilled with
whiteness, and becomes not whiteness but a white thing, whether wood or
stone or whatever the object may be which happens to be coloured white.
And this is true of all sensible objects, hard, warm, and the like,
which are similarly to be regarded, as I was saying before, not as
having any absolute existence, but as being all of them of whatever kind
generated by motion in their intercourse with one another; for of the
agent and patient, as existing in separation, no trustworthy conception,
as they say, can be formed, for the agent has no existence until united
with the patient, and the patient has no existence until united with
the agent; and that which by uniting with something becomes an agent, by
meeting with some other thing is converted into a patient. And from
all these considerations, as I said at first, there arises a general
reflection, that there is no one self-existent thing, but everything
is becoming and in relation; and being must be altogether abolished,
although from habit and ignorance we are compelled even in this
discussion to retain the use of the term. But great philosophers tell us
that we are not to allow either the word 'something,' or 'belonging to
something,' or 'to me,' or 'this,' or 'that,' or any other detaining
name to be used, in the language of nature all things are being created
and destroyed, coming into being and passing into new forms; nor can any
name fix or detain them; he who attempts to fix them is easily refuted.
And this should be the way of speaking, not only of particulars but
of aggregates; such aggregates as are expressed in the word 'man,' or
'stone,' or any name of an animal or of a class. O Theaetetus, are not
these speculations sweet as honey? And do you not like the taste of them
in the mouth?

THEAETETUS: I do not know what to say, Socrates; for, indeed, I cannot
make out whether you are giving your own opinion or only wanting to draw
me out.

SOCRATES: You forget, my friend, that I neither know, nor profess to
know, anything of these matters; you are the person who is in labour, I
am the barren midwife; and this is why I soothe you, and offer you one
good thing after another, that you may taste them. And I hope that I may
at last help to bring your own opinion into the light of day: when this
has been accomplished, then we will determine whether what you have
brought forth is only a wind-egg or a real and genuine birth. Therefore,
keep up your spirits, and answer like a man what you think.

THEAETETUS: Ask me.

SOCRATES: Then once more: Is it your opinion that nothing is but what
becomes?--the good and the noble, as well as all the other things which
we were just now mentioning?

THEAETETUS: When I hear you discoursing in this style, I think that
there is a great deal in what you say, and I am very ready to assent.

SOCRATES: Let us not leave the argument unfinished, then; for there
still remains to be considered an objection which may be raised about
dreams and diseases, in particular about madness, and the various
illusions of hearing and sight, or of other senses. For you know that
in all these cases the esse-percipi theory appears to be unmistakably
refuted, since in dreams and illusions we certainly have false
perceptions; and far from saying that everything is which appears, we
should rather say that nothing is which appears.

THEAETETUS: Very true, Socrates.

SOCRATES: But then, my boy, how can any one contend that knowledge is
perception, or that to every man what appears is?

THEAETETUS: I am afraid to say, Socrates, that I have nothing to answer,
because you rebuked me just now for making this excuse; but I certainly
cannot undertake to argue that madmen or dreamers think truly, when they
imagine, some of them that they are gods, and others that they can fly,
and are flying in their sleep.

SOCRATES: Do you see another question which can be raised about these
phenomena, notably about dreaming and waking?

THEAETETUS: What question?

SOCRATES: A question which I think that you must often have heard
persons ask:--How can you determine whether at this moment we are
sleeping, and all our thoughts are a dream; or whether we are awake, and
talking to one another in the waking state?

THEAETETUS: Indeed, Socrates, I do not know how to prove the one
any more than the other, for in both cases the facts precisely
correspond;--and there is no difficulty in supposing that during all
this discussion we have been talking to one another in a dream; and when
in a dream we seem to be narrating dreams, the resemblance of the two
states is quite astonishing.

SOCRATES: You see, then, that a doubt about the reality of sense is
easily raised, since there may even be a doubt whether we are awake
or in a dream. And as our time is equally divided between sleeping
and waking, in either sphere of existence the soul contends that the
thoughts which are present to our minds at the time are true; and during
one half of our lives we affirm the truth of the one, and, during the
other half, of the other; and are equally confident of both.

THEAETETUS: Most true.

SOCRATES: And may not the same be said of madness and other disorders?
the difference is only that the times are not equal.

THEAETETUS: Certainly.

SOCRATES: And is truth or falsehood to be determined by duration of
time?

THEAETETUS: That would be in many ways ridiculous.

SOCRATES: But can you certainly determine by any other means which of
these opinions is true?

THEAETETUS: I do not think that I can.

SOCRATES: Listen, then, to a statement of the other side of the
argument, which is made by the champions of appearance. They would say,
as I imagine--Can that which is wholly other than something, have the
same quality as that from which it differs? and observe, Theaetetus,
that the word 'other' means not 'partially,' but 'wholly other.'

THEAETETUS: Certainly, putting the question as you do, that which is
wholly other cannot either potentially or in any other way be the same.

SOCRATES: And must therefore be admitted to be unlike?

THEAETETUS: True.

SOCRATES: If, then, anything happens to become like or unlike itself or
another, when it becomes like we call it the same--when unlike, other?

THEAETETUS: Certainly.

SOCRATES: Were we not saying that there are agents many and infinite,
and patients many and infinite?

THEAETETUS: Yes.

SOCRATES: And also that different combinations will produce results
which are not the same, but different?

THEAETETUS: Certainly.

SOCRATES: Let us take you and me, or anything as an example:--There is
Socrates in health, and Socrates sick--Are they like or unlike?

THEAETETUS: You mean to compare Socrates in health as a whole, and
Socrates in sickness as a whole?

SOCRATES: Exactly; that is my meaning.

THEAETETUS: I answer, they are unlike.

SOCRATES: And if unlike, they are other?

THEAETETUS: Certainly.

SOCRATES: And would you not say the same of Socrates sleeping and
waking, or in any of the states which we were mentioning?

THEAETETUS: I should.

SOCRATES: All agents have a different patient in Socrates, accordingly
as he is well or ill.

THEAETETUS: Of course.

SOCRATES: And I who am the patient, and that which is the agent, will
produce something different in each of the two cases?

THEAETETUS: Certainly.

SOCRATES: The wine which I drink when I am in health, appears sweet and
pleasant to me?

THEAETETUS: True.

SOCRATES: For, as has been already acknowledged, the patient and agent
meet together and produce sweetness and a perception of sweetness, which
are in simultaneous motion, and the perception which comes from the
patient makes the tongue percipient, and the quality of sweetness which
arises out of and is moving about the wine, makes the wine both to be
and to appear sweet to the healthy tongue.

THEAETETUS: Certainly; that has been already acknowledged.

SOCRATES: But when I am sick, the wine really acts upon another and a
different person?

THEAETETUS: Yes.

SOCRATES: The combination of the draught of wine, and the Socrates
who is sick, produces quite another result; which is the sensation of
bitterness in the tongue, and the motion and creation of bitterness in
and about the wine, which becomes not bitterness but something bitter;
as I myself become not perception but percipient?

THEAETETUS: True.

SOCRATES: There is no other object of which I shall ever have the same
perception, for another object would give another perception, and would
make the percipient other and different; nor can that object which
affects me, meeting another subject, produce the same, or become
similar, for that too would produce another result from another subject,
and become different.

THEAETETUS: True.

SOCRATES: Neither can I by myself, have this sensation, nor the object
by itself, this quality.

THEAETETUS: Certainly not.

SOCRATES: When I perceive I must become percipient of something--there
can be no such thing as perceiving and perceiving nothing; the object,
whether it become sweet, bitter, or of any other quality, must have
relation to a percipient; nothing can become sweet which is sweet to no
one.

THEAETETUS: Certainly not.

SOCRATES: Then the inference is, that we (the agent and patient) are or
become in relation to one another; there is a law which binds us one to
the other, but not to any other existence, nor each of us to himself;
and therefore we can only be bound to one another; so that whether
a person says that a thing is or becomes, he must say that it is or
becomes to or of or in relation to something else; but he must not
say or allow any one else to say that anything is or becomes
absolutely:--such is our conclusion.

THEAETETUS: Very true, Socrates.

SOCRATES: Then, if that which acts upon me has relation to me and to no
other, I and no other am the percipient of it?

THEAETETUS: Of course.

SOCRATES: Then my perception is true to me, being inseparable from my
own being; and, as Protagoras says, to myself I am judge of what is and
what is not to me.

THEAETETUS: I suppose so.

SOCRATES: How then, if I never err, and if my mind never trips in the
conception of being or becoming, can I fail of knowing that which I
perceive?

THEAETETUS: You cannot.

SOCRATES: Then you were quite right in affirming that knowledge is only
perception; and the meaning turns out to be the same, whether with Homer
and Heracleitus, and all that company, you say that all is motion and
flux, or with the great sage Protagoras, that man is the measure of all
things; or with Theaetetus, that, given these premises, perception is
knowledge. Am I not right, Theaetetus, and is not this your new-born
child, of which I have delivered you? What say you?

THEAETETUS: I cannot but agree, Socrates.

SOCRATES: Then this is the child, however he may turn out, which you and
I have with difficulty brought into the world. And now that he is born,
we must run round the hearth with him, and see whether he is worth
rearing, or is only a wind-egg and a sham. Is he to be reared in any
case, and not exposed? or will you bear to see him rejected, and not get
into a passion if I take away your first-born?

THEODORUS: Theaetetus will not be angry, for he is very good-natured.
But tell me, Socrates, in heaven's name, is this, after all, not the
truth?

SOCRATES: You, Theodorus, are a lover of theories, and now you
innocently fancy that I am a bag full of them, and can easily pull one
out which will overthrow its predecessor. But you do not see that in
reality none of these theories come from me; they all come from him who
talks with me. I only know just enough to extract them from the wisdom
of another, and to receive them in a spirit of fairness. And now I shall
say nothing myself, but shall endeavour to elicit something from our
young friend.

THEODORUS: Do as you say, Socrates; you are quite right.

SOCRATES: Shall I tell you, Theodorus, what amazes me in your
acquaintance Protagoras?

THEODORUS: What is it?

SOCRATES: I am charmed with his doctrine, that what appears is to
each one, but I wonder that he did not begin his book on Truth with a
declaration that a pig or a dog-faced baboon, or some other yet stranger
monster which has sensation, is the measure of all things; then he might
have shown a magnificent contempt for our opinion of him by informing
us at the outset that while we were reverencing him like a God for
his wisdom he was no better than a tadpole, not to speak of his
fellow-men--would not this have produced an overpowering effect? For
if truth is only sensation, and no man can discern another's feelings
better than he, or has any superior right to determine whether his
opinion is true or false, but each, as we have several times repeated,
is to himself the sole judge, and everything that he judges is true and
right, why, my friend, should Protagoras be preferred to the place
of wisdom and instruction, and deserve to be well paid, and we poor
ignoramuses have to go to him, if each one is the measure of his own
wisdom? Must he not be talking 'ad captandum' in all this? I say nothing
of the ridiculous predicament in which my own midwifery and the whole
art of dialectic is placed; for the attempt to supervise or refute the
notions or opinions of others would be a tedious and enormous piece of
folly, if to each man his own are right; and this must be the case if
Protagoras' Truth is the real truth, and the philosopher is not merely
amusing himself by giving oracles out of the shrine of his book.

THEODORUS: He was a friend of mine, Socrates, as you were saying, and
therefore I cannot have him refuted by my lips, nor can I oppose you
when I agree with you; please, then, to take Theaetetus again; he seemed
to answer very nicely.

SOCRATES: If you were to go into a Lacedaemonian palestra, Theodorus,
would you have a right to look on at the naked wrestlers, some of them
making a poor figure, if you did not strip and give them an opportunity
of judging of your own person?

THEODORUS: Why not, Socrates, if they would allow me, as I think you
will, in consideration of my age and stiffness; let some more supple
youth try a fall with you, and do not drag me into the gymnasium.

SOCRATES: Your will is my will, Theodorus, as the proverbial
philosophers say, and therefore I will return to the sage Theaetetus:
Tell me, Theaetetus, in reference to what I was saying, are you not
lost in wonder, like myself, when you find that all of a sudden you are
raised to the level of the wisest of men, or indeed of the gods?--for
you would assume the measure of Protagoras to apply to the gods as well
as men?

THEAETETUS: Certainly I should, and I confess to you that I am lost in
wonder. At first hearing, I was quite satisfied with the doctrine, that
whatever appears is to each one, but now the face of things has changed.

SOCRATES: Why, my dear boy, you are young, and therefore your ear
is quickly caught and your mind influenced by popular arguments.
Protagoras, or some one speaking on his behalf, will doubtless say in
reply,--Good people, young and old, you meet and harangue, and bring
in the gods, whose existence or non-existence I banish from writing and
speech, or you talk about the reason of man being degraded to the level
of the brutes, which is a telling argument with the multitude, but not
one word of proof or demonstration do you offer. All is probability with
you, and yet surely you and Theodorus had better reflect whether you
are disposed to admit of probability and figures of speech in matters
of such importance. He or any other mathematician who argued from
probabilities and likelihoods in geometry, would not be worth an ace.

THEAETETUS: But neither you nor we, Socrates, would be satisfied with
such arguments.

SOCRATES: Then you and Theodorus mean to say that we must look at the
matter in some other way?

THEAETETUS: Yes, in quite another way.

SOCRATES: And the way will be to ask whether perception is or is not the
same as knowledge; for this was the real point of our argument, and with
a view to this we raised (did we not?) those many strange questions.

THEAETETUS: Certainly.

SOCRATES: Shall we say that we know every thing which we see and hear?
for example, shall we say that not having learned, we do not hear the
language of foreigners when they speak to us? or shall we say that
we not only hear, but know what they are saying? Or again, if we see
letters which we do not understand, shall we say that we do not see
them? or shall we aver that, seeing them, we must know them?

THEAETETUS: We shall say, Socrates, that we know what we actually see
and hear of them--that is to say, we see and know the figure and colour
of the letters, and we hear and know the elevation or depression of the
sound of them; but we do not perceive by sight and hearing, or know,
that which grammarians and interpreters teach about them.

SOCRATES: Capital, Theaetetus; and about this there shall be no dispute,
because I want you to grow; but there is another difficulty coming,
which you will also have to repulse.

THEAETETUS: What is it?

SOCRATES: Some one will say, Can a man who has ever known anything, and
still has and preserves a memory of that which he knows, not know that
which he remembers at the time when he remembers? I have, I fear, a
tedious way of putting a simple question, which is only, whether a man
who has learned, and remembers, can fail to know?

THEAETETUS: Impossible, Socrates; the supposition is monstrous.

SOCRATES: Am I talking nonsense, then? Think: is not seeing perceiving,
and is not sight perception?

THEAETETUS: True.

SOCRATES: And if our recent definition holds, every man knows that which
he has seen?

THEAETETUS: Yes.

SOCRATES: And you would admit that there is such a thing as memory?

THEAETETUS: Yes.

SOCRATES: And is memory of something or of nothing?

THEAETETUS: Of something, surely.

SOCRATES: Of things learned and perceived, that is?

THEAETETUS: Certainly.

SOCRATES: Often a man remembers that which he has seen?

THEAETETUS: True.

SOCRATES: And if he closed his eyes, would he forget?

THEAETETUS: Who, Socrates, would dare to say so?

SOCRATES: But we must say so, if the previous argument is to be
maintained.

THEAETETUS: What do you mean? I am not quite sure that I understand you,
though I have a strong suspicion that you are right.

SOCRATES: As thus: he who sees knows, as we say, that which he sees; for
perception and sight and knowledge are admitted to be the same.

THEAETETUS: Certainly.

SOCRATES: But he who saw, and has knowledge of that which he saw,
remembers, when he closes his eyes, that which he no longer sees.

THEAETETUS: True.

SOCRATES: And seeing is knowing, and therefore not-seeing is
not-knowing?

THEAETETUS: Very true.

SOCRATES: Then the inference is, that a man may have attained the
knowledge of something, which he may remember and yet not know, because
he does not see; and this has been affirmed by us to be a monstrous
supposition.

THEAETETUS: Most true.

SOCRATES: Thus, then, the assertion that knowledge and perception are
one, involves a manifest impossibility?

THEAETETUS: Yes.

SOCRATES: Then they must be distinguished?

THEAETETUS: I suppose that they must.

SOCRATES: Once more we shall have to begin, and ask 'What is knowledge?'
and yet, Theaetetus, what are we going to do?

THEAETETUS: About what?

SOCRATES: Like a good-for-nothing cock, without having won the victory,
we walk away from the argument and crow.

THEAETETUS: How do you mean?

SOCRATES: After the manner of disputers (Lys.; Phaedo; Republic), we
were satisfied with mere verbal consistency, and were well pleased if in
this way we could gain an advantage. Although professing not to be mere
Eristics, but philosophers, I suspect that we have unconsciously fallen
into the error of that ingenious class of persons.

THEAETETUS: I do not as yet understand you.

SOCRATES: Then I will try to explain myself: just now we asked the
question, whether a man who had learned and remembered could fail to
know, and we showed that a person who had seen might remember when he
had his eyes shut and could not see, and then he would at the same
time remember and not know. But this was an impossibility. And so the
Protagorean fable came to nought, and yours also, who maintained that
knowledge is the same as perception.

THEAETETUS: True.

SOCRATES: And yet, my friend, I rather suspect that the result would
have been different if Protagoras, who was the father of the first of
the two brats, had been alive; he would have had a great deal to say on
their behalf. But he is dead, and we insult over his orphan child; and
even the guardians whom he left, and of whom our friend Theodorus is
one, are unwilling to give any help, and therefore I suppose that I must
take up his cause myself, and see justice done?

THEODORUS: Not I, Socrates, but rather Callias, the son of Hipponicus,
is guardian of his orphans. I was too soon diverted from the
abstractions of dialectic to geometry. Nevertheless, I shall be grateful
to you if you assist him.

SOCRATES: Very good, Theodorus; you shall see how I will come to the
rescue. If a person does not attend to the meaning of terms as they are
commonly used in argument, he may be involved even in greater paradoxes
than these. Shall I explain this matter to you or to Theaetetus?

THEODORUS: To both of us, and let the younger answer; he will incur less
disgrace if he is discomfited.

SOCRATES: Then now let me ask the awful question, which is this:--Can a
man know and also not know that which he knows?

THEODORUS: How shall we answer, Theaetetus?

THEAETETUS: He cannot, I should say.

SOCRATES: He can, if you maintain that seeing is knowing. When you are
imprisoned in a well, as the saying is, and the self-assured adversary
closes one of your eyes with his hand, and asks whether you can see
his cloak with the eye which he has closed, how will you answer the
inevitable man?

THEAETETUS: I should answer, 'Not with that eye but with the other.'

SOCRATES: Then you see and do not see the same thing at the same time.

THEAETETUS: Yes, in a certain sense.

SOCRATES: None of that, he will reply; I do not ask or bid you answer
in what sense you know, but only whether you know that which you do not
know. You have been proved to see that which you do not see; and you
have already admitted that seeing is knowing, and that not-seeing is
not-knowing: I leave you to draw the inference.

THEAETETUS: Yes; the inference is the contradictory of my assertion.

SOCRATES: Yes, my marvel, and there might have been yet worse things in
store for you, if an opponent had gone on to ask whether you can have a
sharp and also a dull knowledge, and whether you can know near, but not
at a distance, or know the same thing with more or less intensity,
and so on without end. Such questions might have been put to you by a
light-armed mercenary, who argued for pay. He would have lain in wait
for you, and when you took up the position, that sense is knowledge,
he would have made an assault upon hearing, smelling, and the other
senses;--he would have shown you no mercy; and while you were lost in
envy and admiration of his wisdom, he would have got you into his
net, out of which you would not have escaped until you had come to an
understanding about the sum to be paid for your release. Well, you ask,
and how will Protagoras reinforce his position? Shall I answer for him?

THEAETETUS: By all means.

SOCRATES: He will repeat all those things which we have been urging on
his behalf, and then he will close with us in disdain, and say:--The
worthy Socrates asked a little boy, whether the same man could remember
and not know the same thing, and the boy said No, because he was
frightened, and could not see what was coming, and then Socrates made
fun of poor me. The truth is, O slatternly Socrates, that when you ask
questions about any assertion of mine, and the person asked is found
tripping, if he has answered as I should have answered, then I am
refuted, but if he answers something else, then he is refuted and not
I. For do you really suppose that any one would admit the memory which a
man has of an impression which has passed away to be the same with that
which he experienced at the time? Assuredly not. Or would he hesitate to
acknowledge that the same man may know and not know the same thing? Or,
if he is afraid of making this admission, would he ever grant that one
who has become unlike is the same as before he became unlike? Or would
he admit that a man is one at all, and not rather many and infinite as
the changes which take place in him? I speak by the card in order to
avoid entanglements of words. But, O my good sir, he will say, come to
the argument in a more generous spirit; and either show, if you can,
that our sensations are not relative and individual, or, if you admit
them to be so, prove that this does not involve the consequence that the
appearance becomes, or, if you will have the word, is, to the individual
only. As to your talk about pigs and baboons, you are yourself behaving
like a pig, and you teach your hearers to make sport of my writings in
the same ignorant manner; but this is not to your credit. For I declare
that the truth is as I have written, and that each of us is a measure
of existence and of non-existence. Yet one man may be a thousand times
better than another in proportion as different things are and appear
to him. And I am far from saying that wisdom and the wise man have no
existence; but I say that the wise man is he who makes the evils which
appear and are to a man, into goods which are and appear to him. And
I would beg you not to press my words in the letter, but to take the
meaning of them as I will explain them. Remember what has been already
said,--that to the sick man his food appears to be and is bitter, and to
the man in health the opposite of bitter. Now I cannot conceive that one
of these men can be or ought to be made wiser than the other: nor can
you assert that the sick man because he has one impression is foolish,
and the healthy man because he has another is wise; but the one state
requires to be changed into the other, the worse into the better. As
in education, a change of state has to be effected, and the sophist
accomplishes by words the change which the physician works by the aid
of drugs. Not that any one ever made another think truly, who previously
thought falsely. For no one can think what is not, or, think anything
different from that which he feels; and this is always true. But as the
inferior habit of mind has thoughts of kindred nature, so I conceive
that a good mind causes men to have good thoughts; and these which the
inexperienced call true, I maintain to be only better, and not truer
than others. And, O my dear Socrates, I do not call wise men tadpoles:
far from it; I say that they are the physicians of the human body, and
the husbandmen of plants--for the husbandmen also take away the evil and
disordered sensations of plants, and infuse into them good and healthy
sensations--aye and true ones; and the wise and good rhetoricians
make the good instead of the evil to seem just to states; for whatever
appears to a state to be just and fair, so long as it is regarded as
such, is just and fair to it; but the teacher of wisdom causes the good
to take the place of the evil, both in appearance and in reality. And in
like manner the Sophist who is able to train his pupils in this spirit
is a wise man, and deserves to be well paid by them. And so one man is
wiser than another; and no one thinks falsely, and you, whether you will
or not, must endure to be a measure. On these foundations the argument
stands firm, which you, Socrates, may, if you please, overthrow by an
opposite argument, or if you like you may put questions to me--a method
to which no intelligent person will object, quite the reverse. But I
must beg you to put fair questions: for there is great inconsistency
in saying that you have a zeal for virtue, and then always behaving
unfairly in argument. The unfairness of which I complain is that you do
not distinguish between mere disputation and dialectic: the disputer
may trip up his opponent as often as he likes, and make fun; but the
dialectician will be in earnest, and only correct his adversary when
necessary, telling him the errors into which he has fallen through his
own fault, or that of the company which he has previously kept. If
you do so, your adversary will lay the blame of his own confusion and
perplexity on himself, and not on you. He will follow and love you, and
will hate himself, and escape from himself into philosophy, in order
that he may become different from what he was. But the other mode of
arguing, which is practised by the many, will have just the opposite
effect upon him; and as he grows older, instead of turning philosopher,
he will come to hate philosophy. I would recommend you, therefore, as
I said before, not to encourage yourself in this polemical and
controversial temper, but to find out, in a friendly and congenial
spirit, what we really mean when we say that all things are in motion,
and that to every individual and state what appears, is. In this manner
you will consider whether knowledge and sensation are the same or
different, but you will not argue, as you were just now doing, from the
customary use of names and words, which the vulgar pervert in all sorts
of ways, causing infinite perplexity to one another. Such, Theodorus, is
the very slight help which I am able to offer to your old friend; had he
been living, he would have helped himself in a far more gloriose style.

THEODORUS: You are jesting, Socrates; indeed, your defence of him has
been most valorous.

SOCRATES: Thank you, friend; and I hope that you observed Protagoras
bidding us be serious, as the text, 'Man is the measure of all things,'
was a solemn one; and he reproached us with making a boy the medium of
discourse, and said that the boy's timidity was made to tell against his
argument; he also declared that we made a joke of him.

THEODORUS: How could I fail to observe all that, Socrates?

SOCRATES: Well, and shall we do as he says?

THEODORUS: By all means.

SOCRATES: But if his wishes are to be regarded, you and I must take up
the argument, and in all seriousness, and ask and answer one another,
for you see that the rest of us are nothing but boys. In no other way
can we escape the imputation, that in our fresh analysis of his thesis
we are making fun with boys.

THEODORUS: Well, but is not Theaetetus better able to follow a
philosophical enquiry than a great many men who have long beards?

SOCRATES: Yes, Theodorus, but not better than you; and therefore please
not to imagine that I am to defend by every means in my power your
departed friend; and that you are to defend nothing and nobody. At any
rate, my good man, do not sheer off until we know whether you are a
true measure of diagrams, or whether all men are equally measures and
sufficient for themselves in astronomy and geometry, and the other
branches of knowledge in which you are supposed to excel them.

THEODORUS: He who is sitting by you, Socrates, will not easily avoid
being drawn into an argument; and when I said just now that you would
excuse me, and not, like the Lacedaemonians, compel me to strip and
fight, I was talking nonsense--I should rather compare you to Scirrhon,
who threw travellers from the rocks; for the Lacedaemonian rule is
'strip or depart,' but you seem to go about your work more after the
fashion of Antaeus: you will not allow any one who approaches you to
depart until you have stripped him, and he has been compelled to try a
fall with you in argument.

SOCRATES: There, Theodorus, you have hit off precisely the nature of my
complaint; but I am even more pugnacious than the giants of old, for I
have met with no end of heroes; many a Heracles, many a Theseus, mighty
in words, has broken my head; nevertheless I am always at this rough
exercise, which inspires me like a passion. Please, then, to try a fall
with me, whereby you will do yourself good as well as me.

THEODORUS: I consent; lead me whither you will, for I know that you are
like destiny; no man can escape from any argument which you may weave
for him. But I am not disposed to go further than you suggest.

SOCRATES: Once will be enough; and now take particular care that we
do not again unwittingly expose ourselves to the reproach of talking
childishly.

THEODORUS: I will do my best to avoid that error.

SOCRATES: In the first place, let us return to our old objection, and
see whether we were right in blaming and taking offence at Protagoras
on the ground that he assumed all to be equal and sufficient in wisdom;
although he admitted that there was a better and worse, and that in
respect of this, some who as he said were the wise excelled others.

THEODORUS: Very true.

SOCRATES: Had Protagoras been living and answered for himself, instead
of our answering for him, there would have been no need of our reviewing
or reinforcing the argument. But as he is not here, and some one may
accuse us of speaking without authority on his behalf, had we not better
come to a clearer agreement about his meaning, for a great deal may be
at stake?

THEODORUS: True.

SOCRATES: Then let us obtain, not through any third person, but from his
own statement and in the fewest words possible, the basis of agreement.

THEODORUS: In what way?

SOCRATES: In this way:--His words are, 'What seems to a man, is to him.'

THEODORUS: Yes, so he says.

SOCRATES: And are not we, Protagoras, uttering the opinion of man, or
rather of all mankind, when we say that every one thinks himself wiser
than other men in some things, and their inferior in others? In the
hour of danger, when they are in perils of war, or of the sea, or of
sickness, do they not look up to their commanders as if they were
gods, and expect salvation from them, only because they excel them in
knowledge? Is not the world full of men in their several employments,
who are looking for teachers and rulers of themselves and of the
animals? and there are plenty who think that they are able to teach
and able to rule. Now, in all this is implied that ignorance and wisdom
exist among them, at least in their own opinion.

THEODORUS: Certainly.

SOCRATES: And wisdom is assumed by them to be true thought, and
ignorance to be false opinion.

THEODORUS: Exactly.

SOCRATES: How then, Protagoras, would you have us treat the argument?
Shall we say that the opinions of men are always true, or sometimes true
and sometimes false? In either case, the result is the same, and their
opinions are not always true, but sometimes true and sometimes false.
For tell me, Theodorus, do you suppose that you yourself, or any other
follower of Protagoras, would contend that no one deems another ignorant
or mistaken in his opinion?

THEODORUS: The thing is incredible, Socrates.

SOCRATES: And yet that absurdity is necessarily involved in the thesis
which declares man to be the measure of all things.

THEODORUS: How so?

SOCRATES: Why, suppose that you determine in your own mind something to
be true, and declare your opinion to me; let us assume, as he argues,
that this is true to you. Now, if so, you must either say that the rest
of us are not the judges of this opinion or judgment of yours, or that
we judge you always to have a true opinion? But are there not thousands
upon thousands who, whenever you form a judgment, take up arms against
you and are of an opposite judgment and opinion, deeming that you judge
falsely?

THEODORUS: Yes, indeed, Socrates, thousands and tens of thousands, as
Homer says, who give me a world of trouble.

SOCRATES: Well, but are we to assert that what you think is true to you
and false to the ten thousand others?

THEODORUS: No other inference seems to be possible.

SOCRATES: And how about Protagoras himself? If neither he nor the
multitude thought, as indeed they do not think, that man is the measure
of all things, must it not follow that the truth of which Protagoras
wrote would be true to no one? But if you suppose that he himself
thought this, and that the multitude does not agree with him, you must
begin by allowing that in whatever proportion the many are more than
one, in that proportion his truth is more untrue than true.

THEODORUS: That would follow if the truth is supposed to vary with
individual opinion.

SOCRATES: And the best of the joke is, that he acknowledges the truth
of their opinion who believe his own opinion to be false; for he admits
that the opinions of all men are true.

THEODORUS: Certainly.

SOCRATES: And does he not allow that his own opinion is false, if he
admits that the opinion of those who think him false is true?

THEODORUS: Of course.

SOCRATES: Whereas the other side do not admit that they speak falsely?

THEODORUS: They do not.

SOCRATES: And he, as may be inferred from his writings, agrees that this
opinion is also true.

THEODORUS: Clearly.

SOCRATES: Then all mankind, beginning with Protagoras, will contend,
or rather, I should say that he will allow, when he concedes that his
adversary has a true opinion--Protagoras, I say, will himself allow that
neither a dog nor any ordinary man is the measure of anything which he
has not learned--am I not right?

THEODORUS: Yes.

SOCRATES: And the truth of Protagoras being doubted by all, will be true
neither to himself to any one else?

THEODORUS: I think, Socrates, that we are running my old friend too
hard.

SOCRATES: But I do not know that we are going beyond the truth.
Doubtless, as he is older, he may be expected to be wiser than we are.
And if he could only just get his head out of the world below, he would
have overthrown both of us again and again, me for talking nonsense and
you for assenting to me, and have been off and underground in a trice.
But as he is not within call, we must make the best use of our own
faculties, such as they are, and speak out what appears to us to be
true. And one thing which no one will deny is, that there are great
differences in the understandings of men.

THEODORUS: In that opinion I quite agree.

SOCRATES: And is there not most likely to be firm ground in the
distinction which we were indicating on behalf of Protagoras, viz. that
most things, and all immediate sensations, such as hot, dry, sweet,
are only such as they appear; if however difference of opinion is to be
allowed at all, surely we must allow it in respect of health or disease?
for every woman, child, or living creature has not such a knowledge of
what conduces to health as to enable them to cure themselves.

THEODORUS: I quite agree.

SOCRATES: Or again, in politics, while affirming that just and unjust,
honourable and disgraceful, holy and unholy, are in reality to each
state such as the state thinks and makes lawful, and that in determining
these matters no individual or state is wiser than another, still the
followers of Protagoras will not deny that in determining what is or is
not expedient for the community one state is wiser and one counsellor
better than another--they will scarcely venture to maintain, that what
a city enacts in the belief that it is expedient will always be really
expedient. But in the other case, I mean when they speak of justice and
injustice, piety and impiety, they are confident that in nature these
have no existence or essence of their own--the truth is that which is
agreed on at the time of the agreement, and as long as the agreement
lasts; and this is the philosophy of many who do not altogether go along
with Protagoras. Here arises a new question, Theodorus, which threatens
to be more serious than the last.

THEODORUS: Well, Socrates, we have plenty of leisure.

SOCRATES: That is true, and your remark recalls to my mind an
observation which I have often made, that those who have passed their
days in the pursuit of philosophy are ridiculously at fault when they
have to appear and speak in court. How natural is this!

THEODORUS: What do you mean?

SOCRATES: I mean to say, that those who have been trained in philosophy
and liberal pursuits are as unlike those who from their youth upwards
have been knocking about in the courts and such places, as a freeman is
in breeding unlike a slave.

THEODORUS: In what is the difference seen?

SOCRATES: In the leisure spoken of by you, which a freeman can always
command: he has his talk out in peace, and, like ourselves, he wanders
at will from one subject to another, and from a second to a third,--if
the fancy takes him, he begins again, as we are doing now, caring not
whether his words are many or few; his only aim is to attain the truth.
But the lawyer is always in a hurry; there is the water of the clepsydra
driving him on, and not allowing him to expatiate at will: and there is
his adversary standing over him, enforcing his rights; the indictment,
which in their phraseology is termed the affidavit, is recited at
the time: and from this he must not deviate. He is a servant, and is
continually disputing about a fellow-servant before his master, who is
seated, and has the cause in his hands; the trial is never about some
indifferent matter, but always concerns himself; and often the race
is for his life. The consequence has been, that he has become keen and
shrewd; he has learned how to flatter his master in word and indulge him
in deed; but his soul is small and unrighteous. His condition, which has
been that of a slave from his youth upwards, has deprived him of growth
and uprightness and independence; dangers and fears, which were too
much for his truth and honesty, came upon him in early years, when the
tenderness of youth was unequal to them, and he has been driven into
crooked ways; from the first he has practised deception and retaliation,
and has become stunted and warped. And so he has passed out of youth
into manhood, having no soundness in him; and is now, as he thinks,
a master in wisdom. Such is the lawyer, Theodorus. Will you have the
companion picture of the philosopher, who is of our brotherhood; or
shall we return to the argument? Do not let us abuse the freedom of
digression which we claim.

THEODORUS: Nay, Socrates, not until we have finished what we are about;
for you truly said that we belong to a brotherhood which is free, and
are not the servants of the argument; but the argument is our servant,
and must wait our leisure. Who is our judge? Or where is the spectator
having any right to censure or control us, as he might the poets?

SOCRATES: Then, as this is your wish, I will describe the leaders; for
there is no use in talking about the inferior sort. In the first place,
the lords of philosophy have never, from their youth upwards, known
their way to the Agora, or the dicastery, or the council, or any other
political assembly; they neither see nor hear the laws or decrees,
as they are called, of the state written or recited; the eagerness of
political societies in the attainment of offices--clubs, and banquets,
and revels, and singing-maidens,--do not enter even into their dreams.
Whether any event has turned out well or ill in the city, what disgrace
may have descended to any one from his ancestors, male or female, are
matters of which the philosopher no more knows than he can tell, as they
say, how many pints are contained in the ocean. Neither is he conscious
of his ignorance. For he does not hold aloof in order that he may gain a
reputation; but the truth is, that the outer form of him only is in the
city: his mind, disdaining the littlenesses and nothingnesses of human
things, is 'flying all abroad' as Pindar says, measuring earth and
heaven and the things which are under and on the earth and above
the heaven, interrogating the whole nature of each and all in their
entirety, but not condescending to anything which is within reach.

THEODORUS: What do you mean, Socrates?

SOCRATES: I will illustrate my meaning, Theodorus, by the jest which the
clever witty Thracian handmaid is said to have made about Thales, when
he fell into a well as he was looking up at the stars. She said, that he
was so eager to know what was going on in heaven, that he could not see
what was before his feet. This is a jest which is equally applicable to
all philosophers. For the philosopher is wholly unacquainted with his
next-door neighbour; he is ignorant, not only of what he is doing, but
he hardly knows whether he is a man or an animal; he is searching into
the essence of man, and busy in enquiring what belongs to such a nature
to do or suffer different from any other;--I think that you understand
me, Theodorus?

THEODORUS: I do, and what you say is true.

SOCRATES: And thus, my friend, on every occasion, private as well as
public, as I said at first, when he appears in a law-court, or in any
place in which he has to speak of things which are at his feet and
before his eyes, he is the jest, not only of Thracian handmaids but of
the general herd, tumbling into wells and every sort of disaster through
his inexperience. His awkwardness is fearful, and gives the impression
of imbecility. When he is reviled, he has nothing personal to say in
answer to the civilities of his adversaries, for he knows no scandals
of any one, and they do not interest him; and therefore he is laughed at
for his sheepishness; and when others are being praised and glorified,
in the simplicity of his heart he cannot help going into fits of
laughter, so that he seems to be a downright idiot. When he hears a
tyrant or king eulogized, he fancies that he is listening to the
praises of some keeper of cattle--a swineherd, or shepherd, or perhaps a
cowherd, who is congratulated on the quantity of milk which he squeezes
from them; and he remarks that the creature whom they tend, and out of
whom they squeeze the wealth, is of a less tractable and more insidious
nature. Then, again, he observes that the great man is of necessity as
ill-mannered and uneducated as any shepherd--for he has no leisure,
and he is surrounded by a wall, which is his mountain-pen. Hearing
of enormous landed proprietors of ten thousand acres and more, our
philosopher deems this to be a trifle, because he has been accustomed to
think of the whole earth; and when they sing the praises of family, and
say that some one is a gentleman because he can show seven generations
of wealthy ancestors, he thinks that their sentiments only betray a
dull and narrow vision in those who utter them, and who are not educated
enough to look at the whole, nor to consider that every man has had
thousands and ten thousands of progenitors, and among them have been
rich and poor, kings and slaves, Hellenes and barbarians, innumerable.
And when people pride themselves on having a pedigree of twenty-five
ancestors, which goes back to Heracles, the son of Amphitryon, he cannot
understand their poverty of ideas. Why are they unable to calculate that
Amphitryon had a twenty-fifth ancestor, who might have been anybody,
and was such as fortune made him, and he had a fiftieth, and so on? He
amuses himself with the notion that they cannot count, and thinks that a
little arithmetic would have got rid of their senseless vanity. Now, in
all these cases our philosopher is derided by the vulgar, partly because
he is thought to despise them, and also because he is ignorant of what
is before him, and always at a loss.

THEODORUS: That is very true, Socrates.

SOCRATES: But, O my friend, when he draws the other into upper air,
and gets him out of his pleas and rejoinders into the contemplation of
justice and injustice in their own nature and in their difference from
one another and from all other things; or from the commonplaces about
the happiness of a king or of a rich man to the consideration of
government, and of human happiness and misery in general--what they
are, and how a man is to attain the one and avoid the other--when that
narrow, keen, little legal mind is called to account about all this, he
gives the philosopher his revenge; for dizzied by the height at which
he is hanging, whence he looks down into space, which is a strange
experience to him, he being dismayed, and lost, and stammering
broken words, is laughed at, not by Thracian handmaidens or any other
uneducated persons, for they have no eye for the situation, but by every
man who has not been brought up a slave. Such are the two characters,
Theodorus: the one of the freeman, who has been trained in liberty and
leisure, whom you call the philosopher,--him we cannot blame because
he appears simple and of no account when he has to perform some menial
task, such as packing up bed-clothes, or flavouring a sauce or fawning
speech; the other character is that of the man who is able to do all
this kind of service smartly and neatly, but knows not how to wear his
cloak like a gentleman; still less with the music of discourse can he
hymn the true life aright which is lived by immortals or men blessed of
heaven.

THEODORUS: If you could only persuade everybody, Socrates, as you do me,
of the truth of your words, there would be more peace and fewer evils
among men.

SOCRATES: Evils, Theodorus, can never pass away; for there must always
remain something which is antagonistic to good. Having no place among
the gods in heaven, of necessity they hover around the mortal nature,
and this earthly sphere. Wherefore we ought to fly away from earth to
heaven as quickly as we can; and to fly away is to become like God,
as far as this is possible; and to become like him, is to become holy,
just, and wise. But, O my friend, you cannot easily convince mankind
that they should pursue virtue or avoid vice, not merely in order that a
man may seem to be good, which is the reason given by the world, and in
my judgment is only a repetition of an old wives' fable. Whereas,
the truth is that God is never in any way unrighteous--he is perfect
righteousness; and he of us who is the most righteous is most like him.
Herein is seen the true cleverness of a man, and also his nothingness
and want of manhood. For to know this is true wisdom and virtue, and
ignorance of this is manifest folly and vice. All other kinds of wisdom
or cleverness, which seem only, such as the wisdom of politicians, or
the wisdom of the arts, are coarse and vulgar. The unrighteous man, or
the sayer and doer of unholy things, had far better not be encouraged
in the illusion that his roguery is clever; for men glory in their
shame--they fancy that they hear others saying of them, 'These are not
mere good-for-nothing persons, mere burdens of the earth, but such as
men should be who mean to dwell safely in a state.' Let us tell them
that they are all the more truly what they do not think they are because
they do not know it; for they do not know the penalty of injustice,
which above all things they ought to know--not stripes and death, as
they suppose, which evil-doers often escape, but a penalty which cannot
be escaped.

THEODORUS: What is that?

SOCRATES: There are two patterns eternally set before them; the one
blessed and divine, the other godless and wretched: but they do not see
them, or perceive that in their utter folly and infatuation they are
growing like the one and unlike the other, by reason of their evil
deeds; and the penalty is, that they lead a life answering to the
pattern which they are growing like. And if we tell them, that unless
they depart from their cunning, the place of innocence will not receive
them after death; and that here on earth, they will live ever in the
likeness of their own evil selves, and with evil friends--when they hear
this they in their superior cunning will seem to be listening to the
talk of idiots.

THEODORUS: Very true, Socrates.

SOCRATES: Too true, my friend, as I well know; there is, however, one
peculiarity in their case: when they begin to reason in private about
their dislike of philosophy, if they have the courage to hear the
argument out, and do not run away, they grow at last strangely
discontented with themselves; their rhetoric fades away, and they become
helpless as children. These however are digressions from which we must
now desist, or they will overflow, and drown the original argument; to
which, if you please, we will now return.

THEODORUS: For my part, Socrates, I would rather have the digressions,
for at my age I find them easier to follow; but if you wish, let us go
back to the argument.

SOCRATES: Had we not reached the point at which the partisans of the
perpetual flux, who say that things are as they seem to each one, were
confidently maintaining that the ordinances which the state commanded
and thought just, were just to the state which imposed them, while they
were in force; this was especially asserted of justice; but as to the
good, no one had any longer the hardihood to contend of any ordinances
which the state thought and enacted to be good that these, while they
were in force, were really good;--he who said so would be playing with
the name 'good,' and would not touch the real question--it would be a
mockery, would it not?

THEODORUS: Certainly it would.

SOCRATES: He ought not to speak of the name, but of the thing which is
contemplated under the name.

THEODORUS: Right.

SOCRATES: Whatever be the term used, the good or expedient is the aim
of legislation, and as far as she has an opinion, the state imposes all
laws with a view to the greatest expediency; can legislation have any
other aim?

THEODORUS: Certainly not.

SOCRATES: But is the aim attained always? do not mistakes often happen?

THEODORUS: Yes, I think that there are mistakes.

SOCRATES: The possibility of error will be more distinctly recognised,
if we put the question in reference to the whole class under which the
good or expedient falls. That whole class has to do with the future, and
laws are passed under the idea that they will be useful in after-time;
which, in other words, is the future.

THEODORUS: Very true.

SOCRATES: Suppose now, that we ask Protagoras, or one of his disciples,
a question:--O, Protagoras, we will say to him, Man is, as you declare,
the measure of all things--white, heavy, light: of all such things he
is the judge; for he has the criterion of them in himself, and when he
thinks that things are such as he experiences them to be, he thinks what
is and is true to himself. Is it not so?

THEODORUS: Yes.

SOCRATES: And do you extend your doctrine, Protagoras (as we shall
further say), to the future as well as to the present; and has he the
criterion not only of what in his opinion is but of what will be, and do
things always happen to him as he expected? For example, take the case
of heat:--When an ordinary man thinks that he is going to have a fever,
and that this kind of heat is coming on, and another person, who is a
physician, thinks the contrary, whose opinion is likely to prove right?
Or are they both right?--he will have a heat and fever in his own
judgment, and not have a fever in the physician's judgment?

THEODORUS: How ludicrous!

SOCRATES: And the vinegrower, if I am not mistaken, is a better judge of
the sweetness or dryness of the vintage which is not yet gathered than
the harp-player?

THEODORUS: Certainly.

SOCRATES: And in musical composition the musician will know better than
the training master what the training master himself will hereafter
think harmonious or the reverse?

THEODORUS: Of course.

SOCRATES: And the cook will be a better judge than the guest, who is
not a cook, of the pleasure to be derived from the dinner which is in
preparation; for of present or past pleasure we are not as yet arguing;
but can we say that every one will be to himself the best judge of the
pleasure which will seem to be and will be to him in the future?--nay,
would not you, Protagoras, better guess which arguments in a court would
convince any one of us than the ordinary man?

THEODORUS: Certainly, Socrates, he used to profess in the strongest
manner that he was the superior of all men in this respect.

SOCRATES: To be sure, friend: who would have paid a large sum for the
privilege of talking to him, if he had really persuaded his visitors
that neither a prophet nor any other man was better able to judge what
will be and seem to be in the future than every one could for himself?

THEODORUS: Who indeed?

SOCRATES: And legislation and expediency are all concerned with the
future; and every one will admit that states, in passing laws, must
often fail of their highest interests?

THEODORUS: Quite true.

SOCRATES: Then we may fairly argue against your master, that he must
admit one man to be wiser than another, and that the wiser is a measure:
but I, who know nothing, am not at all obliged to accept the honour
which the advocate of Protagoras was just now forcing upon me, whether I
would or not, of being a measure of anything.

THEODORUS: That is the best refutation of him, Socrates; although he is
also caught when he ascribes truth to the opinions of others, who give
the lie direct to his own opinion.

SOCRATES: There are many ways, Theodorus, in which the doctrine that
every opinion of every man is true may be refuted; but there is more
difficulty in proving that states of feeling, which are present to a
man, and out of which arise sensations and opinions in accordance with
them, are also untrue. And very likely I have been talking nonsense
about them; for they may be unassailable, and those who say that there
is clear evidence of them, and that they are matters of knowledge, may
probably be right; in which case our friend Theaetetus was not so far
from the mark when he identified perception and knowledge. And therefore
let us draw nearer, as the advocate of Protagoras desires; and give the
truth of the universal flux a ring: is the theory sound or not? at any
rate, no small war is raging about it, and there are combination not a
few.

THEODORUS: No small, war, indeed, for in Ionia the sect makes rapid
strides; the disciples of Heracleitus are most energetic upholders of
the doctrine.

SOCRATES: Then we are the more bound, my dear Theodorus, to examine the
question from the foundation as it is set forth by themselves.

THEODORUS: Certainly we are. About these speculations of Heracleitus,
which, as you say, are as old as Homer, or even older still, the
Ephesians themselves, who profess to know them, are downright mad, and
you cannot talk with them on the subject. For, in accordance with their
text-books, they are always in motion; but as for dwelling upon an
argument or a question, and quietly asking and answering in turn, they
can no more do so than they can fly; or rather, the determination of
these fellows not to have a particle of rest in them is more than
the utmost powers of negation can express. If you ask any of them a
question, he will produce, as from a quiver, sayings brief and dark, and
shoot them at you; and if you inquire the reason of what he has said,
you will be hit by some other new-fangled word, and will make no way
with any of them, nor they with one another; their great care is, not
to allow of any settled principle either in their arguments or in
their minds, conceiving, as I imagine, that any such principle would be
stationary; for they are at war with the stationary, and do what they
can to drive it out everywhere.

SOCRATES: I suppose, Theodorus, that you have only seen them when they
were fighting, and have never stayed with them in time of peace,
for they are no friends of yours; and their peace doctrines are only
communicated by them at leisure, as I imagine, to those disciples of
theirs whom they want to make like themselves.

THEODORUS: Disciples! my good sir, they have none; men of their sort are
not one another's disciples, but they grow up at their own sweet will,
and get their inspiration anywhere, each of them saying of his neighbour
that he knows nothing. From these men, then, as I was going to remark,
you will never get a reason, whether with their will or without their
will; we must take the question out of their hands, and make the
analysis ourselves, as if we were doing geometrical problem.

SOCRATES: Quite right too; but as touching the aforesaid problem, have
we not heard from the ancients, who concealed their wisdom from the many
in poetical figures, that Oceanus and Tethys, the origin of all things,
are streams, and that nothing is at rest? And now the moderns, in their
superior wisdom, have declared the same openly, that the cobbler too may
hear and learn of them, and no longer foolishly imagine that some things
are at rest and others in motion--having learned that all is motion,
he will duly honour his teachers. I had almost forgotten the opposite
doctrine, Theodorus,

     'Alone Being remains unmoved, which is the name for the all.'

This is the language of Parmenides, Melissus, and their followers, who
stoutly maintain that all being is one and self-contained, and has no
place in which to move. What shall we do, friend, with all these people;
for, advancing step by step, we have imperceptibly got between the
combatants, and, unless we can protect our retreat, we shall pay the
penalty of our rashness--like the players in the palaestra who are
caught upon the line, and are dragged different ways by the two parties.
Therefore I think that we had better begin by considering those whom we
first accosted, 'the river-gods,' and, if we find any truth in them, we
will help them to pull us over, and try to get away from the others. But
if the partisans of 'the whole' appear to speak more truly, we will fly
off from the party which would move the immovable, to them. And if I
find that neither of them have anything reasonable to say, we shall
be in a ridiculous position, having so great a conceit of our own poor
opinion and rejecting that of ancient and famous men. O Theodorus, do
you think that there is any use in proceeding when the danger is so
great?

THEODORUS: Nay, Socrates, not to examine thoroughly what the two parties
have to say would be quite intolerable.

SOCRATES: Then examine we must, since you, who were so reluctant to
begin, are so eager to proceed. The nature of motion appears to be the
question with which we begin. What do they mean when they say that all
things are in motion? Is there only one kind of motion, or, as I rather
incline to think, two? I should like to have your opinion upon this
point in addition to my own, that I may err, if I must err, in your
company; tell me, then, when a thing changes from one place to another,
or goes round in the same place, is not that what is called motion?

THEODORUS: Yes.

SOCRATES: Here then we have one kind of motion. But when a thing,
remaining on the same spot, grows old, or becomes black from being
white, or hard from being soft, or undergoes any other change, may not
this be properly called motion of another kind?

THEODORUS: I think so.

SOCRATES: Say rather that it must be so. Of motion then there are these
two kinds, 'change,' and 'motion in place.'

THEODORUS: You are right.

SOCRATES: And now, having made this distinction, let us address
ourselves to those who say that all is motion, and ask them whether all
things according to them have the two kinds of motion, and are changed
as well as move in place, or is one thing moved in both ways, and
another in one only?

THEODORUS: Indeed, I do not know what to answer; but I think they would
say that all things are moved in both ways.

SOCRATES: Yes, comrade; for, if not, they would have to say that the
same things are in motion and at rest, and there would be no more truth
in saying that all things are in motion, than that all things are at
rest.

THEODORUS: To be sure.

SOCRATES: And if they are to be in motion, and nothing is to be devoid
of motion, all things must always have every sort of motion?

THEODORUS: Most true.

SOCRATES: Consider a further point: did we not understand them to
explain the generation of heat, whiteness, or anything else, in some
such manner as the following:--were they not saying that each of them
is moving between the agent and the patient, together with a perception,
and that the patient ceases to be a perceiving power and becomes a
percipient, and the agent a quale instead of a quality? I suspect that
quality may appear a strange and uncouth term to you, and that you
do not understand the abstract expression. Then I will take concrete
instances: I mean to say that the producing power or agent becomes
neither heat nor whiteness but hot and white, and the like of other
things. For I must repeat what I said before, that neither the agent
nor patient have any absolute existence, but when they come together
and generate sensations and their objects, the one becomes a thing of a
certain quality, and the other a percipient. You remember?

THEODORUS: Of course.

SOCRATES: We may leave the details of their theory unexamined, but
we must not forget to ask them the only question with which we are
concerned: Are all things in motion and flux?

THEODORUS: Yes, they will reply.

SOCRATES: And they are moved in both those ways which we distinguished,
that is to say, they move in place and are also changed?

THEODORUS: Of course, if the motion is to be perfect.

SOCRATES: If they only moved in place and were not changed, we should
be able to say what is the nature of the things which are in motion and
flux?

THEODORUS: Exactly.

SOCRATES: But now, since not even white continues to flow white, and
whiteness itself is a flux or change which is passing into another
colour, and is never to be caught standing still, can the name of any
colour be rightly used at all?

THEODORUS: How is that possible, Socrates, either in the case of this
or of any other quality--if while we are using the word the object is
escaping in the flux?

SOCRATES: And what would you say of perceptions, such as sight and
hearing, or any other kind of perception? Is there any stopping in the
act of seeing and hearing?

THEODORUS: Certainly not, if all things are in motion.

SOCRATES: Then we must not speak of seeing any more than of not-seeing,
nor of any other perception more than of any non-perception, if all
things partake of every kind of motion?

THEODORUS: Certainly not.

SOCRATES: Yet perception is knowledge: so at least Theaetetus and I were
saying.

THEODORUS: Very true.

SOCRATES: Then when we were asked what is knowledge, we no more answered
what is knowledge than what is not knowledge?

THEODORUS: I suppose not.

SOCRATES: Here, then, is a fine result: we corrected our first answer
in our eagerness to prove that nothing is at rest. But if nothing is at
rest, every answer upon whatever subject is equally right: you may say
that a thing is or is not thus; or, if you prefer, 'becomes' thus; and
if we say 'becomes,' we shall not then hamper them with words expressive
of rest.

THEODORUS: Quite true.

SOCRATES: Yes, Theodorus, except in saying 'thus' and 'not thus.' But
you ought not to use the word 'thus,' for there is no motion in 'thus'
or in 'not thus.' The maintainers of the doctrine have as yet no words
in which to express themselves, and must get a new language. I know of
no word that will suit them, except perhaps 'no how,' which is perfectly
indefinite.

THEODORUS: Yes, that is a manner of speaking in which they will be quite
at home.

SOCRATES: And so, Theodorus, we have got rid of your friend without
assenting to his doctrine, that every man is the measure of all
things--a wise man only is a measure; neither can we allow that
knowledge is perception, certainly not on the hypothesis of a perpetual
flux, unless perchance our friend Theaetetus is able to convince us that
it is.

THEODORUS: Very good, Socrates; and now that the argument about the
doctrine of Protagoras has been completed, I am absolved from answering;
for this was the agreement.

THEAETETUS: Not, Theodorus, until you and Socrates have discussed the
doctrine of those who say that all things are at rest, as you were
proposing.

THEODORUS: You, Theaetetus, who are a young rogue, must not instigate
your elders to a breach of faith, but should prepare to answer Socrates
in the remainder of the argument.

THEAETETUS: Yes, if he wishes; but I would rather have heard about the
doctrine of rest.

THEODORUS: Invite Socrates to an argument--invite horsemen to the open
plain; do but ask him, and he will answer.

SOCRATES: Nevertheless, Theodorus, I am afraid that I shall not be able
to comply with the request of Theaetetus.

THEODORUS: Not comply! for what reason?

SOCRATES: My reason is that I have a kind of reverence; not so much for
Melissus and the others, who say that 'All is one and at rest,' as for
the great leader himself, Parmenides, venerable and awful, as in Homeric
language he may be called;--him I should be ashamed to approach in a
spirit unworthy of him. I met him when he was an old man, and I was a
mere youth, and he appeared to me to have a glorious depth of mind.
And I am afraid that we may not understand his words, and may be still
further from understanding his meaning; above all I fear that the nature
of knowledge, which is the main subject of our discussion, may be thrust
out of sight by the unbidden guests who will come pouring in upon our
feast of discourse, if we let them in--besides, the question which is
now stirring is of immense extent, and will be treated unfairly if only
considered by the way; or if treated adequately and at length, will put
into the shade the other question of knowledge. Neither the one nor the
other can be allowed; but I must try by my art of midwifery to deliver
Theaetetus of his conceptions about knowledge.

THEAETETUS: Very well; do so if you will.

SOCRATES: Then now, Theaetetus, take another view of the subject: you
answered that knowledge is perception?

THEAETETUS: I did.

SOCRATES: And if any one were to ask you: With what does a man see black
and white colours? and with what does he hear high and low sounds?--you
would say, if I am not mistaken, 'With the eyes and with the ears.'

THEAETETUS: I should.

SOCRATES: The free use of words and phrases, rather than minute
precision, is generally characteristic of a liberal education, and
the opposite is pedantic; but sometimes precision is necessary, and I
believe that the answer which you have just given is open to the charge
of incorrectness; for which is more correct, to say that we see or hear
with the eyes and with the ears, or through the eyes and through the
ears.

THEAETETUS: I should say 'through,' Socrates, rather than 'with.'

SOCRATES: Yes, my boy, for no one can suppose that in each of us, as
in a sort of Trojan horse, there are perched a number of unconnected
senses, which do not all meet in some one nature, the mind, or whatever
we please to call it, of which they are the instruments, and with which
through them we perceive objects of sense.

THEAETETUS: I agree with you in that opinion.

SOCRATES: The reason why I am thus precise is, because I want to know
whether, when we perceive black and white through the eyes, and again,
other qualities through other organs, we do not perceive them with one
and the same part of ourselves, and, if you were asked, you might refer
all such perceptions to the body. Perhaps, however, I had better allow
you to answer for yourself and not interfere. Tell me, then, are not
the organs through which you perceive warm and hard and light and sweet,
organs of the body?

THEAETETUS: Of the body, certainly.

SOCRATES: And you would admit that what you perceive through one
faculty you cannot perceive through another; the objects of hearing,
for example, cannot be perceived through sight, or the objects of sight
through hearing?

THEAETETUS: Of course not.

SOCRATES: If you have any thought about both of them, this common
perception cannot come to you, either through the one or the other
organ?

THEAETETUS: It cannot.

SOCRATES: How about sounds and colours: in the first place you would
admit that they both exist?

THEAETETUS: Yes.

SOCRATES: And that either of them is different from the other, and the
same with itself?

THEAETETUS: Certainly.

SOCRATES: And that both are two and each of them one?

THEAETETUS: Yes.

SOCRATES: You can further observe whether they are like or unlike one
another?

THEAETETUS: I dare say.

SOCRATES: But through what do you perceive all this about them? for
neither through hearing nor yet through seeing can you apprehend that
which they have in common. Let me give you an illustration of the
point at issue:--If there were any meaning in asking whether sounds and
colours are saline or not, you would be able to tell me what faculty
would consider the question. It would not be sight or hearing, but some
other.

THEAETETUS: Certainly; the faculty of taste.

SOCRATES: Very good; and now tell me what is the power which discerns,
not only in sensible objects, but in all things, universal notions, such
as those which are called being and not-being, and those others
about which we were just asking--what organs will you assign for the
perception of these notions?

THEAETETUS: You are thinking of being and not being, likeness and
unlikeness, sameness and difference, and also of unity and other numbers
which are applied to objects of sense; and you mean to ask, through
what bodily organ the soul perceives odd and even numbers and other
arithmetical conceptions.

SOCRATES: You follow me excellently, Theaetetus; that is precisely what
I am asking.

THEAETETUS: Indeed, Socrates, I cannot answer; my only notion is, that
these, unlike objects of sense, have no separate organ, but that the
mind, by a power of her own, contemplates the universals in all things.

SOCRATES: You are a beauty, Theaetetus, and not ugly, as Theodorus was
saying; for he who utters the beautiful is himself beautiful and good.
And besides being beautiful, you have done me a kindness in releasing me
from a very long discussion, if you are clear that the soul views some
things by herself and others through the bodily organs. For that was my
own opinion, and I wanted you to agree with me.

THEAETETUS: I am quite clear.

SOCRATES: And to which class would you refer being or essence; for this,
of all our notions, is the most universal?

THEAETETUS: I should say, to that class which the soul aspires to know
of herself.

SOCRATES: And would you say this also of like and unlike, same and
other?

THEAETETUS: Yes.

SOCRATES: And would you say the same of the noble and base, and of good
and evil?

THEAETETUS: These I conceive to be notions which are essentially
relative, and which the soul also perceives by comparing in herself
things past and present with the future.

SOCRATES: And does she not perceive the hardness of that which is hard
by the touch, and the softness of that which is soft equally by the
touch?

THEAETETUS: Yes.

SOCRATES: But their essence and what they are, and their opposition
to one another, and the essential nature of this opposition, the soul
herself endeavours to decide for us by the review and comparison of
them?

THEAETETUS: Certainly.

SOCRATES: The simple sensations which reach the soul through the body
are given at birth to men and animals by nature, but their reflections
on the being and use of them are slowly and hardly gained, if they are
ever gained, by education and long experience.

THEAETETUS: Assuredly.

SOCRATES: And can a man attain truth who fails of attaining being?

THEAETETUS: Impossible.

SOCRATES: And can he who misses the truth of anything, have a knowledge
of that thing?

THEAETETUS: He cannot.

SOCRATES: Then knowledge does not consist in impressions of sense, but
in reasoning about them; in that only, and not in the mere impression,
truth and being can be attained?

THEAETETUS: Clearly.

SOCRATES: And would you call the two processes by the same name, when
there is so great a difference between them?

THEAETETUS: That would certainly not be right.

SOCRATES: And what name would you give to seeing, hearing, smelling,
being cold and being hot?

THEAETETUS: I should call all of them perceiving--what other name could
be given to them?

SOCRATES: Perception would be the collective name of them?

THEAETETUS: Certainly.

SOCRATES: Which, as we say, has no part in the attainment of truth any
more than of being?

THEAETETUS: Certainly not.

SOCRATES: And therefore not in science or knowledge?

THEAETETUS: No.

SOCRATES: Then perception, Theaetetus, can never be the same as
knowledge or science?

THEAETETUS: Clearly not, Socrates; and knowledge has now been most
distinctly proved to be different from perception.

SOCRATES: But the original aim of our discussion was to find out rather
what knowledge is than what it is not; at the same time we have made
some progress, for we no longer seek for knowledge in perception at all,
but in that other process, however called, in which the mind is alone
and engaged with being.

THEAETETUS: You mean, Socrates, if I am not mistaken, what is called
thinking or opining.

SOCRATES: You conceive truly. And now, my friend, please to begin
again at this point; and having wiped out of your memory all that has
preceded, see if you have arrived at any clearer view, and once more say
what is knowledge.

THEAETETUS: I cannot say, Socrates, that all opinion is knowledge,
because there may be a false opinion; but I will venture to assert, that
knowledge is true opinion: let this then be my reply; and if this is
hereafter disproved, I must try to find another.

SOCRATES: That is the way in which you ought to answer, Theaetetus, and
not in your former hesitating strain, for if we are bold we shall gain
one of two advantages; either we shall find what we seek, or we shall be
less likely to think that we know what we do not know--in either case we
shall be richly rewarded. And now, what are you saying?--Are there
two sorts of opinion, one true and the other false; and do you define
knowledge to be the true?

THEAETETUS: Yes, according to my present view.

SOCRATES: Is it still worth our while to resume the discussion touching
opinion?

THEAETETUS: To what are you alluding?

SOCRATES: There is a point which often troubles me, and is a great
perplexity to me, both in regard to myself and others. I cannot make out
the nature or origin of the mental experience to which I refer.

THEAETETUS: Pray what is it?

SOCRATES: How there can be false opinion--that difficulty still troubles
the eye of my mind; and I am uncertain whether I shall leave the
question, or begin over again in a new way.

THEAETETUS: Begin again, Socrates,--at least if you think that there is
the slightest necessity for doing so. Were not you and Theodorus just
now remarking very truly, that in discussions of this kind we may take
our own time?

SOCRATES: You are quite right, and perhaps there will be no harm in
retracing our steps and beginning again. Better a little which is well
done, than a great deal imperfectly.

THEAETETUS: Certainly.

SOCRATES: Well, and what is the difficulty? Do we not speak of false
opinion, and say that one man holds a false and another a true opinion,
as though there were some natural distinction between them?

THEAETETUS: We certainly say so.

SOCRATES: All things and everything are either known or not known.
I leave out of view the intermediate conceptions of learning and
forgetting, because they have nothing to do with our present question.

THEAETETUS: There can be no doubt, Socrates, if you exclude these, that
there is no other alternative but knowing or not knowing a thing.

SOCRATES: That point being now determined, must we not say that he who
has an opinion, must have an opinion about something which he knows or
does not know?

THEAETETUS: He must.

SOCRATES: He who knows, cannot but know; and he who does not know,
cannot know?

THEAETETUS: Of course.

SOCRATES: What shall we say then? When a man has a false opinion does
he think that which he knows to be some other thing which he knows, and
knowing both, is he at the same time ignorant of both?

THEAETETUS: That, Socrates, is impossible.

SOCRATES: But perhaps he thinks of something which he does not know as
some other thing which he does not know; for example, he knows neither
Theaetetus nor Socrates, and yet he fancies that Theaetetus is Socrates,
or Socrates Theaetetus?

THEAETETUS: How can he?

SOCRATES: But surely he cannot suppose what he knows to be what he does
not know, or what he does not know to be what he knows?

THEAETETUS: That would be monstrous.

SOCRATES: Where, then, is false opinion? For if all things are either
known or unknown, there can be no opinion which is not comprehended
under this alternative, and so false opinion is excluded.

THEAETETUS: Most true.

SOCRATES: Suppose that we remove the question out of the sphere of
knowing or not knowing, into that of being and not-being.

THEAETETUS: What do you mean?

SOCRATES: May we not suspect the simple truth to be that he who thinks
about anything, that which is not, will necessarily think what is false,
whatever in other respects may be the state of his mind?

THEAETETUS: That, again, is not unlikely, Socrates.

SOCRATES: Then suppose some one to say to us, Theaetetus:--Is
it possible for any man to think that which is not, either as a
self-existent substance or as a predicate of something else? And suppose
that we answer, 'Yes, he can, when he thinks what is not true.'--That
will be our answer?

THEAETETUS: Yes.

SOCRATES: But is there any parallel to this?

THEAETETUS: What do you mean?

SOCRATES: Can a man see something and yet see nothing?

THEAETETUS: Impossible.

SOCRATES: But if he sees any one thing, he sees something that exists.
Do you suppose that what is one is ever to be found among non-existing
things?

THEAETETUS: I do not.

SOCRATES: He then who sees some one thing, sees something which is?

THEAETETUS: Clearly.

SOCRATES: And he who hears anything, hears some one thing, and hears
that which is?

THEAETETUS: Yes.

SOCRATES: And he who touches anything, touches something which is one
and therefore is?

THEAETETUS: That again is true.

SOCRATES: And does not he who thinks, think some one thing?

THEAETETUS: Certainly.

SOCRATES: And does not he who thinks some one thing, think something
which is?

THEAETETUS: I agree.

SOCRATES: Then he who thinks of that which is not, thinks of nothing?

THEAETETUS: Clearly.

SOCRATES: And he who thinks of nothing, does not think at all?

THEAETETUS: Obviously.

SOCRATES: Then no one can think that which is not, either as a
self-existent substance or as a predicate of something else?

THEAETETUS: Clearly not.

SOCRATES: Then to think falsely is different from thinking that which is
not?

THEAETETUS: It would seem so.

SOCRATES: Then false opinion has no existence in us, either in the
sphere of being or of knowledge?

THEAETETUS: Certainly not.

SOCRATES: But may not the following be the description of what we
express by this name?

THEAETETUS: What?

SOCRATES: May we not suppose that false opinion or thought is a sort of
heterodoxy; a person may make an exchange in his mind, and say that one
real object is another real object. For thus he always thinks that which
is, but he puts one thing in place of another; and missing the aim of
his thoughts, he may be truly said to have false opinion.

THEAETETUS: Now you appear to me to have spoken the exact truth: when a
man puts the base in the place of the noble, or the noble in the place
of the base, then he has truly false opinion.

SOCRATES: I see, Theaetetus, that your fear has disappeared, and that
you are beginning to despise me.

THEAETETUS: What makes you say so?

SOCRATES: You think, if I am not mistaken, that your 'truly false' is
safe from censure, and that I shall never ask whether there can be
a swift which is slow, or a heavy which is light, or any other
self-contradictory thing, which works, not according to its own nature,
but according to that of its opposite. But I will not insist upon this,
for I do not wish needlessly to discourage you. And so you are satisfied
that false opinion is heterodoxy, or the thought of something else?

THEAETETUS: I am.

SOCRATES: It is possible then upon your view for the mind to conceive of
one thing as another?

THEAETETUS: True.

SOCRATES: But must not the mind, or thinking power, which misplaces
them, have a conception either of both objects or of one of them?

THEAETETUS: Certainly.

SOCRATES: Either together or in succession?

THEAETETUS: Very good.

SOCRATES: And do you mean by conceiving, the same which I mean?

THEAETETUS: What is that?

SOCRATES: I mean the conversation which the soul holds with herself in
considering of anything. I speak of what I scarcely understand; but the
soul when thinking appears to me to be just talking--asking questions
of herself and answering them, affirming and denying. And when she has
arrived at a decision, either gradually or by a sudden impulse, and has
at last agreed, and does not doubt, this is called her opinion. I
say, then, that to form an opinion is to speak, and opinion is a word
spoken,--I mean, to oneself and in silence, not aloud or to another:
What think you?

THEAETETUS: I agree.

SOCRATES: Then when any one thinks of one thing as another, he is saying
to himself that one thing is another?

THEAETETUS: Yes.

SOCRATES: But do you ever remember saying to yourself that the noble
is certainly base, or the unjust just; or, best of all--have you ever
attempted to convince yourself that one thing is another? Nay, not even
in sleep, did you ever venture to say to yourself that odd is even, or
anything of the kind?

THEAETETUS: Never.

SOCRATES: And do you suppose that any other man, either in his senses
or out of them, ever seriously tried to persuade himself that an ox is a
horse, or that two are one?

THEAETETUS: Certainly not.

SOCRATES: But if thinking is talking to oneself, no one speaking and
thinking of two objects, and apprehending them both in his soul, will
say and think that the one is the other of them, and I must add, that
even you, lover of dispute as you are, had better let the word 'other'
alone (i.e. not insist that 'one' and 'other' are the same (Both words
in Greek are called eteron: compare Parmen.; Euthyd.)). I mean to say,
that no one thinks the noble to be base, or anything of the kind.

THEAETETUS: I will give up the word 'other,' Socrates; and I agree to
what you say.

SOCRATES: If a man has both of them in his thoughts, he cannot think
that the one of them is the other?

THEAETETUS: True.

SOCRATES: Neither, if he has one of them only in his mind and not the
other, can he think that one is the other?

THEAETETUS: True; for we should have to suppose that he apprehends that
which is not in his thoughts at all.

SOCRATES: Then no one who has either both or only one of the two objects
in his mind can think that the one is the other. And therefore, he who
maintains that false opinion is heterodoxy is talking nonsense; for
neither in this, any more than in the previous way, can false opinion
exist in us.

THEAETETUS: No.

SOCRATES: But if, Theaetetus, this is not admitted, we shall be driven
into many absurdities.

THEAETETUS: What are they?

SOCRATES: I will not tell you until I have endeavoured to consider the
matter from every point of view. For I should be ashamed of us if we
were driven in our perplexity to admit the absurd consequences of which
I speak. But if we find the solution, and get away from them, we may
regard them only as the difficulties of others, and the ridicule will
not attach to us. On the other hand, if we utterly fail, I suppose that
we must be humble, and allow the argument to trample us under foot,
as the sea-sick passenger is trampled upon by the sailor, and to do
anything to us. Listen, then, while I tell you how I hope to find a way
out of our difficulty.

THEAETETUS: Let me hear.

SOCRATES: I think that we were wrong in denying that a man could think
what he knew to be what he did not know; and that there is a way in
which such a deception is possible.

THEAETETUS: You mean to say, as I suspected at the time, that I may know
Socrates, and at a distance see some one who is unknown to me, and whom
I mistake for him--then the deception will occur?

SOCRATES: But has not that position been relinquished by us, because
involving the absurdity that we should know and not know the things
which we know?

THEAETETUS: True.

SOCRATES: Let us make the assertion in another form, which may or may
not have a favourable issue; but as we are in a great strait, every
argument should be turned over and tested. Tell me, then, whether I am
right in saying that you may learn a thing which at one time you did not
know?

THEAETETUS: Certainly you may.

SOCRATES: And another and another?

THEAETETUS: Yes.

SOCRATES: I would have you imagine, then, that there exists in the mind
of man a block of wax, which is of different sizes in different men;
harder, moister, and having more or less of purity in one than another,
and in some of an intermediate quality.

THEAETETUS: I see.

SOCRATES: Let us say that this tablet is a gift of Memory, the mother
of the Muses; and that when we wish to remember anything which we have
seen, or heard, or thought in our own minds, we hold the wax to the
perceptions and thoughts, and in that material receive the impression of
them as from the seal of a ring; and that we remember and know what is
imprinted as long as the image lasts; but when the image is effaced, or
cannot be taken, then we forget and do not know.

THEAETETUS: Very good.

SOCRATES: Now, when a person has this knowledge, and is considering
something which he sees or hears, may not false opinion arise in the
following manner?

THEAETETUS: In what manner?

SOCRATES: When he thinks what he knows, sometimes to be what he knows,
and sometimes to be what he does not know. We were wrong before in
denying the possibility of this.

THEAETETUS: And how would you amend the former statement?

SOCRATES: I should begin by making a list of the impossible cases which
must be excluded. (1) No one can think one thing to be another when he
does not perceive either of them, but has the memorial or seal of both
of them in his mind; nor can any mistaking of one thing for another
occur, when he only knows one, and does not know, and has no impression
of the other; nor can he think that one thing which he does not know is
another thing which he does not know, or that what he does not know
is what he knows; nor (2) that one thing which he perceives is another
thing which he perceives, or that something which he perceives is
something which he does not perceive; or that something which he does
not perceive is something else which he does not perceive; or that
something which he does not perceive is something which he perceives;
nor again (3) can he think that something which he knows and perceives,
and of which he has the impression coinciding with sense, is something
else which he knows and perceives, and of which he has the impression
coinciding with sense;--this last case, if possible, is still more
inconceivable than the others; nor (4) can he think that something which
he knows and perceives, and of which he has the memorial coinciding with
sense, is something else which he knows; nor so long as these agree,
can he think that a thing which he knows and perceives is another thing
which he perceives; or that a thing which he does not know and does not
perceive, is the same as another thing which he does not know and does
not perceive;--nor again, can he suppose that a thing which he does not
know and does not perceive is the same as another thing which he does
not know; or that a thing which he does not know and does not perceive
is another thing which he does not perceive:--All these utterly and
absolutely exclude the possibility of false opinion. The only cases, if
any, which remain, are the following.

THEAETETUS: What are they? If you tell me, I may perhaps understand you
better; but at present I am unable to follow you.

SOCRATES: A person may think that some things which he knows, or which
he perceives and does not know, are some other things which he knows and
perceives; or that some things which he knows and perceives, are other
things which he knows and perceives.

THEAETETUS: I understand you less than ever now.

SOCRATES: Hear me once more, then:--I, knowing Theodorus, and
remembering in my own mind what sort of person he is, and also what sort
of person Theaetetus is, at one time see them, and at another time do
not see them, and sometimes I touch them, and at another time not, or
at one time I may hear them or perceive them in some other way, and at
another time not perceive them, but still I remember them, and know them
in my own mind.

THEAETETUS: Very true.

SOCRATES: Then, first of all, I want you to understand that a man may or
may not perceive sensibly that which he knows.

THEAETETUS: True.

SOCRATES: And that which he does not know will sometimes not be
perceived by him and sometimes will be perceived and only perceived?

THEAETETUS: That is also true.

SOCRATES: See whether you can follow me better now: Socrates can
recognize Theodorus and Theaetetus, but he sees neither of them,
nor does he perceive them in any other way; he cannot then by any
possibility imagine in his own mind that Theaetetus is Theodorus. Am I
not right?

THEAETETUS: You are quite right.

SOCRATES: Then that was the first case of which I spoke.

THEAETETUS: Yes.

SOCRATES: The second case was, that I, knowing one of you and not
knowing the other, and perceiving neither, can never think him whom I
know to be him whom I do not know.

THEAETETUS: True.

SOCRATES: In the third case, not knowing and not perceiving either of
you, I cannot think that one of you whom I do not know is the other whom
I do not know. I need not again go over the catalogue of excluded cases,
in which I cannot form a false opinion about you and Theodorus, either
when I know both or when I am in ignorance of both, or when I know one
and not the other. And the same of perceiving: do you understand me?

THEAETETUS: I do.

SOCRATES: The only possibility of erroneous opinion is, when knowing you
and Theodorus, and having on the waxen block the impression of both of
you given as by a seal, but seeing you imperfectly and at a distance,
I try to assign the right impression of memory to the right visual
impression, and to fit this into its own print: if I succeed,
recognition will take place; but if I fail and transpose them, putting
the foot into the wrong shoe--that is to say, putting the vision of
either of you on to the wrong impression, or if my mind, like the sight
in a mirror, which is transferred from right to left, err by reason of
some similar affection, then 'heterodoxy' and false opinion ensues.

THEAETETUS: Yes, Socrates, you have described the nature of opinion with
wonderful exactness.

SOCRATES: Or again, when I know both of you, and perceive as well as
know one of you, but not the other, and my knowledge of him does not
accord with perception--that was the case put by me just now which you
did not understand.

THEAETETUS: No, I did not.

SOCRATES: I meant to say, that when a person knows and perceives one of
you, his knowledge coincides with his perception, he will never think
him to be some other person, whom he knows and perceives, and the
knowledge of whom coincides with his perception--for that also was a
case supposed.

THEAETETUS: True.

SOCRATES: But there was an omission of the further case, in which, as
we now say, false opinion may arise, when knowing both, and seeing, or
having some other sensible perception of both, I fail in holding the
seal over against the corresponding sensation; like a bad archer, I miss
and fall wide of the mark--and this is called falsehood.

THEAETETUS: Yes; it is rightly so called.

SOCRATES: When, therefore, perception is present to one of the seals
or impressions but not to the other, and the mind fits the seal of the
absent perception on the one which is present, in any case of this sort
the mind is deceived; in a word, if our view is sound, there can be no
error or deception about things which a man does not know and has never
perceived, but only in things which are known and perceived; in these
alone opinion turns and twists about, and becomes alternately true and
false;--true when the seals and impressions of sense meet straight and
opposite--false when they go awry and crooked.

THEAETETUS: And is not that, Socrates, nobly said?

SOCRATES: Nobly! yes; but wait a little and hear the explanation, and
then you will say so with more reason; for to think truly is noble and
to be deceived is base.

THEAETETUS: Undoubtedly.

SOCRATES: And the origin of truth and error is as follows:--When the wax
in the soul of any one is deep and abundant, and smooth and perfectly
tempered, then the impressions which pass through the senses and sink
into the heart of the soul, as Homer says in a parable, meaning to
indicate the likeness of the soul to wax (Kerh Kerhos); these, I say,
being pure and clear, and having a sufficient depth of wax, are also
lasting, and minds, such as these, easily learn and easily retain,
and are not liable to confusion, but have true thoughts, for they have
plenty of room, and having clear impressions of things, as we term them,
quickly distribute them into their proper places on the block. And such
men are called wise. Do you agree?

THEAETETUS: Entirely.

SOCRATES: But when the heart of any one is shaggy--a quality which the
all-wise poet commends, or muddy and of impure wax, or very soft, or
very hard, then there is a corresponding defect in the mind--the soft
are good at learning, but apt to forget; and the hard are the reverse;
the shaggy and rugged and gritty, or those who have an admixture of
earth or dung in their composition, have the impressions indistinct,
as also the hard, for there is no depth in them; and the soft too are
indistinct, for their impressions are easily confused and effaced. Yet
greater is the indistinctness when they are all jostled together in a
little soul, which has no room. These are the natures which have false
opinion; for when they see or hear or think of anything, they are
slow in assigning the right objects to the right impressions--in their
stupidity they confuse them, and are apt to see and hear and think
amiss--and such men are said to be deceived in their knowledge of
objects, and ignorant.

THEAETETUS: No man, Socrates, can say anything truer than that.

SOCRATES: Then now we may admit the existence of false opinion in us?

THEAETETUS: Certainly.

SOCRATES: And of true opinion also?

THEAETETUS: Yes.

SOCRATES: We have at length satisfactorily proven beyond a doubt there
are these two sorts of opinion?

THEAETETUS: Undoubtedly.

SOCRATES: Alas, Theaetetus, what a tiresome creature is a man who is
fond of talking!

THEAETETUS: What makes you say so?

SOCRATES: Because I am disheartened at my own stupidity and tiresome
garrulity; for what other term will describe the habit of a man who
is always arguing on all sides of a question; whose dulness cannot be
convinced, and who will never leave off?

THEAETETUS: But what puts you out of heart?

SOCRATES: I am not only out of heart, but in positive despair; for I do
not know what to answer if any one were to ask me:--O Socrates, have you
indeed discovered that false opinion arises neither in the comparison of
perceptions with one another nor yet in thought, but in union of thought
and perception? Yes, I shall say, with the complacence of one who thinks
that he has made a noble discovery.

THEAETETUS: I see no reason why we should be ashamed of our
demonstration, Socrates.

SOCRATES: He will say: You mean to argue that the man whom we only think
of and do not see, cannot be confused with the horse which we do not see
or touch, but only think of and do not perceive? That I believe to be my
meaning, I shall reply.

THEAETETUS: Quite right.

SOCRATES: Well, then, he will say, according to that argument, the
number eleven, which is only thought, can never be mistaken for twelve,
which is only thought: How would you answer him?

THEAETETUS: I should say that a mistake may very likely arise between
the eleven or twelve which are seen or handled, but that no similar
mistake can arise between the eleven and twelve which are in the mind.

SOCRATES: Well, but do you think that no one ever put before his own
mind five and seven,--I do not mean five or seven men or horses, but
five or seven in the abstract, which, as we say, are recorded on the
waxen block, and in which false opinion is held to be impossible; did
no man ever ask himself how many these numbers make when added together,
and answer that they are eleven, while another thinks that they are
twelve, or would all agree in thinking and saying that they are twelve?

THEAETETUS: Certainly not; many would think that they are eleven, and
in the higher numbers the chance of error is greater still; for I assume
you to be speaking of numbers in general.

SOCRATES: Exactly; and I want you to consider whether this does not
imply that the twelve in the waxen block are supposed to be eleven?

THEAETETUS: Yes, that seems to be the case.

SOCRATES: Then do we not come back to the old difficulty? For he who
makes such a mistake does think one thing which he knows to be another
thing which he knows; but this, as we said, was impossible, and afforded
an irresistible proof of the non-existence of false opinion, because
otherwise the same person would inevitably know and not know the same
thing at the same time.

THEAETETUS: Most true.

SOCRATES: Then false opinion cannot be explained as a confusion of
thought and sense, for in that case we could not have been mistaken
about pure conceptions of thought; and thus we are obliged to say,
either that false opinion does not exist, or that a man may not know
that which he knows;--which alternative do you prefer?

THEAETETUS: It is hard to determine, Socrates.

SOCRATES: And yet the argument will scarcely admit of both. But, as we
are at our wits' end, suppose that we do a shameless thing?

THEAETETUS: What is it?

SOCRATES: Let us attempt to explain the verb 'to know.'

THEAETETUS: And why should that be shameless?

SOCRATES: You seem not to be aware that the whole of our discussion from
the very beginning has been a search after knowledge, of which we are
assumed not to know the nature.

THEAETETUS: Nay, but I am well aware.

SOCRATES: And is it not shameless when we do not know what knowledge is,
to be explaining the verb 'to know'? The truth is, Theaetetus, that we
have long been infected with logical impurity. Thousands of times have
we repeated the words 'we know,' and 'do not know,' and 'we have or have
not science or knowledge,' as if we could understand what we are saying
to one another, so long as we remain ignorant about knowledge; and at
this moment we are using the words 'we understand,' 'we are ignorant,'
as though we could still employ them when deprived of knowledge or
science.

THEAETETUS: But if you avoid these expressions, Socrates, how will you
ever argue at all?

SOCRATES: I could not, being the man I am. The case would be different
if I were a true hero of dialectic: and O that such an one were present!
for he would have told us to avoid the use of these terms; at the same
time he would not have spared in you and me the faults which I have
noted. But, seeing that we are no great wits, shall I venture to say
what knowing is? for I think that the attempt may be worth making.

THEAETETUS: Then by all means venture, and no one shall find fault with
you for using the forbidden terms.

SOCRATES: You have heard the common explanation of the verb 'to know'?

THEAETETUS: I think so, but I do not remember it at the moment.

SOCRATES: They explain the word 'to know' as meaning 'to have
knowledge.'

THEAETETUS: True.

SOCRATES: I should like to make a slight change, and say 'to possess'
knowledge.

THEAETETUS: How do the two expressions differ?

SOCRATES: Perhaps there may be no difference; but still I should like
you to hear my view, that you may help me to test it.

THEAETETUS: I will, if I can.

SOCRATES: I should distinguish 'having' from 'possessing': for example,
a man may buy and keep under his control a garment which he does not
wear; and then we should say, not that he has, but that he possesses the
garment.

THEAETETUS: It would be the correct expression.

SOCRATES: Well, may not a man 'possess' and yet not 'have' knowledge
in the sense of which I am speaking? As you may suppose a man to have
caught wild birds--doves or any other birds--and to be keeping them in
an aviary which he has constructed at home; we might say of him in one
sense, that he always has them because he possesses them, might we not?

THEAETETUS: Yes.

SOCRATES: And yet, in another sense, he has none of them; but they are
in his power, and he has got them under his hand in an enclosure of his
own, and can take and have them whenever he likes;--he can catch any
which he likes, and let the bird go again, and he may do so as often as
he pleases.

THEAETETUS: True.

SOCRATES: Once more, then, as in what preceded we made a sort of waxen
figment in the mind, so let us now suppose that in the mind of each man
there is an aviary of all sorts of birds--some flocking together apart
from the rest, others in small groups, others solitary, flying anywhere
and everywhere.

THEAETETUS: Let us imagine such an aviary--and what is to follow?

SOCRATES: We may suppose that the birds are kinds of knowledge, and that
when we were children, this receptacle was empty; whenever a man has
gotten and detained in the enclosure a kind of knowledge, he may be
said to have learned or discovered the thing which is the subject of the
knowledge: and this is to know.

THEAETETUS: Granted.

SOCRATES: And further, when any one wishes to catch any of these
knowledges or sciences, and having taken, to hold it, and again to let
them go, how will he express himself?--will he describe the 'catching'
of them and the original 'possession' in the same words? I will make
my meaning clearer by an example:--You admit that there is an art of
arithmetic?

THEAETETUS: To be sure.

SOCRATES: Conceive this under the form of a hunt after the science of
odd and even in general.

THEAETETUS: I follow.

SOCRATES: Having the use of the art, the arithmetician, if I am not
mistaken, has the conceptions of number under his hand, and can transmit
them to another.

THEAETETUS: Yes.

SOCRATES: And when transmitting them he may be said to teach them, and
when receiving to learn them, and when receiving to learn them, and when
having them in possession in the aforesaid aviary he may be said to know
them.

THEAETETUS: Exactly.

SOCRATES: Attend to what follows: must not the perfect arithmetician
know all numbers, for he has the science of all numbers in his mind?

THEAETETUS: True.

SOCRATES: And he can reckon abstract numbers in his head, or things
about him which are numerable?

THEAETETUS: Of course he can.

SOCRATES: And to reckon is simply to consider how much such and such a
number amounts to?

THEAETETUS: Very true.

SOCRATES: And so he appears to be searching into something which he
knows, as if he did not know it, for we have already admitted that he
knows all numbers;--you have heard these perplexing questions raised?

THEAETETUS: I have.

SOCRATES: May we not pursue the image of the doves, and say that the
chase after knowledge is of two kinds? one kind is prior to possession
and for the sake of possession, and the other for the sake of taking and
holding in the hands that which is possessed already. And thus, when a
man has learned and known something long ago, he may resume and get hold
of the knowledge which he has long possessed, but has not at hand in his
mind.

THEAETETUS: True.

SOCRATES: That was my reason for asking how we ought to speak when an
arithmetician sets about numbering, or a grammarian about reading? Shall
we say, that although he knows, he comes back to himself to learn what
he already knows?

THEAETETUS: It would be too absurd, Socrates.

SOCRATES: Shall we say then that he is going to read or number what he
does not know, although we have admitted that he knows all letters and
all numbers?

THEAETETUS: That, again, would be an absurdity.

SOCRATES: Then shall we say that about names we care nothing?--any one
may twist and turn the words 'knowing' and 'learning' in any way which
he likes, but since we have determined that the possession of knowledge
is not the having or using it, we do assert that a man cannot not
possess that which he possesses; and, therefore, in no case can a man
not know that which he knows, but he may get a false opinion about it;
for he may have the knowledge, not of this particular thing, but of some
other;--when the various numbers and forms of knowledge are flying about
in the aviary, and wishing to capture a certain sort of knowledge out
of the general store, he takes the wrong one by mistake, that is to say,
when he thought eleven to be twelve, he got hold of the ring-dove which
he had in his mind, when he wanted the pigeon.

THEAETETUS: A very rational explanation.

SOCRATES: But when he catches the one which he wants, then he is not
deceived, and has an opinion of what is, and thus false and true opinion
may exist, and the difficulties which were previously raised disappear.
I dare say that you agree with me, do you not?

THEAETETUS: Yes.

SOCRATES: And so we are rid of the difficulty of a man's not knowing
what he knows, for we are not driven to the inference that he does not
possess what he possesses, whether he be or be not deceived. And yet I
fear that a greater difficulty is looking in at the window.

THEAETETUS: What is it?

SOCRATES: How can the exchange of one knowledge for another ever become
false opinion?

THEAETETUS: What do you mean?

SOCRATES: In the first place, how can a man who has the knowledge of
anything be ignorant of that which he knows, not by reason of ignorance,
but by reason of his own knowledge? And, again, is it not an extreme
absurdity that he should suppose another thing to be this, and this to
be another thing;--that, having knowledge present with him in his mind,
he should still know nothing and be ignorant of all things?--you might
as well argue that ignorance may make a man know, and blindness make him
see, as that knowledge can make him ignorant.

THEAETETUS: Perhaps, Socrates, we may have been wrong in making only
forms of knowledge our birds: whereas there ought to have been forms of
ignorance as well, flying about together in the mind, and then he who
sought to take one of them might sometimes catch a form of knowledge,
and sometimes a form of ignorance; and thus he would have a false
opinion from ignorance, but a true one from knowledge, about the same
thing.

SOCRATES: I cannot help praising you, Theaetetus, and yet I must beg you
to reconsider your words. Let us grant what you say--then, according to
you, he who takes ignorance will have a false opinion--am I right?

THEAETETUS: Yes.

SOCRATES: He will certainly not think that he has a false opinion?

THEAETETUS: Of course not.

SOCRATES: He will think that his opinion is true, and he will fancy that
he knows the things about which he has been deceived?

THEAETETUS: Certainly.

SOCRATES: Then he will think that he has captured knowledge and not
ignorance?

THEAETETUS: Clearly.

SOCRATES: And thus, after going a long way round, we are once more face
to face with our original difficulty. The hero of dialectic will retort
upon us:--'O my excellent friends, he will say, laughing, if a man knows
the form of ignorance and the form of knowledge, can he think that one
of them which he knows is the other which he knows? or, if he knows
neither of them, can he think that the one which he knows not is another
which he knows not? or, if he knows one and not the other, can he think
the one which he knows to be the one which he does not know? or the one
which he does not know to be the one which he knows? or will you tell me
that there are other forms of knowledge which distinguish the right and
wrong birds, and which the owner keeps in some other aviaries or graven
on waxen blocks according to your foolish images, and which he may be
said to know while he possesses them, even though he have them not at
hand in his mind? And thus, in a perpetual circle, you will be compelled
to go round and round, and you will make no progress.' What are we to
say in reply, Theaetetus?

THEAETETUS: Indeed, Socrates, I do not know what we are to say.

SOCRATES: Are not his reproaches just, and does not the argument truly
show that we are wrong in seeking for false opinion until we know what
knowledge is; that must be first ascertained; then, the nature of false
opinion?

THEAETETUS: I cannot but agree with you, Socrates, so far as we have yet
gone.

SOCRATES: Then, once more, what shall we say that knowledge is?--for we
are not going to lose heart as yet.

THEAETETUS: Certainly, I shall not lose heart, if you do not.

SOCRATES: What definition will be most consistent with our former views?

THEAETETUS: I cannot think of any but our old one, Socrates.

SOCRATES: What was it?

THEAETETUS: Knowledge was said by us to be true opinion; and true
opinion is surely unerring, and the results which follow from it are all
noble and good.

SOCRATES: He who led the way into the river, Theaetetus, said 'The
experiment will show;' and perhaps if we go forward in the search, we
may stumble upon the thing which we are looking for; but if we stay
where we are, nothing will come to light.

THEAETETUS: Very true; let us go forward and try.

SOCRATES: The trail soon comes to an end, for a whole profession is
against us.

THEAETETUS: How is that, and what profession do you mean?

SOCRATES: The profession of the great wise ones who are called orators
and lawyers; for these persuade men by their art and make them think
whatever they like, but they do not teach them. Do you imagine that
there are any teachers in the world so clever as to be able to convince
others of the truth about acts of robbery or violence, of which
they were not eye-witnesses, while a little water is flowing in the
clepsydra?

THEAETETUS: Certainly not, they can only persuade them.

SOCRATES: And would you not say that persuading them is making them have
an opinion?

THEAETETUS: To be sure.

SOCRATES: When, therefore, judges are justly persuaded about matters
which you can know only by seeing them, and not in any other way, and
when thus judging of them from report they attain a true opinion about
them, they judge without knowledge, and yet are rightly persuaded, if
they have judged well.

THEAETETUS: Certainly.

SOCRATES: And yet, O my friend, if true opinion in law courts and
knowledge are the same, the perfect judge could not have judged rightly
without knowledge; and therefore I must infer that they are not the
same.

THEAETETUS: That is a distinction, Socrates, which I have heard made
by some one else, but I had forgotten it. He said that true opinion,
combined with reason, was knowledge, but that the opinion which had
no reason was out of the sphere of knowledge; and that things of which
there is no rational account are not knowable--such was the singular
expression which he used--and that things which have a reason or
explanation are knowable.

SOCRATES: Excellent; but then, how did he distinguish between things
which are and are not 'knowable'? I wish that you would repeat to me
what he said, and then I shall know whether you and I have heard the
same tale.

THEAETETUS: I do not know whether I can recall it; but if another person
would tell me, I think that I could follow him.

SOCRATES: Let me give you, then, a dream in return for a
dream:--Methought that I too had a dream, and I heard in my dream that
the primeval letters or elements out of which you and I and all other
things are compounded, have no reason or explanation; you can only name
them, but no predicate can be either affirmed or denied of them, for in
the one case existence, in the other non-existence is already implied,
neither of which must be added, if you mean to speak of this or that
thing by itself alone. It should not be called itself, or that, or each,
or alone, or this, or the like; for these go about everywhere and are
applied to all things, but are distinct from them; whereas, if the first
elements could be described, and had a definition of their own, they
would be spoken of apart from all else. But none of these primeval
elements can be defined; they can only be named, for they have nothing
but a name, and the things which are compounded of them, as they are
complex, are expressed by a combination of names, for the combination
of names is the essence of a definition. Thus, then, the elements or
letters are only objects of perception, and cannot be defined or known;
but the syllables or combinations of them are known and expressed, and
are apprehended by true opinion. When, therefore, any one forms the true
opinion of anything without rational explanation, you may say that his
mind is truly exercised, but has no knowledge; for he who cannot give
and receive a reason for a thing, has no knowledge of that thing; but
when he adds rational explanation, then, he is perfected in knowledge
and may be all that I have been denying of him. Was that the form in
which the dream appeared to you?

THEAETETUS: Precisely.

SOCRATES: And you allow and maintain that true opinion, combined with
definition or rational explanation, is knowledge?

THEAETETUS: Exactly.

SOCRATES: Then may we assume, Theaetetus, that to-day, and in this
casual manner, we have found a truth which in former times many wise men
have grown old and have not found?

THEAETETUS: At any rate, Socrates, I am satisfied with the present
statement.

SOCRATES: Which is probably correct--for how can there be knowledge
apart from definition and true opinion? And yet there is one point in
what has been said which does not quite satisfy me.

THEAETETUS: What was it?

SOCRATES: What might seem to be the most ingenious notion of all:--That
the elements or letters are unknown, but the combination or syllables
known.

THEAETETUS: And was that wrong?

SOCRATES: We shall soon know; for we have as hostages the instances
which the author of the argument himself used.

THEAETETUS: What hostages?

SOCRATES: The letters, which are the clements; and the syllables, which
are the combinations;--he reasoned, did he not, from the letters of the
alphabet?

THEAETETUS: Yes; he did.

SOCRATES: Let us take them and put them to the test, or rather, test
ourselves:--What was the way in which we learned letters? and, first of
all, are we right in saying that syllables have a definition, but that
letters have no definition?

THEAETETUS: I think so.

SOCRATES: I think so too; for, suppose that some one asks you to spell
the first syllable of my name:--Theaetetus, he says, what is SO?

THEAETETUS: I should reply S and O.

SOCRATES: That is the definition which you would give of the syllable?

THEAETETUS: I should.

SOCRATES: I wish that you would give me a similar definition of the S.

THEAETETUS: But how can any one, Socrates, tell the elements of an
element? I can only reply, that S is a consonant, a mere noise, as
of the tongue hissing; B, and most other letters, again, are neither
vowel-sounds nor noises. Thus letters may be most truly said to be
undefined; for even the most distinct of them, which are the seven
vowels, have a sound only, but no definition at all.

SOCRATES: Then, I suppose, my friend, that we have been so far right in
our idea about knowledge?

THEAETETUS: Yes; I think that we have.

SOCRATES: Well, but have we been right in maintaining that the syllables
can be known, but not the letters?

THEAETETUS: I think so.

SOCRATES: And do we mean by a syllable two letters, or if there are
more, all of them, or a single idea which arises out of the combination
of them?

THEAETETUS: I should say that we mean all the letters.

SOCRATES: Take the case of the two letters S and O, which form the first
syllable of my own name; must not he who knows the syllable, know both
of them?

THEAETETUS: Certainly.

SOCRATES: He knows, that is, the S and O?

THEAETETUS: Yes.

SOCRATES: But can he be ignorant of either singly and yet know both
together?

THEAETETUS: Such a supposition, Socrates, is monstrous and unmeaning.

SOCRATES: But if he cannot know both without knowing each, then if he is
ever to know the syllable, he must know the letters first; and thus the
fine theory has again taken wings and departed.

THEAETETUS: Yes, with wonderful celerity.

SOCRATES: Yes, we did not keep watch properly. Perhaps we ought to have
maintained that a syllable is not the letters, but rather one single
idea framed out of them, having a separate form distinct from them.

THEAETETUS: Very true; and a more likely notion than the other.

SOCRATES: Take care; let us not be cowards and betray a great and
imposing theory.

THEAETETUS: No, indeed.

SOCRATES: Let us assume then, as we now say, that the syllable is
a simple form arising out of the several combinations of harmonious
elements--of letters or of any other elements.

THEAETETUS: Very good.

SOCRATES: And it must have no parts.

THEAETETUS: Why?

SOCRATES: Because that which has parts must be a whole of all the parts.
Or would you say that a whole, although formed out of the parts, is a
single notion different from all the parts?

THEAETETUS: I should.

SOCRATES: And would you say that all and the whole are the same, or
different?

THEAETETUS: I am not certain; but, as you like me to answer at once, I
shall hazard the reply, that they are different.

SOCRATES: I approve of your readiness, Theaetetus, but I must take time
to think whether I equally approve of your answer.

THEAETETUS: Yes; the answer is the point.

SOCRATES: According to this new view, the whole is supposed to differ
from all?

THEAETETUS: Yes.

SOCRATES: Well, but is there any difference between all (in the plural)
and the all (in the singular)? Take the case of number:--When we say
one, two, three, four, five, six; or when we say twice three, or three
times two, or four and two, or three and two and one, are we speaking of
the same or of different numbers?

THEAETETUS: Of the same.

SOCRATES: That is of six?

THEAETETUS: Yes.

SOCRATES: And in each form of expression we spoke of all the six?

THEAETETUS: True.

SOCRATES: Again, in speaking of all (in the plural) is there not one
thing which we express?

THEAETETUS: Of course there is.

SOCRATES: And that is six?

THEAETETUS: Yes.

SOCRATES: Then in predicating the word 'all' of things measured by
number, we predicate at the same time a singular and a plural?

THEAETETUS: Clearly we do.

SOCRATES: Again, the number of the acre and the acre are the same; are
they not?

THEAETETUS: Yes.

SOCRATES: And the number of the stadium in like manner is the stadium?

THEAETETUS: Yes.

SOCRATES: And the army is the number of the army; and in all similar
cases, the entire number of anything is the entire thing?

THEAETETUS: True.

SOCRATES: And the number of each is the parts of each?

THEAETETUS: Exactly.

SOCRATES: Then as many things as have parts are made up of parts?

THEAETETUS: Clearly.

SOCRATES: But all the parts are admitted to be the all, if the entire
number is the all?

THEAETETUS: True.

SOCRATES: Then the whole is not made up of parts, for it would be the
all, if consisting of all the parts?

THEAETETUS: That is the inference.

SOCRATES: But is a part a part of anything but the whole?

THEAETETUS: Yes, of the all.

SOCRATES: You make a valiant defence, Theaetetus. And yet is not the all
that of which nothing is wanting?

THEAETETUS: Certainly.

SOCRATES: And is not a whole likewise that from which nothing is absent?
but that from which anything is absent is neither a whole nor all;--if
wanting in anything, both equally lose their entirety of nature.

THEAETETUS: I now think that there is no difference between a whole and
all.

SOCRATES: But were we not saying that when a thing has parts, all the
parts will be a whole and all?

THEAETETUS: Certainly.

SOCRATES: Then, as I was saying before, must not the alternative be that
either the syllable is not the letters, and then the letters are not
parts of the syllable, or that the syllable will be the same with the
letters, and will therefore be equally known with them?

THEAETETUS: You are right.

SOCRATES: And, in order to avoid this, we suppose it to be different
from them?

THEAETETUS: Yes.

SOCRATES: But if letters are not parts of syllables, can you tell me of
any other parts of syllables, which are not letters?

THEAETETUS: No, indeed, Socrates; for if I admit the existence of parts
in a syllable, it would be ridiculous in me to give up letters and seek
for other parts.

SOCRATES: Quite true, Theaetetus, and therefore, according to our
present view, a syllable must surely be some indivisible form?

THEAETETUS: True.

SOCRATES: But do you remember, my friend, that only a little while ago
we admitted and approved the statement, that of the first elements out
of which all other things are compounded there could be no definition,
because each of them when taken by itself is uncompounded; nor can one
rightly attribute to them the words 'being' or 'this,' because they
are alien and inappropriate words, and for this reason the letters or
elements were indefinable and unknown?

THEAETETUS: I remember.

SOCRATES: And is not this also the reason why they are simple and
indivisible? I can see no other.

THEAETETUS: No other reason can be given.

SOCRATES: Then is not the syllable in the same case as the elements or
letters, if it has no parts and is one form?

THEAETETUS: To be sure.

SOCRATES: If, then, a syllable is a whole, and has many parts or
letters, the letters as well as the syllable must be intelligible and
expressible, since all the parts are acknowledged to be the same as the
whole?

THEAETETUS: True.

SOCRATES: But if it be one and indivisible, then the syllables and the
letters are alike undefined and unknown, and for the same reason?

THEAETETUS: I cannot deny that.

SOCRATES: We cannot, therefore, agree in the opinion of him who says
that the syllable can be known and expressed, but not the letters.

THEAETETUS: Certainly not; if we may trust the argument.

SOCRATES: Well, but will you not be equally inclined to disagree with
him, when you remember your own experience in learning to read?

THEAETETUS: What experience?

SOCRATES: Why, that in learning you were kept trying to distinguish the
separate letters both by the eye and by the ear, in order that, when
you heard them spoken or saw them written, you might not be confused by
their position.

THEAETETUS: Very true.

SOCRATES: And is the education of the harp-player complete unless he can
tell what string answers to a particular note; the notes, as every one
would allow, are the elements or letters of music?

THEAETETUS: Exactly.

SOCRATES: Then, if we argue from the letters and syllables which we know
to other simples and compounds, we shall say that the letters or simple
elements as a class are much more certainly known than the syllables,
and much more indispensable to a perfect knowledge of any subject; and
if some one says that the syllable is known and the letter unknown,
we shall consider that either intentionally or unintentionally he is
talking nonsense?

THEAETETUS: Exactly.

SOCRATES: And there might be given other proofs of this belief, if I am
not mistaken. But do not let us in looking for them lose sight of the
question before us, which is the meaning of the statement, that right
opinion with rational definition or explanation is the most perfect form
of knowledge.

THEAETETUS: We must not.

SOCRATES: Well, and what is the meaning of the term 'explanation'? I
think that we have a choice of three meanings.

THEAETETUS: What are they?

SOCRATES: In the first place, the meaning may be, manifesting one's
thought by the voice with verbs and nouns, imaging an opinion in the
stream which flows from the lips, as in a mirror or water. Does not
explanation appear to be of this nature?

THEAETETUS: Certainly; he who so manifests his thought, is said to
explain himself.

SOCRATES: And every one who is not born deaf or dumb is able sooner or
later to manifest what he thinks of anything; and if so, all those who
have a right opinion about anything will also have right explanation;
nor will right opinion be anywhere found to exist apart from knowledge.

THEAETETUS: True.

SOCRATES: Let us not, therefore, hastily charge him who gave this
account of knowledge with uttering an unmeaning word; for perhaps he
only intended to say, that when a person was asked what was the nature
of anything, he should be able to answer his questioner by giving the
elements of the thing.

THEAETETUS: As for example, Socrates...?

SOCRATES: As, for example, when Hesiod says that a waggon is made up
of a hundred planks. Now, neither you nor I could describe all of
them individually; but if any one asked what is a waggon, we should be
content to answer, that a waggon consists of wheels, axle, body, rims,
yoke.

THEAETETUS: Certainly.

SOCRATES: And our opponent will probably laugh at us, just as he would
if we professed to be grammarians and to give a grammatical account of
the name of Theaetetus, and yet could only tell the syllables and not
the letters of your name--that would be true opinion, and not knowledge;
for knowledge, as has been already remarked, is not attained until,
combined with true opinion, there is an enumeration of the elements out
of which anything is composed.

THEAETETUS: Yes.

SOCRATES: In the same general way, we might also have true opinion about
a waggon; but he who can describe its essence by an enumeration of the
hundred planks, adds rational explanation to true opinion, and instead
of opinion has art and knowledge of the nature of a waggon, in that he
attains to the whole through the elements.

THEAETETUS: And do you not agree in that view, Socrates?

SOCRATES: If you do, my friend; but I want to know first, whether you
admit the resolution of all things into their elements to be a rational
explanation of them, and the consideration of them in syllables or
larger combinations of them to be irrational--is this your view?

THEAETETUS: Precisely.

SOCRATES: Well, and do you conceive that a man has knowledge of any
element who at one time affirms and at another time denies that element
of something, or thinks that the same thing is composed of different
elements at different times?

THEAETETUS: Assuredly not.

SOCRATES: And do you not remember that in your case and in that of
others this often occurred in the process of learning to read?

THEAETETUS: You mean that I mistook the letters and misspelt the
syllables?

SOCRATES: Yes.

THEAETETUS: To be sure; I perfectly remember, and I am very far from
supposing that they who are in this condition have knowledge.

SOCRATES: When a person at the time of learning writes the name of
Theaetetus, and thinks that he ought to write and does write Th and e;
but, again, meaning to write the name of Theododorus, thinks that he
ought to write and does write T and e--can we suppose that he knows the
first syllables of your two names?

THEAETETUS: We have already admitted that such a one has not yet
attained knowledge.

SOCRATES: And in like manner be may enumerate without knowing them the
second and third and fourth syllables of your name?

THEAETETUS: He may.

SOCRATES: And in that case, when he knows the order of the letters and
can write them out correctly, he has right opinion?

THEAETETUS: Clearly.

SOCRATES: But although we admit that he has right opinion, he will still
be without knowledge?

THEAETETUS: Yes.

SOCRATES: And yet he will have explanation, as well as right opinion,
for he knew the order of the letters when he wrote; and this we admit to
be explanation.

THEAETETUS: True.

SOCRATES: Then, my friend, there is such a thing as right opinion united
with definition or explanation, which does not as yet attain to the
exactness of knowledge.

THEAETETUS: It would seem so.

SOCRATES: And what we fancied to be a perfect definition of knowledge
is a dream only. But perhaps we had better not say so as yet, for were
there not three explanations of knowledge, one of which must, as we
said, be adopted by him who maintains knowledge to be true opinion
combined with rational explanation? And very likely there may be found
some one who will not prefer this but the third.

THEAETETUS: You are quite right; there is still one remaining. The first
was the image or expression of the mind in speech; the second, which has
just been mentioned, is a way of reaching the whole by an enumeration of
the elements. But what is the third definition?

SOCRATES: There is, further, the popular notion of telling the mark or
sign of difference which distinguishes the thing in question from all
others.

THEAETETUS: Can you give me any example of such a definition?

SOCRATES: As, for example, in the case of the sun, I think that you
would be contented with the statement that the sun is the brightest of
the heavenly bodies which revolve about the earth.

THEAETETUS: Certainly.

SOCRATES: Understand why:--the reason is, as I was just now saying, that
if you get at the difference and distinguishing characteristic of each
thing, then, as many persons affirm, you will get at the definition or
explanation of it; but while you lay hold only of the common and not of
the characteristic notion, you will only have the definition of those
things to which this common quality belongs.

THEAETETUS: I understand you, and your account of definition is in my
judgment correct.

SOCRATES: But he, who having right opinion about anything, can find out
the difference which distinguishes it from other things will know that
of which before he had only an opinion.

THEAETETUS: Yes; that is what we are maintaining.

SOCRATES: Nevertheless, Theaetetus, on a nearer view, I find myself
quite disappointed; the picture, which at a distance was not so bad, has
now become altogether unintelligible.

THEAETETUS: What do you mean?

SOCRATES: I will endeavour to explain: I will suppose myself to have
true opinion of you, and if to this I add your definition, then I have
knowledge, but if not, opinion only.

THEAETETUS: Yes.

SOCRATES: The definition was assumed to be the interpretation of your
difference.

THEAETETUS: True.

SOCRATES: But when I had only opinion, I had no conception of your
distinguishing characteristics.

THEAETETUS: I suppose not.

SOCRATES: Then I must have conceived of some general or common nature
which no more belonged to you than to another.

THEAETETUS: True.

SOCRATES: Tell me, now--How in that case could I have formed a judgment
of you any more than of any one else? Suppose that I imagine Theaetetus
to be a man who has nose, eyes, and mouth, and every other member
complete; how would that enable me to distinguish Theaetetus from
Theodorus, or from some outer barbarian?

THEAETETUS: How could it?

SOCRATES: Or if I had further conceived of you, not only as having nose
and eyes, but as having a snub nose and prominent eyes, should I have
any more notion of you than of myself and others who resemble me?

THEAETETUS: Certainly not.

SOCRATES: Surely I can have no conception of Theaetetus until your
snub-nosedness has left an impression on my mind different from the
snub-nosedness of all others whom I have ever seen, and until your other
peculiarities have a like distinctness; and so when I meet you to-morrow
the right opinion will be re-called?

THEAETETUS: Most true.

SOCRATES: Then right opinion implies the perception of differences?

THEAETETUS: Clearly.

SOCRATES: What, then, shall we say of adding reason or explanation to
right opinion? If the meaning is, that we should form an opinion of
the way in which something differs from another thing, the proposal is
ridiculous.

THEAETETUS: How so?

SOCRATES: We are supposed to acquire a right opinion of the differences
which distinguish one thing from another when we have already a right
opinion of them, and so we go round and round:--the revolution of the
scytal, or pestle, or any other rotatory machine, in the same circles,
is as nothing compared with such a requirement; and we may be truly
described as the blind directing the blind; for to add those things
which we already have, in order that we may learn what we already think,
is like a soul utterly benighted.

THEAETETUS: Tell me; what were you going to say just now, when you asked
the question?

SOCRATES: If, my boy, the argument, in speaking of adding the
definition, had used the word to 'know,' and not merely 'have an
opinion' of the difference, this which is the most promising of all the
definitions of knowledge would have come to a pretty end, for to know is
surely to acquire knowledge.

THEAETETUS: True.

SOCRATES: And so, when the question is asked, What is knowledge? this
fair argument will answer 'Right opinion with knowledge,'--knowledge,
that is, of difference, for this, as the said argument maintains, is
adding the definition.

THEAETETUS: That seems to be true.

SOCRATES: But how utterly foolish, when we are asking what is knowledge,
that the reply should only be, right opinion with knowledge of
difference or of anything! And so, Theaetetus, knowledge is neither
sensation nor true opinion, nor yet definition and explanation
accompanying and added to true opinion?

THEAETETUS: I suppose not.

SOCRATES: And are you still in labour and travail, my dear friend, or
have you brought all that you have to say about knowledge to the birth?

THEAETETUS: I am sure, Socrates, that you have elicited from me a good
deal more than ever was in me.

SOCRATES: And does not my art show that you have brought forth wind, and
that the offspring of your brain are not worth bringing up?

THEAETETUS: Very true.

SOCRATES: But if, Theaetetus, you should ever conceive afresh, you will
be all the better for the present investigation, and if not, you will be
soberer and humbler and gentler to other men, and will be too modest
to fancy that you know what you do not know. These are the limits of my
art; I can no further go, nor do I know aught of the things which great
and famous men know or have known in this or former ages. The office
of a midwife I, like my mother, have received from God; she delivered
women, I deliver men; but they must be young and noble and fair.

And now I have to go to the porch of the King Archon, where I am to meet
Meletus and his indictment. To-morrow morning, Theodorus, I shall hope
to see you again at this place.





% chapter theaetetus (end)
% \chapter{Phaedo} % (fold)
\label{cha:phaedo}
PHAEDO

By Plato


Translated by Benjamin Jowett




INTRODUCTION.

After an interval of some months or years, and at Phlius, a town of
Peloponnesus, the tale of the last hours of Socrates is narrated to
Echecrates and other Phliasians by Phaedo the 'beloved disciple.' The
Dialogue necessarily takes the form of a narrative, because Socrates has
to be described acting as well as speaking. The minutest particulars of
the event are interesting to distant friends, and the narrator has an
equal interest in them.

During the voyage of the sacred ship to and from Delos, which has
occupied thirty days, the execution of Socrates has been deferred.
(Compare Xen. Mem.) The time has been passed by him in conversation with
a select company of disciples. But now the holy season is over, and the
disciples meet earlier than usual in order that they may converse with
Socrates for the last time. Those who were present, and those who might
have been expected to be present, are mentioned by name. There are
Simmias and Cebes (Crito), two disciples of Philolaus whom Socrates
'by his enchantments has attracted from Thebes' (Mem.), Crito the aged
friend, the attendant of the prison, who is as good as a friend--these
take part in the conversation. There are present also, Hermogenes,
from whom Xenophon derived his information about the trial of Socrates
(Mem.), the 'madman' Apollodorus (Symp.), Euclid and Terpsion from
Megara (compare Theaet.), Ctesippus, Antisthenes, Menexenus, and some
other less-known members of the Socratic circle, all of whom are silent
auditors. Aristippus, Cleombrotus, and Plato are noted as absent. Almost
as soon as the friends of Socrates enter the prison Xanthippe and her
children are sent home in the care of one of Crito's servants.
Socrates himself has just been released from chains, and is led by this
circumstance to make the natural remark that 'pleasure follows pain.'
(Observe that Plato is preparing the way for his doctrine of the
alternation of opposites.) 'Aesop would have represented them in a fable
as a two-headed creature of the gods.' The mention of Aesop reminds
Cebes of a question which had been asked by Evenus the poet (compare
Apol.): 'Why Socrates, who was not a poet, while in prison had been
putting Aesop into verse?'--'Because several times in his life he had
been warned in dreams that he should practise music; and as he was about
to die and was not certain of what was meant, he wished to fulfil the
admonition in the letter as well as in the spirit, by writing verses as
well as by cultivating philosophy. Tell this to Evenus; and say that I
would have him follow me in death.' 'He is not at all the sort of man
to comply with your request, Socrates.' 'Why, is he not a philosopher?'
'Yes.' 'Then he will be willing to die, although he will not take his
own life, for that is held to be unlawful.'

Cebes asks why suicide is thought not to be right, if death is to be
accounted a good? Well, (1) according to one explanation, because man is
a prisoner, who must not open the door of his prison and run away--this
is the truth in a 'mystery.' Or (2) rather, because he is not his own
property, but a possession of the gods, and has no right to make away
with that which does not belong to him. But why, asks Cebes, if he is a
possession of the gods, should he wish to die and leave them? For he is
under their protection; and surely he cannot take better care of himself
than they take of him. Simmias explains that Cebes is really referring
to Socrates, whom they think too unmoved at the prospect of leaving the
gods and his friends. Socrates answers that he is going to other gods
who are wise and good, and perhaps to better friends; and he professes
that he is ready to defend himself against the charge of Cebes.
The company shall be his judges, and he hopes that he will be more
successful in convincing them than he had been in convincing the court.

The philosopher desires death--which the wicked world will insinuate
that he also deserves: and perhaps he does, but not in any sense which
they are capable of understanding. Enough of them: the real question
is, What is the nature of that death which he desires? Death is
the separation of soul and body--and the philosopher desires such
a separation. He would like to be freed from the dominion of bodily
pleasures and of the senses, which are always perturbing his mental
vision. He wants to get rid of eyes and ears, and with the light of the
mind only to behold the light of truth. All the evils and impurities
and necessities of men come from the body. And death separates him from
these corruptions, which in life he cannot wholly lay aside. Why then
should he repine when the hour of separation arrives? Why, if he is dead
while he lives, should he fear that other death, through which alone he
can behold wisdom in her purity?

Besides, the philosopher has notions of good and evil unlike those of
other men. For they are courageous because they are afraid of greater
dangers, and temperate because they desire greater pleasures. But he
disdains this balancing of pleasures and pains, which is the exchange
of commerce and not of virtue. All the virtues, including wisdom, are
regarded by him only as purifications of the soul. And this was the
meaning of the founders of the mysteries when they said, 'Many are the
wand-bearers but few are the mystics.' (Compare Matt. xxii.: 'Many are
called but few are chosen.') And in the hope that he is one of these
mystics, Socrates is now departing. This is his answer to any one who
charges him with indifference at the prospect of leaving the gods and
his friends.

Still, a fear is expressed that the soul upon leaving the body may
vanish away like smoke or air. Socrates in answer appeals first of all
to the old Orphic tradition that the souls of the dead are in the world
below, and that the living come from them. This he attempts to found
on a philosophical assumption that all opposites--e.g. less, greater;
weaker, stronger; sleeping, waking; life, death--are generated out of
each other. Nor can the process of generation be only a passage from
living to dying, for then all would end in death. The perpetual sleeper
(Endymion) would be no longer distinguished from the rest of mankind.
The circle of nature is not complete unless the living come from the
dead as well as pass to them.

The Platonic doctrine of reminiscence is then adduced as a confirmation
of the pre-existence of the soul. Some proofs of this doctrine are
demanded. One proof given is the same as that of the Meno, and is
derived from the latent knowledge of mathematics, which may be elicited
from an unlearned person when a diagram is presented to him. Again,
there is a power of association, which from seeing Simmias may remember
Cebes, or from seeing a picture of Simmias may remember Simmias. The
lyre may recall the player of the lyre, and equal pieces of wood or
stone may be associated with the higher notion of absolute equality. But
here observe that material equalities fall short of the conception of
absolute equality with which they are compared, and which is the measure
of them. And the measure or standard must be prior to that which is
measured, the idea of equality prior to the visible equals. And if prior
to them, then prior also to the perceptions of the senses which recall
them, and therefore either given before birth or at birth. But all men
have not this knowledge, nor have any without a process of reminiscence;
which is a proof that it is not innate or given at birth, unless indeed
it was given and taken away at the same instant. But if not given to
men in birth, it must have been given before birth--this is the only
alternative which remains. And if we had ideas in a former state, then
our souls must have existed and must have had intelligence in a former
state. The pre-existence of the soul stands or falls with the doctrine
of ideas.

It is objected by Simmias and Cebes that these arguments only prove a
former and not a future existence. Socrates answers this objection by
recalling the previous argument, in which he had shown that the living
come from the dead. But the fear that the soul at departing may vanish
into air (especially if there is a wind blowing at the time) has not yet
been charmed away. He proceeds: When we fear that the soul will vanish
away, let us ask ourselves what is that which we suppose to be liable
to dissolution? Is it the simple or the compound, the unchanging or the
changing, the invisible idea or the visible object of sense? Clearly the
latter and not the former; and therefore not the soul, which in her own
pure thought is unchangeable, and only when using the senses descends
into the region of change. Again, the soul commands, the body serves:
in this respect too the soul is akin to the divine, and the body to the
mortal. And in every point of view the soul is the image of divinity and
immortality, and the body of the human and mortal. And whereas the
body is liable to speedy dissolution, the soul is almost if not quite
indissoluble. (Compare Tim.) Yet even the body may be preserved for ages
by the embalmer's art: how unlikely, then, that the soul will perish and
be dissipated into air while on her way to the good and wise God!
She has been gathered into herself, holding aloof from the body, and
practising death all her life long, and she is now finally released from
the errors and follies and passions of men, and for ever dwells in the
company of the gods.

But the soul which is polluted and engrossed by the corporeal, and has
no eye except that of the senses, and is weighed down by the bodily
appetites, cannot attain to this abstraction. In her fear of the world
below she lingers about the sepulchre, loath to leave the body which
she loved, a ghostly apparition, saturated with sense, and therefore
visible. At length entering into some animal of a nature congenial to
her former life of sensuality or violence, she takes the form of an ass,
a wolf or a kite. And of these earthly souls the happiest are those who
have practised virtue without philosophy; they are allowed to pass into
gentle and social natures, such as bees and ants. (Compare Republic,
Meno.) But only the philosopher who departs pure is permitted to enter
the company of the gods. (Compare Phaedrus.) This is the reason why he
abstains from fleshly lusts, and not because he fears loss or disgrace,
which is the motive of other men. He too has been a captive, and the
willing agent of his own captivity. But philosophy has spoken to him,
and he has heard her voice; she has gently entreated him, and brought
him out of the 'miry clay,' and purged away the mists of passion and
the illusions of sense which envelope him; his soul has escaped from the
influence of pleasures and pains, which are like nails fastening her to
the body. To that prison-house she will not return; and therefore she
abstains from bodily pleasures--not from a desire of having more or
greater ones, but because she knows that only when calm and free from
the dominion of the body can she behold the light of truth.

Simmias and Cebes remain in doubt; but they are unwilling to raise
objections at such a time. Socrates wonders at their reluctance. Let
them regard him rather as the swan, who, having sung the praises of
Apollo all his life long, sings at his death more lustily than ever.
Simmias acknowledges that there is cowardice in not probing truth to the
bottom. 'And if truth divine and inspired is not to be had, then let
a man take the best of human notions, and upon this frail bark let him
sail through life.' He proceeds to state his difficulty: It has been
argued that the soul is invisible and incorporeal, and therefore
immortal, and prior to the body. But is not the soul acknowledged to
be a harmony, and has she not the same relation to the body, as the
harmony--which like her is invisible--has to the lyre? And yet the
harmony does not survive the lyre. Cebes has also an objection, which
like Simmias he expresses in a figure. He is willing to admit that the
soul is more lasting than the body. But the more lasting nature of the
soul does not prove her immortality; for after having worn out many
bodies in a single life, and many more in successive births and
deaths, she may at last perish, or, as Socrates afterwards restates the
objection, the very act of birth may be the beginning of her death, and
her last body may survive her, just as the coat of an old weaver is left
behind him after he is dead, although a man is more lasting than his
coat. And he who would prove the immortality of the soul, must prove not
only that the soul outlives one or many bodies, but that she outlives
them all.

The audience, like the chorus in a play, for a moment interpret the
feelings of the actors; there is a temporary depression, and then the
enquiry is resumed. It is a melancholy reflection that arguments, like
men, are apt to be deceivers; and those who have been often deceived
become distrustful both of arguments and of friends. But this
unfortunate experience should not make us either haters of men or haters
of arguments. The want of health and truth is not in the argument, but
in ourselves. Socrates, who is about to die, is sensible of his own
weakness; he desires to be impartial, but he cannot help feeling that he
has too great an interest in the truth of the argument. And therefore he
would have his friends examine and refute him, if they think that he is
in error.

At his request Simmias and Cebes repeat their objections. They do not
go to the length of denying the pre-existence of ideas. Simmias is of
opinion that the soul is a harmony of the body. But the admission of the
pre-existence of ideas, and therefore of the soul, is at variance with
this. (Compare a parallel difficulty in Theaet.) For a harmony is
an effect, whereas the soul is not an effect, but a cause; a harmony
follows, but the soul leads; a harmony admits of degrees, and the soul
has no degrees. Again, upon the supposition that the soul is a harmony,
why is one soul better than another? Are they more or less harmonized,
or is there one harmony within another? But the soul does not admit of
degrees, and cannot therefore be more or less harmonized. Further, the
soul is often engaged in resisting the affections of the body, as Homer
describes Odysseus 'rebuking his heart.' Could he have written this
under the idea that the soul is a harmony of the body? Nay rather, are
we not contradicting Homer and ourselves in affirming anything of the
sort?

The goddess Harmonia, as Socrates playfully terms the argument of
Simmias, has been happily disposed of; and now an answer has to be given
to the Theban Cadmus. Socrates recapitulates the argument of Cebes,
which, as he remarks, involves the whole question of natural growth or
causation; about this he proposes to narrate his own mental experience.
When he was young he had puzzled himself with physics: he had enquired
into the growth and decay of animals, and the origin of thought, until
at last he began to doubt the self-evident fact that growth is the
result of eating and drinking; and so he arrived at the conclusion that
he was not meant for such enquiries. Nor was he less perplexed with
notions of comparison and number. At first he had imagined himself to
understand differences of greater and less, and to know that ten is two
more than eight, and the like. But now those very notions appeared to
him to contain a contradiction. For how can one be divided into two? Or
two be compounded into one? These are difficulties which Socrates cannot
answer. Of generation and destruction he knows nothing. But he has a
confused notion of another method in which matters of this sort are to
be investigated. (Compare Republic; Charm.)

Then he heard some one reading out of a book of Anaxagoras, that mind is
the cause of all things. And he said to himself: If mind is the cause
of all things, surely mind must dispose them all for the best. The new
teacher will show me this 'order of the best' in man and nature. How
great had been his hopes and how great his disappointment! For he found
that his new friend was anything but consistent in his use of mind as
a cause, and that he soon introduced winds, waters, and other eccentric
notions. (Compare Arist. Metaph.) It was as if a person had said that
Socrates is sitting here because he is made up of bones and muscles,
instead of telling the true reason--that he is here because the
Athenians have thought good to sentence him to death, and he has thought
good to await his sentence. Had his bones and muscles been left by him
to their own ideas of right, they would long ago have taken themselves
off. But surely there is a great confusion of the cause and condition
in all this. And this confusion also leads people into all sorts of
erroneous theories about the position and motions of the earth. None of
them know how much stronger than any Atlas is the power of the best. But
this 'best' is still undiscovered; and in enquiring after the cause, we
can only hope to attain the second best.

Now there is a danger in the contemplation of the nature of things, as
there is a danger in looking at the sun during an eclipse, unless the
precaution is taken of looking only at the image reflected in the water,
or in a glass. (Compare Laws; Republic.) 'I was afraid,' says Socrates,
'that I might injure the eye of the soul. I thought that I had better
return to the old and safe method of ideas. Though I do not mean to say
that he who contemplates existence through the medium of ideas sees
only through a glass darkly, any more than he who contemplates actual
effects.'

If the existence of ideas is granted to him, Socrates is of opinion that
he will then have no difficulty in proving the immortality of the soul.
He will only ask for a further admission:--that beauty is the cause of
the beautiful, greatness the cause of the great, smallness of the small,
and so on of other things. This is a safe and simple answer, which
escapes the contradictions of greater and less (greater by reason of
that which is smaller!), of addition and subtraction, and the other
difficulties of relation. These subtleties he is for leaving to wiser
heads than his own; he prefers to test ideas by the consistency of their
consequences, and, if asked to give an account of them, goes back to
some higher idea or hypothesis which appears to him to be the best,
until at last he arrives at a resting-place. (Republic; Phil.)

The doctrine of ideas, which has long ago received the assent of the
Socratic circle, is now affirmed by the Phliasian auditor to command
the assent of any man of sense. The narrative is continued; Socrates is
desirous of explaining how opposite ideas may appear to co-exist but do
not really co-exist in the same thing or person. For example, Simmias
may be said to have greatness and also smallness, because he is greater
than Socrates and less than Phaedo. And yet Simmias is not really great
and also small, but only when compared to Phaedo and Socrates. I use the
illustration, says Socrates, because I want to show you not only that
ideal opposites exclude one another, but also the opposites in us. I,
for example, having the attribute of smallness remain small, and cannot
become great: the smallness which is in me drives out greatness.

One of the company here remarked that this was inconsistent with the
old assertion that opposites generated opposites. But that, replies
Socrates, was affirmed, not of opposite ideas either in us or in
nature, but of opposition in the concrete--not of life and death, but
of individuals living and dying. When this objection has been removed,
Socrates proceeds: This doctrine of the mutual exclusion of opposites
is not only true of the opposites themselves, but of things which are
inseparable from them. For example, cold and heat are opposed; and fire,
which is inseparable from heat, cannot co-exist with cold, or snow,
which is inseparable from cold, with heat. Again, the number three
excludes the number four, because three is an odd number and four is
an even number, and the odd is opposed to the even. Thus we are able to
proceed a step beyond 'the safe and simple answer.' We may say, not
only that the odd excludes the even, but that the number three, which
participates in oddness, excludes the even. And in like manner, not only
does life exclude death, but the soul, of which life is the inseparable
attribute, also excludes death. And that of which life is the
inseparable attribute is by the force of the terms imperishable. If the
odd principle were imperishable, then the number three would not perish
but remove, on the approach of the even principle. But the immortal is
imperishable; and therefore the soul on the approach of death does not
perish but removes.

Thus all objections appear to be finally silenced. And now the
application has to be made: If the soul is immortal, 'what manner of
persons ought we to be?' having regard not only to time but to eternity.
For death is not the end of all, and the wicked is not released from his
evil by death; but every one carries with him into the world below that
which he is or has become, and that only.

For after death the soul is carried away to judgment, and when she has
received her punishment returns to earth in the course of ages. The wise
soul is conscious of her situation, and follows the attendant angel who
guides her through the windings of the world below; but the impure soul
wanders hither and thither without companion or guide, and is carried
at last to her own place, as the pure soul is also carried away to hers.
'In order that you may understand this, I must first describe to you the
nature and conformation of the earth.'

Now the whole earth is a globe placed in the centre of the heavens, and
is maintained there by the perfection of balance. That which we call the
earth is only one of many small hollows, wherein collect the mists and
waters and the thick lower air; but the true earth is above, and is in
a finer and subtler element. And if, like birds, we could fly to the
surface of the air, in the same manner that fishes come to the top of
the sea, then we should behold the true earth and the true heaven and
the true stars. Our earth is everywhere corrupted and corroded; and even
the land which is fairer than the sea, for that is a mere chaos or waste
of water and mud and sand, has nothing to show in comparison of the
other world. But the heavenly earth is of divers colours, sparkling with
jewels brighter than gold and whiter than any snow, having flowers and
fruits innumerable. And the inhabitants dwell some on the shore of the
sea of air, others in 'islets of the blest,' and they hold converse
with the gods, and behold the sun, moon and stars as they truly are, and
their other blessedness is of a piece with this.

The hollows on the surface of the globe vary in size and shape from that
which we inhabit: but all are connected by passages and perforations in
the interior of the earth. And there is one huge chasm or opening called
Tartarus, into which streams of fire and water and liquid mud are ever
flowing; of these small portions find their way to the surface and
form seas and rivers and volcanoes. There is a perpetual inhalation and
exhalation of the air rising and falling as the waters pass into the
depths of the earth and return again, in their course forming lakes
and rivers, but never descending below the centre of the earth; for on
either side the rivers flowing either way are stopped by a precipice.
These rivers are many and mighty, and there are four principal ones,
Oceanus, Acheron, Pyriphlegethon, and Cocytus. Oceanus is the river
which encircles the earth; Acheron takes an opposite direction, and
after flowing under the earth through desert places, at last reaches the
Acherusian lake,--this is the river at which the souls of the dead await
their return to earth. Pyriphlegethon is a stream of fire, which coils
round the earth and flows into the depths of Tartarus. The fourth river,
Cocytus, is that which is called by the poets the Stygian river, and
passes into and forms the lake Styx, from the waters of which it gains
new and strange powers. This river, too, falls into Tartarus.

The dead are first of all judged according to their deeds, and those who
are incurable are thrust into Tartarus, from which they never come out.
Those who have only committed venial sins are first purified of them,
and then rewarded for the good which they have done. Those who have
committed crimes, great indeed, but not unpardonable, are thrust
into Tartarus, but are cast forth at the end of a year by way of
Pyriphlegethon or Cocytus, and these carry them as far as the Acherusian
lake, where they call upon their victims to let them come out of the
rivers into the lake. And if they prevail, then they are let out and
their sufferings cease: if not, they are borne unceasingly into Tartarus
and back again, until they at last obtain mercy. The pure souls also
receive their reward, and have their abode in the upper earth, and a
select few in still fairer 'mansions.'

Socrates is not prepared to insist on the literal accuracy of this
description, but he is confident that something of the kind is true.
He who has sought after the pleasures of knowledge and rejected the
pleasures of the body, has reason to be of good hope at the approach of
death; whose voice is already speaking to him, and who will one day be
heard calling all men.

The hour has come at which he must drink the poison, and not much
remains to be done. How shall they bury him? That is a question which he
refuses to entertain, for they are burying, not him, but his dead body.
His friends had once been sureties that he would remain, and they shall
now be sureties that he has run away. Yet he would not die without the
customary ceremonies of washing and burial. Shall he make a libation of
the poison? In the spirit he will, but not in the letter. One request he
utters in the very act of death, which has been a puzzle to after ages.
With a sort of irony he remembers that a trifling religious duty is
still unfulfilled, just as above he desires before he departs to compose
a few verses in order to satisfy a scruple about a dream--unless,
indeed, we suppose him to mean, that he was now restored to health, and
made the customary offering to Asclepius in token of his recovery.

*****

1. The doctrine of the immortality of the soul has sunk deep into
the heart of the human race; and men are apt to rebel against any
examination of the nature or grounds of their belief. They do not like
to acknowledge that this, as well as the other 'eternal ideas; of
man, has a history in time, which may be traced in Greek poetry or
philosophy, and also in the Hebrew Scriptures. They convert feeling into
reasoning, and throw a network of dialectics over that which is really
a deeply-rooted instinct. In the same temper which Socrates reproves in
himself they are disposed to think that even fallacies will do no harm,
for they will die with them, and while they live they will gain by the
delusion. And when they consider the numberless bad arguments which have
been pressed into the service of theology, they say, like the companions
of Socrates, 'What argument can we ever trust again?' But there is a
better and higher spirit to be gathered from the Phaedo, as well as from
the other writings of Plato, which says that first principles should
be most constantly reviewed (Phaedo and Crat.), and that the highest
subjects demand of us the greatest accuracy (Republic); also that we
must not become misologists because arguments are apt to be deceivers.

2. In former ages there was a customary rather than a reasoned belief
in the immortality of the soul. It was based on the authority of the
Church, on the necessity of such a belief to morality and the order of
society, on the evidence of an historical fact, and also on analogies
and figures of speech which filled up the void or gave an expression
in words to a cherished instinct. The mass of mankind went on their
way busy with the affairs of this life, hardly stopping to think about
another. But in our own day the question has been reopened, and it is
doubtful whether the belief which in the first ages of Christianity
was the strongest motive of action can survive the conflict with a
scientific age in which the rules of evidence are stricter and the mind
has become more sensitive to criticism. It has faded into the distance
by a natural process as it was removed further and further from the
historical fact on which it has been supposed to rest. Arguments derived
from material things such as the seed and the ear of corn or transitions
in the life of animals from one state of being to another (the chrysalis
and the butterfly) are not 'in pari materia' with arguments from
the visible to the invisible, and are therefore felt to be no longer
applicable. The evidence to the historical fact seems to be weaker than
was once supposed: it is not consistent with itself, and is based upon
documents which are of unknown origin. The immortality of man must be
proved by other arguments than these if it is again to become a living
belief. We must ask ourselves afresh why we still maintain it, and seek
to discover a foundation for it in the nature of God and in the first
principles of morality.

3. At the outset of the discussion we may clear away a confusion. We
certainly do not mean by the immortality of the soul the immortality of
fame, which whether worth having or not can only be ascribed to a very
select class of the whole race of mankind, and even the interest in
these few is comparatively short-lived. To have been a benefactor to the
world, whether in a higher or a lower sphere of life and thought, is a
great thing: to have the reputation of being one, when men have passed
out of the sphere of earthly praise or blame, is hardly worthy of
consideration. The memory of a great man, so far from being immortal,
is really limited to his own generation:--so long as his friends or his
disciples are alive, so long as his books continue to be read, so long
as his political or military successes fill a page in the history of
his country. The praises which are bestowed upon him at his death hardly
last longer than the flowers which are strewed upon his coffin or the
'immortelles' which are laid upon his tomb. Literature makes the most
of its heroes, but the true man is well aware that far from enjoying an
immortality of fame, in a generation or two, or even in a much shorter
time, he will be forgotten and the world will get on without him.

4. Modern philosophy is perplexed at this whole question, which is
sometimes fairly given up and handed over to the realm of faith. The
perplexity should not be forgotten by us when we attempt to submit the
Phaedo of Plato to the requirements of logic. For what idea can we form
of the soul when separated from the body? Or how can the soul be united
with the body and still be independent? Is the soul related to the
body as the ideal to the real, or as the whole to the parts, or as the
subject to the object, or as the cause to the effect, or as the end to
the means? Shall we say with Aristotle, that the soul is the entelechy
or form of an organized living body? or with Plato, that she has a life
of her own? Is the Pythagorean image of the harmony, or that of the
monad, the truer expression? Is the soul related to the body as sight to
the eye, or as the boatman to his boat? (Arist. de Anim.) And in
another state of being is the soul to be conceived of as vanishing into
infinity, hardly possessing an existence which she can call her own,
as in the pantheistic system of Spinoza: or as an individual informing
another body and entering into new relations, but retaining her own
character? (Compare Gorgias.) Or is the opposition of soul and body a
mere illusion, and the true self neither soul nor body, but the union
of the two in the 'I' which is above them? And is death the assertion
of this individuality in the higher nature, and the falling away into
nothingness of the lower? Or are we vainly attempting to pass
the boundaries of human thought? The body and the soul seem to be
inseparable, not only in fact, but in our conceptions of them; and any
philosophy which too closely unites them, or too widely separates them,
either in this life or in another, disturbs the balance of human nature.
No thinker has perfectly adjusted them, or been entirely consistent with
himself in describing their relation to one another. Nor can we
wonder that Plato in the infancy of human thought should have confused
mythology and philosophy, or have mistaken verbal arguments for real
ones.

5. Again, believing in the immortality of the soul, we must still
ask the question of Socrates, 'What is that which we suppose to be
immortal?' Is it the personal and individual element in us, or the
spiritual and universal? Is it the principle of knowledge or of
goodness, or the union of the two? Is it the mere force of life which is
determined to be, or the consciousness of self which cannot be got rid
of, or the fire of genius which refuses to be extinguished? Or is there
a hidden being which is allied to the Author of all existence, who is
because he is perfect, and to whom our ideas of perfection give us a
title to belong? Whatever answer is given by us to these questions,
there still remains the necessity of allowing the permanence of evil, if
not for ever, at any rate for a time, in order that the wicked 'may not
have too good a bargain.' For the annihilation of evil at death, or the
eternal duration of it, seem to involve equal difficulties in the moral
government of the universe. Sometimes we are led by our feelings, rather
than by our reason, to think of the good and wise only as existing in
another life. Why should the mean, the weak, the idiot, the infant,
the herd of men who have never in any proper sense the use of reason,
reappear with blinking eyes in the light of another world? But our
second thought is that the hope of humanity is a common one, and that
all or none will be partakers of immortality. Reason does not allow us
to suppose that we have any greater claims than others, and experience
may often reveal to us unexpected flashes of the higher nature in
those whom we had despised. Why should the wicked suffer any more than
ourselves? had we been placed in their circumstances should we have been
any better than they? The worst of men are objects of pity rather than
of anger to the philanthropist; must they not be equally such to divine
benevolence? Even more than the good they have need of another life; not
that they may be punished, but that they may be educated. These are
a few of the reflections which arise in our minds when we attempt to
assign any form to our conceptions of a future state.

There are some other questions which are disturbing to us because we
have no answer to them. What is to become of the animals in a future
state? Have we not seen dogs more faithful and intelligent than men,
and men who are more stupid and brutal than any animals? Does their life
cease at death, or is there some 'better thing reserved' also for
them? They may be said to have a shadow or imitation of morality, and
imperfect moral claims upon the benevolence of man and upon the justice
of God. We cannot think of the least or lowest of them, the insect, the
bird, the inhabitants of the sea or the desert, as having any place in
a future world, and if not all, why should those who are specially
attached to man be deemed worthy of any exceptional privilege? When we
reason about such a subject, almost at once we degenerate into nonsense.
It is a passing thought which has no real hold on the mind. We may argue
for the existence of animals in a future state from the attributes of
God, or from texts of Scripture ('Are not two sparrows sold for one
farthing?' etc.), but the truth is that we are only filling up the void
of another world with our own fancies. Again, we often talk about
the origin of evil, that great bugbear of theologians, by which they
frighten us into believing any superstition. What answer can be made
to the old commonplace, 'Is not God the author of evil, if he knowingly
permitted, but could have prevented it?' Even if we assume that the
inequalities of this life are rectified by some transposition of human
beings in another, still the existence of the very least evil if it
could have been avoided, seems to be at variance with the love and
justice of God. And so we arrive at the conclusion that we are carrying
logic too far, and that the attempt to frame the world according to a
rule of divine perfection is opposed to experience and had better be
given up. The case of the animals is our own. We must admit that the
Divine Being, although perfect himself, has placed us in a state of life
in which we may work together with him for good, but we are very far
from having attained to it.

6. Again, ideas must be given through something; and we are always prone
to argue about the soul from analogies of outward things which may serve
to embody our thoughts, but are also partly delusive. For we cannot
reason from the natural to the spiritual, or from the outward to the
inward. The progress of physiological science, without bringing us
nearer to the great secret, has tended to remove some erroneous notions
respecting the relations of body and mind, and in this we have the
advantage of the ancients. But no one imagines that any seed of
immortality is to be discerned in our mortal frames. Most people have
been content to rest their belief in another life on the agreement of
the more enlightened part of mankind, and on the inseparable connection
of such a doctrine with the existence of a God--also in a less degree
on the impossibility of doubting about the continued existence of those
whom we love and reverence in this world. And after all has been
said, the figure, the analogy, the argument, are felt to be only
approximations in different forms to an expression of the common
sentiment of the human heart. That we shall live again is far more
certain than that we shall take any particular form of life.

7. When we speak of the immortality of the soul, we must ask further
what we mean by the word immortality. For of the duration of a living
being in countless ages we can form no conception; far less than a three
years' old child of the whole of life. The naked eye might as well try
to see the furthest star in the infinity of heaven. Whether time and
space really exist when we take away the limits of them may be doubted;
at any rate the thought of them when unlimited us so overwhelming to us
as to lose all distinctness. Philosophers have spoken of them as forms
of the human mind, but what is the mind without them? As then infinite
time, or an existence out of time, which are the only possible
explanations of eternal duration, are equally inconceivable to us, let
us substitute for them a hundred or a thousand years after death, and
ask not what will be our employment in eternity, but what will happen to
us in that definite portion of time; or what is now happening to those
who passed out of life a hundred or a thousand years ago. Do we imagine
that the wicked are suffering torments, or that the good are singing the
praises of God, during a period longer than that of a whole life, or
of ten lives of men? Is the suffering physical or mental? And does the
worship of God consist only of praise, or of many forms of service? Who
are the wicked, and who are the good, whom we venture to divide by a
hard and fast line; and in which of the two classes should we place
ourselves and our friends? May we not suspect that we are making
differences of kind, because we are unable to imagine differences
of degree?--putting the whole human race into heaven or hell for the
greater convenience of logical division? Are we not at the same time
describing them both in superlatives, only that we may satisfy the
demands of rhetoric? What is that pain which does not become deadened
after a thousand years? or what is the nature of that pleasure or
happiness which never wearies by monotony? Earthly pleasures and pains
are short in proportion as they are keen; of any others which are both
intense and lasting we have no experience, and can form no idea.
The words or figures of speech which we use are not consistent with
themselves. For are we not imagining Heaven under the similitude of
a church, and Hell as a prison, or perhaps a madhouse or chamber of
horrors? And yet to beings constituted as we are, the monotony of
singing psalms would be as great an infliction as the pains of hell,
and might be even pleasantly interrupted by them. Where are the actions
worthy of rewards greater than those which are conferred on the greatest
benefactors of mankind? And where are the crimes which according to
Plato's merciful reckoning,--more merciful, at any rate, than the
eternal damnation of so-called Christian teachers,--for every ten years
in this life deserve a hundred of punishment in the life to come?
We should be ready to die of pity if we could see the least of the
sufferings which the writers of Infernos and Purgatorios have attributed
to the damned. Yet these joys and terrors seem hardly to exercise an
appreciable influence over the lives of men. The wicked man when old,
is not, as Plato supposes (Republic), more agitated by the terrors of
another world when he is nearer to them, nor the good in an ecstasy at
the joys of which he is soon to be the partaker. Age numbs the sense of
both worlds; and the habit of life is strongest in death. Even the dying
mother is dreaming of her lost children as they were forty or fifty
years before, 'pattering over the boards,' not of reunion with them
in another state of being. Most persons when the last hour comes are
resigned to the order of nature and the will of God. They are not
thinking of Dante's Inferno or Paradiso, or of the Pilgrim's Progress.
Heaven and hell are not realities to them, but words or ideas; the
outward symbols of some great mystery, they hardly know what. Many
noble poems and pictures have been suggested by the traditional
representations of them, which have been fixed in forms of art and can
no longer be altered. Many sermons have been filled with descriptions
of celestial or infernal mansions. But hardly even in childhood did the
thought of heaven and hell supply the motives of our actions, or at any
time seriously affect the substance of our belief.

8. Another life must be described, if at all, in forms of thought
and not of sense. To draw pictures of heaven and hell, whether in the
language of Scripture or any other, adds nothing to our real knowledge,
but may perhaps disguise our ignorance. The truest conception which
we can form of a future life is a state of progress or education--a
progress from evil to good, from ignorance to knowledge. To this we are
led by the analogy of the present life, in which we see different races
and nations of men, and different men and women of the same nation,
in various states or stages of cultivation; some more and some less
developed, and all of them capable of improvement under favourable
circumstances. There are punishments too of children when they are
growing up inflicted by their parents, of elder offenders which are
imposed by the law of the land, of all men at all times of life,
which are attached by the laws of nature to the performance of certain
actions. All these punishments are really educational; that is to say,
they are not intended to retaliate on the offender, but to teach him
a lesson. Also there is an element of chance in them, which is another
name for our ignorance of the laws of nature. There is evil too
inseparable from good (compare Lysis); not always punished here, as good
is not always rewarded. It is capable of being indefinitely diminished;
and as knowledge increases, the element of chance may more and more
disappear.

For we do not argue merely from the analogy of the present state of this
world to another, but from the analogy of a probable future to which we
are tending. The greatest changes of which we have had experience as yet
are due to our increasing knowledge of history and of nature. They
have been produced by a few minds appearing in three or four favoured
nations, in a comparatively short period of time. May we be allowed to
imagine the minds of men everywhere working together during many ages
for the completion of our knowledge? May not the science of physiology
transform the world? Again, the majority of mankind have really
experienced some moral improvement; almost every one feels that he has
tendencies to good, and is capable of becoming better. And these germs
of good are often found to be developed by new circumstances, like
stunted trees when transplanted to a better soil. The differences
between the savage and the civilized man, or between the civilized
man in old and new countries, may be indefinitely increased. The first
difference is the effect of a few thousand, the second of a few hundred
years. We congratulate ourselves that slavery has become industry;
that law and constitutional government have superseded despotism and
violence; that an ethical religion has taken the place of Fetichism.
There may yet come a time when the many may be as well off as the few;
when no one will be weighed down by excessive toil; when the necessity
of providing for the body will not interfere with mental improvement;
when the physical frame may be strengthened and developed; and the
religion of all men may become a reasonable service.

Nothing therefore, either in the present state of man or in the
tendencies of the future, as far as we can entertain conjecture of them,
would lead us to suppose that God governs us vindictively in this
world, and therefore we have no reason to infer that he will govern us
vindictively in another. The true argument from analogy is not, 'This
life is a mixed state of justice and injustice, of great waste, of
sudden casualties, of disproportionate punishments, and therefore the
like inconsistencies, irregularities, injustices are to be expected
in another;' but 'This life is subject to law, and is in a state of
progress, and therefore law and progress may be believed to be the
governing principles of another.' All the analogies of this world would
be against unmeaning punishments inflicted a hundred or a thousand years
after an offence had been committed. Suffering there might be as a
part of education, but not hopeless or protracted; as there might be
a retrogression of individuals or of bodies of men, yet not such as to
interfere with a plan for the improvement of the whole (compare Laws.)

9. But some one will say: That we cannot reason from the seen to the
unseen, and that we are creating another world after the image of this,
just as men in former ages have created gods in their own likeness. And
we, like the companions of Socrates, may feel discouraged at hearing
our favourite 'argument from analogy' thus summarily disposed of. Like
himself, too, we may adduce other arguments in which he seems to have
anticipated us, though he expresses them in different language. For we
feel that the soul partakes of the ideal and invisible; and can never
fall into the error of confusing the external circumstances of man with
his higher self; or his origin with his nature. It is as repugnant to
us as it was to him to imagine that our moral ideas are to be attributed
only to cerebral forces. The value of a human soul, like the value of a
man's life to himself, is inestimable, and cannot be reckoned in earthly
or material things. The human being alone has the consciousness of truth
and justice and love, which is the consciousness of God. And the soul
becoming more conscious of these, becomes more conscious of her own
immortality.

10. The last ground of our belief in immortality, and the strongest, is
the perfection of the divine nature. The mere fact of the existence of
God does not tend to show the continued existence of man. An evil God
or an indifferent God might have had the power, but not the will, to
preserve us. He might have regarded us as fitted to minister to his
service by a succession of existences,--like the animals, without
attributing to each soul an incomparable value. But if he is perfect,
he must will that all rational beings should partake of that perfection
which he himself is. In the words of the Timaeus, he is good, and
therefore he desires that all other things should be as like himself as
possible. And the manner in which he accomplishes this is by permitting
evil, or rather degrees of good, which are otherwise called evil.
For all progress is good relatively to the past, and yet may be
comparatively evil when regarded in the light of the future. Good and
evil are relative terms, and degrees of evil are merely the negative
aspect of degrees of good. Of the absolute goodness of any finite nature
we can form no conception; we are all of us in process of transition
from one degree of good or evil to another. The difficulties which
are urged about the origin or existence of evil are mere dialectical
puzzles, standing in the same relation to Christian philosophy as the
puzzles of the Cynics and Megarians to the philosophy of Plato. They
arise out of the tendency of the human mind to regard good and evil both
as relative and absolute; just as the riddles about motion are to be
explained by the double conception of space or matter, which the human
mind has the power of regarding either as continuous or discrete.

In speaking of divine perfection, we mean to say that God is just and
true and loving, the author of order and not of disorder, of good and
not of evil. Or rather, that he is justice, that he is truth, that he
is love, that he is order, that he is the very progress of which we were
speaking; and that wherever these qualities are present, whether in the
human soul or in the order of nature, there is God. We might still see
him everywhere, if we had not been mistakenly seeking for him apart from
us, instead of in us; away from the laws of nature, instead of in
them. And we become united to him not by mystical absorption, but by
partaking, whether consciously or unconsciously, of that truth and
justice and love which he himself is.

Thus the belief in the immortality of the soul rests at last on the
belief in God. If there is a good and wise God, then there is a progress
of mankind towards perfection; and if there is no progress of men
towards perfection, then there is no good and wise God. We cannot
suppose that the moral government of God of which we see the beginnings
in the world and in ourselves will cease when we pass out of life.

11. Considering the 'feebleness of the human faculties and the
uncertainty of the subject,' we are inclined to believe that the fewer
our words the better. At the approach of death there is not much said;
good men are too honest to go out of the world professing more than they
know. There is perhaps no important subject about which, at any time,
even religious people speak so little to one another. In the fulness
of life the thought of death is mostly awakened by the sight or
recollection of the death of others rather than by the prospect of our
own. We must also acknowledge that there are degrees of the belief in
immortality, and many forms in which it presents itself to the mind.
Some persons will say no more than that they trust in God, and that they
leave all to Him. It is a great part of true religion not to pretend
to know more than we do. Others when they quit this world are comforted
with the hope 'That they will see and know their friends in heaven.' But
it is better to leave them in the hands of God and to be assured that
'no evil shall touch them.' There are others again to whom the belief in
a divine personality has ceased to have any longer a meaning; yet they
are satisfied that the end of all is not here, but that something still
remains to us, 'and some better thing for the good than for the evil.'
They are persuaded, in spite of their theological nihilism, that the
ideas of justice and truth and holiness and love are realities. They
cherish an enthusiastic devotion to the first principles of morality.
Through these they see, or seem to see, darkly, and in a figure, that
the soul is immortal.

But besides differences of theological opinion which must ever prevail
about things unseen, the hope of immortality is weaker or stronger in
men at one time of life than at another; it even varies from day to day.
It comes and goes; the mind, like the sky, is apt to be overclouded.
Other generations of men may have sometimes lived under an 'eclipse of
faith,' to us the total disappearance of it might be compared to the
'sun falling from heaven.' And we may sometimes have to begin again and
acquire the belief for ourselves; or to win it back again when it is
lost. It is really weakest in the hour of death. For Nature, like a kind
mother or nurse, lays us to sleep without frightening us; physicians,
who are the witnesses of such scenes, say that under ordinary
circumstances there is no fear of the future. Often, as Plato tells
us, death is accompanied 'with pleasure.' (Tim.) When the end is still
uncertain, the cry of many a one has been, 'Pray, that I may be taken.'
The last thoughts even of the best men depend chiefly on the accidents
of their bodily state. Pain soon overpowers the desire of life; old age,
like the child, is laid to sleep almost in a moment. The long experience
of life will often destroy the interest which mankind have in it. So
various are the feelings with which different persons draw near to
death; and still more various the forms in which imagination clothes it.
For this alternation of feeling compare the Old Testament,--Psalm vi.;
Isaiah; Eccles.

12. When we think of God and of man in his relation to God; of the
imperfection of our present state and yet of the progress which is
observable in the history of the world and of the human mind; of the
depth and power of our moral ideas which seem to partake of the very
nature of God Himself; when we consider the contrast between the
physical laws to which we are subject and the higher law which raises us
above them and is yet a part of them; when we reflect on our capacity of
becoming the 'spectators of all time and all existence,' and of framing
in our own minds the ideal of a perfect Being; when we see how the
human mind in all the higher religions of the world, including Buddhism,
notwithstanding some aberrations, has tended towards such a belief--we
have reason to think that our destiny is different from that of animals;
and though we cannot altogether shut out the childish fear that the soul
upon leaving the body may 'vanish into thin air,' we have still, so far
as the nature of the subject admits, a hope of immortality with which we
comfort ourselves on sufficient grounds. The denial of the belief takes
the heart out of human life; it lowers men to the level of the material.
As Goethe also says, 'He is dead even in this world who has no belief in
another.'

13. It is well also that we should sometimes think of the forms of
thought under which the idea of immortality is most naturally presented
to us. It is clear that to our minds the risen soul can no longer be
described, as in a picture, by the symbol of a creature half-bird,
half-human, nor in any other form of sense. The multitude of angels, as
in Milton, singing the Almighty's praises, are a noble image, and may
furnish a theme for the poet or the painter, but they are no longer an
adequate expression of the kingdom of God which is within us. Neither is
there any mansion, in this world or another, in which the departed can
be imagined to dwell and carry on their occupations. When this earthly
tabernacle is dissolved, no other habitation or building can take them
in: it is in the language of ideas only that we speak of them.

First of all there is the thought of rest and freedom from pain; they
have gone home, as the common saying is, and the cares of this world
touch them no more. Secondly, we may imagine them as they were at
their best and brightest, humbly fulfilling their daily round of
duties--selfless, childlike, unaffected by the world; when the eye was
single and the whole body seemed to be full of light; when the mind was
clear and saw into the purposes of God. Thirdly, we may think of them
as possessed by a great love of God and man, working out His will at a
further stage in the heavenly pilgrimage. And yet we acknowledge that
these are the things which eye hath not seen nor ear heard and therefore
it hath not entered into the heart of man in any sensible manner to
conceive them. Fourthly, there may have been some moments in our own
lives when we have risen above ourselves, or been conscious of our truer
selves, in which the will of God has superseded our wills, and we have
entered into communion with Him, and been partakers for a brief season
of the Divine truth and love, in which like Christ we have been inspired
to utter the prayer, 'I in them, and thou in me, that we may be all made
perfect in one.' These precious moments, if we have ever known them, are
the nearest approach which we can make to the idea of immortality.

14. Returning now to the earlier stage of human thought which is
represented by the writings of Plato, we find that many of the
same questions have already arisen: there is the same tendency to
materialism; the same inconsistency in the application of the idea of
mind; the same doubt whether the soul is to be regarded as a cause or as
an effect; the same falling back on moral convictions. In the Phaedo the
soul is conscious of her divine nature, and the separation from the body
which has been commenced in this life is perfected in another. Beginning
in mystery, Socrates, in the intermediate part of the Dialogue, attempts
to bring the doctrine of a future life into connection with his theory
of knowledge. In proportion as he succeeds in this, the individual seems
to disappear in a more general notion of the soul; the contemplation of
ideas 'under the form of eternity' takes the place of past and future
states of existence. His language may be compared to that of some modern
philosophers, who speak of eternity, not in the sense of perpetual
duration of time, but as an ever-present quality of the soul. Yet at
the conclusion of the Dialogue, having 'arrived at the end of the
intellectual world' (Republic), he replaces the veil of mythology,
and describes the soul and her attendant genius in the language of the
mysteries or of a disciple of Zoroaster. Nor can we fairly demand of
Plato a consistency which is wanting among ourselves, who acknowledge
that another world is beyond the range of human thought, and yet are
always seeking to represent the mansions of heaven or hell in
the colours of the painter, or in the descriptions of the poet or
rhetorician.

15. The doctrine of the immortality of the soul was not new to the
Greeks in the age of Socrates, but, like the unity of God, had a
foundation in the popular belief. The old Homeric notion of a gibbering
ghost flitting away to Hades; or of a few illustrious heroes enjoying
the isles of the blest; or of an existence divided between the two; or
the Hesiodic, of righteous spirits, who become guardian angels,--had
given place in the mysteries and the Orphic poets to representations,
partly fanciful, of a future state of rewards and punishments. (Laws.)
The reticence of the Greeks on public occasions and in some part of
their literature respecting this 'underground' religion, is not to be
taken as a measure of the diffusion of such beliefs. If Pericles in the
funeral oration is silent on the consolations of immortality, the
poet Pindar and the tragedians on the other hand constantly assume the
continued existence of the dead in an upper or under world. Darius
and Laius are still alive; Antigone will be dear to her brethren after
death; the way to the palace of Cronos is found by those who 'have
thrice departed from evil.' The tragedy of the Greeks is not 'rounded'
by this life, but is deeply set in decrees of fate and mysterious
workings of powers beneath the earth. In the caricature of Aristophanes
there is also a witness to the common sentiment. The Ionian and
Pythagorean philosophies arose, and some new elements were added to the
popular belief. The individual must find an expression as well as the
world. Either the soul was supposed to exist in the form of a magnet, or
of a particle of fire, or of light, or air, or water; or of a number or
of a harmony of number; or to be or have, like the stars, a principle
of motion (Arist. de Anim.). At length Anaxagoras, hardly distinguishing
between life and mind, or between mind human and divine, attained
the pure abstraction; and this, like the other abstractions of Greek
philosophy, sank deep into the human intelligence. The opposition of
the intelligible and the sensible, and of God to the world, supplied an
analogy which assisted in the separation of soul and body. If ideas were
separable from phenomena, mind was also separable from matter; if the
ideas were eternal, the mind that conceived them was eternal too. As
the unity of God was more distinctly acknowledged, the conception of the
human soul became more developed. The succession, or alternation of
life and death, had occurred to Heracleitus. The Eleatic Parmenides had
stumbled upon the modern thesis, that 'thought and being are the same.'
The Eastern belief in transmigration defined the sense of individuality;
and some, like Empedocles, fancied that the blood which they had shed
in another state of being was crying against them, and that for thirty
thousand years they were to be 'fugitives and vagabonds upon the earth.'
The desire of recognizing a lost mother or love or friend in the world
below (Phaedo) was a natural feeling which, in that age as well as in
every other, has given distinctness to the hope of immortality. Nor were
ethical considerations wanting, partly derived from the necessity of
punishing the greater sort of criminals, whom no avenging power of this
world could reach. The voice of conscience, too, was heard reminding
the good man that he was not altogether innocent. (Republic.) To these
indistinct longings and fears an expression was given in the mysteries
and Orphic poets: a 'heap of books' (Republic), passing under the names
of Musaeus and Orpheus in Plato's time, were filled with notions of an
under-world.

16. Yet after all the belief in the individuality of the soul after
death had but a feeble hold on the Greek mind. Like the personality of
God, the personality of man in a future state was not inseparably bound
up with the reality of his existence. For the distinction between the
personal and impersonal, and also between the divine and human, was far
less marked to the Greek than to ourselves. And as Plato readily passes
from the notion of the good to that of God, he also passes almost
imperceptibly to himself and his reader from the future life of the
individual soul to the eternal being of the absolute soul. There has
been a clearer statement and a clearer denial of the belief in modern
times than is found in early Greek philosophy, and hence the comparative
silence on the whole subject which is often remarked in ancient writers,
and particularly in Aristotle. For Plato and Aristotle are not further
removed in their teaching about the immortality of the soul than they
are in their theory of knowledge.

17. Living in an age when logic was beginning to mould human thought,
Plato naturally cast his belief in immortality into a logical form. And
when we consider how much the doctrine of ideas was also one of words,
it is not surprising that he should have fallen into verbal fallacies:
early logic is always mistaking the truth of the form for the truth of
the matter. It is easy to see that the alternation of opposites is
not the same as the generation of them out of each other; and that the
generation of them out of each other, which is the first argument in
the Phaedo, is at variance with their mutual exclusion of each other,
whether in themselves or in us, which is the last. For even if we admit
the distinction which he draws between the opposites and the things
which have the opposites, still individuals fall under the latter class;
and we have to pass out of the region of human hopes and fears to a
conception of an abstract soul which is the impersonation of the ideas.
Such a conception, which in Plato himself is but half expressed, is
unmeaning to us, and relative only to a particular stage in the history
of thought. The doctrine of reminiscence is also a fragment of a former
world, which has no place in the philosophy of modern times. But Plato
had the wonders of psychology just opening to him, and he had not the
explanation of them which is supplied by the analysis of language and
the history of the human mind. The question, 'Whence come our abstract
ideas?' he could only answer by an imaginary hypothesis. Nor is it
difficult to see that his crowning argument is purely verbal, and is
but the expression of an instinctive confidence put into a logical
form:--'The soul is immortal because it contains a principle of
imperishableness.' Nor does he himself seem at all to be aware that
nothing is added to human knowledge by his 'safe and simple answer,'
that beauty is the cause of the beautiful; and that he is merely
reasserting the Eleatic being 'divided by the Pythagorean numbers,'
against the Heracleitean doctrine of perpetual generation. The answer to
the 'very serious question' of generation and destruction is really
the denial of them. For this he would substitute, as in the Republic, a
system of ideas, tested, not by experience, but by their consequences,
and not explained by actual causes, but by a higher, that is, a more
general notion. Consistency with themselves is the only test which is to
be applied to them. (Republic, and Phaedo.)

18. To deal fairly with such arguments, they should be translated as
far as possible into their modern equivalents. 'If the ideas of men are
eternal, their souls are eternal, and if not the ideas, then not the
souls.' Such an argument stands nearly in the same relation to Plato and
his age, as the argument from the existence of God to immortality among
ourselves. 'If God exists, then the soul exists after death; and if
there is no God, there is no existence of the soul after death.' For
the ideas are to his mind the reality, the truth, the principle of
permanence, as well as of intelligence and order in the world. When
Simmias and Cebes say that they are more strongly persuaded of the
existence of ideas than they are of the immortality of the soul, they
represent fairly enough the order of thought in Greek philosophy. And we
might say in the same way that we are more certain of the existence
of God than we are of the immortality of the soul, and are led by the
belief in the one to a belief in the other. The parallel, as Socrates
would say, is not perfect, but agrees in as far as the mind in either
case is regarded as dependent on something above and beyond herself. The
analogy may even be pressed a step further: 'We are more certain of our
ideas of truth and right than we are of the existence of God, and
are led on in the order of thought from one to the other.' Or more
correctly: 'The existence of right and truth is the existence of God,
and can never for a moment be separated from Him.'

19. The main argument of the Phaedo is derived from the existence of
eternal ideas of which the soul is a partaker; the other argument of the
alternation of opposites is replaced by this. And there have not been
wanting philosophers of the idealist school who have imagined that the
doctrine of the immortality of the soul is a theory of knowledge, and
that in what has preceded Plato is accommodating himself to the popular
belief. Such a view can only be elicited from the Phaedo by what may
be termed the transcendental method of interpretation, and is obviously
inconsistent with the Gorgias and the Republic. Those who maintain
it are immediately compelled to renounce the shadow which they have
grasped, as a play of words only. But the truth is, that Plato in his
argument for the immortality of the soul has collected many elements of
proof or persuasion, ethical and mythological as well as dialectical,
which are not easily to be reconciled with one another; and he is as
much in earnest about his doctrine of retribution, which is repeated
in all his more ethical writings, as about his theory of knowledge.
And while we may fairly translate the dialectical into the language of
Hegel, and the religious and mythological into the language of Dante or
Bunyan, the ethical speaks to us still in the same voice, and appeals to
a common feeling.

20. Two arguments of this ethical character occur in the Phaedo. The
first may be described as the aspiration of the soul after another state
of being. Like the Oriental or Christian mystic, the philosopher is
seeking to withdraw from impurities of sense, to leave the world and the
things of the world, and to find his higher self. Plato recognizes in
these aspirations the foretaste of immortality; as Butler and Addison in
modern times have argued, the one from the moral tendencies of mankind,
the other from the progress of the soul towards perfection. In using
this argument Plato has certainly confused the soul which has left the
body, with the soul of the good and wise. (Compare Republic.) Such a
confusion was natural, and arose partly out of the antithesis of soul
and body. The soul in her own essence, and the soul 'clothed upon' with
virtues and graces, were easily interchanged with one another, because
on a subject which passes expression the distinctions of language can
hardly be maintained.

21. The ethical proof of the immortality of the soul is derived from the
necessity of retribution. The wicked would be too well off if their
evil deeds came to an end. It is not to be supposed that an Ardiaeus,
an Archelaus, an Ismenias could ever have suffered the penalty of
their crimes in this world. The manner in which this retribution is
accomplished Plato represents under the figures of mythology. Doubtless
he felt that it was easier to improve than to invent, and that in
religion especially the traditional form was required in order to give
verisimilitude to the myth. The myth too is far more probable to that
age than to ours, and may fairly be regarded as 'one guess among
many' about the nature of the earth, which he cleverly supports by the
indications of geology. Not that he insists on the absolute truth of
his own particular notions: 'no man of sense will be confident in such
matters; but he will be confident that something of the kind is true.'
As in other passages (Gorg., Tim., compare Crito), he wins belief for
his fictions by the moderation of his statements; he does not, like
Dante or Swedenborg, allow himself to be deceived by his own creations.

The Dialogue must be read in the light of the situation. And first of
all we are struck by the calmness of the scene. Like the spectators
at the time, we cannot pity Socrates; his mien and his language are
so noble and fearless. He is the same that he ever was, but milder and
gentler, and he has in no degree lost his interest in dialectics;
he will not forego the delight of an argument in compliance with the
jailer's intimation that he should not heat himself with talking. At
such a time he naturally expresses the hope of his life, that he has
been a true mystic and not a mere retainer or wand-bearer: and he refers
to passages of his personal history. To his old enemies the Comic
poets, and to the proceedings on the trial, he alludes playfully; but he
vividly remembers the disappointment which he felt in reading the books
of Anaxagoras. The return of Xanthippe and his children indicates that
the philosopher is not 'made of oak or rock.' Some other traits of his
character may be noted; for example, the courteous manner in which
he inclines his head to the last objector, or the ironical touch, 'Me
already, as the tragic poet would say, the voice of fate calls;' or
the depreciation of the arguments with which 'he comforted himself and
them;' or his fear of 'misology;' or his references to Homer; or the
playful smile with which he 'talks like a book' about greater and less;
or the allusion to the possibility of finding another teacher among
barbarous races (compare Polit.); or the mysterious reference to another
science (mathematics?) of generation and destruction for which he is
vainly feeling. There is no change in him; only now he is invested with
a sort of sacred character, as the prophet or priest of Apollo the God
of the festival, in whose honour he first of all composes a hymn,
and then like the swan pours forth his dying lay. Perhaps the extreme
elevation of Socrates above his own situation, and the ordinary
interests of life (compare his jeu d'esprit about his burial, in which
for a moment he puts on the 'Silenus mask'), create in the mind of the
reader an impression stronger than could be derived from arguments that
such a one has in him 'a principle which does not admit of death.'

The other persons of the Dialogue may be considered under two heads: (1)
private friends; (2) the respondents in the argument.

First there is Crito, who has been already introduced to us in the
Euthydemus and the Crito; he is the equal in years of Socrates, and
stands in quite a different relation to him from his younger disciples.
He is a man of the world who is rich and prosperous (compare the jest
in the Euthydemus), the best friend of Socrates, who wants to know his
commands, in whose presence he talks to his family, and who performs
the last duty of closing his eyes. It is observable too that, as in the
Euthydemus, Crito shows no aptitude for philosophical discussions. Nor
among the friends of Socrates must the jailer be forgotten, who seems
to have been introduced by Plato in order to show the impression made
by the extraordinary man on the common. The gentle nature of the man
is indicated by his weeping at the announcement of his errand and then
turning away, and also by the words of Socrates to his disciples: 'How
charming the man is! since I have been in prison he has been always
coming to me, and is as good as could be to me.' We are reminded
too that he has retained this gentle nature amid scenes of death and
violence by the contrasts which he draws between the behaviour of
Socrates and of others when about to die.

Another person who takes no part in the philosophical discussion is the
excitable Apollodorus, the same who, in the Symposium, of which he is
the narrator, is called 'the madman,' and who testifies his grief by the
most violent emotions. Phaedo is also present, the 'beloved disciple'
as he may be termed, who is described, if not 'leaning on his bosom,'
as seated next to Socrates, who is playing with his hair. He too, like
Apollodorus, takes no part in the discussion, but he loves above all
things to hear and speak of Socrates after his death. The calmness
of his behaviour, veiling his face when he can no longer restrain
his tears, contrasts with the passionate outcries of the other. At a
particular point the argument is described as falling before the attack
of Simmias. A sort of despair is introduced in the minds of the company.
The effect of this is heightened by the description of Phaedo, who has
been the eye-witness of the scene, and by the sympathy of his Phliasian
auditors who are beginning to think 'that they too can never trust an
argument again.' And the intense interest of the company is communicated
not only to the first auditors, but to us who in a distant country read
the narrative of their emotions after more than two thousand years have
passed away.

The two principal interlocutors are Simmias and Cebes, the disciples of
Philolaus the Pythagorean philosopher of Thebes. Simmias is described
in the Phaedrus as fonder of an argument than any man living; and
Cebes, although finally persuaded by Socrates, is said to be the most
incredulous of human beings. It is Cebes who at the commencement of
the Dialogue asks why 'suicide is held to be unlawful,' and who
first supplies the doctrine of recollection in confirmation of the
pre-existence of the soul. It is Cebes who urges that the pre-existence
does not necessarily involve the future existence of the soul, as is
shown by the illustration of the weaver and his coat. Simmias, on the
other hand, raises the question about harmony and the lyre, which is
naturally put into the mouth of a Pythagorean disciple. It is Simmias,
too, who first remarks on the uncertainty of human knowledge, and
only at last concedes to the argument such a qualified approval as is
consistent with the feebleness of the human faculties. Cebes is the
deeper and more consecutive thinker, Simmias more superficial and
rhetorical; they are distinguished in much the same manner as Adeimantus
and Glaucon in the Republic.

Other persons, Menexenus, Ctesippus, Lysis, are old friends; Evenus
has been already satirized in the Apology; Aeschines and Epigenes
were present at the trial; Euclid and Terpsion will reappear in the
Introduction to the Theaetetus, Hermogenes has already appeared in
the Cratylus. No inference can fairly be drawn from the absence of
Aristippus, nor from the omission of Xenophon, who at the time of
Socrates' death was in Asia. The mention of Plato's own absence seems
like an expression of sorrow, and may, perhaps, be an indication that
the report of the conversation is not to be taken literally.

The place of the Dialogue in the series is doubtful. The doctrine of
ideas is certainly carried beyond the Socratic point of view; in no
other of the writings of Plato is the theory of them so completely
developed. Whether the belief in immortality can be attributed to
Socrates or not is uncertain; the silence of the Memorabilia, and of the
earlier Dialogues of Plato, is an argument to the contrary. Yet in the
Cyropaedia Xenophon has put language into the mouth of the dying Cyrus
which recalls the Phaedo, and may have been derived from the teaching of
Socrates. It may be fairly urged that the greatest religious interest of
mankind could not have been wholly ignored by one who passed his life in
fulfilling the commands of an oracle, and who recognized a Divine plan
in man and nature. (Xen. Mem.) And the language of the Apology and of
the Crito confirms this view.

The Phaedo is not one of the Socratic Dialogues of Plato; nor, on the
other hand, can it be assigned to that later stage of the Platonic
writings at which the doctrine of ideas appears to be forgotten. It
belongs rather to the intermediate period of the Platonic philosophy,
which roughly corresponds to the Phaedrus, Gorgias, Republic,
Theaetetus. Without pretending to determine the real time of their
composition, the Symposium, Meno, Euthyphro, Apology, Phaedo may be
conveniently read by us in this order as illustrative of the life of
Socrates. Another chain may be formed of the Meno, Phaedrus, Phaedo,
in which the immortality of the soul is connected with the doctrine of
ideas. In the Meno the theory of ideas is based on the ancient belief in
transmigration, which reappears again in the Phaedrus as well as in the
Republic and Timaeus, and in all of them is connected with a doctrine of
retribution. In the Phaedrus the immortality of the soul is supposed to
rest on the conception of the soul as a principle of motion, whereas in
the Republic the argument turns on the natural continuance of the soul,
which, if not destroyed by her own proper evil, can hardly be destroyed
by any other. The soul of man in the Timaeus is derived from the Supreme
Creator, and either returns after death to her kindred star, or descends
into the lower life of an animal. The Apology expresses the same view
as the Phaedo, but with less confidence; there the probability of death
being a long sleep is not excluded. The Theaetetus also describes, in a
digression, the desire of the soul to fly away and be with God--'and to
fly to him is to be like him.' The Symposium may be observed to
resemble as well as to differ from the Phaedo. While the first notion of
immortality is only in the way of natural procreation or of posthumous
fame and glory, the higher revelation of beauty, like the good in the
Republic, is the vision of the eternal idea. So deeply rooted in
Plato's mind is the belief in immortality; so various are the forms of
expression which he employs.

As in several other Dialogues, there is more of system in the Phaedo
than appears at first sight. The succession of arguments is based on
previous philosophies; beginning with the mysteries and the Heracleitean
alternation of opposites, and proceeding to the Pythagorean harmony and
transmigration; making a step by the aid of Platonic reminiscence, and
a further step by the help of the nous of Anaxagoras; until at last we
rest in the conviction that the soul is inseparable from the ideas,
and belongs to the world of the invisible and unknown. Then, as in
the Gorgias or Republic, the curtain falls, and the veil of mythology
descends upon the argument. After the confession of Socrates that he is
an interested party, and the acknowledgment that no man of sense will
think the details of his narrative true, but that something of the kind
is true, we return from speculation to practice. He is himself more
confident of immortality than he is of his own arguments; and the
confidence which he expresses is less strong than that which his
cheerfulness and composure in death inspire in us.

Difficulties of two kinds occur in the Phaedo--one kind to be explained
out of contemporary philosophy, the other not admitting of an entire
solution. (1) The difficulty which Socrates says that he experienced in
explaining generation and corruption; the assumption of hypotheses which
proceed from the less general to the more general, and are tested by
their consequences; the puzzle about greater and less; the resort to the
method of ideas, which to us appear only abstract terms,--these are to
be explained out of the position of Socrates and Plato in the history of
philosophy. They were living in a twilight between the sensible and
the intellectual world, and saw no way of connecting them. They
could neither explain the relation of ideas to phenomena, nor their
correlation to one another. The very idea of relation or comparison was
embarrassing to them. Yet in this intellectual uncertainty they had a
conception of a proof from results, and of a moral truth, which remained
unshaken amid the questionings of philosophy. (2) The other is a
difficulty which is touched upon in the Republic as well as in the
Phaedo, and is common to modern and ancient philosophy. Plato is not
altogether satisfied with his safe and simple method of ideas. He wants
to have proved to him by facts that all things are for the best, and
that there is one mind or design which pervades them all. But this
'power of the best' he is unable to explain; and therefore takes refuge
in universal ideas. And are not we at this day seeking to discover that
which Socrates in a glass darkly foresaw?

Some resemblances to the Greek drama may be noted in all the Dialogues
of Plato. The Phaedo is the tragedy of which Socrates is the protagonist
and Simmias and Cebes the secondary performers, standing to them in the
same relation as to Glaucon and Adeimantus in the Republic. No Dialogue
has a greater unity of subject and feeling. Plato has certainly
fulfilled the condition of Greek, or rather of all art, which requires
that scenes of death and suffering should be clothed in beauty. The
gathering of the friends at the commencement of the Dialogue, the
dismissal of Xanthippe, whose presence would have been out of place at
a philosophical discussion, but who returns again with her children to
take a final farewell, the dejection of the audience at the temporary
overthrow of the argument, the picture of Socrates playing with the
hair of Phaedo, the final scene in which Socrates alone retains his
composure--are masterpieces of art. And the chorus at the end might have
interpreted the feeling of the play: 'There can no evil happen to a good
man in life or death.'

'The art of concealing art' is nowhere more perfect than in those
writings of Plato which describe the trial and death of Socrates. Their
charm is their simplicity, which gives them verisimilitude; and yet
they touch, as if incidentally, and because they were suitable to the
occasion, on some of the deepest truths of philosophy. There is nothing
in any tragedy, ancient or modern, nothing in poetry or history (with
one exception), like the last hours of Socrates in Plato. The master
could not be more fitly occupied at such a time than in discoursing of
immortality; nor the disciples more divinely consoled. The arguments,
taken in the spirit and not in the letter, are our arguments; and
Socrates by anticipation may be even thought to refute some 'eccentric
notions; current in our own age. For there are philosophers among
ourselves who do not seem to understand how much stronger is the power
of intelligence, or of the best, than of Atlas, or mechanical force.
How far the words attributed to Socrates were actually uttered by him we
forbear to ask; for no answer can be given to this question. And it
is better to resign ourselves to the feeling of a great work, than to
linger among critical uncertainties.




PHAEDO


PERSONS OF THE DIALOGUE:

Phaedo, who is the narrator of the dialogue to Echecrates of Phlius.
Socrates, Apollodorus, Simmias, Cebes, Crito and an Attendant of the
Prison.

SCENE: The Prison of Socrates.

PLACE OF THE NARRATION: Phlius.



ECHECRATES: Were you yourself, Phaedo, in the prison with Socrates on
the day when he drank the poison?

PHAEDO: Yes, Echecrates, I was.

ECHECRATES: I should so like to hear about his death. What did he say in
his last hours? We were informed that he died by taking poison, but no
one knew anything more; for no Phliasian ever goes to Athens now, and it
is a long time since any stranger from Athens has found his way hither;
so that we had no clear account.

PHAEDO: Did you not hear of the proceedings at the trial?

ECHECRATES: Yes; some one told us about the trial, and we could not
understand why, having been condemned, he should have been put to death,
not at the time, but long afterwards. What was the reason of this?

PHAEDO: An accident, Echecrates: the stern of the ship which the
Athenians send to Delos happened to have been crowned on the day before
he was tried.

ECHECRATES: What is this ship?

PHAEDO: It is the ship in which, according to Athenian tradition,
Theseus went to Crete when he took with him the fourteen youths, and was
the saviour of them and of himself. And they were said to have vowed
to Apollo at the time, that if they were saved they would send a yearly
mission to Delos. Now this custom still continues, and the whole period
of the voyage to and from Delos, beginning when the priest of Apollo
crowns the stern of the ship, is a holy season, during which the city is
not allowed to be polluted by public executions; and when the vessel
is detained by contrary winds, the time spent in going and returning
is very considerable. As I was saying, the ship was crowned on the day
before the trial, and this was the reason why Socrates lay in prison and
was not put to death until long after he was condemned.

ECHECRATES: What was the manner of his death, Phaedo? What was said or
done? And which of his friends were with him? Or did the authorities
forbid them to be present--so that he had no friends near him when he
died?

PHAEDO: No; there were several of them with him.

ECHECRATES: If you have nothing to do, I wish that you would tell me
what passed, as exactly as you can.

PHAEDO: I have nothing at all to do, and will try to gratify your wish.
To be reminded of Socrates is always the greatest delight to me, whether
I speak myself or hear another speak of him.

ECHECRATES: You will have listeners who are of the same mind with you,
and I hope that you will be as exact as you can.

PHAEDO: I had a singular feeling at being in his company. For I
could hardly believe that I was present at the death of a friend, and
therefore I did not pity him, Echecrates; he died so fearlessly, and
his words and bearing were so noble and gracious, that to me he appeared
blessed. I thought that in going to the other world he could not be
without a divine call, and that he would be happy, if any man ever was,
when he arrived there, and therefore I did not pity him as might have
seemed natural at such an hour. But I had not the pleasure which I
usually feel in philosophical discourse (for philosophy was the theme
of which we spoke). I was pleased, but in the pleasure there was also a
strange admixture of pain; for I reflected that he was soon to die, and
this double feeling was shared by us all; we were laughing and weeping
by turns, especially the excitable Apollodorus--you know the sort of
man?

ECHECRATES: Yes.

PHAEDO: He was quite beside himself; and I and all of us were greatly
moved.

ECHECRATES: Who were present?

PHAEDO: Of native Athenians there were, besides Apollodorus, Critobulus
and his father Crito, Hermogenes, Epigenes, Aeschines, Antisthenes;
likewise Ctesippus of the deme of Paeania, Menexenus, and some others;
Plato, if I am not mistaken, was ill.

ECHECRATES: Were there any strangers?

PHAEDO: Yes, there were; Simmias the Theban, and Cebes, and Phaedondes;
Euclid and Terpison, who came from Megara.

ECHECRATES: And was Aristippus there, and Cleombrotus?

PHAEDO: No, they were said to be in Aegina.

ECHECRATES: Any one else?

PHAEDO: I think that these were nearly all.

ECHECRATES: Well, and what did you talk about?

PHAEDO: I will begin at the beginning, and endeavour to repeat the
entire conversation. On the previous days we had been in the habit of
assembling early in the morning at the court in which the trial took
place, and which is not far from the prison. There we used to wait
talking with one another until the opening of the doors (for they were
not opened very early); then we went in and generally passed the day
with Socrates. On the last morning we assembled sooner than usual,
having heard on the day before when we quitted the prison in the evening
that the sacred ship had come from Delos, and so we arranged to meet
very early at the accustomed place. On our arrival the jailer who
answered the door, instead of admitting us, came out and told us to stay
until he called us. 'For the Eleven,' he said, 'are now with Socrates;
they are taking off his chains, and giving orders that he is to die
to-day.' He soon returned and said that we might come in. On entering we
found Socrates just released from chains, and Xanthippe, whom you know,
sitting by him, and holding his child in her arms. When she saw us she
uttered a cry and said, as women will: 'O Socrates, this is the last
time that either you will converse with your friends, or they with you.'
Socrates turned to Crito and said: 'Crito, let some one take her home.'
Some of Crito's people accordingly led her away, crying out and beating
herself. And when she was gone, Socrates, sitting up on the couch, bent
and rubbed his leg, saying, as he was rubbing: How singular is the
thing called pleasure, and how curiously related to pain, which might be
thought to be the opposite of it; for they are never present to a man at
the same instant, and yet he who pursues either is generally compelled
to take the other; their bodies are two, but they are joined by a single
head. And I cannot help thinking that if Aesop had remembered them, he
would have made a fable about God trying to reconcile their strife, and
how, when he could not, he fastened their heads together; and this is
the reason why when one comes the other follows, as I know by my own
experience now, when after the pain in my leg which was caused by the
chain pleasure appears to succeed.

Upon this Cebes said: I am glad, Socrates, that you have mentioned the
name of Aesop. For it reminds me of a question which has been asked by
many, and was asked of me only the day before yesterday by Evenus the
poet--he will be sure to ask it again, and therefore if you would like
me to have an answer ready for him, you may as well tell me what I
should say to him:--he wanted to know why you, who never before wrote
a line of poetry, now that you are in prison are turning Aesop's fables
into verse, and also composing that hymn in honour of Apollo.

Tell him, Cebes, he replied, what is the truth--that I had no idea of
rivalling him or his poems; to do so, as I knew, would be no easy task.
But I wanted to see whether I could purge away a scruple which I felt
about the meaning of certain dreams. In the course of my life I have
often had intimations in dreams 'that I should compose music.' The same
dream came to me sometimes in one form, and sometimes in another, but
always saying the same or nearly the same words: 'Cultivate and make
music,' said the dream. And hitherto I had imagined that this was only
intended to exhort and encourage me in the study of philosophy, which
has been the pursuit of my life, and is the noblest and best of music.
The dream was bidding me do what I was already doing, in the same way
that the competitor in a race is bidden by the spectators to run when he
is already running. But I was not certain of this, for the dream might
have meant music in the popular sense of the word, and being under
sentence of death, and the festival giving me a respite, I thought that
it would be safer for me to satisfy the scruple, and, in obedience to
the dream, to compose a few verses before I departed. And first I made
a hymn in honour of the god of the festival, and then considering that a
poet, if he is really to be a poet, should not only put together words,
but should invent stories, and that I have no invention, I took some
fables of Aesop, which I had ready at hand and which I knew--they were
the first I came upon--and turned them into verse. Tell this to Evenus,
Cebes, and bid him be of good cheer; say that I would have him come
after me if he be a wise man, and not tarry; and that to-day I am likely
to be going, for the Athenians say that I must.

Simmias said: What a message for such a man! having been a frequent
companion of his I should say that, as far as I know him, he will never
take your advice unless he is obliged.

Why, said Socrates,--is not Evenus a philosopher?

I think that he is, said Simmias.

Then he, or any man who has the spirit of philosophy, will be willing to
die, but he will not take his own life, for that is held to be unlawful.

Here he changed his position, and put his legs off the couch on to the
ground, and during the rest of the conversation he remained sitting.

Why do you say, enquired Cebes, that a man ought not to take his own
life, but that the philosopher will be ready to follow the dying?

Socrates replied: And have you, Cebes and Simmias, who are the disciples
of Philolaus, never heard him speak of this?

Yes, but his language was obscure, Socrates.

My words, too, are only an echo; but there is no reason why I should not
repeat what I have heard: and indeed, as I am going to another place,
it is very meet for me to be thinking and talking of the nature of
the pilgrimage which I am about to make. What can I do better in the
interval between this and the setting of the sun?

Then tell me, Socrates, why is suicide held to be unlawful? as I have
certainly heard Philolaus, about whom you were just now asking, affirm
when he was staying with us at Thebes: and there are others who say the
same, although I have never understood what was meant by any of them.

Do not lose heart, replied Socrates, and the day may come when you will
understand. I suppose that you wonder why, when other things which are
evil may be good at certain times and to certain persons, death is to
be the only exception, and why, when a man is better dead, he is not
permitted to be his own benefactor, but must wait for the hand of
another.

Very true, said Cebes, laughing gently and speaking in his native
Boeotian.

I admit the appearance of inconsistency in what I am saying; but
there may not be any real inconsistency after all. There is a doctrine
whispered in secret that man is a prisoner who has no right to open
the door and run away; this is a great mystery which I do not quite
understand. Yet I too believe that the gods are our guardians, and that
we are a possession of theirs. Do you not agree?

Yes, I quite agree, said Cebes.

And if one of your own possessions, an ox or an ass, for example, took
the liberty of putting himself out of the way when you had given no
intimation of your wish that he should die, would you not be angry with
him, and would you not punish him if you could?

Certainly, replied Cebes.

Then, if we look at the matter thus, there may be reason in saying that
a man should wait, and not take his own life until God summons him, as
he is now summoning me.

Yes, Socrates, said Cebes, there seems to be truth in what you say. And
yet how can you reconcile this seemingly true belief that God is our
guardian and we his possessions, with the willingness to die which we
were just now attributing to the philosopher? That the wisest of men
should be willing to leave a service in which they are ruled by the gods
who are the best of rulers, is not reasonable; for surely no wise man
thinks that when set at liberty he can take better care of himself than
the gods take of him. A fool may perhaps think so--he may argue that he
had better run away from his master, not considering that his duty is
to remain to the end, and not to run away from the good, and that there
would be no sense in his running away. The wise man will want to be ever
with him who is better than himself. Now this, Socrates, is the reverse
of what was just now said; for upon this view the wise man should sorrow
and the fool rejoice at passing out of life.

The earnestness of Cebes seemed to please Socrates. Here, said he,
turning to us, is a man who is always inquiring, and is not so easily
convinced by the first thing which he hears.

And certainly, added Simmias, the objection which he is now making does
appear to me to have some force. For what can be the meaning of a truly
wise man wanting to fly away and lightly leave a master who is better
than himself? And I rather imagine that Cebes is referring to you; he
thinks that you are too ready to leave us, and too ready to leave the
gods whom you acknowledge to be our good masters.

Yes, replied Socrates; there is reason in what you say. And so you think
that I ought to answer your indictment as if I were in a court?

We should like you to do so, said Simmias.

Then I must try to make a more successful defence before you than I
did when before the judges. For I am quite ready to admit, Simmias and
Cebes, that I ought to be grieved at death, if I were not persuaded in
the first place that I am going to other gods who are wise and good (of
which I am as certain as I can be of any such matters), and secondly
(though I am not so sure of this last) to men departed, better than
those whom I leave behind; and therefore I do not grieve as I might have
done, for I have good hope that there is yet something remaining for the
dead, and as has been said of old, some far better thing for the good
than for the evil.

But do you mean to take away your thoughts with you, Socrates? said
Simmias. Will you not impart them to us?--for they are a benefit
in which we too are entitled to share. Moreover, if you succeed in
convincing us, that will be an answer to the charge against yourself.

I will do my best, replied Socrates. But you must first let me hear what
Crito wants; he has long been wishing to say something to me.

Only this, Socrates, replied Crito:--the attendant who is to give you
the poison has been telling me, and he wants me to tell you, that you
are not to talk much, talking, he says, increases heat, and this is
apt to interfere with the action of the poison; persons who excite
themselves are sometimes obliged to take a second or even a third dose.

Then, said Socrates, let him mind his business and be prepared to give
the poison twice or even thrice if necessary; that is all.

I knew quite well what you would say, replied Crito; but I was obliged
to satisfy him.

Never mind him, he said.

And now, O my judges, I desire to prove to you that the real philosopher
has reason to be of good cheer when he is about to die, and that after
death he may hope to obtain the greatest good in the other world. And
how this may be, Simmias and Cebes, I will endeavour to explain. For I
deem that the true votary of philosophy is likely to be misunderstood
by other men; they do not perceive that he is always pursuing death and
dying; and if this be so, and he has had the desire of death all his
life long, why when his time comes should he repine at that which he has
been always pursuing and desiring?

Simmias said laughingly: Though not in a laughing humour, you have made
me laugh, Socrates; for I cannot help thinking that the many when they
hear your words will say how truly you have described philosophers, and
our people at home will likewise say that the life which philosophers
desire is in reality death, and that they have found them out to be
deserving of the death which they desire.

And they are right, Simmias, in thinking so, with the exception of the
words 'they have found them out'; for they have not found out either
what is the nature of that death which the true philosopher deserves,
or how he deserves or desires death. But enough of them:--let us discuss
the matter among ourselves: Do we believe that there is such a thing as
death?

To be sure, replied Simmias.

Is it not the separation of soul and body? And to be dead is the
completion of this; when the soul exists in herself, and is released
from the body and the body is released from the soul, what is this but
death?

Just so, he replied.

There is another question, which will probably throw light on our
present inquiry if you and I can agree about it:--Ought the philosopher
to care about the pleasures--if they are to be called pleasures--of
eating and drinking?

Certainly not, answered Simmias.

And what about the pleasures of love--should he care for them?

By no means.

And will he think much of the other ways of indulging the body, for
example, the acquisition of costly raiment, or sandals, or other
adornments of the body? Instead of caring about them, does he not rather
despise anything more than nature needs? What do you say?

I should say that the true philosopher would despise them.

Would you not say that he is entirely concerned with the soul and not
with the body? He would like, as far as he can, to get away from the
body and to turn to the soul.

Quite true.

In matters of this sort philosophers, above all other men, may be
observed in every sort of way to dissever the soul from the communion of
the body.

Very true.

Whereas, Simmias, the rest of the world are of opinion that to him who
has no sense of pleasure and no part in bodily pleasure, life is not
worth having; and that he who is indifferent about them is as good as
dead.

That is also true.

What again shall we say of the actual acquirement of knowledge?--is the
body, if invited to share in the enquiry, a hinderer or a helper? I mean
to say, have sight and hearing any truth in them? Are they not, as the
poets are always telling us, inaccurate witnesses? and yet, if even
they are inaccurate and indistinct, what is to be said of the other
senses?--for you will allow that they are the best of them?

Certainly, he replied.

Then when does the soul attain truth?--for in attempting to consider
anything in company with the body she is obviously deceived.

True.

Then must not true existence be revealed to her in thought, if at all?

Yes.

And thought is best when the mind is gathered into herself and none of
these things trouble her--neither sounds nor sights nor pain nor any
pleasure,--when she takes leave of the body, and has as little as
possible to do with it, when she has no bodily sense or desire, but is
aspiring after true being?

Certainly.

And in this the philosopher dishonours the body; his soul runs away from
his body and desires to be alone and by herself?

That is true.

Well, but there is another thing, Simmias: Is there or is there not an
absolute justice?

Assuredly there is.

And an absolute beauty and absolute good?

Of course.

But did you ever behold any of them with your eyes?

Certainly not.

Or did you ever reach them with any other bodily sense?--and I speak not
of these alone, but of absolute greatness, and health, and strength,
and of the essence or true nature of everything. Has the reality of them
ever been perceived by you through the bodily organs? or rather, is not
the nearest approach to the knowledge of their several natures made
by him who so orders his intellectual vision as to have the most exact
conception of the essence of each thing which he considers?

Certainly.

And he attains to the purest knowledge of them who goes to each with the
mind alone, not introducing or intruding in the act of thought sight
or any other sense together with reason, but with the very light of the
mind in her own clearness searches into the very truth of each; he who
has got rid, as far as he can, of eyes and ears and, so to speak, of the
whole body, these being in his opinion distracting elements which when
they infect the soul hinder her from acquiring truth and knowledge--who,
if not he, is likely to attain the knowledge of true being?

What you say has a wonderful truth in it, Socrates, replied Simmias.

And when real philosophers consider all these things, will they not be
led to make a reflection which they will express in words something like
the following? 'Have we not found,' they will say, 'a path of thought
which seems to bring us and our argument to the conclusion, that while
we are in the body, and while the soul is infected with the evils of the
body, our desire will not be satisfied? and our desire is of the truth.
For the body is a source of endless trouble to us by reason of the mere
requirement of food; and is liable also to diseases which overtake and
impede us in the search after true being: it fills us full of loves, and
lusts, and fears, and fancies of all kinds, and endless foolery, and
in fact, as men say, takes away from us the power of thinking at all.
Whence come wars, and fightings, and factions? whence but from the body
and the lusts of the body? wars are occasioned by the love of money, and
money has to be acquired for the sake and in the service of the body;
and by reason of all these impediments we have no time to give to
philosophy; and, last and worst of all, even if we are at leisure and
betake ourselves to some speculation, the body is always breaking in
upon us, causing turmoil and confusion in our enquiries, and so amazing
us that we are prevented from seeing the truth. It has been proved to us
by experience that if we would have pure knowledge of anything we
must be quit of the body--the soul in herself must behold things in
themselves: and then we shall attain the wisdom which we desire, and of
which we say that we are lovers, not while we live, but after death; for
if while in company with the body, the soul cannot have pure knowledge,
one of two things follows--either knowledge is not to be attained at
all, or, if at all, after death. For then, and not till then, the soul
will be parted from the body and exist in herself alone. In this present
life, I reckon that we make the nearest approach to knowledge when we
have the least possible intercourse or communion with the body, and are
not surfeited with the bodily nature, but keep ourselves pure until the
hour when God himself is pleased to release us. And thus having got rid
of the foolishness of the body we shall be pure and hold converse with
the pure, and know of ourselves the clear light everywhere, which is
no other than the light of truth.' For the impure are not permitted to
approach the pure. These are the sort of words, Simmias, which the true
lovers of knowledge cannot help saying to one another, and thinking. You
would agree; would you not?

Undoubtedly, Socrates.

But, O my friend, if this is true, there is great reason to hope that,
going whither I go, when I have come to the end of my journey, I shall
attain that which has been the pursuit of my life. And therefore I go on
my way rejoicing, and not I only, but every other man who believes that
his mind has been made ready and that he is in a manner purified.

Certainly, replied Simmias.

And what is purification but the separation of the soul from the body,
as I was saying before; the habit of the soul gathering and collecting
herself into herself from all sides out of the body; the dwelling in
her own place alone, as in another life, so also in this, as far as she
can;--the release of the soul from the chains of the body?

Very true, he said.

And this separation and release of the soul from the body is termed
death?

To be sure, he said.

And the true philosophers, and they only, are ever seeking to release
the soul. Is not the separation and release of the soul from the body
their especial study?

That is true.

And, as I was saying at first, there would be a ridiculous contradiction
in men studying to live as nearly as they can in a state of death, and
yet repining when it comes upon them.

Clearly.

And the true philosophers, Simmias, are always occupied in the practice
of dying, wherefore also to them least of all men is death terrible.
Look at the matter thus:--if they have been in every way the enemies of
the body, and are wanting to be alone with the soul, when this desire of
theirs is granted, how inconsistent would they be if they trembled and
repined, instead of rejoicing at their departure to that place where,
when they arrive, they hope to gain that which in life they desired--and
this was wisdom--and at the same time to be rid of the company of their
enemy. Many a man has been willing to go to the world below animated
by the hope of seeing there an earthly love, or wife, or son, and
conversing with them. And will he who is a true lover of wisdom, and is
strongly persuaded in like manner that only in the world below he can
worthily enjoy her, still repine at death? Will he not depart with joy?
Surely he will, O my friend, if he be a true philosopher. For he will
have a firm conviction that there and there only, he can find wisdom
in her purity. And if this be true, he would be very absurd, as I was
saying, if he were afraid of death.

He would, indeed, replied Simmias.

And when you see a man who is repining at the approach of death, is not
his reluctance a sufficient proof that he is not a lover of wisdom, but
a lover of the body, and probably at the same time a lover of either
money or power, or both?

Quite so, he replied.

And is not courage, Simmias, a quality which is specially characteristic
of the philosopher?

Certainly.

There is temperance again, which even by the vulgar is supposed to
consist in the control and regulation of the passions, and in the sense
of superiority to them--is not temperance a virtue belonging to those
only who despise the body, and who pass their lives in philosophy?

Most assuredly.

For the courage and temperance of other men, if you will consider them,
are really a contradiction.

How so?

Well, he said, you are aware that death is regarded by men in general as
a great evil.

Very true, he said.

And do not courageous men face death because they are afraid of yet
greater evils?

That is quite true.

Then all but the philosophers are courageous only from fear, and because
they are afraid; and yet that a man should be courageous from fear, and
because he is a coward, is surely a strange thing.

Very true.

And are not the temperate exactly in the same case? They are temperate
because they are intemperate--which might seem to be a contradiction,
but is nevertheless the sort of thing which happens with this foolish
temperance. For there are pleasures which they are afraid of losing; and
in their desire to keep them, they abstain from some pleasures, because
they are overcome by others; and although to be conquered by pleasure is
called by men intemperance, to them the conquest of pleasure consists in
being conquered by pleasure. And that is what I mean by saying that, in
a sense, they are made temperate through intemperance.

Such appears to be the case.

Yet the exchange of one fear or pleasure or pain for another fear or
pleasure or pain, and of the greater for the less, as if they were
coins, is not the exchange of virtue. O my blessed Simmias, is there not
one true coin for which all things ought to be exchanged?--and that
is wisdom; and only in exchange for this, and in company with this, is
anything truly bought or sold, whether courage or temperance or justice.
And is not all true virtue the companion of wisdom, no matter what fears
or pleasures or other similar goods or evils may or may not attend her?
But the virtue which is made up of these goods, when they are severed
from wisdom and exchanged with one another, is a shadow of virtue only,
nor is there any freedom or health or truth in her; but in the true
exchange there is a purging away of all these things, and temperance,
and justice, and courage, and wisdom herself are the purgation of them.
The founders of the mysteries would appear to have had a real meaning,
and were not talking nonsense when they intimated in a figure long ago
that he who passes unsanctified and uninitiated into the world below
will lie in a slough, but that he who arrives there after initiation and
purification will dwell with the gods. For 'many,' as they say in the
mysteries, 'are the thyrsus-bearers, but few are the mystics,'--meaning,
as I interpret the words, 'the true philosophers.' In the number
of whom, during my whole life, I have been seeking, according to my
ability, to find a place;--whether I have sought in a right way or not,
and whether I have succeeded or not, I shall truly know in a little
while, if God will, when I myself arrive in the other world--such is my
belief. And therefore I maintain that I am right, Simmias and Cebes,
in not grieving or repining at parting from you and my masters in this
world, for I believe that I shall equally find good masters and friends
in another world. But most men do not believe this saying; if then I
succeed in convincing you by my defence better than I did the Athenian
judges, it will be well.

Cebes answered: I agree, Socrates, in the greater part of what you say.
But in what concerns the soul, men are apt to be incredulous; they fear
that when she has left the body her place may be nowhere, and that on
the very day of death she may perish and come to an end--immediately on
her release from the body, issuing forth dispersed like smoke or air
and in her flight vanishing away into nothingness. If she could only be
collected into herself after she has obtained release from the evils of
which you are speaking, there would be good reason to hope, Socrates,
that what you say is true. But surely it requires a great deal of
argument and many proofs to show that when the man is dead his soul yet
exists, and has any force or intelligence.

True, Cebes, said Socrates; and shall I suggest that we converse a
little of the probabilities of these things?

I am sure, said Cebes, that I should greatly like to know your opinion
about them.

I reckon, said Socrates, that no one who heard me now, not even if he
were one of my old enemies, the Comic poets, could accuse me of idle
talking about matters in which I have no concern:--If you please, then,
we will proceed with the inquiry.

Suppose we consider the question whether the souls of men after death
are or are not in the world below. There comes into my mind an ancient
doctrine which affirms that they go from hence into the other world, and
returning hither, are born again from the dead. Now if it be true that
the living come from the dead, then our souls must exist in the other
world, for if not, how could they have been born again? And this would
be conclusive, if there were any real evidence that the living are only
born from the dead; but if this is not so, then other arguments will
have to be adduced.

Very true, replied Cebes.

Then let us consider the whole question, not in relation to man only,
but in relation to animals generally, and to plants, and to everything
of which there is generation, and the proof will be easier. Are not all
things which have opposites generated out of their opposites? I mean
such things as good and evil, just and unjust--and there are innumerable
other opposites which are generated out of opposites. And I want to show
that in all opposites there is of necessity a similar alternation;
I mean to say, for example, that anything which becomes greater must
become greater after being less.

True.

And that which becomes less must have been once greater and then have
become less.

Yes.

And the weaker is generated from the stronger, and the swifter from the
slower.

Very true.

And the worse is from the better, and the more just is from the more
unjust.

Of course.

And is this true of all opposites? and are we convinced that all of them
are generated out of opposites?

Yes.

And in this universal opposition of all things, are there not also two
intermediate processes which are ever going on, from one to the other
opposite, and back again; where there is a greater and a less there is
also an intermediate process of increase and diminution, and that which
grows is said to wax, and that which decays to wane?

Yes, he said.

And there are many other processes, such as division and composition,
cooling and heating, which equally involve a passage into and out of one
another. And this necessarily holds of all opposites, even though not
always expressed in words--they are really generated out of one another,
and there is a passing or process from one to the other of them?

Very true, he replied.

Well, and is there not an opposite of life, as sleep is the opposite of
waking?

True, he said.

And what is it?

Death, he answered.

And these, if they are opposites, are generated the one from the other,
and have there their two intermediate processes also?

Of course.

Now, said Socrates, I will analyze one of the two pairs of opposites
which I have mentioned to you, and also its intermediate processes, and
you shall analyze the other to me. One of them I term sleep, the other
waking. The state of sleep is opposed to the state of waking, and out
of sleeping waking is generated, and out of waking, sleeping; and the
process of generation is in the one case falling asleep, and in the
other waking up. Do you agree?

I entirely agree.

Then, suppose that you analyze life and death to me in the same manner.
Is not death opposed to life?

Yes.

And they are generated one from the other?

Yes.

What is generated from the living?

The dead.

And what from the dead?

I can only say in answer--the living.

Then the living, whether things or persons, Cebes, are generated from
the dead?

That is clear, he replied.

Then the inference is that our souls exist in the world below?

That is true.

And one of the two processes or generations is visible--for surely the
act of dying is visible?

Surely, he said.

What then is to be the result? Shall we exclude the opposite process?
And shall we suppose nature to walk on one leg only? Must we not rather
assign to death some corresponding process of generation?

Certainly, he replied.

And what is that process?

Return to life.

And return to life, if there be such a thing, is the birth of the dead
into the world of the living?

Quite true.

Then here is a new way by which we arrive at the conclusion that the
living come from the dead, just as the dead come from the living; and
this, if true, affords a most certain proof that the souls of the dead
exist in some place out of which they come again.

Yes, Socrates, he said; the conclusion seems to flow necessarily out of
our previous admissions.

And that these admissions were not unfair, Cebes, he said, may be shown,
I think, as follows: If generation were in a straight line only, and
there were no compensation or circle in nature, no turn or return of
elements into their opposites, then you know that all things would at
last have the same form and pass into the same state, and there would be
no more generation of them.

What do you mean? he said.

A simple thing enough, which I will illustrate by the case of sleep,
he replied. You know that if there were no alternation of sleeping
and waking, the tale of the sleeping Endymion would in the end have no
meaning, because all other things would be asleep, too, and he would not
be distinguishable from the rest. Or if there were composition only,
and no division of substances, then the chaos of Anaxagoras would come
again. And in like manner, my dear Cebes, if all things which partook
of life were to die, and after they were dead remained in the form
of death, and did not come to life again, all would at last die, and
nothing would be alive--what other result could there be? For if the
living spring from any other things, and they too die, must not all
things at last be swallowed up in death? (But compare Republic.)

There is no escape, Socrates, said Cebes; and to me your argument seems
to be absolutely true.

Yes, he said, Cebes, it is and must be so, in my opinion; and we have
not been deluded in making these admissions; but I am confident that
there truly is such a thing as living again, and that the living spring
from the dead, and that the souls of the dead are in existence, and that
the good souls have a better portion than the evil.

Cebes added: Your favorite doctrine, Socrates, that knowledge is simply
recollection, if true, also necessarily implies a previous time in
which we have learned that which we now recollect. But this would be
impossible unless our soul had been in some place before existing in the
form of man; here then is another proof of the soul's immortality.

But tell me, Cebes, said Simmias, interposing, what arguments are urged
in favour of this doctrine of recollection. I am not very sure at the
moment that I remember them.

One excellent proof, said Cebes, is afforded by questions. If you put
a question to a person in a right way, he will give a true answer of
himself, but how could he do this unless there were knowledge and right
reason already in him? And this is most clearly shown when he is taken
to a diagram or to anything of that sort. (Compare Meno.)

But if, said Socrates, you are still incredulous, Simmias, I would ask
you whether you may not agree with me when you look at the matter
in another way;--I mean, if you are still incredulous as to whether
knowledge is recollection.

Incredulous, I am not, said Simmias; but I want to have this doctrine
of recollection brought to my own recollection, and, from what Cebes has
said, I am beginning to recollect and be convinced; but I should still
like to hear what you were going to say.

This is what I would say, he replied:--We should agree, if I am not
mistaken, that what a man recollects he must have known at some previous
time.

Very true.

And what is the nature of this knowledge or recollection? I mean to
ask, Whether a person who, having seen or heard or in any way perceived
anything, knows not only that, but has a conception of something
else which is the subject, not of the same but of some other kind of
knowledge, may not be fairly said to recollect that of which he has the
conception?

What do you mean?

I mean what I may illustrate by the following instance:--The knowledge
of a lyre is not the same as the knowledge of a man?

True.

And yet what is the feeling of lovers when they recognize a lyre, or
a garment, or anything else which the beloved has been in the habit of
using? Do not they, from knowing the lyre, form in the mind's eye an
image of the youth to whom the lyre belongs? And this is recollection.
In like manner any one who sees Simmias may remember Cebes; and there
are endless examples of the same thing.

Endless, indeed, replied Simmias.

And recollection is most commonly a process of recovering that which has
been already forgotten through time and inattention.

Very true, he said.

Well; and may you not also from seeing the picture of a horse or a
lyre remember a man? and from the picture of Simmias, you may be led to
remember Cebes?

True.

Or you may also be led to the recollection of Simmias himself?

Quite so.

And in all these cases, the recollection may be derived from things
either like or unlike?

It may be.

And when the recollection is derived from like things, then another
consideration is sure to arise, which is--whether the likeness in any
degree falls short or not of that which is recollected?

Very true, he said.

And shall we proceed a step further, and affirm that there is such a
thing as equality, not of one piece of wood or stone with another, but
that, over and above this, there is absolute equality? Shall we say so?

Say so, yes, replied Simmias, and swear to it, with all the confidence
in life.

And do we know the nature of this absolute essence?

To be sure, he said.

And whence did we obtain our knowledge? Did we not see equalities of
material things, such as pieces of wood and stones, and gather from
them the idea of an equality which is different from them? For you will
acknowledge that there is a difference. Or look at the matter in another
way:--Do not the same pieces of wood or stone appear at one time equal,
and at another time unequal?

That is certain.

But are real equals ever unequal? or is the idea of equality the same as
of inequality?

Impossible, Socrates.

Then these (so-called) equals are not the same with the idea of
equality?

I should say, clearly not, Socrates.

And yet from these equals, although differing from the idea of equality,
you conceived and attained that idea?

Very true, he said.

Which might be like, or might be unlike them?

Yes.

But that makes no difference; whenever from seeing one thing you
conceived another, whether like or unlike, there must surely have been
an act of recollection?

Very true.

But what would you say of equal portions of wood and stone, or other
material equals? and what is the impression produced by them? Are they
equals in the same sense in which absolute equality is equal? or do they
fall short of this perfect equality in a measure?

Yes, he said, in a very great measure too.

And must we not allow, that when I or any one, looking at any object,
observes that the thing which he sees aims at being some other thing,
but falls short of, and cannot be, that other thing, but is inferior, he
who makes this observation must have had a previous knowledge of that to
which the other, although similar, was inferior?

Certainly.

And has not this been our own case in the matter of equals and of
absolute equality?

Precisely.

Then we must have known equality previously to the time when we first
saw the material equals, and reflected that all these apparent equals
strive to attain absolute equality, but fall short of it?

Very true.

And we recognize also that this absolute equality has only been known,
and can only be known, through the medium of sight or touch, or of some
other of the senses, which are all alike in this respect?

Yes, Socrates, as far as the argument is concerned, one of them is the
same as the other.

From the senses then is derived the knowledge that all sensible things
aim at an absolute equality of which they fall short?

Yes.

Then before we began to see or hear or perceive in any way, we must have
had a knowledge of absolute equality, or we could not have referred to
that standard the equals which are derived from the senses?--for to that
they all aspire, and of that they fall short.

No other inference can be drawn from the previous statements.

And did we not see and hear and have the use of our other senses as soon
as we were born?

Certainly.

Then we must have acquired the knowledge of equality at some previous
time?

Yes.

That is to say, before we were born, I suppose?

True.

And if we acquired this knowledge before we were born, and were born
having the use of it, then we also knew before we were born and at the
instant of birth not only the equal or the greater or the less, but all
other ideas; for we are not speaking only of equality, but of beauty,
goodness, justice, holiness, and of all which we stamp with the name of
essence in the dialectical process, both when we ask and when we answer
questions. Of all this we may certainly affirm that we acquired the
knowledge before birth?

We may.

But if, after having acquired, we have not forgotten what in each case
we acquired, then we must always have come into life having knowledge,
and shall always continue to know as long as life lasts--for knowing
is the acquiring and retaining knowledge and not forgetting. Is not
forgetting, Simmias, just the losing of knowledge?

Quite true, Socrates.

But if the knowledge which we acquired before birth was lost by us at
birth, and if afterwards by the use of the senses we recovered what
we previously knew, will not the process which we call learning be a
recovering of the knowledge which is natural to us, and may not this be
rightly termed recollection?

Very true.

So much is clear--that when we perceive something, either by the help of
sight, or hearing, or some other sense, from that perception we are
able to obtain a notion of some other thing like or unlike which is
associated with it but has been forgotten. Whence, as I was saying, one
of two alternatives follows:--either we had this knowledge at birth, and
continued to know through life; or, after birth, those who are said to
learn only remember, and learning is simply recollection.

Yes, that is quite true, Socrates.

And which alternative, Simmias, do you prefer? Had we the knowledge at
our birth, or did we recollect the things which we knew previously to
our birth?

I cannot decide at the moment.

At any rate you can decide whether he who has knowledge will or will not
be able to render an account of his knowledge? What do you say?

Certainly, he will.

But do you think that every man is able to give an account of these very
matters about which we are speaking?

Would that they could, Socrates, but I rather fear that to-morrow, at
this time, there will no longer be any one alive who is able to give an
account of them such as ought to be given.

Then you are not of opinion, Simmias, that all men know these things?

Certainly not.

They are in process of recollecting that which they learned before?

Certainly.

But when did our souls acquire this knowledge?--not since we were born
as men?

Certainly not.

And therefore, previously?

Yes.

Then, Simmias, our souls must also have existed without bodies before
they were in the form of man, and must have had intelligence.

Unless indeed you suppose, Socrates, that these notions are given us at
the very moment of birth; for this is the only time which remains.

Yes, my friend, but if so, when do we lose them? for they are not in
us when we are born--that is admitted. Do we lose them at the moment of
receiving them, or if not at what other time?

No, Socrates, I perceive that I was unconsciously talking nonsense.

Then may we not say, Simmias, that if, as we are always repeating, there
is an absolute beauty, and goodness, and an absolute essence of all
things; and if to this, which is now discovered to have existed in our
former state, we refer all our sensations, and with this compare them,
finding these ideas to be pre-existent and our inborn possession--then
our souls must have had a prior existence, but if not, there would be
no force in the argument? There is the same proof that these ideas must
have existed before we were born, as that our souls existed before we
were born; and if not the ideas, then not the souls.

Yes, Socrates; I am convinced that there is precisely the same necessity
for the one as for the other; and the argument retreats successfully
to the position that the existence of the soul before birth cannot be
separated from the existence of the essence of which you speak. For
there is nothing which to my mind is so patent as that beauty, goodness,
and the other notions of which you were just now speaking, have a most
real and absolute existence; and I am satisfied with the proof.

Well, but is Cebes equally satisfied? for I must convince him too.

I think, said Simmias, that Cebes is satisfied: although he is the most
incredulous of mortals, yet I believe that he is sufficiently convinced
of the existence of the soul before birth. But that after death the soul
will continue to exist is not yet proven even to my own satisfaction.
I cannot get rid of the feeling of the many to which Cebes was
referring--the feeling that when the man dies the soul will be
dispersed, and that this may be the extinction of her. For admitting
that she may have been born elsewhere, and framed out of other elements,
and was in existence before entering the human body, why after having
entered in and gone out again may she not herself be destroyed and come
to an end?

Very true, Simmias, said Cebes; about half of what was required has been
proven; to wit, that our souls existed before we were born:--that the
soul will exist after death as well as before birth is the other half of
which the proof is still wanting, and has to be supplied; when that is
given the demonstration will be complete.

But that proof, Simmias and Cebes, has been already given, said
Socrates, if you put the two arguments together--I mean this and the
former one, in which we admitted that everything living is born of the
dead. For if the soul exists before birth, and in coming to life and
being born can be born only from death and dying, must she not after
death continue to exist, since she has to be born again?--Surely the
proof which you desire has been already furnished. Still I suspect
that you and Simmias would be glad to probe the argument further. Like
children, you are haunted with a fear that when the soul leaves the
body, the wind may really blow her away and scatter her; especially if a
man should happen to die in a great storm and not when the sky is calm.

Cebes answered with a smile: Then, Socrates, you must argue us out of
our fears--and yet, strictly speaking, they are not our fears, but there
is a child within us to whom death is a sort of hobgoblin; him too we
must persuade not to be afraid when he is alone in the dark.

Socrates said: Let the voice of the charmer be applied daily until you
have charmed away the fear.

And where shall we find a good charmer of our fears, Socrates, when you
are gone?

Hellas, he replied, is a large place, Cebes, and has many good men, and
there are barbarous races not a few: seek for him among them all, far
and wide, sparing neither pains nor money; for there is no better way
of spending your money. And you must seek among yourselves too; for you
will not find others better able to make the search.

The search, replied Cebes, shall certainly be made. And now, if
you please, let us return to the point of the argument at which we
digressed.

By all means, replied Socrates; what else should I please?

Very good.

Must we not, said Socrates, ask ourselves what that is which, as we
imagine, is liable to be scattered, and about which we fear? and what
again is that about which we have no fear? And then we may proceed
further to enquire whether that which suffers dispersion is or is not
of the nature of soul--our hopes and fears as to our own souls will turn
upon the answers to these questions.

Very true, he said.

Now the compound or composite may be supposed to be naturally capable,
as of being compounded, so also of being dissolved; but that which is
uncompounded, and that only, must be, if anything is, indissoluble.

Yes; I should imagine so, said Cebes.

And the uncompounded may be assumed to be the same and unchanging,
whereas the compound is always changing and never the same.

I agree, he said.

Then now let us return to the previous discussion. Is that idea or
essence, which in the dialectical process we define as essence or true
existence--whether essence of equality, beauty, or anything else--are
these essences, I say, liable at times to some degree of change? or
are they each of them always what they are, having the same simple
self-existent and unchanging forms, not admitting of variation at all,
or in any way, or at any time?

They must be always the same, Socrates, replied Cebes.

And what would you say of the many beautiful--whether men or horses or
garments or any other things which are named by the same names and may
be called equal or beautiful,--are they all unchanging and the same
always, or quite the reverse? May they not rather be described as almost
always changing and hardly ever the same, either with themselves or with
one another?

The latter, replied Cebes; they are always in a state of change.

And these you can touch and see and perceive with the senses, but
the unchanging things you can only perceive with the mind--they are
invisible and are not seen?

That is very true, he said.

Well, then, added Socrates, let us suppose that there are two sorts of
existences--one seen, the other unseen.

Let us suppose them.

The seen is the changing, and the unseen is the unchanging?

That may be also supposed.

And, further, is not one part of us body, another part soul?

To be sure.

And to which class is the body more alike and akin?

Clearly to the seen--no one can doubt that.

And is the soul seen or not seen?

Not by man, Socrates.

And what we mean by 'seen' and 'not seen' is that which is or is not
visible to the eye of man?

Yes, to the eye of man.

And is the soul seen or not seen?

Not seen.

Unseen then?

Yes.

Then the soul is more like to the unseen, and the body to the seen?

That follows necessarily, Socrates.

And were we not saying long ago that the soul when using the body as an
instrument of perception, that is to say, when using the sense of sight
or hearing or some other sense (for the meaning of perceiving through
the body is perceiving through the senses)--were we not saying that the
soul too is then dragged by the body into the region of the changeable,
and wanders and is confused; the world spins round her, and she is like
a drunkard, when she touches change?

Very true.

But when returning into herself she reflects, then she passes into the
other world, the region of purity, and eternity, and immortality, and
unchangeableness, which are her kindred, and with them she ever lives,
when she is by herself and is not let or hindered; then she ceases
from her erring ways, and being in communion with the unchanging is
unchanging. And this state of the soul is called wisdom?

That is well and truly said, Socrates, he replied.

And to which class is the soul more nearly alike and akin, as far as may
be inferred from this argument, as well as from the preceding one?

I think, Socrates, that, in the opinion of every one who follows the
argument, the soul will be infinitely more like the unchangeable--even
the most stupid person will not deny that.

And the body is more like the changing?

Yes.

Yet once more consider the matter in another light: When the soul and
the body are united, then nature orders the soul to rule and govern, and
the body to obey and serve. Now which of these two functions is akin to
the divine? and which to the mortal? Does not the divine appear to you
to be that which naturally orders and rules, and the mortal to be that
which is subject and servant?

True.

And which does the soul resemble?

The soul resembles the divine, and the body the mortal--there can be no
doubt of that, Socrates.

Then reflect, Cebes: of all which has been said is not this the
conclusion?--that the soul is in the very likeness of the divine,
and immortal, and intellectual, and uniform, and indissoluble, and
unchangeable; and that the body is in the very likeness of the human,
and mortal, and unintellectual, and multiform, and dissoluble, and
changeable. Can this, my dear Cebes, be denied?

It cannot.

But if it be true, then is not the body liable to speedy dissolution?
and is not the soul almost or altogether indissoluble?

Certainly.

And do you further observe, that after a man is dead, the body, or
visible part of him, which is lying in the visible world, and is
called a corpse, and would naturally be dissolved and decomposed and
dissipated, is not dissolved or decomposed at once, but may remain for a
for some time, nay even for a long time, if the constitution be sound at
the time of death, and the season of the year favourable? For the body
when shrunk and embalmed, as the manner is in Egypt, may remain almost
entire through infinite ages; and even in decay, there are still
some portions, such as the bones and ligaments, which are practically
indestructible:--Do you agree?

Yes.

And is it likely that the soul, which is invisible, in passing to the
place of the true Hades, which like her is invisible, and pure, and
noble, and on her way to the good and wise God, whither, if God will, my
soul is also soon to go,--that the soul, I repeat, if this be her nature
and origin, will be blown away and destroyed immediately on quitting the
body, as the many say? That can never be, my dear Simmias and Cebes.
The truth rather is, that the soul which is pure at departing and draws
after her no bodily taint, having never voluntarily during life had
connection with the body, which she is ever avoiding, herself gathered
into herself;--and making such abstraction her perpetual study--which
means that she has been a true disciple of philosophy; and therefore
has in fact been always engaged in the practice of dying? For is not
philosophy the practice of death?--

Certainly--

That soul, I say, herself invisible, departs to the invisible world--to
the divine and immortal and rational: thither arriving, she is secure of
bliss and is released from the error and folly of men, their fears and
wild passions and all other human ills, and for ever dwells, as they say
of the initiated, in company with the gods (compare Apol.). Is not this
true, Cebes?

Yes, said Cebes, beyond a doubt.

But the soul which has been polluted, and is impure at the time of her
departure, and is the companion and servant of the body always, and is
in love with and fascinated by the body and by the desires and pleasures
of the body, until she is led to believe that the truth only exists in
a bodily form, which a man may touch and see and taste, and use for the
purposes of his lusts,--the soul, I mean, accustomed to hate and fear
and avoid the intellectual principle, which to the bodily eye is dark
and invisible, and can be attained only by philosophy;--do you suppose
that such a soul will depart pure and unalloyed?

Impossible, he replied.

She is held fast by the corporeal, which the continual association and
constant care of the body have wrought into her nature.

Very true.

And this corporeal element, my friend, is heavy and weighty and earthy,
and is that element of sight by which a soul is depressed and dragged
down again into the visible world, because she is afraid of the
invisible and of the world below--prowling about tombs and sepulchres,
near which, as they tell us, are seen certain ghostly apparitions
of souls which have not departed pure, but are cloyed with sight and
therefore visible.

(Compare Milton, Comus:--

     'But when lust,
     By unchaste looks, loose gestures, and foul talk,
     But most by lewd and lavish act of sin,
     Lets in defilement to the inward parts,
     The soul grows clotted by contagion,
     Imbodies, and imbrutes, till she quite lose,
     The divine property of her first being.
     Such are those thick and gloomy shadows damp
     Oft seen in charnel vaults and sepulchres,
     Lingering, and sitting by a new made grave,
     As loath to leave the body that it lov'd,
     And linked itself by carnal sensuality
     To a degenerate and degraded state.')

That is very likely, Socrates.

Yes, that is very likely, Cebes; and these must be the souls, not of the
good, but of the evil, which are compelled to wander about such places
in payment of the penalty of their former evil way of life; and they
continue to wander until through the craving after the corporeal which
never leaves them, they are imprisoned finally in another body. And they
may be supposed to find their prisons in the same natures which they
have had in their former lives.

What natures do you mean, Socrates?

What I mean is that men who have followed after gluttony, and
wantonness, and drunkenness, and have had no thought of avoiding them,
would pass into asses and animals of that sort. What do you think?

I think such an opinion to be exceedingly probable.

And those who have chosen the portion of injustice, and tyranny, and
violence, will pass into wolves, or into hawks and kites;--whither else
can we suppose them to go?

Yes, said Cebes; with such natures, beyond question.

And there is no difficulty, he said, in assigning to all of them places
answering to their several natures and propensities?

There is not, he said.

Some are happier than others; and the happiest both in themselves and
in the place to which they go are those who have practised the civil and
social virtues which are called temperance and justice, and are acquired
by habit and attention without philosophy and mind. (Compare Republic.)

Why are they the happiest?

Because they may be expected to pass into some gentle and social kind
which is like their own, such as bees or wasps or ants, or back again
into the form of man, and just and moderate men may be supposed to
spring from them.

Very likely.

No one who has not studied philosophy and who is not entirely pure at
the time of his departure is allowed to enter the company of the Gods,
but the lover of knowledge only. And this is the reason, Simmias and
Cebes, why the true votaries of philosophy abstain from all fleshly
lusts, and hold out against them and refuse to give themselves up to
them,--not because they fear poverty or the ruin of their families, like
the lovers of money, and the world in general; nor like the lovers of
power and honour, because they dread the dishonour or disgrace of evil
deeds.

No, Socrates, that would not become them, said Cebes.

No indeed, he replied; and therefore they who have any care of their
own souls, and do not merely live moulding and fashioning the body, say
farewell to all this; they will not walk in the ways of the blind: and
when philosophy offers them purification and release from evil, they
feel that they ought not to resist her influence, and whither she leads
they turn and follow.

What do you mean, Socrates?

I will tell you, he said. The lovers of knowledge are conscious that
the soul was simply fastened and glued to the body--until philosophy
received her, she could only view real existence through the bars of
a prison, not in and through herself; she was wallowing in the mire of
every sort of ignorance; and by reason of lust had become the principal
accomplice in her own captivity. This was her original state; and
then, as I was saying, and as the lovers of knowledge are well aware,
philosophy, seeing how terrible was her confinement, of which she was
to herself the cause, received and gently comforted her and sought to
release her, pointing out that the eye and the ear and the other senses
are full of deception, and persuading her to retire from them, and
abstain from all but the necessary use of them, and be gathered up and
collected into herself, bidding her trust in herself and her own pure
apprehension of pure existence, and to mistrust whatever comes to her
through other channels and is subject to variation; for such things
are visible and tangible, but what she sees in her own nature is
intelligible and invisible. And the soul of the true philosopher thinks
that she ought not to resist this deliverance, and therefore abstains
from pleasures and desires and pains and fears, as far as she is
able; reflecting that when a man has great joys or sorrows or fears or
desires, he suffers from them, not merely the sort of evil which might
be anticipated--as for example, the loss of his health or property which
he has sacrificed to his lusts--but an evil greater far, which is the
greatest and worst of all evils, and one of which he never thinks.

What is it, Socrates? said Cebes.

The evil is that when the feeling of pleasure or pain is most intense,
every soul of man imagines the objects of this intense feeling to be
then plainest and truest: but this is not so, they are really the things
of sight.

Very true.

And is not this the state in which the soul is most enthralled by the
body?

How so?

Why, because each pleasure and pain is a sort of nail which nails
and rivets the soul to the body, until she becomes like the body, and
believes that to be true which the body affirms to be true; and from
agreeing with the body and having the same delights she is obliged to
have the same habits and haunts, and is not likely ever to be pure at
her departure to the world below, but is always infected by the body;
and so she sinks into another body and there germinates and grows,
and has therefore no part in the communion of the divine and pure and
simple.

Most true, Socrates, answered Cebes.

And this, Cebes, is the reason why the true lovers of knowledge are
temperate and brave; and not for the reason which the world gives.

Certainly not.

Certainly not! The soul of a philosopher will reason in quite another
way; she will not ask philosophy to release her in order that when
released she may deliver herself up again to the thraldom of pleasures
and pains, doing a work only to be undone again, weaving instead of
unweaving her Penelope's web. But she will calm passion, and follow
reason, and dwell in the contemplation of her, beholding the true
and divine (which is not matter of opinion), and thence deriving
nourishment. Thus she seeks to live while she lives, and after death she
hopes to go to her own kindred and to that which is like her, and to be
freed from human ills. Never fear, Simmias and Cebes, that a soul which
has been thus nurtured and has had these pursuits, will at her departure
from the body be scattered and blown away by the winds and be nowhere
and nothing.

When Socrates had done speaking, for a considerable time there was
silence; he himself appeared to be meditating, as most of us were, on
what had been said; only Cebes and Simmias spoke a few words to one
another. And Socrates observing them asked what they thought of the
argument, and whether there was anything wanting? For, said he, there
are many points still open to suspicion and attack, if any one were
disposed to sift the matter thoroughly. Should you be considering
some other matter I say no more, but if you are still in doubt do not
hesitate to say exactly what you think, and let us have anything better
which you can suggest; and if you think that I can be of any use, allow
me to help you.

Simmias said: I must confess, Socrates, that doubts did arise in our
minds, and each of us was urging and inciting the other to put the
question which we wanted to have answered and which neither of us liked
to ask, fearing that our importunity might be troublesome under present
at such a time.

Socrates replied with a smile: O Simmias, what are you saying? I am
not very likely to persuade other men that I do not regard my present
situation as a misfortune, if I cannot even persuade you that I am no
worse off now than at any other time in my life. Will you not allow that
I have as much of the spirit of prophecy in me as the swans? For they,
when they perceive that they must die, having sung all their life long,
do then sing more lustily than ever, rejoicing in the thought that
they are about to go away to the god whose ministers they are. But men,
because they are themselves afraid of death, slanderously affirm of the
swans that they sing a lament at the last, not considering that no bird
sings when cold, or hungry, or in pain, not even the nightingale, nor
the swallow, nor yet the hoopoe; which are said indeed to tune a lay of
sorrow, although I do not believe this to be true of them any more than
of the swans. But because they are sacred to Apollo, they have the gift
of prophecy, and anticipate the good things of another world, wherefore
they sing and rejoice in that day more than they ever did before. And I
too, believing myself to be the consecrated servant of the same God, and
the fellow-servant of the swans, and thinking that I have received from
my master gifts of prophecy which are not inferior to theirs, would not
go out of life less merrily than the swans. Never mind then, if this be
your only objection, but speak and ask anything which you like, while
the eleven magistrates of Athens allow.

Very good, Socrates, said Simmias; then I will tell you my difficulty,
and Cebes will tell you his. I feel myself, (and I daresay that you have
the same feeling), how hard or rather impossible is the attainment of
any certainty about questions such as these in the present life. And yet
I should deem him a coward who did not prove what is said about them to
the uttermost, or whose heart failed him before he had examined them
on every side. For he should persevere until he has achieved one of two
things: either he should discover, or be taught the truth about them;
or, if this be impossible, I would have him take the best and most
irrefragable of human theories, and let this be the raft upon which he
sails through life--not without risk, as I admit, if he cannot find some
word of God which will more surely and safely carry him. And now, as
you bid me, I will venture to question you, and then I shall not have to
reproach myself hereafter with not having said at the time what I think.
For when I consider the matter, either alone or with Cebes, the argument
does certainly appear to me, Socrates, to be not sufficient.

Socrates answered: I dare say, my friend, that you may be right, but I
should like to know in what respect the argument is insufficient.

In this respect, replied Simmias:--Suppose a person to use the same
argument about harmony and the lyre--might he not say that harmony is
a thing invisible, incorporeal, perfect, divine, existing in the lyre
which is harmonized, but that the lyre and the strings are matter and
material, composite, earthy, and akin to mortality? And when some one
breaks the lyre, or cuts and rends the strings, then he who takes this
view would argue as you do, and on the same analogy, that the harmony
survives and has not perished--you cannot imagine, he would say, that
the lyre without the strings, and the broken strings themselves which
are mortal remain, and yet that the harmony, which is of heavenly and
immortal nature and kindred, has perished--perished before the mortal.
The harmony must still be somewhere, and the wood and strings will decay
before anything can happen to that. The thought, Socrates, must have
occurred to your own mind that such is our conception of the soul;
and that when the body is in a manner strung and held together by the
elements of hot and cold, wet and dry, then the soul is the harmony or
due proportionate admixture of them. But if so, whenever the strings of
the body are unduly loosened or overstrained through disease or other
injury, then the soul, though most divine, like other harmonies of music
or of works of art, of course perishes at once, although the material
remains of the body may last for a considerable time, until they are
either decayed or burnt. And if any one maintains that the soul, being
the harmony of the elements of the body, is first to perish in that
which is called death, how shall we answer him?

Socrates looked fixedly at us as his manner was, and said with a smile:
Simmias has reason on his side; and why does not some one of you who
is better able than myself answer him? for there is force in his attack
upon me. But perhaps, before we answer him, we had better also hear what
Cebes has to say that we may gain time for reflection, and when they
have both spoken, we may either assent to them, if there is truth in
what they say, or if not, we will maintain our position. Please to tell
me then, Cebes, he said, what was the difficulty which troubled you?

Cebes said: I will tell you. My feeling is that the argument is where it
was, and open to the same objections which were urged before; for I am
ready to admit that the existence of the soul before entering into
the bodily form has been very ingeniously, and, if I may say so, quite
sufficiently proven; but the existence of the soul after death is still,
in my judgment, unproven. Now my objection is not the same as that of
Simmias; for I am not disposed to deny that the soul is stronger and
more lasting than the body, being of opinion that in all such respects
the soul very far excels the body. Well, then, says the argument to me,
why do you remain unconvinced?--When you see that the weaker continues
in existence after the man is dead, will you not admit that the more
lasting must also survive during the same period of time? Now I will
ask you to consider whether the objection, which, like Simmias, I will
express in a figure, is of any weight. The analogy which I will adduce
is that of an old weaver, who dies, and after his death somebody
says:--He is not dead, he must be alive;--see, there is the coat which
he himself wove and wore, and which remains whole and undecayed. And
then he proceeds to ask of some one who is incredulous, whether a man
lasts longer, or the coat which is in use and wear; and when he is
answered that a man lasts far longer, thinks that he has thus certainly
demonstrated the survival of the man, who is the more lasting, because
the less lasting remains. But that, Simmias, as I would beg you to
remark, is a mistake; any one can see that he who talks thus is talking
nonsense. For the truth is, that the weaver aforesaid, having woven and
worn many such coats, outlived several of them, and was outlived by the
last; but a man is not therefore proved to be slighter and weaker than
a coat. Now the relation of the body to the soul may be expressed in a
similar figure; and any one may very fairly say in like manner that the
soul is lasting, and the body weak and shortlived in comparison. He may
argue in like manner that every soul wears out many bodies, especially
if a man live many years. While he is alive the body deliquesces and
decays, and the soul always weaves another garment and repairs the
waste. But of course, whenever the soul perishes, she must have on her
last garment, and this will survive her; and then at length, when
the soul is dead, the body will show its native weakness, and quickly
decompose and pass away. I would therefore rather not rely on the
argument from superior strength to prove the continued existence of the
soul after death. For granting even more than you affirm to be possible,
and acknowledging not only that the soul existed before birth, but also
that the souls of some exist, and will continue to exist after death,
and will be born and die again and again, and that there is a
natural strength in the soul which will hold out and be born many
times--nevertheless, we may be still inclined to think that she will
weary in the labours of successive births, and may at last succumb in
one of her deaths and utterly perish; and this death and dissolution of
the body which brings destruction to the soul may be unknown to any of
us, for no one of us can have had any experience of it: and if so,
then I maintain that he who is confident about death has but a foolish
confidence, unless he is able to prove that the soul is altogether
immortal and imperishable. But if he cannot prove the soul's
immortality, he who is about to die will always have reason to fear that
when the body is disunited, the soul also may utterly perish.

All of us, as we afterwards remarked to one another, had an unpleasant
feeling at hearing what they said. When we had been so firmly convinced
before, now to have our faith shaken seemed to introduce a confusion and
uncertainty, not only into the previous argument, but into any future
one; either we were incapable of forming a judgment, or there were no
grounds of belief.

ECHECRATES: There I feel with you--by heaven I do, Phaedo, and when you
were speaking, I was beginning to ask myself the same question: What
argument can I ever trust again? For what could be more convincing than
the argument of Socrates, which has now fallen into discredit? That
the soul is a harmony is a doctrine which has always had a wonderful
attraction for me, and, when mentioned, came back to me at once, as my
own original conviction. And now I must begin again and find another
argument which will assure me that when the man is dead the soul
survives. Tell me, I implore you, how did Socrates proceed? Did he
appear to share the unpleasant feeling which you mention? or did he
calmly meet the attack? And did he answer forcibly or feebly? Narrate
what passed as exactly as you can.

PHAEDO: Often, Echecrates, I have wondered at Socrates, but never more
than on that occasion. That he should be able to answer was nothing,
but what astonished me was, first, the gentle and pleasant and approving
manner in which he received the words of the young men, and then his
quick sense of the wound which had been inflicted by the argument, and
the readiness with which he healed it. He might be compared to a general
rallying his defeated and broken army, urging them to accompany him and
return to the field of argument.

ECHECRATES: What followed?

PHAEDO: You shall hear, for I was close to him on his right hand, seated
on a sort of stool, and he on a couch which was a good deal higher.
He stroked my head, and pressed the hair upon my neck--he had a way of
playing with my hair; and then he said: To-morrow, Phaedo, I suppose
that these fair locks of yours will be severed.

Yes, Socrates, I suppose that they will, I replied.

Not so, if you will take my advice.

What shall I do with them? I said.

To-day, he replied, and not to-morrow, if this argument dies and we
cannot bring it to life again, you and I will both shave our locks; and
if I were you, and the argument got away from me, and I could not hold
my ground against Simmias and Cebes, I would myself take an oath, like
the Argives, not to wear hair any more until I had renewed the conflict
and defeated them.

Yes, I said, but Heracles himself is said not to be a match for two.

Summon me then, he said, and I will be your Iolaus until the sun goes
down.

I summon you rather, I rejoined, not as Heracles summoning Iolaus, but
as Iolaus might summon Heracles.

That will do as well, he said. But first let us take care that we avoid
a danger.

Of what nature? I said.

Lest we become misologists, he replied, no worse thing can happen to a
man than this. For as there are misanthropists or haters of men, there
are also misologists or haters of ideas, and both spring from the same
cause, which is ignorance of the world. Misanthropy arises out of the
too great confidence of inexperience;--you trust a man and think him
altogether true and sound and faithful, and then in a little while he
turns out to be false and knavish; and then another and another, and
when this has happened several times to a man, especially when it
happens among those whom he deems to be his own most trusted and
familiar friends, and he has often quarreled with them, he at last hates
all men, and believes that no one has any good in him at all. You must
have observed this trait of character?

I have.

And is not the feeling discreditable? Is it not obvious that such an
one having to deal with other men, was clearly without any experience of
human nature; for experience would have taught him the true state of
the case, that few are the good and few the evil, and that the great
majority are in the interval between them.

What do you mean? I said.

I mean, he replied, as you might say of the very large and very small,
that nothing is more uncommon than a very large or very small man; and
this applies generally to all extremes, whether of great and small, or
swift and slow, or fair and foul, or black and white: and whether
the instances you select be men or dogs or anything else, few are the
extremes, but many are in the mean between them. Did you never observe
this?

Yes, I said, I have.

And do you not imagine, he said, that if there were a competition in
evil, the worst would be found to be very few?

Yes, that is very likely, I said.

Yes, that is very likely, he replied; although in this respect arguments
are unlike men--there I was led on by you to say more than I had
intended; but the point of comparison was, that when a simple man who
has no skill in dialectics believes an argument to be true which he
afterwards imagines to be false, whether really false or not, and
then another and another, he has no longer any faith left, and great
disputers, as you know, come to think at last that they have grown to be
the wisest of mankind; for they alone perceive the utter unsoundness and
instability of all arguments, or indeed, of all things, which, like the
currents in the Euripus, are going up and down in never-ceasing ebb and
flow.

That is quite true, I said.

Yes, Phaedo, he replied, and how melancholy, if there be such a thing as
truth or certainty or possibility of knowledge--that a man should have
lighted upon some argument or other which at first seemed true and then
turned out to be false, and instead of blaming himself and his own want
of wit, because he is annoyed, should at last be too glad to transfer
the blame from himself to arguments in general: and for ever afterwards
should hate and revile them, and lose truth and the knowledge of
realities.

Yes, indeed, I said; that is very melancholy.

Let us then, in the first place, he said, be careful of allowing or of
admitting into our souls the notion that there is no health or soundness
in any arguments at all. Rather say that we have not yet attained to
soundness in ourselves, and that we must struggle manfully and do our
best to gain health of mind--you and all other men having regard to the
whole of your future life, and I myself in the prospect of death. For at
this moment I am sensible that I have not the temper of a philosopher;
like the vulgar, I am only a partisan. Now the partisan, when he is
engaged in a dispute, cares nothing about the rights of the question,
but is anxious only to convince his hearers of his own assertions.
And the difference between him and me at the present moment is merely
this--that whereas he seeks to convince his hearers that what he says is
true, I am rather seeking to convince myself; to convince my hearers
is a secondary matter with me. And do but see how much I gain by the
argument. For if what I say is true, then I do well to be persuaded of
the truth, but if there be nothing after death, still, during the short
time that remains, I shall not distress my friends with lamentations,
and my ignorance will not last, but will die with me, and therefore
no harm will be done. This is the state of mind, Simmias and Cebes, in
which I approach the argument. And I would ask you to be thinking of
the truth and not of Socrates: agree with me, if I seem to you to be
speaking the truth; or if not, withstand me might and main, that I may
not deceive you as well as myself in my enthusiasm, and like the bee,
leave my sting in you before I die.

And now let us proceed, he said. And first of all let me be sure that
I have in my mind what you were saying. Simmias, if I remember rightly,
has fears and misgivings whether the soul, although a fairer and diviner
thing than the body, being as she is in the form of harmony, may not
perish first. On the other hand, Cebes appeared to grant that the soul
was more lasting than the body, but he said that no one could know
whether the soul, after having worn out many bodies, might not perish
herself and leave her last body behind her; and that this is death,
which is the destruction not of the body but of the soul, for in the
body the work of destruction is ever going on. Are not these, Simmias
and Cebes, the points which we have to consider?

They both agreed to this statement of them.

He proceeded: And did you deny the force of the whole preceding
argument, or of a part only?

Of a part only, they replied.

And what did you think, he said, of that part of the argument in which
we said that knowledge was recollection, and hence inferred that the
soul must have previously existed somewhere else before she was enclosed
in the body?

Cebes said that he had been wonderfully impressed by that part of the
argument, and that his conviction remained absolutely unshaken. Simmias
agreed, and added that he himself could hardly imagine the possibility
of his ever thinking differently.

But, rejoined Socrates, you will have to think differently, my Theban
friend, if you still maintain that harmony is a compound, and that the
soul is a harmony which is made out of strings set in the frame of the
body; for you will surely never allow yourself to say that a harmony is
prior to the elements which compose it.

Never, Socrates.

But do you not see that this is what you imply when you say that the
soul existed before she took the form and body of man, and was made up
of elements which as yet had no existence? For harmony is not like
the soul, as you suppose; but first the lyre, and the strings, and the
sounds exist in a state of discord, and then harmony is made last of
all, and perishes first. And how can such a notion of the soul as this
agree with the other?

Not at all, replied Simmias.

And yet, he said, there surely ought to be harmony in a discourse of
which harmony is the theme.

There ought, replied Simmias.

But there is no harmony, he said, in the two propositions that knowledge
is recollection, and that the soul is a harmony. Which of them will you
retain?

I think, he replied, that I have a much stronger faith, Socrates, in the
first of the two, which has been fully demonstrated to me, than in
the latter, which has not been demonstrated at all, but rests only on
probable and plausible grounds; and is therefore believed by the many. I
know too well that these arguments from probabilities are impostors, and
unless great caution is observed in the use of them, they are apt to
be deceptive--in geometry, and in other things too. But the doctrine of
knowledge and recollection has been proven to me on trustworthy grounds;
and the proof was that the soul must have existed before she came into
the body, because to her belongs the essence of which the very name
implies existence. Having, as I am convinced, rightly accepted this
conclusion, and on sufficient grounds, I must, as I suppose, cease to
argue or allow others to argue that the soul is a harmony.

Let me put the matter, Simmias, he said, in another point of view: Do
you imagine that a harmony or any other composition can be in a state
other than that of the elements, out of which it is compounded?

Certainly not.

Or do or suffer anything other than they do or suffer?

He agreed.

Then a harmony does not, properly speaking, lead the parts or elements
which make up the harmony, but only follows them.

He assented.

For harmony cannot possibly have any motion, or sound, or other quality
which is opposed to its parts.

That would be impossible, he replied.

And does not the nature of every harmony depend upon the manner in which
the elements are harmonized?

I do not understand you, he said.

I mean to say that a harmony admits of degrees, and is more of a
harmony, and more completely a harmony, when more truly and fully
harmonized, to any extent which is possible; and less of a harmony, and
less completely a harmony, when less truly and fully harmonized.

True.

But does the soul admit of degrees? or is one soul in the very least
degree more or less, or more or less completely, a soul than another?

Not in the least.

Yet surely of two souls, one is said to have intelligence and virtue,
and to be good, and the other to have folly and vice, and to be an evil
soul: and this is said truly?

Yes, truly.

But what will those who maintain the soul to be a harmony say of this
presence of virtue and vice in the soul?--will they say that here is
another harmony, and another discord, and that the virtuous soul is
harmonized, and herself being a harmony has another harmony within her,
and that the vicious soul is inharmonical and has no harmony within her?

I cannot tell, replied Simmias; but I suppose that something of the sort
would be asserted by those who say that the soul is a harmony.

And we have already admitted that no soul is more a soul than another;
which is equivalent to admitting that harmony is not more or less
harmony, or more or less completely a harmony?

Quite true.

And that which is not more or less a harmony is not more or less
harmonized?

True.

And that which is not more or less harmonized cannot have more or less
of harmony, but only an equal harmony?

Yes, an equal harmony.

Then one soul not being more or less absolutely a soul than another, is
not more or less harmonized?

Exactly.

And therefore has neither more nor less of discord, nor yet of harmony?

She has not.

And having neither more nor less of harmony or of discord, one soul
has no more vice or virtue than another, if vice be discord and virtue
harmony?

Not at all more.

Or speaking more correctly, Simmias, the soul, if she is a harmony, will
never have any vice; because a harmony, being absolutely a harmony, has
no part in the inharmonical.

No.

And therefore a soul which is absolutely a soul has no vice?

How can she have, if the previous argument holds?

Then, if all souls are equally by their nature souls, all souls of all
living creatures will be equally good?

I agree with you, Socrates, he said.

And can all this be true, think you? he said; for these are the
consequences which seem to follow from the assumption that the soul is a
harmony?

It cannot be true.

Once more, he said, what ruler is there of the elements of human nature
other than the soul, and especially the wise soul? Do you know of any?

Indeed, I do not.

And is the soul in agreement with the affections of the body? or is she
at variance with them? For example, when the body is hot and thirsty,
does not the soul incline us against drinking? and when the body
is hungry, against eating? And this is only one instance out of ten
thousand of the opposition of the soul to the things of the body.

Very true.

But we have already acknowledged that the soul, being a harmony, can
never utter a note at variance with the tensions and relaxations and
vibrations and other affections of the strings out of which she is
composed; she can only follow, she cannot lead them?

It must be so, he replied.

And yet do we not now discover the soul to be doing the exact
opposite--leading the elements of which she is believed to be composed;
almost always opposing and coercing them in all sorts of ways throughout
life, sometimes more violently with the pains of medicine and gymnastic;
then again more gently; now threatening, now admonishing the desires,
passions, fears, as if talking to a thing which is not herself, as Homer
in the Odyssee represents Odysseus doing in the words--

'He beat his breast, and thus reproached his heart: Endure, my heart;
far worse hast thou endured!'

Do you think that Homer wrote this under the idea that the soul is a
harmony capable of being led by the affections of the body, and not
rather of a nature which should lead and master them--herself a far
diviner thing than any harmony?

Yes, Socrates, I quite think so.

Then, my friend, we can never be right in saying that the soul is a
harmony, for we should contradict the divine Homer, and contradict
ourselves.

True, he said.

Thus much, said Socrates, of Harmonia, your Theban goddess, who has
graciously yielded to us; but what shall I say, Cebes, to her husband
Cadmus, and how shall I make peace with him?

I think that you will discover a way of propitiating him, said Cebes; I
am sure that you have put the argument with Harmonia in a manner that
I could never have expected. For when Simmias was mentioning his
difficulty, I quite imagined that no answer could be given to him, and
therefore I was surprised at finding that his argument could not sustain
the first onset of yours, and not impossibly the other, whom you call
Cadmus, may share a similar fate.

Nay, my good friend, said Socrates, let us not boast, lest some evil eye
should put to flight the word which I am about to speak. That, however,
may be left in the hands of those above, while I draw near in Homeric
fashion, and try the mettle of your words. Here lies the point:--You
want to have it proven to you that the soul is imperishable and
immortal, and the philosopher who is confident in death appears to you
to have but a vain and foolish confidence, if he believes that he will
fare better in the world below than one who has led another sort of
life, unless he can prove this; and you say that the demonstration of
the strength and divinity of the soul, and of her existence prior to our
becoming men, does not necessarily imply her immortality. Admitting the
soul to be longlived, and to have known and done much in a former state,
still she is not on that account immortal; and her entrance into
the human form may be a sort of disease which is the beginning of
dissolution, and may at last, after the toils of life are over, end in
that which is called death. And whether the soul enters into the body
once only or many times, does not, as you say, make any difference in
the fears of individuals. For any man, who is not devoid of sense,
must fear, if he has no knowledge and can give no account of the soul's
immortality. This, or something like this, I suspect to be your notion,
Cebes; and I designedly recur to it in order that nothing may escape us,
and that you may, if you wish, add or subtract anything.

But, said Cebes, as far as I see at present, I have nothing to add or
subtract: I mean what you say that I mean.

Socrates paused awhile, and seemed to be absorbed in reflection. At
length he said: You are raising a tremendous question, Cebes, involving
the whole nature of generation and corruption, about which, if you like,
I will give you my own experience; and if anything which I say is likely
to avail towards the solution of your difficulty you may make use of it.

I should very much like, said Cebes, to hear what you have to say.

Then I will tell you, said Socrates. When I was young, Cebes, I had a
prodigious desire to know that department of philosophy which is called
the investigation of nature; to know the causes of things, and why
a thing is and is created or destroyed appeared to me to be a lofty
profession; and I was always agitating myself with the consideration of
questions such as these:--Is the growth of animals the result of some
decay which the hot and cold principle contracts, as some have said? Is
the blood the element with which we think, or the air, or the fire? or
perhaps nothing of the kind--but the brain may be the originating
power of the perceptions of hearing and sight and smell, and memory
and opinion may come from them, and science may be based on memory and
opinion when they have attained fixity. And then I went on to examine
the corruption of them, and then to the things of heaven and earth, and
at last I concluded myself to be utterly and absolutely incapable
of these enquiries, as I will satisfactorily prove to you. For I was
fascinated by them to such a degree that my eyes grew blind to things
which I had seemed to myself, and also to others, to know quite well; I
forgot what I had before thought self-evident truths; e.g. such a fact
as that the growth of man is the result of eating and drinking; for when
by the digestion of food flesh is added to flesh and bone to bone, and
whenever there is an aggregation of congenial elements, the lesser
bulk becomes larger and the small man great. Was not that a reasonable
notion?

Yes, said Cebes, I think so.

Well; but let me tell you something more. There was a time when I
thought that I understood the meaning of greater and less pretty well;
and when I saw a great man standing by a little one, I fancied that one
was taller than the other by a head; or one horse would appear to
be greater than another horse: and still more clearly did I seem to
perceive that ten is two more than eight, and that two cubits are more
than one, because two is the double of one.

And what is now your notion of such matters? said Cebes.

I should be far enough from imagining, he replied, that I knew the cause
of any of them, by heaven I should; for I cannot satisfy myself that,
when one is added to one, the one to which the addition is made becomes
two, or that the two units added together make two by reason of the
addition. I cannot understand how, when separated from the other, each
of them was one and not two, and now, when they are brought together,
the mere juxtaposition or meeting of them should be the cause of their
becoming two: neither can I understand how the division of one is the
way to make two; for then a different cause would produce the same
effect,--as in the former instance the addition and juxtaposition of one
to one was the cause of two, in this the separation and subtraction of
one from the other would be the cause. Nor am I any longer satisfied
that I understand the reason why one or anything else is either
generated or destroyed or is at all, but I have in my mind some confused
notion of a new method, and can never admit the other.

Then I heard some one reading, as he said, from a book of Anaxagoras,
that mind was the disposer and cause of all, and I was delighted at this
notion, which appeared quite admirable, and I said to myself: If mind
is the disposer, mind will dispose all for the best, and put each
particular in the best place; and I argued that if any one desired to
find out the cause of the generation or destruction or existence of
anything, he must find out what state of being or doing or suffering was
best for that thing, and therefore a man had only to consider the best
for himself and others, and then he would also know the worse, since the
same science comprehended both. And I rejoiced to think that I had found
in Anaxagoras a teacher of the causes of existence such as I desired,
and I imagined that he would tell me first whether the earth is flat or
round; and whichever was true, he would proceed to explain the cause and
the necessity of this being so, and then he would teach me the nature of
the best and show that this was best; and if he said that the earth was
in the centre, he would further explain that this position was the best,
and I should be satisfied with the explanation given, and not want any
other sort of cause. And I thought that I would then go on and ask him
about the sun and moon and stars, and that he would explain to me their
comparative swiftness, and their returnings and various states, active
and passive, and how all of them were for the best. For I could not
imagine that when he spoke of mind as the disposer of them, he would
give any other account of their being as they are, except that this was
best; and I thought that when he had explained to me in detail the cause
of each and the cause of all, he would go on to explain to me what was
best for each and what was good for all. These hopes I would not have
sold for a large sum of money, and I seized the books and read them as
fast as I could in my eagerness to know the better and the worse.

What expectations I had formed, and how grievously was I disappointed!
As I proceeded, I found my philosopher altogether forsaking mind or any
other principle of order, but having recourse to air, and ether, and
water, and other eccentricities. I might compare him to a person who
began by maintaining generally that mind is the cause of the actions
of Socrates, but who, when he endeavoured to explain the causes of my
several actions in detail, went on to show that I sit here because my
body is made up of bones and muscles; and the bones, as he would say,
are hard and have joints which divide them, and the muscles are elastic,
and they cover the bones, which have also a covering or environment of
flesh and skin which contains them; and as the bones are lifted at their
joints by the contraction or relaxation of the muscles, I am able
to bend my limbs, and this is why I am sitting here in a curved
posture--that is what he would say, and he would have a similar
explanation of my talking to you, which he would attribute to sound, and
air, and hearing, and he would assign ten thousand other causes of the
same sort, forgetting to mention the true cause, which is, that the
Athenians have thought fit to condemn me, and accordingly I have thought
it better and more right to remain here and undergo my sentence; for
I am inclined to think that these muscles and bones of mine would have
gone off long ago to Megara or Boeotia--by the dog they would, if they
had been moved only by their own idea of what was best, and if I had not
chosen the better and nobler part, instead of playing truant and running
away, of enduring any punishment which the state inflicts. There is
surely a strange confusion of causes and conditions in all this. It may
be said, indeed, that without bones and muscles and the other parts
of the body I cannot execute my purposes. But to say that I do as I do
because of them, and that this is the way in which mind acts, and
not from the choice of the best, is a very careless and idle mode of
speaking. I wonder that they cannot distinguish the cause from the
condition, which the many, feeling about in the dark, are always
mistaking and misnaming. And thus one man makes a vortex all round and
steadies the earth by the heaven; another gives the air as a support to
the earth, which is a sort of broad trough. Any power which in arranging
them as they are arranges them for the best never enters into their
minds; and instead of finding any superior strength in it, they rather
expect to discover another Atlas of the world who is stronger and more
everlasting and more containing than the good;--of the obligatory and
containing power of the good they think nothing; and yet this is the
principle which I would fain learn if any one would teach me. But as I
have failed either to discover myself, or to learn of any one else,
the nature of the best, I will exhibit to you, if you like, what I have
found to be the second best mode of enquiring into the cause.

I should very much like to hear, he replied.

Socrates proceeded:--I thought that as I had failed in the contemplation
of true existence, I ought to be careful that I did not lose the eye of
my soul; as people may injure their bodily eye by observing and gazing
on the sun during an eclipse, unless they take the precaution of only
looking at the image reflected in the water, or in some similar medium.
So in my own case, I was afraid that my soul might be blinded altogether
if I looked at things with my eyes or tried to apprehend them by the
help of the senses. And I thought that I had better have recourse to the
world of mind and seek there the truth of existence. I dare say that
the simile is not perfect--for I am very far from admitting that he who
contemplates existences through the medium of thought, sees them only
'through a glass darkly,' any more than he who considers them in action
and operation. However, this was the method which I adopted: I first
assumed some principle which I judged to be the strongest, and then I
affirmed as true whatever seemed to agree with this, whether relating
to the cause or to anything else; and that which disagreed I regarded
as untrue. But I should like to explain my meaning more clearly, as I do
not think that you as yet understand me.

No indeed, replied Cebes, not very well.

There is nothing new, he said, in what I am about to tell you; but
only what I have been always and everywhere repeating in the previous
discussion and on other occasions: I want to show you the nature of that
cause which has occupied my thoughts. I shall have to go back to those
familiar words which are in the mouth of every one, and first of all
assume that there is an absolute beauty and goodness and greatness, and
the like; grant me this, and I hope to be able to show you the nature of
the cause, and to prove the immortality of the soul.

Cebes said: You may proceed at once with the proof, for I grant you
this.

Well, he said, then I should like to know whether you agree with me
in the next step; for I cannot help thinking, if there be anything
beautiful other than absolute beauty should there be such, that it can
be beautiful only in as far as it partakes of absolute beauty--and I
should say the same of everything. Do you agree in this notion of the
cause?

Yes, he said, I agree.

He proceeded: I know nothing and can understand nothing of any other of
those wise causes which are alleged; and if a person says to me that
the bloom of colour, or form, or any such thing is a source of beauty,
I leave all that, which is only confusing to me, and simply and singly,
and perhaps foolishly, hold and am assured in my own mind that nothing
makes a thing beautiful but the presence and participation of beauty in
whatever way or manner obtained; for as to the manner I am uncertain,
but I stoutly contend that by beauty all beautiful things become
beautiful. This appears to me to be the safest answer which I can give,
either to myself or to another, and to this I cling, in the persuasion
that this principle will never be overthrown, and that to myself or
to any one who asks the question, I may safely reply, That by beauty
beautiful things become beautiful. Do you not agree with me?

I do.

And that by greatness only great things become great and greater
greater, and by smallness the less become less?

True.

Then if a person were to remark that A is taller by a head than B, and
B less by a head than A, you would refuse to admit his statement, and
would stoutly contend that what you mean is only that the greater is
greater by, and by reason of, greatness, and the less is less only by,
and by reason of, smallness; and thus you would avoid the danger of
saying that the greater is greater and the less less by the measure of
the head, which is the same in both, and would also avoid the monstrous
absurdity of supposing that the greater man is greater by reason of the
head, which is small. You would be afraid to draw such an inference,
would you not?

Indeed, I should, said Cebes, laughing.

In like manner you would be afraid to say that ten exceeded eight by,
and by reason of, two; but would say by, and by reason of, number; or
you would say that two cubits exceed one cubit not by a half, but by
magnitude?-for there is the same liability to error in all these cases.

Very true, he said.

Again, would you not be cautious of affirming that the addition of
one to one, or the division of one, is the cause of two? And you would
loudly asseverate that you know of no way in which anything comes
into existence except by participation in its own proper essence,
and consequently, as far as you know, the only cause of two is
the participation in duality--this is the way to make two, and the
participation in one is the way to make one. You would say: I will let
alone puzzles of division and addition--wiser heads than mine may answer
them; inexperienced as I am, and ready to start, as the proverb says,
at my own shadow, I cannot afford to give up the sure ground of a
principle. And if any one assails you there, you would not mind him,
or answer him, until you had seen whether the consequences which follow
agree with one another or not, and when you are further required to give
an explanation of this principle, you would go on to assume a higher
principle, and a higher, until you found a resting-place in the best of
the higher; but you would not confuse the principle and the consequences
in your reasoning, like the Eristics--at least if you wanted to discover
real existence. Not that this confusion signifies to them, who never
care or think about the matter at all, for they have the wit to be well
pleased with themselves however great may be the turmoil of their ideas.
But you, if you are a philosopher, will certainly do as I say.

What you say is most true, said Simmias and Cebes, both speaking at
once.

ECHECRATES: Yes, Phaedo; and I do not wonder at their assenting. Any
one who has the least sense will acknowledge the wonderful clearness of
Socrates' reasoning.

PHAEDO: Certainly, Echecrates; and such was the feeling of the whole
company at the time.

ECHECRATES: Yes, and equally of ourselves, who were not of the company,
and are now listening to your recital. But what followed?

PHAEDO: After all this had been admitted, and they had that ideas exist,
and that other things participate in them and derive their names from
them, Socrates, if I remember rightly, said:--

This is your way of speaking; and yet when you say that Simmias is
greater than Socrates and less than Phaedo, do you not predicate of
Simmias both greatness and smallness?

Yes, I do.

But still you allow that Simmias does not really exceed Socrates, as
the words may seem to imply, because he is Simmias, but by reason of the
size which he has; just as Simmias does not exceed Socrates because he
is Simmias, any more than because Socrates is Socrates, but because he
has smallness when compared with the greatness of Simmias?

True.

And if Phaedo exceeds him in size, this is not because Phaedo is
Phaedo, but because Phaedo has greatness relatively to Simmias, who is
comparatively smaller?

That is true.

And therefore Simmias is said to be great, and is also said to be small,
because he is in a mean between them, exceeding the smallness of the one
by his greatness, and allowing the greatness of the other to exceed his
smallness. He added, laughing, I am speaking like a book, but I believe
that what I am saying is true.

Simmias assented.

I speak as I do because I want you to agree with me in thinking, not
only that absolute greatness will never be great and also small, but
that greatness in us or in the concrete will never admit the small or
admit of being exceeded: instead of this, one of two things will happen,
either the greater will fly or retire before the opposite, which is the
less, or at the approach of the less has already ceased to exist; but
will not, if allowing or admitting of smallness, be changed by that;
even as I, having received and admitted smallness when compared with
Simmias, remain just as I was, and am the same small person. And as the
idea of greatness cannot condescend ever to be or become small, in like
manner the smallness in us cannot be or become great; nor can any other
opposite which remains the same ever be or become its own opposite, but
either passes away or perishes in the change.

That, replied Cebes, is quite my notion.

Hereupon one of the company, though I do not exactly remember which of
them, said: In heaven's name, is not this the direct contrary of what
was admitted before--that out of the greater came the less and out of
the less the greater, and that opposites were simply generated from
opposites; but now this principle seems to be utterly denied.

Socrates inclined his head to the speaker and listened. I like your
courage, he said, in reminding us of this. But you do not observe that
there is a difference in the two cases. For then we were speaking of
opposites in the concrete, and now of the essential opposite which, as
is affirmed, neither in us nor in nature can ever be at variance with
itself: then, my friend, we were speaking of things in which opposites
are inherent and which are called after them, but now about the
opposites which are inherent in them and which give their name to them;
and these essential opposites will never, as we maintain, admit of
generation into or out of one another. At the same time, turning to
Cebes, he said: Are you at all disconcerted, Cebes, at our friend's
objection?

No, I do not feel so, said Cebes; and yet I cannot deny that I am often
disturbed by objections.

Then we are agreed after all, said Socrates, that the opposite will
never in any case be opposed to itself?

To that we are quite agreed, he replied.

Yet once more let me ask you to consider the question from another point
of view, and see whether you agree with me:--There is a thing which you
term heat, and another thing which you term cold?

Certainly.

But are they the same as fire and snow?

Most assuredly not.

Heat is a thing different from fire, and cold is not the same with snow?

Yes.

And yet you will surely admit, that when snow, as was before said, is
under the influence of heat, they will not remain snow and heat; but at
the advance of the heat, the snow will either retire or perish?

Very true, he replied.

And the fire too at the advance of the cold will either retire or
perish; and when the fire is under the influence of the cold, they will
not remain as before, fire and cold.

That is true, he said.

And in some cases the name of the idea is not only attached to the idea
in an eternal connection, but anything else which, not being the idea,
exists only in the form of the idea, may also lay claim to it. I will
try to make this clearer by an example:--The odd number is always called
by the name of odd?

Very true.

But is this the only thing which is called odd? Are there not other
things which have their own name, and yet are called odd, because,
although not the same as oddness, they are never without oddness?--that
is what I mean to ask--whether numbers such as the number three are not
of the class of odd. And there are many other examples: would you not
say, for example, that three may be called by its proper name, and also
be called odd, which is not the same with three? and this may be said
not only of three but also of five, and of every alternate number--each
of them without being oddness is odd, and in the same way two and
four, and the other series of alternate numbers, has every number even,
without being evenness. Do you agree?

Of course.

Then now mark the point at which I am aiming:--not only do essential
opposites exclude one another, but also concrete things, which, although
not in themselves opposed, contain opposites; these, I say, likewise
reject the idea which is opposed to that which is contained in them,
and when it approaches them they either perish or withdraw. For example;
Will not the number three endure annihilation or anything sooner than be
converted into an even number, while remaining three?

Very true, said Cebes.

And yet, he said, the number two is certainly not opposed to the number
three?

It is not.

Then not only do opposite ideas repel the advance of one another, but
also there are other natures which repel the approach of opposites.

Very true, he said.

Suppose, he said, that we endeavour, if possible, to determine what
these are.

By all means.

Are they not, Cebes, such as compel the things of which they have
possession, not only to take their own form, but also the form of some
opposite?

What do you mean?

I mean, as I was just now saying, and as I am sure that you know, that
those things which are possessed by the number three must not only be
three in number, but must also be odd.

Quite true.

And on this oddness, of which the number three has the impress, the
opposite idea will never intrude?

No.

And this impress was given by the odd principle?

Yes.

And to the odd is opposed the even?

True.

Then the idea of the even number will never arrive at three?

No.

Then three has no part in the even?

None.

Then the triad or number three is uneven?

Very true.

To return then to my distinction of natures which are not opposed, and
yet do not admit opposites--as, in the instance given, three, although
not opposed to the even, does not any the more admit of the even, but
always brings the opposite into play on the other side; or as two does
not receive the odd, or fire the cold--from these examples (and there
are many more of them) perhaps you may be able to arrive at the general
conclusion, that not only opposites will not receive opposites, but also
that nothing which brings the opposite will admit the opposite of
that which it brings, in that to which it is brought. And here let me
recapitulate--for there is no harm in repetition. The number five will
not admit the nature of the even, any more than ten, which is the
double of five, will admit the nature of the odd. The double has another
opposite, and is not strictly opposed to the odd, but nevertheless
rejects the odd altogether. Nor again will parts in the ratio 3:2, nor
any fraction in which there is a half, nor again in which there is a
third, admit the notion of the whole, although they are not opposed to
the whole: You will agree?

Yes, he said, I entirely agree and go along with you in that.

And now, he said, let us begin again; and do not you answer my question
in the words in which I ask it: let me have not the old safe answer of
which I spoke at first, but another equally safe, of which the truth
will be inferred by you from what has been just said. I mean that if any
one asks you 'what that is, of which the inherence makes the body
hot,' you will reply not heat (this is what I call the safe and
stupid answer), but fire, a far superior answer, which we are now in a
condition to give. Or if any one asks you 'why a body is diseased,' you
will not say from disease, but from fever; and instead of saying that
oddness is the cause of odd numbers, you will say that the monad is the
cause of them: and so of things in general, as I dare say that you will
understand sufficiently without my adducing any further examples.

Yes, he said, I quite understand you.

Tell me, then, what is that of which the inherence will render the body
alive?

The soul, he replied.

And is this always the case?

Yes, he said, of course.

Then whatever the soul possesses, to that she comes bearing life?

Yes, certainly.

And is there any opposite to life?

There is, he said.

And what is that?

Death.

Then the soul, as has been acknowledged, will never receive the opposite
of what she brings.

Impossible, replied Cebes.

And now, he said, what did we just now call that principle which repels
the even?

The odd.

And that principle which repels the musical, or the just?

The unmusical, he said, and the unjust.

And what do we call the principle which does not admit of death?

The immortal, he said.

And does the soul admit of death?

No.

Then the soul is immortal?

Yes, he said.

And may we say that this has been proven?

Yes, abundantly proven, Socrates, he replied.

Supposing that the odd were imperishable, must not three be
imperishable?

Of course.

And if that which is cold were imperishable, when the warm principle
came attacking the snow, must not the snow have retired whole and
unmelted--for it could never have perished, nor could it have remained
and admitted the heat?

True, he said.

Again, if the uncooling or warm principle were imperishable, the fire
when assailed by cold would not have perished or have been extinguished,
but would have gone away unaffected?

Certainly, he said.

And the same may be said of the immortal: if the immortal is also
imperishable, the soul when attacked by death cannot perish; for the
preceding argument shows that the soul will not admit of death, or ever
be dead, any more than three or the odd number will admit of the even,
or fire or the heat in the fire, of the cold. Yet a person may say: 'But
although the odd will not become even at the approach of the even, why
may not the odd perish and the even take the place of the odd?' Now to
him who makes this objection, we cannot answer that the odd principle is
imperishable; for this has not been acknowledged, but if this had been
acknowledged, there would have been no difficulty in contending that
at the approach of the even the odd principle and the number three took
their departure; and the same argument would have held good of fire and
heat and any other thing.

Very true.

And the same may be said of the immortal: if the immortal is also
imperishable, then the soul will be imperishable as well as immortal;
but if not, some other proof of her imperishableness will have to be
given.

No other proof is needed, he said; for if the immortal, being eternal,
is liable to perish, then nothing is imperishable.

Yes, replied Socrates, and yet all men will agree that God, and the
essential form of life, and the immortal in general, will never perish.

Yes, all men, he said--that is true; and what is more, gods, if I am not
mistaken, as well as men.

Seeing then that the immortal is indestructible, must not the soul, if
she is immortal, be also imperishable?

Most certainly.

Then when death attacks a man, the mortal portion of him may be supposed
to die, but the immortal retires at the approach of death and is
preserved safe and sound?

True.

Then, Cebes, beyond question, the soul is immortal and imperishable, and
our souls will truly exist in another world!

I am convinced, Socrates, said Cebes, and have nothing more to object;
but if my friend Simmias, or any one else, has any further objection to
make, he had better speak out, and not keep silence, since I do not know
to what other season he can defer the discussion, if there is anything
which he wants to say or to have said.

But I have nothing more to say, replied Simmias; nor can I see any
reason for doubt after what has been said. But I still feel and cannot
help feeling uncertain in my own mind, when I think of the greatness of
the subject and the feebleness of man.

Yes, Simmias, replied Socrates, that is well said: and I may add that
first principles, even if they appear certain, should be carefully
considered; and when they are satisfactorily ascertained, then, with a
sort of hesitating confidence in human reason, you may, I think, follow
the course of the argument; and if that be plain and clear, there will
be no need for any further enquiry.

Very true.

But then, O my friends, he said, if the soul is really immortal, what
care should be taken of her, not only in respect of the portion of time
which is called life, but of eternity! And the danger of neglecting her
from this point of view does indeed appear to be awful. If death had
only been the end of all, the wicked would have had a good bargain in
dying, for they would have been happily quit not only of their body, but
of their own evil together with their souls. But now, inasmuch as the
soul is manifestly immortal, there is no release or salvation from evil
except the attainment of the highest virtue and wisdom. For the soul
when on her progress to the world below takes nothing with her but
nurture and education; and these are said greatly to benefit or greatly
to injure the departed, at the very beginning of his journey thither.

For after death, as they say, the genius of each individual, to whom
he belonged in life, leads him to a certain place in which the dead are
gathered together, whence after judgment has been given they pass into
the world below, following the guide, who is appointed to conduct them
from this world to the other: and when they have there received their
due and remained their time, another guide brings them back again after
many revolutions of ages. Now this way to the other world is not, as
Aeschylus says in the Telephus, a single and straight path--if that were
so no guide would be needed, for no one could miss it; but there are
many partings of the road, and windings, as I infer from the rites and
sacrifices which are offered to the gods below in places where three
ways meet on earth. The wise and orderly soul follows in the straight
path and is conscious of her surroundings; but the soul which desires
the body, and which, as I was relating before, has long been fluttering
about the lifeless frame and the world of sight, is after many struggles
and many sufferings hardly and with violence carried away by her
attendant genius, and when she arrives at the place where the other
souls are gathered, if she be impure and have done impure deeds, whether
foul murders or other crimes which are the brothers of these, and the
works of brothers in crime--from that soul every one flees and turns
away; no one will be her companion, no one her guide, but alone she
wanders in extremity of evil until certain times are fulfilled, and
when they are fulfilled, she is borne irresistibly to her own fitting
habitation; as every pure and just soul which has passed through life in
the company and under the guidance of the gods has also her own proper
home.

Now the earth has divers wonderful regions, and is indeed in nature
and extent very unlike the notions of geographers, as I believe on the
authority of one who shall be nameless.

What do you mean, Socrates? said Simmias. I have myself heard many
descriptions of the earth, but I do not know, and I should very much
like to know, in which of these you put faith.

And I, Simmias, replied Socrates, if I had the art of Glaucus would tell
you; although I know not that the art of Glaucus could prove the truth
of my tale, which I myself should never be able to prove, and even if
I could, I fear, Simmias, that my life would come to an end before the
argument was completed. I may describe to you, however, the form and
regions of the earth according to my conception of them.

That, said Simmias, will be enough.

Well, then, he said, my conviction is, that the earth is a round body
in the centre of the heavens, and therefore has no need of air or any
similar force to be a support, but is kept there and hindered from
falling or inclining any way by the equability of the surrounding heaven
and by her own equipoise. For that which, being in equipoise, is in the
centre of that which is equably diffused, will not incline any way in
any degree, but will always remain in the same state and not deviate.
And this is my first notion.

Which is surely a correct one, said Simmias.

Also I believe that the earth is very vast, and that we who dwell in
the region extending from the river Phasis to the Pillars of Heracles
inhabit a small portion only about the sea, like ants or frogs about a
marsh, and that there are other inhabitants of many other like places;
for everywhere on the face of the earth there are hollows of various
forms and sizes, into which the water and the mist and the lower
air collect. But the true earth is pure and situated in the pure
heaven--there are the stars also; and it is the heaven which is commonly
spoken of by us as the ether, and of which our own earth is the sediment
gathering in the hollows beneath. But we who live in these hollows are
deceived into the notion that we are dwelling above on the surface of
the earth; which is just as if a creature who was at the bottom of the
sea were to fancy that he was on the surface of the water, and that the
sea was the heaven through which he saw the sun and the other stars,
he having never come to the surface by reason of his feebleness and
sluggishness, and having never lifted up his head and seen, nor ever
heard from one who had seen, how much purer and fairer the world above
is than his own. And such is exactly our case: for we are dwelling in a
hollow of the earth, and fancy that we are on the surface; and the air
we call the heaven, in which we imagine that the stars move. But the
fact is, that owing to our feebleness and sluggishness we are prevented
from reaching the surface of the air: for if any man could arrive at the
exterior limit, or take the wings of a bird and come to the top, then
like a fish who puts his head out of the water and sees this world, he
would see a world beyond; and, if the nature of man could sustain the
sight, he would acknowledge that this other world was the place of the
true heaven and the true light and the true earth. For our earth, and
the stones, and the entire region which surrounds us, are spoilt and
corroded, as in the sea all things are corroded by the brine, neither
is there any noble or perfect growth, but caverns only, and sand, and an
endless slough of mud: and even the shore is not to be compared to the
fairer sights of this world. And still less is this our world to be
compared with the other. Of that upper earth which is under the heaven,
I can tell you a charming tale, Simmias, which is well worth hearing.

And we, Socrates, replied Simmias, shall be charmed to listen to you.

The tale, my friend, he said, is as follows:--In the first place, the
earth, when looked at from above, is in appearance streaked like one of
those balls which have leather coverings in twelve pieces, and is decked
with various colours, of which the colours used by painters on earth are
in a manner samples. But there the whole earth is made up of them,
and they are brighter far and clearer than ours; there is a purple of
wonderful lustre, also the radiance of gold, and the white which is in
the earth is whiter than any chalk or snow. Of these and other colours
the earth is made up, and they are more in number and fairer than the
eye of man has ever seen; the very hollows (of which I was speaking)
filled with air and water have a colour of their own, and are seen like
light gleaming amid the diversity of the other colours, so that the
whole presents a single and continuous appearance of variety in unity.
And in this fair region everything that grows--trees, and flowers, and
fruits--are in a like degree fairer than any here; and there are hills,
having stones in them in a like degree smoother, and more transparent,
and fairer in colour than our highly-valued emeralds and sardonyxes and
jaspers, and other gems, which are but minute fragments of them: for
there all the stones are like our precious stones, and fairer still
(compare Republic). The reason is, that they are pure, and not, like
our precious stones, infected or corroded by the corrupt briny elements
which coagulate among us, and which breed foulness and disease both in
earth and stones, as well as in animals and plants. They are the jewels
of the upper earth, which also shines with gold and silver and the like,
and they are set in the light of day and are large and abundant and in
all places, making the earth a sight to gladden the beholder's eye.
And there are animals and men, some in a middle region, others dwelling
about the air as we dwell about the sea; others in islands which the air
flows round, near the continent: and in a word, the air is used by them
as the water and the sea are by us, and the ether is to them what the
air is to us. Moreover, the temperament of their seasons is such that
they have no disease, and live much longer than we do, and have
sight and hearing and smell, and all the other senses, in far greater
perfection, in the same proportion that air is purer than water or the
ether than air. Also they have temples and sacred places in which the
gods really dwell, and they hear their voices and receive their answers,
and are conscious of them and hold converse with them, and they see the
sun, moon, and stars as they truly are, and their other blessedness is
of a piece with this.

Such is the nature of the whole earth, and of the things which are
around the earth; and there are divers regions in the hollows on the
face of the globe everywhere, some of them deeper and more extended than
that which we inhabit, others deeper but with a narrower opening
than ours, and some are shallower and also wider. All have numerous
perforations, and there are passages broad and narrow in the interior of
the earth, connecting them with one another; and there flows out of and
into them, as into basins, a vast tide of water, and huge subterranean
streams of perennial rivers, and springs hot and cold, and a great fire,
and great rivers of fire, and streams of liquid mud, thin or thick (like
the rivers of mud in Sicily, and the lava streams which follow them),
and the regions about which they happen to flow are filled up with them.
And there is a swinging or see-saw in the interior of the earth which
moves all this up and down, and is due to the following cause:--There is
a chasm which is the vastest of them all, and pierces right through the
whole earth; this is that chasm which Homer describes in the words,--

     'Far off, where is the inmost depth beneath the earth;'

and which he in other places, and many other poets, have called
Tartarus. And the see-saw is caused by the streams flowing into and out
of this chasm, and they each have the nature of the soil through which
they flow. And the reason why the streams are always flowing in and out,
is that the watery element has no bed or bottom, but is swinging and
surging up and down, and the surrounding wind and air do the same; they
follow the water up and down, hither and thither, over the earth--just
as in the act of respiration the air is always in process of inhalation
and exhalation;--and the wind swinging with the water in and out
produces fearful and irresistible blasts: when the waters retire with
a rush into the lower parts of the earth, as they are called, they flow
through the earth in those regions, and fill them up like water raised
by a pump, and then when they leave those regions and rush back hither,
they again fill the hollows here, and when these are filled, flow
through subterranean channels and find their way to their several
places, forming seas, and lakes, and rivers, and springs. Thence they
again enter the earth, some of them making a long circuit into many
lands, others going to a few places and not so distant; and again fall
into Tartarus, some at a point a good deal lower than that at which they
rose, and others not much lower, but all in some degree lower than the
point from which they came. And some burst forth again on the opposite
side, and some on the same side, and some wind round the earth with one
or many folds like the coils of a serpent, and descend as far as they
can, but always return and fall into the chasm. The rivers flowing in
either direction can descend only to the centre and no further, for
opposite to the rivers is a precipice.

Now these rivers are many, and mighty, and diverse, and there are four
principal ones, of which the greatest and outermost is that called
Oceanus, which flows round the earth in a circle; and in the opposite
direction flows Acheron, which passes under the earth through desert
places into the Acherusian lake: this is the lake to the shores of
which the souls of the many go when they are dead, and after waiting an
appointed time, which is to some a longer and to some a shorter time,
they are sent back to be born again as animals. The third river passes
out between the two, and near the place of outlet pours into a vast
region of fire, and forms a lake larger than the Mediterranean Sea,
boiling with water and mud; and proceeding muddy and turbid, and winding
about the earth, comes, among other places, to the extremities of the
Acherusian Lake, but mingles not with the waters of the lake, and after
making many coils about the earth plunges into Tartarus at a deeper
level. This is that Pyriphlegethon, as the stream is called, which
throws up jets of fire in different parts of the earth. The fourth river
goes out on the opposite side, and falls first of all into a wild and
savage region, which is all of a dark-blue colour, like lapis lazuli;
and this is that river which is called the Stygian river, and falls into
and forms the Lake Styx, and after falling into the lake and receiving
strange powers in the waters, passes under the earth, winding round
in the opposite direction, and comes near the Acherusian lake from the
opposite side to Pyriphlegethon. And the water of this river too mingles
with no other, but flows round in a circle and falls into Tartarus over
against Pyriphlegethon; and the name of the river, as the poets say, is
Cocytus.

Such is the nature of the other world; and when the dead arrive at the
place to which the genius of each severally guides them, first of all,
they have sentence passed upon them, as they have lived well and piously
or not. And those who appear to have lived neither well nor ill, go to
the river Acheron, and embarking in any vessels which they may find, are
carried in them to the lake, and there they dwell and are purified of
their evil deeds, and having suffered the penalty of the wrongs which
they have done to others, they are absolved, and receive the rewards of
their good deeds, each of them according to his deserts. But those who
appear to be incurable by reason of the greatness of their crimes--who
have committed many and terrible deeds of sacrilege, murders foul and
violent, or the like--such are hurled into Tartarus which is their
suitable destiny, and they never come out. Those again who have
committed crimes, which, although great, are not irremediable--who in
a moment of anger, for example, have done violence to a father or a
mother, and have repented for the remainder of their lives, or, who
have taken the life of another under the like extenuating
circumstances--these are plunged into Tartarus, the pains of which they
are compelled to undergo for a year, but at the end of the year the
wave casts them forth--mere homicides by way of Cocytus, parricides and
matricides by Pyriphlegethon--and they are borne to the Acherusian lake,
and there they lift up their voices and call upon the victims whom they
have slain or wronged, to have pity on them, and to be kind to them,
and let them come out into the lake. And if they prevail, then they come
forth and cease from their troubles; but if not, they are carried back
again into Tartarus and from thence into the rivers unceasingly, until
they obtain mercy from those whom they have wronged: for that is the
sentence inflicted upon them by their judges. Those too who have been
pre-eminent for holiness of life are released from this earthly prison,
and go to their pure home which is above, and dwell in the purer earth;
and of these, such as have duly purified themselves with philosophy live
henceforth altogether without the body, in mansions fairer still which
may not be described, and of which the time would fail me to tell.

Wherefore, Simmias, seeing all these things, what ought not we to do
that we may obtain virtue and wisdom in this life? Fair is the prize,
and the hope great!

A man of sense ought not to say, nor will I be very confident, that the
description which I have given of the soul and her mansions is exactly
true. But I do say that, inasmuch as the soul is shown to be immortal,
he may venture to think, not improperly or unworthily, that something of
the kind is true. The venture is a glorious one, and he ought to comfort
himself with words like these, which is the reason why I lengthen out
the tale. Wherefore, I say, let a man be of good cheer about his soul,
who having cast away the pleasures and ornaments of the body as alien to
him and working harm rather than good, has sought after the pleasures of
knowledge; and has arrayed the soul, not in some foreign attire, but
in her own proper jewels, temperance, and justice, and courage, and
nobility, and truth--in these adorned she is ready to go on her journey
to the world below, when her hour comes. You, Simmias and Cebes, and all
other men, will depart at some time or other. Me already, as the tragic
poet would say, the voice of fate calls. Soon I must drink the poison;
and I think that I had better repair to the bath first, in order that
the women may not have the trouble of washing my body after I am dead.

When he had done speaking, Crito said: And have you any commands for us,
Socrates--anything to say about your children, or any other matter in
which we can serve you?

Nothing particular, Crito, he replied: only, as I have always told
you, take care of yourselves; that is a service which you may be ever
rendering to me and mine and to all of us, whether you promise to do so
or not. But if you have no thought for yourselves, and care not to walk
according to the rule which I have prescribed for you, not now for the
first time, however much you may profess or promise at the moment, it
will be of no avail.

We will do our best, said Crito: And in what way shall we bury you?

In any way that you like; but you must get hold of me, and take care
that I do not run away from you. Then he turned to us, and added with a
smile:--I cannot make Crito believe that I am the same Socrates who have
been talking and conducting the argument; he fancies that I am the other
Socrates whom he will soon see, a dead body--and he asks, How shall he
bury me? And though I have spoken many words in the endeavour to show
that when I have drunk the poison I shall leave you and go to the joys
of the blessed,--these words of mine, with which I was comforting you
and myself, have had, as I perceive, no effect upon Crito. And therefore
I want you to be surety for me to him now, as at the trial he was surety
to the judges for me: but let the promise be of another sort; for he
was surety for me to the judges that I would remain, and you must be my
surety to him that I shall not remain, but go away and depart; and then
he will suffer less at my death, and not be grieved when he sees my body
being burned or buried. I would not have him sorrow at my hard lot, or
say at the burial, Thus we lay out Socrates, or, Thus we follow him to
the grave or bury him; for false words are not only evil in themselves,
but they infect the soul with evil. Be of good cheer, then, my dear
Crito, and say that you are burying my body only, and do with that
whatever is usual, and what you think best.

When he had spoken these words, he arose and went into a chamber to
bathe; Crito followed him and told us to wait. So we remained behind,
talking and thinking of the subject of discourse, and also of the
greatness of our sorrow; he was like a father of whom we were being
bereaved, and we were about to pass the rest of our lives as orphans.
When he had taken the bath his children were brought to him--(he had two
young sons and an elder one); and the women of his family also came,
and he talked to them and gave them a few directions in the presence of
Crito; then he dismissed them and returned to us.

Now the hour of sunset was near, for a good deal of time had passed
while he was within. When he came out, he sat down with us again after
his bath, but not much was said. Soon the jailer, who was the servant of
the Eleven, entered and stood by him, saying:--To you, Socrates, whom
I know to be the noblest and gentlest and best of all who ever came to
this place, I will not impute the angry feelings of other men, who rage
and swear at me, when, in obedience to the authorities, I bid them drink
the poison--indeed, I am sure that you will not be angry with me; for
others, as you are aware, and not I, are to blame. And so fare you well,
and try to bear lightly what must needs be--you know my errand. Then
bursting into tears he turned away and went out.

Socrates looked at him and said: I return your good wishes, and will do
as you bid. Then turning to us, he said, How charming the man is: since
I have been in prison he has always been coming to see me, and at times
he would talk to me, and was as good to me as could be, and now see how
generously he sorrows on my account. We must do as he says, Crito; and
therefore let the cup be brought, if the poison is prepared: if not, let
the attendant prepare some.

Yet, said Crito, the sun is still upon the hill-tops, and I know that
many a one has taken the draught late, and after the announcement has
been made to him, he has eaten and drunk, and enjoyed the society of his
beloved; do not hurry--there is time enough.

Socrates said: Yes, Crito, and they of whom you speak are right in so
acting, for they think that they will be gainers by the delay; but I am
right in not following their example, for I do not think that I should
gain anything by drinking the poison a little later; I should only be
ridiculous in my own eyes for sparing and saving a life which is already
forfeit. Please then to do as I say, and not to refuse me.

Crito made a sign to the servant, who was standing by; and he went out,
and having been absent for some time, returned with the jailer
carrying the cup of poison. Socrates said: You, my good friend, who
are experienced in these matters, shall give me directions how I am to
proceed. The man answered: You have only to walk about until your legs
are heavy, and then to lie down, and the poison will act. At the same
time he handed the cup to Socrates, who in the easiest and gentlest
manner, without the least fear or change of colour or feature, looking
at the man with all his eyes, Echecrates, as his manner was, took the
cup and said: What do you say about making a libation out of this cup
to any god? May I, or not? The man answered: We only prepare, Socrates,
just so much as we deem enough. I understand, he said: but I may
and must ask the gods to prosper my journey from this to the other
world--even so--and so be it according to my prayer. Then raising the
cup to his lips, quite readily and cheerfully he drank off the poison.
And hitherto most of us had been able to control our sorrow; but now
when we saw him drinking, and saw too that he had finished the draught,
we could no longer forbear, and in spite of myself my own tears were
flowing fast; so that I covered my face and wept, not for him, but at
the thought of my own calamity in having to part from such a friend. Nor
was I the first; for Crito, when he found himself unable to restrain his
tears, had got up, and I followed; and at that moment, Apollodorus, who
had been weeping all the time, broke out in a loud and passionate cry
which made cowards of us all. Socrates alone retained his calmness: What
is this strange outcry? he said. I sent away the women mainly in order
that they might not misbehave in this way, for I have been told that
a man should die in peace. Be quiet, then, and have patience. When we
heard his words we were ashamed, and refrained our tears; and he walked
about until, as he said, his legs began to fail, and then he lay on his
back, according to the directions, and the man who gave him the poison
now and then looked at his feet and legs; and after a while he pressed
his foot hard, and asked him if he could feel; and he said, No; and then
his leg, and so upwards and upwards, and showed us that he was cold and
stiff. And he felt them himself, and said: When the poison reaches the
heart, that will be the end. He was beginning to grow cold about the
groin, when he uncovered his face, for he had covered himself up,
and said--they were his last words--he said: Crito, I owe a cock to
Asclepius; will you remember to pay the debt? The debt shall be
paid, said Crito; is there anything else? There was no answer to
this question; but in a minute or two a movement was heard, and the
attendants uncovered him; his eyes were set, and Crito closed his eyes
and mouth.

Such was the end, Echecrates, of our friend; concerning whom I may
truly say, that of all the men of his time whom I have known, he was the
wisest and justest and best.


% chapter phaedo (end)
%!TEX root = ancient_text.tex

\chapter{Republic} % (fold)
\label{cha:republic}





THE REPUBLIC

By Plato


Translated by Benjamin Jowett



THE REPUBLIC.




PERSONS OF THE DIALOGUE.

Socrates, who is the narrator.

Glaucon.

Adeimantus.

Polemarchus.

Cephalus.

Thrasymachus.

Cleitophon.

And others who are mute auditors.

The scene is laid in the house of Cephalus at the Piraeus; and the whole
dialogue is narrated by Socrates the day after it actually took place to
Timaeus, Hermocrates, Critias, and a nameless person, who are introduced
in the Timaeus.


\section{Book I} % (fold)
\label{sec:book_i}



BOOK I.

I went down yesterday to the Piraeus with Glaucon the son of Ariston,
that I might offer up my prayers to the goddess (Bendis, the Thracian
Artemis.); and also because I wanted to see in what manner they would
celebrate the festival, which was a new thing. I was delighted with the
procession of the inhabitants; but that of the Thracians was equally,
if not more, beautiful. When we had finished our prayers and viewed the
spectacle, we turned in the direction of the city; and at that instant
Polemarchus the son of Cephalus chanced to catch sight of us from a
distance as we were starting on our way home, and told his servant to
run and bid us wait for him. The servant took hold of me by the cloak
behind, and said: Polemarchus desires you to wait.

I turned round, and asked him where his master was.

There he is, said the youth, coming after you, if you will only wait.

Certainly we will, said Glaucon; and in a few minutes Polemarchus
appeared, and with him Adeimantus, Glaucon's brother, Niceratus the son
of Nicias, and several others who had been at the procession.

Polemarchus said to me: I perceive, Socrates, that you and your
companion are already on your way to the city.

You are not far wrong, I said.

But do you see, he rejoined, how many we are?

Of course.

And are you stronger than all these? for if not, you will have to remain
where you are.

May there not be the alternative, I said, that we may persuade you to
let us go?

But can you persuade us, if we refuse to listen to you? he said.

Certainly not, replied Glaucon.

Then we are not going to listen; of that you may be assured.

Adeimantus added: Has no one told you of the torch-race on horseback in
honour of the goddess which will take place in the evening?

With horses! I replied: That is a novelty. Will horsemen carry torches
and pass them one to another during the race?

Yes, said Polemarchus, and not only so, but a festival will be
celebrated at night, which you certainly ought to see. Let us rise soon
after supper and see this festival; there will be a gathering of young
men, and we will have a good talk. Stay then, and do not be perverse.

Glaucon said: I suppose, since you insist, that we must.

Very good, I replied.

Accordingly we went with Polemarchus to his house; and there we found
his brothers Lysias and Euthydemus, and with them Thrasymachus the
Chalcedonian, Charmantides the Paeanian, and Cleitophon the son of
Aristonymus. There too was Cephalus the father of Polemarchus, whom I
had not seen for a long time, and I thought him very much aged. He was
seated on a cushioned chair, and had a garland on his head, for he had
been sacrificing in the court; and there were some other chairs in the
room arranged in a semicircle, upon which we sat down by him. He saluted
me eagerly, and then he said:--

You don't come to see me, Socrates, as often as you ought: If I were
still able to go and see you I would not ask you to come to me. But
at my age I can hardly get to the city, and therefore you should come
oftener to the Piraeus. For let me tell you, that the more the pleasures
of the body fade away, the greater to me is the pleasure and charm
of conversation. Do not then deny my request, but make our house your
resort and keep company with these young men; we are old friends, and
you will be quite at home with us.

I replied: There is nothing which for my part I like better, Cephalus,
than conversing with aged men; for I regard them as travellers who
have gone a journey which I too may have to go, and of whom I ought to
enquire, whether the way is smooth and easy, or rugged and difficult.
And this is a question which I should like to ask of you who have
arrived at that time which the poets call the ``threshold of old age--Is
life harder towards the end, or what report do you give of it?''

I will tell you, Socrates, he said, what my own feeling is. Men of my
age flock together; we are birds of a feather, as the old proverb says;
and at our meetings the tale of my acquaintance commonly is--I cannot
eat, I cannot drink; the pleasures of youth and love are fled away:
there was a good time once, but now that is gone, and life is no longer
life. Some complain of the slights which are put upon them by relations,
and they will tell you sadly of how many evils their old age is the
cause. But to me, Socrates, these complainers seem to blame that which
is not really in fault. For if old age were the cause, I too being old,
and every other old man, would have felt as they do. But this is not
my own experience, nor that of others whom I have known. How well I
remember the aged poet Sophocles, when in answer to the question, How
does love suit with age, Sophocles,--are you still the man you were?
Peace, he replied; most gladly have I escaped the thing of which you
speak; I feel as if I had escaped from a mad and furious master. His
words have often occurred to my mind since, and they seem as good to
me now as at the time when he uttered them. For certainly old age has
a great sense of calm and freedom; when the passions relax their hold,
then, as Sophocles says, we are freed from the grasp not of one mad
master only, but of many. The truth is, Socrates, that these regrets,
and also the complaints about relations, are to be attributed to the
same cause, which is not old age, but men's characters and tempers; for
he who is of a calm and happy nature will hardly feel the pressure of
age, but to him who is of an opposite disposition youth and age are
equally a burden.

I listened in admiration, and wanting to draw him out, that he might go
on--Yes, Cephalus, I said: but I rather suspect that people in general
are not convinced by you when you speak thus; they think that old
age sits lightly upon you, not because of your happy disposition, but
because you are rich, and wealth is well known to be a great comforter.

You are right, he replied; they are not convinced: and there is
something in what they say; not, however, so much as they imagine. I
might answer them as Themistocles answered the Seriphian who was abusing
him and saying that he was famous, not for his own merits but because he
was an Athenian: ``If you had been a native of my country or I of yours,
neither of us would have been famous.'' And to those who are not rich and
are impatient of old age, the same reply may be made; for to the good
poor man old age cannot be a light burden, nor can a bad rich man ever
have peace with himself.

May I ask, Cephalus, whether your fortune was for the most part
inherited or acquired by you?

Acquired! Socrates; do you want to know how much I acquired? In the art
of making money I have been midway between my father and grandfather:
for my grandfather, whose name I bear, doubled and trebled the value of
his patrimony, that which he inherited being much what I possess now;
but my father Lysanias reduced the property below what it is at present:
and I shall be satisfied if I leave to these my sons not less but a
little more than I received.

That was why I asked you the question, I replied, because I see that you
are indifferent about money, which is a characteristic rather of those
who have inherited their fortunes than of those who have acquired them;
the makers of fortunes have a second love of money as a creation of
their own, resembling the affection of authors for their own poems, or
of parents for their children, besides that natural love of it for the
sake of use and profit which is common to them and all men. And hence
they are very bad company, for they can talk about nothing but the
praises of wealth.

That is true, he said.

Yes, that is very true, but may I ask another question?--What do you
consider to be the greatest blessing which you have reaped from your
wealth?

One, he said, of which I could not expect easily to convince others.
For let me tell you, Socrates, that when a man thinks himself to be near
death, fears and cares enter into his mind which he never had before;
the tales of a world below and the punishment which is exacted there
of deeds done here were once a laughing matter to him, but now he
is tormented with the thought that they may be true: either from the
weakness of age, or because he is now drawing nearer to that other
place, he has a clearer view of these things; suspicions and alarms
crowd thickly upon him, and he begins to reflect and consider what
wrongs he has done to others. And when he finds that the sum of his
transgressions is great he will many a time like a child start up in his
sleep for fear, and he is filled with dark forebodings. But to him who
is conscious of no sin, sweet hope, as Pindar charmingly says, is the
kind nurse of his age:

``Hope,'' he says, ``cherishes the soul of him who lives in justice
and holiness, and is the nurse of his age and the companion of his
journey;--hope which is mightiest to sway the restless soul of man.''

How admirable are his words! And the great blessing of riches, I do not
say to every man, but to a good man, is, that he has had no occasion to
deceive or to defraud others, either intentionally or unintentionally;
and when he departs to the world below he is not in any apprehension
about offerings due to the gods or debts which he owes to men. Now to
this peace of mind the possession of wealth greatly contributes; and
therefore I say, that, setting one thing against another, of the many
advantages which wealth has to give, to a man of sense this is in my
opinion the greatest.

Well said, Cephalus, I replied; but as concerning justice, what is
it?--to speak the truth and to pay your debts--no more than this? And
even to this are there not exceptions? Suppose that a friend when in his
right mind has deposited arms with me and he asks for them when he is
not in his right mind, ought I to give them back to him? No one would
say that I ought or that I should be right in doing so, any more than
they would say that I ought always to speak the truth to one who is in
his condition.

You are quite right, he replied.

But then, I said, speaking the truth and paying your debts is not a
correct definition of justice.

Quite correct, Socrates, if Simonides is to be believed, said
Polemarchus interposing.

I fear, said Cephalus, that I must go now, for I have to look after the
sacrifices, and I hand over the argument to Polemarchus and the company.

Is not Polemarchus your heir? I said.

To be sure, he answered, and went away laughing to the sacrifices.

Tell me then, O thou heir of the argument, what did Simonides say, and
according to you truly say, about justice?

He said that the repayment of a debt is just, and in saying so he
appears to me to be right.

I should be sorry to doubt the word of such a wise and inspired man, but
his meaning, though probably clear to you, is the reverse of clear to
me. For he certainly does not mean, as we were just now saying, that I
ought to return a deposit of arms or of anything else to one who asks
for it when he is not in his right senses; and yet a deposit cannot be
denied to be a debt.

True.

Then when the person who asks me is not in his right mind I am by no
means to make the return?

Certainly not.

When Simonides said that the repayment of a debt was justice, he did not
mean to include that case?

Certainly not; for he thinks that a friend ought always to do good to a
friend and never evil.

You mean that the return of a deposit of gold which is to the injury of
the receiver, if the two parties are friends, is not the repayment of a
debt,--that is what you would imagine him to say?

Yes.

And are enemies also to receive what we owe to them?

To be sure, he said, they are to receive what we owe them, and an enemy,
as I take it, owes to an enemy that which is due or proper to him--that
is to say, evil.

Simonides, then, after the manner of poets, would seem to have spoken
darkly of the nature of justice; for he really meant to say that justice
is the giving to each man what is proper to him, and this he termed a
debt.

That must have been his meaning, he said.

By heaven! I replied; and if we asked him what due or proper thing is
given by medicine, and to whom, what answer do you think that he would
make to us?

He would surely reply that medicine gives drugs and meat and drink to
human bodies.

And what due or proper thing is given by cookery, and to what?

Seasoning to food.

And what is that which justice gives, and to whom?

If, Socrates, we are to be guided at all by the analogy of the preceding
instances, then justice is the art which gives good to friends and evil
to enemies.

That is his meaning then?

I think so.

And who is best able to do good to his friends and evil to his enemies
in time of sickness?

The physician.

Or when they are on a voyage, amid the perils of the sea?

The pilot.

And in what sort of actions or with a view to what result is the just
man most able to do harm to his enemy and good to his friend?

In going to war against the one and in making alliances with the other.

But when a man is well, my dear Polemarchus, there is no need of a
physician?

No.

And he who is not on a voyage has no need of a pilot?

No.

Then in time of peace justice will be of no use?

I am very far from thinking so.

You think that justice may be of use in peace as well as in war?

Yes.

Like husbandry for the acquisition of corn?

Yes.

Or like shoemaking for the acquisition of shoes,--that is what you mean?

Yes.

And what similar use or power of acquisition has justice in time of
peace?

In contracts, Socrates, justice is of use.

And by contracts you mean partnerships?

Exactly.

But is the just man or the skilful player a more useful and better
partner at a game of draughts?

The skilful player.

And in the laying of bricks and stones is the just man a more useful or
better partner than the builder?

Quite the reverse.

Then in what sort of partnership is the just man a better partner than
the harp-player, as in playing the harp the harp-player is certainly a
better partner than the just man?

In a money partnership.

Yes, Polemarchus, but surely not in the use of money; for you do not
want a just man to be your counsellor in the purchase or sale of a
horse; a man who is knowing about horses would be better for that, would
he not?

Certainly.

And when you want to buy a ship, the shipwright or the pilot would be
better?

True.

Then what is that joint use of silver or gold in which the just man is
to be preferred?

When you want a deposit to be kept safely.

You mean when money is not wanted, but allowed to lie?

Precisely.

That is to say, justice is useful when money is useless?

That is the inference.

And when you want to keep a pruning-hook safe, then justice is useful to
the individual and to the state; but when you want to use it, then the
art of the vine-dresser?

Clearly.

And when you want to keep a shield or a lyre, and not to use them, you
would say that justice is useful; but when you want to use them, then
the art of the soldier or of the musician?

Certainly.

And so of all other things;--justice is useful when they are useless,
and useless when they are useful?

That is the inference.

Then justice is not good for much. But let us consider this further
point: Is not he who can best strike a blow in a boxing match or in any
kind of fighting best able to ward off a blow?

Certainly.

And he who is most skilful in preventing or escaping from a disease is
best able to create one?

True.

And he is the best guard of a camp who is best able to steal a march
upon the enemy?

Certainly.

Then he who is a good keeper of anything is also a good thief?

That, I suppose, is to be inferred.

Then if the just man is good at keeping money, he is good at stealing
it.

That is implied in the argument.

Then after all the just man has turned out to be a thief. And this is
a lesson which I suspect you must have learnt out of Homer; for he,
speaking of Autolycus, the maternal grandfather of Odysseus, who is a
favourite of his, affirms that

``He was excellent above all men in theft and perjury.''

And so, you and Homer and Simonides are agreed that justice is an art of
theft; to be practised however ``for the good of friends and for the harm
of enemies,''--that was what you were saying?

No, certainly not that, though I do not now know what I did say; but I
still stand by the latter words.

Well, there is another question: By friends and enemies do we mean those
who are so really, or only in seeming?

Surely, he said, a man may be expected to love those whom he thinks
good, and to hate those whom he thinks evil.

Yes, but do not persons often err about good and evil: many who are not
good seem to be so, and conversely?

That is true.

Then to them the good will be enemies and the evil will be their
friends? True.

And in that case they will be right in doing good to the evil and evil
to the good?

Clearly.

But the good are just and would not do an injustice?

True.

Then according to your argument it is just to injure those who do no
wrong?

Nay, Socrates; the doctrine is immoral.

Then I suppose that we ought to do good to the just and harm to the
unjust?

I like that better.

But see the consequence:--Many a man who is ignorant of human nature
has friends who are bad friends, and in that case he ought to do harm to
them; and he has good enemies whom he ought to benefit; but, if so, we
shall be saying the very opposite of that which we affirmed to be the
meaning of Simonides.

Very true, he said: and I think that we had better correct an error
into which we seem to have fallen in the use of the words ``friend'' and
``enemy.''

What was the error, Polemarchus? I asked.

We assumed that he is a friend who seems to be or who is thought good.

And how is the error to be corrected?

We should rather say that he is a friend who is, as well as seems, good;
and that he who seems only, and is not good, only seems to be and is not
a friend; and of an enemy the same may be said.

You would argue that the good are our friends and the bad our enemies?

Yes.

And instead of saying simply as we did at first, that it is just to do
good to our friends and harm to our enemies, we should further say: It
is just to do good to our friends when they are good and harm to our
enemies when they are evil?

Yes, that appears to me to be the truth.

But ought the just to injure any one at all?

Undoubtedly he ought to injure those who are both wicked and his
enemies.

When horses are injured, are they improved or deteriorated?

The latter.

Deteriorated, that is to say, in the good qualities of horses, not of
dogs?

Yes, of horses.

And dogs are deteriorated in the good qualities of dogs, and not of
horses?

Of course.

And will not men who are injured be deteriorated in that which is the
proper virtue of man?

Certainly.

And that human virtue is justice?

To be sure.

Then men who are injured are of necessity made unjust?

That is the result.

But can the musician by his art make men unmusical?

Certainly not.

Or the horseman by his art make them bad horsemen?

Impossible.

And can the just by justice make men unjust, or speaking generally, can
the good by virtue make them bad?

Assuredly not.

Any more than heat can produce cold?

It cannot.

Or drought moisture?

Clearly not.

Nor can the good harm any one?

Impossible.

And the just is the good?

Certainly.

Then to injure a friend or any one else is not the act of a just man,
but of the opposite, who is the unjust?

I think that what you say is quite true, Socrates.

Then if a man says that justice consists in the repayment of debts, and
that good is the debt which a just man owes to his friends, and evil the
debt which he owes to his enemies,--to say this is not wise; for it is
not true, if, as has been clearly shown, the injuring of another can be
in no case just.

I agree with you, said Polemarchus.

Then you and I are prepared to take up arms against any one who
attributes such a saying to Simonides or Bias or Pittacus, or any other
wise man or seer?

I am quite ready to do battle at your side, he said.

Shall I tell you whose I believe the saying to be?

Whose?

I believe that Periander or Perdiccas or Xerxes or Ismenias the Theban,
or some other rich and mighty man, who had a great opinion of his own
power, was the first to say that justice is ``doing good to your friends
and harm to your enemies.''

Most true, he said.

Yes, I said; but if this definition of justice also breaks down, what
other can be offered?

Several times in the course of the discussion Thrasymachus had made an
attempt to get the argument into his own hands, and had been put down
by the rest of the company, who wanted to hear the end. But when
Polemarchus and I had done speaking and there was a pause, he could no
longer hold his peace; and, gathering himself up, he came at us like a
wild beast, seeking to devour us. We were quite panic-stricken at the
sight of him.

He roared out to the whole company: What folly, Socrates, has taken
possession of you all? And why, sillybillies, do you knock under to
one another? I say that if you want really to know what justice is,
you should not only ask but answer, and you should not seek honour to
yourself from the refutation of an opponent, but have your own answer;
for there is many a one who can ask and cannot answer. And now I will
not have you say that justice is duty or advantage or profit or gain
or interest, for this sort of nonsense will not do for me; I must have
clearness and accuracy.

I was panic-stricken at his words, and could not look at him without
trembling. Indeed I believe that if I had not fixed my eye upon him, I
should have been struck dumb: but when I saw his fury rising, I looked
at him first, and was therefore able to reply to him.

Thrasymachus, I said, with a quiver, don't be hard upon us. Polemarchus
and I may have been guilty of a little mistake in the argument, but I
can assure you that the error was not intentional. If we were seeking
for a piece of gold, you would not imagine that we were ``knocking under
to one another,'' and so losing our chance of finding it. And why, when
we are seeking for justice, a thing more precious than many pieces of
gold, do you say that we are weakly yielding to one another and not
doing our utmost to get at the truth? Nay, my good friend, we are most
willing and anxious to do so, but the fact is that we cannot. And if so,
you people who know all things should pity us and not be angry with us.

How characteristic of Socrates! he replied, with a bitter laugh;--that's
your ironical style! Did I not foresee--have I not already told you,
that whatever he was asked he would refuse to answer, and try irony or
any other shuffle, in order that he might avoid answering?

You are a philosopher, Thrasymachus, I replied, and well know that if
you ask a person what numbers make up twelve, taking care to prohibit
him whom you ask from answering twice six, or three times four, or six
times two, or four times three, ``for this sort of nonsense will not do
for me,''--then obviously, if that is your way of putting the
question, no one can answer you. But suppose that he were to retort,
``Thrasymachus, what do you mean? If one of these numbers which you
interdict be the true answer to the question, am I falsely to say some
other number which is not the right one?--is that your meaning?''--How
would you answer him?

Just as if the two cases were at all alike! he said.

Why should they not be? I replied; and even if they are not, but only
appear to be so to the person who is asked, ought he not to say what he
thinks, whether you and I forbid him or not?

I presume then that you are going to make one of the interdicted
answers?

I dare say that I may, notwithstanding the danger, if upon reflection I
approve of any of them.

But what if I give you an answer about justice other and better, he
said, than any of these? What do you deserve to have done to you?

Done to me!--as becomes the ignorant, I must learn from the wise--that
is what I deserve to have done to me.

What, and no payment! a pleasant notion!

I will pay when I have the money, I replied.

But you have, Socrates, said Glaucon: and you, Thrasymachus, need be
under no anxiety about money, for we will all make a contribution for
Socrates.

Yes, he replied, and then Socrates will do as he always does--refuse to
answer himself, but take and pull to pieces the answer of some one else.

Why, my good friend, I said, how can any one answer who knows, and says
that he knows, just nothing; and who, even if he has some faint notions
of his own, is told by a man of authority not to utter them? The
natural thing is, that the speaker should be some one like yourself
who professes to know and can tell what he knows. Will you then kindly
answer, for the edification of the company and of myself?

Glaucon and the rest of the company joined in my request, and
Thrasymachus, as any one might see, was in reality eager to speak;
for he thought that he had an excellent answer, and would distinguish
himself. But at first he affected to insist on my answering; at length
he consented to begin. Behold, he said, the wisdom of Socrates; he
refuses to teach himself, and goes about learning of others, to whom he
never even says Thank you.

That I learn of others, I replied, is quite true; but that I am
ungrateful I wholly deny. Money I have none, and therefore I pay in
praise, which is all I have; and how ready I am to praise any one who
appears to me to speak well you will very soon find out when you answer;
for I expect that you will answer well.

Listen, then, he said; I proclaim that justice is nothing else than
the interest of the stronger. And now why do you not praise me? But of
course you won't.

Let me first understand you, I replied. Justice, as you say, is the
interest of the stronger. What, Thrasymachus, is the meaning of this?
You cannot mean to say that because Polydamas, the pancratiast, is
stronger than we are, and finds the eating of beef conducive to his
bodily strength, that to eat beef is therefore equally for our good who
are weaker than he is, and right and just for us?

That's abominable of you, Socrates; you take the words in the sense
which is most damaging to the argument.

Not at all, my good sir, I said; I am trying to understand them; and I
wish that you would be a little clearer.

Well, he said, have you never heard that forms of government differ;
there are tyrannies, and there are democracies, and there are
aristocracies?

Yes, I know.

And the government is the ruling power in each state?

Certainly.

And the different forms of government make laws democratical,
aristocratical, tyrannical, with a view to their several interests;
and these laws, which are made by them for their own interests, are the
justice which they deliver to their subjects, and him who transgresses
them they punish as a breaker of the law, and unjust. And that is what
I mean when I say that in all states there is the same principle of
justice, which is the interest of the government; and as the government
must be supposed to have power, the only reasonable conclusion is, that
everywhere there is one principle of justice, which is the interest of
the stronger.

Now I understand you, I said; and whether you are right or not I will
try to discover. But let me remark, that in defining justice you have
yourself used the word ``interest'' which you forbade me to use. It is
true, however, that in your definition the words ``of the stronger'' are
added.

A small addition, you must allow, he said.

Great or small, never mind about that: we must first enquire whether
what you are saying is the truth. Now we are both agreed that justice
is interest of some sort, but you go on to say ``of the stronger''; about
this addition I am not so sure, and must therefore consider further.

Proceed.

I will; and first tell me, Do you admit that it is just for subjects to
obey their rulers?

I do.

But are the rulers of states absolutely infallible, or are they
sometimes liable to err?

To be sure, he replied, they are liable to err.

Then in making their laws they may sometimes make them rightly, and
sometimes not?

True.

When they make them rightly, they make them agreeably to their interest;
when they are mistaken, contrary to their interest; you admit that?

Yes.

And the laws which they make must be obeyed by their subjects,--and that
is what you call justice?

Doubtless.

Then justice, according to your argument, is not only obedience to the
interest of the stronger but the reverse?

What is that you are saying? he asked.

I am only repeating what you are saying, I believe. But let us consider:
Have we not admitted that the rulers may be mistaken about their own
interest in what they command, and also that to obey them is justice?
Has not that been admitted?

Yes.

Then you must also have acknowledged justice not to be for the interest
of the stronger, when the rulers unintentionally command things to be
done which are to their own injury. For if, as you say, justice is the
obedience which the subject renders to their commands, in that case, O
wisest of men, is there any escape from the conclusion that the weaker
are commanded to do, not what is for the interest, but what is for the
injury of the stronger?

Nothing can be clearer, Socrates, said Polemarchus.

Yes, said Cleitophon, interposing, if you are allowed to be his witness.

But there is no need of any witness, said Polemarchus, for Thrasymachus
himself acknowledges that rulers may sometimes command what is not for
their own interest, and that for subjects to obey them is justice.

Yes, Polemarchus,--Thrasymachus said that for subjects to do what was
commanded by their rulers is just.

Yes, Cleitophon, but he also said that justice is the interest of the
stronger, and, while admitting both these propositions, he further
acknowledged that the stronger may command the weaker who are his
subjects to do what is not for his own interest; whence follows that
justice is the injury quite as much as the interest of the stronger.

But, said Cleitophon, he meant by the interest of the stronger what the
stronger thought to be his interest,--this was what the weaker had to
do; and this was affirmed by him to be justice.

Those were not his words, rejoined Polemarchus.

Never mind, I replied, if he now says that they are, let us accept his
statement. Tell me, Thrasymachus, I said, did you mean by justice what
the stronger thought to be his interest, whether really so or not?

Certainly not, he said. Do you suppose that I call him who is mistaken
the stronger at the time when he is mistaken?

Yes, I said, my impression was that you did so, when you admitted that
the ruler was not infallible but might be sometimes mistaken.

You argue like an informer, Socrates. Do you mean, for example, that he
who is mistaken about the sick is a physician in that he is mistaken?
or that he who errs in arithmetic or grammar is an arithmetician or
grammarian at the time when he is making the mistake, in respect of the
mistake? True, we say that the physician or arithmetician or grammarian
has made a mistake, but this is only a way of speaking; for the fact is
that neither the grammarian nor any other person of skill ever makes a
mistake in so far as he is what his name implies; they none of them
err unless their skill fails them, and then they cease to be skilled
artists. No artist or sage or ruler errs at the time when he is what
his name implies; though he is commonly said to err, and I adopted the
common mode of speaking. But to be perfectly accurate, since you are
such a lover of accuracy, we should say that the ruler, in so far as he
is a ruler, is unerring, and, being unerring, always commands that which
is for his own interest; and the subject is required to execute his
commands; and therefore, as I said at first and now repeat, justice is
the interest of the stronger.

Indeed, Thrasymachus, and do I really appear to you to argue like an
informer?

Certainly, he replied.

And do you suppose that I ask these questions with any design of
injuring you in the argument?

Nay, he replied, ``suppose'' is not the word--I know it; but you will be
found out, and by sheer force of argument you will never prevail.

I shall not make the attempt, my dear man; but to avoid any
misunderstanding occurring between us in future, let me ask, in what
sense do you speak of a ruler or stronger whose interest, as you were
saying, he being the superior, it is just that the inferior should
execute--is he a ruler in the popular or in the strict sense of the
term?

In the strictest of all senses, he said. And now cheat and play the
informer if you can; I ask no quarter at your hands. But you never will
be able, never.

And do you imagine, I said, that I am such a madman as to try and cheat,
Thrasymachus? I might as well shave a lion.

Why, he said, you made the attempt a minute ago, and you failed.

Enough, I said, of these civilities. It will be better that I should ask
you a question: Is the physician, taken in that strict sense of which
you are speaking, a healer of the sick or a maker of money? And remember
that I am now speaking of the true physician.

A healer of the sick, he replied.

And the pilot--that is to say, the true pilot--is he a captain of
sailors or a mere sailor?

A captain of sailors.

The circumstance that he sails in the ship is not to be taken into
account; neither is he to be called a sailor; the name pilot by which he
is distinguished has nothing to do with sailing, but is significant of
his skill and of his authority over the sailors.

Very true, he said.

Now, I said, every art has an interest?

Certainly.

For which the art has to consider and provide?

Yes, that is the aim of art.

And the interest of any art is the perfection of it--this and nothing
else?

What do you mean?

I mean what I may illustrate negatively by the example of the body.
Suppose you were to ask me whether the body is self-sufficing or has
wants, I should reply: Certainly the body has wants; for the body may
be ill and require to be cured, and has therefore interests to which
the art of medicine ministers; and this is the origin and intention of
medicine, as you will acknowledge. Am I not right?

Quite right, he replied.

But is the art of medicine or any other art faulty or deficient in any
quality in the same way that the eye may be deficient in sight or the
ear fail of hearing, and therefore requires another art to provide
for the interests of seeing and hearing--has art in itself, I say, any
similar liability to fault or defect, and does every art require another
supplementary art to provide for its interests, and that another and
another without end? Or have the arts to look only after their
own interests? Or have they no need either of themselves or of
another?--having no faults or defects, they have no need to correct
them, either by the exercise of their own art or of any other; they have
only to consider the interest of their subject-matter. For every art
remains pure and faultless while remaining true--that is to say, while
perfect and unimpaired. Take the words in your precise sense, and tell
me whether I am not right.

Yes, clearly.

Then medicine does not consider the interest of medicine, but the
interest of the body?

True, he said.

Nor does the art of horsemanship consider the interests of the art of
horsemanship, but the interests of the horse; neither do any other arts
care for themselves, for they have no needs; they care only for that
which is the subject of their art?

True, he said.

But surely, Thrasymachus, the arts are the superiors and rulers of their
own subjects?

To this he assented with a good deal of reluctance.

Then, I said, no science or art considers or enjoins the interest of the
stronger or superior, but only the interest of the subject and weaker?

He made an attempt to contest this proposition also, but finally
acquiesced.

Then, I continued, no physician, in so far as he is a physician,
considers his own good in what he prescribes, but the good of his
patient; for the true physician is also a ruler having the human body as
a subject, and is not a mere money-maker; that has been admitted?

Yes.

And the pilot likewise, in the strict sense of the term, is a ruler of
sailors and not a mere sailor?

That has been admitted.

And such a pilot and ruler will provide and prescribe for the interest
of the sailor who is under him, and not for his own or the ruler's
interest?

He gave a reluctant ``Yes.''

Then, I said, Thrasymachus, there is no one in any rule who, in so far
as he is a ruler, considers or enjoins what is for his own interest, but
always what is for the interest of his subject or suitable to his art;
to that he looks, and that alone he considers in everything which he
says and does.

When we had got to this point in the argument, and every one saw that
the definition of justice had been completely upset, Thrasymachus,
instead of replying to me, said: Tell me, Socrates, have you got a
nurse?

Why do you ask such a question, I said, when you ought rather to be
answering?

Because she leaves you to snivel, and never wipes your nose: she has not
even taught you to know the shepherd from the sheep.

What makes you say that? I replied.

Because you fancy that the shepherd or neatherd fattens or tends the
sheep or oxen with a view to their own good and not to the good of
himself or his master; and you further imagine that the rulers of
states, if they are true rulers, never think of their subjects as sheep,
and that they are not studying their own advantage day and night. Oh,
no; and so entirely astray are you in your ideas about the just and
unjust as not even to know that justice and the just are in reality
another's good; that is to say, the interest of the ruler and stronger,
and the loss of the subject and servant; and injustice the opposite; for
the unjust is lord over the truly simple and just: he is the stronger,
and his subjects do what is for his interest, and minister to his
happiness, which is very far from being their own. Consider further,
most foolish Socrates, that the just is always a loser in comparison
with the unjust. First of all, in private contracts: wherever the unjust
is the partner of the just you will find that, when the partnership is
dissolved, the unjust man has always more and the just less. Secondly,
in their dealings with the State: when there is an income-tax, the just
man will pay more and the unjust less on the same amount of income;
and when there is anything to be received the one gains nothing and the
other much. Observe also what happens when they take an office; there is
the just man neglecting his affairs and perhaps suffering other losses,
and getting nothing out of the public, because he is just; moreover he
is hated by his friends and acquaintance for refusing to serve them in
unlawful ways. But all this is reversed in the case of the unjust man.
I am speaking, as before, of injustice on a large scale in which the
advantage of the unjust is most apparent; and my meaning will be most
clearly seen if we turn to that highest form of injustice in which the
criminal is the happiest of men, and the sufferers or those who refuse
to do injustice are the most miserable--that is to say tyranny, which by
fraud and force takes away the property of others, not little by little
but wholesale; comprehending in one, things sacred as well as profane,
private and public; for which acts of wrong, if he were detected
perpetrating any one of them singly, he would be punished and incur
great disgrace--they who do such wrong in particular cases are called
robbers of temples, and man-stealers and burglars and swindlers and
thieves. But when a man besides taking away the money of the citizens
has made slaves of them, then, instead of these names of reproach, he is
termed happy and blessed, not only by the citizens but by all who
hear of his having achieved the consummation of injustice. For mankind
censure injustice, fearing that they may be the victims of it and not
because they shrink from committing it. And thus, as I have shown,
Socrates, injustice, when on a sufficient scale, has more strength and
freedom and mastery than justice; and, as I said at first, justice is
the interest of the stronger, whereas injustice is a man's own profit
and interest.

Thrasymachus, when he had thus spoken, having, like a bath-man, deluged
our ears with his words, had a mind to go away. But the company
would not let him; they insisted that he should remain and defend his
position; and I myself added my own humble request that he would not
leave us. Thrasymachus, I said to him, excellent man, how suggestive
are your remarks! And are you going to run away before you have fairly
taught or learned whether they are true or not? Is the attempt to
determine the way of man's life so small a matter in your eyes--to
determine how life may be passed by each one of us to the greatest
advantage?

And do I differ from you, he said, as to the importance of the enquiry?

You appear rather, I replied, to have no care or thought about us,
Thrasymachus--whether we live better or worse from not knowing what you
say you know, is to you a matter of indifference. Prithee, friend,
do not keep your knowledge to yourself; we are a large party; and any
benefit which you confer upon us will be amply rewarded. For my own
part I openly declare that I am not convinced, and that I do not believe
injustice to be more gainful than justice, even if uncontrolled and
allowed to have free play. For, granting that there may be an unjust
man who is able to commit injustice either by fraud or force, still this
does not convince me of the superior advantage of injustice, and there
may be others who are in the same predicament with myself. Perhaps we
may be wrong; if so, you in your wisdom should convince us that we are
mistaken in preferring justice to injustice.

And how am I to convince you, he said, if you are not already convinced
by what I have just said; what more can I do for you? Would you have me
put the proof bodily into your souls?

Heaven forbid! I said; I would only ask you to be consistent; or, if you
change, change openly and let there be no deception. For I must remark,
Thrasymachus, if you will recall what was previously said, that although
you began by defining the true physician in an exact sense, you did not
observe a like exactness when speaking of the shepherd; you thought that
the shepherd as a shepherd tends the sheep not with a view to their own
good, but like a mere diner or banquetter with a view to the pleasures
of the table; or, again, as a trader for sale in the market, and not as
a shepherd. Yet surely the art of the shepherd is concerned only with
the good of his subjects; he has only to provide the best for them,
since the perfection of the art is already ensured whenever all the
requirements of it are satisfied. And that was what I was saying just
now about the ruler. I conceived that the art of the ruler, considered
as ruler, whether in a state or in private life, could only regard the
good of his flock or subjects; whereas you seem to think that the rulers
in states, that is to say, the true rulers, like being in authority.

Think! Nay, I am sure of it.

Then why in the case of lesser offices do men never take them willingly
without payment, unless under the idea that they govern for the
advantage not of themselves but of others? Let me ask you a question:
Are not the several arts different, by reason of their each having a
separate function? And, my dear illustrious friend, do say what you
think, that we may make a little progress.

Yes, that is the difference, he replied.

And each art gives us a particular good and not merely a general
one--medicine, for example, gives us health; navigation, safety at sea,
and so on?

Yes, he said.

And the art of payment has the special function of giving pay: but we do
not confuse this with other arts, any more than the art of the pilot is
to be confused with the art of medicine, because the health of the pilot
may be improved by a sea voyage. You would not be inclined to say, would
you, that navigation is the art of medicine, at least if we are to adopt
your exact use of language?

Certainly not.

Or because a man is in good health when he receives pay you would not
say that the art of payment is medicine?

I should not.

Nor would you say that medicine is the art of receiving pay because a
man takes fees when he is engaged in healing?

Certainly not.

And we have admitted, I said, that the good of each art is specially
confined to the art?

Yes.

Then, if there be any good which all artists have in common, that is to
be attributed to something of which they all have the common use?

True, he replied.

And when the artist is benefited by receiving pay the advantage is
gained by an additional use of the art of pay, which is not the art
professed by him?

He gave a reluctant assent to this.

Then the pay is not derived by the several artists from their respective
arts. But the truth is, that while the art of medicine gives health, and
the art of the builder builds a house, another art attends them which
is the art of pay. The various arts may be doing their own business and
benefiting that over which they preside, but would the artist receive
any benefit from his art unless he were paid as well?

I suppose not.

But does he therefore confer no benefit when he works for nothing?

Certainly, he confers a benefit.

Then now, Thrasymachus, there is no longer any doubt that neither arts
nor governments provide for their own interests; but, as we were before
saying, they rule and provide for the interests of their subjects who
are the weaker and not the stronger--to their good they attend and
not to the good of the superior. And this is the reason, my dear
Thrasymachus, why, as I was just now saying, no one is willing to
govern; because no one likes to take in hand the reformation of evils
which are not his concern without remuneration. For, in the execution of
his work, and in giving his orders to another, the true artist does not
regard his own interest, but always that of his subjects; and therefore
in order that rulers may be willing to rule, they must be paid in one of
three modes of payment, money, or honour, or a penalty for refusing.

What do you mean, Socrates? said Glaucon. The first two modes of payment
are intelligible enough, but what the penalty is I do not understand, or
how a penalty can be a payment.

You mean that you do not understand the nature of this payment which to
the best men is the great inducement to rule? Of course you know that
ambition and avarice are held to be, as indeed they are, a disgrace?

Very true.

And for this reason, I said, money and honour have no attraction for
them; good men do not wish to be openly demanding payment for governing
and so to get the name of hirelings, nor by secretly helping themselves
out of the public revenues to get the name of thieves. And not being
ambitious they do not care about honour. Wherefore necessity must be
laid upon them, and they must be induced to serve from the fear of
punishment. And this, as I imagine, is the reason why the forwardness
to take office, instead of waiting to be compelled, has been deemed
dishonourable. Now the worst part of the punishment is that he who
refuses to rule is liable to be ruled by one who is worse than himself.
And the fear of this, as I conceive, induces the good to take office,
not because they would, but because they cannot help--not under the idea
that they are going to have any benefit or enjoyment themselves, but as
a necessity, and because they are not able to commit the task of ruling
to any one who is better than themselves, or indeed as good. For there
is reason to think that if a city were composed entirely of good men,
then to avoid office would be as much an object of contention as to
obtain office is at present; then we should have plain proof that the
true ruler is not meant by nature to regard his own interest, but that
of his subjects; and every one who knew this would choose rather to
receive a benefit from another than to have the trouble of conferring
one. So far am I from agreeing with Thrasymachus that justice is the
interest of the stronger. This latter question need not be further
discussed at present; but when Thrasymachus says that the life of the
unjust is more advantageous than that of the just, his new statement
appears to me to be of a far more serious character. Which of us has
spoken truly? And which sort of life, Glaucon, do you prefer?

I for my part deem the life of the just to be the more advantageous, he
answered.

Did you hear all the advantages of the unjust which Thrasymachus was
rehearsing?

Yes, I heard him, he replied, but he has not convinced me.

Then shall we try to find some way of convincing him, if we can, that he
is saying what is not true?

Most certainly, he replied.

If, I said, he makes a set speech and we make another recounting all the
advantages of being just, and he answers and we rejoin, there must be a
numbering and measuring of the goods which are claimed on either side,
and in the end we shall want judges to decide; but if we proceed in our
enquiry as we lately did, by making admissions to one another, we shall
unite the offices of judge and advocate in our own persons.

Very good, he said.

And which method do I understand you to prefer? I said.

That which you propose.

Well, then, Thrasymachus, I said, suppose you begin at the beginning and
answer me. You say that perfect injustice is more gainful than perfect
justice?

Yes, that is what I say, and I have given you my reasons.

And what is your view about them? Would you call one of them virtue and
the other vice?

Certainly.

I suppose that you would call justice virtue and injustice vice?

What a charming notion! So likely too, seeing that I affirm injustice to
be profitable and justice not.

What else then would you say?

The opposite, he replied.

And would you call justice vice?

No, I would rather say sublime simplicity.

Then would you call injustice malignity?

No; I would rather say discretion.

And do the unjust appear to you to be wise and good?

Yes, he said; at any rate those of them who are able to be perfectly
unjust, and who have the power of subduing states and nations; but
perhaps you imagine me to be talking of cutpurses. Even this profession
if undetected has advantages, though they are not to be compared with
those of which I was just now speaking.

I do not think that I misapprehend your meaning, Thrasymachus, I
replied; but still I cannot hear without amazement that you class
injustice with wisdom and virtue, and justice with the opposite.

Certainly I do so class them.

Now, I said, you are on more substantial and almost unanswerable ground;
for if the injustice which you were maintaining to be profitable had
been admitted by you as by others to be vice and deformity, an answer
might have been given to you on received principles; but now I perceive
that you will call injustice honourable and strong, and to the unjust
you will attribute all the qualities which were attributed by us before
to the just, seeing that you do not hesitate to rank injustice with
wisdom and virtue.

You have guessed most infallibly, he replied.

Then I certainly ought not to shrink from going through with the
argument so long as I have reason to think that you, Thrasymachus, are
speaking your real mind; for I do believe that you are now in earnest
and are not amusing yourself at our expense.

I may be in earnest or not, but what is that to you?--to refute the
argument is your business.

Very true, I said; that is what I have to do: But will you be so good
as answer yet one more question? Does the just man try to gain any
advantage over the just?

Far otherwise; if he did he would not be the simple amusing creature
which he is.

And would he try to go beyond just action?

He would not.

And how would he regard the attempt to gain an advantage over the
unjust; would that be considered by him as just or unjust?

He would think it just, and would try to gain the advantage; but he
would not be able.

Whether he would or would not be able, I said, is not to the point. My
question is only whether the just man, while refusing to have more than
another just man, would wish and claim to have more than the unjust?

Yes, he would.

And what of the unjust--does he claim to have more than the just man and
to do more than is just?

Of course, he said, for he claims to have more than all men.

And the unjust man will strive and struggle to obtain more than the
unjust man or action, in order that he may have more than all?

True.

We may put the matter thus, I said--the just does not desire more than
his like but more than his unlike, whereas the unjust desires more than
both his like and his unlike?

Nothing, he said, can be better than that statement.

And the unjust is good and wise, and the just is neither?

Good again, he said.

And is not the unjust like the wise and good and the just unlike them?

Of course, he said, he who is of a certain nature, is like those who are
of a certain nature; he who is not, not.

Each of them, I said, is such as his like is?

Certainly, he replied.

Very good, Thrasymachus, I said; and now to take the case of the arts:
you would admit that one man is a musician and another not a musician?

Yes.

And which is wise and which is foolish?

Clearly the musician is wise, and he who is not a musician is foolish.

And he is good in as far as he is wise, and bad in as far as he is
foolish?

Yes.

And you would say the same sort of thing of the physician?

Yes.

And do you think, my excellent friend, that a musician when he adjusts
the lyre would desire or claim to exceed or go beyond a musician in the
tightening and loosening the strings?

I do not think that he would.

But he would claim to exceed the non-musician?

Of course.

And what would you say of the physician? In prescribing meats and drinks
would he wish to go beyond another physician or beyond the practice of
medicine?

He would not.

But he would wish to go beyond the non-physician?

Yes.

And about knowledge and ignorance in general; see whether you think that
any man who has knowledge ever would wish to have the choice of saying
or doing more than another man who has knowledge. Would he not rather
say or do the same as his like in the same case?

That, I suppose, can hardly be denied.

And what of the ignorant? would he not desire to have more than either
the knowing or the ignorant?

I dare say.

And the knowing is wise?

Yes.

And the wise is good?

True.

Then the wise and good will not desire to gain more than his like, but
more than his unlike and opposite?

I suppose so.

Whereas the bad and ignorant will desire to gain more than both?

Yes.

But did we not say, Thrasymachus, that the unjust goes beyond both his
like and unlike? Were not these your words?

They were.

And you also said that the just will not go beyond his like but his
unlike?

Yes.

Then the just is like the wise and good, and the unjust like the evil
and ignorant?

That is the inference.

And each of them is such as his like is?

That was admitted.

Then the just has turned out to be wise and good and the unjust evil and
ignorant.

Thrasymachus made all these admissions, not fluently, as I repeat
them, but with extreme reluctance; it was a hot summer's day, and the
perspiration poured from him in torrents; and then I saw what I had
never seen before, Thrasymachus blushing. As we were now agreed that
justice was virtue and wisdom, and injustice vice and ignorance, I
proceeded to another point:

Well, I said, Thrasymachus, that matter is now settled; but were we not
also saying that injustice had strength; do you remember?

Yes, I remember, he said, but do not suppose that I approve of what you
are saying or have no answer; if however I were to answer, you would be
quite certain to accuse me of haranguing; therefore either permit me to
have my say out, or if you would rather ask, do so, and I will answer
``Very good,'' as they say to story-telling old women, and will nod ``Yes''
and ``No.''

Certainly not, I said, if contrary to your real opinion.

Yes, he said, I will, to please you, since you will not let me speak.
What else would you have?

Nothing in the world, I said; and if you are so disposed I will ask and
you shall answer.

Proceed.

Then I will repeat the question which I asked before, in order that
our examination of the relative nature of justice and injustice may be
carried on regularly. A statement was made that injustice is stronger
and more powerful than justice, but now justice, having been identified
with wisdom and virtue, is easily shown to be stronger than injustice,
if injustice is ignorance; this can no longer be questioned by any one.
But I want to view the matter, Thrasymachus, in a different way: You
would not deny that a state may be unjust and may be unjustly attempting
to enslave other states, or may have already enslaved them, and may be
holding many of them in subjection?

True, he replied; and I will add that the best and most perfectly unjust
state will be most likely to do so.

I know, I said, that such was your position; but what I would further
consider is, whether this power which is possessed by the superior state
can exist or be exercised without justice or only with justice.

If you are right in your view, and justice is wisdom, then only with
justice; but if I am right, then without justice.

I am delighted, Thrasymachus, to see you not only nodding assent and
dissent, but making answers which are quite excellent.

That is out of civility to you, he replied.

You are very kind, I said; and would you have the goodness also to
inform me, whether you think that a state, or an army, or a band of
robbers and thieves, or any other gang of evil-doers could act at all if
they injured one another?

No indeed, he said, they could not.

But if they abstained from injuring one another, then they might act
together better?

Yes.

And this is because injustice creates divisions and hatreds and
fighting, and justice imparts harmony and friendship; is not that true,
Thrasymachus?

I agree, he said, because I do not wish to quarrel with you.

How good of you, I said; but I should like to know also whether
injustice, having this tendency to arouse hatred, wherever existing,
among slaves or among freemen, will not make them hate one another and
set them at variance and render them incapable of common action?

Certainly.

And even if injustice be found in two only, will they not quarrel and
fight, and become enemies to one another and to the just?

They will.

And suppose injustice abiding in a single person, would your wisdom say
that she loses or that she retains her natural power?

Let us assume that she retains her power.

Yet is not the power which injustice exercises of such a nature that
wherever she takes up her abode, whether in a city, in an army, in a
family, or in any other body, that body is, to begin with, rendered
incapable of united action by reason of sedition and distraction; and
does it not become its own enemy and at variance with all that opposes
it, and with the just? Is not this the case?

Yes, certainly.

And is not injustice equally fatal when existing in a single person; in
the first place rendering him incapable of action because he is not
at unity with himself, and in the second place making him an enemy to
himself and the just? Is not that true, Thrasymachus?

Yes.

And O my friend, I said, surely the gods are just?

Granted that they are.

But if so, the unjust will be the enemy of the gods, and the just will
be their friend?

Feast away in triumph, and take your fill of the argument; I will not
oppose you, lest I should displease the company.

Well then, proceed with your answers, and let me have the remainder of
my repast. For we have already shown that the just are clearly wiser and
better and abler than the unjust, and that the unjust are incapable of
common action; nay more, that to speak as we did of men who are evil
acting at any time vigorously together, is not strictly true, for
if they had been perfectly evil, they would have laid hands upon one
another; but it is evident that there must have been some remnant of
justice in them, which enabled them to combine; if there had not been
they would have injured one another as well as their victims; they
were but half-villains in their enterprises; for had they been whole
villains, and utterly unjust, they would have been utterly incapable of
action. That, as I believe, is the truth of the matter, and not what you
said at first. But whether the just have a better and happier life than
the unjust is a further question which we also proposed to consider. I
think that they have, and for the reasons which I have given; but still
I should like to examine further, for no light matter is at stake,
nothing less than the rule of human life.

Proceed.

I will proceed by asking a question: Would you not say that a horse has
some end?

I should.

And the end or use of a horse or of anything would be that which could
not be accomplished, or not so well accomplished, by any other thing?

I do not understand, he said.

Let me explain: Can you see, except with the eye?

Certainly not.

Or hear, except with the ear?

No.

These then may be truly said to be the ends of these organs?

They may.

But you can cut off a vine-branch with a dagger or with a chisel, and in
many other ways?

Of course.

And yet not so well as with a pruning-hook made for the purpose?

True.

May we not say that this is the end of a pruning-hook?

We may.

Then now I think you will have no difficulty in understanding my meaning
when I asked the question whether the end of anything would be that
which could not be accomplished, or not so well accomplished, by any
other thing?

I understand your meaning, he said, and assent.

And that to which an end is appointed has also an excellence? Need I ask
again whether the eye has an end?

It has.

And has not the eye an excellence?

Yes.

And the ear has an end and an excellence also?

True.

And the same is true of all other things; they have each of them an end
and a special excellence?

That is so.

Well, and can the eyes fulfil their end if they are wanting in their own
proper excellence and have a defect instead?

How can they, he said, if they are blind and cannot see?

You mean to say, if they have lost their proper excellence, which is
sight; but I have not arrived at that point yet. I would rather ask
the question more generally, and only enquire whether the things which
fulfil their ends fulfil them by their own proper excellence, and fail
of fulfilling them by their own defect?

Certainly, he replied.

I might say the same of the ears; when deprived of their own proper
excellence they cannot fulfil their end?

True.

And the same observation will apply to all other things?

I agree.

Well; and has not the soul an end which nothing else can fulfil? for
example, to superintend and command and deliberate and the like. Are not
these functions proper to the soul, and can they rightly be assigned to
any other?

To no other.

And is not life to be reckoned among the ends of the soul?

Assuredly, he said.

And has not the soul an excellence also?

Yes.

And can she or can she not fulfil her own ends when deprived of that
excellence?

She cannot.

Then an evil soul must necessarily be an evil ruler and superintendent,
and the good soul a good ruler?

Yes, necessarily.

And we have admitted that justice is the excellence of the soul, and
injustice the defect of the soul?

That has been admitted.

Then the just soul and the just man will live well, and the unjust man
will live ill?

That is what your argument proves.

And he who lives well is blessed and happy, and he who lives ill the
reverse of happy?

Certainly.

Then the just is happy, and the unjust miserable?

So be it.

But happiness and not misery is profitable.

Of course.

Then, my blessed Thrasymachus, injustice can never be more profitable
than justice.

Let this, Socrates, he said, be your entertainment at the Bendidea.

For which I am indebted to you, I said, now that you have grown gentle
towards me and have left off scolding. Nevertheless, I have not been
well entertained; but that was my own fault and not yours. As an epicure
snatches a taste of every dish which is successively brought to table,
he not having allowed himself time to enjoy the one before, so have I
gone from one subject to another without having discovered what I sought
at first, the nature of justice. I left that enquiry and turned away
to consider whether justice is virtue and wisdom or evil and folly; and
when there arose a further question about the comparative advantages of
justice and injustice, I could not refrain from passing on to that. And
the result of the whole discussion has been that I know nothing at all.
For I know not what justice is, and therefore I am not likely to know
whether it is or is not a virtue, nor can I say whether the just man is
happy or unhappy.

% section book_i (end)

\section{Book II} % (fold)
\label{sec:book_ii}



BOOK II.

With these words I was thinking that I had made an end of the
discussion; but the end, in truth, proved to be only a beginning. For
Glaucon, who is always the most pugnacious of men, was dissatisfied at
Thrasymachus' retirement; he wanted to have the battle out. So he said
to me: Socrates, do you wish really to persuade us, or only to seem to
have persuaded us, that to be just is always better than to be unjust?

I should wish really to persuade you, I replied, if I could.

Then you certainly have not succeeded. Let me ask you now:--How would
you arrange goods--are there not some which we welcome for their
own sakes, and independently of their consequences, as, for example,
harmless pleasures and enjoyments, which delight us at the time,
although nothing follows from them?

I agree in thinking that there is such a class, I replied.

Is there not also a second class of goods, such as knowledge, sight,
health, which are desirable not only in themselves, but also for their
results?

Certainly, I said.

And would you not recognize a third class, such as gymnastic, and the
care of the sick, and the physician's art; also the various ways of
money-making--these do us good but we regard them as disagreeable; and
no one would choose them for their own sakes, but only for the sake of
some reward or result which flows from them?

There is, I said, this third class also. But why do you ask?

Because I want to know in which of the three classes you would place
justice?

In the highest class, I replied,--among those goods which he who would
be happy desires both for their own sake and for the sake of their
results.

Then the many are of another mind; they think that justice is to be
reckoned in the troublesome class, among goods which are to be pursued
for the sake of rewards and of reputation, but in themselves are
disagreeable and rather to be avoided.

I know, I said, that this is their manner of thinking, and that this was
the thesis which Thrasymachus was maintaining just now, when he censured
justice and praised injustice. But I am too stupid to be convinced by
him.

I wish, he said, that you would hear me as well as him, and then I shall
see whether you and I agree. For Thrasymachus seems to me, like a snake,
to have been charmed by your voice sooner than he ought to have been;
but to my mind the nature of justice and injustice have not yet been
made clear. Setting aside their rewards and results, I want to know what
they are in themselves, and how they inwardly work in the soul. If you,
please, then, I will revive the argument of Thrasymachus. And first I
will speak of the nature and origin of justice according to the common
view of them. Secondly, I will show that all men who practise justice do
so against their will, of necessity, but not as a good. And thirdly, I
will argue that there is reason in this view, for the life of the unjust
is after all better far than the life of the just--if what they say
is true, Socrates, since I myself am not of their opinion. But still I
acknowledge that I am perplexed when I hear the voices of Thrasymachus
and myriads of others dinning in my ears; and, on the other hand, I have
never yet heard the superiority of justice to injustice maintained by
any one in a satisfactory way. I want to hear justice praised in respect
of itself; then I shall be satisfied, and you are the person from whom
I think that I am most likely to hear this; and therefore I will praise
the unjust life to the utmost of my power, and my manner of speaking
will indicate the manner in which I desire to hear you too praising
justice and censuring injustice. Will you say whether you approve of my
proposal?

Indeed I do; nor can I imagine any theme about which a man of sense
would oftener wish to converse.

I am delighted, he replied, to hear you say so, and shall begin by
speaking, as I proposed, of the nature and origin of justice.

They say that to do injustice is, by nature, good; to suffer injustice,
evil; but that the evil is greater than the good. And so when men have
both done and suffered injustice and have had experience of both, not
being able to avoid the one and obtain the other, they think that they
had better agree among themselves to have neither; hence there arise
laws and mutual covenants; and that which is ordained by law is termed
by them lawful and just. This they affirm to be the origin and nature of
justice;--it is a mean or compromise, between the best of all, which is
to do injustice and not be punished, and the worst of all, which is to
suffer injustice without the power of retaliation; and justice, being at
a middle point between the two, is tolerated not as a good, but as
the lesser evil, and honoured by reason of the inability of men to do
injustice. For no man who is worthy to be called a man would ever submit
to such an agreement if he were able to resist; he would be mad if he
did. Such is the received account, Socrates, of the nature and origin of
justice.

Now that those who practise justice do so involuntarily and because they
have not the power to be unjust will best appear if we imagine something
of this kind: having given both to the just and the unjust power to do
what they will, let us watch and see whither desire will lead them;
then we shall discover in the very act the just and unjust man to be
proceeding along the same road, following their interest, which all
natures deem to be their good, and are only diverted into the path of
justice by the force of law. The liberty which we are supposing may be
most completely given to them in the form of such a power as is said
to have been possessed by Gyges, the ancestor of Croesus the Lydian.
According to the tradition, Gyges was a shepherd in the service of
the king of Lydia; there was a great storm, and an earthquake made an
opening in the earth at the place where he was feeding his flock. Amazed
at the sight, he descended into the opening, where, among other marvels,
he beheld a hollow brazen horse, having doors, at which he stooping and
looking in saw a dead body of stature, as appeared to him, more than
human, and having nothing on but a gold ring; this he took from the
finger of the dead and reascended. Now the shepherds met together,
according to custom, that they might send their monthly report about the
flocks to the king; into their assembly he came having the ring on his
finger, and as he was sitting among them he chanced to turn the collet
of the ring inside his hand, when instantly he became invisible to the
rest of the company and they began to speak of him as if he were no
longer present. He was astonished at this, and again touching the ring
he turned the collet outwards and reappeared; he made several trials
of the ring, and always with the same result--when he turned the collet
inwards he became invisible, when outwards he reappeared. Whereupon he
contrived to be chosen one of the messengers who were sent to the court;
whereas soon as he arrived he seduced the queen, and with her help
conspired against the king and slew him, and took the kingdom. Suppose
now that there were two such magic rings, and the just put on one of
them and the unjust the other; no man can be imagined to be of such an
iron nature that he would stand fast in justice. No man would keep his
hands off what was not his own when he could safely take what he
liked out of the market, or go into houses and lie with any one at
his pleasure, or kill or release from prison whom he would, and in all
respects be like a God among men. Then the actions of the just would be
as the actions of the unjust; they would both come at last to the same
point. And this we may truly affirm to be a great proof that a man is
just, not willingly or because he thinks that justice is any good to him
individually, but of necessity, for wherever any one thinks that he
can safely be unjust, there he is unjust. For all men believe in their
hearts that injustice is far more profitable to the individual than
justice, and he who argues as I have been supposing, will say that they
are right. If you could imagine any one obtaining this power of becoming
invisible, and never doing any wrong or touching what was another's, he
would be thought by the lookers-on to be a most wretched idiot, although
they would praise him to one another's faces, and keep up appearances
with one another from a fear that they too might suffer injustice.
Enough of this.

Now, if we are to form a real judgment of the life of the just and
unjust, we must isolate them; there is no other way; and how is the
isolation to be effected? I answer: Let the unjust man be entirely
unjust, and the just man entirely just; nothing is to be taken away from
either of them, and both are to be perfectly furnished for the work
of their respective lives. First, let the unjust be like other
distinguished masters of craft; like the skilful pilot or physician, who
knows intuitively his own powers and keeps within their limits, and who,
if he fails at any point, is able to recover himself. So let the unjust
make his unjust attempts in the right way, and lie hidden if he means
to be great in his injustice: (he who is found out is nobody:) for
the highest reach of injustice is, to be deemed just when you are not.
Therefore I say that in the perfectly unjust man we must assume the most
perfect injustice; there is to be no deduction, but we must allow
him, while doing the most unjust acts, to have acquired the greatest
reputation for justice. If he have taken a false step he must be able to
recover himself; he must be one who can speak with effect, if any of his
deeds come to light, and who can force his way where force is required
by his courage and strength, and command of money and friends. And at
his side let us place the just man in his nobleness and simplicity,
wishing, as Aeschylus says, to be and not to seem good. There must be no
seeming, for if he seem to be just he will be honoured and rewarded, and
then we shall not know whether he is just for the sake of justice or
for the sake of honours and rewards; therefore, let him be clothed in
justice only, and have no other covering; and he must be imagined in a
state of life the opposite of the former. Let him be the best of men,
and let him be thought the worst; then he will have been put to the
proof; and we shall see whether he will be affected by the fear of
infamy and its consequences. And let him continue thus to the hour of
death; being just and seeming to be unjust. When both have reached the
uttermost extreme, the one of justice and the other of injustice, let
judgment be given which of them is the happier of the two.

Heavens! my dear Glaucon, I said, how energetically you polish them
up for the decision, first one and then the other, as if they were two
statues.

I do my best, he said. And now that we know what they are like there
is no difficulty in tracing out the sort of life which awaits either
of them. This I will proceed to describe; but as you may think the
description a little too coarse, I ask you to suppose, Socrates, that
the words which follow are not mine.--Let me put them into the mouths of
the eulogists of injustice: They will tell you that the just man who is
thought unjust will be scourged, racked, bound--will have his eyes
burnt out; and, at last, after suffering every kind of evil, he will be
impaled: Then he will understand that he ought to seem only, and not to
be, just; the words of Aeschylus may be more truly spoken of the unjust
than of the just. For the unjust is pursuing a reality; he does not live
with a view to appearances--he wants to be really unjust and not to seem
only:--

``His mind has a soil deep and fertile, Out of which spring his prudent
counsels.''

In the first place, he is thought just, and therefore bears rule in the
city; he can marry whom he will, and give in marriage to whom he
will; also he can trade and deal where he likes, and always to his own
advantage, because he has no misgivings about injustice; and at every
contest, whether in public or private, he gets the better of his
antagonists, and gains at their expense, and is rich, and out of his
gains he can benefit his friends, and harm his enemies; moreover, he
can offer sacrifices, and dedicate gifts to the gods abundantly and
magnificently, and can honour the gods or any man whom he wants to
honour in a far better style than the just, and therefore he is likely
to be dearer than they are to the gods. And thus, Socrates, gods and men
are said to unite in making the life of the unjust better than the life
of the just.

I was going to say something in answer to Glaucon, when Adeimantus, his
brother, interposed: Socrates, he said, you do not suppose that there is
nothing more to be urged?

Why, what else is there? I answered.

The strongest point of all has not been even mentioned, he replied.

Well, then, according to the proverb, ``Let brother help brother''--if
he fails in any part do you assist him; although I must confess that
Glaucon has already said quite enough to lay me in the dust, and take
from me the power of helping justice.

Nonsense, he replied. But let me add something more: There is another
side to Glaucon's argument about the praise and censure of justice
and injustice, which is equally required in order to bring out what I
believe to be his meaning. Parents and tutors are always telling their
sons and their wards that they are to be just; but why? not for the sake
of justice, but for the sake of character and reputation; in the hope of
obtaining for him who is reputed just some of those offices, marriages,
and the like which Glaucon has enumerated among the advantages accruing
to the unjust from the reputation of justice. More, however, is made of
appearances by this class of persons than by the others; for they
throw in the good opinion of the gods, and will tell you of a shower of
benefits which the heavens, as they say, rain upon the pious; and this
accords with the testimony of the noble Hesiod and Homer, the first of
whom says, that the gods make the oaks of the just--

   ``To bear acorns at their summit, and bees in the middle;
   And the sheep are bowed down with the weight of their fleeces,''

and many other blessings of a like kind are provided for them. And Homer
has a very similar strain; for he speaks of one whose fame is--

``As the fame of some blameless king who, like a god, Maintains justice;
to whom the black earth brings forth Wheat and barley, whose trees are
bowed with fruit, And his sheep never fail to bear, and the sea gives
him fish.''

Still grander are the gifts of heaven which Musaeus and his son
vouchsafe to the just; they take them down into the world below, where
they have the saints lying on couches at a feast, everlastingly drunk,
crowned with garlands; their idea seems to be that an immortality of
drunkenness is the highest meed of virtue. Some extend their rewards
yet further; the posterity, as they say, of the faithful and just shall
survive to the third and fourth generation. This is the style in which
they praise justice. But about the wicked there is another strain; they
bury them in a slough in Hades, and make them carry water in a sieve;
also while they are yet living they bring them to infamy, and inflict
upon them the punishments which Glaucon described as the portion of the
just who are reputed to be unjust; nothing else does their invention
supply. Such is their manner of praising the one and censuring the
other.

Once more, Socrates, I will ask you to consider another way of speaking
about justice and injustice, which is not confined to the poets, but
is found in prose writers. The universal voice of mankind is always
declaring that justice and virtue are honourable, but grievous and
toilsome; and that the pleasures of vice and injustice are easy of
attainment, and are only censured by law and opinion. They say also that
honesty is for the most part less profitable than dishonesty; and they
are quite ready to call wicked men happy, and to honour them both in
public and private when they are rich or in any other way influential,
while they despise and overlook those who may be weak and poor, even
though acknowledging them to be better than the others. But most
extraordinary of all is their mode of speaking about virtue and the
gods: they say that the gods apportion calamity and misery to many good
men, and good and happiness to the wicked. And mendicant prophets go to
rich men's doors and persuade them that they have a power committed
to them by the gods of making an atonement for a man's own or his
ancestor's sins by sacrifices or charms, with rejoicings and feasts; and
they promise to harm an enemy, whether just or unjust, at a small cost;
with magic arts and incantations binding heaven, as they say, to execute
their will. And the poets are the authorities to whom they appeal, now
smoothing the path of vice with the words of Hesiod;--

``Vice may be had in abundance without trouble; the way is smooth and her
dwelling-place is near. But before virtue the gods have set toil,''

and a tedious and uphill road: then citing Homer as a witness that the
gods may be influenced by men; for he also says:--

``The gods, too, may be turned from their purpose; and men pray to them
and avert their wrath by sacrifices and soothing entreaties, and by
libations and the odour of fat, when they have sinned and transgressed.''

And they produce a host of books written by Musaeus and Orpheus,
who were children of the Moon and the Muses--that is what they
say--according to which they perform their ritual, and persuade not only
individuals, but whole cities, that expiations and atonements for sin
may be made by sacrifices and amusements which fill a vacant hour, and
are equally at the service of the living and the dead; the latter sort
they call mysteries, and they redeem us from the pains of hell, but if
we neglect them no one knows what awaits us.

He proceeded: And now when the young hear all this said about virtue and
vice, and the way in which gods and men regard them, how are their minds
likely to be affected, my dear Socrates,--those of them, I mean, who are
quickwitted, and, like bees on the wing, light on every flower, and from
all that they hear are prone to draw conclusions as to what manner of
persons they should be and in what way they should walk if they would
make the best of life? Probably the youth will say to himself in the
words of Pindar--

``Can I by justice or by crooked ways of deceit ascend a loftier tower
which may be a fortress to me all my days?''

For what men say is that, if I am really just and am not also thought
just profit there is none, but the pain and loss on the other hand
are unmistakeable. But if, though unjust, I acquire the reputation of
justice, a heavenly life is promised to me. Since then, as philosophers
prove, appearance tyrannizes over truth and is lord of happiness, to
appearance I must devote myself. I will describe around me a picture and
shadow of virtue to be the vestibule and exterior of my house; behind I
will trail the subtle and crafty fox, as Archilochus, greatest of sages,
recommends. But I hear some one exclaiming that the concealment of
wickedness is often difficult; to which I answer, Nothing great is easy.
Nevertheless, the argument indicates this, if we would be happy, to be
the path along which we should proceed. With a view to concealment we
will establish secret brotherhoods and political clubs. And there
are professors of rhetoric who teach the art of persuading courts and
assemblies; and so, partly by persuasion and partly by force, I shall
make unlawful gains and not be punished. Still I hear a voice saying
that the gods cannot be deceived, neither can they be compelled. But
what if there are no gods? or, suppose them to have no care of human
things--why in either case should we mind about concealment? And even if
there are gods, and they do care about us, yet we know of them only
from tradition and the genealogies of the poets; and these are the very
persons who say that they may be influenced and turned by ``sacrifices
and soothing entreaties and by offerings.'' Let us be consistent then,
and believe both or neither. If the poets speak truly, why then we had
better be unjust, and offer of the fruits of injustice; for if we are
just, although we may escape the vengeance of heaven, we shall lose the
gains of injustice; but, if we are unjust, we shall keep the gains, and
by our sinning and praying, and praying and sinning, the gods will be
propitiated, and we shall not be punished. ``But there is a world below
in which either we or our posterity will suffer for our unjust deeds.''
Yes, my friend, will be the reflection, but there are mysteries and
atoning deities, and these have great power. That is what mighty
cities declare; and the children of the gods, who were their poets and
prophets, bear a like testimony.

On what principle, then, shall we any longer choose justice rather than
the worst injustice? when, if we only unite the latter with a deceitful
regard to appearances, we shall fare to our mind both with gods and
men, in life and after death, as the most numerous and the highest
authorities tell us. Knowing all this, Socrates, how can a man who
has any superiority of mind or person or rank or wealth, be willing to
honour justice; or indeed to refrain from laughing when he hears justice
praised? And even if there should be some one who is able to disprove
the truth of my words, and who is satisfied that justice is best, still
he is not angry with the unjust, but is very ready to forgive them,
because he also knows that men are not just of their own free will;
unless, peradventure, there be some one whom the divinity within him may
have inspired with a hatred of injustice, or who has attained knowledge
of the truth--but no other man. He only blames injustice who, owing to
cowardice or age or some weakness, has not the power of being unjust.
And this is proved by the fact that when he obtains the power, he
immediately becomes unjust as far as he can be.

The cause of all this, Socrates, was indicated by us at the beginning of
the argument, when my brother and I told you how astonished we were to
find that of all the professing panegyrists of justice--beginning with
the ancient heroes of whom any memorial has been preserved to us, and
ending with the men of our own time--no one has ever blamed injustice or
praised justice except with a view to the glories, honours, and benefits
which flow from them. No one has ever adequately described either in
verse or prose the true essential nature of either of them abiding in
the soul, and invisible to any human or divine eye; or shown that of
all the things of a man's soul which he has within him, justice is
the greatest good, and injustice the greatest evil. Had this been the
universal strain, had you sought to persuade us of this from our youth
upwards, we should not have been on the watch to keep one another from
doing wrong, but every one would have been his own watchman, because
afraid, if he did wrong, of harbouring in himself the greatest of
evils. I dare say that Thrasymachus and others would seriously hold the
language which I have been merely repeating, and words even stronger
than these about justice and injustice, grossly, as I conceive,
perverting their true nature. But I speak in this vehement manner, as
I must frankly confess to you, because I want to hear from you the
opposite side; and I would ask you to show not only the superiority
which justice has over injustice, but what effect they have on the
possessor of them which makes the one to be a good and the other an evil
to him. And please, as Glaucon requested of you, to exclude reputations;
for unless you take away from each of them his true reputation and
add on the false, we shall say that you do not praise justice, but the
appearance of it; we shall think that you are only exhorting us to keep
injustice dark, and that you really agree with Thrasymachus in thinking
that justice is another's good and the interest of the stronger, and
that injustice is a man's own profit and interest, though injurious to
the weaker. Now as you have admitted that justice is one of that highest
class of goods which are desired indeed for their results, but in a far
greater degree for their own sakes--like sight or hearing or knowledge
or health, or any other real and natural and not merely conventional
good--I would ask you in your praise of justice to regard one point
only: I mean the essential good and evil which justice and injustice
work in the possessors of them. Let others praise justice and censure
injustice, magnifying the rewards and honours of the one and abusing the
other; that is a manner of arguing which, coming from them, I am
ready to tolerate, but from you who have spent your whole life in the
consideration of this question, unless I hear the contrary from your own
lips, I expect something better. And therefore, I say, not only prove to
us that justice is better than injustice, but show what they either of
them do to the possessor of them, which makes the one to be a good and
the other an evil, whether seen or unseen by gods and men.

I had always admired the genius of Glaucon and Adeimantus, but on
hearing these words I was quite delighted, and said: Sons of an
illustrious father, that was not a bad beginning of the Elegiac verses
which the admirer of Glaucon made in honour of you after you had
distinguished yourselves at the battle of Megara:--

``Sons of Ariston,'' he sang, ``divine offspring of an illustrious hero.''

The epithet is very appropriate, for there is something truly divine in
being able to argue as you have done for the superiority of injustice,
and remaining unconvinced by your own arguments. And I do believe that
you are not convinced--this I infer from your general character, for had
I judged only from your speeches I should have mistrusted you. But
now, the greater my confidence in you, the greater is my difficulty in
knowing what to say. For I am in a strait between two; on the one hand I
feel that I am unequal to the task; and my inability is brought home to
me by the fact that you were not satisfied with the answer which I made
to Thrasymachus, proving, as I thought, the superiority which justice
has over injustice. And yet I cannot refuse to help, while breath and
speech remain to me; I am afraid that there would be an impiety in being
present when justice is evil spoken of and not lifting up a hand in her
defence. And therefore I had best give such help as I can.

Glaucon and the rest entreated me by all means not to let the question
drop, but to proceed in the investigation. They wanted to arrive at the
truth, first, about the nature of justice and injustice, and secondly,
about their relative advantages. I told them, what I really thought,
that the enquiry would be of a serious nature, and would require very
good eyes. Seeing then, I said, that we are no great wits, I think that
we had better adopt a method which I may illustrate thus; suppose that
a short-sighted person had been asked by some one to read small letters
from a distance; and it occurred to some one else that they might be
found in another place which was larger and in which the letters were
larger--if they were the same and he could read the larger letters
first, and then proceed to the lesser--this would have been thought a
rare piece of good fortune.

Very true, said Adeimantus; but how does the illustration apply to our
enquiry?

I will tell you, I replied; justice, which is the subject of our
enquiry, is, as you know, sometimes spoken of as the virtue of an
individual, and sometimes as the virtue of a State.

True, he replied.

And is not a State larger than an individual?

It is.

Then in the larger the quantity of justice is likely to be larger and
more easily discernible. I propose therefore that we enquire into the
nature of justice and injustice, first as they appear in the State, and
secondly in the individual, proceeding from the greater to the lesser
and comparing them.

That, he said, is an excellent proposal.

And if we imagine the State in process of creation, we shall see the
justice and injustice of the State in process of creation also.

I dare say.

When the State is completed there may be a hope that the object of our
search will be more easily discovered.

Yes, far more easily.

But ought we to attempt to construct one? I said; for to do so, as I am
inclined to think, will be a very serious task. Reflect therefore.

I have reflected, said Adeimantus, and am anxious that you should
proceed.

A State, I said, arises, as I conceive, out of the needs of mankind;
no one is self-sufficing, but all of us have many wants. Can any other
origin of a State be imagined?

There can be no other.

Then, as we have many wants, and many persons are needed to supply them,
one takes a helper for one purpose and another for another; and when
these partners and helpers are gathered together in one habitation the
body of inhabitants is termed a State.

True, he said.

And they exchange with one another, and one gives, and another receives,
under the idea that the exchange will be for their good.

Very true.

Then, I said, let us begin and create in idea a State; and yet the true
creator is necessity, who is the mother of our invention.

Of course, he replied.

Now the first and greatest of necessities is food, which is the
condition of life and existence.

Certainly.

The second is a dwelling, and the third clothing and the like.

True.

And now let us see how our city will be able to supply this great
demand: We may suppose that one man is a husbandman, another a builder,
some one else a weaver--shall we add to them a shoemaker, or perhaps
some other purveyor to our bodily wants?

Quite right.

The barest notion of a State must include four or five men.

Clearly.

And how will they proceed? Will each bring the result of his labours
into a common stock?--the individual husbandman, for example, producing
for four, and labouring four times as long and as much as he need in the
provision of food with which he supplies others as well as himself;
or will he have nothing to do with others and not be at the trouble of
producing for them, but provide for himself alone a fourth of the food
in a fourth of the time, and in the remaining three fourths of his time
be employed in making a house or a coat or a pair of shoes, having no
partnership with others, but supplying himself all his own wants?

Adeimantus thought that he should aim at producing food only and not at
producing everything.

Probably, I replied, that would be the better way; and when I hear you
say this, I am myself reminded that we are not all alike; there
are diversities of natures among us which are adapted to different
occupations.

Very true.

And will you have a work better done when the workman has many
occupations, or when he has only one?

When he has only one.

Further, there can be no doubt that a work is spoilt when not done at
the right time?

No doubt.

For business is not disposed to wait until the doer of the business is
at leisure; but the doer must follow up what he is doing, and make the
business his first object.

He must.

And if so, we must infer that all things are produced more plentifully
and easily and of a better quality when one man does one thing which is
natural to him and does it at the right time, and leaves other things.

Undoubtedly.

Then more than four citizens will be required; for the husbandman will
not make his own plough or mattock, or other implements of agriculture,
if they are to be good for anything. Neither will the builder make
his tools--and he too needs many; and in like manner the weaver and
shoemaker.

True.

Then carpenters, and smiths, and many other artisans, will be sharers in
our little State, which is already beginning to grow?

True.

Yet even if we add neatherds, shepherds, and other herdsmen, in order
that our husbandmen may have oxen to plough with, and builders as well
as husbandmen may have draught cattle, and curriers and weavers fleeces
and hides,--still our State will not be very large.

That is true; yet neither will it be a very small State which contains
all these.

Then, again, there is the situation of the city--to find a place where
nothing need be imported is wellnigh impossible.

Impossible.

Then there must be another class of citizens who will bring the required
supply from another city?

There must.

But if the trader goes empty-handed, having nothing which they require
who would supply his need, he will come back empty-handed.

That is certain.

And therefore what they produce at home must be not only enough for
themselves, but such both in quantity and quality as to accommodate
those from whom their wants are supplied.

Very true.

Then more husbandmen and more artisans will be required?

They will.

Not to mention the importers and exporters, who are called merchants?

Yes.

Then we shall want merchants?

We shall.

And if merchandise is to be carried over the sea, skilful sailors will
also be needed, and in considerable numbers?

Yes, in considerable numbers.

Then, again, within the city, how will they exchange their productions?
To secure such an exchange was, as you will remember, one of our
principal objects when we formed them into a society and constituted a
State.

Clearly they will buy and sell.

Then they will need a market-place, and a money-token for purposes of
exchange.

Certainly.

Suppose now that a husbandman, or an artisan, brings some production
to market, and he comes at a time when there is no one to exchange with
him,--is he to leave his calling and sit idle in the market-place?

Not at all; he will find people there who, seeing the want, undertake
the office of salesmen. In well-ordered states they are commonly those
who are the weakest in bodily strength, and therefore of little use for
any other purpose; their duty is to be in the market, and to give money
in exchange for goods to those who desire to sell and to take money from
those who desire to buy.

This want, then, creates a class of retail-traders in our State. Is
not ``retailer'' the term which is applied to those who sit in the
market-place engaged in buying and selling, while those who wander from
one city to another are called merchants?

Yes, he said.

And there is another class of servants, who are intellectually hardly
on the level of companionship; still they have plenty of bodily strength
for labour, which accordingly they sell, and are called, if I do not
mistake, hirelings, hire being the name which is given to the price of
their labour.

True.

Then hirelings will help to make up our population?

Yes.

And now, Adeimantus, is our State matured and perfected?

I think so.

Where, then, is justice, and where is injustice, and in what part of the
State did they spring up?

Probably in the dealings of these citizens with one another. I cannot
imagine that they are more likely to be found any where else.

I dare say that you are right in your suggestion, I said; we had better
think the matter out, and not shrink from the enquiry.

Let us then consider, first of all, what will be their way of life,
now that we have thus established them. Will they not produce corn, and
wine, and clothes, and shoes, and build houses for themselves? And
when they are housed, they will work, in summer, commonly, stripped and
barefoot, but in winter substantially clothed and shod. They will feed
on barley-meal and flour of wheat, baking and kneading them, making
noble cakes and loaves; these they will serve up on a mat of reeds or on
clean leaves, themselves reclining the while upon beds strewn with yew
or myrtle. And they and their children will feast, drinking of the wine
which they have made, wearing garlands on their heads, and hymning the
praises of the gods, in happy converse with one another. And they will
take care that their families do not exceed their means; having an eye
to poverty or war.

But, said Glaucon, interposing, you have not given them a relish to
their meal.

True, I replied, I had forgotten; of course they must have a
relish--salt, and olives, and cheese, and they will boil roots and herbs
such as country people prepare; for a dessert we shall give them figs,
and peas, and beans; and they will roast myrtle-berries and acorns
at the fire, drinking in moderation. And with such a diet they may be
expected to live in peace and health to a good old age, and bequeath a
similar life to their children after them.

Yes, Socrates, he said, and if you were providing for a city of pigs,
how else would you feed the beasts?

But what would you have, Glaucon? I replied.

Why, he said, you should give them the ordinary conveniences of life.
People who are to be comfortable are accustomed to lie on sofas, and
dine off tables, and they should have sauces and sweets in the modern
style.

Yes, I said, now I understand: the question which you would have me
consider is, not only how a State, but how a luxurious State is created;
and possibly there is no harm in this, for in such a State we shall be
more likely to see how justice and injustice originate. In my opinion
the true and healthy constitution of the State is the one which I have
described. But if you wish also to see a State at fever-heat, I have
no objection. For I suspect that many will not be satisfied with the
simpler way of life. They will be for adding sofas, and tables,
and other furniture; also dainties, and perfumes, and incense, and
courtesans, and cakes, all these not of one sort only, but in every
variety; we must go beyond the necessaries of which I was at first
speaking, such as houses, and clothes, and shoes: the arts of the
painter and the embroiderer will have to be set in motion, and gold and
ivory and all sorts of materials must be procured.

True, he said.

Then we must enlarge our borders; for the original healthy State is
no longer sufficient. Now will the city have to fill and swell with a
multitude of callings which are not required by any natural want; such
as the whole tribe of hunters and actors, of whom one large class
have to do with forms and colours; another will be the votaries of
music--poets and their attendant train of rhapsodists, players, dancers,
contractors; also makers of divers kinds of articles, including women's
dresses. And we shall want more servants. Will not tutors be also in
request, and nurses wet and dry, tirewomen and barbers, as well as
confectioners and cooks; and swineherds, too, who were not needed and
therefore had no place in the former edition of our State, but are
needed now? They must not be forgotten: and there will be animals of
many other kinds, if people eat them.

Certainly.

And living in this way we shall have much greater need of physicians
than before?

Much greater.

And the country which was enough to support the original inhabitants
will be too small now, and not enough?

Quite true.

Then a slice of our neighbours'' land will be wanted by us for pasture
and tillage, and they will want a slice of ours, if, like ourselves,
they exceed the limit of necessity, and give themselves up to the
unlimited accumulation of wealth?

That, Socrates, will be inevitable.

And so we shall go to war, Glaucon. Shall we not?

Most certainly, he replied.

Then without determining as yet whether war does good or harm, thus much
we may affirm, that now we have discovered war to be derived from causes
which are also the causes of almost all the evils in States, private as
well as public.

Undoubtedly.

And our State must once more enlarge; and this time the enlargement will
be nothing short of a whole army, which will have to go out and fight
with the invaders for all that we have, as well as for the things and
persons whom we were describing above.

Why? he said; are they not capable of defending themselves?

No, I said; not if we were right in the principle which was acknowledged
by all of us when we were framing the State: the principle, as you will
remember, was that one man cannot practise many arts with success.

Very true, he said.

But is not war an art?

Certainly.

And an art requiring as much attention as shoemaking?

Quite true.

And the shoemaker was not allowed by us to be a husbandman, or a weaver,
or a builder--in order that we might have our shoes well made; but to
him and to every other worker was assigned one work for which he was by
nature fitted, and at that he was to continue working all his life long
and at no other; he was not to let opportunities slip, and then he would
become a good workman. Now nothing can be more important than that
the work of a soldier should be well done. But is war an art so easily
acquired that a man may be a warrior who is also a husbandman, or
shoemaker, or other artisan; although no one in the world would be a
good dice or draught player who merely took up the game as a recreation,
and had not from his earliest years devoted himself to this and nothing
else? No tools will make a man a skilled workman, or master of defence,
nor be of any use to him who has not learned how to handle them, and has
never bestowed any attention upon them. How then will he who takes up
a shield or other implement of war become a good fighter all in a day,
whether with heavy-armed or any other kind of troops?

Yes, he said, the tools which would teach men their own use would be
beyond price.

And the higher the duties of the guardian, I said, the more time, and
skill, and art, and application will be needed by him?

No doubt, he replied.

Will he not also require natural aptitude for his calling?

Certainly.

Then it will be our duty to select, if we can, natures which are fitted
for the task of guarding the city?

It will.

And the selection will be no easy matter, I said; but we must be brave
and do our best.

We must.

Is not the noble youth very like a well-bred dog in respect of guarding
and watching?

What do you mean?

I mean that both of them ought to be quick to see, and swift to overtake
the enemy when they see him; and strong too if, when they have caught
him, they have to fight with him.

All these qualities, he replied, will certainly be required by them.

Well, and your guardian must be brave if he is to fight well?

Certainly.

And is he likely to be brave who has no spirit, whether horse or dog
or any other animal? Have you never observed how invincible and
unconquerable is spirit and how the presence of it makes the soul of any
creature to be absolutely fearless and indomitable?

I have.

Then now we have a clear notion of the bodily qualities which are
required in the guardian.

True.

And also of the mental ones; his soul is to be full of spirit?

Yes.

But are not these spirited natures apt to be savage with one another,
and with everybody else?

A difficulty by no means easy to overcome, he replied.

Whereas, I said, they ought to be dangerous to their enemies, and gentle
to their friends; if not, they will destroy themselves without waiting
for their enemies to destroy them.

True, he said.

What is to be done then? I said; how shall we find a gentle nature which
has also a great spirit, for the one is the contradiction of the other?

True.

He will not be a good guardian who is wanting in either of these two
qualities; and yet the combination of them appears to be impossible; and
hence we must infer that to be a good guardian is impossible.

I am afraid that what you say is true, he replied.

Here feeling perplexed I began to think over what had preceded.--My
friend, I said, no wonder that we are in a perplexity; for we have lost
sight of the image which we had before us.

What do you mean? he said.

I mean to say that there do exist natures gifted with those opposite
qualities.

And where do you find them?

Many animals, I replied, furnish examples of them; our friend the dog
is a very good one: you know that well-bred dogs are perfectly gentle to
their familiars and acquaintances, and the reverse to strangers.

Yes, I know.

Then there is nothing impossible or out of the order of nature in our
finding a guardian who has a similar combination of qualities?

Certainly not.

Would not he who is fitted to be a guardian, besides the spirited
nature, need to have the qualities of a philosopher?

I do not apprehend your meaning.

The trait of which I am speaking, I replied, may be also seen in the
dog, and is remarkable in the animal.

What trait?

Why, a dog, whenever he sees a stranger, is angry; when an acquaintance,
he welcomes him, although the one has never done him any harm, nor the
other any good. Did this never strike you as curious?

The matter never struck me before; but I quite recognise the truth of
your remark.

And surely this instinct of the dog is very charming;--your dog is a
true philosopher.

Why?

Why, because he distinguishes the face of a friend and of an enemy only
by the criterion of knowing and not knowing. And must not an animal be a
lover of learning who determines what he likes and dislikes by the test
of knowledge and ignorance?

Most assuredly.

And is not the love of learning the love of wisdom, which is philosophy?

They are the same, he replied.

And may we not say confidently of man also, that he who is likely to be
gentle to his friends and acquaintances, must by nature be a lover of
wisdom and knowledge?

That we may safely affirm.

Then he who is to be a really good and noble guardian of the State will
require to unite in himself philosophy and spirit and swiftness and
strength?

Undoubtedly.

Then we have found the desired natures; and now that we have found them,
how are they to be reared and educated? Is not this an enquiry which
may be expected to throw light on the greater enquiry which is our final
end--How do justice and injustice grow up in States? for we do not want
either to omit what is to the point or to draw out the argument to an
inconvenient length.

Adeimantus thought that the enquiry would be of great service to us.

Then, I said, my dear friend, the task must not be given up, even if
somewhat long.

Certainly not.

Come then, and let us pass a leisure hour in story-telling, and our
story shall be the education of our heroes.

By all means.

And what shall be their education? Can we find a better than the
traditional sort?--and this has two divisions, gymnastic for the body,
and music for the soul.

True.

Shall we begin education with music, and go on to gymnastic afterwards?

By all means.

And when you speak of music, do you include literature or not?

I do.

And literature may be either true or false?

Yes.

And the young should be trained in both kinds, and we begin with the
false?

I do not understand your meaning, he said.

You know, I said, that we begin by telling children stories which,
though not wholly destitute of truth, are in the main fictitious;
and these stories are told them when they are not of an age to learn
gymnastics.

Very true.

That was my meaning when I said that we must teach music before
gymnastics.

Quite right, he said.

You know also that the beginning is the most important part of any work,
especially in the case of a young and tender thing; for that is the time
at which the character is being formed and the desired impression is
more readily taken.

Quite true.

And shall we just carelessly allow children to hear any casual tales
which may be devised by casual persons, and to receive into their minds
ideas for the most part the very opposite of those which we should wish
them to have when they are grown up?

We cannot.

Then the first thing will be to establish a censorship of the writers of
fiction, and let the censors receive any tale of fiction which is good,
and reject the bad; and we will desire mothers and nurses to tell their
children the authorised ones only. Let them fashion the mind with such
tales, even more fondly than they mould the body with their hands; but
most of those which are now in use must be discarded.

Of what tales are you speaking? he said.

You may find a model of the lesser in the greater, I said; for they are
necessarily of the same type, and there is the same spirit in both of
them.

Very likely, he replied; but I do not as yet know what you would term
the greater.

Those, I said, which are narrated by Homer and Hesiod, and the rest of
the poets, who have ever been the great story-tellers of mankind.

But which stories do you mean, he said; and what fault do you find with
them?

A fault which is most serious, I said; the fault of telling a lie, and,
what is more, a bad lie.

But when is this fault committed?

Whenever an erroneous representation is made of the nature of gods and
heroes,--as when a painter paints a portrait not having the shadow of a
likeness to the original.

Yes, he said, that sort of thing is certainly very blameable; but what
are the stories which you mean?

First of all, I said, there was that greatest of all lies in high
places, which the poet told about Uranus, and which was a bad lie
too,--I mean what Hesiod says that Uranus did, and how Cronus retaliated
on him. The doings of Cronus, and the sufferings which in turn his son
inflicted upon him, even if they were true, ought certainly not to be
lightly told to young and thoughtless persons; if possible, they had
better be buried in silence. But if there is an absolute necessity
for their mention, a chosen few might hear them in a mystery, and
they should sacrifice not a common (Eleusinian) pig, but some huge and
unprocurable victim; and then the number of the hearers will be very few
indeed.

Why, yes, said he, those stories are extremely objectionable.

Yes, Adeimantus, they are stories not to be repeated in our State; the
young man should not be told that in committing the worst of crimes he
is far from doing anything outrageous; and that even if he chastises his
father when he does wrong, in whatever manner, he will only be following
the example of the first and greatest among the gods.

I entirely agree with you, he said; in my opinion those stories are
quite unfit to be repeated.

Neither, if we mean our future guardians to regard the habit of
quarrelling among themselves as of all things the basest, should
any word be said to them of the wars in heaven, and of the plots and
fightings of the gods against one another, for they are not true. No,
we shall never mention the battles of the giants, or let them be
embroidered on garments; and we shall be silent about the innumerable
other quarrels of gods and heroes with their friends and relatives.
If they would only believe us we would tell them that quarrelling
is unholy, and that never up to this time has there been any quarrel
between citizens; this is what old men and old women should begin by
telling children; and when they grow up, the poets also should be told
to compose for them in a similar spirit. But the narrative of Hephaestus
binding Here his mother, or how on another occasion Zeus sent him flying
for taking her part when she was being beaten, and all the battles of
the gods in Homer--these tales must not be admitted into our State,
whether they are supposed to have an allegorical meaning or not. For
a young person cannot judge what is allegorical and what is literal;
anything that he receives into his mind at that age is likely to become
indelible and unalterable; and therefore it is most important that the
tales which the young first hear should be models of virtuous thoughts.

There you are right, he replied; but if any one asks where are such
models to be found and of what tales are you speaking--how shall we
answer him?

I said to him, You and I, Adeimantus, at this moment are not poets,
but founders of a State: now the founders of a State ought to know the
general forms in which poets should cast their tales, and the limits
which must be observed by them, but to make the tales is not their
business.

Very true, he said; but what are these forms of theology which you mean?

Something of this kind, I replied:--God is always to be represented as
he truly is, whatever be the sort of poetry, epic, lyric or tragic, in
which the representation is given.

Right.

And is he not truly good? and must he not be represented as such?

Certainly.

And no good thing is hurtful?

No, indeed.

And that which is not hurtful hurts not?

Certainly not.

And that which hurts not does no evil?

No.

And can that which does no evil be a cause of evil?

Impossible.

And the good is advantageous?

Yes.

And therefore the cause of well-being?

Yes.

It follows therefore that the good is not the cause of all things, but
of the good only?

Assuredly.

Then God, if he be good, is not the author of all things, as the many
assert, but he is the cause of a few things only, and not of most things
that occur to men. For few are the goods of human life, and many are the
evils, and the good is to be attributed to God alone; of the evils the
causes are to be sought elsewhere, and not in him.

That appears to me to be most true, he said.

Then we must not listen to Homer or to any other poet who is guilty of
the folly of saying that two casks

``Lie at the threshold of Zeus, full of lots, one of good, the other of
evil lots,''

and that he to whom Zeus gives a mixture of the two

``Sometimes meets with evil fortune, at other times with good;''

but that he to whom is given the cup of unmingled ill,

``Him wild hunger drives o'er the beauteous earth.''

And again--

``Zeus, who is the dispenser of good and evil to us.''

And if any one asserts that the violation of oaths and treaties, which
was really the work of Pandarus, was brought about by Athene and Zeus,
or that the strife and contention of the gods was instigated by Themis
and Zeus, he shall not have our approval; neither will we allow our
young men to hear the words of Aeschylus, that

``God plants guilt among men when he desires utterly to destroy a house.''

And if a poet writes of the sufferings of Niobe--the subject of the
tragedy in which these iambic verses occur--or of the house of Pelops,
or of the Trojan war or on any similar theme, either we must not permit
him to say that these are the works of God, or if they are of God, he
must devise some explanation of them such as we are seeking; he must say
that God did what was just and right, and they were the better for being
punished; but that those who are punished are miserable, and that God
is the author of their misery--the poet is not to be permitted to say;
though he may say that the wicked are miserable because they require
to be punished, and are benefited by receiving punishment from God;
but that God being good is the author of evil to any one is to be
strenuously denied, and not to be said or sung or heard in verse or
prose by any one whether old or young in any well-ordered commonwealth.
Such a fiction is suicidal, ruinous, impious.

I agree with you, he replied, and am ready to give my assent to the law.

Let this then be one of our rules and principles concerning the gods, to
which our poets and reciters will be expected to conform,--that God is
not the author of all things, but of good only.

That will do, he said.

And what do you think of a second principle? Shall I ask you whether God
is a magician, and of a nature to appear insidiously now in one shape,
and now in another--sometimes himself changing and passing into
many forms, sometimes deceiving us with the semblance of such
transformations; or is he one and the same immutably fixed in his own
proper image?

I cannot answer you, he said, without more thought.

Well, I said; but if we suppose a change in anything, that change must
be effected either by the thing itself, or by some other thing?

Most certainly.

And things which are at their best are also least liable to be altered
or discomposed; for example, when healthiest and strongest, the human
frame is least liable to be affected by meats and drinks, and the plant
which is in the fullest vigour also suffers least from winds or the heat
of the sun or any similar causes.

Of course.

And will not the bravest and wisest soul be least confused or deranged
by any external influence?

True.

And the same principle, as I should suppose, applies to all composite
things--furniture, houses, garments: when good and well made, they are
least altered by time and circumstances.

Very true.

Then everything which is good, whether made by art or nature, or both,
is least liable to suffer change from without?

True.

But surely God and the things of God are in every way perfect?

Of course they are.

Then he can hardly be compelled by external influence to take many
shapes?

He cannot.

But may he not change and transform himself?

Clearly, he said, that must be the case if he is changed at all.

And will he then change himself for the better and fairer, or for the
worse and more unsightly?

If he change at all he can only change for the worse, for we cannot
suppose him to be deficient either in virtue or beauty.

Very true, Adeimantus; but then, would any one, whether God or man,
desire to make himself worse?

Impossible.

Then it is impossible that God should ever be willing to change; being,
as is supposed, the fairest and best that is conceivable, every God
remains absolutely and for ever in his own form.

That necessarily follows, he said, in my judgment.

Then, I said, my dear friend, let none of the poets tell us that

``The gods, taking the disguise of strangers from other lands, walk up
and down cities in all sorts of forms;''

and let no one slander Proteus and Thetis, neither let any one, either
in tragedy or in any other kind of poetry, introduce Here disguised in
the likeness of a priestess asking an alms

``For the life-giving daughters of Inachus the river of Argos;''

--let us have no more lies of that sort. Neither must we have mothers
under the influence of the poets scaring their children with a bad
version of these myths--telling how certain gods, as they say, ``Go about
by night in the likeness of so many strangers and in divers forms;'' but
let them take heed lest they make cowards of their children, and at the
same time speak blasphemy against the gods.

Heaven forbid, he said.

But although the gods are themselves unchangeable, still by witchcraft
and deception they may make us think that they appear in various forms?

Perhaps, he replied.

Well, but can you imagine that God will be willing to lie, whether in
word or deed, or to put forth a phantom of himself?

I cannot say, he replied.

Do you not know, I said, that the true lie, if such an expression may be
allowed, is hated of gods and men?

What do you mean? he said.

I mean that no one is willingly deceived in that which is the truest and
highest part of himself, or about the truest and highest matters; there,
above all, he is most afraid of a lie having possession of him.

Still, he said, I do not comprehend you.

The reason is, I replied, that you attribute some profound meaning to
my words; but I am only saying that deception, or being deceived
or uninformed about the highest realities in the highest part of
themselves, which is the soul, and in that part of them to have and to
hold the lie, is what mankind least like;--that, I say, is what they
utterly detest.

There is nothing more hateful to them.

And, as I was just now remarking, this ignorance in the soul of him who
is deceived may be called the true lie; for the lie in words is only a
kind of imitation and shadowy image of a previous affection of the soul,
not pure unadulterated falsehood. Am I not right?

Perfectly right.

The true lie is hated not only by the gods, but also by men?

Yes.

Whereas the lie in words is in certain cases useful and not hateful; in
dealing with enemies--that would be an instance; or again, when those
whom we call our friends in a fit of madness or illusion are going to
do some harm, then it is useful and is a sort of medicine or
preventive; also in the tales of mythology, of which we were just now
speaking--because we do not know the truth about ancient times, we make
falsehood as much like truth as we can, and so turn it to account.

Very true, he said.

But can any of these reasons apply to God? Can we suppose that he is
ignorant of antiquity, and therefore has recourse to invention?

That would be ridiculous, he said.

Then the lying poet has no place in our idea of God?

I should say not.

Or perhaps he may tell a lie because he is afraid of enemies?

That is inconceivable.

But he may have friends who are senseless or mad?

But no mad or senseless person can be a friend of God.

Then no motive can be imagined why God should lie?

None whatever.

Then the superhuman and divine is absolutely incapable of falsehood?

Yes.

Then is God perfectly simple and true both in word and deed; he changes
not; he deceives not, either by sign or word, by dream or waking vision.

Your thoughts, he said, are the reflection of my own.

You agree with me then, I said, that this is the second type or form in
which we should write and speak about divine things. The gods are not
magicians who transform themselves, neither do they deceive mankind in
any way.

I grant that.

Then, although we are admirers of Homer, we do not admire the lying
dream which Zeus sends to Agamemnon; neither will we praise the verses
of Aeschylus in which Thetis says that Apollo at her nuptials

``Was celebrating in song her fair progeny whose days were to be long,
and to know no sickness. And when he had spoken of my lot as in all
things blessed of heaven he raised a note of triumph and cheered my
soul. And I thought that the word of Phoebus, being divine and full of
prophecy, would not fail. And now he himself who uttered the strain,
he who was present at the banquet, and who said this--he it is who has
slain my son.''

These are the kind of sentiments about the gods which will arouse our
anger; and he who utters them shall be refused a chorus; neither shall
we allow teachers to make use of them in the instruction of the young,
meaning, as we do, that our guardians, as far as men can be, should be
true worshippers of the gods and like them.

I entirely agree, he said, in these principles, and promise to make them
my laws.


% section book_ii (end)

% BOOK III.

% Such then, I said, are our principles of theology--some tales are to be
% told, and others are not to be told to our disciples from their youth
% upwards, if we mean them to honour the gods and their parents, and to
% value friendship with one another.

% Yes; and I think that our principles are right, he said.

% But if they are to be courageous, must they not learn other lessons
% besides these, and lessons of such a kind as will take away the fear of
% death? Can any man be courageous who has the fear of death in him?

% Certainly not, he said.

% And can he be fearless of death, or will he choose death in battle
% rather than defeat and slavery, who believes the world below to be real
% and terrible?

% Impossible.

% Then we must assume a control over the narrators of this class of tales
% as well as over the others, and beg them not simply to revile but rather
% to commend the world below, intimating to them that their descriptions
% are untrue, and will do harm to our future warriors.

% That will be our duty, he said.

% Then, I said, we shall have to obliterate many obnoxious passages,
% beginning with the verses,

% ``I would rather be a serf on the land of a poor and portionless man than
% rule over all the dead who have come to nought.''

% We must also expunge the verse, which tells us how Pluto feared,

% ``Lest the mansions grim and squalid which the gods abhor should be seen
% both of mortals and immortals.''

% And again:--

% ``O heavens! verily in the house of Hades there is soul and ghostly form
% but no mind at all!''

% Again of Tiresias:--

% ``(To him even after death did Persephone grant mind,) that he alone
% should be wise; but the other souls are flitting shades.''

% Again:--

% ``The soul flying from the limbs had gone to Hades, lamenting her fate,
% leaving manhood and youth.''

% Again:--

% ``And the soul, with shrilling cry, passed like smoke beneath the earth.''

% And,--

% ``As bats in hollow of mystic cavern, whenever any of them has dropped
% out of the string and falls from the rock, fly shrilling and cling
% to one another, so did they with shrilling cry hold together as they
% moved.''

% And we must beg Homer and the other poets not to be angry if we strike
% out these and similar passages, not because they are unpoetical, or
% unattractive to the popular ear, but because the greater the poetical
% charm of them, the less are they meet for the ears of boys and men who
% are meant to be free, and who should fear slavery more than death.

% Undoubtedly.

% Also we shall have to reject all the terrible and appalling names which
% describe the world below--Cocytus and Styx, ghosts under the earth, and
% sapless shades, and any similar words of which the very mention causes a
% shudder to pass through the inmost soul of him who hears them. I do not
% say that these horrible stories may not have a use of some kind; but
% there is a danger that the nerves of our guardians may be rendered too
% excitable and effeminate by them.

% There is a real danger, he said.

% Then we must have no more of them.

% True.

% Another and a nobler strain must be composed and sung by us.

% Clearly.

% And shall we proceed to get rid of the weepings and wailings of famous
% men?

% They will go with the rest.

% But shall we be right in getting rid of them? Reflect: our principle is
% that the good man will not consider death terrible to any other good man
% who is his comrade.

% Yes; that is our principle.

% And therefore he will not sorrow for his departed friend as though he
% had suffered anything terrible?

% He will not.

% Such an one, as we further maintain, is sufficient for himself and his
% own happiness, and therefore is least in need of other men.

% True, he said.

% And for this reason the loss of a son or brother, or the deprivation of
% fortune, is to him of all men least terrible.

% Assuredly.

% And therefore he will be least likely to lament, and will bear with the
% greatest equanimity any misfortune of this sort which may befall him.

% Yes, he will feel such a misfortune far less than another.

% Then we shall be right in getting rid of the lamentations of famous men,
% and making them over to women (and not even to women who are good for
% anything), or to men of a baser sort, that those who are being educated
% by us to be the defenders of their country may scorn to do the like.

% That will be very right.

% Then we will once more entreat Homer and the other poets not to depict
% Achilles, who is the son of a goddess, first lying on his side, then on
% his back, and then on his face; then starting up and sailing in a frenzy
% along the shores of the barren sea; now taking the sooty ashes in both
% his hands and pouring them over his head, or weeping and wailing in the
% various modes which Homer has delineated. Nor should he describe Priam
% the kinsman of the gods as praying and beseeching,

% ``Rolling in the dirt, calling each man loudly by his name.''

% Still more earnestly will we beg of him at all events not to introduce
% the gods lamenting and saying,

% ``Alas! my misery! Alas! that I bore the bravest to my sorrow.''

% But if he must introduce the gods, at any rate let him not dare so
% completely to misrepresent the greatest of the gods, as to make him
% say--

% ``O heavens! with my eyes verily I behold a dear friend of mine chased
% round and round the city, and my heart is sorrowful.''

% Or again:--

% Woe is me that I am fated to have Sarpedon, dearest of men to me,
% subdued at the hands of Patroclus the son of Menoetius.''

% For if, my sweet Adeimantus, our youth seriously listen to such unworthy
% representations of the gods, instead of laughing at them as they ought,
% hardly will any of them deem that he himself, being but a man, can be
% dishonoured by similar actions; neither will he rebuke any inclination
% which may arise in his mind to say and do the like. And instead
% of having any shame or self-control, he will be always whining and
% lamenting on slight occasions.

% Yes, he said, that is most true.

% Yes, I replied; but that surely is what ought not to be, as the argument
% has just proved to us; and by that proof we must abide until it is
% disproved by a better.

% It ought not to be.

% Neither ought our guardians to be given to laughter. For a fit of
% laughter which has been indulged to excess almost always produces a
% violent reaction.

% So I believe.

% Then persons of worth, even if only mortal men, must not be represented
% as overcome by laughter, and still less must such a representation of
% the gods be allowed.

% Still less of the gods, as you say, he replied.

% Then we shall not suffer such an expression to be used about the gods as
% that of Homer when he describes how

% ``Inextinguishable laughter arose among the blessed gods, when they saw
% Hephaestus bustling about the mansion.''

% On your views, we must not admit them.

% On my views, if you like to father them on me; that we must not admit
% them is certain.

% Again, truth should be highly valued; if, as we were saying, a lie is
% useless to the gods, and useful only as a medicine to men, then the
% use of such medicines should be restricted to physicians; private
% individuals have no business with them.

% Clearly not, he said.

% Then if any one at all is to have the privilege of lying, the rulers of
% the State should be the persons; and they, in their dealings either with
% enemies or with their own citizens, may be allowed to lie for the public
% good. But nobody else should meddle with anything of the kind; and
% although the rulers have this privilege, for a private man to lie to
% them in return is to be deemed a more heinous fault than for the patient
% or the pupil of a gymnasium not to speak the truth about his own bodily
% illnesses to the physician or to the trainer, or for a sailor not to
% tell the captain what is happening about the ship and the rest of the
% crew, and how things are going with himself or his fellow sailors.

% Most true, he said.

% If, then, the ruler catches anybody beside himself lying in the State,

% ``Any of the craftsmen, whether he be priest or physician or carpenter,''

% he will punish him for introducing a practice which is equally
% subversive and destructive of ship or State.

% Most certainly, he said, if our idea of the State is ever carried out.

% In the next place our youth must be temperate?

% Certainly.

% Are not the chief elements of temperance, speaking generally, obedience
% to commanders and self-control in sensual pleasures?

% True.

% Then we shall approve such language as that of Diomede in Homer,

% ``Friend, sit still and obey my word,''

% and the verses which follow,

% ``The Greeks marched breathing prowess, ...in silent awe of their
% leaders,''

% and other sentiments of the same kind.

% We shall.

% What of this line,

% ``O heavy with wine, who hast the eyes of a dog and the heart of a stag,''

% and of the words which follow? Would you say that these, or any similar
% impertinences which private individuals are supposed to address to their
% rulers, whether in verse or prose, are well or ill spoken?

% They are ill spoken.

% They may very possibly afford some amusement, but they do not conduce
% to temperance. And therefore they are likely to do harm to our young
% men--you would agree with me there?

% Yes.

% And then, again, to make the wisest of men say that nothing in his
% opinion is more glorious than

% ``When the tables are full of bread and meat, and the cup-bearer carries
% round wine which he draws from the bowl and pours into the cups,''

% is it fit or conducive to temperance for a young man to hear such words?
% Or the verse

% ``The saddest of fates is to die and meet destiny from hunger?''

% What would you say again to the tale of Zeus, who, while other gods and
% men were asleep and he the only person awake, lay devising plans, but
% forgot them all in a moment through his lust, and was so completely
% overcome at the sight of Here that he would not even go into the hut,
% but wanted to lie with her on the ground, declaring that he had never
% been in such a state of rapture before, even when they first met one
% another

% ``Without the knowledge of their parents;''

% or that other tale of how Hephaestus, because of similar goings on, cast
% a chain around Ares and Aphrodite?

% Indeed, he said, I am strongly of opinion that they ought not to hear
% that sort of thing.

% But any deeds of endurance which are done or told by famous men, these
% they ought to see and hear; as, for example, what is said in the verses,

% ``He smote his breast, and thus reproached his heart, Endure, my heart;
% far worse hast thou endured!''

% Certainly, he said.

% In the next place, we must not let them be receivers of gifts or lovers
% of money.

% Certainly not.

% Neither must we sing to them of

% ``Gifts persuading gods, and persuading reverend kings.''

% Neither is Phoenix, the tutor of Achilles, to be approved or deemed to
% have given his pupil good counsel when he told him that he should take
% the gifts of the Greeks and assist them; but that without a gift he
% should not lay aside his anger. Neither will we believe or acknowledge
% Achilles himself to have been such a lover of money that he took
% Agamemnon's gifts, or that when he had received payment he restored the
% dead body of Hector, but that without payment he was unwilling to do so.

% Undoubtedly, he said, these are not sentiments which can be approved.

% Loving Homer as I do, I hardly like to say that in attributing these
% feelings to Achilles, or in believing that they are truly attributed
% to him, he is guilty of downright impiety. As little can I believe the
% narrative of his insolence to Apollo, where he says,

% ``Thou hast wronged me, O far-darter, most abominable of deities. Verily
% I would be even with thee, if I had only the power;''

% or his insubordination to the river-god, on whose divinity he is ready
% to lay hands; or his offering to the dead Patroclus of his own hair,
% which had been previously dedicated to the other river-god Spercheius,
% and that he actually performed this vow; or that he dragged Hector round
% the tomb of Patroclus, and slaughtered the captives at the pyre; of all
% this I cannot believe that he was guilty, any more than I can allow
% our citizens to believe that he, the wise Cheiron's pupil, the son of a
% goddess and of Peleus who was the gentlest of men and third in descent
% from Zeus, was so disordered in his wits as to be at one time the slave
% of two seemingly inconsistent passions, meanness, not untainted by
% avarice, combined with overweening contempt of gods and men.

% You are quite right, he replied.

% And let us equally refuse to believe, or allow to be repeated, the tale
% of Theseus son of Poseidon, or of Peirithous son of Zeus, going forth as
% they did to perpetrate a horrid rape; or of any other hero or son of
% a god daring to do such impious and dreadful things as they falsely
% ascribe to them in our day: and let us further compel the poets to
% declare either that these acts were not done by them, or that they
% were not the sons of gods;--both in the same breath they shall not be
% permitted to affirm. We will not have them trying to persuade our youth
% that the gods are the authors of evil, and that heroes are no better
% than men--sentiments which, as we were saying, are neither pious nor
% true, for we have already proved that evil cannot come from the gods.

% Assuredly not.

% And further they are likely to have a bad effect on those who hear them;
% for everybody will begin to excuse his own vices when he is convinced
% that similar wickednesses are always being perpetrated by--

% ``The kindred of the gods, the relatives of Zeus, whose ancestral altar,
% the altar of Zeus, is aloft in air on the peak of Ida,''

% and who have

% ``the blood of deities yet flowing in their veins.''

% And therefore let us put an end to such tales, lest they engender laxity
% of morals among the young.

% By all means, he replied.

% But now that we are determining what classes of subjects are or are not
% to be spoken of, let us see whether any have been omitted by us. The
% manner in which gods and demigods and heroes and the world below should
% be treated has been already laid down.

% Very true.

% And what shall we say about men? That is clearly the remaining portion
% of our subject.

% Clearly so.

% But we are not in a condition to answer this question at present, my
% friend.

% Why not?

% Because, if I am not mistaken, we shall have to say that about men poets
% and story-tellers are guilty of making the gravest misstatements when
% they tell us that wicked men are often happy, and the good miserable;
% and that injustice is profitable when undetected, but that justice is a
% man's own loss and another's gain--these things we shall forbid them to
% utter, and command them to sing and say the opposite.

% To be sure we shall, he replied.

% But if you admit that I am right in this, then I shall maintain that you
% have implied the principle for which we have been all along contending.

% I grant the truth of your inference.

% That such things are or are not to be said about men is a question which
% we cannot determine until we have discovered what justice is, and how
% naturally advantageous to the possessor, whether he seem to be just or
% not.

% Most true, he said.

% Enough of the subjects of poetry: let us now speak of the style; and
% when this has been considered, both matter and manner will have been
% completely treated.

% I do not understand what you mean, said Adeimantus.

% Then I must make you understand; and perhaps I may be more intelligible
% if I put the matter in this way. You are aware, I suppose, that all
% mythology and poetry is a narration of events, either past, present, or
% to come?

% Certainly, he replied.

% And narration may be either simple narration, or imitation, or a union
% of the two?

% That again, he said, I do not quite understand.

% I fear that I must be a ridiculous teacher when I have so much
% difficulty in making myself apprehended. Like a bad speaker, therefore,
% I will not take the whole of the subject, but will break a piece off in
% illustration of my meaning. You know the first lines of the Iliad,
% in which the poet says that Chryses prayed Agamemnon to release his
% daughter, and that Agamemnon flew into a passion with him; whereupon
% Chryses, failing of his object, invoked the anger of the God against the
% Achaeans. Now as far as these lines,

% ``And he prayed all the Greeks, but especially the two sons of Atreus,
% the chiefs of the people,''

% the poet is speaking in his own person; he never leads us to suppose
% that he is any one else. But in what follows he takes the person of
% Chryses, and then he does all that he can to make us believe that the
% speaker is not Homer, but the aged priest himself. And in this double
% form he has cast the entire narrative of the events which occurred at
% Troy and in Ithaca and throughout the Odyssey.

% Yes.

% And a narrative it remains both in the speeches which the poet recites
% from time to time and in the intermediate passages?

% Quite true.

% But when the poet speaks in the person of another, may we not say that
% he assimilates his style to that of the person who, as he informs you,
% is going to speak?

% Certainly.

% And this assimilation of himself to another, either by the use of voice
% or gesture, is the imitation of the person whose character he assumes?

% Of course.

% Then in this case the narrative of the poet may be said to proceed by
% way of imitation?

% Very true.

% Or, if the poet everywhere appears and never conceals himself, then
% again the imitation is dropped, and his poetry becomes simple narration.
% However, in order that I may make my meaning quite clear, and that you
% may no more say, ``I don't understand,'' I will show how the change might
% be effected. If Homer had said, ``The priest came, having his daughter's
% ransom in his hands, supplicating the Achaeans, and above all the
% kings;'' and then if, instead of speaking in the person of Chryses,
% he had continued in his own person, the words would have been, not
% imitation, but simple narration. The passage would have run as follows
% (I am no poet, and therefore I drop the metre), ``The priest came and
% prayed the gods on behalf of the Greeks that they might capture Troy
% and return safely home, but begged that they would give him back his
% daughter, and take the ransom which he brought, and respect the God.
% Thus he spoke, and the other Greeks revered the priest and assented. But
% Agamemnon was wroth, and bade him depart and not come again, lest the
% staff and chaplets of the God should be of no avail to him--the daughter
% of Chryses should not be released, he said--she should grow old with him
% in Argos. And then he told him to go away and not to provoke him, if he
% intended to get home unscathed. And the old man went away in fear and
% silence, and, when he had left the camp, he called upon Apollo by his
% many names, reminding him of everything which he had done pleasing to
% him, whether in building his temples, or in offering sacrifice, and
% praying that his good deeds might be returned to him, and that the
% Achaeans might expiate his tears by the arrows of the god,'--and so on.
% In this way the whole becomes simple narrative.

% I understand, he said.

% Or you may suppose the opposite case--that the intermediate passages are
% omitted, and the dialogue only left.

% That also, he said, I understand; you mean, for example, as in tragedy.

% You have conceived my meaning perfectly; and if I mistake not, what you
% failed to apprehend before is now made clear to you, that poetry and
% mythology are, in some cases, wholly imitative--instances of this are
% supplied by tragedy and comedy; there is likewise the opposite style,
% in which the poet is the only speaker--of this the dithyramb affords
% the best example; and the combination of both is found in epic, and in
% several other styles of poetry. Do I take you with me?

% Yes, he said; I see now what you meant.

% I will ask you to remember also what I began by saying, that we had done
% with the subject and might proceed to the style.

% Yes, I remember.

% In saying this, I intended to imply that we must come to an
% understanding about the mimetic art,--whether the poets, in narrating
% their stories, are to be allowed by us to imitate, and if so, whether
% in whole or in part, and if the latter, in what parts; or should all
% imitation be prohibited?

% You mean, I suspect, to ask whether tragedy and comedy shall be admitted
% into our State?

% Yes, I said; but there may be more than this in question: I really do
% not know as yet, but whither the argument may blow, thither we go.

% And go we will, he said.

% Then, Adeimantus, let me ask you whether our guardians ought to be
% imitators; or rather, has not this question been decided by the rule
% already laid down that one man can only do one thing well, and not many;
% and that if he attempt many, he will altogether fail of gaining much
% reputation in any?

% Certainly.

% And this is equally true of imitation; no one man can imitate many
% things as well as he would imitate a single one?

% He cannot.

% Then the same person will hardly be able to play a serious part in life,
% and at the same time to be an imitator and imitate many other parts as
% well; for even when two species of imitation are nearly allied, the same
% persons cannot succeed in both, as, for example, the writers of tragedy
% and comedy--did you not just now call them imitations?

% Yes, I did; and you are right in thinking that the same persons cannot
% succeed in both.

% Any more than they can be rhapsodists and actors at once?

% True.

% Neither are comic and tragic actors the same; yet all these things are
% but imitations.

% They are so.

% And human nature, Adeimantus, appears to have been coined into yet
% smaller pieces, and to be as incapable of imitating many things well, as
% of performing well the actions of which the imitations are copies.

% Quite true, he replied.

% If then we adhere to our original notion and bear in mind that
% our guardians, setting aside every other business, are to dedicate
% themselves wholly to the maintenance of freedom in the State, making
% this their craft, and engaging in no work which does not bear on this
% end, they ought not to practise or imitate anything else; if they
% imitate at all, they should imitate from youth upward only those
% characters which are suitable to their profession--the courageous,
% temperate, holy, free, and the like; but they should not depict or be
% skilful at imitating any kind of illiberality or baseness, lest from
% imitation they should come to be what they imitate. Did you never
% observe how imitations, beginning in early youth and continuing far into
% life, at length grow into habits and become a second nature, affecting
% body, voice, and mind?

% Yes, certainly, he said.

% Then, I said, we will not allow those for whom we profess a care and of
% whom we say that they ought to be good men, to imitate a woman, whether
% young or old, quarrelling with her husband, or striving and vaunting
% against the gods in conceit of her happiness, or when she is in
% affliction, or sorrow, or weeping; and certainly not one who is in
% sickness, love, or labour.

% Very right, he said.

% Neither must they represent slaves, male or female, performing the
% offices of slaves?

% They must not.

% And surely not bad men, whether cowards or any others, who do the
% reverse of what we have just been prescribing, who scold or mock or
% revile one another in drink or out of drink, or who in any other manner
% sin against themselves and their neighbours in word or deed, as the
% manner of such is. Neither should they be trained to imitate the action
% or speech of men or women who are mad or bad; for madness, like vice, is
% to be known but not to be practised or imitated.

% Very true, he replied.

% Neither may they imitate smiths or other artificers, or oarsmen, or
% boatswains, or the like?

% How can they, he said, when they are not allowed to apply their minds to
% the callings of any of these?

% Nor may they imitate the neighing of horses, the bellowing of bulls, the
% murmur of rivers and roll of the ocean, thunder, and all that sort of
% thing?

% Nay, he said, if madness be forbidden, neither may they copy the
% behaviour of madmen.

% You mean, I said, if I understand you aright, that there is one sort of
% narrative style which may be employed by a truly good man when he has
% anything to say, and that another sort will be used by a man of an
% opposite character and education.

% And which are these two sorts? he asked.

% Suppose, I answered, that a just and good man in the course of a
% narration comes on some saying or action of another good man,--I should
% imagine that he will like to personate him, and will not be ashamed of
% this sort of imitation: he will be most ready to play the part of the
% good man when he is acting firmly and wisely; in a less degree when
% he is overtaken by illness or love or drink, or has met with any other
% disaster. But when he comes to a character which is unworthy of him, he
% will not make a study of that; he will disdain such a person, and will
% assume his likeness, if at all, for a moment only when he is performing
% some good action; at other times he will be ashamed to play a part which
% he has never practised, nor will he like to fashion and frame himself
% after the baser models; he feels the employment of such an art, unless
% in jest, to be beneath him, and his mind revolts at it.

% So I should expect, he replied.

% Then he will adopt a mode of narration such as we have illustrated
% out of Homer, that is to say, his style will be both imitative and
% narrative; but there will be very little of the former, and a great deal
% of the latter. Do you agree?

% Certainly, he said; that is the model which such a speaker must
% necessarily take.

% But there is another sort of character who will narrate anything, and,
% the worse he is, the more unscrupulous he will be; nothing will be too
% bad for him: and he will be ready to imitate anything, not as a joke,
% but in right good earnest, and before a large company. As I was just now
% saying, he will attempt to represent the roll of thunder, the noise of
% wind and hail, or the creaking of wheels, and pulleys, and the various
% sounds of flutes, pipes, trumpets, and all sorts of instruments: he will
% bark like a dog, bleat like a sheep, or crow like a cock; his entire art
% will consist in imitation of voice and gesture, and there will be very
% little narration.

% That, he said, will be his mode of speaking.

% These, then, are the two kinds of style?

% Yes.

% And you would agree with me in saying that one of them is simple and has
% but slight changes; and if the harmony and rhythm are also chosen
% for their simplicity, the result is that the speaker, if he speaks
% correctly, is always pretty much the same in style, and he will keep
% within the limits of a single harmony (for the changes are not great),
% and in like manner he will make use of nearly the same rhythm?

% That is quite true, he said.

% Whereas the other requires all sorts of harmonies and all sorts of
% rhythms, if the music and the style are to correspond, because the style
% has all sorts of changes.

% That is also perfectly true, he replied.

% And do not the two styles, or the mixture of the two, comprehend all
% poetry, and every form of expression in words? No one can say anything
% except in one or other of them or in both together.

% They include all, he said.

% And shall we receive into our State all the three styles, or one only of
% the two unmixed styles? or would you include the mixed?

% I should prefer only to admit the pure imitator of virtue.

% Yes, I said, Adeimantus, but the mixed style is also very charming: and
% indeed the pantomimic, which is the opposite of the one chosen by you,
% is the most popular style with children and their attendants, and with
% the world in general.

% I do not deny it.

% But I suppose you would argue that such a style is unsuitable to our
% State, in which human nature is not twofold or manifold, for one man
% plays one part only?

% Yes; quite unsuitable.

% And this is the reason why in our State, and in our State only, we
% shall find a shoemaker to be a shoemaker and not a pilot also, and a
% husbandman to be a husbandman and not a dicast also, and a soldier a
% soldier and not a trader also, and the same throughout?

% True, he said.

% And therefore when any one of these pantomimic gentlemen, who are so
% clever that they can imitate anything, comes to us, and makes a proposal
% to exhibit himself and his poetry, we will fall down and worship him as
% a sweet and holy and wonderful being; but we must also inform him that
% in our State such as he are not permitted to exist; the law will not
% allow them. And so when we have anointed him with myrrh, and set a
% garland of wool upon his head, we shall send him away to another city.
% For we mean to employ for our souls'' health the rougher and severer poet
% or story-teller, who will imitate the style of the virtuous only, and
% will follow those models which we prescribed at first when we began the
% education of our soldiers.

% We certainly will, he said, if we have the power.

% Then now, my friend, I said, that part of music or literary education
% which relates to the story or myth may be considered to be finished; for
% the matter and manner have both been discussed.

% I think so too, he said.

% Next in order will follow melody and song.

% That is obvious.

% Every one can see already what we ought to say about them, if we are to
% be consistent with ourselves.

% I fear, said Glaucon, laughing, that the word ``every one'' hardly
% includes me, for I cannot at the moment say what they should be; though
% I may guess.

% At any rate you can tell that a song or ode has three parts--the words,
% the melody, and the rhythm; that degree of knowledge I may presuppose?

% Yes, he said; so much as that you may.

% And as for the words, there will surely be no difference between words
% which are and which are not set to music; both will conform to the same
% laws, and these have been already determined by us?

% Yes.

% And the melody and rhythm will depend upon the words?

% Certainly.

% We were saying, when we spoke of the subject-matter, that we had no need
% of lamentation and strains of sorrow?

% True.

% And which are the harmonies expressive of sorrow? You are musical, and
% can tell me.

% The harmonies which you mean are the mixed or tenor Lydian, and the
% full-toned or bass Lydian, and such like.

% These then, I said, must be banished; even to women who have a character
% to maintain they are of no use, and much less to men.

% Certainly.

% In the next place, drunkenness and softness and indolence are utterly
% unbecoming the character of our guardians.

% Utterly unbecoming.

% And which are the soft or drinking harmonies?

% The Ionian, he replied, and the Lydian; they are termed ``relaxed.''

% Well, and are these of any military use?

% Quite the reverse, he replied; and if so the Dorian and the Phrygian are
% the only ones which you have left.

% I answered: Of the harmonies I know nothing, but I want to have one
% warlike, to sound the note or accent which a brave man utters in the
% hour of danger and stern resolve, or when his cause is failing, and he
% is going to wounds or death or is overtaken by some other evil, and
% at every such crisis meets the blows of fortune with firm step and a
% determination to endure; and another to be used by him in times of peace
% and freedom of action, when there is no pressure of necessity, and he is
% seeking to persuade God by prayer, or man by instruction and admonition,
% or on the other hand, when he is expressing his willingness to yield to
% persuasion or entreaty or admonition, and which represents him when
% by prudent conduct he has attained his end, not carried away by his
% success, but acting moderately and wisely under the circumstances, and
% acquiescing in the event. These two harmonies I ask you to leave;
% the strain of necessity and the strain of freedom, the strain of the
% unfortunate and the strain of the fortunate, the strain of courage, and
% the strain of temperance; these, I say, leave.

% And these, he replied, are the Dorian and Phrygian harmonies of which I
% was just now speaking.

% Then, I said, if these and these only are to be used in our songs and
% melodies, we shall not want multiplicity of notes or a panharmonic
% scale?

% I suppose not.

% Then we shall not maintain the artificers of lyres with three
% corners and complex scales, or the makers of any other many-stringed
% curiously-harmonised instruments?

% Certainly not.

% But what do you say to flute-makers and flute-players? Would you admit
% them into our State when you reflect that in this composite use of
% harmony the flute is worse than all the stringed instruments put
% together; even the panharmonic music is only an imitation of the flute?

% Clearly not.

% There remain then only the lyre and the harp for use in the city, and
% the shepherds may have a pipe in the country.

% That is surely the conclusion to be drawn from the argument.

% The preferring of Apollo and his instruments to Marsyas and his
% instruments is not at all strange, I said.

% Not at all, he replied.

% And so, by the dog of Egypt, we have been unconsciously purging the
% State, which not long ago we termed luxurious.

% And we have done wisely, he replied.

% Then let us now finish the purgation, I said. Next in order to
% harmonies, rhythms will naturally follow, and they should be subject to
% the same rules, for we ought not to seek out complex systems of metre,
% or metres of every kind, but rather to discover what rhythms are the
% expressions of a courageous and harmonious life; and when we have found
% them, we shall adapt the foot and the melody to words having a like
% spirit, not the words to the foot and melody. To say what these rhythms
% are will be your duty--you must teach me them, as you have already
% taught me the harmonies.

% But, indeed, he replied, I cannot tell you. I only know that there
% are some three principles of rhythm out of which metrical systems are
% framed, just as in sounds there are four notes (i.e. the four notes of
% the tetrachord.) out of which all the harmonies are composed; that is
% an observation which I have made. But of what sort of lives they are
% severally the imitations I am unable to say.

% Then, I said, we must take Damon into our counsels; and he will tell us
% what rhythms are expressive of meanness, or insolence, or fury, or other
% unworthiness, and what are to be reserved for the expression of opposite
% feelings. And I think that I have an indistinct recollection of his
% mentioning a complex Cretic rhythm; also a dactylic or heroic, and he
% arranged them in some manner which I do not quite understand, making
% the rhythms equal in the rise and fall of the foot, long and short
% alternating; and, unless I am mistaken, he spoke of an iambic as well
% as of a trochaic rhythm, and assigned to them short and long quantities.
% Also in some cases he appeared to praise or censure the movement of the
% foot quite as much as the rhythm; or perhaps a combination of the two;
% for I am not certain what he meant. These matters, however, as I was
% saying, had better be referred to Damon himself, for the analysis of
% the subject would be difficult, you know? (Socrates expresses himself
% carelessly in accordance with his assumed ignorance of the details of
% the subject. In the first part of the sentence he appears to be speaking
% of paeonic rhythms which are in the ratio of 3/2; in the second part, of
% dactylic and anapaestic rhythms, which are in the ratio of 1/1; in the
% last clause, of iambic and trochaic rhythms, which are in the ratio of
% 1/2 or 2/1.)

% Rather so, I should say.

% But there is no difficulty in seeing that grace or the absence of grace
% is an effect of good or bad rhythm.

% None at all.

% And also that good and bad rhythm naturally assimilate to a good and bad
% style; and that harmony and discord in like manner follow style; for our
% principle is that rhythm and harmony are regulated by the words, and not
% the words by them.

% Just so, he said, they should follow the words.

% And will not the words and the character of the style depend on the
% temper of the soul?

% Yes.

% And everything else on the style?

% Yes.

% Then beauty of style and harmony and grace and good rhythm depend on
% simplicity,--I mean the true simplicity of a rightly and nobly ordered
% mind and character, not that other simplicity which is only an euphemism
% for folly?

% Very true, he replied.

% And if our youth are to do their work in life, must they not make these
% graces and harmonies their perpetual aim?

% They must.

% And surely the art of the painter and every other creative and
% constructive art are full of them,--weaving, embroidery, architecture,
% and every kind of manufacture; also nature, animal and vegetable,--in
% all of them there is grace or the absence of grace. And ugliness and
% discord and inharmonious motion are nearly allied to ill words and ill
% nature, as grace and harmony are the twin sisters of goodness and virtue
% and bear their likeness.

% That is quite true, he said.

% But shall our superintendence go no further, and are the poets only to
% be required by us to express the image of the good in their works, on
% pain, if they do anything else, of expulsion from our State? Or is the
% same control to be extended to other artists, and are they also to be
% prohibited from exhibiting the opposite forms of vice and intemperance
% and meanness and indecency in sculpture and building and the other
% creative arts; and is he who cannot conform to this rule of ours to be
% prevented from practising his art in our State, lest the taste of our
% citizens be corrupted by him? We would not have our guardians grow up
% amid images of moral deformity, as in some noxious pasture, and there
% browse and feed upon many a baneful herb and flower day by day, little
% by little, until they silently gather a festering mass of corruption
% in their own soul. Let our artists rather be those who are gifted to
% discern the true nature of the beautiful and graceful; then will our
% youth dwell in a land of health, amid fair sights and sounds, and
% receive the good in everything; and beauty, the effluence of fair works,
% shall flow into the eye and ear, like a health-giving breeze from a
% purer region, and insensibly draw the soul from earliest years into
% likeness and sympathy with the beauty of reason.

% There can be no nobler training than that, he replied.

% And therefore, I said, Glaucon, musical training is a more potent
% instrument than any other, because rhythm and harmony find their way
% into the inward places of the soul, on which they mightily fasten,
% imparting grace, and making the soul of him who is rightly educated
% graceful, or of him who is ill-educated ungraceful; and also because
% he who has received this true education of the inner being will most
% shrewdly perceive omissions or faults in art and nature, and with a true
% taste, while he praises and rejoices over and receives into his soul the
% good, and becomes noble and good, he will justly blame and hate the bad,
% now in the days of his youth, even before he is able to know the reason
% why; and when reason comes he will recognise and salute the friend with
% whom his education has made him long familiar.

% Yes, he said, I quite agree with you in thinking that our youth should
% be trained in music and on the grounds which you mention.

% Just as in learning to read, I said, we were satisfied when we knew
% the letters of the alphabet, which are very few, in all their recurring
% sizes and combinations; not slighting them as unimportant whether they
% occupy a space large or small, but everywhere eager to make them out;
% and not thinking ourselves perfect in the art of reading until we
% recognise them wherever they are found:

% True--

% Or, as we recognise the reflection of letters in the water, or in a
% mirror, only when we know the letters themselves; the same art and study
% giving us the knowledge of both:

% Exactly--

% Even so, as I maintain, neither we nor our guardians, whom we have to
% educate, can ever become musical until we and they know the essential
% forms of temperance, courage, liberality, magnificence, and their
% kindred, as well as the contrary forms, in all their combinations,
% and can recognise them and their images wherever they are found, not
% slighting them either in small things or great, but believing them all
% to be within the sphere of one art and study.

% Most assuredly.

% And when a beautiful soul harmonizes with a beautiful form, and the two
% are cast in one mould, that will be the fairest of sights to him who has
% an eye to see it?

% The fairest indeed.

% And the fairest is also the loveliest?

% That may be assumed.

% And the man who has the spirit of harmony will be most in love with the
% loveliest; but he will not love him who is of an inharmonious soul?

% That is true, he replied, if the deficiency be in his soul; but if there
% be any merely bodily defect in another he will be patient of it, and
% will love all the same.

% I perceive, I said, that you have or have had experiences of this sort,
% and I agree. But let me ask you another question: Has excess of pleasure
% any affinity to temperance?

% How can that be? he replied; pleasure deprives a man of the use of his
% faculties quite as much as pain.

% Or any affinity to virtue in general?

% None whatever.

% Any affinity to wantonness and intemperance?

% Yes, the greatest.

% And is there any greater or keener pleasure than that of sensual love?

% No, nor a madder.

% Whereas true love is a love of beauty and order--temperate and
% harmonious?

% Quite true, he said.

% Then no intemperance or madness should be allowed to approach true love?

% Certainly not.

% Then mad or intemperate pleasure must never be allowed to come near the
% lover and his beloved; neither of them can have any part in it if their
% love is of the right sort?

% No, indeed, Socrates, it must never come near them.

% Then I suppose that in the city which we are founding you would make a
% law to the effect that a friend should use no other familiarity to
% his love than a father would use to his son, and then only for a noble
% purpose, and he must first have the other's consent; and this rule is
% to limit him in all his intercourse, and he is never to be seen going
% further, or, if he exceeds, he is to be deemed guilty of coarseness and
% bad taste.

% I quite agree, he said.

% Thus much of music, which makes a fair ending; for what should be the
% end of music if not the love of beauty?

% I agree, he said.

% After music comes gymnastic, in which our youth are next to be trained.

% Certainly.

% Gymnastic as well as music should begin in early years; the training
% in it should be careful and should continue through life. Now my belief
% is,--and this is a matter upon which I should like to have your opinion
% in confirmation of my own, but my own belief is,--not that the good body
% by any bodily excellence improves the soul, but, on the contrary, that
% the good soul, by her own excellence, improves the body as far as this
% may be possible. What do you say?

% Yes, I agree.

% Then, to the mind when adequately trained, we shall be right in handing
% over the more particular care of the body; and in order to avoid
% prolixity we will now only give the general outlines of the subject.

% Very good.

% That they must abstain from intoxication has been already remarked by
% us; for of all persons a guardian should be the last to get drunk and
% not know where in the world he is.

% Yes, he said; that a guardian should require another guardian to take
% care of him is ridiculous indeed.

% But next, what shall we say of their food; for the men are in training
% for the great contest of all--are they not?

% Yes, he said.

% And will the habit of body of our ordinary athletes be suited to them?

% Why not?

% I am afraid, I said, that a habit of body such as they have is but a
% sleepy sort of thing, and rather perilous to health. Do you not observe
% that these athletes sleep away their lives, and are liable to most
% dangerous illnesses if they depart, in ever so slight a degree, from
% their customary regimen?

% Yes, I do.

% Then, I said, a finer sort of training will be required for our warrior
% athletes, who are to be like wakeful dogs, and to see and hear with the
% utmost keenness; amid the many changes of water and also of food, of
% summer heat and winter cold, which they will have to endure when on a
% campaign, they must not be liable to break down in health.

% That is my view.

% The really excellent gymnastic is twin sister of that simple music which
% we were just now describing.

% How so?

% Why, I conceive that there is a gymnastic which, like our music, is
% simple and good; and especially the military gymnastic.

% What do you mean?

% My meaning may be learned from Homer; he, you know, feeds his heroes at
% their feasts, when they are campaigning, on soldiers'' fare; they have
% no fish, although they are on the shores of the Hellespont, and they
% are not allowed boiled meats but only roast, which is the food most
% convenient for soldiers, requiring only that they should light a fire,
% and not involving the trouble of carrying about pots and pans.

% True.

% And I can hardly be mistaken in saying that sweet sauces are nowhere
% mentioned in Homer. In proscribing them, however, he is not singular;
% all professional athletes are well aware that a man who is to be in good
% condition should take nothing of the kind.

% Yes, he said; and knowing this, they are quite right in not taking them.

% Then you would not approve of Syracusan dinners, and the refinements of
% Sicilian cookery?

% I think not.

% Nor, if a man is to be in condition, would you allow him to have a
% Corinthian girl as his fair friend?

% Certainly not.

% Neither would you approve of the delicacies, as they are thought, of
% Athenian confectionary?

% Certainly not.

% All such feeding and living may be rightly compared by us to melody and
% song composed in the panharmonic style, and in all the rhythms.

% Exactly.

% There complexity engendered licence, and here disease; whereas
% simplicity in music was the parent of temperance in the soul; and
% simplicity in gymnastic of health in the body.

% Most true, he said.

% But when intemperance and diseases multiply in a State, halls of justice
% and medicine are always being opened; and the arts of the doctor and the
% lawyer give themselves airs, finding how keen is the interest which not
% only the slaves but the freemen of a city take about them.

% Of course.

% And yet what greater proof can there be of a bad and disgraceful state
% of education than this, that not only artisans and the meaner sort of
% people need the skill of first-rate physicians and judges, but also
% those who would profess to have had a liberal education? Is it not
% disgraceful, and a great sign of want of good-breeding, that a man
% should have to go abroad for his law and physic because he has none of
% his own at home, and must therefore surrender himself into the hands of
% other men whom he makes lords and judges over him?

% Of all things, he said, the most disgraceful.

% Would you say ``most,'' I replied, when you consider that there is
% a further stage of the evil in which a man is not only a life-long
% litigant, passing all his days in the courts, either as plaintiff or
% defendant, but is actually led by his bad taste to pride himself on his
% litigiousness; he imagines that he is a master in dishonesty; able to
% take every crooked turn, and wriggle into and out of every hole,
% bending like a withy and getting out of the way of justice: and all
% for what?--in order to gain small points not worth mentioning, he not
% knowing that so to order his life as to be able to do without a napping
% judge is a far higher and nobler sort of thing. Is not that still more
% disgraceful?

% Yes, he said, that is still more disgraceful.

% Well, I said, and to require the help of medicine, not when a wound
% has to be cured, or on occasion of an epidemic, but just because, by
% indolence and a habit of life such as we have been describing, men
% fill themselves with waters and winds, as if their bodies were a marsh,
% compelling the ingenious sons of Asclepius to find more names for
% diseases, such as flatulence and catarrh; is not this, too, a disgrace?

% Yes, he said, they do certainly give very strange and newfangled names
% to diseases.

% Yes, I said, and I do not believe that there were any such diseases in
% the days of Asclepius; and this I infer from the circumstance that the
% hero Eurypylus, after he has been wounded in Homer, drinks a posset of
% Pramnian wine well besprinkled with barley-meal and grated cheese, which
% are certainly inflammatory, and yet the sons of Asclepius who were
% at the Trojan war do not blame the damsel who gives him the drink, or
% rebuke Patroclus, who is treating his case.

% Well, he said, that was surely an extraordinary drink to be given to a
% person in his condition.

% Not so extraordinary, I replied, if you bear in mind that in former
% days, as is commonly said, before the time of Herodicus, the guild of
% Asclepius did not practise our present system of medicine, which may be
% said to educate diseases. But Herodicus, being a trainer, and himself of
% a sickly constitution, by a combination of training and doctoring found
% out a way of torturing first and chiefly himself, and secondly the rest
% of the world.

% How was that? he said.

% By the invention of lingering death; for he had a mortal disease which
% he perpetually tended, and as recovery was out of the question, he
% passed his entire life as a valetudinarian; he could do nothing but
% attend upon himself, and he was in constant torment whenever he departed
% in anything from his usual regimen, and so dying hard, by the help of
% science he struggled on to old age.

% A rare reward of his skill!

% Yes, I said; a reward which a man might fairly expect who never
% understood that, if Asclepius did not instruct his descendants
% in valetudinarian arts, the omission arose, not from ignorance or
% inexperience of such a branch of medicine, but because he knew that in
% all well-ordered states every individual has an occupation to which he
% must attend, and has therefore no leisure to spend in continually being
% ill. This we remark in the case of the artisan, but, ludicrously enough,
% do not apply the same rule to people of the richer sort.

% How do you mean? he said.

% I mean this: When a carpenter is ill he asks the physician for a rough
% and ready cure; an emetic or a purge or a cautery or the knife,--these
% are his remedies. And if some one prescribes for him a course of
% dietetics, and tells him that he must swathe and swaddle his head, and
% all that sort of thing, he replies at once that he has no time to be
% ill, and that he sees no good in a life which is spent in nursing
% his disease to the neglect of his customary employment; and therefore
% bidding good-bye to this sort of physician, he resumes his ordinary
% habits, and either gets well and lives and does his business, or, if his
% constitution fails, he dies and has no more trouble.

% Yes, he said, and a man in his condition of life ought to use the art of
% medicine thus far only.

% Has he not, I said, an occupation; and what profit would there be in his
% life if he were deprived of his occupation?

% Quite true, he said.

% But with the rich man this is otherwise; of him we do not say that he
% has any specially appointed work which he must perform, if he would
% live.

% He is generally supposed to have nothing to do.

% Then you never heard of the saying of Phocylides, that as soon as a man
% has a livelihood he should practise virtue?

% Nay, he said, I think that he had better begin somewhat sooner.

% Let us not have a dispute with him about this, I said; but rather ask
% ourselves: Is the practice of virtue obligatory on the rich man, or
% can he live without it? And if obligatory on him, then let us raise
% a further question, whether this dieting of disorders, which is an
% impediment to the application of the mind in carpentering and the
% mechanical arts, does not equally stand in the way of the sentiment of
% Phocylides?

% Of that, he replied, there can be no doubt; such excessive care of the
% body, when carried beyond the rules of gymnastic, is most inimical to
% the practice of virtue.

% Yes, indeed, I replied, and equally incompatible with the management of
% a house, an army, or an office of state; and, what is most important
% of all, irreconcileable with any kind of study or thought or
% self-reflection--there is a constant suspicion that headache and
% giddiness are to be ascribed to philosophy, and hence all practising or
% making trial of virtue in the higher sense is absolutely stopped; for
% a man is always fancying that he is being made ill, and is in constant
% anxiety about the state of his body.

% Yes, likely enough.

% And therefore our politic Asclepius may be supposed to have exhibited
% the power of his art only to persons who, being generally of healthy
% constitution and habits of life, had a definite ailment; such as these
% he cured by purges and operations, and bade them live as usual, herein
% consulting the interests of the State; but bodies which disease had
% penetrated through and through he would not have attempted to cure
% by gradual processes of evacuation and infusion: he did not want to
% lengthen out good-for-nothing lives, or to have weak fathers begetting
% weaker sons;--if a man was not able to live in the ordinary way he
% had no business to cure him; for such a cure would have been of no use
% either to himself, or to the State.

% Then, he said, you regard Asclepius as a statesman.

% Clearly; and his character is further illustrated by his sons. Note that
% they were heroes in the days of old and practised the medicines of which
% I am speaking at the siege of Troy: You will remember how, when Pandarus
% wounded Menelaus, they

% ``Sucked the blood out of the wound, and sprinkled soothing remedies,''

% but they never prescribed what the patient was afterwards to eat or
% drink in the case of Menelaus, any more than in the case of Eurypylus;
% the remedies, as they conceived, were enough to heal any man who before
% he was wounded was healthy and regular in his habits; and even though he
% did happen to drink a posset of Pramnian wine, he might get well all the
% same. But they would have nothing to do with unhealthy and intemperate
% subjects, whose lives were of no use either to themselves or others; the
% art of medicine was not designed for their good, and though they were as
% rich as Midas, the sons of Asclepius would have declined to attend them.

% They were very acute persons, those sons of Asclepius.

% Naturally so, I replied. Nevertheless, the tragedians and Pindar
% disobeying our behests, although they acknowledge that Asclepius was the
% son of Apollo, say also that he was bribed into healing a rich man
% who was at the point of death, and for this reason he was struck by
% lightning. But we, in accordance with the principle already affirmed by
% us, will not believe them when they tell us both;--if he was the son of
% a god, we maintain that he was not avaricious; or, if he was avaricious,
% he was not the son of a god.

% All that, Socrates, is excellent; but I should like to put a question to
% you: Ought there not to be good physicians in a State, and are not the
% best those who have treated the greatest number of constitutions good
% and bad? and are not the best judges in like manner those who are
% acquainted with all sorts of moral natures?

% Yes, I said, I too would have good judges and good physicians. But do
% you know whom I think good?

% Will you tell me?

% I will, if I can. Let me however note that in the same question you join
% two things which are not the same.

% How so? he asked.

% Why, I said, you join physicians and judges. Now the most skilful
% physicians are those who, from their youth upwards, have combined with
% the knowledge of their art the greatest experience of disease; they
% had better not be robust in health, and should have had all manner of
% diseases in their own persons. For the body, as I conceive, is not the
% instrument with which they cure the body; in that case we could not
% allow them ever to be or to have been sickly; but they cure the body
% with the mind, and the mind which has become and is sick can cure
% nothing.

% That is very true, he said.

% But with the judge it is otherwise; since he governs mind by mind; he
% ought not therefore to have been trained among vicious minds, and to
% have associated with them from youth upwards, and to have gone through
% the whole calendar of crime, only in order that he may quickly infer
% the crimes of others as he might their bodily diseases from his own
% self-consciousness; the honourable mind which is to form a healthy
% judgment should have had no experience or contamination of evil habits
% when young. And this is the reason why in youth good men often appear to
% be simple, and are easily practised upon by the dishonest, because they
% have no examples of what evil is in their own souls.

% Yes, he said, they are far too apt to be deceived.

% Therefore, I said, the judge should not be young; he should have learned
% to know evil, not from his own soul, but from late and long observation
% of the nature of evil in others: knowledge should be his guide, not
% personal experience.

% Yes, he said, that is the ideal of a judge.

% Yes, I replied, and he will be a good man (which is my answer to your
% question); for he is good who has a good soul. But the cunning and
% suspicious nature of which we spoke,--he who has committed many crimes,
% and fancies himself to be a master in wickedness, when he is amongst
% his fellows, is wonderful in the precautions which he takes, because he
% judges of them by himself: but when he gets into the company of men of
% virtue, who have the experience of age, he appears to be a fool again,
% owing to his unseasonable suspicions; he cannot recognise an honest man,
% because he has no pattern of honesty in himself; at the same time, as
% the bad are more numerous than the good, and he meets with them oftener,
% he thinks himself, and is by others thought to be, rather wise than
% foolish.

% Most true, he said.

% Then the good and wise judge whom we are seeking is not this man, but
% the other; for vice cannot know virtue too, but a virtuous nature,
% educated by time, will acquire a knowledge both of virtue and vice: the
% virtuous, and not the vicious, man has wisdom--in my opinion.

% And in mine also.

% This is the sort of medicine, and this is the sort of law, which you
% will sanction in your state. They will minister to better natures,
% giving health both of soul and of body; but those who are diseased in
% their bodies they will leave to die, and the corrupt and incurable souls
% they will put an end to themselves.

% That is clearly the best thing both for the patients and for the State.

% And thus our youth, having been educated only in that simple music
% which, as we said, inspires temperance, will be reluctant to go to law.

% Clearly.

% And the musician, who, keeping to the same track, is content to practise
% the simple gymnastic, will have nothing to do with medicine unless in
% some extreme case.

% That I quite believe.

% The very exercises and tolls which he undergoes are intended to
% stimulate the spirited element of his nature, and not to increase his
% strength; he will not, like common athletes, use exercise and regimen to
% develope his muscles.

% Very right, he said.

% Neither are the two arts of music and gymnastic really designed, as is
% often supposed, the one for the training of the soul, the other for the
% training of the body.

% What then is the real object of them?

% I believe, I said, that the teachers of both have in view chiefly the
% improvement of the soul.

% How can that be? he asked.

% Did you never observe, I said, the effect on the mind itself of
% exclusive devotion to gymnastic, or the opposite effect of an exclusive
% devotion to music?

% In what way shown? he said.

% The one producing a temper of hardness and ferocity, the other of
% softness and effeminacy, I replied.

% Yes, he said, I am quite aware that the mere athlete becomes too much of
% a savage, and that the mere musician is melted and softened beyond what
% is good for him.

% Yet surely, I said, this ferocity only comes from spirit, which, if
% rightly educated, would give courage, but, if too much intensified, is
% liable to become hard and brutal.

% That I quite think.

% On the other hand the philosopher will have the quality of gentleness.
% And this also, when too much indulged, will turn to softness, but, if
% educated rightly, will be gentle and moderate.

% True.

% And in our opinion the guardians ought to have both these qualities?

% Assuredly.

% And both should be in harmony?

% Beyond question.

% And the harmonious soul is both temperate and courageous?

% Yes.

% And the inharmonious is cowardly and boorish?

% Very true.

% And, when a man allows music to play upon him and to pour into his soul
% through the funnel of his ears those sweet and soft and melancholy airs
% of which we were just now speaking, and his whole life is passed in
% warbling and the delights of song; in the first stage of the process
% the passion or spirit which is in him is tempered like iron, and made
% useful, instead of brittle and useless. But, if he carries on the
% softening and soothing process, in the next stage he begins to melt and
% waste, until he has wasted away his spirit and cut out the sinews of his
% soul; and he becomes a feeble warrior.

% Very true.

% If the element of spirit is naturally weak in him the change is speedily
% accomplished, but if he have a good deal, then the power of music
% weakening the spirit renders him excitable;--on the least provocation
% he flames up at once, and is speedily extinguished; instead of having
% spirit he grows irritable and passionate and is quite impracticable.

% Exactly.

% And so in gymnastics, if a man takes violent exercise and is a great
% feeder, and the reverse of a great student of music and philosophy, at
% first the high condition of his body fills him with pride and spirit,
% and he becomes twice the man that he was.

% Certainly.

% And what happens? if he do nothing else, and holds no converse with the
% Muses, does not even that intelligence which there may be in him, having
% no taste of any sort of learning or enquiry or thought or culture,
% grow feeble and dull and blind, his mind never waking up or receiving
% nourishment, and his senses not being purged of their mists?

% True, he said.

% And he ends by becoming a hater of philosophy, uncivilized, never using
% the weapon of persuasion,--he is like a wild beast, all violence and
% fierceness, and knows no other way of dealing; and he lives in all
% ignorance and evil conditions, and has no sense of propriety and grace.

% That is quite true, he said.

% And as there are two principles of human nature, one the spirited
% and the other the philosophical, some God, as I should say, has given
% mankind two arts answering to them (and only indirectly to the soul
% and body), in order that these two principles (like the strings of
% an instrument) may be relaxed or drawn tighter until they are duly
% harmonized.

% That appears to be the intention.

% And he who mingles music with gymnastic in the fairest proportions, and
% best attempers them to the soul, may be rightly called the true musician
% and harmonist in a far higher sense than the tuner of the strings.

% You are quite right, Socrates.

% And such a presiding genius will be always required in our State if the
% government is to last.

% Yes, he will be absolutely necessary.

% Such, then, are our principles of nurture and education: Where would be
% the use of going into further details about the dances of our citizens,
% or about their hunting and coursing, their gymnastic and equestrian
% contests? For these all follow the general principle, and having found
% that, we shall have no difficulty in discovering them.

% I dare say that there will be no difficulty.

% Very good, I said; then what is the next question? Must we not ask who
% are to be rulers and who subjects?

% Certainly.

% There can be no doubt that the elder must rule the younger.

% Clearly.

% And that the best of these must rule.

% That is also clear.

% Now, are not the best husbandmen those who are most devoted to
% husbandry?

% Yes.

% And as we are to have the best of guardians for our city, must they not
% be those who have most the character of guardians?

% Yes.

% And to this end they ought to be wise and efficient, and to have a
% special care of the State?

% True.

% And a man will be most likely to care about that which he loves?

% To be sure.

% And he will be most likely to love that which he regards as having the
% same interests with himself, and that of which the good or evil fortune
% is supposed by him at any time most to affect his own?

% Very true, he replied.

% Then there must be a selection. Let us note among the guardians those
% who in their whole life show the greatest eagerness to do what is for
% the good of their country, and the greatest repugnance to do what is
% against her interests.

% Those are the right men.

% And they will have to be watched at every age, in order that we may see
% whether they preserve their resolution, and never, under the influence
% either of force or enchantment, forget or cast off their sense of duty
% to the State.

% How cast off? he said.

% I will explain to you, I replied. A resolution may go out of a man's
% mind either with his will or against his will; with his will when he
% gets rid of a falsehood and learns better, against his will whenever he
% is deprived of a truth.

% I understand, he said, the willing loss of a resolution; the meaning of
% the unwilling I have yet to learn.

% Why, I said, do you not see that men are unwillingly deprived of good,
% and willingly of evil? Is not to have lost the truth an evil, and to
% possess the truth a good? and you would agree that to conceive things as
% they are is to possess the truth?

% Yes, he replied; I agree with you in thinking that mankind are deprived
% of truth against their will.

% And is not this involuntary deprivation caused either by theft, or
% force, or enchantment?

% Still, he replied, I do not understand you.

% I fear that I must have been talking darkly, like the tragedians. I only
% mean that some men are changed by persuasion and that others forget;
% argument steals away the hearts of one class, and time of the other; and
% this I call theft. Now you understand me?

% Yes.

% Those again who are forced, are those whom the violence of some pain or
% grief compels to change their opinion.

% I understand, he said, and you are quite right.

% And you would also acknowledge that the enchanted are those who change
% their minds either under the softer influence of pleasure, or the
% sterner influence of fear?

% Yes, he said; everything that deceives may be said to enchant.

% Therefore, as I was just now saying, we must enquire who are the best
% guardians of their own conviction that what they think the interest
% of the State is to be the rule of their lives. We must watch them from
% their youth upwards, and make them perform actions in which they are
% most likely to forget or to be deceived, and he who remembers and is
% not deceived is to be selected, and he who fails in the trial is to be
% rejected. That will be the way?

% Yes.

% And there should also be toils and pains and conflicts prescribed for
% them, in which they will be made to give further proof of the same
% qualities.

% Very right, he replied.

% And then, I said, we must try them with enchantments--that is the third
% sort of test--and see what will be their behaviour: like those who take
% colts amid noise and tumult to see if they are of a timid nature, so
% must we take our youth amid terrors of some kind, and again pass them
% into pleasures, and prove them more thoroughly than gold is proved in
% the furnace, that we may discover whether they are armed against
% all enchantments, and of a noble bearing always, good guardians of
% themselves and of the music which they have learned, and retaining under
% all circumstances a rhythmical and harmonious nature, such as will be
% most serviceable to the individual and to the State. And he who at every
% age, as boy and youth and in mature life, has come out of the trial
% victorious and pure, shall be appointed a ruler and guardian of the
% State; he shall be honoured in life and death, and shall receive
% sepulture and other memorials of honour, the greatest that we have to
% give. But him who fails, we must reject. I am inclined to think that
% this is the sort of way in which our rulers and guardians should be
% chosen and appointed. I speak generally, and not with any pretension to
% exactness.

% And, speaking generally, I agree with you, he said.

% And perhaps the word ``guardian'' in the fullest sense ought to be applied
% to this higher class only who preserve us against foreign enemies and
% maintain peace among our citizens at home, that the one may not have the
% will, or the others the power, to harm us. The young men whom we
% before called guardians may be more properly designated auxiliaries and
% supporters of the principles of the rulers.

% I agree with you, he said.

% How then may we devise one of those needful falsehoods of which we
% lately spoke--just one royal lie which may deceive the rulers, if that
% be possible, and at any rate the rest of the city?

% What sort of lie? he said.

% Nothing new, I replied; only an old Phoenician tale (Laws) of what has
% often occurred before now in other places, (as the poets say, and have
% made the world believe,) though not in our time, and I do not know
% whether such an event could ever happen again, or could now even be made
% probable, if it did.

% How your words seem to hesitate on your lips!

% You will not wonder, I replied, at my hesitation when you have heard.

% Speak, he said, and fear not.

% Well then, I will speak, although I really know not how to look you
% in the face, or in what words to utter the audacious fiction, which
% I propose to communicate gradually, first to the rulers, then to the
% soldiers, and lastly to the people. They are to be told that their youth
% was a dream, and the education and training which they received from
% us, an appearance only; in reality during all that time they were being
% formed and fed in the womb of the earth, where they themselves and their
% arms and appurtenances were manufactured; when they were completed, the
% earth, their mother, sent them up; and so, their country being their
% mother and also their nurse, they are bound to advise for her good, and
% to defend her against attacks, and her citizens they are to regard as
% children of the earth and their own brothers.

% You had good reason, he said, to be ashamed of the lie which you were
% going to tell.

% True, I replied, but there is more coming; I have only told you half.
% Citizens, we shall say to them in our tale, you are brothers, yet God
% has framed you differently. Some of you have the power of command, and
% in the composition of these he has mingled gold, wherefore also
% they have the greatest honour; others he has made of silver, to be
% auxiliaries; others again who are to be husbandmen and craftsmen he has
% composed of brass and iron; and the species will generally be preserved
% in the children. But as all are of the same original stock, a golden
% parent will sometimes have a silver son, or a silver parent a golden
% son. And God proclaims as a first principle to the rulers, and above all
% else, that there is nothing which they should so anxiously guard, or of
% which they are to be such good guardians, as of the purity of the race.
% They should observe what elements mingle in their offspring; for if the
% son of a golden or silver parent has an admixture of brass and iron,
% then nature orders a transposition of ranks, and the eye of the ruler
% must not be pitiful towards the child because he has to descend in the
% scale and become a husbandman or artisan, just as there may be sons of
% artisans who having an admixture of gold or silver in them are raised
% to honour, and become guardians or auxiliaries. For an oracle says that
% when a man of brass or iron guards the State, it will be destroyed. Such
% is the tale; is there any possibility of making our citizens believe in
% it?

% Not in the present generation, he replied; there is no way of
% accomplishing this; but their sons may be made to believe in the tale,
% and their sons'' sons, and posterity after them.

% I see the difficulty, I replied; yet the fostering of such a belief will
% make them care more for the city and for one another. Enough, however,
% of the fiction, which may now fly abroad upon the wings of rumour, while
% we arm our earth-born heroes, and lead them forth under the command of
% their rulers. Let them look round and select a spot whence they can best
% suppress insurrection, if any prove refractory within, and also defend
% themselves against enemies, who like wolves may come down on the fold
% from without; there let them encamp, and when they have encamped, let
% them sacrifice to the proper Gods and prepare their dwellings.

% Just so, he said.

% And their dwellings must be such as will shield them against the cold of
% winter and the heat of summer.

% I suppose that you mean houses, he replied.

% Yes, I said; but they must be the houses of soldiers, and not of
% shop-keepers.

% What is the difference? he said.

% That I will endeavour to explain, I replied. To keep watch-dogs, who,
% from want of discipline or hunger, or some evil habit or other, would
% turn upon the sheep and worry them, and behave not like dogs but wolves,
% would be a foul and monstrous thing in a shepherd?

% Truly monstrous, he said.

% And therefore every care must be taken that our auxiliaries, being
% stronger than our citizens, may not grow to be too much for them and
% become savage tyrants instead of friends and allies?

% Yes, great care should be taken.

% And would not a really good education furnish the best safeguard?

% But they are well-educated already, he replied.

% I cannot be so confident, my dear Glaucon, I said; I am much more
% certain that they ought to be, and that true education, whatever that
% may be, will have the greatest tendency to civilize and humanize them
% in their relations to one another, and to those who are under their
% protection.

% Very true, he replied.

% And not only their education, but their habitations, and all that
% belongs to them, should be such as will neither impair their virtue as
% guardians, nor tempt them to prey upon the other citizens. Any man of
% sense must acknowledge that.

% He must.

% Then now let us consider what will be their way of life, if they are to
% realize our idea of them. In the first place, none of them should have
% any property of his own beyond what is absolutely necessary; neither
% should they have a private house or store closed against any one who has
% a mind to enter; their provisions should be only such as are required
% by trained warriors, who are men of temperance and courage; they should
% agree to receive from the citizens a fixed rate of pay, enough to meet
% the expenses of the year and no more; and they will go to mess and live
% together like soldiers in a camp. Gold and silver we will tell them
% that they have from God; the diviner metal is within them, and they have
% therefore no need of the dross which is current among men, and ought not
% to pollute the divine by any such earthly admixture; for that commoner
% metal has been the source of many unholy deeds, but their own is
% undefiled. And they alone of all the citizens may not touch or handle
% silver or gold, or be under the same roof with them, or wear them, or
% drink from them. And this will be their salvation, and they will be the
% saviours of the State. But should they ever acquire homes or lands
% or moneys of their own, they will become housekeepers and husbandmen
% instead of guardians, enemies and tyrants instead of allies of the other
% citizens; hating and being hated, plotting and being plotted against,
% they will pass their whole life in much greater terror of internal than
% of external enemies, and the hour of ruin, both to themselves and to the
% rest of the State, will be at hand. For all which reasons may we not
% say that thus shall our State be ordered, and that these shall be the
% regulations appointed by us for guardians concerning their houses and
% all other matters?

% Yes, said Glaucon.




% BOOK IV.

% Here Adeimantus interposed a question: How would you answer, Socrates,
% said he, if a person were to say that you are making these people
% miserable, and that they are the cause of their own unhappiness; the
% city in fact belongs to them, but they are none the better for it;
% whereas other men acquire lands, and build large and handsome houses,
% and have everything handsome about them, offering sacrifices to the gods
% on their own account, and practising hospitality; moreover, as you were
% saying just now, they have gold and silver, and all that is usual among
% the favourites of fortune; but our poor citizens are no better than
% mercenaries who are quartered in the city and are always mounting guard?

% Yes, I said; and you may add that they are only fed, and not paid in
% addition to their food, like other men; and therefore they cannot, if
% they would, take a journey of pleasure; they have no money to spend on
% a mistress or any other luxurious fancy, which, as the world goes, is
% thought to be happiness; and many other accusations of the same nature
% might be added.

% But, said he, let us suppose all this to be included in the charge.

% You mean to ask, I said, what will be our answer?

% Yes.

% If we proceed along the old path, my belief, I said, is that we shall
% find the answer. And our answer will be that, even as they are, our
% guardians may very likely be the happiest of men; but that our aim in
% founding the State was not the disproportionate happiness of any one
% class, but the greatest happiness of the whole; we thought that in a
% State which is ordered with a view to the good of the whole we should
% be most likely to find justice, and in the ill-ordered State injustice:
% and, having found them, we might then decide which of the two is the
% happier. At present, I take it, we are fashioning the happy State,
% not piecemeal, or with a view of making a few happy citizens, but as a
% whole; and by-and-by we will proceed to view the opposite kind of State.
% Suppose that we were painting a statue, and some one came up to us
% and said, Why do you not put the most beautiful colours on the most
% beautiful parts of the body--the eyes ought to be purple, but you have
% made them black--to him we might fairly answer, Sir, you would not
% surely have us beautify the eyes to such a degree that they are no
% longer eyes; consider rather whether, by giving this and the other
% features their due proportion, we make the whole beautiful. And so I say
% to you, do not compel us to assign to the guardians a sort of happiness
% which will make them anything but guardians; for we too can clothe our
% husbandmen in royal apparel, and set crowns of gold on their heads, and
% bid them till the ground as much as they like, and no more. Our potters
% also might be allowed to repose on couches, and feast by the fireside,
% passing round the winecup, while their wheel is conveniently at hand,
% and working at pottery only as much as they like; in this way we might
% make every class happy--and then, as you imagine, the whole State would
% be happy. But do not put this idea into our heads; for, if we listen
% to you, the husbandman will be no longer a husbandman, the potter will
% cease to be a potter, and no one will have the character of any distinct
% class in the State. Now this is not of much consequence where the
% corruption of society, and pretension to be what you are not, is
% confined to cobblers; but when the guardians of the laws and of the
% government are only seeming and not real guardians, then see how they
% turn the State upside down; and on the other hand they alone have the
% power of giving order and happiness to the State. We mean our guardians
% to be true saviours and not the destroyers of the State, whereas our
% opponent is thinking of peasants at a festival, who are enjoying a life
% of revelry, not of citizens who are doing their duty to the State. But,
% if so, we mean different things, and he is speaking of something which
% is not a State. And therefore we must consider whether in appointing
% our guardians we would look to their greatest happiness individually, or
% whether this principle of happiness does not rather reside in the State
% as a whole. But if the latter be the truth, then the guardians and
% auxiliaries, and all others equally with them, must be compelled or
% induced to do their own work in the best way. And thus the whole State
% will grow up in a noble order, and the several classes will receive the
% proportion of happiness which nature assigns to them.

% I think that you are quite right.

% I wonder whether you will agree with another remark which occurs to me.

% What may that be?

% There seem to be two causes of the deterioration of the arts.

% What are they?

% Wealth, I said, and poverty.

% How do they act?

% The process is as follows: When a potter becomes rich, will he, think
% you, any longer take the same pains with his art?

% Certainly not.

% He will grow more and more indolent and careless?

% Very true.

% And the result will be that he becomes a worse potter?

% Yes; he greatly deteriorates.

% But, on the other hand, if he has no money, and cannot provide himself
% with tools or instruments, he will not work equally well himself, nor
% will he teach his sons or apprentices to work equally well.

% Certainly not.

% Then, under the influence either of poverty or of wealth, workmen and
% their work are equally liable to degenerate?

% That is evident.

% Here, then, is a discovery of new evils, I said, against which
% the guardians will have to watch, or they will creep into the city
% unobserved.

% What evils?

% Wealth, I said, and poverty; the one is the parent of luxury and
% indolence, and the other of meanness and viciousness, and both of
% discontent.

% That is very true, he replied; but still I should like to know,
% Socrates, how our city will be able to go to war, especially against an
% enemy who is rich and powerful, if deprived of the sinews of war.

% There would certainly be a difficulty, I replied, in going to war with
% one such enemy; but there is no difficulty where there are two of them.

% How so? he asked.

% In the first place, I said, if we have to fight, our side will be
% trained warriors fighting against an army of rich men.

% That is true, he said.

% And do you not suppose, Adeimantus, that a single boxer who was
% perfect in his art would easily be a match for two stout and well-to-do
% gentlemen who were not boxers?

% Hardly, if they came upon him at once.

% What, now, I said, if he were able to run away and then turn and strike
% at the one who first came up? And supposing he were to do this several
% times under the heat of a scorching sun, might he not, being an expert,
% overturn more than one stout personage?

% Certainly, he said, there would be nothing wonderful in that.

% And yet rich men probably have a greater superiority in the science and
% practise of boxing than they have in military qualities.

% Likely enough.

% Then we may assume that our athletes will be able to fight with two or
% three times their own number?

% I agree with you, for I think you right.

% And suppose that, before engaging, our citizens send an embassy to one
% of the two cities, telling them what is the truth: Silver and gold we
% neither have nor are permitted to have, but you may; do you therefore
% come and help us in war, and take the spoils of the other city: Who,
% on hearing these words, would choose to fight against lean wiry dogs,
% rather than, with the dogs on their side, against fat and tender sheep?

% That is not likely; and yet there might be a danger to the poor State if
% the wealth of many States were to be gathered into one.

% But how simple of you to use the term State at all of any but our own!

% Why so?

% You ought to speak of other States in the plural number; not one of
% them is a city, but many cities, as they say in the game. For indeed any
% city, however small, is in fact divided into two, one the city of the
% poor, the other of the rich; these are at war with one another; and in
% either there are many smaller divisions, and you would be altogether
% beside the mark if you treated them all as a single State. But if you
% deal with them as many, and give the wealth or power or persons of the
% one to the others, you will always have a great many friends and not
% many enemies. And your State, while the wise order which has now been
% prescribed continues to prevail in her, will be the greatest of States,
% I do not mean to say in reputation or appearance, but in deed and truth,
% though she number not more than a thousand defenders. A single State
% which is her equal you will hardly find, either among Hellenes or
% barbarians, though many that appear to be as great and many times
% greater.

% That is most true, he said.

% And what, I said, will be the best limit for our rulers to fix when they
% are considering the size of the State and the amount of territory which
% they are to include, and beyond which they will not go?

% What limit would you propose?

% I would allow the State to increase so far as is consistent with unity;
% that, I think, is the proper limit.

% Very good, he said.

% Here then, I said, is another order which will have to be conveyed to
% our guardians: Let our city be accounted neither large nor small, but
% one and self-sufficing.

% And surely, said he, this is not a very severe order which we impose
% upon them.

% And the other, said I, of which we were speaking before is lighter
% still,--I mean the duty of degrading the offspring of the guardians when
% inferior, and of elevating into the rank of guardians the offspring of
% the lower classes, when naturally superior. The intention was, that, in
% the case of the citizens generally, each individual should be put to the
% use for which nature intended him, one to one work, and then every man
% would do his own business, and be one and not many; and so the whole
% city would be one and not many.

% Yes, he said; that is not so difficult.

% The regulations which we are prescribing, my good Adeimantus, are not,
% as might be supposed, a number of great principles, but trifles all,
% if care be taken, as the saying is, of the one great thing,--a thing,
% however, which I would rather call, not great, but sufficient for our
% purpose.

% What may that be? he asked.

% Education, I said, and nurture: If our citizens are well educated,
% and grow into sensible men, they will easily see their way through all
% these, as well as other matters which I omit; such, for example, as
% marriage, the possession of women and the procreation of children, which
% will all follow the general principle that friends have all things in
% common, as the proverb says.

% That will be the best way of settling them.

% Also, I said, the State, if once started well, moves with accumulating
% force like a wheel. For good nurture and education implant good
% constitutions, and these good constitutions taking root in a good
% education improve more and more, and this improvement affects the breed
% in man as in other animals.

% Very possibly, he said.

% Then to sum up: This is the point to which, above all, the attention of
% our rulers should be directed,--that music and gymnastic be preserved in
% their original form, and no innovation made. They must do their utmost
% to maintain them intact. And when any one says that mankind most regard

% ``The newest song which the singers have,''

% they will be afraid that he may be praising, not new songs, but a new
% kind of song; and this ought not to be praised, or conceived to be the
% meaning of the poet; for any musical innovation is full of danger to the
% whole State, and ought to be prohibited. So Damon tells me, and I
% can quite believe him;--he says that when modes of music change, the
% fundamental laws of the State always change with them.

% Yes, said Adeimantus; and you may add my suffrage to Damon's and your
% own.

% Then, I said, our guardians must lay the foundations of their fortress
% in music?

% Yes, he said; the lawlessness of which you speak too easily steals in.

% Yes, I replied, in the form of amusement; and at first sight it appears
% harmless.

% Why, yes, he said, and there is no harm; were it not that little by
% little this spirit of licence, finding a home, imperceptibly penetrates
% into manners and customs; whence, issuing with greater force, it invades
% contracts between man and man, and from contracts goes on to laws and
% constitutions, in utter recklessness, ending at last, Socrates, by an
% overthrow of all rights, private as well as public.

% Is that true? I said.

% That is my belief, he replied.

% Then, as I was saying, our youth should be trained from the first in
% a stricter system, for if amusements become lawless, and the youths
% themselves become lawless, they can never grow up into well-conducted
% and virtuous citizens.

% Very true, he said.

% And when they have made a good beginning in play, and by the help of
% music have gained the habit of good order, then this habit of order, in
% a manner how unlike the lawless play of the others! will accompany them
% in all their actions and be a principle of growth to them, and if there
% be any fallen places in the State will raise them up again.

% Very true, he said.

% Thus educated, they will invent for themselves any lesser rules which
% their predecessors have altogether neglected.

% What do you mean?

% I mean such things as these:--when the young are to be silent before
% their elders; how they are to show respect to them by standing and
% making them sit; what honour is due to parents; what garments or shoes
% are to be worn; the mode of dressing the hair; deportment and manners in
% general. You would agree with me?

% Yes.

% But there is, I think, small wisdom in legislating about such
% matters,--I doubt if it is ever done; nor are any precise written
% enactments about them likely to be lasting.

% Impossible.

% It would seem, Adeimantus, that the direction in which education starts
% a man, will determine his future life. Does not like always attract
% like?

% To be sure.

% Until some one rare and grand result is reached which may be good, and
% may be the reverse of good?

% That is not to be denied.

% And for this reason, I said, I shall not attempt to legislate further
% about them.

% Naturally enough, he replied.

% Well, and about the business of the agora, and the ordinary dealings
% between man and man, or again about agreements with artisans; about
% insult and injury, or the commencement of actions, and the appointment
% of juries, what would you say? there may also arise questions about
% any impositions and exactions of market and harbour dues which may
% be required, and in general about the regulations of markets, police,
% harbours, and the like. But, oh heavens! shall we condescend to
% legislate on any of these particulars?

% I think, he said, that there is no need to impose laws about them on
% good men; what regulations are necessary they will find out soon enough
% for themselves.

% Yes, I said, my friend, if God will only preserve to them the laws which
% we have given them.

% And without divine help, said Adeimantus, they will go on for ever
% making and mending their laws and their lives in the hope of attaining
% perfection.

% You would compare them, I said, to those invalids who, having no
% self-restraint, will not leave off their habits of intemperance?

% Exactly.

% Yes, I said; and what a delightful life they lead! they are always
% doctoring and increasing and complicating their disorders, and always
% fancying that they will be cured by any nostrum which anybody advises
% them to try.

% Such cases are very common, he said, with invalids of this sort.

% Yes, I replied; and the charming thing is that they deem him their worst
% enemy who tells them the truth, which is simply that, unless they give
% up eating and drinking and wenching and idling, neither drug nor cautery
% nor spell nor amulet nor any other remedy will avail.

% Charming! he replied. I see nothing charming in going into a passion
% with a man who tells you what is right.

% These gentlemen, I said, do not seem to be in your good graces.

% Assuredly not.

% Nor would you praise the behaviour of States which act like the men whom
% I was just now describing. For are there not ill-ordered States in
% which the citizens are forbidden under pain of death to alter the
% constitution; and yet he who most sweetly courts those who live under
% this regime and indulges them and fawns upon them and is skilful in
% anticipating and gratifying their humours is held to be a great and
% good statesman--do not these States resemble the persons whom I was
% describing?

% Yes, he said; the States are as bad as the men; and I am very far from
% praising them.

% But do you not admire, I said, the coolness and dexterity of these ready
% ministers of political corruption?

% Yes, he said, I do; but not of all of them, for there are some whom
% the applause of the multitude has deluded into the belief that they are
% really statesmen, and these are not much to be admired.

% What do you mean? I said; you should have more feeling for them. When a
% man cannot measure, and a great many others who cannot measure declare
% that he is four cubits high, can he help believing what they say?

% Nay, he said, certainly not in that case.

% Well, then, do not be angry with them; for are they not as good as a
% play, trying their hand at paltry reforms such as I was describing; they
% are always fancying that by legislation they will make an end of frauds
% in contracts, and the other rascalities which I was mentioning, not
% knowing that they are in reality cutting off the heads of a hydra?

% Yes, he said; that is just what they are doing.

% I conceive, I said, that the true legislator will not trouble
% himself with this class of enactments whether concerning laws or the
% constitution either in an ill-ordered or in a well-ordered State; for
% in the former they are quite useless, and in the latter there will be no
% difficulty in devising them; and many of them will naturally flow out of
% our previous regulations.

% What, then, he said, is still remaining to us of the work of
% legislation?

% Nothing to us, I replied; but to Apollo, the God of Delphi, there
% remains the ordering of the greatest and noblest and chiefest things of
% all.

% Which are they? he said.

% The institution of temples and sacrifices, and the entire service of
% gods, demigods, and heroes; also the ordering of the repositories of
% the dead, and the rites which have to be observed by him who would
% propitiate the inhabitants of the world below. These are matters of
% which we are ignorant ourselves, and as founders of a city we should be
% unwise in trusting them to any interpreter but our ancestral deity. He
% is the god who sits in the centre, on the navel of the earth, and he is
% the interpreter of religion to all mankind.

% You are right, and we will do as you propose.

% But where, amid all this, is justice? son of Ariston, tell me where. Now
% that our city has been made habitable, light a candle and search, and
% get your brother and Polemarchus and the rest of our friends to help,
% and let us see where in it we can discover justice and where injustice,
% and in what they differ from one another, and which of them the man who
% would be happy should have for his portion, whether seen or unseen by
% gods and men.

% Nonsense, said Glaucon: did you not promise to search yourself, saying
% that for you not to help justice in her need would be an impiety?

% I do not deny that I said so, and as you remind me, I will be as good as
% my word; but you must join.

% We will, he replied.

% Well, then, I hope to make the discovery in this way: I mean to begin
% with the assumption that our State, if rightly ordered, is perfect.

% That is most certain.

% And being perfect, is therefore wise and valiant and temperate and just.

% That is likewise clear.

% And whichever of these qualities we find in the State, the one which is
% not found will be the residue?

% Very good.

% If there were four things, and we were searching for one of them,
% wherever it might be, the one sought for might be known to us from the
% first, and there would be no further trouble; or we might know the other
% three first, and then the fourth would clearly be the one left.

% Very true, he said.

% And is not a similar method to be pursued about the virtues, which are
% also four in number?

% Clearly.

% First among the virtues found in the State, wisdom comes into view, and
% in this I detect a certain peculiarity.

% What is that?

% The State which we have been describing is said to be wise as being good
% in counsel?

% Very true.

% And good counsel is clearly a kind of knowledge, for not by ignorance,
% but by knowledge, do men counsel well?

% Clearly.

% And the kinds of knowledge in a State are many and diverse?

% Of course.

% There is the knowledge of the carpenter; but is that the sort of
% knowledge which gives a city the title of wise and good in counsel?

% Certainly not; that would only give a city the reputation of skill in
% carpentering.

% Then a city is not to be called wise because possessing a knowledge
% which counsels for the best about wooden implements?

% Certainly not.

% Nor by reason of a knowledge which advises about brazen pots, I said,
% nor as possessing any other similar knowledge?

% Not by reason of any of them, he said.

% Nor yet by reason of a knowledge which cultivates the earth; that would
% give the city the name of agricultural?

% Yes.

% Well, I said, and is there any knowledge in our recently-founded State
% among any of the citizens which advises, not about any particular thing
% in the State, but about the whole, and considers how a State can best
% deal with itself and with other States?

% There certainly is.

% And what is this knowledge, and among whom is it found? I asked.

% It is the knowledge of the guardians, he replied, and is found among
% those whom we were just now describing as perfect guardians.

% And what is the name which the city derives from the possession of this
% sort of knowledge?

% The name of good in counsel and truly wise.

% And will there be in our city more of these true guardians or more
% smiths?

% The smiths, he replied, will be far more numerous.

% Will not the guardians be the smallest of all the classes who receive a
% name from the profession of some kind of knowledge?

% Much the smallest.

% And so by reason of the smallest part or class, and of the knowledge
% which resides in this presiding and ruling part of itself, the whole
% State, being thus constituted according to nature, will be wise; and
% this, which has the only knowledge worthy to be called wisdom, has been
% ordained by nature to be of all classes the least.

% Most true.

% Thus, then, I said, the nature and place in the State of one of the four
% virtues has somehow or other been discovered.

% And, in my humble opinion, very satisfactorily discovered, he replied.

% Again, I said, there is no difficulty in seeing the nature of courage,
% and in what part that quality resides which gives the name of courageous
% to the State.

% How do you mean?

% Why, I said, every one who calls any State courageous or cowardly, will
% be thinking of the part which fights and goes out to war on the State's
% behalf.

% No one, he replied, would ever think of any other.

% The rest of the citizens may be courageous or may be cowardly, but their
% courage or cowardice will not, as I conceive, have the effect of making
% the city either the one or the other.

% Certainly not.

% The city will be courageous in virtue of a portion of herself which
% preserves under all circumstances that opinion about the nature of
% things to be feared and not to be feared in which our legislator
% educated them; and this is what you term courage.

% I should like to hear what you are saying once more, for I do not think
% that I perfectly understand you.

% I mean that courage is a kind of salvation.

% Salvation of what?

% Of the opinion respecting things to be feared, what they are and of
% what nature, which the law implants through education; and I mean by the
% words ``under all circumstances'' to intimate that in pleasure or in pain,
% or under the influence of desire or fear, a man preserves, and does not
% lose this opinion. Shall I give you an illustration?

% If you please.

% You know, I said, that dyers, when they want to dye wool for making the
% true sea-purple, begin by selecting their white colour first; this they
% prepare and dress with much care and pains, in order that the white
% ground may take the purple hue in full perfection. The dyeing then
% proceeds; and whatever is dyed in this manner becomes a fast colour,
% and no washing either with lyes or without them can take away the bloom.
% But, when the ground has not been duly prepared, you will have noticed
% how poor is the look either of purple or of any other colour.

% Yes, he said; I know that they have a washed-out and ridiculous
% appearance.

% Then now, I said, you will understand what our object was in selecting
% our soldiers, and educating them in music and gymnastic; we were
% contriving influences which would prepare them to take the dye of the
% laws in perfection, and the colour of their opinion about dangers and
% of every other opinion was to be indelibly fixed by their nurture
% and training, not to be washed away by such potent lyes as
% pleasure--mightier agent far in washing the soul than any soda or lye;
% or by sorrow, fear, and desire, the mightiest of all other solvents. And
% this sort of universal saving power of true opinion in conformity with
% law about real and false dangers I call and maintain to be courage,
% unless you disagree.

% But I agree, he replied; for I suppose that you mean to exclude mere
% uninstructed courage, such as that of a wild beast or of a slave--this,
% in your opinion, is not the courage which the law ordains, and ought to
% have another name.

% Most certainly.

% Then I may infer courage to be such as you describe?

% Why, yes, said I, you may, and if you add the words ``of a citizen,''
% you will not be far wrong;--hereafter, if you like, we will carry the
% examination further, but at present we are seeking not for courage but
% justice; and for the purpose of our enquiry we have said enough.

% You are right, he replied.

% Two virtues remain to be discovered in the State--first, temperance, and
% then justice which is the end of our search.

% Very true.

% Now, can we find justice without troubling ourselves about temperance?

% I do not know how that can be accomplished, he said, nor do I desire
% that justice should be brought to light and temperance lost sight of;
% and therefore I wish that you would do me the favour of considering
% temperance first.

% Certainly, I replied, I should not be justified in refusing your
% request.

% Then consider, he said.

% Yes, I replied; I will; and as far as I can at present see, the virtue
% of temperance has more of the nature of harmony and symphony than the
% preceding.

% How so? he asked.

% Temperance, I replied, is the ordering or controlling of certain
% pleasures and desires; this is curiously enough implied in the saying of
% ``a man being his own master;'' and other traces of the same notion may be
% found in language.

% No doubt, he said.

% There is something ridiculous in the expression ``master of himself;'' for
% the master is also the servant and the servant the master; and in all
% these modes of speaking the same person is denoted.

% Certainly.

% The meaning is, I believe, that in the human soul there is a better and
% also a worse principle; and when the better has the worse under control,
% then a man is said to be master of himself; and this is a term of
% praise: but when, owing to evil education or association, the better
% principle, which is also the smaller, is overwhelmed by the greater mass
% of the worse--in this case he is blamed and is called the slave of self
% and unprincipled.

% Yes, there is reason in that.

% And now, I said, look at our newly-created State, and there you will
% find one of these two conditions realized; for the State, as you
% will acknowledge, may be justly called master of itself, if the words
% ``temperance'' and ``self-mastery'' truly express the rule of the better
% part over the worse.

% Yes, he said, I see that what you say is true.

% Let me further note that the manifold and complex pleasures and desires
% and pains are generally found in children and women and servants, and in
% the freemen so called who are of the lowest and more numerous class.

% Certainly, he said.

% Whereas the simple and moderate desires which follow reason, and are
% under the guidance of mind and true opinion, are to be found only in a
% few, and those the best born and best educated.

% Very true.

% These two, as you may perceive, have a place in our State; and the
% meaner desires of the many are held down by the virtuous desires and
% wisdom of the few.

% That I perceive, he said.

% Then if there be any city which may be described as master of its own
% pleasures and desires, and master of itself, ours may claim such a
% designation?

% Certainly, he replied.

% It may also be called temperate, and for the same reasons?

% Yes.

% And if there be any State in which rulers and subjects will be agreed as
% to the question who are to rule, that again will be our State?

% Undoubtedly.

% And the citizens being thus agreed among themselves, in which class will
% temperance be found--in the rulers or in the subjects?

% In both, as I should imagine, he replied.

% Do you observe that we were not far wrong in our guess that temperance
% was a sort of harmony?

% Why so?

% Why, because temperance is unlike courage and wisdom, each of which
% resides in a part only, the one making the State wise and the other
% valiant; not so temperance, which extends to the whole, and runs through
% all the notes of the scale, and produces a harmony of the weaker and the
% stronger and the middle class, whether you suppose them to be stronger
% or weaker in wisdom or power or numbers or wealth, or anything else.
% Most truly then may we deem temperance to be the agreement of the
% naturally superior and inferior, as to the right to rule of either, both
% in states and individuals.

% I entirely agree with you.

% And so, I said, we may consider three out of the four virtues to have
% been discovered in our State. The last of those qualities which make a
% state virtuous must be justice, if we only knew what that was.

% The inference is obvious.

% The time then has arrived, Glaucon, when, like huntsmen, we should
% surround the cover, and look sharp that justice does not steal away, and
% pass out of sight and escape us; for beyond a doubt she is somewhere in
% this country: watch therefore and strive to catch a sight of her, and if
% you see her first, let me know.

% Would that I could! but you should regard me rather as a follower who
% has just eyes enough to see what you show him--that is about as much as
% I am good for.

% Offer up a prayer with me and follow.

% I will, but you must show me the way.

% Here is no path, I said, and the wood is dark and perplexing; still we
% must push on.

% Let us push on.

% Here I saw something: Halloo! I said, I begin to perceive a track, and I
% believe that the quarry will not escape.

% Good news, he said.

% Truly, I said, we are stupid fellows.

% Why so?

% Why, my good sir, at the beginning of our enquiry, ages ago, there was
% justice tumbling out at our feet, and we never saw her; nothing could be
% more ridiculous. Like people who go about looking for what they have
% in their hands--that was the way with us--we looked not at what we
% were seeking, but at what was far off in the distance; and therefore, I
% suppose, we missed her.

% What do you mean?

% I mean to say that in reality for a long time past we have been talking
% of justice, and have failed to recognise her.

% I grow impatient at the length of your exordium.

% Well then, tell me, I said, whether I am right or not: You remember the
% original principle which we were always laying down at the foundation
% of the State, that one man should practise one thing only, the thing to
% which his nature was best adapted;--now justice is this principle or a
% part of it.

% Yes, we often said that one man should do one thing only.

% Further, we affirmed that justice was doing one's own business, and not
% being a busybody; we said so again and again, and many others have said
% the same to us.

% Yes, we said so.

% Then to do one's own business in a certain way may be assumed to be
% justice. Can you tell me whence I derive this inference?

% I cannot, but I should like to be told.

% Because I think that this is the only virtue which remains in the
% State when the other virtues of temperance and courage and wisdom are
% abstracted; and, that this is the ultimate cause and condition of the
% existence of all of them, and while remaining in them is also their
% preservative; and we were saying that if the three were discovered by
% us, justice would be the fourth or remaining one.

% That follows of necessity.

% If we are asked to determine which of these four qualities by its
% presence contributes most to the excellence of the State, whether the
% agreement of rulers and subjects, or the preservation in the soldiers of
% the opinion which the law ordains about the true nature of dangers, or
% wisdom and watchfulness in the rulers, or whether this other which I am
% mentioning, and which is found in children and women, slave and freeman,
% artisan, ruler, subject,--the quality, I mean, of every one doing his
% own work, and not being a busybody, would claim the palm--the question
% is not so easily answered.

% Certainly, he replied, there would be a difficulty in saying which.

% Then the power of each individual in the State to do his own work
% appears to compete with the other political virtues, wisdom, temperance,
% courage.

% Yes, he said.

% And the virtue which enters into this competition is justice?

% Exactly.

% Let us look at the question from another point of view: Are not
% the rulers in a State those to whom you would entrust the office of
% determining suits at law?

% Certainly.

% And are suits decided on any other ground but that a man may neither
% take what is another's, nor be deprived of what is his own?

% Yes; that is their principle.

% Which is a just principle?

% Yes.

% Then on this view also justice will be admitted to be the having and
% doing what is a man's own, and belongs to him?

% Very true.

% Think, now, and say whether you agree with me or not. Suppose a
% carpenter to be doing the business of a cobbler, or a cobbler of a
% carpenter; and suppose them to exchange their implements or their
% duties, or the same person to be doing the work of both, or whatever be
% the change; do you think that any great harm would result to the State?

% Not much.

% But when the cobbler or any other man whom nature designed to be a
% trader, having his heart lifted up by wealth or strength or the number
% of his followers, or any like advantage, attempts to force his way
% into the class of warriors, or a warrior into that of legislators and
% guardians, for which he is unfitted, and either to take the implements
% or the duties of the other; or when one man is trader, legislator, and
% warrior all in one, then I think you will agree with me in saying that
% this interchange and this meddling of one with another is the ruin of
% the State.

% Most true.

% Seeing then, I said, that there are three distinct classes, any meddling
% of one with another, or the change of one into another, is the greatest
% harm to the State, and may be most justly termed evil-doing?

% Precisely.

% And the greatest degree of evil-doing to one's own city would be termed
% by you injustice?

% Certainly.

% This then is injustice; and on the other hand when the trader, the
% auxiliary, and the guardian each do their own business, that is justice,
% and will make the city just.

% I agree with you.

% We will not, I said, be over-positive as yet; but if, on trial, this
% conception of justice be verified in the individual as well as in
% the State, there will be no longer any room for doubt; if it be not
% verified, we must have a fresh enquiry. First let us complete the old
% investigation, which we began, as you remember, under the impression
% that, if we could previously examine justice on the larger scale, there
% would be less difficulty in discerning her in the individual. That
% larger example appeared to be the State, and accordingly we constructed
% as good a one as we could, knowing well that in the good State justice
% would be found. Let the discovery which we made be now applied to the
% individual--if they agree, we shall be satisfied; or, if there be a
% difference in the individual, we will come back to the State and
% have another trial of the theory. The friction of the two when rubbed
% together may possibly strike a light in which justice will shine forth,
% and the vision which is then revealed we will fix in our souls.

% That will be in regular course; let us do as you say.

% I proceeded to ask: When two things, a greater and less, are called by
% the same name, are they like or unlike in so far as they are called the
% same?

% Like, he replied.

% The just man then, if we regard the idea of justice only, will be like
% the just State?

% He will.

% And a State was thought by us to be just when the three classes in the
% State severally did their own business; and also thought to be temperate
% and valiant and wise by reason of certain other affections and qualities
% of these same classes?

% True, he said.

% And so of the individual; we may assume that he has the same three
% principles in his own soul which are found in the State; and he may be
% rightly described in the same terms, because he is affected in the same
% manner?

% Certainly, he said.

% Once more then, O my friend, we have alighted upon an easy
% question--whether the soul has these three principles or not?

% An easy question! Nay, rather, Socrates, the proverb holds that hard is
% the good.

% Very true, I said; and I do not think that the method which we are
% employing is at all adequate to the accurate solution of this question;
% the true method is another and a longer one. Still we may arrive at a
% solution not below the level of the previous enquiry.

% May we not be satisfied with that? he said;--under the circumstances, I
% am quite content.

% I too, I replied, shall be extremely well satisfied.

% Then faint not in pursuing the speculation, he said.

% Must we not acknowledge, I said, that in each of us there are the same
% principles and habits which there are in the State; and that from the
% individual they pass into the State?--how else can they come there? Take
% the quality of passion or spirit;--it would be ridiculous to imagine
% that this quality, when found in States, is not derived from the
% individuals who are supposed to possess it, e.g. the Thracians,
% Scythians, and in general the northern nations; and the same may be said
% of the love of knowledge, which is the special characteristic of our
% part of the world, or of the love of money, which may, with equal truth,
% be attributed to the Phoenicians and Egyptians.

% Exactly so, he said.

% There is no difficulty in understanding this.

% None whatever.

% But the question is not quite so easy when we proceed to ask whether
% these principles are three or one; whether, that is to say, we learn
% with one part of our nature, are angry with another, and with a third
% part desire the satisfaction of our natural appetites; or whether the
% whole soul comes into play in each sort of action--to determine that is
% the difficulty.

% Yes, he said; there lies the difficulty.

% Then let us now try and determine whether they are the same or
% different.

% How can we? he asked.

% I replied as follows: The same thing clearly cannot act or be acted upon
% in the same part or in relation to the same thing at the same time,
% in contrary ways; and therefore whenever this contradiction occurs in
% things apparently the same, we know that they are really not the same,
% but different.

% Good.

% For example, I said, can the same thing be at rest and in motion at the
% same time in the same part?

% Impossible.

% Still, I said, let us have a more precise statement of terms, lest we
% should hereafter fall out by the way. Imagine the case of a man who is
% standing and also moving his hands and his head, and suppose a person
% to say that one and the same person is in motion and at rest at the same
% moment--to such a mode of speech we should object, and should rather say
% that one part of him is in motion while another is at rest.

% Very true.

% And suppose the objector to refine still further, and to draw the nice
% distinction that not only parts of tops, but whole tops, when they spin
% round with their pegs fixed on the spot, are at rest and in motion at
% the same time (and he may say the same of anything which revolves in the
% same spot), his objection would not be admitted by us, because in
% such cases things are not at rest and in motion in the same parts of
% themselves; we should rather say that they have both an axis and a
% circumference, and that the axis stands still, for there is no deviation
% from the perpendicular; and that the circumference goes round. But if,
% while revolving, the axis inclines either to the right or left, forwards
% or backwards, then in no point of view can they be at rest.

% That is the correct mode of describing them, he replied.

% Then none of these objections will confuse us, or incline us to believe
% that the same thing at the same time, in the same part or in relation to
% the same thing, can act or be acted upon in contrary ways.

% Certainly not, according to my way of thinking.

% Yet, I said, that we may not be compelled to examine all such
% objections, and prove at length that they are untrue, let us assume
% their absurdity, and go forward on the understanding that hereafter, if
% this assumption turn out to be untrue, all the consequences which follow
% shall be withdrawn.

% Yes, he said, that will be the best way.

% Well, I said, would you not allow that assent and dissent, desire and
% aversion, attraction and repulsion, are all of them opposites, whether
% they are regarded as active or passive (for that makes no difference in
% the fact of their opposition)?

% Yes, he said, they are opposites.

% Well, I said, and hunger and thirst, and the desires in general, and
% again willing and wishing,--all these you would refer to the classes
% already mentioned. You would say--would you not?--that the soul of him
% who desires is seeking after the object of his desire; or that he is
% drawing to himself the thing which he wishes to possess: or again,
% when a person wants anything to be given him, his mind, longing for the
% realization of his desire, intimates his wish to have it by a nod of
% assent, as if he had been asked a question?

% Very true.

% And what would you say of unwillingness and dislike and the absence of
% desire; should not these be referred to the opposite class of repulsion
% and rejection?

% Certainly.

% Admitting this to be true of desire generally, let us suppose a
% particular class of desires, and out of these we will select hunger and
% thirst, as they are termed, which are the most obvious of them?

% Let us take that class, he said.

% The object of one is food, and of the other drink?

% Yes.

% And here comes the point: is not thirst the desire which the soul has of
% drink, and of drink only; not of drink qualified by anything else; for
% example, warm or cold, or much or little, or, in a word, drink of any
% particular sort: but if the thirst be accompanied by heat, then the
% desire is of cold drink; or, if accompanied by cold, then of warm drink;
% or, if the thirst be excessive, then the drink which is desired will be
% excessive; or, if not great, the quantity of drink will also be small:
% but thirst pure and simple will desire drink pure and simple, which is
% the natural satisfaction of thirst, as food is of hunger?

% Yes, he said; the simple desire is, as you say, in every case of the
% simple object, and the qualified desire of the qualified object.

% But here a confusion may arise; and I should wish to guard against an
% opponent starting up and saying that no man desires drink only, but good
% drink, or food only, but good food; for good is the universal object of
% desire, and thirst being a desire, will necessarily be thirst after good
% drink; and the same is true of every other desire.

% Yes, he replied, the opponent might have something to say.

% Nevertheless I should still maintain, that of relatives some have a
% quality attached to either term of the relation; others are simple and
% have their correlatives simple.

% I do not know what you mean.

% Well, you know of course that the greater is relative to the less?

% Certainly.

% And the much greater to the much less?

% Yes.

% And the sometime greater to the sometime less, and the greater that is
% to be to the less that is to be?

% Certainly, he said.

% And so of more and less, and of other correlative terms, such as the
% double and the half, or again, the heavier and the lighter, the swifter
% and the slower; and of hot and cold, and of any other relatives;--is not
% this true of all of them?

% Yes.

% And does not the same principle hold in the sciences? The object of
% science is knowledge (assuming that to be the true definition), but
% the object of a particular science is a particular kind of knowledge;
% I mean, for example, that the science of house-building is a kind of
% knowledge which is defined and distinguished from other kinds and is
% therefore termed architecture.

% Certainly.

% Because it has a particular quality which no other has?

% Yes.

% And it has this particular quality because it has an object of a
% particular kind; and this is true of the other arts and sciences?

% Yes.

% Now, then, if I have made myself clear, you will understand my original
% meaning in what I said about relatives. My meaning was, that if one term
% of a relation is taken alone, the other is taken alone; if one term
% is qualified, the other is also qualified. I do not mean to say that
% relatives may not be disparate, or that the science of health is
% healthy, or of disease necessarily diseased, or that the sciences of
% good and evil are therefore good and evil; but only that, when the term
% science is no longer used absolutely, but has a qualified object which
% in this case is the nature of health and disease, it becomes defined,
% and is hence called not merely science, but the science of medicine.

% I quite understand, and I think as you do.

% Would you not say that thirst is one of these essentially relative
% terms, having clearly a relation--

% Yes, thirst is relative to drink.

% And a certain kind of thirst is relative to a certain kind of drink; but
% thirst taken alone is neither of much nor little, nor of good nor bad,
% nor of any particular kind of drink, but of drink only?

% Certainly.

% Then the soul of the thirsty one, in so far as he is thirsty, desires
% only drink; for this he yearns and tries to obtain it?

% That is plain.

% And if you suppose something which pulls a thirsty soul away from drink,
% that must be different from the thirsty principle which draws him like
% a beast to drink; for, as we were saying, the same thing cannot at the
% same time with the same part of itself act in contrary ways about the
% same.

% Impossible.

% No more than you can say that the hands of the archer push and pull the
% bow at the same time, but what you say is that one hand pushes and the
% other pulls.

% Exactly so, he replied.

% And might a man be thirsty, and yet unwilling to drink?

% Yes, he said, it constantly happens.

% And in such a case what is one to say? Would you not say that there
% was something in the soul bidding a man to drink, and something else
% forbidding him, which is other and stronger than the principle which
% bids him?

% I should say so.

% And the forbidding principle is derived from reason, and that which bids
% and attracts proceeds from passion and disease?

% Clearly.

% Then we may fairly assume that they are two, and that they differ from
% one another; the one with which a man reasons, we may call the rational
% principle of the soul, the other, with which he loves and hungers and
% thirsts and feels the flutterings of any other desire, may be termed
% the irrational or appetitive, the ally of sundry pleasures and
% satisfactions?

% Yes, he said, we may fairly assume them to be different.

% Then let us finally determine that there are two principles existing in
% the soul. And what of passion, or spirit? Is it a third, or akin to one
% of the preceding?

% I should be inclined to say--akin to desire.

% Well, I said, there is a story which I remember to have heard, and in
% which I put faith. The story is, that Leontius, the son of Aglaion,
% coming up one day from the Piraeus, under the north wall on the outside,
% observed some dead bodies lying on the ground at the place of execution.
% He felt a desire to see them, and also a dread and abhorrence of them;
% for a time he struggled and covered his eyes, but at length the desire
% got the better of him; and forcing them open, he ran up to the dead
% bodies, saying, Look, ye wretches, take your fill of the fair sight.

% I have heard the story myself, he said.

% The moral of the tale is, that anger at times goes to war with desire,
% as though they were two distinct things.

% Yes; that is the meaning, he said.

% And are there not many other cases in which we observe that when a man's
% desires violently prevail over his reason, he reviles himself, and is
% angry at the violence within him, and that in this struggle, which is
% like the struggle of factions in a State, his spirit is on the side of
% his reason;--but for the passionate or spirited element to take part
% with the desires when reason decides that she should not be opposed,
% is a sort of thing which I believe that you never observed occurring in
% yourself, nor, as I should imagine, in any one else?

% Certainly not.

% Suppose that a man thinks he has done a wrong to another, the nobler
% he is the less able is he to feel indignant at any suffering, such as
% hunger, or cold, or any other pain which the injured person may inflict
% upon him--these he deems to be just, and, as I say, his anger refuses to
% be excited by them.

% True, he said.

% But when he thinks that he is the sufferer of the wrong, then he boils
% and chafes, and is on the side of what he believes to be justice; and
% because he suffers hunger or cold or other pain he is only the more
% determined to persevere and conquer. His noble spirit will not be
% quelled until he either slays or is slain; or until he hears the voice
% of the shepherd, that is, reason, bidding his dog bark no more.

% The illustration is perfect, he replied; and in our State, as we were
% saying, the auxiliaries were to be dogs, and to hear the voice of the
% rulers, who are their shepherds.

% I perceive, I said, that you quite understand me; there is, however, a
% further point which I wish you to consider.

% What point?

% You remember that passion or spirit appeared at first sight to be a kind
% of desire, but now we should say quite the contrary; for in the conflict
% of the soul spirit is arrayed on the side of the rational principle.

% Most assuredly.

% But a further question arises: Is passion different from reason also, or
% only a kind of reason; in which latter case, instead of three principles
% in the soul, there will only be two, the rational and the concupiscent;
% or rather, as the State was composed of three classes, traders,
% auxiliaries, counsellors, so may there not be in the individual soul a
% third element which is passion or spirit, and when not corrupted by bad
% education is the natural auxiliary of reason?

% Yes, he said, there must be a third.

% Yes, I replied, if passion, which has already been shown to be different
% from desire, turn out also to be different from reason.

% But that is easily proved:--We may observe even in young children that
% they are full of spirit almost as soon as they are born, whereas some
% of them never seem to attain to the use of reason, and most of them late
% enough.

% Excellent, I said, and you may see passion equally in brute animals,
% which is a further proof of the truth of what you are saying. And we may
% once more appeal to the words of Homer, which have been already quoted
% by us,

% ``He smote his breast, and thus rebuked his soul,''

% for in this verse Homer has clearly supposed the power which reasons
% about the better and worse to be different from the unreasoning anger
% which is rebuked by it.

% Very true, he said.

% And so, after much tossing, we have reached land, and are fairly agreed
% that the same principles which exist in the State exist also in the
% individual, and that they are three in number.

% Exactly.

% Must we not then infer that the individual is wise in the same way, and
% in virtue of the same quality which makes the State wise?

% Certainly.

% Also that the same quality which constitutes courage in the State
% constitutes courage in the individual, and that both the State and the
% individual bear the same relation to all the other virtues?

% Assuredly.

% And the individual will be acknowledged by us to be just in the same way
% in which the State is just?

% That follows, of course.

% We cannot but remember that the justice of the State consisted in each
% of the three classes doing the work of its own class?

% We are not very likely to have forgotten, he said.

% We must recollect that the individual in whom the several qualities of
% his nature do their own work will be just, and will do his own work?

% Yes, he said, we must remember that too.

% And ought not the rational principle, which is wise, and has the care of
% the whole soul, to rule, and the passionate or spirited principle to be
% the subject and ally?

% Certainly.

% And, as we were saying, the united influence of music and gymnastic will
% bring them into accord, nerving and sustaining the reason with noble
% words and lessons, and moderating and soothing and civilizing the
% wildness of passion by harmony and rhythm?

% Quite true, he said.

% And these two, thus nurtured and educated, and having learned truly to
% know their own functions, will rule over the concupiscent, which in each
% of us is the largest part of the soul and by nature most insatiable of
% gain; over this they will keep guard, lest, waxing great and strong with
% the fulness of bodily pleasures, as they are termed, the concupiscent
% soul, no longer confined to her own sphere, should attempt to enslave
% and rule those who are not her natural-born subjects, and overturn the
% whole life of man?

% Very true, he said.

% Both together will they not be the best defenders of the whole soul and
% the whole body against attacks from without; the one counselling, and
% the other fighting under his leader, and courageously executing his
% commands and counsels?

% True.

% And he is to be deemed courageous whose spirit retains in pleasure and
% in pain the commands of reason about what he ought or ought not to fear?

% Right, he replied.

% And him we call wise who has in him that little part which rules, and
% which proclaims these commands; that part too being supposed to have a
% knowledge of what is for the interest of each of the three parts and of
% the whole?

% Assuredly.

% And would you not say that he is temperate who has these same elements
% in friendly harmony, in whom the one ruling principle of reason, and
% the two subject ones of spirit and desire are equally agreed that reason
% ought to rule, and do not rebel?

% Certainly, he said, that is the true account of temperance whether in
% the State or individual.

% And surely, I said, we have explained again and again how and by virtue
% of what quality a man will be just.

% That is very certain.

% And is justice dimmer in the individual, and is her form different, or
% is she the same which we found her to be in the State?

% There is no difference in my opinion, he said.

% Because, if any doubt is still lingering in our minds, a few commonplace
% instances will satisfy us of the truth of what I am saying.

% What sort of instances do you mean?

% If the case is put to us, must we not admit that the just State, or
% the man who is trained in the principles of such a State, will be less
% likely than the unjust to make away with a deposit of gold or silver?
% Would any one deny this?

% No one, he replied.

% Will the just man or citizen ever be guilty of sacrilege or theft, or
% treachery either to his friends or to his country?

% Never.

% Neither will he ever break faith where there have been oaths or
% agreements?

% Impossible.

% No one will be less likely to commit adultery, or to dishonour his
% father and mother, or to fail in his religious duties?

% No one.

% And the reason is that each part of him is doing its own business,
% whether in ruling or being ruled?

% Exactly so.

% Are you satisfied then that the quality which makes such men and such
% states is justice, or do you hope to discover some other?

% Not I, indeed.

% Then our dream has been realized; and the suspicion which we entertained
% at the beginning of our work of construction, that some divine power
% must have conducted us to a primary form of justice, has now been
% verified?

% Yes, certainly.

% And the division of labour which required the carpenter and the
% shoemaker and the rest of the citizens to be doing each his own
% business, and not another's, was a shadow of justice, and for that
% reason it was of use?

% Clearly.

% But in reality justice was such as we were describing, being concerned
% however, not with the outward man, but with the inward, which is the
% true self and concernment of man: for the just man does not permit the
% several elements within him to interfere with one another, or any of
% them to do the work of others,--he sets in order his own inner life, and
% is his own master and his own law, and at peace with himself; and when
% he has bound together the three principles within him, which may be
% compared to the higher, lower, and middle notes of the scale, and the
% intermediate intervals--when he has bound all these together, and is
% no longer many, but has become one entirely temperate and perfectly
% adjusted nature, then he proceeds to act, if he has to act, whether in
% a matter of property, or in the treatment of the body, or in some affair
% of politics or private business; always thinking and calling that which
% preserves and co-operates with this harmonious condition, just and good
% action, and the knowledge which presides over it, wisdom, and that which
% at any time impairs this condition, he will call unjust action, and the
% opinion which presides over it ignorance.

% You have said the exact truth, Socrates.

% Very good; and if we were to affirm that we had discovered the just man
% and the just State, and the nature of justice in each of them, we should
% not be telling a falsehood?

% Most certainly not.

% May we say so, then?

% Let us say so.

% And now, I said, injustice has to be considered.

% Clearly.

% Must not injustice be a strife which arises among the three
% principles--a meddlesomeness, and interference, and rising up of a part
% of the soul against the whole, an assertion of unlawful authority, which
% is made by a rebellious subject against a true prince, of whom he is the
% natural vassal,--what is all this confusion and delusion but injustice,
% and intemperance and cowardice and ignorance, and every form of vice?

% Exactly so.

% And if the nature of justice and injustice be known, then the meaning of
% acting unjustly and being unjust, or, again, of acting justly, will also
% be perfectly clear?

% What do you mean? he said.

% Why, I said, they are like disease and health; being in the soul just
% what disease and health are in the body.

% How so? he said.

% Why, I said, that which is healthy causes health, and that which is
% unhealthy causes disease.

% Yes.

% And just actions cause justice, and unjust actions cause injustice?

% That is certain.

% And the creation of health is the institution of a natural order and
% government of one by another in the parts of the body; and the creation
% of disease is the production of a state of things at variance with this
% natural order?

% True.

% And is not the creation of justice the institution of a natural order
% and government of one by another in the parts of the soul, and the
% creation of injustice the production of a state of things at variance
% with the natural order?

% Exactly so, he said.

% Then virtue is the health and beauty and well-being of the soul, and
% vice the disease and weakness and deformity of the same?

% True.

% And do not good practices lead to virtue, and evil practices to vice?

% Assuredly.

% Still our old question of the comparative advantage of justice and
% injustice has not been answered: Which is the more profitable, to be
% just and act justly and practise virtue, whether seen or unseen of
% gods and men, or to be unjust and act unjustly, if only unpunished and
% unreformed?

% In my judgment, Socrates, the question has now become ridiculous. We
% know that, when the bodily constitution is gone, life is no longer
% endurable, though pampered with all kinds of meats and drinks, and
% having all wealth and all power; and shall we be told that when the
% very essence of the vital principle is undermined and corrupted, life
% is still worth having to a man, if only he be allowed to do whatever he
% likes with the single exception that he is not to acquire justice and
% virtue, or to escape from injustice and vice; assuming them both to be
% such as we have described?

% Yes, I said, the question is, as you say, ridiculous. Still, as we are
% near the spot at which we may see the truth in the clearest manner with
% our own eyes, let us not faint by the way.

% Certainly not, he replied.

% Come up hither, I said, and behold the various forms of vice, those of
% them, I mean, which are worth looking at.

% I am following you, he replied: proceed.

% I said, The argument seems to have reached a height from which, as from
% some tower of speculation, a man may look down and see that virtue
% is one, but that the forms of vice are innumerable; there being four
% special ones which are deserving of note.

% What do you mean? he said.

% I mean, I replied, that there appear to be as many forms of the soul as
% there are distinct forms of the State.

% How many?

% There are five of the State, and five of the soul, I said.

% What are they?

% The first, I said, is that which we have been describing, and which may
% be said to have two names, monarchy and aristocracy, accordingly as rule
% is exercised by one distinguished man or by many.

% True, he replied.

% But I regard the two names as describing one form only; for whether the
% government is in the hands of one or many, if the governors have been
% trained in the manner which we have supposed, the fundamental laws of
% the State will be maintained.

% That is true, he replied.


\section{Book V} % (fold)
\label{sec:book_v}



BOOK V.

Such is the good and true City or State, and the good and true man is
of the same pattern; and if this is right every other is wrong; and the
evil is one which affects not only the ordering of the State, but also
the regulation of the individual soul, and is exhibited in four forms.

What are they? he said.

I was proceeding to tell the order in which the four evil forms appeared
to me to succeed one another, when Polemarchus, who was sitting a little
way off, just beyond Adeimantus, began to whisper to him: stretching
forth his hand, he took hold of the upper part of his coat by the
shoulder, and drew him towards him, leaning forward himself so as to be
quite close and saying something in his ear, of which I only caught the
words, ``Shall we let him off, or what shall we do?''

Certainly not, said Adeimantus, raising his voice.

Who is it, I said, whom you are refusing to let off?

You, he said.

I repeated, Why am I especially not to be let off?

Why, he said, we think that you are lazy, and mean to cheat us out of a
whole chapter which is a very important part of the story; and you fancy
that we shall not notice your airy way of proceeding; as if it were
self-evident to everybody, that in the matter of women and children
``friends have all things in common.''

And was I not right, Adeimantus?

Yes, he said; but what is right in this particular case, like everything
else, requires to be explained; for community may be of many kinds.
Please, therefore, to say what sort of community you mean. We have been
long expecting that you would tell us something about the family life
of your citizens--how they will bring children into the world, and rear
them when they have arrived, and, in general, what is the nature of this
community of women and children--for we are of opinion that the right
or wrong management of such matters will have a great and paramount
influence on the State for good or for evil. And now, since the question
is still undetermined, and you are taking in hand another State, we have
resolved, as you heard, not to let you go until you give an account of
all this.

To that resolution, said Glaucon, you may regard me as saying Agreed.

And without more ado, said Thrasymachus, you may consider us all to be
equally agreed.

I said, You know not what you are doing in thus assailing me: What an
argument are you raising about the State! Just as I thought that I had
finished, and was only too glad that I had laid this question to sleep,
and was reflecting how fortunate I was in your acceptance of what I then
said, you ask me to begin again at the very foundation, ignorant of what
a hornet's nest of words you are stirring. Now I foresaw this gathering
trouble, and avoided it.

For what purpose do you conceive that we have come here, said
Thrasymachus,--to look for gold, or to hear discourse?

Yes, but discourse should have a limit.

Yes, Socrates, said Glaucon, and the whole of life is the only limit
which wise men assign to the hearing of such discourses. But never mind
about us; take heart yourself and answer the question in your own way:
What sort of community of women and children is this which is to prevail
among our guardians? and how shall we manage the period between birth
and education, which seems to require the greatest care? Tell us how
these things will be.

Yes, my simple friend, but the answer is the reverse of easy; many more
doubts arise about this than about our previous conclusions. For the
practicability of what is said may be doubted; and looked at in another
point of view, whether the scheme, if ever so practicable, would be for
the best, is also doubtful. Hence I feel a reluctance to approach the
subject, lest our aspiration, my dear friend, should turn out to be a
dream only.

Fear not, he replied, for your audience will not be hard upon you; they
are not sceptical or hostile.

I said: My good friend, I suppose that you mean to encourage me by these
words.

Yes, he said.

Then let me tell you that you are doing just the reverse; the
encouragement which you offer would have been all very well had I myself
believed that I knew what I was talking about: to declare the truth
about matters of high interest which a man honours and loves among wise
men who love him need occasion no fear or faltering in his mind; but to
carry on an argument when you are yourself only a hesitating enquirer,
which is my condition, is a dangerous and slippery thing; and the danger
is not that I shall be laughed at (of which the fear would be childish),
but that I shall miss the truth where I have most need to be sure of my
footing, and drag my friends after me in my fall. And I pray Nemesis not
to visit upon me the words which I am going to utter. For I do indeed
believe that to be an involuntary homicide is a less crime than to be a
deceiver about beauty or goodness or justice in the matter of laws.
And that is a risk which I would rather run among enemies than among
friends, and therefore you do well to encourage me.

Glaucon laughed and said: Well then, Socrates, in case you and your
argument do us any serious injury you shall be acquitted beforehand of
the homicide, and shall not be held to be a deceiver; take courage then
and speak.

Well, I said, the law says that when a man is acquitted he is free from
guilt, and what holds at law may hold in argument.

Then why should you mind?

Well, I replied, I suppose that I must retrace my steps and say what I
perhaps ought to have said before in the proper place. The part of the
men has been played out, and now properly enough comes the turn of the
women. Of them I will proceed to speak, and the more readily since I am
invited by you.

For men born and educated like our citizens, the only way, in my
opinion, of arriving at a right conclusion about the possession and
use of women and children is to follow the path on which we originally
started, when we said that the men were to be the guardians and
watchdogs of the herd.

True.

Let us further suppose the birth and education of our women to be
subject to similar or nearly similar regulations; then we shall see
whether the result accords with our design.

What do you mean?

What I mean may be put into the form of a question, I said: Are dogs
divided into hes and shes, or do they both share equally in hunting and
in keeping watch and in the other duties of dogs? or do we entrust to
the males the entire and exclusive care of the flocks, while we leave
the females at home, under the idea that the bearing and suckling their
puppies is labour enough for them?

No, he said, they share alike; the only difference between them is that
the males are stronger and the females weaker.

But can you use different animals for the same purpose, unless they are
bred and fed in the same way?

You cannot.

Then, if women are to have the same duties as men, they must have the
same nurture and education?

Yes.

The education which was assigned to the men was music and gymnastic.

Yes.

Then women must be taught music and gymnastic and also the art of war,
which they must practise like the men?

That is the inference, I suppose.

I should rather expect, I said, that several of our proposals, if they
are carried out, being unusual, may appear ridiculous.

No doubt of it.

Yes, and the most ridiculous thing of all will be the sight of women
naked in the palaestra, exercising with the men, especially when they
are no longer young; they certainly will not be a vision of beauty, any
more than the enthusiastic old men who in spite of wrinkles and ugliness
continue to frequent the gymnasia.

Yes, indeed, he said: according to present notions the proposal would be
thought ridiculous.

But then, I said, as we have determined to speak our minds, we must not
fear the jests of the wits which will be directed against this sort of
innovation; how they will talk of women's attainments both in music
and gymnastic, and above all about their wearing armour and riding upon
horseback!

Very true, he replied.

Yet having begun we must go forward to the rough places of the law; at
the same time begging of these gentlemen for once in their life to be
serious. Not long ago, as we shall remind them, the Hellenes were of the
opinion, which is still generally received among the barbarians, that
the sight of a naked man was ridiculous and improper; and when first the
Cretans and then the Lacedaemonians introduced the custom, the wits of
that day might equally have ridiculed the innovation.

No doubt.

But when experience showed that to let all things be uncovered was far
better than to cover them up, and the ludicrous effect to the outward
eye vanished before the better principle which reason asserted, then the
man was perceived to be a fool who directs the shafts of his ridicule
at any other sight but that of folly and vice, or seriously inclines to
weigh the beautiful by any other standard but that of the good.

Very true, he replied.

First, then, whether the question is to be put in jest or in earnest,
let us come to an understanding about the nature of woman: Is she
capable of sharing either wholly or partially in the actions of men, or
not at all? And is the art of war one of those arts in which she can or
can not share? That will be the best way of commencing the enquiry, and
will probably lead to the fairest conclusion.

That will be much the best way.

Shall we take the other side first and begin by arguing against
ourselves; in this manner the adversary's position will not be
undefended.

Why not? he said.

Then let us put a speech into the mouths of our opponents. They will
say: ``Socrates and Glaucon, no adversary need convict you, for you
yourselves, at the first foundation of the State, admitted the principle
that everybody was to do the one work suited to his own nature.'' And
certainly, if I am not mistaken, such an admission was made by us. ``And
do not the natures of men and women differ very much indeed?'' And we
shall reply: Of course they do. Then we shall be asked, ``Whether the
tasks assigned to men and to women should not be different, and such as
are agreeable to their different natures?'' Certainly they should. ``But
if so, have you not fallen into a serious inconsistency in saying that
men and women, whose natures are so entirely different, ought to perform
the same actions?'--What defence will you make for us, my good Sir,
against any one who offers these objections?''

That is not an easy question to answer when asked suddenly; and I shall
and I do beg of you to draw out the case on our side.

These are the objections, Glaucon, and there are many others of a like
kind, which I foresaw long ago; they made me afraid and reluctant to
take in hand any law about the possession and nurture of women and
children.

By Zeus, he said, the problem to be solved is anything but easy.

Why yes, I said, but the fact is that when a man is out of his depth,
whether he has fallen into a little swimming bath or into mid ocean, he
has to swim all the same.

Very true.

And must not we swim and try to reach the shore: we will hope that
Arion's dolphin or some other miraculous help may save us?

I suppose so, he said.

Well then, let us see if any way of escape can be found. We
acknowledged--did we not? that different natures ought to have different
pursuits, and that men's and women's natures are different. And now
what are we saying?--that different natures ought to have the same
pursuits,--this is the inconsistency which is charged upon us.

Precisely.

Verily, Glaucon, I said, glorious is the power of the art of
contradiction!

Why do you say so?

Because I think that many a man falls into the practice against his
will. When he thinks that he is reasoning he is really disputing, just
because he cannot define and divide, and so know that of which he is
speaking; and he will pursue a merely verbal opposition in the spirit of
contention and not of fair discussion.

Yes, he replied, such is very often the case; but what has that to do
with us and our argument?

A great deal; for there is certainly a danger of our getting
unintentionally into a verbal opposition.

In what way?

Why we valiantly and pugnaciously insist upon the verbal truth, that
different natures ought to have different pursuits, but we never
considered at all what was the meaning of sameness or difference of
nature, or why we distinguished them when we assigned different pursuits
to different natures and the same to the same natures.

Why, no, he said, that was never considered by us.

I said: Suppose that by way of illustration we were to ask the question
whether there is not an opposition in nature between bald men and hairy
men; and if this is admitted by us, then, if bald men are cobblers, we
should forbid the hairy men to be cobblers, and conversely?

That would be a jest, he said.

Yes, I said, a jest; and why? because we never meant when we constructed
the State, that the opposition of natures should extend to every
difference, but only to those differences which affected the pursuit
in which the individual is engaged; we should have argued, for example,
that a physician and one who is in mind a physician may be said to have
the same nature.

True.

Whereas the physician and the carpenter have different natures?

Certainly.

And if, I said, the male and female sex appear to differ in their
fitness for any art or pursuit, we should say that such pursuit or art
ought to be assigned to one or the other of them; but if the difference
consists only in women bearing and men begetting children, this does not
amount to a proof that a woman differs from a man in respect of the
sort of education she should receive; and we shall therefore continue
to maintain that our guardians and their wives ought to have the same
pursuits.

Very true, he said.

Next, we shall ask our opponent how, in reference to any of the pursuits
or arts of civic life, the nature of a woman differs from that of a man?

That will be quite fair.

And perhaps he, like yourself, will reply that to give a sufficient
answer on the instant is not easy; but after a little reflection there
is no difficulty.

Yes, perhaps.

Suppose then that we invite him to accompany us in the argument, and
then we may hope to show him that there is nothing peculiar in the
constitution of women which would affect them in the administration of
the State.

By all means.

Let us say to him: Come now, and we will ask you a question:--when you
spoke of a nature gifted or not gifted in any respect, did you mean to
say that one man will acquire a thing easily, another with difficulty; a
little learning will lead the one to discover a great deal; whereas
the other, after much study and application, no sooner learns than he
forgets; or again, did you mean, that the one has a body which is a
good servant to his mind, while the body of the other is a hindrance to
him?--would not these be the sort of differences which distinguish the
man gifted by nature from the one who is ungifted?

No one will deny that.

And can you mention any pursuit of mankind in which the male sex has not
all these gifts and qualities in a higher degree than the female? Need
I waste time in speaking of the art of weaving, and the management of
pancakes and preserves, in which womankind does really appear to be
great, and in which for her to be beaten by a man is of all things the
most absurd?

You are quite right, he replied, in maintaining the general inferiority
of the female sex: although many women are in many things superior to
many men, yet on the whole what you say is true.

And if so, my friend, I said, there is no special faculty of
administration in a state which a woman has because she is a woman, or
which a man has by virtue of his sex, but the gifts of nature are alike
diffused in both; all the pursuits of men are the pursuits of women
also, but in all of them a woman is inferior to a man.

Very true.

Then are we to impose all our enactments on men and none of them on
women?

That will never do.

One woman has a gift of healing, another not; one is a musician, and
another has no music in her nature?

Very true.

And one woman has a turn for gymnastic and military exercises, and
another is unwarlike and hates gymnastics?

Certainly.

And one woman is a philosopher, and another is an enemy of philosophy;
one has spirit, and another is without spirit?

That is also true.

Then one woman will have the temper of a guardian, and another not. Was
not the selection of the male guardians determined by differences of
this sort?

Yes.

Men and women alike possess the qualities which make a guardian; they
differ only in their comparative strength or weakness.

Obviously.

And those women who have such qualities are to be selected as the
companions and colleagues of men who have similar qualities and whom
they resemble in capacity and in character?

Very true.

And ought not the same natures to have the same pursuits?

They ought.

Then, as we were saying before, there is nothing unnatural in assigning
music and gymnastic to the wives of the guardians--to that point we come
round again.

Certainly not.

The law which we then enacted was agreeable to nature, and therefore not
an impossibility or mere aspiration; and the contrary practice, which
prevails at present, is in reality a violation of nature.

That appears to be true.

We had to consider, first, whether our proposals were possible, and
secondly whether they were the most beneficial?

Yes.

And the possibility has been acknowledged?

Yes.

The very great benefit has next to be established?

Quite so.

You will admit that the same education which makes a man a good guardian
will make a woman a good guardian; for their original nature is the
same?

Yes.

I should like to ask you a question.

What is it?

Would you say that all men are equal in excellence, or is one man better
than another?

The latter.

And in the commonwealth which we were founding do you conceive the
guardians who have been brought up on our model system to be more
perfect men, or the cobblers whose education has been cobbling?

What a ridiculous question!

You have answered me, I replied: Well, and may we not further say that
our guardians are the best of our citizens?

By far the best.

And will not their wives be the best women?

Yes, by far the best.

And can there be anything better for the interests of the State than
that the men and women of a State should be as good as possible?

There can be nothing better.

And this is what the arts of music and gymnastic, when present in such
manner as we have described, will accomplish?

Certainly.

Then we have made an enactment not only possible but in the highest
degree beneficial to the State?

True.

Then let the wives of our guardians strip, for their virtue will be
their robe, and let them share in the toils of war and the defence of
their country; only in the distribution of labours the lighter are to be
assigned to the women, who are the weaker natures, but in other respects
their duties are to be the same. And as for the man who laughs at naked
women exercising their bodies from the best of motives, in his laughter
he is plucking

``A fruit of unripe wisdom,''

and he himself is ignorant of what he is laughing at, or what he is
about;--for that is, and ever will be, the best of sayings, That the
useful is the noble and the hurtful is the base.

Very true.

Here, then, is one difficulty in our law about women, which we may say
that we have now escaped; the wave has not swallowed us up alive for
enacting that the guardians of either sex should have all their
pursuits in common; to the utility and also to the possibility of this
arrangement the consistency of the argument with itself bears witness.

Yes, that was a mighty wave which you have escaped.

Yes, I said, but a greater is coming; you will not think much of this
when you see the next.

Go on; let me see.

The law, I said, which is the sequel of this and of all that has
preceded, is to the following effect,--'that the wives of our guardians
are to be common, and their children are to be common, and no parent is
to know his own child, nor any child his parent.''

Yes, he said, that is a much greater wave than the other; and
the possibility as well as the utility of such a law are far more
questionable.

I do not think, I said, that there can be any dispute about the very
great utility of having wives and children in common; the possibility is
quite another matter, and will be very much disputed.

I think that a good many doubts may be raised about both.

You imply that the two questions must be combined, I replied. Now I
meant that you should admit the utility; and in this way, as I thought,
I should escape from one of them, and then there would remain only the
possibility.

But that little attempt is detected, and therefore you will please to
give a defence of both.

Well, I said, I submit to my fate. Yet grant me a little favour: let
me feast my mind with the dream as day dreamers are in the habit of
feasting themselves when they are walking alone; for before they have
discovered any means of effecting their wishes--that is a matter which
never troubles them--they would rather not tire themselves by thinking
about possibilities; but assuming that what they desire is already
granted to them, they proceed with their plan, and delight in detailing
what they mean to do when their wish has come true--that is a way which
they have of not doing much good to a capacity which was never good for
much. Now I myself am beginning to lose heart, and I should like, with
your permission, to pass over the question of possibility at present.
Assuming therefore the possibility of the proposal, I shall now proceed
to enquire how the rulers will carry out these arrangements, and I shall
demonstrate that our plan, if executed, will be of the greatest benefit
to the State and to the guardians. First of all, then, if you have no
objection, I will endeavour with your help to consider the advantages of
the measure; and hereafter the question of possibility.

I have no objection; proceed.

First, I think that if our rulers and their auxiliaries are to be worthy
of the name which they bear, there must be willingness to obey in the
one and the power of command in the other; the guardians must themselves
obey the laws, and they must also imitate the spirit of them in any
details which are entrusted to their care.

That is right, he said.

You, I said, who are their legislator, having selected the men, will now
select the women and give them to them;--they must be as far as possible
of like natures with them; and they must live in common houses and meet
at common meals. None of them will have anything specially his or her
own; they will be together, and will be brought up together, and
will associate at gymnastic exercises. And so they will be drawn by
a necessity of their natures to have intercourse with each
other--necessity is not too strong a word, I think?

Yes, he said;--necessity, not geometrical, but another sort of necessity
which lovers know, and which is far more convincing and constraining to
the mass of mankind.

True, I said; and this, Glaucon, like all the rest, must proceed after
an orderly fashion; in a city of the blessed, licentiousness is an
unholy thing which the rulers will forbid.

Yes, he said, and it ought not to be permitted.

Then clearly the next thing will be to make matrimony sacred in the
highest degree, and what is most beneficial will be deemed sacred?

Exactly.

And how can marriages be made most beneficial?--that is a question which
I put to you, because I see in your house dogs for hunting, and of the
nobler sort of birds not a few. Now, I beseech you, do tell me, have you
ever attended to their pairing and breeding?

In what particulars?

Why, in the first place, although they are all of a good sort, are not
some better than others?

True.

And do you breed from them all indifferently, or do you take care to
breed from the best only?

From the best.

And do you take the oldest or the youngest, or only those of ripe age?

I choose only those of ripe age.

And if care was not taken in the breeding, your dogs and birds would
greatly deteriorate?

Certainly.

And the same of horses and animals in general?

Undoubtedly.

Good heavens! my dear friend, I said, what consummate skill will our
rulers need if the same principle holds of the human species!

Certainly, the same principle holds; but why does this involve any
particular skill?

Because, I said, our rulers will often have to practise upon the body
corporate with medicines. Now you know that when patients do not require
medicines, but have only to be put under a regimen, the inferior sort
of practitioner is deemed to be good enough; but when medicine has to be
given, then the doctor should be more of a man.

That is quite true, he said; but to what are you alluding?

I mean, I replied, that our rulers will find a considerable dose of
falsehood and deceit necessary for the good of their subjects: we were
saying that the use of all these things regarded as medicines might be
of advantage.

And we were very right.

And this lawful use of them seems likely to be often needed in the
regulations of marriages and births.

How so?

Why, I said, the principle has been already laid down that the best of
either sex should be united with the best as often, and the inferior
with the inferior, as seldom as possible; and that they should rear the
offspring of the one sort of union, but not of the other, if the flock
is to be maintained in first-rate condition. Now these goings on must be
a secret which the rulers only know, or there will be a further
danger of our herd, as the guardians may be termed, breaking out into
rebellion.

Very true.

Had we not better appoint certain festivals at which we will bring
together the brides and bridegrooms, and sacrifices will be offered and
suitable hymeneal songs composed by our poets: the number of weddings is
a matter which must be left to the discretion of the rulers, whose aim
will be to preserve the average of population? There are many other
things which they will have to consider, such as the effects of wars and
diseases and any similar agencies, in order as far as this is possible
to prevent the State from becoming either too large or too small.

Certainly, he replied.

We shall have to invent some ingenious kind of lots which the less
worthy may draw on each occasion of our bringing them together, and then
they will accuse their own ill-luck and not the rulers.

To be sure, he said.

And I think that our braver and better youth, besides their other
honours and rewards, might have greater facilities of intercourse with
women given them; their bravery will be a reason, and such fathers ought
to have as many sons as possible.

True.

And the proper officers, whether male or female or both, for offices are
to be held by women as well as by men--

Yes--

The proper officers will take the offspring of the good parents to the
pen or fold, and there they will deposit them with certain nurses who
dwell in a separate quarter; but the offspring of the inferior, or of
the better when they chance to be deformed, will be put away in some
mysterious, unknown place, as they should be.

Yes, he said, that must be done if the breed of the guardians is to be
kept pure.

They will provide for their nurture, and will bring the mothers to the
fold when they are full of milk, taking the greatest possible care that
no mother recognises her own child; and other wet-nurses may be engaged
if more are required. Care will also be taken that the process of
suckling shall not be protracted too long; and the mothers will have no
getting up at night or other trouble, but will hand over all this sort
of thing to the nurses and attendants.

You suppose the wives of our guardians to have a fine easy time of it
when they are having children.

Why, said I, and so they ought. Let us, however, proceed with our
scheme. We were saying that the parents should be in the prime of life?

Very true.

And what is the prime of life? May it not be defined as a period of
about twenty years in a woman's life, and thirty in a man's?

Which years do you mean to include?

A woman, I said, at twenty years of age may begin to bear children to
the State, and continue to bear them until forty; a man may begin at
five-and-twenty, when he has passed the point at which the pulse of life
beats quickest, and continue to beget children until he be fifty-five.

Certainly, he said, both in men and women those years are the prime of
physical as well as of intellectual vigour.

Any one above or below the prescribed ages who takes part in the public
hymeneals shall be said to have done an unholy and unrighteous thing;
the child of which he is the father, if it steals into life, will have
been conceived under auspices very unlike the sacrifices and prayers,
which at each hymeneal priestesses and priest and the whole city will
offer, that the new generation may be better and more useful than their
good and useful parents, whereas his child will be the offspring of
darkness and strange lust.

Very true, he replied.

And the same law will apply to any one of those within the prescribed
age who forms a connection with any woman in the prime of life without
the sanction of the rulers; for we shall say that he is raising up a
bastard to the State, uncertified and unconsecrated.

Very true, he replied.

This applies, however, only to those who are within the specified age:
after that we allow them to range at will, except that a man may not
marry his daughter or his daughter's daughter, or his mother or his
mother's mother; and women, on the other hand, are prohibited from
marrying their sons or fathers, or son's son or father's father, and
so on in either direction. And we grant all this, accompanying the
permission with strict orders to prevent any embryo which may come into
being from seeing the light; and if any force a way to the birth, the
parents must understand that the offspring of such an union cannot be
maintained, and arrange accordingly.

That also, he said, is a reasonable proposition. But how will they know
who are fathers and daughters, and so on?

They will never know. The way will be this:--dating from the day of the
hymeneal, the bridegroom who was then married will call all the male
children who are born in the seventh and tenth month afterwards his
sons, and the female children his daughters, and they will call him
father, and he will call their children his grandchildren, and they will
call the elder generation grandfathers and grandmothers. All who were
begotten at the time when their fathers and mothers came together will
be called their brothers and sisters, and these, as I was saying, will
be forbidden to inter-marry. This, however, is not to be understood as
an absolute prohibition of the marriage of brothers and sisters; if the
lot favours them, and they receive the sanction of the Pythian oracle,
the law will allow them.

Quite right, he replied.

Such is the scheme, Glaucon, according to which the guardians of our
State are to have their wives and families in common. And now you would
have the argument show that this community is consistent with the rest
of our polity, and also that nothing can be better--would you not?

Yes, certainly.

Shall we try to find a common basis by asking of ourselves what ought
to be the chief aim of the legislator in making laws and in the
organization of a State,--what is the greatest good, and what is the
greatest evil, and then consider whether our previous description has
the stamp of the good or of the evil?

By all means.

Can there be any greater evil than discord and distraction and plurality
where unity ought to reign? or any greater good than the bond of unity?

There cannot.

And there is unity where there is community of pleasures and
pains--where all the citizens are glad or grieved on the same occasions
of joy and sorrow?

No doubt.

Yes; and where there is no common but only private feeling a State is
disorganized--when you have one half of the world triumphing and the
other plunged in grief at the same events happening to the city or the
citizens?

Certainly.

Such differences commonly originate in a disagreement about the use of
the terms ``mine'' and ``not mine,'' ``his'' and ``not his.''

Exactly so.

And is not that the best-ordered State in which the greatest number of
persons apply the terms ``mine'' and ``not mine'' in the same way to the
same thing?

Quite true.

Or that again which most nearly approaches to the condition of the
individual--as in the body, when but a finger of one of us is hurt, the
whole frame, drawn towards the soul as a centre and forming one kingdom
under the ruling power therein, feels the hurt and sympathizes all
together with the part affected, and we say that the man has a pain in
his finger; and the same expression is used about any other part of the
body, which has a sensation of pain at suffering or of pleasure at the
alleviation of suffering.

Very true, he replied; and I agree with you that in the best-ordered
State there is the nearest approach to this common feeling which you
describe.

Then when any one of the citizens experiences any good or evil, the
whole State will make his case their own, and will either rejoice or
sorrow with him?

Yes, he said, that is what will happen in a well-ordered State.

It will now be time, I said, for us to return to our State and see
whether this or some other form is most in accordance with these
fundamental principles.

Very good.

Our State like every other has rulers and subjects?

True.

All of whom will call one another citizens?

Of course.

But is there not another name which people give to their rulers in other
States?

Generally they call them masters, but in democratic States they simply
call them rulers.

And in our State what other name besides that of citizens do the people
give the rulers?

They are called saviours and helpers, he replied.

And what do the rulers call the people?

Their maintainers and foster-fathers.

And what do they call them in other States?

Slaves.

And what do the rulers call one another in other States?

Fellow-rulers.

And what in ours?

Fellow-guardians.

Did you ever know an example in any other State of a ruler who would
speak of one of his colleagues as his friend and of another as not being
his friend?

Yes, very often.

And the friend he regards and describes as one in whom he has an
interest, and the other as a stranger in whom he has no interest?

Exactly.

But would any of your guardians think or speak of any other guardian as
a stranger?

Certainly he would not; for every one whom they meet will be regarded
by them either as a brother or sister, or father or mother, or son or
daughter, or as the child or parent of those who are thus connected with
him.

Capital, I said; but let me ask you once more: Shall they be a family in
name only; or shall they in all their actions be true to the name? For
example, in the use of the word ``father,'' would the care of a father be
implied and the filial reverence and duty and obedience to him which the
law commands; and is the violator of these duties to be regarded as an
impious and unrighteous person who is not likely to receive much good
either at the hands of God or of man? Are these to be or not to be the
strains which the children will hear repeated in their ears by all the
citizens about those who are intimated to them to be their parents and
the rest of their kinsfolk?

These, he said, and none other; for what can be more ridiculous than for
them to utter the names of family ties with the lips only and not to act
in the spirit of them?

Then in our city the language of harmony and concord will be more often
heard than in any other. As I was describing before, when any one is
well or ill, the universal word will be ``with me it is well'' or ``it is
ill.''

Most true.

And agreeably to this mode of thinking and speaking, were we not saying
that they will have their pleasures and pains in common?

Yes, and so they will.

And they will have a common interest in the same thing which they will
alike call ``my own,'' and having this common interest they will have a
common feeling of pleasure and pain?

Yes, far more so than in other States.

And the reason of this, over and above the general constitution of the
State, will be that the guardians will have a community of women and
children?

That will be the chief reason.

And this unity of feeling we admitted to be the greatest good, as was
implied in our own comparison of a well-ordered State to the relation of
the body and the members, when affected by pleasure or pain?

That we acknowledged, and very rightly.

Then the community of wives and children among our citizens is clearly
the source of the greatest good to the State?

Certainly.

And this agrees with the other principle which we were affirming,--that
the guardians were not to have houses or lands or any other property;
their pay was to be their food, which they were to receive from the
other citizens, and they were to have no private expenses; for we
intended them to preserve their true character of guardians.

Right, he replied.

Both the community of property and the community of families, as I am
saying, tend to make them more truly guardians; they will not tear
the city in pieces by differing about ``mine'' and ``not mine;'' each man
dragging any acquisition which he has made into a separate house of his
own, where he has a separate wife and children and private pleasures and
pains; but all will be affected as far as may be by the same pleasures
and pains because they are all of one opinion about what is near and
dear to them, and therefore they all tend towards a common end.

Certainly, he replied.

And as they have nothing but their persons which they can call their
own, suits and complaints will have no existence among them; they will
be delivered from all those quarrels of which money or children or
relations are the occasion.

Of course they will.

Neither will trials for assault or insult ever be likely to occur among
them. For that equals should defend themselves against equals we shall
maintain to be honourable and right; we shall make the protection of the
person a matter of necessity.

That is good, he said.

Yes; and there is a further good in the law; viz. that if a man has a
quarrel with another he will satisfy his resentment then and there, and
not proceed to more dangerous lengths.

Certainly.

To the elder shall be assigned the duty of ruling and chastising the
younger.

Clearly.

Nor can there be a doubt that the younger will not strike or do any
other violence to an elder, unless the magistrates command him; nor will
he slight him in any way. For there are two guardians, shame and fear,
mighty to prevent him: shame, which makes men refrain from laying hands
on those who are to them in the relation of parents; fear, that the
injured one will be succoured by the others who are his brothers, sons,
fathers.

That is true, he replied.

Then in every way the laws will help the citizens to keep the peace with
one another?

Yes, there will be no want of peace.

And as the guardians will never quarrel among themselves there will be
no danger of the rest of the city being divided either against them or
against one another.

None whatever.

I hardly like even to mention the little meannesses of which they will
be rid, for they are beneath notice: such, for example, as the
flattery of the rich by the poor, and all the pains and pangs which
men experience in bringing up a family, and in finding money to buy
necessaries for their household, borrowing and then repudiating, getting
how they can, and giving the money into the hands of women and slaves
to keep--the many evils of so many kinds which people suffer in this way
are mean enough and obvious enough, and not worth speaking of.

Yes, he said, a man has no need of eyes in order to perceive that.

And from all these evils they will be delivered, and their life will be
blessed as the life of Olympic victors and yet more blessed.

How so?

The Olympic victor, I said, is deemed happy in receiving a part only of
the blessedness which is secured to our citizens, who have won a more
glorious victory and have a more complete maintenance at the public
cost. For the victory which they have won is the salvation of the whole
State; and the crown with which they and their children are crowned is
the fulness of all that life needs; they receive rewards from the
hands of their country while living, and after death have an honourable
burial.

Yes, he said, and glorious rewards they are.

Do you remember, I said, how in the course of the previous discussion
some one who shall be nameless accused us of making our guardians
unhappy--they had nothing and might have possessed all things--to whom
we replied that, if an occasion offered, we might perhaps hereafter
consider this question, but that, as at present advised, we would make
our guardians truly guardians, and that we were fashioning the State
with a view to the greatest happiness, not of any particular class, but
of the whole?

Yes, I remember.

And what do you say, now that the life of our protectors is made out to
be far better and nobler than that of Olympic victors--is the life of
shoemakers, or any other artisans, or of husbandmen, to be compared with
it?

Certainly not.

At the same time I ought here to repeat what I have said elsewhere, that
if any of our guardians shall try to be happy in such a manner that
he will cease to be a guardian, and is not content with this safe and
harmonious life, which, in our judgment, is of all lives the best, but
infatuated by some youthful conceit of happiness which gets up into his
head shall seek to appropriate the whole state to himself, then he will
have to learn how wisely Hesiod spoke, when he said, ``half is more than
the whole.''

If he were to consult me, I should say to him: Stay where you are, when
you have the offer of such a life.

You agree then, I said, that men and women are to have a common way of
life such as we have described--common education, common children; and
they are to watch over the citizens in common whether abiding in the
city or going out to war; they are to keep watch together, and to hunt
together like dogs; and always and in all things, as far as they are
able, women are to share with the men? And in so doing they will do what
is best, and will not violate, but preserve the natural relation of the
sexes.

I agree with you, he replied.

The enquiry, I said, has yet to be made, whether such a community
be found possible--as among other animals, so also among men--and if
possible, in what way possible?

You have anticipated the question which I was about to suggest.

There is no difficulty, I said, in seeing how war will be carried on by
them.

How?

Why, of course they will go on expeditions together; and will take with
them any of their children who are strong enough, that, after the manner
of the artisan's child, they may look on at the work which they will
have to do when they are grown up; and besides looking on they will
have to help and be of use in war, and to wait upon their fathers and
mothers. Did you never observe in the arts how the potters'' boys look on
and help, long before they touch the wheel?

Yes, I have.

And shall potters be more careful in educating their children and in
giving them the opportunity of seeing and practising their duties than
our guardians will be?

The idea is ridiculous, he said.

There is also the effect on the parents, with whom, as with other
animals, the presence of their young ones will be the greatest incentive
to valour.

That is quite true, Socrates; and yet if they are defeated, which may
often happen in war, how great the danger is! the children will be lost
as well as their parents, and the State will never recover.

True, I said; but would you never allow them to run any risk?

I am far from saying that.

Well, but if they are ever to run a risk should they not do so on some
occasion when, if they escape disaster, they will be the better for it?

Clearly.

Whether the future soldiers do or do not see war in the days of their
youth is a very important matter, for the sake of which some risk may
fairly be incurred.

Yes, very important.

This then must be our first step,--to make our children spectators
of war; but we must also contrive that they shall be secured against
danger; then all will be well.

True.

Their parents may be supposed not to be blind to the risks of war, but
to know, as far as human foresight can, what expeditions are safe and
what dangerous?

That may be assumed.

And they will take them on the safe expeditions and be cautious about
the dangerous ones?

True.

And they will place them under the command of experienced veterans who
will be their leaders and teachers?

Very properly.

Still, the dangers of war cannot be always foreseen; there is a good
deal of chance about them?

True.

Then against such chances the children must be at once furnished with
wings, in order that in the hour of need they may fly away and escape.

What do you mean? he said.

I mean that we must mount them on horses in their earliest youth, and
when they have learnt to ride, take them on horseback to see war: the
horses must not be spirited and warlike, but the most tractable and yet
the swiftest that can be had. In this way they will get an excellent
view of what is hereafter to be their own business; and if there is
danger they have only to follow their elder leaders and escape.

I believe that you are right, he said.

Next, as to war; what are to be the relations of your soldiers to one
another and to their enemies? I should be inclined to propose that the
soldier who leaves his rank or throws away his arms, or is guilty of any
other act of cowardice, should be degraded into the rank of a husbandman
or artisan. What do you think?

By all means, I should say.

And he who allows himself to be taken prisoner may as well be made a
present of to his enemies; he is their lawful prey, and let them do what
they like with him.

Certainly.

But the hero who has distinguished himself, what shall be done to
him? In the first place, he shall receive honour in the army from his
youthful comrades; every one of them in succession shall crown him. What
do you say?

I approve.

And what do you say to his receiving the right hand of fellowship?

To that too, I agree.

But you will hardly agree to my next proposal.

What is your proposal?

That he should kiss and be kissed by them.

Most certainly, and I should be disposed to go further, and say: Let
no one whom he has a mind to kiss refuse to be kissed by him while the
expedition lasts. So that if there be a lover in the army, whether
his love be youth or maiden, he may be more eager to win the prize of
valour.

Capital, I said. That the brave man is to have more wives than others
has been already determined: and he is to have first choices in such
matters more than others, in order that he may have as many children as
possible?

Agreed.

Again, there is another manner in which, according to Homer, brave
youths should be honoured; for he tells how Ajax, after he had
distinguished himself in battle, was rewarded with long chines, which
seems to be a compliment appropriate to a hero in the flower of his age,
being not only a tribute of honour but also a very strengthening thing.

Most true, he said.

Then in this, I said, Homer shall be our teacher; and we too, at
sacrifices and on the like occasions, will honour the brave according to
the measure of their valour, whether men or women, with hymns and those
other distinctions which we were mentioning; also with

``seats of precedence, and meats and full cups;''

and in honouring them, we shall be at the same time training them.

That, he replied, is excellent.

Yes, I said; and when a man dies gloriously in war shall we not say, in
the first place, that he is of the golden race?

To be sure.

Nay, have we not the authority of Hesiod for affirming that when they
are dead

``They are holy angels upon the earth, authors of good, averters of evil,
the guardians of speech-gifted men''?

Yes; and we accept his authority.

We must learn of the god how we are to order the sepulture of divine and
heroic personages, and what is to be their special distinction; and we
must do as he bids?

By all means.

And in ages to come we will reverence them and kneel before their
sepulchres as at the graves of heroes. And not only they but any who are
deemed pre-eminently good, whether they die from age, or in any other
way, shall be admitted to the same honours.

That is very right, he said.

Next, how shall our soldiers treat their enemies? What about this?

In what respect do you mean?

First of all, in regard to slavery? Do you think it right that Hellenes
should enslave Hellenic States, or allow others to enslave them, if
they can help? Should not their custom be to spare them, considering
the danger which there is that the whole race may one day fall under the
yoke of the barbarians?

To spare them is infinitely better.

Then no Hellene should be owned by them as a slave; that is a rule which
they will observe and advise the other Hellenes to observe.

Certainly, he said; they will in this way be united against the
barbarians and will keep their hands off one another.

Next as to the slain; ought the conquerors, I said, to take anything
but their armour? Does not the practice of despoiling an enemy afford
an excuse for not facing the battle? Cowards skulk about the dead,
pretending that they are fulfilling a duty, and many an army before now
has been lost from this love of plunder.

Very true.

And is there not illiberality and avarice in robbing a corpse, and also
a degree of meanness and womanishness in making an enemy of the dead
body when the real enemy has flown away and left only his fighting
gear behind him,--is not this rather like a dog who cannot get at his
assailant, quarrelling with the stones which strike him instead?

Very like a dog, he said.

Then we must abstain from spoiling the dead or hindering their burial?

Yes, he replied, we most certainly must.

Neither shall we offer up arms at the temples of the gods, least of all
the arms of Hellenes, if we care to maintain good feeling with other
Hellenes; and, indeed, we have reason to fear that the offering of
spoils taken from kinsmen may be a pollution unless commanded by the god
himself?

Very true.

Again, as to the devastation of Hellenic territory or the burning of
houses, what is to be the practice?

May I have the pleasure, he said, of hearing your opinion?

Both should be forbidden, in my judgment; I would take the annual
produce and no more. Shall I tell you why?

Pray do.

Why, you see, there is a difference in the names ``discord'' and ``war,''
and I imagine that there is also a difference in their natures; the one
is expressive of what is internal and domestic, the other of what is
external and foreign; and the first of the two is termed discord, and
only the second, war.

That is a very proper distinction, he replied.

And may I not observe with equal propriety that the Hellenic race is all
united together by ties of blood and friendship, and alien and strange
to the barbarians?

Very good, he said.

And therefore when Hellenes fight with barbarians and barbarians with
Hellenes, they will be described by us as being at war when they fight,
and by nature enemies, and this kind of antagonism should be called war;
but when Hellenes fight with one another we shall say that Hellas is
then in a state of disorder and discord, they being by nature friends;
and such enmity is to be called discord.

I agree.

Consider then, I said, when that which we have acknowledged to be
discord occurs, and a city is divided, if both parties destroy the lands
and burn the houses of one another, how wicked does the strife appear!
No true lover of his country would bring himself to tear in pieces his
own nurse and mother: There might be reason in the conqueror depriving
the conquered of their harvest, but still they would have the idea of
peace in their hearts and would not mean to go on fighting for ever.

Yes, he said, that is a better temper than the other.

And will not the city, which you are founding, be an Hellenic city?

It ought to be, he replied.

Then will not the citizens be good and civilized?

Yes, very civilized.

And will they not be lovers of Hellas, and think of Hellas as their own
land, and share in the common temples?

Most certainly.

And any difference which arises among them will be regarded by them as
discord only--a quarrel among friends, which is not to be called a war?

Certainly not.

Then they will quarrel as those who intend some day to be reconciled?

Certainly.

They will use friendly correction, but will not enslave or destroy their
opponents; they will be correctors, not enemies?

Just so.

And as they are Hellenes themselves they will not devastate Hellas, nor
will they burn houses, nor ever suppose that the whole population of a
city--men, women, and children--are equally their enemies, for they know
that the guilt of war is always confined to a few persons and that the
many are their friends. And for all these reasons they will be unwilling
to waste their lands and rase their houses; their enmity to them will
only last until the many innocent sufferers have compelled the guilty
few to give satisfaction?

I agree, he said, that our citizens should thus deal with their Hellenic
enemies; and with barbarians as the Hellenes now deal with one another.

Then let us enact this law also for our guardians:--that they are
neither to devastate the lands of Hellenes nor to burn their houses.

Agreed; and we may agree also in thinking that these, like all our
previous enactments, are very good.

But still I must say, Socrates, that if you are allowed to go on in
this way you will entirely forget the other question which at the
commencement of this discussion you thrust aside:--Is such an order of
things possible, and how, if at all? For I am quite ready to acknowledge
that the plan which you propose, if only feasible, would do all sorts of
good to the State. I will add, what you have omitted, that your citizens
will be the bravest of warriors, and will never leave their ranks, for
they will all know one another, and each will call the other father,
brother, son; and if you suppose the women to join their armies, whether
in the same rank or in the rear, either as a terror to the enemy, or as
auxiliaries in case of need, I know that they will then be absolutely
invincible; and there are many domestic advantages which might also be
mentioned and which I also fully acknowledge: but, as I admit all these
advantages and as many more as you please, if only this State of yours
were to come into existence, we need say no more about them; assuming
then the existence of the State, let us now turn to the question of
possibility and ways and means--the rest may be left.

If I loiter for a moment, you instantly make a raid upon me, I said, and
have no mercy; I have hardly escaped the first and second waves, and you
seem not to be aware that you are now bringing upon me the third, which
is the greatest and heaviest. When you have seen and heard the third
wave, I think you will be more considerate and will acknowledge
that some fear and hesitation was natural respecting a proposal so
extraordinary as that which I have now to state and investigate.

The more appeals of this sort which you make, he said, the more
determined are we that you shall tell us how such a State is possible:
speak out and at once.

Let me begin by reminding you that we found our way hither in the search
after justice and injustice.

True, he replied; but what of that?

I was only going to ask whether, if we have discovered them, we are to
require that the just man should in nothing fail of absolute justice; or
may we be satisfied with an approximation, and the attainment in him of
a higher degree of justice than is to be found in other men?

The approximation will be enough.

We were enquiring into the nature of absolute justice and into the
character of the perfectly just, and into injustice and the perfectly
unjust, that we might have an ideal. We were to look at these in order
that we might judge of our own happiness and unhappiness according to
the standard which they exhibited and the degree in which we resembled
them, but not with any view of showing that they could exist in fact.

True, he said.

Would a painter be any the worse because, after having delineated with
consummate art an ideal of a perfectly beautiful man, he was unable to
show that any such man could ever have existed?

He would be none the worse.

Well, and were we not creating an ideal of a perfect State?

To be sure.

And is our theory a worse theory because we are unable to prove the
possibility of a city being ordered in the manner described?

Surely not, he replied.

That is the truth, I said. But if, at your request, I am to try and show
how and under what conditions the possibility is highest, I must ask
you, having this in view, to repeat your former admissions.

What admissions?

I want to know whether ideals are ever fully realized in language?
Does not the word express more than the fact, and must not the actual,
whatever a man may think, always, in the nature of things, fall short of
the truth? What do you say?

I agree.

Then you must not insist on my proving that the actual State will in
every respect coincide with the ideal: if we are only able to discover
how a city may be governed nearly as we proposed, you will admit that we
have discovered the possibility which you demand; and will be contented.
I am sure that I should be contented--will not you?

Yes, I will.

Let me next endeavour to show what is that fault in States which is the
cause of their present maladministration, and what is the least change
which will enable a State to pass into the truer form; and let the
change, if possible, be of one thing only, or, if not, of two; at any
rate, let the changes be as few and slight as possible.

Certainly, he replied.

I think, I said, that there might be a reform of the State if only one
change were made, which is not a slight or easy though still a possible
one.

What is it? he said.

Now then, I said, I go to meet that which I liken to the greatest of
the waves; yet shall the word be spoken, even though the wave break and
drown me in laughter and dishonour; and do you mark my words.

Proceed.

I said: ``Until philosophers are kings, or the kings and princes of this
world have the spirit and power of philosophy, and political greatness
and wisdom meet in one, and those commoner natures who pursue either
to the exclusion of the other are compelled to stand aside, cities
will never have rest from their evils,--nor the human race, as I
believe,--and then only will this our State have a possibility of life
and behold the light of day.'' Such was the thought, my dear Glaucon,
which I would fain have uttered if it had not seemed too extravagant;
for to be convinced that in no other State can there be happiness
private or public is indeed a hard thing.

Socrates, what do you mean? I would have you consider that the word
which you have uttered is one at which numerous persons, and very
respectable persons too, in a figure pulling off their coats all in a
moment, and seizing any weapon that comes to hand, will run at you might
and main, before you know where you are, intending to do heaven knows
what; and if you don't prepare an answer, and put yourself in motion,
you will be ``pared by their fine wits,'' and no mistake.

You got me into the scrape, I said.

And I was quite right; however, I will do all I can to get you out of
it; but I can only give you good-will and good advice, and, perhaps, I
may be able to fit answers to your questions better than another--that
is all. And now, having such an auxiliary, you must do your best to show
the unbelievers that you are right.

I ought to try, I said, since you offer me such invaluable assistance.
And I think that, if there is to be a chance of our escaping, we must
explain to them whom we mean when we say that philosophers are to rule
in the State; then we shall be able to defend ourselves: There will be
discovered to be some natures who ought to study philosophy and to be
leaders in the State; and others who are not born to be philosophers,
and are meant to be followers rather than leaders.

Then now for a definition, he said.

Follow me, I said, and I hope that I may in some way or other be able to
give you a satisfactory explanation.

Proceed.

I dare say that you remember, and therefore I need not remind you, that
a lover, if he is worthy of the name, ought to show his love, not to
some one part of that which he loves, but to the whole.

I really do not understand, and therefore beg of you to assist my
memory.

Another person, I said, might fairly reply as you do; but a man of
pleasure like yourself ought to know that all who are in the flower of
youth do somehow or other raise a pang or emotion in a lover's breast,
and are thought by him to be worthy of his affectionate regards. Is not
this a way which you have with the fair: one has a snub nose, and you
praise his charming face; the hook-nose of another has, you say, a
royal look; while he who is neither snub nor hooked has the grace of
regularity: the dark visage is manly, the fair are children of the gods;
and as to the sweet ``honey pale,'' as they are called, what is the very
name but the invention of a lover who talks in diminutives, and is not
averse to paleness if appearing on the cheek of youth? In a word, there
is no excuse which you will not make, and nothing which you will not
say, in order not to lose a single flower that blooms in the spring-time
of youth.

If you make me an authority in matters of love, for the sake of the
argument, I assent.

And what do you say of lovers of wine? Do you not see them doing the
same? They are glad of any pretext of drinking any wine.

Very good.

And the same is true of ambitious men; if they cannot command an army,
they are willing to command a file; and if they cannot be honoured by
really great and important persons, they are glad to be honoured by
lesser and meaner people,--but honour of some kind they must have.

Exactly.

Once more let me ask: Does he who desires any class of goods, desire the
whole class or a part only?

The whole.

And may we not say of the philosopher that he is a lover, not of a part
of wisdom only, but of the whole?

Yes, of the whole.

And he who dislikes learning, especially in youth, when he has no power
of judging what is good and what is not, such an one we maintain not
to be a philosopher or a lover of knowledge, just as he who refuses his
food is not hungry, and may be said to have a bad appetite and not a
good one?

Very true, he said.

Whereas he who has a taste for every sort of knowledge and who is
curious to learn and is never satisfied, may be justly termed a
philosopher? Am I not right?

Glaucon said: If curiosity makes a philosopher, you will find many a
strange being will have a title to the name. All the lovers of sights
have a delight in learning, and must therefore be included. Musical
amateurs, too, are a folk strangely out of place among philosophers, for
they are the last persons in the world who would come to anything like
a philosophical discussion, if they could help, while they run about at
the Dionysiac festivals as if they had let out their ears to hear every
chorus; whether the performance is in town or country--that makes no
difference--they are there. Now are we to maintain that all these and
any who have similar tastes, as well as the professors of quite minor
arts, are philosophers?

Certainly not, I replied; they are only an imitation.

He said: Who then are the true philosophers?

Those, I said, who are lovers of the vision of truth.

That is also good, he said; but I should like to know what you mean?

To another, I replied, I might have a difficulty in explaining; but I am
sure that you will admit a proposition which I am about to make.

What is the proposition?

That since beauty is the opposite of ugliness, they are two?

Certainly.

And inasmuch as they are two, each of them is one?

True again.

And of just and unjust, good and evil, and of every other class, the
same remark holds: taken singly, each of them is one; but from the
various combinations of them with actions and things and with one
another, they are seen in all sorts of lights and appear many?

Very true.

And this is the distinction which I draw between the sight-loving,
art-loving, practical class and those of whom I am speaking, and who are
alone worthy of the name of philosophers.

How do you distinguish them? he said.

The lovers of sounds and sights, I replied, are, as I conceive, fond of
fine tones and colours and forms and all the artificial products that
are made out of them, but their mind is incapable of seeing or loving
absolute beauty.

True, he replied.

Few are they who are able to attain to the sight of this.

Very true.

And he who, having a sense of beautiful things has no sense of absolute
beauty, or who, if another lead him to a knowledge of that beauty is
unable to follow--of such an one I ask, Is he awake or in a dream
only? Reflect: is not the dreamer, sleeping or waking, one who likens
dissimilar things, who puts the copy in the place of the real object?

I should certainly say that such an one was dreaming.

But take the case of the other, who recognises the existence of absolute
beauty and is able to distinguish the idea from the objects which
participate in the idea, neither putting the objects in the place of the
idea nor the idea in the place of the objects--is he a dreamer, or is he
awake?

He is wide awake.

And may we not say that the mind of the one who knows has knowledge, and
that the mind of the other, who opines only, has opinion?

Certainly.

But suppose that the latter should quarrel with us and dispute our
statement, can we administer any soothing cordial or advice to him,
without revealing to him that there is sad disorder in his wits?

We must certainly offer him some good advice, he replied.

Come, then, and let us think of something to say to him. Shall we begin
by assuring him that he is welcome to any knowledge which he may have,
and that we are rejoiced at his having it? But we should like to ask him
a question: Does he who has knowledge know something or nothing? (You
must answer for him.)

I answer that he knows something.

Something that is or is not?

Something that is; for how can that which is not ever be known?

And are we assured, after looking at the matter from many points of
view, that absolute being is or may be absolutely known, but that the
utterly non-existent is utterly unknown?

Nothing can be more certain.

Good. But if there be anything which is of such a nature as to be and
not to be, that will have a place intermediate between pure being and
the absolute negation of being?

Yes, between them.

And, as knowledge corresponded to being and ignorance of necessity to
not-being, for that intermediate between being and not-being there has
to be discovered a corresponding intermediate between ignorance and
knowledge, if there be such?

Certainly.

Do we admit the existence of opinion?

Undoubtedly.

As being the same with knowledge, or another faculty?

Another faculty.

Then opinion and knowledge have to do with different kinds of matter
corresponding to this difference of faculties?

Yes.

And knowledge is relative to being and knows being. But before I proceed
further I will make a division.

What division?

I will begin by placing faculties in a class by themselves: they are
powers in us, and in all other things, by which we do as we do. Sight
and hearing, for example, I should call faculties. Have I clearly
explained the class which I mean?

Yes, I quite understand.

Then let me tell you my view about them. I do not see them, and
therefore the distinctions of figure, colour, and the like, which enable
me to discern the differences of some things, do not apply to them. In
speaking of a faculty I think only of its sphere and its result; and
that which has the same sphere and the same result I call the same
faculty, but that which has another sphere and another result I call
different. Would that be your way of speaking?

Yes.

And will you be so very good as to answer one more question? Would you
say that knowledge is a faculty, or in what class would you place it?

Certainly knowledge is a faculty, and the mightiest of all faculties.

And is opinion also a faculty?

Certainly, he said; for opinion is that with which we are able to form
an opinion.

And yet you were acknowledging a little while ago that knowledge is not
the same as opinion?

Why, yes, he said: how can any reasonable being ever identify that which
is infallible with that which errs?

An excellent answer, proving, I said, that we are quite conscious of a
distinction between them.

Yes.

Then knowledge and opinion having distinct powers have also distinct
spheres or subject-matters?

That is certain.

Being is the sphere or subject-matter of knowledge, and knowledge is to
know the nature of being?

Yes.

And opinion is to have an opinion?

Yes.

And do we know what we opine? or is the subject-matter of opinion the
same as the subject-matter of knowledge?

Nay, he replied, that has been already disproven; if difference in
faculty implies difference in the sphere or subject-matter, and if, as
we were saying, opinion and knowledge are distinct faculties, then the
sphere of knowledge and of opinion cannot be the same.

Then if being is the subject-matter of knowledge, something else must be
the subject-matter of opinion?

Yes, something else.

Well then, is not-being the subject-matter of opinion? or, rather, how
can there be an opinion at all about not-being? Reflect: when a man
has an opinion, has he not an opinion about something? Can he have an
opinion which is an opinion about nothing?

Impossible.

He who has an opinion has an opinion about some one thing?

Yes.

And not-being is not one thing but, properly speaking, nothing?

True.

Of not-being, ignorance was assumed to be the necessary correlative; of
being, knowledge?

True, he said.

Then opinion is not concerned either with being or with not-being?

Not with either.

And can therefore neither be ignorance nor knowledge?

That seems to be true.

But is opinion to be sought without and beyond either of them, in
a greater clearness than knowledge, or in a greater darkness than
ignorance?

In neither.

Then I suppose that opinion appears to you to be darker than knowledge,
but lighter than ignorance?

Both; and in no small degree.

And also to be within and between them?

Yes.

Then you would infer that opinion is intermediate?

No question.

But were we not saying before, that if anything appeared to be of a sort
which is and is not at the same time, that sort of thing would appear
also to lie in the interval between pure being and absolute not-being;
and that the corresponding faculty is neither knowledge nor ignorance,
but will be found in the interval between them?

True.

And in that interval there has now been discovered something which we
call opinion?

There has.

Then what remains to be discovered is the object which partakes equally
of the nature of being and not-being, and cannot rightly be termed
either, pure and simple; this unknown term, when discovered, we may
truly call the subject of opinion, and assign each to their proper
faculty,--the extremes to the faculties of the extremes and the mean to
the faculty of the mean.

True.

This being premised, I would ask the gentleman who is of opinion that
there is no absolute or unchangeable idea of beauty--in whose opinion
the beautiful is the manifold--he, I say, your lover of beautiful
sights, who cannot bear to be told that the beautiful is one, and the
just is one, or that anything is one--to him I would appeal, saying,
Will you be so very kind, sir, as to tell us whether, of all these
beautiful things, there is one which will not be found ugly; or of the
just, which will not be found unjust; or of the holy, which will not
also be unholy?

No, he replied; the beautiful will in some point of view be found ugly;
and the same is true of the rest.

And may not the many which are doubles be also halves?--doubles, that
is, of one thing, and halves of another?

Quite true.

And things great and small, heavy and light, as they are termed, will
not be denoted by these any more than by the opposite names?

True; both these and the opposite names will always attach to all of
them.

And can any one of those many things which are called by particular
names be said to be this rather than not to be this?

He replied: They are like the punning riddles which are asked at feasts
or the children's puzzle about the eunuch aiming at the bat, with
what he hit him, as they say in the puzzle, and upon what the bat
was sitting. The individual objects of which I am speaking are also
a riddle, and have a double sense: nor can you fix them in your mind,
either as being or not-being, or both, or neither.

Then what will you do with them? I said. Can they have a better place
than between being and not-being? For they are clearly not in greater
darkness or negation than not-being, or more full of light and existence
than being.

That is quite true, he said.

Thus then we seem to have discovered that the many ideas which the
multitude entertain about the beautiful and about all other things are
tossing about in some region which is half-way between pure being and
pure not-being?

We have.

Yes; and we had before agreed that anything of this kind which we might
find was to be described as matter of opinion, and not as matter of
knowledge; being the intermediate flux which is caught and detained by
the intermediate faculty.

Quite true.

Then those who see the many beautiful, and who yet neither see absolute
beauty, nor can follow any guide who points the way thither; who see the
many just, and not absolute justice, and the like,--such persons may be
said to have opinion but not knowledge?

That is certain.

But those who see the absolute and eternal and immutable may be said to
know, and not to have opinion only?

Neither can that be denied.

The one love and embrace the subjects of knowledge, the other those of
opinion? The latter are the same, as I dare say you will remember, who
listened to sweet sounds and gazed upon fair colours, but would not
tolerate the existence of absolute beauty.

Yes, I remember.

Shall we then be guilty of any impropriety in calling them lovers of
opinion rather than lovers of wisdom, and will they be very angry with
us for thus describing them?

I shall tell them not to be angry; no man should be angry at what is
true.

But those who love the truth in each thing are to be called lovers of
wisdom and not lovers of opinion.

Assuredly.


% section book_v (end)

\section{Book VI} % (fold)
\label{sec:book_vi}



BOOK VI.

And thus, Glaucon, after the argument has gone a weary way, the true and
the false philosophers have at length appeared in view.

I do not think, he said, that the way could have been shortened.

I suppose not, I said; and yet I believe that we might have had a better
view of both of them if the discussion could have been confined to this
one subject and if there were not many other questions awaiting us,
which he who desires to see in what respect the life of the just differs
from that of the unjust must consider.

And what is the next question? he asked.

Surely, I said, the one which follows next in order. Inasmuch as
philosophers only are able to grasp the eternal and unchangeable,
and those who wander in the region of the many and variable are not
philosophers, I must ask you which of the two classes should be the
rulers of our State?

And how can we rightly answer that question?

Whichever of the two are best able to guard the laws and institutions of
our State--let them be our guardians.

Very good.

Neither, I said, can there be any question that the guardian who is to
keep anything should have eyes rather than no eyes?

There can be no question of that.

And are not those who are verily and indeed wanting in the knowledge
of the true being of each thing, and who have in their souls no clear
pattern, and are unable as with a painter's eye to look at the absolute
truth and to that original to repair, and having perfect vision of the
other world to order the laws about beauty, goodness, justice in this,
if not already ordered, and to guard and preserve the order of them--are
not such persons, I ask, simply blind?

Truly, he replied, they are much in that condition.

And shall they be our guardians when there are others who, besides being
their equals in experience and falling short of them in no particular of
virtue, also know the very truth of each thing?

There can be no reason, he said, for rejecting those who have this
greatest of all great qualities; they must always have the first place
unless they fail in some other respect.

Suppose then, I said, that we determine how far they can unite this and
the other excellences.

By all means.

In the first place, as we began by observing, the nature of the
philosopher has to be ascertained. We must come to an understanding
about him, and, when we have done so, then, if I am not mistaken, we
shall also acknowledge that such an union of qualities is possible, and
that those in whom they are united, and those only, should be rulers in
the State.

What do you mean?

Let us suppose that philosophical minds always love knowledge of a sort
which shows them the eternal nature not varying from generation and
corruption.

Agreed.

And further, I said, let us agree that they are lovers of all true
being; there is no part whether greater or less, or more or less
honourable, which they are willing to renounce; as we said before of the
lover and the man of ambition.

True.

And if they are to be what we were describing, is there not another
quality which they should also possess?

What quality?

Truthfulness: they will never intentionally receive into their mind
falsehood, which is their detestation, and they will love the truth.

Yes, that may be safely affirmed of them.

``May be,'' my friend, I replied, is not the word; say rather ``must be
affirmed:'' for he whose nature is amorous of anything cannot help loving
all that belongs or is akin to the object of his affections.

Right, he said.

And is there anything more akin to wisdom than truth?

How can there be?

Can the same nature be a lover of wisdom and a lover of falsehood?

Never.

The true lover of learning then must from his earliest youth, as far as
in him lies, desire all truth?

Assuredly.

But then again, as we know by experience, he whose desires are strong
in one direction will have them weaker in others; they will be like a
stream which has been drawn off into another channel.

True.

He whose desires are drawn towards knowledge in every form will be
absorbed in the pleasures of the soul, and will hardly feel bodily
pleasure--I mean, if he be a true philosopher and not a sham one.

That is most certain.

Such an one is sure to be temperate and the reverse of covetous; for the
motives which make another man desirous of having and spending, have no
place in his character.

Very true.

Another criterion of the philosophical nature has also to be considered.

What is that?

There should be no secret corner of illiberality; nothing can be more
antagonistic than meanness to a soul which is ever longing after the
whole of things both divine and human.

Most true, he replied.

Then how can he who has magnificence of mind and is the spectator of all
time and all existence, think much of human life?

He cannot.

Or can such an one account death fearful?

No indeed.

Then the cowardly and mean nature has no part in true philosophy?

Certainly not.

Or again: can he who is harmoniously constituted, who is not covetous or
mean, or a boaster, or a coward--can he, I say, ever be unjust or hard
in his dealings?

Impossible.

Then you will soon observe whether a man is just and gentle, or rude
and unsociable; these are the signs which distinguish even in youth the
philosophical nature from the unphilosophical.

True.

There is another point which should be remarked.

What point?

Whether he has or has not a pleasure in learning; for no one will love
that which gives him pain, and in which after much toil he makes little
progress.

Certainly not.

And again, if he is forgetful and retains nothing of what he learns,
will he not be an empty vessel?

That is certain.

Labouring in vain, he must end in hating himself and his fruitless
occupation? Yes.

Then a soul which forgets cannot be ranked among genuine philosophic
natures; we must insist that the philosopher should have a good memory?

Certainly.

And once more, the inharmonious and unseemly nature can only tend to
disproportion?

Undoubtedly.

And do you consider truth to be akin to proportion or to disproportion?

To proportion.

Then, besides other qualities, we must try to find a naturally
well-proportioned and gracious mind, which will move spontaneously
towards the true being of everything.

Certainly.

Well, and do not all these qualities, which we have been enumerating, go
together, and are they not, in a manner, necessary to a soul, which is
to have a full and perfect participation of being?

They are absolutely necessary, he replied.

And must not that be a blameless study which he only can pursue who has
the gift of a good memory, and is quick to learn,--noble, gracious, the
friend of truth, justice, courage, temperance, who are his kindred?

The god of jealousy himself, he said, could find no fault with such a
study.

And to men like him, I said, when perfected by years and education, and
to these only you will entrust the State.

Here Adeimantus interposed and said: To these statements, Socrates, no
one can offer a reply; but when you talk in this way, a strange feeling
passes over the minds of your hearers: They fancy that they are led
astray a little at each step in the argument, owing to their own want of
skill in asking and answering questions; these littles accumulate, and
at the end of the discussion they are found to have sustained a mighty
overthrow and all their former notions appear to be turned upside down.
And as unskilful players of draughts are at last shut up by their
more skilful adversaries and have no piece to move, so they too find
themselves shut up at last; for they have nothing to say in this new
game of which words are the counters; and yet all the time they are in
the right. The observation is suggested to me by what is now occurring.
For any one of us might say, that although in words he is not able
to meet you at each step of the argument, he sees as a fact that the
votaries of philosophy, when they carry on the study, not only in youth
as a part of education, but as the pursuit of their maturer years, most
of them become strange monsters, not to say utter rogues, and that those
who may be considered the best of them are made useless to the world by
the very study which you extol.

Well, and do you think that those who say so are wrong?

I cannot tell, he replied; but I should like to know what is your
opinion.

Hear my answer; I am of opinion that they are quite right.

Then how can you be justified in saying that cities will not cease from
evil until philosophers rule in them, when philosophers are acknowledged
by us to be of no use to them?

You ask a question, I said, to which a reply can only be given in a
parable.

Yes, Socrates; and that is a way of speaking to which you are not at all
accustomed, I suppose.

I perceive, I said, that you are vastly amused at having plunged me into
such a hopeless discussion; but now hear the parable, and then you will
be still more amused at the meagreness of my imagination: for the manner
in which the best men are treated in their own States is so grievous
that no single thing on earth is comparable to it; and therefore, if
I am to plead their cause, I must have recourse to fiction, and put
together a figure made up of many things, like the fabulous unions of
goats and stags which are found in pictures. Imagine then a fleet or a
ship in which there is a captain who is taller and stronger than any of
the crew, but he is a little deaf and has a similar infirmity in sight,
and his knowledge of navigation is not much better. The sailors are
quarrelling with one another about the steering--every one is of opinion
that he has a right to steer, though he has never learned the art of
navigation and cannot tell who taught him or when he learned, and will
further assert that it cannot be taught, and they are ready to cut in
pieces any one who says the contrary. They throng about the captain,
begging and praying him to commit the helm to them; and if at any time
they do not prevail, but others are preferred to them, they kill the
others or throw them overboard, and having first chained up the noble
captain's senses with drink or some narcotic drug, they mutiny and take
possession of the ship and make free with the stores; thus, eating
and drinking, they proceed on their voyage in such manner as might be
expected of them. Him who is their partisan and cleverly aids them in
their plot for getting the ship out of the captain's hands into their
own whether by force or persuasion, they compliment with the name of
sailor, pilot, able seaman, and abuse the other sort of man, whom they
call a good-for-nothing; but that the true pilot must pay attention
to the year and seasons and sky and stars and winds, and whatever else
belongs to his art, if he intends to be really qualified for the command
of a ship, and that he must and will be the steerer, whether other
people like or not--the possibility of this union of authority with the
steerer's art has never seriously entered into their thoughts or been
made part of their calling. Now in vessels which are in a state of
mutiny and by sailors who are mutineers, how will the true pilot be
regarded? Will he not be called by them a prater, a star-gazer, a
good-for-nothing?

Of course, said Adeimantus.

Then you will hardly need, I said, to hear the interpretation of the
figure, which describes the true philosopher in his relation to the
State; for you understand already.

Certainly.

Then suppose you now take this parable to the gentleman who is surprised
at finding that philosophers have no honour in their cities; explain
it to him and try to convince him that their having honour would be far
more extraordinary.

I will.

Say to him, that, in deeming the best votaries of philosophy to be
useless to the rest of the world, he is right; but also tell him to
attribute their uselessness to the fault of those who will not use them,
and not to themselves. The pilot should not humbly beg the sailors to be
commanded by him--that is not the order of nature; neither are ``the wise
to go to the doors of the rich''--the ingenious author of this saying
told a lie--but the truth is, that, when a man is ill, whether he
be rich or poor, to the physician he must go, and he who wants to
be governed, to him who is able to govern. The ruler who is good for
anything ought not to beg his subjects to be ruled by him; although
the present governors of mankind are of a different stamp; they may be
justly compared to the mutinous sailors, and the true helmsmen to those
who are called by them good-for-nothings and star-gazers.

Precisely so, he said.

For these reasons, and among men like these, philosophy, the noblest
pursuit of all, is not likely to be much esteemed by those of the
opposite faction; not that the greatest and most lasting injury is done
to her by her opponents, but by her own professing followers, the same
of whom you suppose the accuser to say, that the greater number of them
are arrant rogues, and the best are useless; in which opinion I agreed.

Yes.

And the reason why the good are useless has now been explained?

True.

Then shall we proceed to show that the corruption of the majority is
also unavoidable, and that this is not to be laid to the charge of
philosophy any more than the other?

By all means.

And let us ask and answer in turn, first going back to the description
of the gentle and noble nature. Truth, as you will remember, was his
leader, whom he followed always and in all things; failing in this, he
was an impostor, and had no part or lot in true philosophy.

Yes, that was said.

Well, and is not this one quality, to mention no others, greatly at
variance with present notions of him?

Certainly, he said.

And have we not a right to say in his defence, that the true lover of
knowledge is always striving after being--that is his nature; he will
not rest in the multiplicity of individuals which is an appearance only,
but will go on--the keen edge will not be blunted, nor the force of his
desire abate until he have attained the knowledge of the true nature
of every essence by a sympathetic and kindred power in the soul, and by
that power drawing near and mingling and becoming incorporate with very
being, having begotten mind and truth, he will have knowledge and will
live and grow truly, and then, and not till then, will he cease from his
travail.

Nothing, he said, can be more just than such a description of him.

And will the love of a lie be any part of a philosopher's nature? Will
he not utterly hate a lie?

He will.

And when truth is the captain, we cannot suspect any evil of the band
which he leads?

Impossible.

Justice and health of mind will be of the company, and temperance will
follow after?

True, he replied.

Neither is there any reason why I should again set in array the
philosopher's virtues, as you will doubtless remember that courage,
magnificence, apprehension, memory, were his natural gifts. And you
objected that, although no one could deny what I then said, still, if
you leave words and look at facts, the persons who are thus described
are some of them manifestly useless, and the greater number utterly
depraved; we were then led to enquire into the grounds of these
accusations, and have now arrived at the point of asking why are
the majority bad, which question of necessity brought us back to the
examination and definition of the true philosopher.

Exactly.

And we have next to consider the corruptions of the philosophic nature,
why so many are spoiled and so few escape spoiling--I am speaking of
those who were said to be useless but not wicked--and, when we have done
with them, we will speak of the imitators of philosophy, what manner of
men are they who aspire after a profession which is above them and of
which they are unworthy, and then, by their manifold inconsistencies,
bring upon philosophy, and upon all philosophers, that universal
reprobation of which we speak.

What are these corruptions? he said.

I will see if I can explain them to you. Every one will admit that a
nature having in perfection all the qualities which we required in a
philosopher, is a rare plant which is seldom seen among men.

Rare indeed.

And what numberless and powerful causes tend to destroy these rare
natures!

What causes?

In the first place there are their own virtues, their courage,
temperance, and the rest of them, every one of which praiseworthy
qualities (and this is a most singular circumstance) destroys and
distracts from philosophy the soul which is the possessor of them.

That is very singular, he replied.

Then there are all the ordinary goods of life--beauty, wealth, strength,
rank, and great connections in the State--you understand the sort of
things--these also have a corrupting and distracting effect.

I understand; but I should like to know more precisely what you mean
about them.

Grasp the truth as a whole, I said, and in the right way; you will then
have no difficulty in apprehending the preceding remarks, and they will
no longer appear strange to you.

And how am I to do so? he asked.

Why, I said, we know that all germs or seeds, whether vegetable or
animal, when they fail to meet with proper nutriment or climate or soil,
in proportion to their vigour, are all the more sensitive to the want of
a suitable environment, for evil is a greater enemy to what is good than
to what is not.

Very true.

There is reason in supposing that the finest natures, when under alien
conditions, receive more injury than the inferior, because the contrast
is greater.

Certainly.

And may we not say, Adeimantus, that the most gifted minds, when they
are ill-educated, become pre-eminently bad? Do not great crimes and
the spirit of pure evil spring out of a fulness of nature ruined by
education rather than from any inferiority, whereas weak natures are
scarcely capable of any very great good or very great evil?

There I think that you are right.

And our philosopher follows the same analogy--he is like a plant which,
having proper nurture, must necessarily grow and mature into all virtue,
but, if sown and planted in an alien soil, becomes the most noxious of
all weeds, unless he be preserved by some divine power. Do you really
think, as people so often say, that our youth are corrupted by Sophists,
or that private teachers of the art corrupt them in any degree worth
speaking of? Are not the public who say these things the greatest of all
Sophists? And do they not educate to perfection young and old, men and
women alike, and fashion them after their own hearts?

When is this accomplished? he said.

When they meet together, and the world sits down at an assembly, or in
a court of law, or a theatre, or a camp, or in any other popular resort,
and there is a great uproar, and they praise some things which are
being said or done, and blame other things, equally exaggerating both,
shouting and clapping their hands, and the echo of the rocks and the
place in which they are assembled redoubles the sound of the praise or
blame--at such a time will not a young man's heart, as they say, leap
within him? Will any private training enable him to stand firm against
the overwhelming flood of popular opinion? or will he be carried away
by the stream? Will he not have the notions of good and evil which the
public in general have--he will do as they do, and as they are, such
will he be?

Yes, Socrates; necessity will compel him.

And yet, I said, there is a still greater necessity, which has not been
mentioned.

What is that?

The gentle force of attainder or confiscation or death, which, as you
are aware, these new Sophists and educators, who are the public, apply
when their words are powerless.

Indeed they do; and in right good earnest.

Now what opinion of any other Sophist, or of any private person, can be
expected to overcome in such an unequal contest?

None, he replied.

No, indeed, I said, even to make the attempt is a great piece of folly;
there neither is, nor has been, nor is ever likely to be, any different
type of character which has had no other training in virtue but that
which is supplied by public opinion--I speak, my friend, of human virtue
only; what is more than human, as the proverb says, is not included:
for I would not have you ignorant that, in the present evil state of
governments, whatever is saved and comes to good is saved by the power
of God, as we may truly say.

I quite assent, he replied.

Then let me crave your assent also to a further observation.

What are you going to say?

Why, that all those mercenary individuals, whom the many call Sophists
and whom they deem to be their adversaries, do, in fact, teach nothing
but the opinion of the many, that is to say, the opinions of their
assemblies; and this is their wisdom. I might compare them to a man who
should study the tempers and desires of a mighty strong beast who is
fed by him--he would learn how to approach and handle him, also at what
times and from what causes he is dangerous or the reverse, and what
is the meaning of his several cries, and by what sounds, when another
utters them, he is soothed or infuriated; and you may suppose further,
that when, by continually attending upon him, he has become perfect in
all this, he calls his knowledge wisdom, and makes of it a system or
art, which he proceeds to teach, although he has no real notion of what
he means by the principles or passions of which he is speaking, but
calls this honourable and that dishonourable, or good or evil, or just
or unjust, all in accordance with the tastes and tempers of the great
brute. Good he pronounces to be that in which the beast delights and
evil to be that which he dislikes; and he can give no other account
of them except that the just and noble are the necessary, having never
himself seen, and having no power of explaining to others the nature
of either, or the difference between them, which is immense. By heaven,
would not such an one be a rare educator?

Indeed he would.

And in what way does he who thinks that wisdom is the discernment of
the tempers and tastes of the motley multitude, whether in painting
or music, or, finally, in politics, differ from him whom I have been
describing? For when a man consorts with the many, and exhibits to
them his poem or other work of art or the service which he has done
the State, making them his judges when he is not obliged, the so-called
necessity of Diomede will oblige him to produce whatever they
praise. And yet the reasons are utterly ludicrous which they give in
confirmation of their own notions about the honourable and good. Did you
ever hear any of them which were not?

No, nor am I likely to hear.

You recognise the truth of what I have been saying? Then let me ask you
to consider further whether the world will ever be induced to believe in
the existence of absolute beauty rather than of the many beautiful, or
of the absolute in each kind rather than of the many in each kind?

Certainly not.

Then the world cannot possibly be a philosopher?

Impossible.

And therefore philosophers must inevitably fall under the censure of the
world?

They must.

And of individuals who consort with the mob and seek to please them?

That is evident.

Then, do you see any way in which the philosopher can be preserved in
his calling to the end? and remember what we were saying of him, that
he was to have quickness and memory and courage and magnificence--these
were admitted by us to be the true philosopher's gifts.

Yes.

Will not such an one from his early childhood be in all things first
among all, especially if his bodily endowments are like his mental ones?

Certainly, he said.

And his friends and fellow-citizens will want to use him as he gets
older for their own purposes?

No question.

Falling at his feet, they will make requests to him and do him honour
and flatter him, because they want to get into their hands now, the
power which he will one day possess.

That often happens, he said.

And what will a man such as he is be likely to do under such
circumstances, especially if he be a citizen of a great city, rich
and noble, and a tall proper youth? Will he not be full of boundless
aspirations, and fancy himself able to manage the affairs of Hellenes
and of barbarians, and having got such notions into his head will he
not dilate and elevate himself in the fulness of vain pomp and senseless
pride?

To be sure he will.

Now, when he is in this state of mind, if some one gently comes to him
and tells him that he is a fool and must get understanding, which can
only be got by slaving for it, do you think that, under such adverse
circumstances, he will be easily induced to listen?

Far otherwise.

And even if there be some one who through inherent goodness or natural
reasonableness has had his eyes opened a little and is humbled and taken
captive by philosophy, how will his friends behave when they think that
they are likely to lose the advantage which they were hoping to reap
from his companionship? Will they not do and say anything to prevent him
from yielding to his better nature and to render his teacher powerless,
using to this end private intrigues as well as public prosecutions?

There can be no doubt of it.

And how can one who is thus circumstanced ever become a philosopher?

Impossible.

Then were we not right in saying that even the very qualities which
make a man a philosopher may, if he be ill-educated, divert him from
philosophy, no less than riches and their accompaniments and the other
so-called goods of life?

We were quite right.

Thus, my excellent friend, is brought about all that ruin and failure
which I have been describing of the natures best adapted to the best of
all pursuits; they are natures which we maintain to be rare at any time;
this being the class out of which come the men who are the authors of
the greatest evil to States and individuals; and also of the greatest
good when the tide carries them in that direction; but a small man never
was the doer of any great thing either to individuals or to States.

That is most true, he said.

And so philosophy is left desolate, with her marriage rite incomplete:
for her own have fallen away and forsaken her, and while they are
leading a false and unbecoming life, other unworthy persons, seeing that
she has no kinsmen to be her protectors, enter in and dishonour her; and
fasten upon her the reproaches which, as you say, her reprovers utter,
who affirm of her votaries that some are good for nothing, and that the
greater number deserve the severest punishment.

That is certainly what people say.

Yes; and what else would you expect, I said, when you think of the puny
creatures who, seeing this land open to them--a land well stocked with
fair names and showy titles--like prisoners running out of prison into a
sanctuary, take a leap out of their trades into philosophy; those who
do so being probably the cleverest hands at their own miserable crafts?
For, although philosophy be in this evil case, still there remains a
dignity about her which is not to be found in the arts. And many are
thus attracted by her whose natures are imperfect and whose souls are
maimed and disfigured by their meannesses, as their bodies are by their
trades and crafts. Is not this unavoidable?

Yes.

Are they not exactly like a bald little tinker who has just got out of
durance and come into a fortune; he takes a bath and puts on a new coat,
and is decked out as a bridegroom going to marry his master's daughter,
who is left poor and desolate?

A most exact parallel.

What will be the issue of such marriages? Will they not be vile and
bastard?

There can be no question of it.

And when persons who are unworthy of education approach philosophy and
make an alliance with her who is in a rank above them what sort of
ideas and opinions are likely to be generated? Will they not be sophisms
captivating to the ear, having nothing in them genuine, or worthy of or
akin to true wisdom?

No doubt, he said.

Then, Adeimantus, I said, the worthy disciples of philosophy will be but
a small remnant: perchance some noble and well-educated person, detained
by exile in her service, who in the absence of corrupting influences
remains devoted to her; or some lofty soul born in a mean city, the
politics of which he contemns and neglects; and there may be a gifted
few who leave the arts, which they justly despise, and come to her;--or
peradventure there are some who are restrained by our friend Theages''
bridle; for everything in the life of Theages conspired to divert him
from philosophy; but ill-health kept him away from politics. My own case
of the internal sign is hardly worth mentioning, for rarely, if ever,
has such a monitor been given to any other man. Those who belong to this
small class have tasted how sweet and blessed a possession philosophy
is, and have also seen enough of the madness of the multitude; and they
know that no politician is honest, nor is there any champion of justice
at whose side they may fight and be saved. Such an one may be compared
to a man who has fallen among wild beasts--he will not join in the
wickedness of his fellows, but neither is he able singly to resist all
their fierce natures, and therefore seeing that he would be of no use to
the State or to his friends, and reflecting that he would have to throw
away his life without doing any good either to himself or others, he
holds his peace, and goes his own way. He is like one who, in the storm
of dust and sleet which the driving wind hurries along, retires
under the shelter of a wall; and seeing the rest of mankind full of
wickedness, he is content, if only he can live his own life and be pure
from evil or unrighteousness, and depart in peace and good-will, with
bright hopes.

Yes, he said, and he will have done a great work before he departs.

A great work--yes; but not the greatest, unless he find a State suitable
to him; for in a State which is suitable to him, he will have a larger
growth and be the saviour of his country, as well as of himself.

The causes why philosophy is in such an evil name have now been
sufficiently explained: the injustice of the charges against her has
been shown--is there anything more which you wish to say?

Nothing more on that subject, he replied; but I should like to know
which of the governments now existing is in your opinion the one adapted
to her.

Not any of them, I said; and that is precisely the accusation which I
bring against them--not one of them is worthy of the philosophic nature,
and hence that nature is warped and estranged;--as the exotic seed
which is sown in a foreign land becomes denaturalized, and is wont to be
overpowered and to lose itself in the new soil, even so this growth
of philosophy, instead of persisting, degenerates and receives another
character. But if philosophy ever finds in the State that perfection
which she herself is, then will be seen that she is in truth divine, and
that all other things, whether natures of men or institutions, are but
human;--and now, I know, that you are going to ask, What that State is:

No, he said; there you are wrong, for I was going to ask another
question--whether it is the State of which we are the founders and
inventors, or some other?

Yes, I replied, ours in most respects; but you may remember my saying
before, that some living authority would always be required in the
State having the same idea of the constitution which guided you when as
legislator you were laying down the laws.

That was said, he replied.

Yes, but not in a satisfactory manner; you frightened us by interposing
objections, which certainly showed that the discussion would be long and
difficult; and what still remains is the reverse of easy.

What is there remaining?

The question how the study of philosophy may be so ordered as not to be
the ruin of the State: All great attempts are attended with risk; ``hard
is the good,'' as men say.

Still, he said, let the point be cleared up, and the enquiry will then
be complete.

I shall not be hindered, I said, by any want of will, but, if at all,
by a want of power: my zeal you may see for yourselves; and please to
remark in what I am about to say how boldly and unhesitatingly I declare
that States should pursue philosophy, not as they do now, but in a
different spirit.

In what manner?

At present, I said, the students of philosophy are quite young;
beginning when they are hardly past childhood, they devote only the time
saved from moneymaking and housekeeping to such pursuits; and even those
of them who are reputed to have most of the philosophic spirit, when
they come within sight of the great difficulty of the subject, I mean
dialectic, take themselves off. In after life when invited by some one
else, they may, perhaps, go and hear a lecture, and about this they make
much ado, for philosophy is not considered by them to be their
proper business: at last, when they grow old, in most cases they are
extinguished more truly than Heracleitus'' sun, inasmuch as they never
light up again. (Heraclitus said that the sun was extinguished every
evening and relighted every morning.)

But what ought to be their course?

Just the opposite. In childhood and youth their study, and what
philosophy they learn, should be suited to their tender years: during
this period while they are growing up towards manhood, the chief and
special care should be given to their bodies that they may have them
to use in the service of philosophy; as life advances and the intellect
begins to mature, let them increase the gymnastics of the soul; but
when the strength of our citizens fails and is past civil and military
duties, then let them range at will and engage in no serious labour,
as we intend them to live happily here, and to crown this life with a
similar happiness in another.

How truly in earnest you are, Socrates! he said; I am sure of that; and
yet most of your hearers, if I am not mistaken, are likely to be still
more earnest in their opposition to you, and will never be convinced;
Thrasymachus least of all.

Do not make a quarrel, I said, between Thrasymachus and me, who have
recently become friends, although, indeed, we were never enemies; for I
shall go on striving to the utmost until I either convert him and other
men, or do something which may profit them against the day when they
live again, and hold the like discourse in another state of existence.

You are speaking of a time which is not very near.

Rather, I replied, of a time which is as nothing in comparison with
eternity. Nevertheless, I do not wonder that the many refuse to believe;
for they have never seen that of which we are now speaking realized;
they have seen only a conventional imitation of philosophy, consisting
of words artificially brought together, not like these of ours having
a natural unity. But a human being who in word and work is perfectly
moulded, as far as he can be, into the proportion and likeness of
virtue--such a man ruling in a city which bears the same image, they
have never yet seen, neither one nor many of them--do you think that
they ever did?

No indeed.

No, my friend, and they have seldom, if ever, heard free and noble
sentiments; such as men utter when they are earnestly and by every means
in their power seeking after truth for the sake of knowledge, while
they look coldly on the subtleties of controversy, of which the end is
opinion and strife, whether they meet with them in the courts of law or
in society.

They are strangers, he said, to the words of which you speak.

And this was what we foresaw, and this was the reason why truth forced
us to admit, not without fear and hesitation, that neither cities nor
States nor individuals will ever attain perfection until the small
class of philosophers whom we termed useless but not corrupt are
providentially compelled, whether they will or not, to take care of the
State, and until a like necessity be laid on the State to obey them; or
until kings, or if not kings, the sons of kings or princes, are divinely
inspired with a true love of true philosophy. That either or both of
these alternatives are impossible, I see no reason to affirm: if
they were so, we might indeed be justly ridiculed as dreamers and
visionaries. Am I not right?

Quite right.

If then, in the countless ages of the past, or at the present hour in
some foreign clime which is far away and beyond our ken, the perfected
philosopher is or has been or hereafter shall be compelled by a superior
power to have the charge of the State, we are ready to assert to the
death, that this our constitution has been, and is--yea, and will be
whenever the Muse of Philosophy is queen. There is no impossibility in
all this; that there is a difficulty, we acknowledge ourselves.

My opinion agrees with yours, he said.

But do you mean to say that this is not the opinion of the multitude?

I should imagine not, he replied.

O my friend, I said, do not attack the multitude: they will change their
minds, if, not in an aggressive spirit, but gently and with the view
of soothing them and removing their dislike of over-education, you show
them your philosophers as they really are and describe as you were just
now doing their character and profession, and then mankind will see that
he of whom you are speaking is not such as they supposed--if they view
him in this new light, they will surely change their notion of him, and
answer in another strain. Who can be at enmity with one who loves them,
who that is himself gentle and free from envy will be jealous of one
in whom there is no jealousy? Nay, let me answer for you, that in a few
this harsh temper may be found but not in the majority of mankind.

I quite agree with you, he said.

And do you not also think, as I do, that the harsh feeling which the
many entertain towards philosophy originates in the pretenders, who rush
in uninvited, and are always abusing them, and finding fault with them,
who make persons instead of things the theme of their conversation? and
nothing can be more unbecoming in philosophers than this.

It is most unbecoming.

For he, Adeimantus, whose mind is fixed upon true being, has surely no
time to look down upon the affairs of earth, or to be filled with malice
and envy, contending against men; his eye is ever directed towards
things fixed and immutable, which he sees neither injuring nor injured
by one another, but all in order moving according to reason; these he
imitates, and to these he will, as far as he can, conform himself. Can a
man help imitating that with which he holds reverential converse?

Impossible.

And the philosopher holding converse with the divine order, becomes
orderly and divine, as far as the nature of man allows; but like every
one else, he will suffer from detraction.

Of course.

And if a necessity be laid upon him of fashioning, not only himself,
but human nature generally, whether in States or individuals, into
that which he beholds elsewhere, will he, think you, be an unskilful
artificer of justice, temperance, and every civil virtue?

Anything but unskilful.

And if the world perceives that what we are saying about him is the
truth, will they be angry with philosophy? Will they disbelieve us, when
we tell them that no State can be happy which is not designed by artists
who imitate the heavenly pattern?

They will not be angry if they understand, he said. But how will they
draw out the plan of which you are speaking?

They will begin by taking the State and the manners of men, from which,
as from a tablet, they will rub out the picture, and leave a clean
surface. This is no easy task. But whether easy or not, herein will lie
the difference between them and every other legislator,--they will have
nothing to do either with individual or State, and will inscribe no
laws, until they have either found, or themselves made, a clean surface.

They will be very right, he said.

Having effected this, they will proceed to trace an outline of the
constitution?

No doubt.

And when they are filling in the work, as I conceive, they will often
turn their eyes upwards and downwards: I mean that they will first look
at absolute justice and beauty and temperance, and again at the human
copy; and will mingle and temper the various elements of life into the
image of a man; and this they will conceive according to that other
image, which, when existing among men, Homer calls the form and likeness
of God.

Very true, he said.

And one feature they will erase, and another they will put in, until
they have made the ways of men, as far as possible, agreeable to the
ways of God?

Indeed, he said, in no way could they make a fairer picture.

And now, I said, are we beginning to persuade those whom you described
as rushing at us with might and main, that the painter of constitutions
is such an one as we are praising; at whom they were so very indignant
because to his hands we committed the State; and are they growing a
little calmer at what they have just heard?

Much calmer, if there is any sense in them.

Why, where can they still find any ground for objection? Will they doubt
that the philosopher is a lover of truth and being?

They would not be so unreasonable.

Or that his nature, being such as we have delineated, is akin to the
highest good?

Neither can they doubt this.

But again, will they tell us that such a nature, placed under favourable
circumstances, will not be perfectly good and wise if any ever was? Or
will they prefer those whom we have rejected?

Surely not.

Then will they still be angry at our saying, that, until philosophers
bear rule, States and individuals will have no rest from evil, nor will
this our imaginary State ever be realized?

I think that they will be less angry.

Shall we assume that they are not only less angry but quite gentle,
and that they have been converted and for very shame, if for no other
reason, cannot refuse to come to terms?

By all means, he said.

Then let us suppose that the reconciliation has been effected. Will any
one deny the other point, that there may be sons of kings or princes who
are by nature philosophers?

Surely no man, he said.

And when they have come into being will any one say that they must of
necessity be destroyed; that they can hardly be saved is not denied even
by us; but that in the whole course of ages no single one of them can
escape--who will venture to affirm this?

Who indeed!

But, said I, one is enough; let there be one man who has a city obedient
to his will, and he might bring into existence the ideal polity about
which the world is so incredulous.

Yes, one is enough.

The ruler may impose the laws and institutions which we have been
describing, and the citizens may possibly be willing to obey them?

Certainly.

And that others should approve, of what we approve, is no miracle or
impossibility?

I think not.

But we have sufficiently shown, in what has preceded, that all this, if
only possible, is assuredly for the best.

We have.

And now we say not only that our laws, if they could be enacted, would
be for the best, but also that the enactment of them, though difficult,
is not impossible.

Very good.

And so with pain and toil we have reached the end of one subject, but
more remains to be discussed;--how and by what studies and pursuits will
the saviours of the constitution be created, and at what ages are they
to apply themselves to their several studies?

Certainly.

I omitted the troublesome business of the possession of women, and the
procreation of children, and the appointment of the rulers, because
I knew that the perfect State would be eyed with jealousy and was
difficult of attainment; but that piece of cleverness was not of much
service to me, for I had to discuss them all the same. The women and
children are now disposed of, but the other question of the rulers must
be investigated from the very beginning. We were saying, as you will
remember, that they were to be lovers of their country, tried by the
test of pleasures and pains, and neither in hardships, nor in dangers,
nor at any other critical moment were to lose their patriotism--he was
to be rejected who failed, but he who always came forth pure, like gold
tried in the refiner's fire, was to be made a ruler, and to receive
honours and rewards in life and after death. This was the sort of thing
which was being said, and then the argument turned aside and veiled her
face; not liking to stir the question which has now arisen.

I perfectly remember, he said.

Yes, my friend, I said, and I then shrank from hazarding the bold
word; but now let me dare to say--that the perfect guardian must be a
philosopher.

Yes, he said, let that be affirmed.

And do not suppose that there will be many of them; for the gifts which
were deemed by us to be essential rarely grow together; they are mostly
found in shreds and patches.

What do you mean? he said.

You are aware, I replied, that quick intelligence, memory, sagacity,
cleverness, and similar qualities, do not often grow together, and that
persons who possess them and are at the same time high-spirited and
magnanimous are not so constituted by nature as to live orderly and in a
peaceful and settled manner; they are driven any way by their impulses,
and all solid principle goes out of them.

Very true, he said.

On the other hand, those steadfast natures which can better be depended
upon, which in a battle are impregnable to fear and immovable, are
equally immovable when there is anything to be learned; they are
always in a torpid state, and are apt to yawn and go to sleep over any
intellectual toil.

Quite true.

And yet we were saying that both qualities were necessary in those to
whom the higher education is to be imparted, and who are to share in any
office or command.

Certainly, he said.

And will they be a class which is rarely found?

Yes, indeed.

Then the aspirant must not only be tested in those labours and dangers
and pleasures which we mentioned before, but there is another kind of
probation which we did not mention--he must be exercised also in many
kinds of knowledge, to see whether the soul will be able to endure the
highest of all, or will faint under them, as in any other studies and
exercises.

Yes, he said, you are quite right in testing him. But what do you mean
by the highest of all knowledge?

You may remember, I said, that we divided the soul into three parts; and
distinguished the several natures of justice, temperance, courage, and
wisdom?

Indeed, he said, if I had forgotten, I should not deserve to hear more.

And do you remember the word of caution which preceded the discussion of
them?

To what do you refer?

We were saying, if I am not mistaken, that he who wanted to see them in
their perfect beauty must take a longer and more circuitous way, at
the end of which they would appear; but that we could add on a popular
exposition of them on a level with the discussion which had preceded.
And you replied that such an exposition would be enough for you, and so
the enquiry was continued in what to me seemed to be a very inaccurate
manner; whether you were satisfied or not, it is for you to say.

Yes, he said, I thought and the others thought that you gave us a fair
measure of truth.

But, my friend, I said, a measure of such things which in any degree
falls short of the whole truth is not fair measure; for nothing
imperfect is the measure of anything, although persons are too apt to be
contented and think that they need search no further.

Not an uncommon case when people are indolent.

Yes, I said; and there cannot be any worse fault in a guardian of the
State and of the laws.

True.

The guardian then, I said, must be required to take the longer circuit,
and toil at learning as well as at gymnastics, or he will never reach
the highest knowledge of all which, as we were just now saying, is his
proper calling.

What, he said, is there a knowledge still higher than this--higher than
justice and the other virtues?

Yes, I said, there is. And of the virtues too we must behold not the
outline merely, as at present--nothing short of the most finished
picture should satisfy us. When little things are elaborated with an
infinity of pains, in order that they may appear in their full beauty
and utmost clearness, how ridiculous that we should not think the
highest truths worthy of attaining the highest accuracy!

A right noble thought; but do you suppose that we shall refrain from
asking you what is this highest knowledge?

Nay, I said, ask if you will; but I am certain that you have heard the
answer many times, and now you either do not understand me or, as I
rather think, you are disposed to be troublesome; for you have often
been told that the idea of good is the highest knowledge, and that all
other things become useful and advantageous only by their use of this.
You can hardly be ignorant that of this I was about to speak, concerning
which, as you have often heard me say, we know so little; and, without
which, any other knowledge or possession of any kind will profit us
nothing. Do you think that the possession of all other things is of
any value if we do not possess the good? or the knowledge of all other
things if we have no knowledge of beauty and goodness?

Assuredly not.

You are further aware that most people affirm pleasure to be the good,
but the finer sort of wits say it is knowledge?

Yes.

And you are aware too that the latter cannot explain what they mean by
knowledge, but are obliged after all to say knowledge of the good?

How ridiculous!

Yes, I said, that they should begin by reproaching us with our ignorance
of the good, and then presume our knowledge of it--for the good they
define to be knowledge of the good, just as if we understood them when
they use the term `good'--this is of course ridiculous.

Most true, he said.

And those who make pleasure their good are in equal perplexity; for they
are compelled to admit that there are bad pleasures as well as good.

Certainly.

And therefore to acknowledge that bad and good are the same?

True.

There can be no doubt about the numerous difficulties in which this
question is involved.

There can be none.

Further, do we not see that many are willing to do or to have or to seem
to be what is just and honourable without the reality; but no one is
satisfied with the appearance of good--the reality is what they seek; in
the case of the good, appearance is despised by every one.

Very true, he said.

Of this then, which every soul of man pursues and makes the end of all
his actions, having a presentiment that there is such an end, and
yet hesitating because neither knowing the nature nor having the same
assurance of this as of other things, and therefore losing whatever
good there is in other things,--of a principle such and so great as this
ought the best men in our State, to whom everything is entrusted, to be
in the darkness of ignorance?

Certainly not, he said.

I am sure, I said, that he who does not know how the beautiful and
the just are likewise good will be but a sorry guardian of them; and
I suspect that no one who is ignorant of the good will have a true
knowledge of them.

That, he said, is a shrewd suspicion of yours.

And if we only have a guardian who has this knowledge our State will be
perfectly ordered?

Of course, he replied; but I wish that you would tell me whether you
conceive this supreme principle of the good to be knowledge or pleasure,
or different from either?

Aye, I said, I knew all along that a fastidious gentleman like you would
not be contented with the thoughts of other people about these matters.

True, Socrates; but I must say that one who like you has passed a
lifetime in the study of philosophy should not be always repeating the
opinions of others, and never telling his own.

Well, but has any one a right to say positively what he does not know?

Not, he said, with the assurance of positive certainty; he has no right
to do that: but he may say what he thinks, as a matter of opinion.

And do you not know, I said, that all mere opinions are bad, and the
best of them blind? You would not deny that those who have any true
notion without intelligence are only like blind men who feel their way
along the road?

Very true.

And do you wish to behold what is blind and crooked and base, when
others will tell you of brightness and beauty?

Still, I must implore you, Socrates, said Glaucon, not to turn away just
as you are reaching the goal; if you will only give such an explanation
of the good as you have already given of justice and temperance and the
other virtues, we shall be satisfied.

Yes, my friend, and I shall be at least equally satisfied, but I cannot
help fearing that I shall fail, and that my indiscreet zeal will bring
ridicule upon me. No, sweet sirs, let us not at present ask what is the
actual nature of the good, for to reach what is now in my thoughts
would be an effort too great for me. But of the child of the good who
is likest him, I would fain speak, if I could be sure that you wished to
hear--otherwise, not.

By all means, he said, tell us about the child, and you shall remain in
our debt for the account of the parent.

I do indeed wish, I replied, that I could pay, and you receive, the
account of the parent, and not, as now, of the offspring only; take,
however, this latter by way of interest, and at the same time have a
care that I do not render a false account, although I have no intention
of deceiving you.

Yes, we will take all the care that we can: proceed.

Yes, I said, but I must first come to an understanding with you, and
remind you of what I have mentioned in the course of this discussion,
and at many other times.

What?

The old story, that there is a many beautiful and a many good, and so
of other things which we describe and define; to all of them the term
``many'' is applied.

True, he said.

And there is an absolute beauty and an absolute good, and of other
things to which the term ``many'' is applied there is an absolute; for
they may be brought under a single idea, which is called the essence of
each.

Very true.

The many, as we say, are seen but not known, and the ideas are known but
not seen.

Exactly.

And what is the organ with which we see the visible things?

The sight, he said.

And with the hearing, I said, we hear, and with the other senses
perceive the other objects of sense?

True.

But have you remarked that sight is by far the most costly and complex
piece of workmanship which the artificer of the senses ever contrived?

No, I never have, he said.

Then reflect; has the ear or voice need of any third or additional
nature in order that the one may be able to hear and the other to be
heard?

Nothing of the sort.

No, indeed, I replied; and the same is true of most, if not all, the
other senses--you would not say that any of them requires such an
addition?

Certainly not.

But you see that without the addition of some other nature there is no
seeing or being seen?

How do you mean?

Sight being, as I conceive, in the eyes, and he who has eyes wanting to
see; colour being also present in them, still unless there be a third
nature specially adapted to the purpose, the owner of the eyes will see
nothing and the colours will be invisible.

Of what nature are you speaking?

Of that which you term light, I replied.

True, he said.

Noble, then, is the bond which links together sight and visibility, and
great beyond other bonds by no small difference of nature; for light is
their bond, and light is no ignoble thing?

Nay, he said, the reverse of ignoble.

And which, I said, of the gods in heaven would you say was the lord of
this element? Whose is that light which makes the eye to see perfectly
and the visible to appear?

You mean the sun, as you and all mankind say.

May not the relation of sight to this deity be described as follows?

How?

Neither sight nor the eye in which sight resides is the sun?

No.

Yet of all the organs of sense the eye is the most like the sun?

By far the most like.

And the power which the eye possesses is a sort of effluence which is
dispensed from the sun?

Exactly.

Then the sun is not sight, but the author of sight who is recognised by
sight?

True, he said.

And this is he whom I call the child of the good, whom the good begat in
his own likeness, to be in the visible world, in relation to sight
and the things of sight, what the good is in the intellectual world in
relation to mind and the things of mind:

Will you be a little more explicit? he said.

Why, you know, I said, that the eyes, when a person directs them towards
objects on which the light of day is no longer shining, but the moon
and stars only, see dimly, and are nearly blind; they seem to have no
clearness of vision in them?

Very true.

But when they are directed towards objects on which the sun shines, they
see clearly and there is sight in them?

Certainly.

And the soul is like the eye: when resting upon that on which truth and
being shine, the soul perceives and understands, and is radiant with
intelligence; but when turned towards the twilight of becoming and
perishing, then she has opinion only, and goes blinking about, and
is first of one opinion and then of another, and seems to have no
intelligence?

Just so.

Now, that which imparts truth to the known and the power of knowing to
the knower is what I would have you term the idea of good, and this
you will deem to be the cause of science, and of truth in so far as
the latter becomes the subject of knowledge; beautiful too, as are both
truth and knowledge, you will be right in esteeming this other nature as
more beautiful than either; and, as in the previous instance, light and
sight may be truly said to be like the sun, and yet not to be the sun,
so in this other sphere, science and truth may be deemed to be like the
good, but not the good; the good has a place of honour yet higher.

What a wonder of beauty that must be, he said, which is the author of
science and truth, and yet surpasses them in beauty; for you surely
cannot mean to say that pleasure is the good?

God forbid, I replied; but may I ask you to consider the image in
another point of view?

In what point of view?

You would say, would you not, that the sun is not only the author of
visibility in all visible things, but of generation and nourishment and
growth, though he himself is not generation?

Certainly.

In like manner the good may be said to be not only the author of
knowledge to all things known, but of their being and essence, and yet
the good is not essence, but far exceeds essence in dignity and power.

Glaucon said, with a ludicrous earnestness: By the light of heaven, how
amazing!

Yes, I said, and the exaggeration may be set down to you; for you made
me utter my fancies.

And pray continue to utter them; at any rate let us hear if there is
anything more to be said about the similitude of the sun.

Yes, I said, there is a great deal more.

Then omit nothing, however slight.

I will do my best, I said; but I should think that a great deal will
have to be omitted.

I hope not, he said.

You have to imagine, then, that there are two ruling powers, and that
one of them is set over the intellectual world, the other over the
visible. I do not say heaven, lest you should fancy that I am playing
upon the name ('ourhanoz, orhatoz'). May I suppose that you have this
distinction of the visible and intelligible fixed in your mind?

I have.

Now take a line which has been cut into two unequal parts, and divide
each of them again in the same proportion, and suppose the two
main divisions to answer, one to the visible and the other to the
intelligible, and then compare the subdivisions in respect of their
clearness and want of clearness, and you will find that the first
section in the sphere of the visible consists of images. And by images I
mean, in the first place, shadows, and in the second place, reflections
in water and in solid, smooth and polished bodies and the like: Do you
understand?

Yes, I understand.

Imagine, now, the other section, of which this is only the resemblance,
to include the animals which we see, and everything that grows or is
made.

Very good.

Would you not admit that both the sections of this division have
different degrees of truth, and that the copy is to the original as the
sphere of opinion is to the sphere of knowledge?

Most undoubtedly.

Next proceed to consider the manner in which the sphere of the
intellectual is to be divided.

In what manner?

Thus:--There are two subdivisions, in the lower of which the soul uses
the figures given by the former division as images; the enquiry can only
be hypothetical, and instead of going upwards to a principle descends
to the other end; in the higher of the two, the soul passes out of
hypotheses, and goes up to a principle which is above hypotheses, making
no use of images as in the former case, but proceeding only in and
through the ideas themselves.

I do not quite understand your meaning, he said.

Then I will try again; you will understand me better when I have made
some preliminary remarks. You are aware that students of geometry,
arithmetic, and the kindred sciences assume the odd and the even and the
figures and three kinds of angles and the like in their several branches
of science; these are their hypotheses, which they and every body are
supposed to know, and therefore they do not deign to give any account of
them either to themselves or others; but they begin with them, and go
on until they arrive at last, and in a consistent manner, at their
conclusion?

Yes, he said, I know.

And do you not know also that although they make use of the visible
forms and reason about them, they are thinking not of these, but of the
ideals which they resemble; not of the figures which they draw, but
of the absolute square and the absolute diameter, and so on--the forms
which they draw or make, and which have shadows and reflections in water
of their own, are converted by them into images, but they are really
seeking to behold the things themselves, which can only be seen with the
eye of the mind?

That is true.

And of this kind I spoke as the intelligible, although in the search
after it the soul is compelled to use hypotheses; not ascending to
a first principle, because she is unable to rise above the region of
hypothesis, but employing the objects of which the shadows below are
resemblances in their turn as images, they having in relation to the
shadows and reflections of them a greater distinctness, and therefore a
higher value.

I understand, he said, that you are speaking of the province of geometry
and the sister arts.

And when I speak of the other division of the intelligible, you will
understand me to speak of that other sort of knowledge which reason
herself attains by the power of dialectic, using the hypotheses not as
first principles, but only as hypotheses--that is to say, as steps and
points of departure into a world which is above hypotheses, in order
that she may soar beyond them to the first principle of the whole; and
clinging to this and then to that which depends on this, by successive
steps she descends again without the aid of any sensible object, from
ideas, through ideas, and in ideas she ends.

I understand you, he replied; not perfectly, for you seem to me to
be describing a task which is really tremendous; but, at any rate, I
understand you to say that knowledge and being, which the science of
dialectic contemplates, are clearer than the notions of the arts, as
they are termed, which proceed from hypotheses only: these are also
contemplated by the understanding, and not by the senses: yet, because
they start from hypotheses and do not ascend to a principle, those who
contemplate them appear to you not to exercise the higher reason
upon them, although when a first principle is added to them they are
cognizable by the higher reason. And the habit which is concerned
with geometry and the cognate sciences I suppose that you would term
understanding and not reason, as being intermediate between opinion and
reason.

You have quite conceived my meaning, I said; and now, corresponding to
these four divisions, let there be four faculties in the soul--reason
answering to the highest, understanding to the second, faith (or
conviction) to the third, and perception of shadows to the last--and let
there be a scale of them, and let us suppose that the several faculties
have clearness in the same degree that their objects have truth.

I understand, he replied, and give my assent, and accept your
arrangement.

% section book_vi (end)

\section{Book VII} % (fold)
\label{sec:book_vii}



BOOK VII.

And now, I said, let me show in a figure how far our nature is
enlightened or unenlightened:--Behold! human beings living in a
underground den, which has a mouth open towards the light and reaching
all along the den; here they have been from their childhood, and have
their legs and necks chained so that they cannot move, and can only
see before them, being prevented by the chains from turning round
their heads. Above and behind them a fire is blazing at a distance, and
between the fire and the prisoners there is a raised way; and you will
see, if you look, a low wall built along the way, like the screen which
marionette players have in front of them, over which they show the
puppets.

I see.

And do you see, I said, men passing along the wall carrying all sorts of
vessels, and statues and figures of animals made of wood and stone and
various materials, which appear over the wall? Some of them are talking,
others silent.

You have shown me a strange image, and they are strange prisoners.

Like ourselves, I replied; and they see only their own shadows, or the
shadows of one another, which the fire throws on the opposite wall of
the cave?

True, he said; how could they see anything but the shadows if they were
never allowed to move their heads?

And of the objects which are being carried in like manner they would
only see the shadows?

Yes, he said.

And if they were able to converse with one another, would they not
suppose that they were naming what was actually before them?

Very true.

And suppose further that the prison had an echo which came from the
other side, would they not be sure to fancy when one of the passers-by
spoke that the voice which they heard came from the passing shadow?

No question, he replied.

To them, I said, the truth would be literally nothing but the shadows of
the images.

That is certain.

And now look again, and see what will naturally follow if the prisoners
are released and disabused of their error. At first, when any of them is
liberated and compelled suddenly to stand up and turn his neck round and
walk and look towards the light, he will suffer sharp pains; the glare
will distress him, and he will be unable to see the realities of which
in his former state he had seen the shadows; and then conceive some one
saying to him, that what he saw before was an illusion, but that now,
when he is approaching nearer to being and his eye is turned towards
more real existence, he has a clearer vision,--what will be his reply?
And you may further imagine that his instructor is pointing to the
objects as they pass and requiring him to name them,--will he not be
perplexed? Will he not fancy that the shadows which he formerly saw are
truer than the objects which are now shown to him?

Far truer.

And if he is compelled to look straight at the light, will he not have
a pain in his eyes which will make him turn away to take refuge in the
objects of vision which he can see, and which he will conceive to be in
reality clearer than the things which are now being shown to him?

True, he said.

And suppose once more, that he is reluctantly dragged up a steep and
rugged ascent, and held fast until he is forced into the presence of
the sun himself, is he not likely to be pained and irritated? When he
approaches the light his eyes will be dazzled, and he will not be able
to see anything at all of what are now called realities.

Not all in a moment, he said.

He will require to grow accustomed to the sight of the upper world.
And first he will see the shadows best, next the reflections of men and
other objects in the water, and then the objects themselves; then he
will gaze upon the light of the moon and the stars and the spangled
heaven; and he will see the sky and the stars by night better than the
sun or the light of the sun by day?

Certainly.

Last of all he will be able to see the sun, and not mere reflections of
him in the water, but he will see him in his own proper place, and not
in another; and he will contemplate him as he is.

Certainly.

He will then proceed to argue that this is he who gives the season and
the years, and is the guardian of all that is in the visible world, and
in a certain way the cause of all things which he and his fellows have
been accustomed to behold?

Clearly, he said, he would first see the sun and then reason about him.

And when he remembered his old habitation, and the wisdom of the den
and his fellow-prisoners, do you not suppose that he would felicitate
himself on the change, and pity them?

Certainly, he would.

And if they were in the habit of conferring honours among themselves
on those who were quickest to observe the passing shadows and to remark
which of them went before, and which followed after, and which were
together; and who were therefore best able to draw conclusions as to the
future, do you think that he would care for such honours and glories, or
envy the possessors of them? Would he not say with Homer,

``Better to be the poor servant of a poor master,''

and to endure anything, rather than think as they do and live after
their manner?

Yes, he said, I think that he would rather suffer anything than
entertain these false notions and live in this miserable manner.

Imagine once more, I said, such an one coming suddenly out of the sun
to be replaced in his old situation; would he not be certain to have his
eyes full of darkness?

To be sure, he said.

And if there were a contest, and he had to compete in measuring the
shadows with the prisoners who had never moved out of the den, while
his sight was still weak, and before his eyes had become steady (and the
time which would be needed to acquire this new habit of sight might be
very considerable), would he not be ridiculous? Men would say of him
that up he went and down he came without his eyes; and that it was
better not even to think of ascending; and if any one tried to loose
another and lead him up to the light, let them only catch the offender,
and they would put him to death.

No question, he said.

This entire allegory, I said, you may now append, dear Glaucon, to the
previous argument; the prison-house is the world of sight, the light of
the fire is the sun, and you will not misapprehend me if you interpret
the journey upwards to be the ascent of the soul into the intellectual
world according to my poor belief, which, at your desire, I have
expressed--whether rightly or wrongly God knows. But, whether true or
false, my opinion is that in the world of knowledge the idea of good
appears last of all, and is seen only with an effort; and, when seen,
is also inferred to be the universal author of all things beautiful and
right, parent of light and of the lord of light in this visible world,
and the immediate source of reason and truth in the intellectual; and
that this is the power upon which he who would act rationally either in
public or private life must have his eye fixed.

I agree, he said, as far as I am able to understand you.

Moreover, I said, you must not wonder that those who attain to this
beatific vision are unwilling to descend to human affairs; for their
souls are ever hastening into the upper world where they desire to
dwell; which desire of theirs is very natural, if our allegory may be
trusted.

Yes, very natural.

And is there anything surprising in one who passes from divine
contemplations to the evil state of man, misbehaving himself in a
ridiculous manner; if, while his eyes are blinking and before he has
become accustomed to the surrounding darkness, he is compelled to fight
in courts of law, or in other places, about the images or the shadows of
images of justice, and is endeavouring to meet the conceptions of those
who have never yet seen absolute justice?

Anything but surprising, he replied.

Any one who has common sense will remember that the bewilderments of the
eyes are of two kinds, and arise from two causes, either from coming out
of the light or from going into the light, which is true of the mind's
eye, quite as much as of the bodily eye; and he who remembers this when
he sees any one whose vision is perplexed and weak, will not be too
ready to laugh; he will first ask whether that soul of man has come out
of the brighter life, and is unable to see because unaccustomed to the
dark, or having turned from darkness to the day is dazzled by excess
of light. And he will count the one happy in his condition and state of
being, and he will pity the other; or, if he have a mind to laugh at the
soul which comes from below into the light, there will be more reason
in this than in the laugh which greets him who returns from above out of
the light into the den.

That, he said, is a very just distinction.

But then, if I am right, certain professors of education must be wrong
when they say that they can put a knowledge into the soul which was not
there before, like sight into blind eyes.

They undoubtedly say this, he replied.

Whereas, our argument shows that the power and capacity of learning
exists in the soul already; and that just as the eye was unable to turn
from darkness to light without the whole body, so too the instrument of
knowledge can only by the movement of the whole soul be turned from the
world of becoming into that of being, and learn by degrees to endure
the sight of being, and of the brightest and best of being, or in other
words, of the good.

Very true.

And must there not be some art which will effect conversion in the
easiest and quickest manner; not implanting the faculty of sight, for
that exists already, but has been turned in the wrong direction, and is
looking away from the truth?

Yes, he said, such an art may be presumed.

And whereas the other so-called virtues of the soul seem to be akin to
bodily qualities, for even when they are not originally innate they can
be implanted later by habit and exercise, the virtue of wisdom more than
anything else contains a divine element which always remains, and by
this conversion is rendered useful and profitable; or, on the other
hand, hurtful and useless. Did you never observe the narrow intelligence
flashing from the keen eye of a clever rogue--how eager he is, how
clearly his paltry soul sees the way to his end; he is the reverse of
blind, but his keen eye-sight is forced into the service of evil, and he
is mischievous in proportion to his cleverness?

Very true, he said.

But what if there had been a circumcision of such natures in the days
of their youth; and they had been severed from those sensual pleasures,
such as eating and drinking, which, like leaden weights, were attached
to them at their birth, and which drag them down and turn the vision
of their souls upon the things that are below--if, I say, they had been
released from these impediments and turned in the opposite direction,
the very same faculty in them would have seen the truth as keenly as
they see what their eyes are turned to now.

Very likely.

Yes, I said; and there is another thing which is likely, or rather a
necessary inference from what has preceded, that neither the uneducated
and uninformed of the truth, nor yet those who never make an end of
their education, will be able ministers of State; not the former,
because they have no single aim of duty which is the rule of all their
actions, private as well as public; nor the latter, because they will
not act at all except upon compulsion, fancying that they are already
dwelling apart in the islands of the blest.

Very true, he replied.

Then, I said, the business of us who are the founders of the State
will be to compel the best minds to attain that knowledge which we have
already shown to be the greatest of all--they must continue to ascend
until they arrive at the good; but when they have ascended and seen
enough we must not allow them to do as they do now.

What do you mean?

I mean that they remain in the upper world: but this must not be
allowed; they must be made to descend again among the prisoners in the
den, and partake of their labours and honours, whether they are worth
having or not.

But is not this unjust? he said; ought we to give them a worse life,
when they might have a better?

You have again forgotten, my friend, I said, the intention of the
legislator, who did not aim at making any one class in the State happy
above the rest; the happiness was to be in the whole State, and he
held the citizens together by persuasion and necessity, making them
benefactors of the State, and therefore benefactors of one another;
to this end he created them, not to please themselves, but to be his
instruments in binding up the State.

True, he said, I had forgotten.

Observe, Glaucon, that there will be no injustice in compelling our
philosophers to have a care and providence of others; we shall explain
to them that in other States, men of their class are not obliged to
share in the toils of politics: and this is reasonable, for they grow up
at their own sweet will, and the government would rather not have them.
Being self-taught, they cannot be expected to show any gratitude for a
culture which they have never received. But we have brought you into
the world to be rulers of the hive, kings of yourselves and of the other
citizens, and have educated you far better and more perfectly than they
have been educated, and you are better able to share in the double duty.
Wherefore each of you, when his turn comes, must go down to the general
underground abode, and get the habit of seeing in the dark. When you
have acquired the habit, you will see ten thousand times better than the
inhabitants of the den, and you will know what the several images are,
and what they represent, because you have seen the beautiful and just
and good in their truth. And thus our State, which is also yours, will
be a reality, and not a dream only, and will be administered in a spirit
unlike that of other States, in which men fight with one another about
shadows only and are distracted in the struggle for power, which in
their eyes is a great good. Whereas the truth is that the State in which
the rulers are most reluctant to govern is always the best and most
quietly governed, and the State in which they are most eager, the worst.

Quite true, he replied.

And will our pupils, when they hear this, refuse to take their turn at
the toils of State, when they are allowed to spend the greater part of
their time with one another in the heavenly light?

Impossible, he answered; for they are just men, and the commands which
we impose upon them are just; there can be no doubt that every one of
them will take office as a stern necessity, and not after the fashion of
our present rulers of State.

Yes, my friend, I said; and there lies the point. You must contrive for
your future rulers another and a better life than that of a ruler, and
then you may have a well-ordered State; for only in the State which
offers this, will they rule who are truly rich, not in silver and gold,
but in virtue and wisdom, which are the true blessings of life. Whereas
if they go to the administration of public affairs, poor and hungering
after their own private advantage, thinking that hence they are to
snatch the chief good, order there can never be; for they will be
fighting about office, and the civil and domestic broils which thus
arise will be the ruin of the rulers themselves and of the whole State.

Most true, he replied.

And the only life which looks down upon the life of political ambition
is that of true philosophy. Do you know of any other?

Indeed, I do not, he said.

And those who govern ought not to be lovers of the task? For, if they
are, there will be rival lovers, and they will fight.

No question.

Who then are those whom we shall compel to be guardians? Surely they
will be the men who are wisest about affairs of State, and by whom the
State is best administered, and who at the same time have other honours
and another and a better life than that of politics?

They are the men, and I will choose them, he replied.

And now shall we consider in what way such guardians will be produced,
and how they are to be brought from darkness to light,--as some are said
to have ascended from the world below to the gods?

By all means, he replied.

The process, I said, is not the turning over of an oyster-shell (In
allusion to a game in which two parties fled or pursued according as an
oyster-shell which was thrown into the air fell with the dark or light
side uppermost.), but the turning round of a soul passing from a day
which is little better than night to the true day of being, that is, the
ascent from below, which we affirm to be true philosophy?

Quite so.

And should we not enquire what sort of knowledge has the power of
effecting such a change?

Certainly.

What sort of knowledge is there which would draw the soul from becoming
to being? And another consideration has just occurred to me: You will
remember that our young men are to be warrior athletes?

Yes, that was said.

Then this new kind of knowledge must have an additional quality?

What quality?

Usefulness in war.

Yes, if possible.

There were two parts in our former scheme of education, were there not?

Just so.

There was gymnastic which presided over the growth and decay of the
body, and may therefore be regarded as having to do with generation and
corruption?

True.

Then that is not the knowledge which we are seeking to discover?

No.

But what do you say of music, which also entered to a certain extent
into our former scheme?

Music, he said, as you will remember, was the counterpart of gymnastic,
and trained the guardians by the influences of habit, by harmony making
them harmonious, by rhythm rhythmical, but not giving them science; and
the words, whether fabulous or possibly true, had kindred elements of
rhythm and harmony in them. But in music there was nothing which tended
to that good which you are now seeking.

You are most accurate, I said, in your recollection; in music there
certainly was nothing of the kind. But what branch of knowledge is
there, my dear Glaucon, which is of the desired nature; since all the
useful arts were reckoned mean by us?

Undoubtedly; and yet if music and gymnastic are excluded, and the arts
are also excluded, what remains?

Well, I said, there may be nothing left of our special subjects; and
then we shall have to take something which is not special, but of
universal application.

What may that be?

A something which all arts and sciences and intelligences use in common,
and which every one first has to learn among the elements of education.

What is that?

The little matter of distinguishing one, two, and three--in a word,
number and calculation:--do not all arts and sciences necessarily
partake of them?

Yes.

Then the art of war partakes of them?

To be sure.

Then Palamedes, whenever he appears in tragedy, proves Agamemnon
ridiculously unfit to be a general. Did you never remark how he declares
that he had invented number, and had numbered the ships and set in array
the ranks of the army at Troy; which implies that they had never been
numbered before, and Agamemnon must be supposed literally to have been
incapable of counting his own feet--how could he if he was ignorant of
number? And if that is true, what sort of general must he have been?

I should say a very strange one, if this was as you say.

Can we deny that a warrior should have a knowledge of arithmetic?

Certainly he should, if he is to have the smallest understanding of
military tactics, or indeed, I should rather say, if he is to be a man
at all.

I should like to know whether you have the same notion which I have of
this study?

What is your notion?

It appears to me to be a study of the kind which we are seeking, and
which leads naturally to reflection, but never to have been rightly
used; for the true use of it is simply to draw the soul towards being.

Will you explain your meaning? he said.

I will try, I said; and I wish you would share the enquiry with me,
and say ``yes'' or ``no'' when I attempt to distinguish in my own mind what
branches of knowledge have this attracting power, in order that we may
have clearer proof that arithmetic is, as I suspect, one of them.

Explain, he said.

I mean to say that objects of sense are of two kinds; some of them do
not invite thought because the sense is an adequate judge of them; while
in the case of other objects sense is so untrustworthy that further
enquiry is imperatively demanded.

You are clearly referring, he said, to the manner in which the senses
are imposed upon by distance, and by painting in light and shade.

No, I said, that is not at all my meaning.

Then what is your meaning?

When speaking of uninviting objects, I mean those which do not pass from
one sensation to the opposite; inviting objects are those which do; in
this latter case the sense coming upon the object, whether at a distance
or near, gives no more vivid idea of anything in particular than of its
opposite. An illustration will make my meaning clearer:--here are three
fingers--a little finger, a second finger, and a middle finger.

Very good.

You may suppose that they are seen quite close: And here comes the
point.

What is it?

Each of them equally appears a finger, whether seen in the middle or
at the extremity, whether white or black, or thick or thin--it makes no
difference; a finger is a finger all the same. In these cases a man is
not compelled to ask of thought the question what is a finger? for the
sight never intimates to the mind that a finger is other than a finger.

True.

And therefore, I said, as we might expect, there is nothing here which
invites or excites intelligence.

There is not, he said.

But is this equally true of the greatness and smallness of the fingers?
Can sight adequately perceive them? and is no difference made by the
circumstance that one of the fingers is in the middle and another at
the extremity? And in like manner does the touch adequately perceive the
qualities of thickness or thinness, of softness or hardness? And so of
the other senses; do they give perfect intimations of such matters? Is
not their mode of operation on this wise--the sense which is concerned
with the quality of hardness is necessarily concerned also with the
quality of softness, and only intimates to the soul that the same thing
is felt to be both hard and soft?

You are quite right, he said.

And must not the soul be perplexed at this intimation which the sense
gives of a hard which is also soft? What, again, is the meaning of
light and heavy, if that which is light is also heavy, and that which is
heavy, light?

Yes, he said, these intimations which the soul receives are very curious
and require to be explained.

Yes, I said, and in these perplexities the soul naturally summons to her
aid calculation and intelligence, that she may see whether the several
objects announced to her are one or two.

True.

And if they turn out to be two, is not each of them one and different?

Certainly.

And if each is one, and both are two, she will conceive the two as in
a state of division, for if there were undivided they could only be
conceived of as one?

True.

The eye certainly did see both small and great, but only in a confused
manner; they were not distinguished.

Yes.

Whereas the thinking mind, intending to light up the chaos, was
compelled to reverse the process, and look at small and great as
separate and not confused.

Very true.

Was not this the beginning of the enquiry ``What is great?'' and ``What is
small?''

Exactly so.

And thus arose the distinction of the visible and the intelligible.

Most true.

This was what I meant when I spoke of impressions which invited the
intellect, or the reverse--those which are simultaneous with opposite
impressions, invite thought; those which are not simultaneous do not.

I understand, he said, and agree with you.

And to which class do unity and number belong?

I do not know, he replied.

Think a little and you will see that what has preceded will supply the
answer; for if simple unity could be adequately perceived by the sight
or by any other sense, then, as we were saying in the case of the
finger, there would be nothing to attract towards being; but when there
is some contradiction always present, and one is the reverse of one and
involves the conception of plurality, then thought begins to be aroused
within us, and the soul perplexed and wanting to arrive at a decision
asks ``What is absolute unity?'' This is the way in which the study of the
one has a power of drawing and converting the mind to the contemplation
of true being.

And surely, he said, this occurs notably in the case of one; for we see
the same thing to be both one and infinite in multitude?

Yes, I said; and this being true of one must be equally true of all
number?

Certainly.

And all arithmetic and calculation have to do with number?

Yes.

And they appear to lead the mind towards truth?

Yes, in a very remarkable manner.

Then this is knowledge of the kind for which we are seeking, having a
double use, military and philosophical; for the man of war must learn
the art of number or he will not know how to array his troops, and the
philosopher also, because he has to rise out of the sea of change and
lay hold of true being, and therefore he must be an arithmetician.

That is true.

And our guardian is both warrior and philosopher?

Certainly.

Then this is a kind of knowledge which legislation may fitly prescribe;
and we must endeavour to persuade those who are to be the principal men
of our State to go and learn arithmetic, not as amateurs, but they must
carry on the study until they see the nature of numbers with the mind
only; nor again, like merchants or retail-traders, with a view to buying
or selling, but for the sake of their military use, and of the soul
herself; and because this will be the easiest way for her to pass from
becoming to truth and being.

That is excellent, he said.

Yes, I said, and now having spoken of it, I must add how charming the
science is! and in how many ways it conduces to our desired end, if
pursued in the spirit of a philosopher, and not of a shopkeeper!

How do you mean?

I mean, as I was saying, that arithmetic has a very great and elevating
effect, compelling the soul to reason about abstract number, and
rebelling against the introduction of visible or tangible objects into
the argument. You know how steadily the masters of the art repel and
ridicule any one who attempts to divide absolute unity when he is
calculating, and if you divide, they multiply (Meaning either (1)
that they integrate the number because they deny the possibility of
fractions; or (2) that division is regarded by them as a process of
multiplication, for the fractions of one continue to be units.), taking
care that one shall continue one and not become lost in fractions.

That is very true.

Now, suppose a person were to say to them: O my friends, what are these
wonderful numbers about which you are reasoning, in which, as you say,
there is a unity such as you demand, and each unit is equal, invariable,
indivisible,--what would they answer?

They would answer, as I should conceive, that they were speaking of
those numbers which can only be realized in thought.

Then you see that this knowledge may be truly called necessary,
necessitating as it clearly does the use of the pure intelligence in the
attainment of pure truth?

Yes; that is a marked characteristic of it.

And have you further observed, that those who have a natural talent for
calculation are generally quick at every other kind of knowledge; and
even the dull, if they have had an arithmetical training, although they
may derive no other advantage from it, always become much quicker than
they would otherwise have been.

Very true, he said.

And indeed, you will not easily find a more difficult study, and not
many as difficult.

You will not.

And, for all these reasons, arithmetic is a kind of knowledge in which
the best natures should be trained, and which must not be given up.

I agree.

Let this then be made one of our subjects of education. And next, shall
we enquire whether the kindred science also concerns us?

You mean geometry?

Exactly so.

Clearly, he said, we are concerned with that part of geometry which
relates to war; for in pitching a camp, or taking up a position,
or closing or extending the lines of an army, or any other military
manoeuvre, whether in actual battle or on a march, it will make all the
difference whether a general is or is not a geometrician.

Yes, I said, but for that purpose a very little of either geometry or
calculation will be enough; the question relates rather to the greater
and more advanced part of geometry--whether that tends in any degree
to make more easy the vision of the idea of good; and thither, as I was
saying, all things tend which compel the soul to turn her gaze towards
that place, where is the full perfection of being, which she ought, by
all means, to behold.

True, he said.

Then if geometry compels us to view being, it concerns us; if becoming
only, it does not concern us?

Yes, that is what we assert.

Yet anybody who has the least acquaintance with geometry will not deny
that such a conception of the science is in flat contradiction to the
ordinary language of geometricians.

How so?

They have in view practice only, and are always speaking, in a narrow
and ridiculous manner, of squaring and extending and applying and the
like--they confuse the necessities of geometry with those of daily life;
whereas knowledge is the real object of the whole science.

Certainly, he said.

Then must not a further admission be made?

What admission?

That the knowledge at which geometry aims is knowledge of the eternal,
and not of aught perishing and transient.

That, he replied, may be readily allowed, and is true.

Then, my noble friend, geometry will draw the soul towards truth,
and create the spirit of philosophy, and raise up that which is now
unhappily allowed to fall down.

Nothing will be more likely to have such an effect.

Then nothing should be more sternly laid down than that the inhabitants
of your fair city should by all means learn geometry. Moreover the
science has indirect effects, which are not small.

Of what kind? he said.

There are the military advantages of which you spoke, I said; and in all
departments of knowledge, as experience proves, any one who has studied
geometry is infinitely quicker of apprehension than one who has not.

Yes indeed, he said, there is an infinite difference between them.

Then shall we propose this as a second branch of knowledge which our
youth will study?

Let us do so, he replied.

And suppose we make astronomy the third--what do you say?

I am strongly inclined to it, he said; the observation of the seasons
and of months and years is as essential to the general as it is to the
farmer or sailor.

I am amused, I said, at your fear of the world, which makes you guard
against the appearance of insisting upon useless studies; and I quite
admit the difficulty of believing that in every man there is an eye
of the soul which, when by other pursuits lost and dimmed, is by these
purified and re-illumined; and is more precious far than ten thousand
bodily eyes, for by it alone is truth seen. Now there are two classes of
persons: one class of those who will agree with you and will take
your words as a revelation; another class to whom they will be utterly
unmeaning, and who will naturally deem them to be idle tales, for they
see no sort of profit which is to be obtained from them. And therefore
you had better decide at once with which of the two you are proposing to
argue. You will very likely say with neither, and that your chief aim in
carrying on the argument is your own improvement; at the same time you
do not grudge to others any benefit which they may receive.

I think that I should prefer to carry on the argument mainly on my own
behalf.

Then take a step backward, for we have gone wrong in the order of the
sciences.

What was the mistake? he said.

After plane geometry, I said, we proceeded at once to solids in
revolution, instead of taking solids in themselves; whereas after the
second dimension the third, which is concerned with cubes and dimensions
of depth, ought to have followed.

That is true, Socrates; but so little seems to be known as yet about
these subjects.

Why, yes, I said, and for two reasons:--in the first place, no
government patronises them; this leads to a want of energy in the
pursuit of them, and they are difficult; in the second place, students
cannot learn them unless they have a director. But then a director
can hardly be found, and even if he could, as matters now stand,
the students, who are very conceited, would not attend to him. That,
however, would be otherwise if the whole State became the director of
these studies and gave honour to them; then disciples would want to
come, and there would be continuous and earnest search, and discoveries
would be made; since even now, disregarded as they are by the world, and
maimed of their fair proportions, and although none of their votaries
can tell the use of them, still these studies force their way by their
natural charm, and very likely, if they had the help of the State, they
would some day emerge into light.

Yes, he said, there is a remarkable charm in them. But I do not clearly
understand the change in the order. First you began with a geometry of
plane surfaces?

Yes, I said.

And you placed astronomy next, and then you made a step backward?

Yes, and I have delayed you by my hurry; the ludicrous state of solid
geometry, which, in natural order, should have followed, made me pass
over this branch and go on to astronomy, or motion of solids.

True, he said.

Then assuming that the science now omitted would come into existence
if encouraged by the State, let us go on to astronomy, which will be
fourth.

The right order, he replied. And now, Socrates, as you rebuked the
vulgar manner in which I praised astronomy before, my praise shall
be given in your own spirit. For every one, as I think, must see that
astronomy compels the soul to look upwards and leads us from this world
to another.

Every one but myself, I said; to every one else this may be clear, but
not to me.

And what then would you say?

I should rather say that those who elevate astronomy into philosophy
appear to me to make us look downwards and not upwards.

What do you mean? he asked.

You, I replied, have in your mind a truly sublime conception of our
knowledge of the things above. And I dare say that if a person were to
throw his head back and study the fretted ceiling, you would still think
that his mind was the percipient, and not his eyes. And you are very
likely right, and I may be a simpleton: but, in my opinion, that
knowledge only which is of being and of the unseen can make the soul
look upwards, and whether a man gapes at the heavens or blinks on the
ground, seeking to learn some particular of sense, I would deny that he
can learn, for nothing of that sort is matter of science; his soul is
looking downwards, not upwards, whether his way to knowledge is by water
or by land, whether he floats, or only lies on his back.

I acknowledge, he said, the justice of your rebuke. Still, I should like
to ascertain how astronomy can be learned in any manner more conducive
to that knowledge of which we are speaking?

I will tell you, I said: The starry heaven which we behold is wrought
upon a visible ground, and therefore, although the fairest and most
perfect of visible things, must necessarily be deemed inferior far to
the true motions of absolute swiftness and absolute slowness, which are
relative to each other, and carry with them that which is contained in
them, in the true number and in every true figure. Now, these are to be
apprehended by reason and intelligence, but not by sight.

True, he replied.

The spangled heavens should be used as a pattern and with a view to that
higher knowledge; their beauty is like the beauty of figures or pictures
excellently wrought by the hand of Daedalus, or some other great artist,
which we may chance to behold; any geometrician who saw them would
appreciate the exquisiteness of their workmanship, but he would never
dream of thinking that in them he could find the true equal or the true
double, or the truth of any other proportion.

No, he replied, such an idea would be ridiculous.

And will not a true astronomer have the same feeling when he looks at
the movements of the stars? Will he not think that heaven and the things
in heaven are framed by the Creator of them in the most perfect manner?
But he will never imagine that the proportions of night and day, or of
both to the month, or of the month to the year, or of the stars to these
and to one another, and any other things that are material and visible
can also be eternal and subject to no deviation--that would be absurd;
and it is equally absurd to take so much pains in investigating their
exact truth.

I quite agree, though I never thought of this before.

Then, I said, in astronomy, as in geometry, we should employ problems,
and let the heavens alone if we would approach the subject in the right
way and so make the natural gift of reason to be of any real use.

That, he said, is a work infinitely beyond our present astronomers.

Yes, I said; and there are many other things which must also have a
similar extension given to them, if our legislation is to be of any
value. But can you tell me of any other suitable study?

No, he said, not without thinking.

Motion, I said, has many forms, and not one only; two of them are
obvious enough even to wits no better than ours; and there are others,
as I imagine, which may be left to wiser persons.

But where are the two?

There is a second, I said, which is the counterpart of the one already
named.

And what may that be?

The second, I said, would seem relatively to the ears to be what the
first is to the eyes; for I conceive that as the eyes are designed to
look up at the stars, so are the ears to hear harmonious motions; and
these are sister sciences--as the Pythagoreans say, and we, Glaucon,
agree with them?

Yes, he replied.

But this, I said, is a laborious study, and therefore we had better go
and learn of them; and they will tell us whether there are any other
applications of these sciences. At the same time, we must not lose sight
of our own higher object.

What is that?

There is a perfection which all knowledge ought to reach, and which our
pupils ought also to attain, and not to fall short of, as I was saying
that they did in astronomy. For in the science of harmony, as you
probably know, the same thing happens. The teachers of harmony compare
the sounds and consonances which are heard only, and their labour, like
that of the astronomers, is in vain.

Yes, by heaven! he said; and tis as good as a play to hear them talking
about their condensed notes, as they call them; they put their ears
close alongside of the strings like persons catching a sound from their
neighbour's wall--one set of them declaring that they distinguish an
intermediate note and have found the least interval which should be
the unit of measurement; the others insisting that the two sounds have
passed into the same--either party setting their ears before their
understanding.

You mean, I said, those gentlemen who tease and torture the strings and
rack them on the pegs of the instrument: I might carry on the metaphor
and speak after their manner of the blows which the plectrum gives,
and make accusations against the strings, both of backwardness and
forwardness to sound; but this would be tedious, and therefore I will
only say that these are not the men, and that I am referring to the
Pythagoreans, of whom I was just now proposing to enquire about harmony.
For they too are in error, like the astronomers; they investigate the
numbers of the harmonies which are heard, but they never attain to
problems--that is to say, they never reach the natural harmonies of
number, or reflect why some numbers are harmonious and others not.

That, he said, is a thing of more than mortal knowledge.

A thing, I replied, which I would rather call useful; that is, if sought
after with a view to the beautiful and good; but if pursued in any other
spirit, useless.

Very true, he said.

Now, when all these studies reach the point of inter-communion and
connection with one another, and come to be considered in their mutual
affinities, then, I think, but not till then, will the pursuit of them
have a value for our objects; otherwise there is no profit in them.

I suspect so; but you are speaking, Socrates, of a vast work.

What do you mean? I said; the prelude or what? Do you not know that all
this is but the prelude to the actual strain which we have to learn? For
you surely would not regard the skilled mathematician as a dialectician?

Assuredly not, he said; I have hardly ever known a mathematician who was
capable of reasoning.

But do you imagine that men who are unable to give and take a reason
will have the knowledge which we require of them?

Neither can this be supposed.

And so, Glaucon, I said, we have at last arrived at the hymn of
dialectic. This is that strain which is of the intellect only, but which
the faculty of sight will nevertheless be found to imitate; for sight,
as you may remember, was imagined by us after a while to behold the
real animals and stars, and last of all the sun himself. And so with
dialectic; when a person starts on the discovery of the absolute by
the light of reason only, and without any assistance of sense, and
perseveres until by pure intelligence he arrives at the perception
of the absolute good, he at last finds himself at the end of the
intellectual world, as in the case of sight at the end of the visible.

Exactly, he said.

Then this is the progress which you call dialectic?

True.

But the release of the prisoners from chains, and their translation
from the shadows to the images and to the light, and the ascent from the
underground den to the sun, while in his presence they are vainly trying
to look on animals and plants and the light of the sun, but are able to
perceive even with their weak eyes the images in the water (which are
divine), and are the shadows of true existence (not shadows of images
cast by a light of fire, which compared with the sun is only an
image)--this power of elevating the highest principle in the soul to
the contemplation of that which is best in existence, with which we may
compare the raising of that faculty which is the very light of the body
to the sight of that which is brightest in the material and visible
world--this power is given, as I was saying, by all that study and
pursuit of the arts which has been described.

I agree in what you are saying, he replied, which may be hard to
believe, yet, from another point of view, is harder still to deny. This,
however, is not a theme to be treated of in passing only, but will have
to be discussed again and again. And so, whether our conclusion be true
or false, let us assume all this, and proceed at once from the prelude
or preamble to the chief strain (A play upon the Greek word, which means
both ``law'' and ``strain.'', and describe that in like manner. Say, then,
what is the nature and what are the divisions of dialectic, and what
are the paths which lead thither; for these paths will also lead to our
final rest.

Dear Glaucon, I said, you will not be able to follow me here, though
I would do my best, and you should behold not an image only but the
absolute truth, according to my notion. Whether what I told you would
or would not have been a reality I cannot venture to say; but you would
have seen something like reality; of that I am confident.

Doubtless, he replied.

But I must also remind you, that the power of dialectic alone can reveal
this, and only to one who is a disciple of the previous sciences.

Of that assertion you may be as confident as of the last.

And assuredly no one will argue that there is any other method
of comprehending by any regular process all true existence or of
ascertaining what each thing is in its own nature; for the arts in
general are concerned with the desires or opinions of men, or are
cultivated with a view to production and construction, or for the
preservation of such productions and constructions; and as to the
mathematical sciences which, as we were saying, have some apprehension
of true being--geometry and the like--they only dream about being,
but never can they behold the waking reality so long as they leave the
hypotheses which they use unexamined, and are unable to give an account
of them. For when a man knows not his own first principle, and when the
conclusion and intermediate steps are also constructed out of he knows
not what, how can he imagine that such a fabric of convention can ever
become science?

Impossible, he said.

Then dialectic, and dialectic alone, goes directly to the first
principle and is the only science which does away with hypotheses in
order to make her ground secure; the eye of the soul, which is literally
buried in an outlandish slough, is by her gentle aid lifted upwards;
and she uses as handmaids and helpers in the work of conversion, the
sciences which we have been discussing. Custom terms them sciences,
but they ought to have some other name, implying greater clearness
than opinion and less clearness than science: and this, in our previous
sketch, was called understanding. But why should we dispute about names
when we have realities of such importance to consider?

Why indeed, he said, when any name will do which expresses the thought
of the mind with clearness?

At any rate, we are satisfied, as before, to have four divisions;
two for intellect and two for opinion, and to call the first division
science, the second understanding, the third belief, and the fourth
perception of shadows, opinion being concerned with becoming, and
intellect with being; and so to make a proportion:--

As being is to becoming, so is pure intellect to opinion. And as
intellect is to opinion, so is science to belief, and understanding to
the perception of shadows.

But let us defer the further correlation and subdivision of the subjects
of opinion and of intellect, for it will be a long enquiry, many times
longer than this has been.

As far as I understand, he said, I agree.

And do you also agree, I said, in describing the dialectician as one who
attains a conception of the essence of each thing? And he who does not
possess and is therefore unable to impart this conception, in
whatever degree he fails, may in that degree also be said to fail in
intelligence? Will you admit so much?

Yes, he said; how can I deny it?

And you would say the same of the conception of the good? Until the
person is able to abstract and define rationally the idea of good,
and unless he can run the gauntlet of all objections, and is ready to
disprove them, not by appeals to opinion, but to absolute truth, never
faltering at any step of the argument--unless he can do all this, you
would say that he knows neither the idea of good nor any other good; he
apprehends only a shadow, if anything at all, which is given by opinion
and not by science;--dreaming and slumbering in this life, before he
is well awake here, he arrives at the world below, and has his final
quietus.

In all that I should most certainly agree with you.

And surely you would not have the children of your ideal State, whom you
are nurturing and educating--if the ideal ever becomes a reality--you
would not allow the future rulers to be like posts (Literally ``lines,''
probably the starting-point of a race-course.), having no reason in
them, and yet to be set in authority over the highest matters?

Certainly not.

Then you will make a law that they shall have such an education as
will enable them to attain the greatest skill in asking and answering
questions?

Yes, he said, you and I together will make it.

Dialectic, then, as you will agree, is the coping-stone of the sciences,
and is set over them; no other science can be placed higher--the nature
of knowledge can no further go?

I agree, he said.

But to whom we are to assign these studies, and in what way they are to
be assigned, are questions which remain to be considered.

Yes, clearly.

You remember, I said, how the rulers were chosen before?

Certainly, he said.

The same natures must still be chosen, and the preference again given
to the surest and the bravest, and, if possible, to the fairest; and,
having noble and generous tempers, they should also have the natural
gifts which will facilitate their education.

And what are these?

Such gifts as keenness and ready powers of acquisition; for the mind
more often faints from the severity of study than from the severity of
gymnastics: the toil is more entirely the mind's own, and is not shared
with the body.

Very true, he replied.

Further, he of whom we are in search should have a good memory, and be
an unwearied solid man who is a lover of labour in any line; or he will
never be able to endure the great amount of bodily exercise and to go
through all the intellectual discipline and study which we require of
him.

Certainly, he said; he must have natural gifts.

The mistake at present is, that those who study philosophy have no
vocation, and this, as I was before saying, is the reason why she has
fallen into disrepute: her true sons should take her by the hand and not
bastards.

What do you mean?

In the first place, her votary should not have a lame or halting
industry--I mean, that he should not be half industrious and half idle:
as, for example, when a man is a lover of gymnastic and hunting, and all
other bodily exercises, but a hater rather than a lover of the labour
of learning or listening or enquiring. Or the occupation to which he
devotes himself may be of an opposite kind, and he may have the other
sort of lameness.

Certainly, he said.

And as to truth, I said, is not a soul equally to be deemed halt and
lame which hates voluntary falsehood and is extremely indignant at
herself and others when they tell lies, but is patient of involuntary
falsehood, and does not mind wallowing like a swinish beast in the mire
of ignorance, and has no shame at being detected?

To be sure.

And, again, in respect of temperance, courage, magnificence, and every
other virtue, should we not carefully distinguish between the true son
and the bastard? for where there is no discernment of such qualities
states and individuals unconsciously err; and the state makes a ruler,
and the individual a friend, of one who, being defective in some part of
virtue, is in a figure lame or a bastard.

That is very true, he said.

All these things, then, will have to be carefully considered by us; and
if only those whom we introduce to this vast system of education and
training are sound in body and mind, justice herself will have nothing
to say against us, and we shall be the saviours of the constitution and
of the State; but, if our pupils are men of another stamp, the reverse
will happen, and we shall pour a still greater flood of ridicule on
philosophy than she has to endure at present.

That would not be creditable.

Certainly not, I said; and yet perhaps, in thus turning jest into
earnest I am equally ridiculous.

In what respect?

I had forgotten, I said, that we were not serious, and spoke with too
much excitement. For when I saw philosophy so undeservedly trampled
under foot of men I could not help feeling a sort of indignation at the
authors of her disgrace: and my anger made me too vehement.

Indeed! I was listening, and did not think so.

But I, who am the speaker, felt that I was. And now let me remind you
that, although in our former selection we chose old men, we must not do
so in this. Solon was under a delusion when he said that a man when he
grows old may learn many things--for he can no more learn much than he
can run much; youth is the time for any extraordinary toil.

Of course.

And, therefore, calculation and geometry and all the other elements of
instruction, which are a preparation for dialectic, should be presented
to the mind in childhood; not, however, under any notion of forcing our
system of education.

Why not?

Because a freeman ought not to be a slave in the acquisition of
knowledge of any kind. Bodily exercise, when compulsory, does no harm
to the body; but knowledge which is acquired under compulsion obtains no
hold on the mind.

Very true.

Then, my good friend, I said, do not use compulsion, but let early
education be a sort of amusement; you will then be better able to find
out the natural bent.

That is a very rational notion, he said.

Do you remember that the children, too, were to be taken to see the
battle on horseback; and that if there were no danger they were to be
brought close up and, like young hounds, have a taste of blood given
them?

Yes, I remember.

The same practice may be followed, I said, in all these things--labours,
lessons, dangers--and he who is most at home in all of them ought to be
enrolled in a select number.

At what age?

At the age when the necessary gymnastics are over: the period whether of
two or three years which passes in this sort of training is useless for
any other purpose; for sleep and exercise are unpropitious to learning;
and the trial of who is first in gymnastic exercises is one of the most
important tests to which our youth are subjected.

Certainly, he replied.

After that time those who are selected from the class of twenty years
old will be promoted to higher honour, and the sciences which they
learned without any order in their early education will now be brought
together, and they will be able to see the natural relationship of them
to one another and to true being.

Yes, he said, that is the only kind of knowledge which takes lasting
root.

Yes, I said; and the capacity for such knowledge is the great criterion
of dialectical talent: the comprehensive mind is always the dialectical.

I agree with you, he said.

These, I said, are the points which you must consider; and those who
have most of this comprehension, and who are most steadfast in their
learning, and in their military and other appointed duties, when they
have arrived at the age of thirty have to be chosen by you out of the
select class, and elevated to higher honour; and you will have to prove
them by the help of dialectic, in order to learn which of them is able
to give up the use of sight and the other senses, and in company with
truth to attain absolute being: And here, my friend, great caution is
required.

Why great caution?

Do you not remark, I said, how great is the evil which dialectic has
introduced?

What evil? he said.

The students of the art are filled with lawlessness.

Quite true, he said.

Do you think that there is anything so very unnatural or inexcusable in
their case? or will you make allowance for them?

In what way make allowance?

I want you, I said, by way of parallel, to imagine a supposititious son
who is brought up in great wealth; he is one of a great and numerous
family, and has many flatterers. When he grows up to manhood, he learns
that his alleged are not his real parents; but who the real are he
is unable to discover. Can you guess how he will be likely to behave
towards his flatterers and his supposed parents, first of all during the
period when he is ignorant of the false relation, and then again when he
knows? Or shall I guess for you?

If you please.

Then I should say, that while he is ignorant of the truth he will be
likely to honour his father and his mother and his supposed relations
more than the flatterers; he will be less inclined to neglect them when
in need, or to do or say anything against them; and he will be less
willing to disobey them in any important matter.

He will.

But when he has made the discovery, I should imagine that he would
diminish his honour and regard for them, and would become more devoted
to the flatterers; their influence over him would greatly increase; he
would now live after their ways, and openly associate with them, and,
unless he were of an unusually good disposition, he would trouble
himself no more about his supposed parents or other relations.

Well, all that is very probable. But how is the image applicable to the
disciples of philosophy?

In this way: you know that there are certain principles about justice
and honour, which were taught us in childhood, and under their parental
authority we have been brought up, obeying and honouring them.

That is true.

There are also opposite maxims and habits of pleasure which flatter and
attract the soul, but do not influence those of us who have any sense of
right, and they continue to obey and honour the maxims of their fathers.

True.

Now, when a man is in this state, and the questioning spirit asks what
is fair or honourable, and he answers as the legislator has taught him,
and then arguments many and diverse refute his words, until he is driven
into believing that nothing is honourable any more than dishonourable,
or just and good any more than the reverse, and so of all the notions
which he most valued, do you think that he will still honour and obey
them as before?

Impossible.

And when he ceases to think them honourable and natural as heretofore,
and he fails to discover the true, can he be expected to pursue any life
other than that which flatters his desires?

He cannot.

And from being a keeper of the law he is converted into a breaker of it?

Unquestionably.

Now all this is very natural in students of philosophy such as I have
described, and also, as I was just now saying, most excusable.

Yes, he said; and, I may add, pitiable.

Therefore, that your feelings may not be moved to pity about our
citizens who are now thirty years of age, every care must be taken in
introducing them to dialectic.

Certainly.

There is a danger lest they should taste the dear delight too early; for
youngsters, as you may have observed, when they first get the taste
in their mouths, argue for amusement, and are always contradicting and
refuting others in imitation of those who refute them; like puppy-dogs,
they rejoice in pulling and tearing at all who come near them.

Yes, he said, there is nothing which they like better.

And when they have made many conquests and received defeats at the hands
of many, they violently and speedily get into a way of not believing
anything which they believed before, and hence, not only they, but
philosophy and all that relates to it is apt to have a bad name with the
rest of the world.

Too true, he said.

But when a man begins to get older, he will no longer be guilty of such
insanity; he will imitate the dialectician who is seeking for truth, and
not the eristic, who is contradicting for the sake of amusement; and the
greater moderation of his character will increase instead of diminishing
the honour of the pursuit.

Very true, he said.

And did we not make special provision for this, when we said that the
disciples of philosophy were to be orderly and steadfast, not, as now,
any chance aspirant or intruder?

Very true.

Suppose, I said, the study of philosophy to take the place of gymnastics
and to be continued diligently and earnestly and exclusively for twice
the number of years which were passed in bodily exercise--will that be
enough?

Would you say six or four years? he asked.

Say five years, I replied; at the end of the time they must be sent down
again into the den and compelled to hold any military or other office
which young men are qualified to hold: in this way they will get their
experience of life, and there will be an opportunity of trying whether,
when they are drawn all manner of ways by temptation, they will stand
firm or flinch.

And how long is this stage of their lives to last?

Fifteen years, I answered; and when they have reached fifty years of
age, then let those who still survive and have distinguished themselves
in every action of their lives and in every branch of knowledge come at
last to their consummation: the time has now arrived at which they must
raise the eye of the soul to the universal light which lightens all
things, and behold the absolute good; for that is the pattern according
to which they are to order the State and the lives of individuals, and
the remainder of their own lives also; making philosophy their chief
pursuit, but, when their turn comes, toiling also at politics and ruling
for the public good, not as though they were performing some heroic
action, but simply as a matter of duty; and when they have brought up in
each generation others like themselves and left them in their place to
be governors of the State, then they will depart to the Islands of the
Blest and dwell there; and the city will give them public memorials and
sacrifices and honour them, if the Pythian oracle consent, as demigods,
but if not, as in any case blessed and divine.

You are a sculptor, Socrates, and have made statues of our governors
faultless in beauty.

Yes, I said, Glaucon, and of our governesses too; for you must not
suppose that what I have been saying applies to men only and not to
women as far as their natures can go.

There you are right, he said, since we have made them to share in all
things like the men.

Well, I said, and you would agree (would you not?) that what has
been said about the State and the government is not a mere dream, and
although difficult not impossible, but only possible in the way which
has been supposed; that is to say, when the true philosopher kings are
born in a State, one or more of them, despising the honours of this
present world which they deem mean and worthless, esteeming above all
things right and the honour that springs from right, and regarding
justice as the greatest and most necessary of all things, whose
ministers they are, and whose principles will be exalted by them when
they set in order their own city?

How will they proceed?

They will begin by sending out into the country all the inhabitants of
the city who are more than ten years old, and will take possession of
their children, who will be unaffected by the habits of their parents;
these they will train in their own habits and laws, I mean in the laws
which we have given them: and in this way the State and constitution of
which we were speaking will soonest and most easily attain happiness,
and the nation which has such a constitution will gain most.

Yes, that will be the best way. And I think, Socrates, that you have
very well described how, if ever, such a constitution might come into
being.

Enough then of the perfect State, and of the man who bears its
image--there is no difficulty in seeing how we shall describe him.

There is no difficulty, he replied; and I agree with you in thinking
that nothing more need be said.




% BOOK VIII.

% And so, Glaucon, we have arrived at the conclusion that in the perfect
% State wives and children are to be in common; and that all education
% and the pursuits of war and peace are also to be common, and the best
% philosophers and the bravest warriors are to be their kings?

% That, replied Glaucon, has been acknowledged.

% Yes, I said; and we have further acknowledged that the governors, when
% appointed themselves, will take their soldiers and place them in houses
% such as we were describing, which are common to all, and contain nothing
% private, or individual; and about their property, you remember what we
% agreed?

% Yes, I remember that no one was to have any of the ordinary possessions
% of mankind; they were to be warrior athletes and guardians, receiving
% from the other citizens, in lieu of annual payment, only their
% maintenance, and they were to take care of themselves and of the whole
% State.

% True, I said; and now that this division of our task is concluded, let
% us find the point at which we digressed, that we may return into the old
% path.

% There is no difficulty in returning; you implied, then as now, that you
% had finished the description of the State: you said that such a State
% was good, and that the man was good who answered to it, although, as now
% appears, you had more excellent things to relate both of State and man.
% And you said further, that if this was the true form, then the others
% were false; and of the false forms, you said, as I remember, that there
% were four principal ones, and that their defects, and the defects of
% the individuals corresponding to them, were worth examining. When we had
% seen all the individuals, and finally agreed as to who was the best and
% who was the worst of them, we were to consider whether the best was not
% also the happiest, and the worst the most miserable. I asked you
% what were the four forms of government of which you spoke, and then
% Polemarchus and Adeimantus put in their word; and you began again, and
% have found your way to the point at which we have now arrived.

% Your recollection, I said, is most exact.

% Then, like a wrestler, he replied, you must put yourself again in the
% same position; and let me ask the same questions, and do you give me the
% same answer which you were about to give me then.

% Yes, if I can, I will, I said.

% I shall particularly wish to hear what were the four constitutions of
% which you were speaking.

% That question, I said, is easily answered: the four governments of which
% I spoke, so far as they have distinct names, are, first, those of Crete
% and Sparta, which are generally applauded; what is termed oligarchy
% comes next; this is not equally approved, and is a form of government
% which teems with evils: thirdly, democracy, which naturally follows
% oligarchy, although very different: and lastly comes tyranny, great
% and famous, which differs from them all, and is the fourth and worst
% disorder of a State. I do not know, do you? of any other constitution
% which can be said to have a distinct character. There are lordships and
% principalities which are bought and sold, and some other intermediate
% forms of government. But these are nondescripts and may be found equally
% among Hellenes and among barbarians.

% Yes, he replied, we certainly hear of many curious forms of government
% which exist among them.

% Do you know, I said, that governments vary as the dispositions of men
% vary, and that there must be as many of the one as there are of the
% other? For we cannot suppose that States are made of ``oak and rock,'' and
% not out of the human natures which are in them, and which in a figure
% turn the scale and draw other things after them?

% Yes, he said, the States are as the men are; they grow out of human
% characters.

% Then if the constitutions of States are five, the dispositions of
% individual minds will also be five?

% Certainly.

% Him who answers to aristocracy, and whom we rightly call just and good,
% we have already described.

% We have.

% Then let us now proceed to describe the inferior sort of natures, being
% the contentious and ambitious, who answer to the Spartan polity; also
% the oligarchical, democratical, and tyrannical. Let us place the most
% just by the side of the most unjust, and when we see them we shall be
% able to compare the relative happiness or unhappiness of him who leads
% a life of pure justice or pure injustice. The enquiry will then be
% completed. And we shall know whether we ought to pursue injustice,
% as Thrasymachus advises, or in accordance with the conclusions of the
% argument to prefer justice.

% Certainly, he replied, we must do as you say.

% Shall we follow our old plan, which we adopted with a view to clearness,
% of taking the State first and then proceeding to the individual, and
% begin with the government of honour?--I know of no name for such a
% government other than timocracy, or perhaps timarchy. We will compare
% with this the like character in the individual; and, after that,
% consider oligarchy and the oligarchical man; and then again we will turn
% our attention to democracy and the democratical man; and lastly, we
% will go and view the city of tyranny, and once more take a look into the
% tyrant's soul, and try to arrive at a satisfactory decision.

% That way of viewing and judging of the matter will be very suitable.

% First, then, I said, let us enquire how timocracy (the government of
% honour) arises out of aristocracy (the government of the best). Clearly,
% all political changes originate in divisions of the actual governing
% power; a government which is united, however small, cannot be moved.

% Very true, he said.

% In what way, then, will our city be moved, and in what manner will the
% two classes of auxiliaries and rulers disagree among themselves or with
% one another? Shall we, after the manner of Homer, pray the Muses to tell
% us ``how discord first arose'? Shall we imagine them in solemn mockery,
% to play and jest with us as if we were children, and to address us in a
% lofty tragic vein, making believe to be in earnest?

% How would they address us?

% After this manner:--A city which is thus constituted can hardly be
% shaken; but, seeing that everything which has a beginning has also an
% end, even a constitution such as yours will not last for ever, but will
% in time be dissolved. And this is the dissolution:--In plants that grow
% in the earth, as well as in animals that move on the earth's surface,
% fertility and sterility of soul and body occur when the circumferences
% of the circles of each are completed, which in short-lived existences
% pass over a short space, and in long-lived ones over a long space. But
% to the knowledge of human fecundity and sterility all the wisdom and
% education of your rulers will not attain; the laws which regulate them
% will not be discovered by an intelligence which is alloyed with sense,
% but will escape them, and they will bring children into the world when
% they ought not. Now that which is of divine birth has a period which is
% contained in a perfect number (i.e. a cyclical number, such as 6, which
% is equal to the sum of its divisors 1, 2, 3, so that when the circle
% or time represented by 6 is completed, the lesser times or rotations
% represented by 1, 2, 3 are also completed.), but the period of
% human birth is comprehended in a number in which first increments
% by involution and evolution (or squared and cubed) obtaining three
% intervals and four terms of like and unlike, waxing and waning numbers,
% make all the terms commensurable and agreeable to one another. (Probably
% the numbers 3, 4, 5, 6 of which the three first = the sides of the
% Pythagorean triangle. The terms will then be 3 cubed, 4 cubed, 5 cubed,
% which together = 6 cubed = 216.) The base of these (3) with a third
% added (4) when combined with five (20) and raised to the third power
% furnishes two harmonies; the first a square which is a hundred times
% as great (400 = 4 x 100) (Or the first a square which is 100 x 100 =
% 10,000. The whole number will then be 17,500 = a square of 100, and an
% oblong of 100 by 75.), and the other a figure having one side equal to
% the former, but oblong, consisting of a hundred numbers squared upon
% rational diameters of a square (i.e. omitting fractions), the side of
% which is five (7 x 7 = 49 x 100 = 4900), each of them being less by one
% (than the perfect square which includes the fractions, sc. 50) or less
% by (Or, ``consisting of two numbers squared upon irrational diameters,''
% etc. = 100. For other explanations of the passage see Introduction.) two
% perfect squares of irrational diameters (of a square the side of which
% is five = 50 + 50 = 100); and a hundred cubes of three (27 x 100 = 2700
% + 4900 + 400 = 8000). Now this number represents a geometrical figure
% which has control over the good and evil of births. For when your
% guardians are ignorant of the law of births, and unite bride and
% bridegroom out of season, the children will not be goodly or
% fortunate. And though only the best of them will be appointed by their
% predecessors, still they will be unworthy to hold their fathers'' places,
% and when they come into power as guardians, they will soon be found
% to fail in taking care of us, the Muses, first by under-valuing music;
% which neglect will soon extend to gymnastic; and hence the young men of
% your State will be less cultivated. In the succeeding generation rulers
% will be appointed who have lost the guardian power of testing the metal
% of your different races, which, like Hesiod's, are of gold and silver
% and brass and iron. And so iron will be mingled with silver, and brass
% with gold, and hence there will arise dissimilarity and inequality and
% irregularity, which always and in all places are causes of hatred
% and war. This the Muses affirm to be the stock from which discord has
% sprung, wherever arising; and this is their answer to us.

% Yes, and we may assume that they answer truly.

% Why, yes, I said, of course they answer truly; how can the Muses speak
% falsely?

% And what do the Muses say next?

% When discord arose, then the two races were drawn different ways: the
% iron and brass fell to acquiring money and land and houses and gold and
% silver; but the gold and silver races, not wanting money but having the
% true riches in their own nature, inclined towards virtue and the ancient
% order of things. There was a battle between them, and at last they
% agreed to distribute their land and houses among individual owners;
% and they enslaved their friends and maintainers, whom they had formerly
% protected in the condition of freemen, and made of them subjects and
% servants; and they themselves were engaged in war and in keeping a watch
% against them.

% I believe that you have rightly conceived the origin of the change.

% And the new government which thus arises will be of a form intermediate
% between oligarchy and aristocracy?

% Very true.

% Such will be the change, and after the change has been made, how will
% they proceed? Clearly, the new State, being in a mean between oligarchy
% and the perfect State, will partly follow one and partly the other, and
% will also have some peculiarities.

% True, he said.

% In the honour given to rulers, in the abstinence of the warrior class
% from agriculture, handicrafts, and trade in general, in the institution
% of common meals, and in the attention paid to gymnastics and military
% training--in all these respects this State will resemble the former.

% True.

% But in the fear of admitting philosophers to power, because they are no
% longer to be had simple and earnest, but are made up of mixed elements;
% and in turning from them to passionate and less complex characters, who
% are by nature fitted for war rather than peace; and in the value set
% by them upon military stratagems and contrivances, and in the waging of
% everlasting wars--this State will be for the most part peculiar.

% Yes.

% Yes, I said; and men of this stamp will be covetous of money, like those
% who live in oligarchies; they will have, a fierce secret longing after
% gold and silver, which they will hoard in dark places, having magazines
% and treasuries of their own for the deposit and concealment of them;
% also castles which are just nests for their eggs, and in which they will
% spend large sums on their wives, or on any others whom they please.

% That is most true, he said.

% And they are miserly because they have no means of openly acquiring the
% money which they prize; they will spend that which is another man's on
% the gratification of their desires, stealing their pleasures and running
% away like children from the law, their father: they have been schooled
% not by gentle influences but by force, for they have neglected her
% who is the true Muse, the companion of reason and philosophy, and have
% honoured gymnastic more than music.

% Undoubtedly, he said, the form of government which you describe is a
% mixture of good and evil.

% Why, there is a mixture, I said; but one thing, and one thing only, is
% predominantly seen,--the spirit of contention and ambition; and these
% are due to the prevalence of the passionate or spirited element.

% Assuredly, he said.

% Such is the origin and such the character of this State, which has been
% described in outline only; the more perfect execution was not required,
% for a sketch is enough to show the type of the most perfectly just and
% most perfectly unjust; and to go through all the States and all the
% characters of men, omitting none of them, would be an interminable
% labour.

% Very true, he replied.

% Now what man answers to this form of government-how did he come into
% being, and what is he like?

% I think, said Adeimantus, that in the spirit of contention which
% characterises him, he is not unlike our friend Glaucon.

% Perhaps, I said, he may be like him in that one point; but there are
% other respects in which he is very different.

% In what respects?

% He should have more of self-assertion and be less cultivated, and yet
% a friend of culture; and he should be a good listener, but no speaker.
% Such a person is apt to be rough with slaves, unlike the educated man,
% who is too proud for that; and he will also be courteous to freemen, and
% remarkably obedient to authority; he is a lover of power and a lover of
% honour; claiming to be a ruler, not because he is eloquent, or on any
% ground of that sort, but because he is a soldier and has performed feats
% of arms; he is also a lover of gymnastic exercises and of the chase.

% Yes, that is the type of character which answers to timocracy.

% Such an one will despise riches only when he is young; but as he gets
% older he will be more and more attracted to them, because he has a
% piece of the avaricious nature in him, and is not single-minded towards
% virtue, having lost his best guardian.

% Who was that? said Adeimantus.

% Philosophy, I said, tempered with music, who comes and takes up her
% abode in a man, and is the only saviour of his virtue throughout life.

% Good, he said.

% Such, I said, is the timocratical youth, and he is like the timocratical
% State.

% Exactly.

% His origin is as follows:--He is often the young son of a brave father,
% who dwells in an ill-governed city, of which he declines the honours
% and offices, and will not go to law, or exert himself in any way, but is
% ready to waive his rights in order that he may escape trouble.

% And how does the son come into being?

% The character of the son begins to develope when he hears his mother
% complaining that her husband has no place in the government, of which
% the consequence is that she has no precedence among other women.
% Further, when she sees her husband not very eager about money, and
% instead of battling and railing in the law courts or assembly, taking
% whatever happens to him quietly; and when she observes that his thoughts
% always centre in himself, while he treats her with very considerable
% indifference, she is annoyed, and says to her son that his father is
% only half a man and far too easy-going: adding all the other complaints
% about her own ill-treatment which women are so fond of rehearsing.

% Yes, said Adeimantus, they give us plenty of them, and their complaints
% are so like themselves.

% And you know, I said, that the old servants also, who are supposed to
% be attached to the family, from time to time talk privately in the same
% strain to the son; and if they see any one who owes money to his father,
% or is wronging him in any way, and he fails to prosecute them, they tell
% the youth that when he grows up he must retaliate upon people of this
% sort, and be more of a man than his father. He has only to walk abroad
% and he hears and sees the same sort of thing: those who do their own
% business in the city are called simpletons, and held in no esteem, while
% the busy-bodies are honoured and applauded. The result is that the young
% man, hearing and seeing all these things--hearing, too, the words of
% his father, and having a nearer view of his way of life, and making
% comparisons of him and others--is drawn opposite ways: while his father
% is watering and nourishing the rational principle in his soul, the
% others are encouraging the passionate and appetitive; and he being not
% originally of a bad nature, but having kept bad company, is at last
% brought by their joint influence to a middle point, and gives up the
% kingdom which is within him to the middle principle of contentiousness
% and passion, and becomes arrogant and ambitious.

% You seem to me to have described his origin perfectly.

% Then we have now, I said, the second form of government and the second
% type of character?

% We have.

% Next, let us look at another man who, as Aeschylus says,

% ``Is set over against another State;''

% or rather, as our plan requires, begin with the State.

% By all means.

% I believe that oligarchy follows next in order.

% And what manner of government do you term oligarchy?

% A government resting on a valuation of property, in which the rich have
% power and the poor man is deprived of it.

% I understand, he replied.

% Ought I not to begin by describing how the change from timocracy to
% oligarchy arises?

% Yes.

% Well, I said, no eyes are required in order to see how the one passes
% into the other.

% How?

% The accumulation of gold in the treasury of private individuals is the
% ruin of timocracy; they invent illegal modes of expenditure; for what do
% they or their wives care about the law?

% Yes, indeed.

% And then one, seeing another grow rich, seeks to rival him, and thus the
% great mass of the citizens become lovers of money.

% Likely enough.

% And so they grow richer and richer, and the more they think of making
% a fortune the less they think of virtue; for when riches and virtue are
% placed together in the scales of the balance, the one always rises as
% the other falls.

% True.

% And in proportion as riches and rich men are honoured in the State,
% virtue and the virtuous are dishonoured.

% Clearly.

% And what is honoured is cultivated, and that which has no honour is
% neglected.

% That is obvious.

% And so at last, instead of loving contention and glory, men become
% lovers of trade and money; they honour and look up to the rich man, and
% make a ruler of him, and dishonour the poor man.

% They do so.

% They next proceed to make a law which fixes a sum of money as the
% qualification of citizenship; the sum is higher in one place and lower
% in another, as the oligarchy is more or less exclusive; and they allow
% no one whose property falls below the amount fixed to have any share in
% the government. These changes in the constitution they effect by force
% of arms, if intimidation has not already done their work.

% Very true.

% And this, speaking generally, is the way in which oligarchy is
% established.

% Yes, he said; but what are the characteristics of this form of
% government, and what are the defects of which we were speaking?

% First of all, I said, consider the nature of the qualification. Just
% think what would happen if pilots were to be chosen according to their
% property, and a poor man were refused permission to steer, even though
% he were a better pilot?

% You mean that they would shipwreck?

% Yes; and is not this true of the government of anything?

% I should imagine so.

% Except a city?--or would you include a city?

% Nay, he said, the case of a city is the strongest of all, inasmuch as
% the rule of a city is the greatest and most difficult of all.

% This, then, will be the first great defect of oligarchy?

% Clearly.

% And here is another defect which is quite as bad.

% What defect?

% The inevitable division: such a State is not one, but two States, the
% one of poor, the other of rich men; and they are living on the same spot
% and always conspiring against one another.

% That, surely, is at least as bad.

% Another discreditable feature is, that, for a like reason, they are
% incapable of carrying on any war. Either they arm the multitude, and
% then they are more afraid of them than of the enemy; or, if they do not
% call them out in the hour of battle, they are oligarchs indeed, few to
% fight as they are few to rule. And at the same time their fondness for
% money makes them unwilling to pay taxes.

% How discreditable!

% And, as we said before, under such a constitution the same persons have
% too many callings--they are husbandmen, tradesmen, warriors, all in one.
% Does that look well?

% Anything but well.

% There is another evil which is, perhaps, the greatest of all, and to
% which this State first begins to be liable.

% What evil?

% A man may sell all that he has, and another may acquire his property;
% yet after the sale he may dwell in the city of which he is no longer a
% part, being neither trader, nor artisan, nor horseman, nor hoplite, but
% only a poor, helpless creature.

% Yes, that is an evil which also first begins in this State.

% The evil is certainly not prevented there; for oligarchies have both the
% extremes of great wealth and utter poverty.

% True.

% But think again: In his wealthy days, while he was spending his money,
% was a man of this sort a whit more good to the State for the purposes
% of citizenship? Or did he only seem to be a member of the ruling
% body, although in truth he was neither ruler nor subject, but just a
% spendthrift?

% As you say, he seemed to be a ruler, but was only a spendthrift.

% May we not say that this is the drone in the house who is like the drone
% in the honeycomb, and that the one is the plague of the city as the
% other is of the hive?

% Just so, Socrates.

% And God has made the flying drones, Adeimantus, all without stings,
% whereas of the walking drones he has made some without stings but others
% have dreadful stings; of the stingless class are those who in their old
% age end as paupers; of the stingers come all the criminal class, as they
% are termed.

% Most true, he said.

% Clearly then, whenever you see paupers in a State, somewhere in that
% neighborhood there are hidden away thieves, and cut-purses and robbers
% of temples, and all sorts of malefactors.

% Clearly.

% Well, I said, and in oligarchical States do you not find paupers?

% Yes, he said; nearly everybody is a pauper who is not a ruler.

% And may we be so bold as to affirm that there are also many criminals to
% be found in them, rogues who have stings, and whom the authorities are
% careful to restrain by force?

% Certainly, we may be so bold.

% The existence of such persons is to be attributed to want of education,
% ill-training, and an evil constitution of the State?

% True.

% Such, then, is the form and such are the evils of oligarchy; and there
% may be many other evils.

% Very likely.

% Then oligarchy, or the form of government in which the rulers are
% elected for their wealth, may now be dismissed. Let us next proceed to
% consider the nature and origin of the individual who answers to this
% State.

% By all means.

% Does not the timocratical man change into the oligarchical on this wise?

% How?

% A time arrives when the representative of timocracy has a son: at first
% he begins by emulating his father and walking in his footsteps, but
% presently he sees him of a sudden foundering against the State as upon
% a sunken reef, and he and all that he has is lost; he may have been
% a general or some other high officer who is brought to trial under a
% prejudice raised by informers, and either put to death, or exiled, or
% deprived of the privileges of a citizen, and all his property taken from
% him.

% Nothing more likely.

% And the son has seen and known all this--he is a ruined man, and his
% fear has taught him to knock ambition and passion headforemost from his
% bosom's throne; humbled by poverty he takes to money-making and by mean
% and miserly savings and hard work gets a fortune together. Is not such
% an one likely to seat the concupiscent and covetous element on the
% vacant throne and to suffer it to play the great king within him, girt
% with tiara and chain and scimitar?

% Most true, he replied.

% And when he has made reason and spirit sit down on the ground obediently
% on either side of their sovereign, and taught them to know their place,
% he compels the one to think only of how lesser sums may be turned into
% larger ones, and will not allow the other to worship and admire anything
% but riches and rich men, or to be ambitious of anything so much as the
% acquisition of wealth and the means of acquiring it.

% Of all changes, he said, there is none so speedy or so sure as the
% conversion of the ambitious youth into the avaricious one.

% And the avaricious, I said, is the oligarchical youth?

% Yes, he said; at any rate the individual out of whom he came is like the
% State out of which oligarchy came.

% Let us then consider whether there is any likeness between them.

% Very good.

% First, then, they resemble one another in the value which they set upon
% wealth?

% Certainly.

% Also in their penurious, laborious character; the individual only
% satisfies his necessary appetites, and confines his expenditure to them;
% his other desires he subdues, under the idea that they are unprofitable.

% True.

% He is a shabby fellow, who saves something out of everything and makes a
% purse for himself; and this is the sort of man whom the vulgar applaud.
% Is he not a true image of the State which he represents?

% He appears to me to be so; at any rate money is highly valued by him as
% well as by the State.

% You see that he is not a man of cultivation, I said.

% I imagine not, he said; had he been educated he would never have made a
% blind god director of his chorus, or given him chief honour.

% Excellent! I said. Yet consider: Must we not further admit that owing to
% this want of cultivation there will be found in him dronelike desires as
% of pauper and rogue, which are forcibly kept down by his general habit
% of life?

% True.

% Do you know where you will have to look if you want to discover his
% rogueries?

% Where must I look?

% You should see him where he has some great opportunity of acting
% dishonestly, as in the guardianship of an orphan.

% Aye.

% It will be clear enough then that in his ordinary dealings which give
% him a reputation for honesty he coerces his bad passions by an enforced
% virtue; not making them see that they are wrong, or taming them by
% reason, but by necessity and fear constraining them, and because he
% trembles for his possessions.

% To be sure.

% Yes, indeed, my dear friend, but you will find that the natural desires
% of the drone commonly exist in him all the same whenever he has to spend
% what is not his own.

% Yes, and they will be strong in him too.

% The man, then, will be at war with himself; he will be two men, and not
% one; but, in general, his better desires will be found to prevail over
% his inferior ones.

% True.

% For these reasons such an one will be more respectable than most people;
% yet the true virtue of a unanimous and harmonious soul will flee far
% away and never come near him.

% I should expect so.

% And surely, the miser individually will be an ignoble competitor in a
% State for any prize of victory, or other object of honourable ambition;
% he will not spend his money in the contest for glory; so afraid is he of
% awakening his expensive appetites and inviting them to help and join in
% the struggle; in true oligarchical fashion he fights with a small part
% only of his resources, and the result commonly is that he loses the
% prize and saves his money.

% Very true.

% Can we any longer doubt, then, that the miser and money-maker answers to
% the oligarchical State?

% There can be no doubt.

% Next comes democracy; of this the origin and nature have still to
% be considered by us; and then we will enquire into the ways of the
% democratic man, and bring him up for judgment.

% That, he said, is our method.

% Well, I said, and how does the change from oligarchy into democracy
% arise? Is it not on this wise?--The good at which such a State aims is
% to become as rich as possible, a desire which is insatiable?

% What then?

% The rulers, being aware that their power rests upon their wealth, refuse
% to curtail by law the extravagance of the spendthrift youth because
% they gain by their ruin; they take interest from them and buy up their
% estates and thus increase their own wealth and importance?

% To be sure.

% There can be no doubt that the love of wealth and the spirit of
% moderation cannot exist together in citizens of the same state to any
% considerable extent; one or the other will be disregarded.

% That is tolerably clear.

% And in oligarchical States, from the general spread of carelessness and
% extravagance, men of good family have often been reduced to beggary?

% Yes, often.

% And still they remain in the city; there they are, ready to sting and
% fully armed, and some of them owe money, some have forfeited their
% citizenship; a third class are in both predicaments; and they hate
% and conspire against those who have got their property, and against
% everybody else, and are eager for revolution.

% That is true.

% On the other hand, the men of business, stooping as they walk, and
% pretending not even to see those whom they have already ruined, insert
% their sting--that is, their money--into some one else who is not on
% his guard against them, and recover the parent sum many times over
% multiplied into a family of children: and so they make drone and pauper
% to abound in the State.

% Yes, he said, there are plenty of them--that is certain.

% The evil blazes up like a fire; and they will not extinguish it, either
% by restricting a man's use of his own property, or by another remedy:

% What other?

% One which is the next best, and has the advantage of compelling the
% citizens to look to their characters:--Let there be a general rule that
% every one shall enter into voluntary contracts at his own risk, and
% there will be less of this scandalous money-making, and the evils of
% which we were speaking will be greatly lessened in the State.

% Yes, they will be greatly lessened.

% At present the governors, induced by the motives which I have named,
% treat their subjects badly; while they and their adherents, especially
% the young men of the governing class, are habituated to lead a life
% of luxury and idleness both of body and mind; they do nothing, and are
% incapable of resisting either pleasure or pain.

% Very true.

% They themselves care only for making money, and are as indifferent as
% the pauper to the cultivation of virtue.

% Yes, quite as indifferent.

% Such is the state of affairs which prevails among them. And often rulers
% and their subjects may come in one another's way, whether on a journey
% or on some other occasion of meeting, on a pilgrimage or a march,
% as fellow-soldiers or fellow-sailors; aye and they may observe the
% behaviour of each other in the very moment of danger--for where danger
% is, there is no fear that the poor will be despised by the rich--and
% very likely the wiry sunburnt poor man may be placed in battle at the
% side of a wealthy one who has never spoilt his complexion and has
% plenty of superfluous flesh--when he sees such an one puffing and at his
% wits'-end, how can he avoid drawing the conclusion that men like him are
% only rich because no one has the courage to despoil them? And when they
% meet in private will not people be saying to one another ``Our warriors
% are not good for much'?

% Yes, he said, I am quite aware that this is their way of talking.

% And, as in a body which is diseased the addition of a touch from without
% may bring on illness, and sometimes even when there is no external
% provocation a commotion may arise within--in the same way wherever there
% is weakness in the State there is also likely to be illness, of which
% the occasion may be very slight, the one party introducing from without
% their oligarchical, the other their democratical allies, and then
% the State falls sick, and is at war with herself; and may be at times
% distracted, even when there is no external cause.

% Yes, surely.

% And then democracy comes into being after the poor have conquered their
% opponents, slaughtering some and banishing some, while to the remainder
% they give an equal share of freedom and power; and this is the form of
% government in which the magistrates are commonly elected by lot.

% Yes, he said, that is the nature of democracy, whether the revolution
% has been effected by arms, or whether fear has caused the opposite party
% to withdraw.

% And now what is their manner of life, and what sort of a government have
% they? for as the government is, such will be the man.

% Clearly, he said.

% In the first place, are they not free; and is not the city full of
% freedom and frankness--a man may say and do what he likes?

% ``Tis said so, he replied.

% And where freedom is, the individual is clearly able to order for
% himself his own life as he pleases?

% Clearly.

% Then in this kind of State there will be the greatest variety of human
% natures?

% There will.

% This, then, seems likely to be the fairest of States, being like an
% embroidered robe which is spangled with every sort of flower. And just
% as women and children think a variety of colours to be of all things
% most charming, so there are many men to whom this State, which is
% spangled with the manners and characters of mankind, will appear to be
% the fairest of States.

% Yes.

% Yes, my good Sir, and there will be no better in which to look for a
% government.

% Why?

% Because of the liberty which reigns there--they have a complete
% assortment of constitutions; and he who has a mind to establish a State,
% as we have been doing, must go to a democracy as he would to a bazaar at
% which they sell them, and pick out the one that suits him; then, when he
% has made his choice, he may found his State.

% He will be sure to have patterns enough.

% And there being no necessity, I said, for you to govern in this State,
% even if you have the capacity, or to be governed, unless you like, or
% go to war when the rest go to war, or to be at peace when others are
% at peace, unless you are so disposed--there being no necessity also,
% because some law forbids you to hold office or be a dicast, that you
% should not hold office or be a dicast, if you have a fancy--is not this
% a way of life which for the moment is supremely delightful?

% For the moment, yes.

% And is not their humanity to the condemned in some cases quite charming?
% Have you not observed how, in a democracy, many persons, although they
% have been sentenced to death or exile, just stay where they are and walk
% about the world--the gentleman parades like a hero, and nobody sees or
% cares?

% Yes, he replied, many and many a one.

% See too, I said, the forgiving spirit of democracy, and the ``don't
% care'' about trifles, and the disregard which she shows of all the fine
% principles which we solemnly laid down at the foundation of the city--as
% when we said that, except in the case of some rarely gifted nature,
% there never will be a good man who has not from his childhood been used
% to play amid things of beauty and make of them a joy and a study--how
% grandly does she trample all these fine notions of ours under her feet,
% never giving a thought to the pursuits which make a statesman, and
% promoting to honour any one who professes to be the people's friend.

% Yes, she is of a noble spirit.

% These and other kindred characteristics are proper to democracy, which
% is a charming form of government, full of variety and disorder, and
% dispensing a sort of equality to equals and unequals alike.

% We know her well.

% Consider now, I said, what manner of man the individual is, or rather
% consider, as in the case of the State, how he comes into being.

% Very good, he said.

% Is not this the way--he is the son of the miserly and oligarchical
% father who has trained him in his own habits?

% Exactly.

% And, like his father, he keeps under by force the pleasures which are of
% the spending and not of the getting sort, being those which are called
% unnecessary?

% Obviously.

% Would you like, for the sake of clearness, to distinguish which are the
% necessary and which are the unnecessary pleasures?

% I should.

% Are not necessary pleasures those of which we cannot get rid, and of
% which the satisfaction is a benefit to us? And they are rightly called
% so, because we are framed by nature to desire both what is beneficial
% and what is necessary, and cannot help it.

% True.

% We are not wrong therefore in calling them necessary?

% We are not.

% And the desires of which a man may get rid, if he takes pains from his
% youth upwards--of which the presence, moreover, does no good, and in
% some cases the reverse of good--shall we not be right in saying that all
% these are unnecessary?

% Yes, certainly.

% Suppose we select an example of either kind, in order that we may have a
% general notion of them?

% Very good.

% Will not the desire of eating, that is, of simple food and condiments,
% in so far as they are required for health and strength, be of the
% necessary class?

% That is what I should suppose.

% The pleasure of eating is necessary in two ways; it does us good and it
% is essential to the continuance of life?

% Yes.

% But the condiments are only necessary in so far as they are good for
% health?

% Certainly.

% And the desire which goes beyond this, of more delicate food, or other
% luxuries, which might generally be got rid of, if controlled and trained
% in youth, and is hurtful to the body, and hurtful to the soul in the
% pursuit of wisdom and virtue, may be rightly called unnecessary?

% Very true.

% May we not say that these desires spend, and that the others make money
% because they conduce to production?

% Certainly.

% And of the pleasures of love, and all other pleasures, the same holds
% good?

% True.

% And the drone of whom we spoke was he who was surfeited in pleasures
% and desires of this sort, and was the slave of the unnecessary desires,
% whereas he who was subject to the necessary only was miserly and
% oligarchical?

% Very true.

% Again, let us see how the democratical man grows out of the
% oligarchical: the following, as I suspect, is commonly the process.

% What is the process?

% When a young man who has been brought up as we were just now describing,
% in a vulgar and miserly way, has tasted drones'' honey and has come to
% associate with fierce and crafty natures who are able to provide for
% him all sorts of refinements and varieties of pleasure--then, as you may
% imagine, the change will begin of the oligarchical principle within him
% into the democratical?

% Inevitably.

% And as in the city like was helping like, and the change was effected by
% an alliance from without assisting one division of the citizens, so too
% the young man is changed by a class of desires coming from without
% to assist the desires within him, that which is akin and alike again
% helping that which is akin and alike?

% Certainly.

% And if there be any ally which aids the oligarchical principle within
% him, whether the influence of a father or of kindred, advising or
% rebuking him, then there arises in his soul a faction and an opposite
% faction, and he goes to war with himself.

% It must be so.

% And there are times when the democratical principle gives way to the
% oligarchical, and some of his desires die, and others are banished;
% a spirit of reverence enters into the young man's soul and order is
% restored.

% Yes, he said, that sometimes happens.

% And then, again, after the old desires have been driven out, fresh ones
% spring up, which are akin to them, and because he their father does not
% know how to educate them, wax fierce and numerous.

% Yes, he said, that is apt to be the way.

% They draw him to his old associates, and holding secret intercourse with
% them, breed and multiply in him.

% Very true.

% At length they seize upon the citadel of the young man's soul, which
% they perceive to be void of all accomplishments and fair pursuits and
% true words, which make their abode in the minds of men who are dear to
% the gods, and are their best guardians and sentinels.

% None better.

% False and boastful conceits and phrases mount upwards and take their
% place.

% They are certain to do so.

% And so the young man returns into the country of the lotus-eaters, and
% takes up his dwelling there in the face of all men; and if any help be
% sent by his friends to the oligarchical part of him, the aforesaid vain
% conceits shut the gate of the king's fastness; and they will neither
% allow the embassy itself to enter, nor if private advisers offer the
% fatherly counsel of the aged will they listen to them or receive them.
% There is a battle and they gain the day, and then modesty, which
% they call silliness, is ignominiously thrust into exile by them, and
% temperance, which they nickname unmanliness, is trampled in the mire and
% cast forth; they persuade men that moderation and orderly expenditure
% are vulgarity and meanness, and so, by the help of a rabble of evil
% appetites, they drive them beyond the border.

% Yes, with a will.

% And when they have emptied and swept clean the soul of him who is now in
% their power and who is being initiated by them in great mysteries, the
% next thing is to bring back to their house insolence and anarchy and
% waste and impudence in bright array having garlands on their heads, and
% a great company with them, hymning their praises and calling them by
% sweet names; insolence they term breeding, and anarchy liberty, and
% waste magnificence, and impudence courage. And so the young man
% passes out of his original nature, which was trained in the school of
% necessity, into the freedom and libertinism of useless and unnecessary
% pleasures.

% Yes, he said, the change in him is visible enough.

% After this he lives on, spending his money and labour and time on
% unnecessary pleasures quite as much as on necessary ones; but if he be
% fortunate, and is not too much disordered in his wits, when years have
% elapsed, and the heyday of passion is over--supposing that he then
% re-admits into the city some part of the exiled virtues, and does not
% wholly give himself up to their successors--in that case he balances his
% pleasures and lives in a sort of equilibrium, putting the government of
% himself into the hands of the one which comes first and wins the turn;
% and when he has had enough of that, then into the hands of another; he
% despises none of them but encourages them all equally.

% Very true, he said.

% Neither does he receive or let pass into the fortress any true word of
% advice; if any one says to him that some pleasures are the satisfactions
% of good and noble desires, and others of evil desires, and that he ought
% to use and honour some and chastise and master the others--whenever this
% is repeated to him he shakes his head and says that they are all alike,
% and that one is as good as another.

% Yes, he said; that is the way with him.

% Yes, I said, he lives from day to day indulging the appetite of the
% hour; and sometimes he is lapped in drink and strains of the flute; then
% he becomes a water-drinker, and tries to get thin; then he takes a turn
% at gymnastics; sometimes idling and neglecting everything, then once
% more living the life of a philosopher; often he is busy with politics,
% and starts to his feet and says and does whatever comes into his head;
% and, if he is emulous of any one who is a warrior, off he is in that
% direction, or of men of business, once more in that. His life has
% neither law nor order; and this distracted existence he terms joy and
% bliss and freedom; and so he goes on.

% Yes, he replied, he is all liberty and equality.

% Yes, I said; his life is motley and manifold and an epitome of the
% lives of many;--he answers to the State which we described as fair
% and spangled. And many a man and many a woman will take him for their
% pattern, and many a constitution and many an example of manners is
% contained in him.

% Just so.

% Let him then be set over against democracy; he may truly be called the
% democratic man.

% Let that be his place, he said.

% Last of all comes the most beautiful of all, man and State alike,
% tyranny and the tyrant; these we have now to consider.

% Quite true, he said.

% Say then, my friend, In what manner does tyranny arise?--that it has a
% democratic origin is evident.

% Clearly.

% And does not tyranny spring from democracy in the same manner as
% democracy from oligarchy--I mean, after a sort?

% How?

% The good which oligarchy proposed to itself and the means by which it
% was maintained was excess of wealth--am I not right?

% Yes.

% And the insatiable desire of wealth and the neglect of all other things
% for the sake of money-getting was also the ruin of oligarchy?

% True.

% And democracy has her own good, of which the insatiable desire brings
% her to dissolution?

% What good?

% Freedom, I replied; which, as they tell you in a democracy, is the glory
% of the State--and that therefore in a democracy alone will the freeman
% of nature deign to dwell.

% Yes; the saying is in every body's mouth.

% I was going to observe, that the insatiable desire of this and the
% neglect of other things introduces the change in democracy, which
% occasions a demand for tyranny.

% How so?

% When a democracy which is thirsting for freedom has evil cup-bearers
% presiding over the feast, and has drunk too deeply of the strong wine of
% freedom, then, unless her rulers are very amenable and give a plentiful
% draught, she calls them to account and punishes them, and says that they
% are cursed oligarchs.

% Yes, he replied, a very common occurrence.

% Yes, I said; and loyal citizens are insultingly termed by her slaves who
% hug their chains and men of naught; she would have subjects who are like
% rulers, and rulers who are like subjects: these are men after her own
% heart, whom she praises and honours both in private and public. Now, in
% such a State, can liberty have any limit?

% Certainly not.

% By degrees the anarchy finds a way into private houses, and ends by
% getting among the animals and infecting them.

% How do you mean?

% I mean that the father grows accustomed to descend to the level of his
% sons and to fear them, and the son is on a level with his father, he
% having no respect or reverence for either of his parents; and this is
% his freedom, and the metic is equal with the citizen and the citizen
% with the metic, and the stranger is quite as good as either.

% Yes, he said, that is the way.

% And these are not the only evils, I said--there are several lesser ones:
% In such a state of society the master fears and flatters his scholars,
% and the scholars despise their masters and tutors; young and old are
% all alike; and the young man is on a level with the old, and is ready
% to compete with him in word or deed; and old men condescend to the
% young and are full of pleasantry and gaiety; they are loth to be thought
% morose and authoritative, and therefore they adopt the manners of the
% young.

% Quite true, he said.

% The last extreme of popular liberty is when the slave bought with money,
% whether male or female, is just as free as his or her purchaser; nor
% must I forget to tell of the liberty and equality of the two sexes in
% relation to each other.

% Why not, as Aeschylus says, utter the word which rises to our lips?

% That is what I am doing, I replied; and I must add that no one who
% does not know would believe, how much greater is the liberty which the
% animals who are under the dominion of man have in a democracy than in
% any other State: for truly, the she-dogs, as the proverb says, are as
% good as their she-mistresses, and the horses and asses have a way of
% marching along with all the rights and dignities of freemen; and they
% will run at any body who comes in their way if he does not leave
% the road clear for them: and all things are just ready to burst with
% liberty.

% When I take a country walk, he said, I often experience what you
% describe. You and I have dreamed the same thing.

% And above all, I said, and as the result of all, see how sensitive the
% citizens become; they chafe impatiently at the least touch of authority,
% and at length, as you know, they cease to care even for the laws,
% written or unwritten; they will have no one over them.

% Yes, he said, I know it too well.

% Such, my friend, I said, is the fair and glorious beginning out of which
% springs tyranny.

% Glorious indeed, he said. But what is the next step?

% The ruin of oligarchy is the ruin of democracy; the same disease
% magnified and intensified by liberty overmasters democracy--the truth
% being that the excessive increase of anything often causes a reaction in
% the opposite direction; and this is the case not only in the seasons and
% in vegetable and animal life, but above all in forms of government.

% True.

% The excess of liberty, whether in States or individuals, seems only to
% pass into excess of slavery.

% Yes, the natural order.

% And so tyranny naturally arises out of democracy, and the most
% aggravated form of tyranny and slavery out of the most extreme form of
% liberty?

% As we might expect.

% That, however, was not, as I believe, your question--you rather desired
% to know what is that disorder which is generated alike in oligarchy and
% democracy, and is the ruin of both?

% Just so, he replied.

% Well, I said, I meant to refer to the class of idle spendthrifts,
% of whom the more courageous are the leaders and the more timid the
% followers, the same whom we were comparing to drones, some stingless,
% and others having stings.

% A very just comparison.

% These two classes are the plagues of every city in which they are
% generated, being what phlegm and bile are to the body. And the good
% physician and lawgiver of the State ought, like the wise bee-master, to
% keep them at a distance and prevent, if possible, their ever coming in;
% and if they have anyhow found a way in, then he should have them and
% their cells cut out as speedily as possible.

% Yes, by all means, he said.

% Then, in order that we may see clearly what we are doing, let us imagine
% democracy to be divided, as indeed it is, into three classes; for in the
% first place freedom creates rather more drones in the democratic than
% there were in the oligarchical State.

% That is true.

% And in the democracy they are certainly more intensified.

% How so?

% Because in the oligarchical State they are disqualified and driven from
% office, and therefore they cannot train or gather strength; whereas in a
% democracy they are almost the entire ruling power, and while the keener
% sort speak and act, the rest keep buzzing about the bema and do not
% suffer a word to be said on the other side; hence in democracies almost
% everything is managed by the drones.

% Very true, he said.

% Then there is another class which is always being severed from the mass.

% What is that?

% They are the orderly class, which in a nation of traders is sure to be
% the richest.

% Naturally so.

% They are the most squeezable persons and yield the largest amount of
% honey to the drones.

% Why, he said, there is little to be squeezed out of people who have
% little.

% And this is called the wealthy class, and the drones feed upon them.

% That is pretty much the case, he said.

% The people are a third class, consisting of those who work with their
% own hands; they are not politicians, and have not much to live upon.
% This, when assembled, is the largest and most powerful class in a
% democracy.

% True, he said; but then the multitude is seldom willing to congregate
% unless they get a little honey.

% And do they not share? I said. Do not their leaders deprive the rich
% of their estates and distribute them among the people; at the same time
% taking care to reserve the larger part for themselves?

% Why, yes, he said, to that extent the people do share.

% And the persons whose property is taken from them are compelled to
% defend themselves before the people as they best can?

% What else can they do?

% And then, although they may have no desire of change, the others charge
% them with plotting against the people and being friends of oligarchy?

% True.

% And the end is that when they see the people, not of their own accord,
% but through ignorance, and because they are deceived by informers,
% seeking to do them wrong, then at last they are forced to become
% oligarchs in reality; they do not wish to be, but the sting of the
% drones torments them and breeds revolution in them.

% That is exactly the truth.

% Then come impeachments and judgments and trials of one another.

% True.

% The people have always some champion whom they set over them and nurse
% into greatness.

% Yes, that is their way.

% This and no other is the root from which a tyrant springs; when he first
% appears above ground he is a protector.

% Yes, that is quite clear.

% How then does a protector begin to change into a tyrant? Clearly when
% he does what the man is said to do in the tale of the Arcadian temple of
% Lycaean Zeus.

% What tale?

% The tale is that he who has tasted the entrails of a single human victim
% minced up with the entrails of other victims is destined to become a
% wolf. Did you never hear it?

% Oh, yes.

% And the protector of the people is like him; having a mob entirely at
% his disposal, he is not restrained from shedding the blood of kinsmen;
% by the favourite method of false accusation he brings them into court
% and murders them, making the life of man to disappear, and with unholy
% tongue and lips tasting the blood of his fellow citizens; some he kills
% and others he banishes, at the same time hinting at the abolition of
% debts and partition of lands: and after this, what will be his destiny?
% Must he not either perish at the hands of his enemies, or from being a
% man become a wolf--that is, a tyrant?

% Inevitably.

% This, I said, is he who begins to make a party against the rich?

% The same.

% After a while he is driven out, but comes back, in spite of his enemies,
% a tyrant full grown.

% That is clear.

% And if they are unable to expel him, or to get him condemned to death by
% a public accusation, they conspire to assassinate him.

% Yes, he said, that is their usual way.

% Then comes the famous request for a body-guard, which is the device of
% all those who have got thus far in their tyrannical career--'Let not the
% people's friend,'' as they say, ``be lost to them.''

% Exactly.

% The people readily assent; all their fears are for him--they have none
% for themselves.

% Very true.

% And when a man who is wealthy and is also accused of being an enemy of
% the people sees this, then, my friend, as the oracle said to Croesus,

% ``By pebbly Hermus'' shore he flees and rests not, and is not ashamed to
% be a coward.''

% And quite right too, said he, for if he were, he would never be ashamed
% again.

% But if he is caught he dies.

% Of course.

% And he, the protector of whom we spoke, is to be seen, not ``larding the
% plain'' with his bulk, but himself the overthrower of many, standing up
% in the chariot of State with the reins in his hand, no longer protector,
% but tyrant absolute.

% No doubt, he said.

% And now let us consider the happiness of the man, and also of the State
% in which a creature like him is generated.

% Yes, he said, let us consider that.

% At first, in the early days of his power, he is full of smiles, and
% he salutes every one whom he meets;--he to be called a tyrant, who is
% making promises in public and also in private! liberating debtors, and
% distributing land to the people and his followers, and wanting to be so
% kind and good to every one!

% Of course, he said.

% But when he has disposed of foreign enemies by conquest or treaty, and
% there is nothing to fear from them, then he is always stirring up some
% war or other, in order that the people may require a leader.

% To be sure.

% Has he not also another object, which is that they may be impoverished
% by payment of taxes, and thus compelled to devote themselves to their
% daily wants and therefore less likely to conspire against him?

% Clearly.

% And if any of them are suspected by him of having notions of freedom,
% and of resistance to his authority, he will have a good pretext for
% destroying them by placing them at the mercy of the enemy; and for all
% these reasons the tyrant must be always getting up a war.

% He must.

% Now he begins to grow unpopular.

% A necessary result.

% Then some of those who joined in setting him up, and who are in power,
% speak their minds to him and to one another, and the more courageous of
% them cast in his teeth what is being done.

% Yes, that may be expected.

% And the tyrant, if he means to rule, must get rid of them; he cannot
% stop while he has a friend or an enemy who is good for anything.

% He cannot.

% And therefore he must look about him and see who is valiant, who is
% high-minded, who is wise, who is wealthy; happy man, he is the enemy
% of them all, and must seek occasion against them whether he will or no,
% until he has made a purgation of the State.

% Yes, he said, and a rare purgation.

% Yes, I said, not the sort of purgation which the physicians make of the
% body; for they take away the worse and leave the better part, but he
% does the reverse.

% If he is to rule, I suppose that he cannot help himself.

% What a blessed alternative, I said:--to be compelled to dwell only with
% the many bad, and to be by them hated, or not to live at all!

% Yes, that is the alternative.

% And the more detestable his actions are to the citizens the more
% satellites and the greater devotion in them will he require?

% Certainly.

% And who are the devoted band, and where will he procure them?

% They will flock to him, he said, of their own accord, if he pays them.

% By the dog! I said, here are more drones, of every sort and from every
% land.

% Yes, he said, there are.

% But will he not desire to get them on the spot?

% How do you mean?

% He will rob the citizens of their slaves; he will then set them free and
% enrol them in his body-guard.

% To be sure, he said; and he will be able to trust them best of all.

% What a blessed creature, I said, must this tyrant be; he has put to
% death the others and has these for his trusted friends.

% Yes, he said; they are quite of his sort.

% Yes, I said, and these are the new citizens whom he has called into
% existence, who admire him and are his companions, while the good hate
% and avoid him.

% Of course.

% Verily, then, tragedy is a wise thing and Euripides a great tragedian.

% Why so?

% Why, because he is the author of the pregnant saying,

% ``Tyrants are wise by living with the wise;''

% and he clearly meant to say that they are the wise whom the tyrant makes
% his companions.

% Yes, he said, and he also praises tyranny as godlike; and many other
% things of the same kind are said by him and by the other poets.

% And therefore, I said, the tragic poets being wise men will forgive us
% and any others who live after our manner if we do not receive them into
% our State, because they are the eulogists of tyranny.

% Yes, he said, those who have the wit will doubtless forgive us.

% But they will continue to go to other cities and attract mobs, and
% hire voices fair and loud and persuasive, and draw the cities over to
% tyrannies and democracies.

% Very true.

% Moreover, they are paid for this and receive honour--the greatest
% honour, as might be expected, from tyrants, and the next greatest from
% democracies; but the higher they ascend our constitution hill, the more
% their reputation fails, and seems unable from shortness of breath to
% proceed further.

% True.

% But we are wandering from the subject: Let us therefore return and
% enquire how the tyrant will maintain that fair and numerous and various
% and ever-changing army of his.

% If, he said, there are sacred treasures in the city, he will confiscate
% and spend them; and in so far as the fortunes of attainted persons may
% suffice, he will be able to diminish the taxes which he would otherwise
% have to impose upon the people.

% And when these fail?

% Why, clearly, he said, then he and his boon companions, whether male or
% female, will be maintained out of his father's estate.

% You mean to say that the people, from whom he has derived his being,
% will maintain him and his companions?

% Yes, he said; they cannot help themselves.

% But what if the people fly into a passion, and aver that a grown-up son
% ought not to be supported by his father, but that the father should be
% supported by the son? The father did not bring him into being, or settle
% him in life, in order that when his son became a man he should himself
% be the servant of his own servants and should support him and his rabble
% of slaves and companions; but that his son should protect him, and that
% by his help he might be emancipated from the government of the rich and
% aristocratic, as they are termed. And so he bids him and his companions
% depart, just as any other father might drive out of the house a riotous
% son and his undesirable associates.

% By heaven, he said, then the parent will discover what a monster he has
% been fostering in his bosom; and, when he wants to drive him out, he
% will find that he is weak and his son strong.

% Why, you do not mean to say that the tyrant will use violence? What!
% beat his father if he opposes him?

% Yes, he will, having first disarmed him.

% Then he is a parricide, and a cruel guardian of an aged parent; and this
% is real tyranny, about which there can be no longer a mistake: as the
% saying is, the people who would escape the smoke which is the slavery of
% freemen, has fallen into the fire which is the tyranny of slaves. Thus
% liberty, getting out of all order and reason, passes into the harshest
% and bitterest form of slavery.

% True, he said.

% Very well; and may we not rightly say that we have sufficiently
% discussed the nature of tyranny, and the manner of the transition from
% democracy to tyranny?

% Yes, quite enough, he said.




% BOOK IX.

% Last of all comes the tyrannical man; about whom we have once more to
% ask, how is he formed out of the democratical? and how does he live, in
% happiness or in misery?

% Yes, he said, he is the only one remaining.

% There is, however, I said, a previous question which remains unanswered.

% What question?

% I do not think that we have adequately determined the nature and number
% of the appetites, and until this is accomplished the enquiry will always
% be confused.

% Well, he said, it is not too late to supply the omission.

% Very true, I said; and observe the point which I want to understand:
% Certain of the unnecessary pleasures and appetites I conceive to be
% unlawful; every one appears to have them, but in some persons they are
% controlled by the laws and by reason, and the better desires prevail
% over them--either they are wholly banished or they become few and weak;
% while in the case of others they are stronger, and there are more of
% them.

% Which appetites do you mean?

% I mean those which are awake when the reasoning and human and ruling
% power is asleep; then the wild beast within us, gorged with meat or
% drink, starts up and having shaken off sleep, goes forth to satisfy
% his desires; and there is no conceivable folly or crime--not excepting
% incest or any other unnatural union, or parricide, or the eating of
% forbidden food--which at such a time, when he has parted company with
% all shame and sense, a man may not be ready to commit.

% Most true, he said.

% But when a man's pulse is healthy and temperate, and when before going
% to sleep he has awakened his rational powers, and fed them on noble
% thoughts and enquiries, collecting himself in meditation; after having
% first indulged his appetites neither too much nor too little, but just
% enough to lay them to sleep, and prevent them and their enjoyments and
% pains from interfering with the higher principle--which he leaves in
% the solitude of pure abstraction, free to contemplate and aspire to
% the knowledge of the unknown, whether in past, present, or future: when
% again he has allayed the passionate element, if he has a quarrel against
% any one--I say, when, after pacifying the two irrational principles, he
% rouses up the third, which is reason, before he takes his rest, then,
% as you know, he attains truth most nearly, and is least likely to be the
% sport of fantastic and lawless visions.

% I quite agree.

% In saying this I have been running into a digression; but the point
% which I desire to note is that in all of us, even in good men, there is
% a lawless wild-beast nature, which peers out in sleep. Pray, consider
% whether I am right, and you agree with me.

% Yes, I agree.

% And now remember the character which we attributed to the democratic
% man. He was supposed from his youth upwards to have been trained under
% a miserly parent, who encouraged the saving appetites in him, but
% discountenanced the unnecessary, which aim only at amusement and
% ornament?

% True.

% And then he got into the company of a more refined, licentious sort of
% people, and taking to all their wanton ways rushed into the opposite
% extreme from an abhorrence of his father's meanness. At last, being a
% better man than his corruptors, he was drawn in both directions until he
% halted midway and led a life, not of vulgar and slavish passion, but
% of what he deemed moderate indulgence in various pleasures. After this
% manner the democrat was generated out of the oligarch?

% Yes, he said; that was our view of him, and is so still.

% And now, I said, years will have passed away, and you must conceive this
% man, such as he is, to have a son, who is brought up in his father's
% principles.

% I can imagine him.

% Then you must further imagine the same thing to happen to the son
% which has already happened to the father:--he is drawn into a perfectly
% lawless life, which by his seducers is termed perfect liberty; and his
% father and friends take part with his moderate desires, and the opposite
% party assist the opposite ones. As soon as these dire magicians and
% tyrant-makers find that they are losing their hold on him, they contrive
% to implant in him a master passion, to be lord over his idle and
% spendthrift lusts--a sort of monstrous winged drone--that is the only
% image which will adequately describe him.

% Yes, he said, that is the only adequate image of him.

% And when his other lusts, amid clouds of incense and perfumes and
% garlands and wines, and all the pleasures of a dissolute life, now let
% loose, come buzzing around him, nourishing to the utmost the sting of
% desire which they implant in his drone-like nature, then at last this
% lord of the soul, having Madness for the captain of his guard, breaks
% out into a frenzy: and if he finds in himself any good opinions or
% appetites in process of formation, and there is in him any sense of
% shame remaining, to these better principles he puts an end, and casts
% them forth until he has purged away temperance and brought in madness to
% the full.

% Yes, he said, that is the way in which the tyrannical man is generated.

% And is not this the reason why of old love has been called a tyrant?

% I should not wonder.

% Further, I said, has not a drunken man also the spirit of a tyrant?

% He has.

% And you know that a man who is deranged and not right in his mind, will
% fancy that he is able to rule, not only over men, but also over the
% gods?

% That he will.

% And the tyrannical man in the true sense of the word comes into being
% when, either under the influence of nature, or habit, or both, he
% becomes drunken, lustful, passionate? O my friend, is not that so?

% Assuredly.

% Such is the man and such is his origin. And next, how does he live?

% Suppose, as people facetiously say, you were to tell me.

% I imagine, I said, at the next step in his progress, that there will be
% feasts and carousals and revellings and courtezans, and all that sort
% of thing; Love is the lord of the house within him, and orders all the
% concerns of his soul.

% That is certain.

% Yes; and every day and every night desires grow up many and formidable,
% and their demands are many.

% They are indeed, he said.

% His revenues, if he has any, are soon spent.

% True.

% Then comes debt and the cutting down of his property.

% Of course.

% When he has nothing left, must not his desires, crowding in the nest
% like young ravens, be crying aloud for food; and he, goaded on by them,
% and especially by love himself, who is in a manner the captain of them,
% is in a frenzy, and would fain discover whom he can defraud or despoil
% of his property, in order that he may gratify them?

% Yes, that is sure to be the case.

% He must have money, no matter how, if he is to escape horrid pains and
% pangs.

% He must.

% And as in himself there was a succession of pleasures, and the new got
% the better of the old and took away their rights, so he being younger
% will claim to have more than his father and his mother, and if he has
% spent his own share of the property, he will take a slice of theirs.

% No doubt he will.

% And if his parents will not give way, then he will try first of all to
% cheat and deceive them.

% Very true.

% And if he fails, then he will use force and plunder them.

% Yes, probably.

% And if the old man and woman fight for their own, what then, my friend?
% Will the creature feel any compunction at tyrannizing over them?

% Nay, he said, I should not feel at all comfortable about his parents.

% But, O heavens! Adeimantus, on account of some new-fangled love of a
% harlot, who is anything but a necessary connection, can you believe that
% he would strike the mother who is his ancient friend and necessary
% to his very existence, and would place her under the authority of the
% other, when she is brought under the same roof with her; or that, under
% like circumstances, he would do the same to his withered old father,
% first and most indispensable of friends, for the sake of some
% newly-found blooming youth who is the reverse of indispensable?

% Yes, indeed, he said; I believe that he would.

% Truly, then, I said, a tyrannical son is a blessing to his father and
% mother.

% He is indeed, he replied.

% He first takes their property, and when that fails, and pleasures are
% beginning to swarm in the hive of his soul, then he breaks into a house,
% or steals the garments of some nightly wayfarer; next he proceeds to
% clear a temple. Meanwhile the old opinions which he had when a child,
% and which gave judgment about good and evil, are overthrown by those
% others which have just been emancipated, and are now the body-guard of
% love and share his empire. These in his democratic days, when he was
% still subject to the laws and to his father, were only let loose in
% the dreams of sleep. But now that he is under the dominion of love, he
% becomes always and in waking reality what he was then very rarely and in
% a dream only; he will commit the foulest murder, or eat forbidden food,
% or be guilty of any other horrid act. Love is his tyrant, and lives
% lordly in him and lawlessly, and being himself a king, leads him on, as
% a tyrant leads a State, to the performance of any reckless deed by which
% he can maintain himself and the rabble of his associates, whether those
% whom evil communications have brought in from without, or those whom
% he himself has allowed to break loose within him by reason of a similar
% evil nature in himself. Have we not here a picture of his way of life?

% Yes, indeed, he said.

% And if there are only a few of them in the State, and the rest of the
% people are well disposed, they go away and become the body-guard or
% mercenary soldiers of some other tyrant who may probably want them for a
% war; and if there is no war, they stay at home and do many little pieces
% of mischief in the city.

% What sort of mischief?

% For example, they are the thieves, burglars, cut-purses, foot-pads,
% robbers of temples, man-stealers of the community; or if they are able
% to speak they turn informers, and bear false witness, and take bribes.

% A small catalogue of evils, even if the perpetrators of them are few in
% number.

% Yes, I said; but small and great are comparative terms, and all these
% things, in the misery and evil which they inflict upon a State, do not
% come within a thousand miles of the tyrant; when this noxious class and
% their followers grow numerous and become conscious of their strength,
% assisted by the infatuation of the people, they choose from among
% themselves the one who has most of the tyrant in his own soul, and him
% they create their tyrant.

% Yes, he said, and he will be the most fit to be a tyrant.

% If the people yield, well and good; but if they resist him, as he began
% by beating his own father and mother, so now, if he has the power, he
% beats them, and will keep his dear old fatherland or motherland, as the
% Cretans say, in subjection to his young retainers whom he has introduced
% to be their rulers and masters. This is the end of his passions and
% desires.

% Exactly.

% When such men are only private individuals and before they get power,
% this is their character; they associate entirely with their own
% flatterers or ready tools; or if they want anything from anybody, they
% in their turn are equally ready to bow down before them: they profess
% every sort of affection for them; but when they have gained their point
% they know them no more.

% Yes, truly.

% They are always either the masters or servants and never the friends of
% anybody; the tyrant never tastes of true freedom or friendship.

% Certainly not.

% And may we not rightly call such men treacherous?

% No question.

% Also they are utterly unjust, if we were right in our notion of justice?

% Yes, he said, and we were perfectly right.

% Let us then sum up in a word, I said, the character of the worst man: he
% is the waking reality of what we dreamed.

% Most true.

% And this is he who being by nature most of a tyrant bears rule, and the
% longer he lives the more of a tyrant he becomes.

% That is certain, said Glaucon, taking his turn to answer.

% And will not he who has been shown to be the wickedest, be also the most
% miserable? and he who has tyrannized longest and most, most continually
% and truly miserable; although this may not be the opinion of men in
% general?

% Yes, he said, inevitably.

% And must not the tyrannical man be like the tyrannical State, and
% the democratical man like the democratical State; and the same of the
% others?

% Certainly.

% And as State is to State in virtue and happiness, so is man in relation
% to man?

% To be sure.

% Then comparing our original city, which was under a king, and the city
% which is under a tyrant, how do they stand as to virtue?

% They are the opposite extremes, he said, for one is the very best and
% the other is the very worst.

% There can be no mistake, I said, as to which is which, and therefore
% I will at once enquire whether you would arrive at a similar decision
% about their relative happiness and misery. And here we must not allow
% ourselves to be panic-stricken at the apparition of the tyrant, who is
% only a unit and may perhaps have a few retainers about him; but let us
% go as we ought into every corner of the city and look all about, and
% then we will give our opinion.

% A fair invitation, he replied; and I see, as every one must, that a
% tyranny is the wretchedest form of government, and the rule of a king
% the happiest.

% And in estimating the men too, may I not fairly make a like request,
% that I should have a judge whose mind can enter into and see through
% human nature? he must not be like a child who looks at the outside and
% is dazzled at the pompous aspect which the tyrannical nature assumes to
% the beholder, but let him be one who has a clear insight. May I suppose
% that the judgment is given in the hearing of us all by one who is able
% to judge, and has dwelt in the same place with him, and been present at
% his dally life and known him in his family relations, where he may be
% seen stripped of his tragedy attire, and again in the hour of public
% danger--he shall tell us about the happiness and misery of the tyrant
% when compared with other men?

% That again, he said, is a very fair proposal.

% Shall I assume that we ourselves are able and experienced judges and
% have before now met with such a person? We shall then have some one who
% will answer our enquiries.

% By all means.

% Let me ask you not to forget the parallel of the individual and the
% State; bearing this in mind, and glancing in turn from one to the other
% of them, will you tell me their respective conditions?

% What do you mean? he asked.

% Beginning with the State, I replied, would you say that a city which is
% governed by a tyrant is free or enslaved?

% No city, he said, can be more completely enslaved.

% And yet, as you see, there are freemen as well as masters in such a
% State?

% Yes, he said, I see that there are--a few; but the people, speaking
% generally, and the best of them are miserably degraded and enslaved.

% Then if the man is like the State, I said, must not the same rule
% prevail? his soul is full of meanness and vulgarity--the best elements
% in him are enslaved; and there is a small ruling part, which is also the
% worst and maddest.

% Inevitably.

% And would you say that the soul of such an one is the soul of a freeman,
% or of a slave?

% He has the soul of a slave, in my opinion.

% And the State which is enslaved under a tyrant is utterly incapable of
% acting voluntarily?

% Utterly incapable.

% And also the soul which is under a tyrant (I am speaking of the soul
% taken as a whole) is least capable of doing what she desires; there is a
% gadfly which goads her, and she is full of trouble and remorse?

% Certainly.

% And is the city which is under a tyrant rich or poor?

% Poor.

% And the tyrannical soul must be always poor and insatiable?

% True.

% And must not such a State and such a man be always full of fear?

% Yes, indeed.

% Is there any State in which you will find more of lamentation and sorrow
% and groaning and pain?

% Certainly not.

% And is there any man in whom you will find more of this sort of misery
% than in the tyrannical man, who is in a fury of passions and desires?

% Impossible.

% Reflecting upon these and similar evils, you held the tyrannical State
% to be the most miserable of States?

% And I was right, he said.

% Certainly, I said. And when you see the same evils in the tyrannical
% man, what do you say of him?

% I say that he is by far the most miserable of all men.

% There, I said, I think that you are beginning to go wrong.

% What do you mean?

% I do not think that he has as yet reached the utmost extreme of misery.

% Then who is more miserable?

% One of whom I am about to speak.

% Who is that?

% He who is of a tyrannical nature, and instead of leading a private life
% has been cursed with the further misfortune of being a public tyrant.

% From what has been said, I gather that you are right.

% Yes, I replied, but in this high argument you should be a little more
% certain, and should not conjecture only; for of all questions, this
% respecting good and evil is the greatest.

% Very true, he said.

% Let me then offer you an illustration, which may, I think, throw a light
% upon this subject.

% What is your illustration?

% The case of rich individuals in cities who possess many slaves: from
% them you may form an idea of the tyrant's condition, for they both have
% slaves; the only difference is that he has more slaves.

% Yes, that is the difference.

% You know that they live securely and have nothing to apprehend from
% their servants?

% What should they fear?

% Nothing. But do you observe the reason of this?

% Yes; the reason is, that the whole city is leagued together for the
% protection of each individual.

% Very true, I said. But imagine one of these owners, the master say of
% some fifty slaves, together with his family and property and slaves,
% carried off by a god into the wilderness, where there are no freemen to
% help him--will he not be in an agony of fear lest he and his wife and
% children should be put to death by his slaves?

% Yes, he said, he will be in the utmost fear.

% The time has arrived when he will be compelled to flatter divers of his
% slaves, and make many promises to them of freedom and other things, much
% against his will--he will have to cajole his own servants.

% Yes, he said, that will be the only way of saving himself.

% And suppose the same god, who carried him away, to surround him with
% neighbours who will not suffer one man to be the master of another, and
% who, if they could catch the offender, would take his life?

% His case will be still worse, if you suppose him to be everywhere
% surrounded and watched by enemies.

% And is not this the sort of prison in which the tyrant will be bound--he
% who being by nature such as we have described, is full of all sorts of
% fears and lusts? His soul is dainty and greedy, and yet alone, of all
% men in the city, he is never allowed to go on a journey, or to see the
% things which other freemen desire to see, but he lives in his hole like
% a woman hidden in the house, and is jealous of any other citizen who
% goes into foreign parts and sees anything of interest.

% Very true, he said.

% And amid evils such as these will not he who is ill-governed in his own
% person--the tyrannical man, I mean--whom you just now decided to be the
% most miserable of all--will not he be yet more miserable when, instead
% of leading a private life, he is constrained by fortune to be a public
% tyrant? He has to be master of others when he is not master of himself:
% he is like a diseased or paralytic man who is compelled to pass his
% life, not in retirement, but fighting and combating with other men.

% Yes, he said, the similitude is most exact.

% Is not his case utterly miserable? and does not the actual tyrant lead a
% worse life than he whose life you determined to be the worst?

% Certainly.

% He who is the real tyrant, whatever men may think, is the real slave,
% and is obliged to practise the greatest adulation and servility, and to
% be the flatterer of the vilest of mankind. He has desires which he is
% utterly unable to satisfy, and has more wants than any one, and is truly
% poor, if you know how to inspect the whole soul of him: all his life
% long he is beset with fear and is full of convulsions and distractions,
% even as the State which he resembles: and surely the resemblance holds?

% Very true, he said.

% Moreover, as we were saying before, he grows worse from having power: he
% becomes and is of necessity more jealous, more faithless, more unjust,
% more friendless, more impious, than he was at first; he is the purveyor
% and cherisher of every sort of vice, and the consequence is that he is
% supremely miserable, and that he makes everybody else as miserable as
% himself.

% No man of any sense will dispute your words.

% Come then, I said, and as the general umpire in theatrical contests
% proclaims the result, do you also decide who in your opinion is first
% in the scale of happiness, and who second, and in what order the others
% follow: there are five of them in all--they are the royal, timocratical,
% oligarchical, democratical, tyrannical.

% The decision will be easily given, he replied; they shall be choruses
% coming on the stage, and I must judge them in the order in which they
% enter, by the criterion of virtue and vice, happiness and misery.

% Need we hire a herald, or shall I announce, that the son of Ariston (the
% best) has decided that the best and justest is also the happiest, and
% that this is he who is the most royal man and king over himself; and
% that the worst and most unjust man is also the most miserable, and that
% this is he who being the greatest tyrant of himself is also the greatest
% tyrant of his State?

% Make the proclamation yourself, he said.

% And shall I add, ``whether seen or unseen by gods and men'?

% Let the words be added.

% Then this, I said, will be our first proof; and there is another, which
% may also have some weight.

% What is that?

% The second proof is derived from the nature of the soul: seeing that
% the individual soul, like the State, has been divided by us into three
% principles, the division may, I think, furnish a new demonstration.

% Of what nature?

% It seems to me that to these three principles three pleasures
% correspond; also three desires and governing powers.

% How do you mean? he said.

% There is one principle with which, as we were saying, a man learns,
% another with which he is angry; the third, having many forms, has no
% special name, but is denoted by the general term appetitive, from
% the extraordinary strength and vehemence of the desires of eating and
% drinking and the other sensual appetites which are the main elements of
% it; also money-loving, because such desires are generally satisfied by
% the help of money.

% That is true, he said.

% If we were to say that the loves and pleasures of this third part were
% concerned with gain, we should then be able to fall back on a single
% notion; and might truly and intelligibly describe this part of the soul
% as loving gain or money.

% I agree with you.

% Again, is not the passionate element wholly set on ruling and conquering
% and getting fame?

% True.

% Suppose we call it the contentious or ambitious--would the term be
% suitable?

% Extremely suitable.

% On the other hand, every one sees that the principle of knowledge is
% wholly directed to the truth, and cares less than either of the others
% for gain or fame.

% Far less.

% ``Lover of wisdom,'' ``lover of knowledge,'' are titles which we may fitly
% apply to that part of the soul?

% Certainly.

% One principle prevails in the souls of one class of men, another in
% others, as may happen?

% Yes.

% Then we may begin by assuming that there are three classes of
% men--lovers of wisdom, lovers of honour, lovers of gain?

% Exactly.

% And there are three kinds of pleasure, which are their several objects?

% Very true.

% Now, if you examine the three classes of men, and ask of them in turn
% which of their lives is pleasantest, each will be found praising his
% own and depreciating that of others: the money-maker will contrast the
% vanity of honour or of learning if they bring no money with the solid
% advantages of gold and silver?

% True, he said.

% And the lover of honour--what will be his opinion? Will he not think
% that the pleasure of riches is vulgar, while the pleasure of learning,
% if it brings no distinction, is all smoke and nonsense to him?

% Very true.

% And are we to suppose, I said, that the philosopher sets any value on
% other pleasures in comparison with the pleasure of knowing the truth,
% and in that pursuit abiding, ever learning, not so far indeed from the
% heaven of pleasure? Does he not call the other pleasures necessary,
% under the idea that if there were no necessity for them, he would rather
% not have them?

% There can be no doubt of that, he replied.

% Since, then, the pleasures of each class and the life of each are in
% dispute, and the question is not which life is more or less honourable,
% or better or worse, but which is the more pleasant or painless--how
% shall we know who speaks truly?

% I cannot myself tell, he said.

% Well, but what ought to be the criterion? Is any better than experience
% and wisdom and reason?

% There cannot be a better, he said.

% Then, I said, reflect. Of the three individuals, which has the greatest
% experience of all the pleasures which we enumerated? Has the lover of
% gain, in learning the nature of essential truth, greater experience of
% the pleasure of knowledge than the philosopher has of the pleasure of
% gain?

% The philosopher, he replied, has greatly the advantage; for he has
% of necessity always known the taste of the other pleasures from his
% childhood upwards: but the lover of gain in all his experience has not
% of necessity tasted--or, I should rather say, even had he desired, could
% hardly have tasted--the sweetness of learning and knowing truth.

% Then the lover of wisdom has a great advantage over the lover of gain,
% for he has a double experience?

% Yes, very great.

% Again, has he greater experience of the pleasures of honour, or the
% lover of honour of the pleasures of wisdom?

% Nay, he said, all three are honoured in proportion as they attain their
% object; for the rich man and the brave man and the wise man alike have
% their crowd of admirers, and as they all receive honour they all have
% experience of the pleasures of honour; but the delight which is to be
% found in the knowledge of true being is known to the philosopher only.

% His experience, then, will enable him to judge better than any one?

% Far better.

% And he is the only one who has wisdom as well as experience?

% Certainly.

% Further, the very faculty which is the instrument of judgment is not
% possessed by the covetous or ambitious man, but only by the philosopher?

% What faculty?

% Reason, with whom, as we were saying, the decision ought to rest.

% Yes.

% And reasoning is peculiarly his instrument?

% Certainly.

% If wealth and gain were the criterion, then the praise or blame of the
% lover of gain would surely be the most trustworthy?

% Assuredly.

% Or if honour or victory or courage, in that case the judgment of the
% ambitious or pugnacious would be the truest?

% Clearly.

% But since experience and wisdom and reason are the judges--

% The only inference possible, he replied, is that pleasures which are
% approved by the lover of wisdom and reason are the truest.

% And so we arrive at the result, that the pleasure of the intelligent
% part of the soul is the pleasantest of the three, and that he of us in
% whom this is the ruling principle has the pleasantest life.

% Unquestionably, he said, the wise man speaks with authority when he
% approves of his own life.

% And what does the judge affirm to be the life which is next, and the
% pleasure which is next?

% Clearly that of the soldier and lover of honour; who is nearer to
% himself than the money-maker.

% Last comes the lover of gain?

% Very true, he said.

% Twice in succession, then, has the just man overthrown the unjust in
% this conflict; and now comes the third trial, which is dedicated to
% Olympian Zeus the saviour: a sage whispers in my ear that no pleasure
% except that of the wise is quite true and pure--all others are a shadow
% only; and surely this will prove the greatest and most decisive of
% falls?

% Yes, the greatest; but will you explain yourself?

% I will work out the subject and you shall answer my questions.

% Proceed.

% Say, then, is not pleasure opposed to pain?

% True.

% And there is a neutral state which is neither pleasure nor pain?

% There is.

% A state which is intermediate, and a sort of repose of the soul about
% either--that is what you mean?

% Yes.

% You remember what people say when they are sick?

% What do they say?

% That after all nothing is pleasanter than health. But then they never
% knew this to be the greatest of pleasures until they were ill.

% Yes, I know, he said.

% And when persons are suffering from acute pain, you must have heard them
% say that there is nothing pleasanter than to get rid of their pain?

% I have.

% And there are many other cases of suffering in which the mere rest and
% cessation of pain, and not any positive enjoyment, is extolled by them
% as the greatest pleasure?

% Yes, he said; at the time they are pleased and well content to be at
% rest.

% Again, when pleasure ceases, that sort of rest or cessation will be
% painful?

% Doubtless, he said.

% Then the intermediate state of rest will be pleasure and will also be
% pain?

% So it would seem.

% But can that which is neither become both?

% I should say not.

% And both pleasure and pain are motions of the soul, are they not?

% Yes.

% But that which is neither was just now shown to be rest and not motion,
% and in a mean between them?

% Yes.

% How, then, can we be right in supposing that the absence of pain is
% pleasure, or that the absence of pleasure is pain?

% Impossible.

% This then is an appearance only and not a reality; that is to say, the
% rest is pleasure at the moment and in comparison of what is painful,
% and painful in comparison of what is pleasant; but all these
% representations, when tried by the test of true pleasure, are not real
% but a sort of imposition?

% That is the inference.

% Look at the other class of pleasures which have no antecedent pains and
% you will no longer suppose, as you perhaps may at present, that pleasure
% is only the cessation of pain, or pain of pleasure.

% What are they, he said, and where shall I find them?

% There are many of them: take as an example the pleasures of smell, which
% are very great and have no antecedent pains; they come in a moment, and
% when they depart leave no pain behind them.

% Most true, he said.

% Let us not, then, be induced to believe that pure pleasure is the
% cessation of pain, or pain of pleasure.

% No.

% Still, the more numerous and violent pleasures which reach the soul
% through the body are generally of this sort--they are reliefs of pain.

% That is true.

% And the anticipations of future pleasures and pains are of a like
% nature?

% Yes.

% Shall I give you an illustration of them?

% Let me hear.

% You would allow, I said, that there is in nature an upper and lower and
% middle region?

% I should.

% And if a person were to go from the lower to the middle region, would
% he not imagine that he is going up; and he who is standing in the middle
% and sees whence he has come, would imagine that he is already in the
% upper region, if he has never seen the true upper world?

% To be sure, he said; how can he think otherwise?

% But if he were taken back again he would imagine, and truly imagine,
% that he was descending?

% No doubt.

% All that would arise out of his ignorance of the true upper and middle
% and lower regions?

% Yes.

% Then can you wonder that persons who are inexperienced in the truth, as
% they have wrong ideas about many other things, should also have wrong
% ideas about pleasure and pain and the intermediate state; so that when
% they are only being drawn towards the painful they feel pain and think
% the pain which they experience to be real, and in like manner, when
% drawn away from pain to the neutral or intermediate state, they firmly
% believe that they have reached the goal of satiety and pleasure; they,
% not knowing pleasure, err in contrasting pain with the absence of pain,
% which is like contrasting black with grey instead of white--can you
% wonder, I say, at this?

% No, indeed; I should be much more disposed to wonder at the opposite.

% Look at the matter thus:--Hunger, thirst, and the like, are inanitions
% of the bodily state?

% Yes.

% And ignorance and folly are inanitions of the soul?

% True.

% And food and wisdom are the corresponding satisfactions of either?

% Certainly.

% And is the satisfaction derived from that which has less or from that
% which has more existence the truer?

% Clearly, from that which has more.

% What classes of things have a greater share of pure existence in your
% judgment--those of which food and drink and condiments and all kinds of
% sustenance are examples, or the class which contains true opinion
% and knowledge and mind and all the different kinds of virtue? Put
% the question in this way:--Which has a more pure being--that which is
% concerned with the invariable, the immortal, and the true, and is of
% such a nature, and is found in such natures; or that which is concerned
% with and found in the variable and mortal, and is itself variable and
% mortal?

% Far purer, he replied, is the being of that which is concerned with the
% invariable.

% And does the essence of the invariable partake of knowledge in the same
% degree as of essence?

% Yes, of knowledge in the same degree.

% And of truth in the same degree?

% Yes.

% And, conversely, that which has less of truth will also have less of
% essence?

% Necessarily.

% Then, in general, those kinds of things which are in the service of the
% body have less of truth and essence than those which are in the service
% of the soul?

% Far less.

% And has not the body itself less of truth and essence than the soul?

% Yes.

% What is filled with more real existence, and actually has a more real
% existence, is more really filled than that which is filled with less
% real existence and is less real?

% Of course.

% And if there be a pleasure in being filled with that which is according
% to nature, that which is more really filled with more real being
% will more really and truly enjoy true pleasure; whereas that which
% participates in less real being will be less truly and surely satisfied,
% and will participate in an illusory and less real pleasure?

% Unquestionably.

% Those then who know not wisdom and virtue, and are always busy with
% gluttony and sensuality, go down and up again as far as the mean; and
% in this region they move at random throughout life, but they never pass
% into the true upper world; thither they neither look, nor do they ever
% find their way, neither are they truly filled with true being, nor do
% they taste of pure and abiding pleasure. Like cattle, with their eyes
% always looking down and their heads stooping to the earth, that is,
% to the dining-table, they fatten and feed and breed, and, in their
% excessive love of these delights, they kick and butt at one another with
% horns and hoofs which are made of iron; and they kill one another by
% reason of their insatiable lust. For they fill themselves with that
% which is not substantial, and the part of themselves which they fill is
% also unsubstantial and incontinent.

% Verily, Socrates, said Glaucon, you describe the life of the many like
% an oracle.

% Their pleasures are mixed with pains--how can they be otherwise? For
% they are mere shadows and pictures of the true, and are coloured by
% contrast, which exaggerates both light and shade, and so they implant
% in the minds of fools insane desires of themselves; and they are fought
% about as Stesichorus says that the Greeks fought about the shadow of
% Helen at Troy in ignorance of the truth.

% Something of that sort must inevitably happen.

% And must not the like happen with the spirited or passionate element
% of the soul? Will not the passionate man who carries his passion into
% action, be in the like case, whether he is envious and ambitious, or
% violent and contentious, or angry and discontented, if he be seeking
% to attain honour and victory and the satisfaction of his anger without
% reason or sense?

% Yes, he said, the same will happen with the spirited element also.

% Then may we not confidently assert that the lovers of money and honour,
% when they seek their pleasures under the guidance and in the company
% of reason and knowledge, and pursue after and win the pleasures which
% wisdom shows them, will also have the truest pleasures in the highest
% degree which is attainable to them, inasmuch as they follow truth; and
% they will have the pleasures which are natural to them, if that which is
% best for each one is also most natural to him?

% Yes, certainly; the best is the most natural.

% And when the whole soul follows the philosophical principle, and there
% is no division, the several parts are just, and do each of them their
% own business, and enjoy severally the best and truest pleasures of which
% they are capable?

% Exactly.

% But when either of the two other principles prevails, it fails in
% attaining its own pleasure, and compels the rest to pursue after a
% pleasure which is a shadow only and which is not their own?

% True.

% And the greater the interval which separates them from philosophy and
% reason, the more strange and illusive will be the pleasure?

% Yes.

% And is not that farthest from reason which is at the greatest distance
% from law and order?

% Clearly.

% And the lustful and tyrannical desires are, as we saw, at the greatest
% distance? Yes.

% And the royal and orderly desires are nearest?

% Yes.

% Then the tyrant will live at the greatest distance from true or natural
% pleasure, and the king at the least?

% Certainly.

% But if so, the tyrant will live most unpleasantly, and the king most
% pleasantly?

% Inevitably.

% Would you know the measure of the interval which separates them?

% Will you tell me?

% There appear to be three pleasures, one genuine and two spurious: now
% the transgression of the tyrant reaches a point beyond the spurious; he
% has run away from the region of law and reason, and taken up his abode
% with certain slave pleasures which are his satellites, and the measure
% of his inferiority can only be expressed in a figure.

% How do you mean?

% I assume, I said, that the tyrant is in the third place from the
% oligarch; the democrat was in the middle?

% Yes.

% And if there is truth in what has preceded, he will be wedded to an
% image of pleasure which is thrice removed as to truth from the pleasure
% of the oligarch?

% He will.

% And the oligarch is third from the royal; since we count as one royal
% and aristocratical?

% Yes, he is third.

% Then the tyrant is removed from true pleasure by the space of a number
% which is three times three?

% Manifestly.

% The shadow then of tyrannical pleasure determined by the number of
% length will be a plane figure.

% Certainly.

% And if you raise the power and make the plane a solid, there is no
% difficulty in seeing how vast is the interval by which the tyrant is
% parted from the king.

% Yes; the arithmetician will easily do the sum.

% Or if some person begins at the other end and measures the interval by
% which the king is parted from the tyrant in truth of pleasure, he will
% find him, when the multiplication is completed, living 729 times more
% pleasantly, and the tyrant more painfully by this same interval.

% What a wonderful calculation! And how enormous is the distance which
% separates the just from the unjust in regard to pleasure and pain!

% Yet a true calculation, I said, and a number which nearly concerns human
% life, if human beings are concerned with days and nights and months and
% years. (729 NEARLY equals the number of days and nights in the year.)

% Yes, he said, human life is certainly concerned with them.

% Then if the good and just man be thus superior in pleasure to the evil
% and unjust, his superiority will be infinitely greater in propriety of
% life and in beauty and virtue?

% Immeasurably greater.

% Well, I said, and now having arrived at this stage of the argument, we
% may revert to the words which brought us hither: Was not some one saying
% that injustice was a gain to the perfectly unjust who was reputed to be
% just?

% Yes, that was said.

% Now then, having determined the power and quality of justice and
% injustice, let us have a little conversation with him.

% What shall we say to him?

% Let us make an image of the soul, that he may have his own words
% presented before his eyes.

% Of what sort?

% An ideal image of the soul, like the composite creations of ancient
% mythology, such as the Chimera or Scylla or Cerberus, and there are many
% others in which two or more different natures are said to grow into one.

% There are said of have been such unions.

% Then do you now model the form of a multitudinous, many-headed monster,
% having a ring of heads of all manner of beasts, tame and wild, which he
% is able to generate and metamorphose at will.

% You suppose marvellous powers in the artist; but, as language is more
% pliable than wax or any similar substance, let there be such a model as
% you propose.

% Suppose now that you make a second form as of a lion, and a third of a
% man, the second smaller than the first, and the third smaller than the
% second.

% That, he said, is an easier task; and I have made them as you say.

% And now join them, and let the three grow into one.

% That has been accomplished.

% Next fashion the outside of them into a single image, as of a man, so
% that he who is not able to look within, and sees only the outer hull,
% may believe the beast to be a single human creature.

% I have done so, he said.

% And now, to him who maintains that it is profitable for the human
% creature to be unjust, and unprofitable to be just, let us reply
% that, if he be right, it is profitable for this creature to feast
% the multitudinous monster and strengthen the lion and the lion-like
% qualities, but to starve and weaken the man, who is consequently liable
% to be dragged about at the mercy of either of the other two; and he is
% not to attempt to familiarize or harmonize them with one another--he
% ought rather to suffer them to fight and bite and devour one another.

% Certainly, he said; that is what the approver of injustice says.

% To him the supporter of justice makes answer that he should ever so
% speak and act as to give the man within him in some way or other the
% most complete mastery over the entire human creature. He should watch
% over the many-headed monster like a good husbandman, fostering and
% cultivating the gentle qualities, and preventing the wild ones from
% growing; he should be making the lion-heart his ally, and in common care
% of them all should be uniting the several parts with one another and
% with himself.

% Yes, he said, that is quite what the maintainer of justice say.

% And so from every point of view, whether of pleasure, honour, or
% advantage, the approver of justice is right and speaks the truth, and
% the disapprover is wrong and false and ignorant?

% Yes, from every point of view.

% Come, now, and let us gently reason with the unjust, who is not
% intentionally in error. ``Sweet Sir,'' we will say to him, ``what think
% you of things esteemed noble and ignoble? Is not the noble that which
% subjects the beast to the man, or rather to the god in man; and the
% ignoble that which subjects the man to the beast?'' He can hardly avoid
% saying Yes--can he now?

% Not if he has any regard for my opinion.

% But, if he agree so far, we may ask him to answer another question:
% ``Then how would a man profit if he received gold and silver on the
% condition that he was to enslave the noblest part of him to the worst?
% Who can imagine that a man who sold his son or daughter into slavery for
% money, especially if he sold them into the hands of fierce and evil men,
% would be the gainer, however large might be the sum which he
% received? And will any one say that he is not a miserable caitiff who
% remorselessly sells his own divine being to that which is most godless
% and detestable? Eriphyle took the necklace as the price of her husband's
% life, but he is taking a bribe in order to compass a worse ruin.''

% Yes, said Glaucon, far worse--I will answer for him.

% Has not the intemperate been censured of old, because in him the huge
% multiform monster is allowed to be too much at large?

% Clearly.

% And men are blamed for pride and bad temper when the lion and serpent
% element in them disproportionately grows and gains strength?

% Yes.

% And luxury and softness are blamed, because they relax and weaken this
% same creature, and make a coward of him?

% Very true.

% And is not a man reproached for flattery and meanness who subordinates
% the spirited animal to the unruly monster, and, for the sake of money,
% of which he can never have enough, habituates him in the days of his
% youth to be trampled in the mire, and from being a lion to become a
% monkey?

% True, he said.

% And why are mean employments and manual arts a reproach? Only because
% they imply a natural weakness of the higher principle; the individual is
% unable to control the creatures within him, but has to court them, and
% his great study is how to flatter them.

% Such appears to be the reason.

% And therefore, being desirous of placing him under a rule like that of
% the best, we say that he ought to be the servant of the best, in whom
% the Divine rules; not, as Thrasymachus supposed, to the injury of the
% servant, but because every one had better be ruled by divine wisdom
% dwelling within him; or, if this be impossible, then by an external
% authority, in order that we may be all, as far as possible, under the
% same government, friends and equals.

% True, he said.

% And this is clearly seen to be the intention of the law, which is the
% ally of the whole city; and is seen also in the authority which we
% exercise over children, and the refusal to let them be free until we
% have established in them a principle analogous to the constitution of
% a state, and by cultivation of this higher element have set up in their
% hearts a guardian and ruler like our own, and when this is done they may
% go their ways.

% Yes, he said, the purpose of the law is manifest.

% From what point of view, then, and on what ground can we say that a man
% is profited by injustice or intemperance or other baseness, which will
% make him a worse man, even though he acquire money or power by his
% wickedness?

% From no point of view at all.

% What shall he profit, if his injustice be undetected and unpunished?
% He who is undetected only gets worse, whereas he who is detected and
% punished has the brutal part of his nature silenced and humanized; the
% gentler element in him is liberated, and his whole soul is perfected and
% ennobled by the acquirement of justice and temperance and wisdom, more
% than the body ever is by receiving gifts of beauty, strength and health,
% in proportion as the soul is more honourable than the body.

% Certainly, he said.

% To this nobler purpose the man of understanding will devote the energies
% of his life. And in the first place, he will honour studies which
% impress these qualities on his soul and will disregard others?

% Clearly, he said.

% In the next place, he will regulate his bodily habit and training, and
% so far will he be from yielding to brutal and irrational pleasures, that
% he will regard even health as quite a secondary matter; his first object
% will be not that he may be fair or strong or well, unless he is likely
% thereby to gain temperance, but he will always desire so to attemper the
% body as to preserve the harmony of the soul?

% Certainly he will, if he has true music in him.

% And in the acquisition of wealth there is a principle of order and
% harmony which he will also observe; he will not allow himself to be
% dazzled by the foolish applause of the world, and heap up riches to his
% own infinite harm?

% Certainly not, he said.

% He will look at the city which is within him, and take heed that no
% disorder occur in it, such as might arise either from superfluity or
% from want; and upon this principle he will regulate his property and
% gain or spend according to his means.

% Very true.

% And, for the same reason, he will gladly accept and enjoy such honours
% as he deems likely to make him a better man; but those, whether private
% or public, which are likely to disorder his life, he will avoid?

% Then, if that is his motive, he will not be a statesman.

% By the dog of Egypt, he will! in the city which is his own he certainly
% will, though in the land of his birth perhaps not, unless he have a
% divine call.

% I understand; you mean that he will be a ruler in the city of which we
% are the founders, and which exists in idea only; for I do not believe
% that there is such an one anywhere on earth?

% In heaven, I replied, there is laid up a pattern of it, methinks, which
% he who desires may behold, and beholding, may set his own house in
% order. But whether such an one exists, or ever will exist in fact, is no
% matter; for he will live after the manner of that city, having nothing
% to do with any other.

% I think so, he said.


\section{Book X} % (fold)
\label{sec:book_x}



BOOK X.

Of the many excellences which I perceive in the order of our State,
there is none which upon reflection pleases me better than the rule
about poetry.

To what do you refer?

To the rejection of imitative poetry, which certainly ought not to be
received; as I see far more clearly now that the parts of the soul have
been distinguished.

What do you mean?

Speaking in confidence, for I should not like to have my words repeated
to the tragedians and the rest of the imitative tribe--but I do not
mind saying to you, that all poetical imitations are ruinous to the
understanding of the hearers, and that the knowledge of their true
nature is the only antidote to them.

Explain the purport of your remark.

Well, I will tell you, although I have always from my earliest youth had
an awe and love of Homer, which even now makes the words falter on
my lips, for he is the great captain and teacher of the whole of that
charming tragic company; but a man is not to be reverenced more than the
truth, and therefore I will speak out.

Very good, he said.

Listen to me then, or rather, answer me.

Put your question.

Can you tell me what imitation is? for I really do not know.

A likely thing, then, that I should know.

Why not? for the duller eye may often see a thing sooner than the
keener.

Very true, he said; but in your presence, even if I had any faint
notion, I could not muster courage to utter it. Will you enquire
yourself?

Well then, shall we begin the enquiry in our usual manner: Whenever a
number of individuals have a common name, we assume them to have also a
corresponding idea or form:--do you understand me?

I do.

Let us take any common instance; there are beds and tables in the
world--plenty of them, are there not?

Yes.

But there are only two ideas or forms of them--one the idea of a bed,
the other of a table.

True.

And the maker of either of them makes a bed or he makes a table for our
use, in accordance with the idea--that is our way of speaking in this
and similar instances--but no artificer makes the ideas themselves: how
could he?

Impossible.

And there is another artist,--I should like to know what you would say
of him.

Who is he?

One who is the maker of all the works of all other workmen.

What an extraordinary man!

Wait a little, and there will be more reason for your saying so. For
this is he who is able to make not only vessels of every kind, but
plants and animals, himself and all other things--the earth and heaven,
and the things which are in heaven or under the earth; he makes the gods
also.

He must be a wizard and no mistake.

Oh! you are incredulous, are you? Do you mean that there is no such
maker or creator, or that in one sense there might be a maker of all
these things but in another not? Do you see that there is a way in which
you could make them all yourself?

What way?

An easy way enough; or rather, there are many ways in which the feat
might be quickly and easily accomplished, none quicker than that of
turning a mirror round and round--you would soon enough make the sun and
the heavens, and the earth and yourself, and other animals and plants,
and all the other things of which we were just now speaking, in the
mirror.

Yes, he said; but they would be appearances only.

Very good, I said, you are coming to the point now. And the painter too
is, as I conceive, just such another--a creator of appearances, is he
not?

Of course.

But then I suppose you will say that what he creates is untrue. And yet
there is a sense in which the painter also creates a bed?

Yes, he said, but not a real bed.

And what of the maker of the bed? were you not saying that he too makes,
not the idea which, according to our view, is the essence of the bed,
but only a particular bed?

Yes, I did.

Then if he does not make that which exists he cannot make true
existence, but only some semblance of existence; and if any one were to
say that the work of the maker of the bed, or of any other workman, has
real existence, he could hardly be supposed to be speaking the truth.

At any rate, he replied, philosophers would say that he was not speaking
the truth.

No wonder, then, that his work too is an indistinct expression of truth.

No wonder.

Suppose now that by the light of the examples just offered we enquire
who this imitator is?

If you please.

Well then, here are three beds: one existing in nature, which is made by
God, as I think that we may say--for no one else can be the maker?

No.

There is another which is the work of the carpenter?

Yes.

And the work of the painter is a third?

Yes.

Beds, then, are of three kinds, and there are three artists who
superintend them: God, the maker of the bed, and the painter?

Yes, there are three of them.

God, whether from choice or from necessity, made one bed in nature and
one only; two or more such ideal beds neither ever have been nor ever
will be made by God.

Why is that?

Because even if He had made but two, a third would still appear behind
them which both of them would have for their idea, and that would be the
ideal bed and not the two others.

Very true, he said.

God knew this, and He desired to be the real maker of a real bed, not
a particular maker of a particular bed, and therefore He created a bed
which is essentially and by nature one only.

So we believe.

Shall we, then, speak of Him as the natural author or maker of the bed?

Yes, he replied; inasmuch as by the natural process of creation He is
the author of this and of all other things.

And what shall we say of the carpenter--is not he also the maker of the
bed?

Yes.

But would you call the painter a creator and maker?

Certainly not.

Yet if he is not the maker, what is he in relation to the bed?

I think, he said, that we may fairly designate him as the imitator of
that which the others make.

Good, I said; then you call him who is third in the descent from nature
an imitator?

Certainly, he said.

And the tragic poet is an imitator, and therefore, like all other
imitators, he is thrice removed from the king and from the truth?

That appears to be so.

Then about the imitator we are agreed. And what about the painter?--I
would like to know whether he may be thought to imitate that which
originally exists in nature, or only the creations of artists?

The latter.

As they are or as they appear? you have still to determine this.

What do you mean?

I mean, that you may look at a bed from different points of view,
obliquely or directly or from any other point of view, and the bed will
appear different, but there is no difference in reality. And the same of
all things.

Yes, he said, the difference is only apparent.

Now let me ask you another question: Which is the art of painting
designed to be--an imitation of things as they are, or as they
appear--of appearance or of reality?

Of appearance.

Then the imitator, I said, is a long way off the truth, and can do all
things because he lightly touches on a small part of them, and that part
an image. For example: A painter will paint a cobbler, carpenter, or
any other artist, though he knows nothing of their arts; and, if he is
a good artist, he may deceive children or simple persons, when he shows
them his picture of a carpenter from a distance, and they will fancy
that they are looking at a real carpenter.

Certainly.

And whenever any one informs us that he has found a man who knows all
the arts, and all things else that anybody knows, and every single thing
with a higher degree of accuracy than any other man--whoever tells us
this, I think that we can only imagine him to be a simple creature who
is likely to have been deceived by some wizard or actor whom he met, and
whom he thought all-knowing, because he himself was unable to analyse
the nature of knowledge and ignorance and imitation.

Most true.

And so, when we hear persons saying that the tragedians, and Homer, who
is at their head, know all the arts and all things human, virtue as well
as vice, and divine things too, for that the good poet cannot compose
well unless he knows his subject, and that he who has not this knowledge
can never be a poet, we ought to consider whether here also there may
not be a similar illusion. Perhaps they may have come across imitators
and been deceived by them; they may not have remembered when they saw
their works that these were but imitations thrice removed from the
truth, and could easily be made without any knowledge of the truth,
because they are appearances only and not realities? Or, after all, they
may be in the right, and poets do really know the things about which
they seem to the many to speak so well?

The question, he said, should by all means be considered.

Now do you suppose that if a person were able to make the original as
well as the image, he would seriously devote himself to the image-making
branch? Would he allow imitation to be the ruling principle of his life,
as if he had nothing higher in him?

I should say not.

The real artist, who knew what he was imitating, would be interested in
realities and not in imitations; and would desire to leave as memorials
of himself works many and fair; and, instead of being the author of
encomiums, he would prefer to be the theme of them.

Yes, he said, that would be to him a source of much greater honour and
profit.

Then, I said, we must put a question to Homer; not about medicine, or
any of the arts to which his poems only incidentally refer: we are not
going to ask him, or any other poet, whether he has cured patients
like Asclepius, or left behind him a school of medicine such as the
Asclepiads were, or whether he only talks about medicine and other arts
at second-hand; but we have a right to know respecting military tactics,
politics, education, which are the chiefest and noblest subjects of his
poems, and we may fairly ask him about them. ``Friend Homer,'' then we say
to him, ``if you are only in the second remove from truth in what you say
of virtue, and not in the third--not an image maker or imitator--and
if you are able to discern what pursuits make men better or worse in
private or public life, tell us what State was ever better governed by
your help? The good order of Lacedaemon is due to Lycurgus, and many
other cities great and small have been similarly benefited by others;
but who says that you have been a good legislator to them and have done
them any good? Italy and Sicily boast of Charondas, and there is Solon
who is renowned among us; but what city has anything to say about you?''
Is there any city which he might name?

I think not, said Glaucon; not even the Homerids themselves pretend that
he was a legislator.

Well, but is there any war on record which was carried on successfully
by him, or aided by his counsels, when he was alive?

There is not.

Or is there any invention of his, applicable to the arts or to human
life, such as Thales the Milesian or Anacharsis the Scythian, and other
ingenious men have conceived, which is attributed to him?

There is absolutely nothing of the kind.

But, if Homer never did any public service, was he privately a guide or
teacher of any? Had he in his lifetime friends who loved to associate
with him, and who handed down to posterity an Homeric way of life, such
as was established by Pythagoras who was so greatly beloved for his
wisdom, and whose followers are to this day quite celebrated for the
order which was named after him?

Nothing of the kind is recorded of him. For surely, Socrates,
Creophylus, the companion of Homer, that child of flesh, whose name
always makes us laugh, might be more justly ridiculed for his stupidity,
if, as is said, Homer was greatly neglected by him and others in his own
day when he was alive?

Yes, I replied, that is the tradition. But can you imagine, Glaucon,
that if Homer had really been able to educate and improve mankind--if he
had possessed knowledge and not been a mere imitator--can you imagine,
I say, that he would not have had many followers, and been honoured and
loved by them? Protagoras of Abdera, and Prodicus of Ceos, and a host of
others, have only to whisper to their contemporaries: ``You will never be
able to manage either your own house or your own State until you appoint
us to be your ministers of education''--and this ingenious device of
theirs has such an effect in making men love them that their companions
all but carry them about on their shoulders. And is it conceivable that
the contemporaries of Homer, or again of Hesiod, would have allowed
either of them to go about as rhapsodists, if they had really been able
to make mankind virtuous? Would they not have been as unwilling to part
with them as with gold, and have compelled them to stay at home with
them? Or, if the master would not stay, then the disciples would have
followed him about everywhere, until they had got education enough?

Yes, Socrates, that, I think, is quite true.

Then must we not infer that all these poetical individuals, beginning
with Homer, are only imitators; they copy images of virtue and the like,
but the truth they never reach? The poet is like a painter who, as
we have already observed, will make a likeness of a cobbler though he
understands nothing of cobbling; and his picture is good enough for
those who know no more than he does, and judge only by colours and
figures.

Quite so.

In like manner the poet with his words and phrases may be said to lay on
the colours of the several arts, himself understanding their nature only
enough to imitate them; and other people, who are as ignorant as he is,
and judge only from his words, imagine that if he speaks of cobbling,
or of military tactics, or of anything else, in metre and harmony and
rhythm, he speaks very well--such is the sweet influence which melody
and rhythm by nature have. And I think that you must have observed again
and again what a poor appearance the tales of poets make when stripped
of the colours which music puts upon them, and recited in simple prose.

Yes, he said.

They are like faces which were never really beautiful, but only
blooming; and now the bloom of youth has passed away from them?

Exactly.

Here is another point: The imitator or maker of the image knows nothing
of true existence; he knows appearances only. Am I not right?

Yes.

Then let us have a clear understanding, and not be satisfied with half
an explanation.

Proceed.

Of the painter we say that he will paint reins, and he will paint a bit?

Yes.

And the worker in leather and brass will make them?

Certainly.

But does the painter know the right form of the bit and reins? Nay,
hardly even the workers in brass and leather who make them; only the
horseman who knows how to use them--he knows their right form.

Most true.

And may we not say the same of all things?

What?

That there are three arts which are concerned with all things: one which
uses, another which makes, a third which imitates them?

Yes.

And the excellence or beauty or truth of every structure, animate or
inanimate, and of every action of man, is relative to the use for which
nature or the artist has intended them.

True.

Then the user of them must have the greatest experience of them, and
he must indicate to the maker the good or bad qualities which develop
themselves in use; for example, the flute-player will tell the
flute-maker which of his flutes is satisfactory to the performer; he
will tell him how he ought to make them, and the other will attend to
his instructions?

Of course.

The one knows and therefore speaks with authority about the goodness and
badness of flutes, while the other, confiding in him, will do what he is
told by him?

True.

The instrument is the same, but about the excellence or badness of it
the maker will only attain to a correct belief; and this he will gain
from him who knows, by talking to him and being compelled to hear what
he has to say, whereas the user will have knowledge?

True.

But will the imitator have either? Will he know from use whether or no
his drawing is correct or beautiful? or will he have right opinion
from being compelled to associate with another who knows and gives him
instructions about what he should draw?

Neither.

Then he will no more have true opinion than he will have knowledge about
the goodness or badness of his imitations?

I suppose not.

The imitative artist will be in a brilliant state of intelligence about
his own creations?

Nay, very much the reverse.

And still he will go on imitating without knowing what makes a thing
good or bad, and may be expected therefore to imitate only that which
appears to be good to the ignorant multitude?

Just so.

Thus far then we are pretty well agreed that the imitator has no
knowledge worth mentioning of what he imitates. Imitation is only a kind
of play or sport, and the tragic poets, whether they write in Iambic or
in Heroic verse, are imitators in the highest degree?

Very true.

And now tell me, I conjure you, has not imitation been shown by us to be
concerned with that which is thrice removed from the truth?

Certainly.

And what is the faculty in man to which imitation is addressed?

What do you mean?

I will explain: The body which is large when seen near, appears small
when seen at a distance?

True.

And the same object appears straight when looked at out of the water,
and crooked when in the water; and the concave becomes convex, owing to
the illusion about colours to which the sight is liable. Thus every sort
of confusion is revealed within us; and this is that weakness of the
human mind on which the art of conjuring and of deceiving by light and
shadow and other ingenious devices imposes, having an effect upon us
like magic.

True.

And the arts of measuring and numbering and weighing come to the
rescue of the human understanding--there is the beauty of them--and the
apparent greater or less, or more or heavier, no longer have the mastery
over us, but give way before calculation and measure and weight?

Most true.

And this, surely, must be the work of the calculating and rational
principle in the soul?

To be sure.

And when this principle measures and certifies that some things are
equal, or that some are greater or less than others, there occurs an
apparent contradiction?

True.

But were we not saying that such a contradiction is impossible--the same
faculty cannot have contrary opinions at the same time about the same
thing?

Very true.

Then that part of the soul which has an opinion contrary to measure is
not the same with that which has an opinion in accordance with measure?

True.

And the better part of the soul is likely to be that which trusts to
measure and calculation?

Certainly.

And that which is opposed to them is one of the inferior principles of
the soul?

No doubt.

This was the conclusion at which I was seeking to arrive when I said
that painting or drawing, and imitation in general, when doing their own
proper work, are far removed from truth, and the companions and friends
and associates of a principle within us which is equally removed from
reason, and that they have no true or healthy aim.

Exactly.

The imitative art is an inferior who marries an inferior, and has
inferior offspring.

Very true.

And is this confined to the sight only, or does it extend to the hearing
also, relating in fact to what we term poetry?

Probably the same would be true of poetry.

Do not rely, I said, on a probability derived from the analogy of
painting; but let us examine further and see whether the faculty with
which poetical imitation is concerned is good or bad.

By all means.

We may state the question thus:--Imitation imitates the actions of men,
whether voluntary or involuntary, on which, as they imagine, a good or
bad result has ensued, and they rejoice or sorrow accordingly. Is there
anything more?

No, there is nothing else.

But in all this variety of circumstances is the man at unity with
himself--or rather, as in the instance of sight there was confusion and
opposition in his opinions about the same things, so here also is there
not strife and inconsistency in his life? Though I need hardly raise the
question again, for I remember that all this has been already admitted;
and the soul has been acknowledged by us to be full of these and ten
thousand similar oppositions occurring at the same moment?

And we were right, he said.

Yes, I said, thus far we were right; but there was an omission which
must now be supplied.

What was the omission?

Were we not saying that a good man, who has the misfortune to lose his
son or anything else which is most dear to him, will bear the loss with
more equanimity than another?

Yes.

But will he have no sorrow, or shall we say that although he cannot help
sorrowing, he will moderate his sorrow?

The latter, he said, is the truer statement.

Tell me: will he be more likely to struggle and hold out against his
sorrow when he is seen by his equals, or when he is alone?

It will make a great difference whether he is seen or not.

When he is by himself he will not mind saying or doing many things which
he would be ashamed of any one hearing or seeing him do?

True.

There is a principle of law and reason in him which bids him resist, as
well as a feeling of his misfortune which is forcing him to indulge his
sorrow?

True.

But when a man is drawn in two opposite directions, to and from the same
object, this, as we affirm, necessarily implies two distinct principles
in him?

Certainly.

One of them is ready to follow the guidance of the law?

How do you mean?

The law would say that to be patient under suffering is best, and that
we should not give way to impatience, as there is no knowing whether
such things are good or evil; and nothing is gained by impatience; also,
because no human thing is of serious importance, and grief stands in the
way of that which at the moment is most required.

What is most required? he asked.

That we should take counsel about what has happened, and when the dice
have been thrown order our affairs in the way which reason deems best;
not, like children who have had a fall, keeping hold of the part struck
and wasting time in setting up a howl, but always accustoming the soul
forthwith to apply a remedy, raising up that which is sickly and fallen,
banishing the cry of sorrow by the healing art.

Yes, he said, that is the true way of meeting the attacks of fortune.

Yes, I said; and the higher principle is ready to follow this suggestion
of reason?

Clearly.

And the other principle, which inclines us to recollection of our
troubles and to lamentation, and can never have enough of them, we may
call irrational, useless, and cowardly?

Indeed, we may.

And does not the latter--I mean the rebellious principle--furnish a
great variety of materials for imitation? Whereas the wise and calm
temperament, being always nearly equable, is not easy to imitate or
to appreciate when imitated, especially at a public festival when a
promiscuous crowd is assembled in a theatre. For the feeling represented
is one to which they are strangers.

Certainly.

Then the imitative poet who aims at being popular is not by nature made,
nor is his art intended, to please or to affect the rational principle
in the soul; but he will prefer the passionate and fitful temper, which
is easily imitated?

Clearly.

And now we may fairly take him and place him by the side of the painter,
for he is like him in two ways: first, inasmuch as his creations have an
inferior degree of truth--in this, I say, he is like him; and he is
also like him in being concerned with an inferior part of the soul; and
therefore we shall be right in refusing to admit him into a well-ordered
State, because he awakens and nourishes and strengthens the feelings
and impairs the reason. As in a city when the evil are permitted to have
authority and the good are put out of the way, so in the soul of man,
as we maintain, the imitative poet implants an evil constitution, for he
indulges the irrational nature which has no discernment of greater
and less, but thinks the same thing at one time great and at another
small--he is a manufacturer of images and is very far removed from the
truth.

Exactly.

But we have not yet brought forward the heaviest count in our
accusation:--the power which poetry has of harming even the good (and
there are very few who are not harmed), is surely an awful thing?

Yes, certainly, if the effect is what you say.

Hear and judge: The best of us, as I conceive, when we listen to a
passage of Homer, or one of the tragedians, in which he represents
some pitiful hero who is drawling out his sorrows in a long oration, or
weeping, and smiting his breast--the best of us, you know, delight in
giving way to sympathy, and are in raptures at the excellence of the
poet who stirs our feelings most.

Yes, of course I know.

But when any sorrow of our own happens to us, then you may observe that
we pride ourselves on the opposite quality--we would fain be quiet and
patient; this is the manly part, and the other which delighted us in the
recitation is now deemed to be the part of a woman.

Very true, he said.

Now can we be right in praising and admiring another who is doing that
which any one of us would abominate and be ashamed of in his own person?

No, he said, that is certainly not reasonable.

Nay, I said, quite reasonable from one point of view.

What point of view?

If you consider, I said, that when in misfortune we feel a natural
hunger and desire to relieve our sorrow by weeping and lamentation, and
that this feeling which is kept under control in our own calamities is
satisfied and delighted by the poets;--the better nature in each of
us, not having been sufficiently trained by reason or habit, allows the
sympathetic element to break loose because the sorrow is another's;
and the spectator fancies that there can be no disgrace to himself in
praising and pitying any one who comes telling him what a good man he
is, and making a fuss about his troubles; he thinks that the pleasure
is a gain, and why should he be supercilious and lose this and the poem
too? Few persons ever reflect, as I should imagine, that from the evil
of other men something of evil is communicated to themselves. And so
the feeling of sorrow which has gathered strength at the sight of the
misfortunes of others is with difficulty repressed in our own.

How very true!

And does not the same hold also of the ridiculous? There are jests which
you would be ashamed to make yourself, and yet on the comic stage, or
indeed in private, when you hear them, you are greatly amused by them,
and are not at all disgusted at their unseemliness;--the case of pity
is repeated;--there is a principle in human nature which is disposed to
raise a laugh, and this which you once restrained by reason, because you
were afraid of being thought a buffoon, is now let out again; and
having stimulated the risible faculty at the theatre, you are betrayed
unconsciously to yourself into playing the comic poet at home.

Quite true, he said.

And the same may be said of lust and anger and all the other affections,
of desire and pain and pleasure, which are held to be inseparable
from every action--in all of them poetry feeds and waters the passions
instead of drying them up; she lets them rule, although they ought to be
controlled, if mankind are ever to increase in happiness and virtue.

I cannot deny it.

Therefore, Glaucon, I said, whenever you meet with any of the eulogists
of Homer declaring that he has been the educator of Hellas, and that he
is profitable for education and for the ordering of human things, and
that you should take him up again and again and get to know him and
regulate your whole life according to him, we may love and honour those
who say these things--they are excellent people, as far as their lights
extend; and we are ready to acknowledge that Homer is the greatest
of poets and first of tragedy writers; but we must remain firm in our
conviction that hymns to the gods and praises of famous men are the only
poetry which ought to be admitted into our State. For if you go beyond
this and allow the honeyed muse to enter, either in epic or lyric verse,
not law and the reason of mankind, which by common consent have ever
been deemed best, but pleasure and pain will be the rulers in our State.

That is most true, he said.

And now since we have reverted to the subject of poetry, let this our
defence serve to show the reasonableness of our former judgment in
sending away out of our State an art having the tendencies which we have
described; for reason constrained us. But that she may not impute to us
any harshness or want of politeness, let us tell her that there is an
ancient quarrel between philosophy and poetry; of which there are many
proofs, such as the saying of ``the yelping hound howling at her lord,''
or of one ``mighty in the vain talk of fools,'' and ``the mob of sages
circumventing Zeus,'' and the ``subtle thinkers who are beggars after
all''; and there are innumerable other signs of ancient enmity between
them. Notwithstanding this, let us assure our sweet friend and the
sister arts of imitation, that if she will only prove her title to exist
in a well-ordered State we shall be delighted to receive her--we are
very conscious of her charms; but we may not on that account betray the
truth. I dare say, Glaucon, that you are as much charmed by her as I am,
especially when she appears in Homer?

Yes, indeed, I am greatly charmed.

Shall I propose, then, that she be allowed to return from exile, but
upon this condition only--that she make a defence of herself in lyrical
or some other metre?

Certainly.

And we may further grant to those of her defenders who are lovers of
poetry and yet not poets the permission to speak in prose on her behalf:
let them show not only that she is pleasant but also useful to States
and to human life, and we will listen in a kindly spirit; for if this
can be proved we shall surely be the gainers--I mean, if there is a use
in poetry as well as a delight?

Certainly, he said, we shall be the gainers.

If her defence fails, then, my dear friend, like other persons who are
enamoured of something, but put a restraint upon themselves when they
think their desires are opposed to their interests, so too must we after
the manner of lovers give her up, though not without a struggle. We too
are inspired by that love of poetry which the education of noble States
has implanted in us, and therefore we would have her appear at her best
and truest; but so long as she is unable to make good her defence,
this argument of ours shall be a charm to us, which we will repeat to
ourselves while we listen to her strains; that we may not fall away into
the childish love of her which captivates the many. At all events we
are well aware that poetry being such as we have described is not to be
regarded seriously as attaining to the truth; and he who listens to her,
fearing for the safety of the city which is within him, should be on his
guard against her seductions and make our words his law.

Yes, he said, I quite agree with you.

Yes, I said, my dear Glaucon, for great is the issue at stake, greater
than appears, whether a man is to be good or bad. And what will any one
be profited if under the influence of honour or money or power, aye, or
under the excitement of poetry, he neglect justice and virtue?

Yes, he said; I have been convinced by the argument, as I believe that
any one else would have been.

And yet no mention has been made of the greatest prizes and rewards
which await virtue.

What, are there any greater still? If there are, they must be of an
inconceivable greatness.

Why, I said, what was ever great in a short time? The whole period of
three score years and ten is surely but a little thing in comparison
with eternity?

Say rather ``nothing,'' he replied.

And should an immortal being seriously think of this little space rather
than of the whole?

Of the whole, certainly. But why do you ask?

Are you not aware, I said, that the soul of man is immortal and
imperishable?

He looked at me in astonishment, and said: No, by heaven: And are you
really prepared to maintain this?

Yes, I said, I ought to be, and you too--there is no difficulty in
proving it.

I see a great difficulty; but I should like to hear you state this
argument of which you make so light.

Listen then.

I am attending.

There is a thing which you call good and another which you call evil?

Yes, he replied.

Would you agree with me in thinking that the corrupting and destroying
element is the evil, and the saving and improving element the good?

Yes.

And you admit that every thing has a good and also an evil; as
ophthalmia is the evil of the eyes and disease of the whole body; as
mildew is of corn, and rot of timber, or rust of copper and iron: in
everything, or in almost everything, there is an inherent evil and
disease?

Yes, he said.

And anything which is infected by any of these evils is made evil, and
at last wholly dissolves and dies?

True.

The vice and evil which is inherent in each is the destruction of each;
and if this does not destroy them there is nothing else that will; for
good certainly will not destroy them, nor again, that which is neither
good nor evil.

Certainly not.

If, then, we find any nature which having this inherent corruption
cannot be dissolved or destroyed, we may be certain that of such a
nature there is no destruction?

That may be assumed.

Well, I said, and is there no evil which corrupts the soul?

Yes, he said, there are all the evils which we were just now passing in
review: unrighteousness, intemperance, cowardice, ignorance.

But does any of these dissolve or destroy her?--and here do not let us
fall into the error of supposing that the unjust and foolish man, when
he is detected, perishes through his own injustice, which is an evil
of the soul. Take the analogy of the body: The evil of the body is a
disease which wastes and reduces and annihilates the body; and all the
things of which we were just now speaking come to annihilation through
their own corruption attaching to them and inhering in them and so
destroying them. Is not this true?

Yes.

Consider the soul in like manner. Does the injustice or other evil which
exists in the soul waste and consume her? Do they by attaching to the
soul and inhering in her at last bring her to death, and so separate her
from the body?

Certainly not.

And yet, I said, it is unreasonable to suppose that anything can perish
from without through affection of external evil which could not be
destroyed from within by a corruption of its own?

It is, he replied.

Consider, I said, Glaucon, that even the badness of food, whether
staleness, decomposition, or any other bad quality, when confined to
the actual food, is not supposed to destroy the body; although, if the
badness of food communicates corruption to the body, then we should say
that the body has been destroyed by a corruption of itself, which is
disease, brought on by this; but that the body, being one thing, can be
destroyed by the badness of food, which is another, and which does not
engender any natural infection--this we shall absolutely deny?

Very true.

And, on the same principle, unless some bodily evil can produce an evil
of the soul, we must not suppose that the soul, which is one thing, can
be dissolved by any merely external evil which belongs to another?

Yes, he said, there is reason in that.

Either, then, let us refute this conclusion, or, while it remains
unrefuted, let us never say that fever, or any other disease, or the
knife put to the throat, or even the cutting up of the whole body into
the minutest pieces, can destroy the soul, until she herself is proved
to become more unholy or unrighteous in consequence of these things
being done to the body; but that the soul, or anything else if not
destroyed by an internal evil, can be destroyed by an external one, is
not to be affirmed by any man.

And surely, he replied, no one will ever prove that the souls of men
become more unjust in consequence of death.

But if some one who would rather not admit the immortality of the soul
boldly denies this, and says that the dying do really become more
evil and unrighteous, then, if the speaker is right, I suppose that
injustice, like disease, must be assumed to be fatal to the unjust, and
that those who take this disorder die by the natural inherent power of
destruction which evil has, and which kills them sooner or later, but
in quite another way from that in which, at present, the wicked receive
death at the hands of others as the penalty of their deeds?

Nay, he said, in that case injustice, if fatal to the unjust, will not
be so very terrible to him, for he will be delivered from evil. But I
rather suspect the opposite to be the truth, and that injustice which,
if it have the power, will murder others, keeps the murderer alive--aye,
and well awake too; so far removed is her dwelling-place from being a
house of death.

True, I said; if the inherent natural vice or evil of the soul is unable
to kill or destroy her, hardly will that which is appointed to be the
destruction of some other body, destroy a soul or anything else except
that of which it was appointed to be the destruction.

Yes, that can hardly be.

But the soul which cannot be destroyed by an evil, whether inherent
or external, must exist for ever, and if existing for ever, must be
immortal?

Certainly.

That is the conclusion, I said; and, if a true conclusion, then the
souls must always be the same, for if none be destroyed they will not
diminish in number. Neither will they increase, for the increase of the
immortal natures must come from something mortal, and all things would
thus end in immortality.

Very true.

But this we cannot believe--reason will not allow us--any more than we
can believe the soul, in her truest nature, to be full of variety and
difference and dissimilarity.

What do you mean? he said.

The soul, I said, being, as is now proven, immortal, must be the fairest
of compositions and cannot be compounded of many elements?

Certainly not.

Her immortality is demonstrated by the previous argument, and there are
many other proofs; but to see her as she really is, not as we now behold
her, marred by communion with the body and other miseries, you must
contemplate her with the eye of reason, in her original purity; and
then her beauty will be revealed, and justice and injustice and all the
things which we have described will be manifested more clearly. Thus
far, we have spoken the truth concerning her as she appears at present,
but we must remember also that we have seen her only in a condition
which may be compared to that of the sea-god Glaucus, whose original
image can hardly be discerned because his natural members are broken
off and crushed and damaged by the waves in all sorts of ways, and
incrustations have grown over them of seaweed and shells and stones, so
that he is more like some monster than he is to his own natural form.
And the soul which we behold is in a similar condition, disfigured by
ten thousand ills. But not there, Glaucon, not there must we look.

Where then?

At her love of wisdom. Let us see whom she affects, and what society and
converse she seeks in virtue of her near kindred with the immortal
and eternal and divine; also how different she would become if wholly
following this superior principle, and borne by a divine impulse out of
the ocean in which she now is, and disengaged from the stones and shells
and things of earth and rock which in wild variety spring up around her
because she feeds upon earth, and is overgrown by the good things of
this life as they are termed: then you would see her as she is, and know
whether she have one shape only or many, or what her nature is. Of her
affections and of the forms which she takes in this present life I think
that we have now said enough.

True, he replied.

And thus, I said, we have fulfilled the conditions of the argument; we
have not introduced the rewards and glories of justice, which, as you
were saying, are to be found in Homer and Hesiod; but justice in her own
nature has been shown to be best for the soul in her own nature. Let a
man do what is just, whether he have the ring of Gyges or not, and even
if in addition to the ring of Gyges he put on the helmet of Hades.

Very true.

And now, Glaucon, there will be no harm in further enumerating how
many and how great are the rewards which justice and the other virtues
procure to the soul from gods and men, both in life and after death.

Certainly not, he said.

Will you repay me, then, what you borrowed in the argument?

What did I borrow?

The assumption that the just man should appear unjust and the unjust
just: for you were of opinion that even if the true state of the case
could not possibly escape the eyes of gods and men, still this admission
ought to be made for the sake of the argument, in order that pure
justice might be weighed against pure injustice. Do you remember?

I should be much to blame if I had forgotten.

Then, as the cause is decided, I demand on behalf of justice that the
estimation in which she is held by gods and men and which we acknowledge
to be her due should now be restored to her by us; since she has been
shown to confer reality, and not to deceive those who truly possess her,
let what has been taken from her be given back, that so she may win that
palm of appearance which is hers also, and which she gives to her own.

The demand, he said, is just.

In the first place, I said--and this is the first thing which you will
have to give back--the nature both of the just and unjust is truly known
to the gods.

Granted.

And if they are both known to them, one must be the friend and the other
the enemy of the gods, as we admitted from the beginning?

True.

And the friend of the gods may be supposed to receive from them all
things at their best, excepting only such evil as is the necessary
consequence of former sins?

Certainly.

Then this must be our notion of the just man, that even when he is in
poverty or sickness, or any other seeming misfortune, all things will
in the end work together for good to him in life and death: for the gods
have a care of any one whose desire is to become just and to be like
God, as far as man can attain the divine likeness, by the pursuit of
virtue?

Yes, he said; if he is like God he will surely not be neglected by him.

And of the unjust may not the opposite be supposed?

Certainly.

Such, then, are the palms of victory which the gods give the just?

That is my conviction.

And what do they receive of men? Look at things as they really are, and
you will see that the clever unjust are in the case of runners, who run
well from the starting-place to the goal but not back again from the
goal: they go off at a great pace, but in the end only look foolish,
slinking away with their ears draggling on their shoulders, and without
a crown; but the true runner comes to the finish and receives the prize
and is crowned. And this is the way with the just; he who endures to the
end of every action and occasion of his entire life has a good report
and carries off the prize which men have to bestow.

True.

And now you must allow me to repeat of the just the blessings which you
were attributing to the fortunate unjust. I shall say of them, what you
were saying of the others, that as they grow older, they become rulers
in their own city if they care to be; they marry whom they like and give
in marriage to whom they will; all that you said of the others I now say
of these. And, on the other hand, of the unjust I say that the greater
number, even though they escape in their youth, are found out at last
and look foolish at the end of their course, and when they come to be
old and miserable are flouted alike by stranger and citizen; they are
beaten and then come those things unfit for ears polite, as you truly
term them; they will be racked and have their eyes burned out, as you
were saying. And you may suppose that I have repeated the remainder of
your tale of horrors. But will you let me assume, without reciting them,
that these things are true?

Certainly, he said, what you say is true.

These, then, are the prizes and rewards and gifts which are bestowed
upon the just by gods and men in this present life, in addition to the
other good things which justice of herself provides.

Yes, he said; and they are fair and lasting.

And yet, I said, all these are as nothing either in number or greatness
in comparison with those other recompenses which await both just and
unjust after death. And you ought to hear them, and then both just and
unjust will have received from us a full payment of the debt which the
argument owes to them.

Speak, he said; there are few things which I would more gladly hear.

Well, I said, I will tell you a tale; not one of the tales which
Odysseus tells to the hero Alcinous, yet this too is a tale of a hero,
Er the son of Armenius, a Pamphylian by birth. He was slain in battle,
and ten days afterwards, when the bodies of the dead were taken up
already in a state of corruption, his body was found unaffected by
decay, and carried away home to be buried. And on the twelfth day, as he
was lying on the funeral pile, he returned to life and told them what he
had seen in the other world. He said that when his soul left the body
he went on a journey with a great company, and that they came to a
mysterious place at which there were two openings in the earth; they
were near together, and over against them were two other openings in the
heaven above. In the intermediate space there were judges seated, who
commanded the just, after they had given judgment on them and had bound
their sentences in front of them, to ascend by the heavenly way on the
right hand; and in like manner the unjust were bidden by them to descend
by the lower way on the left hand; these also bore the symbols of their
deeds, but fastened on their backs. He drew near, and they told him that
he was to be the messenger who would carry the report of the other world
to men, and they bade him hear and see all that was to be heard and seen
in that place. Then he beheld and saw on one side the souls departing at
either opening of heaven and earth when sentence had been given on them;
and at the two other openings other souls, some ascending out of the
earth dusty and worn with travel, some descending out of heaven clean
and bright. And arriving ever and anon they seemed to have come from a
long journey, and they went forth with gladness into the meadow, where
they encamped as at a festival; and those who knew one another embraced
and conversed, the souls which came from earth curiously enquiring about
the things above, and the souls which came from heaven about the things
beneath. And they told one another of what had happened by the way,
those from below weeping and sorrowing at the remembrance of the things
which they had endured and seen in their journey beneath the earth
(now the journey lasted a thousand years), while those from above were
describing heavenly delights and visions of inconceivable beauty. The
story, Glaucon, would take too long to tell; but the sum was this:--He
said that for every wrong which they had done to any one they suffered
tenfold; or once in a hundred years--such being reckoned to be the
length of man's life, and the penalty being thus paid ten times in a
thousand years. If, for example, there were any who had been the cause
of many deaths, or had betrayed or enslaved cities or armies, or been
guilty of any other evil behaviour, for each and all of their offences
they received punishment ten times over, and the rewards of beneficence
and justice and holiness were in the same proportion. I need hardly
repeat what he said concerning young children dying almost as soon
as they were born. Of piety and impiety to gods and parents, and of
murderers, there were retributions other and greater far which he
described. He mentioned that he was present when one of the spirits
asked another, ``Where is Ardiaeus the Great?'' (Now this Ardiaeus lived
a thousand years before the time of Er: he had been the tyrant of
some city of Pamphylia, and had murdered his aged father and his elder
brother, and was said to have committed many other abominable crimes.)
The answer of the other spirit was: ``He comes not hither and will never
come.'' And this, said he, was one of the dreadful sights which we
ourselves witnessed. We were at the mouth of the cavern, and, having
completed all our experiences, were about to reascend, when of a sudden
Ardiaeus appeared and several others, most of whom were tyrants; and
there were also besides the tyrants private individuals who had been
great criminals: they were just, as they fancied, about to return into
the upper world, but the mouth, instead of admitting them, gave a roar,
whenever any of these incurable sinners or some one who had not been
sufficiently punished tried to ascend; and then wild men of fiery
aspect, who were standing by and heard the sound, seized and carried
them off; and Ardiaeus and others they bound head and foot and hand, and
threw them down and flayed them with scourges, and dragged them along
the road at the side, carding them on thorns like wool, and declaring
to the passers-by what were their crimes, and that they were being taken
away to be cast into hell. And of all the many terrors which they had
endured, he said that there was none like the terror which each of them
felt at that moment, lest they should hear the voice; and when there was
silence, one by one they ascended with exceeding joy. These, said Er,
were the penalties and retributions, and there were blessings as great.

Now when the spirits which were in the meadow had tarried seven days,
on the eighth they were obliged to proceed on their journey, and, on the
fourth day after, he said that they came to a place where they could
see from above a line of light, straight as a column, extending right
through the whole heaven and through the earth, in colour resembling the
rainbow, only brighter and purer; another day's journey brought them to
the place, and there, in the midst of the light, they saw the ends of
the chains of heaven let down from above: for this light is the belt
of heaven, and holds together the circle of the universe, like the
under-girders of a trireme. From these ends is extended the spindle of
Necessity, on which all the revolutions turn. The shaft and hook of this
spindle are made of steel, and the whorl is made partly of steel and
also partly of other materials. Now the whorl is in form like the whorl
used on earth; and the description of it implied that there is one large
hollow whorl which is quite scooped out, and into this is fitted another
lesser one, and another, and another, and four others, making eight
in all, like vessels which fit into one another; the whorls show their
edges on the upper side, and on their lower side all together form one
continuous whorl. This is pierced by the spindle, which is driven home
through the centre of the eighth. The first and outermost whorl has the
rim broadest, and the seven inner whorls are narrower, in the following
proportions--the sixth is next to the first in size, the fourth next
to the sixth; then comes the eighth; the seventh is fifth, the fifth
is sixth, the third is seventh, last and eighth comes the second.
The largest (or fixed stars) is spangled, and the seventh (or sun) is
brightest; the eighth (or moon) coloured by the reflected light of the
seventh; the second and fifth (Saturn and Mercury) are in colour like
one another, and yellower than the preceding; the third (Venus) has the
whitest light; the fourth (Mars) is reddish; the sixth (Jupiter) is in
whiteness second. Now the whole spindle has the same motion; but, as the
whole revolves in one direction, the seven inner circles move slowly in
the other, and of these the swiftest is the eighth; next in swiftness
are the seventh, sixth, and fifth, which move together; third in
swiftness appeared to move according to the law of this reversed motion
the fourth; the third appeared fourth and the second fifth. The spindle
turns on the knees of Necessity; and on the upper surface of each circle
is a siren, who goes round with them, hymning a single tone or note. The
eight together form one harmony; and round about, at equal intervals,
there is another band, three in number, each sitting upon her throne:
these are the Fates, daughters of Necessity, who are clothed in white
robes and have chaplets upon their heads, Lachesis and Clotho
and Atropos, who accompany with their voices the harmony of the
sirens--Lachesis singing of the past, Clotho of the present, Atropos of
the future; Clotho from time to time assisting with a touch of her right
hand the revolution of the outer circle of the whorl or spindle, and
Atropos with her left hand touching and guiding the inner ones, and
Lachesis laying hold of either in turn, first with one hand and then
with the other.

When Er and the spirits arrived, their duty was to go at once to
Lachesis; but first of all there came a prophet who arranged them in
order; then he took from the knees of Lachesis lots and samples of
lives, and having mounted a high pulpit, spoke as follows: ``Hear the
word of Lachesis, the daughter of Necessity. Mortal souls, behold a new
cycle of life and mortality. Your genius will not be allotted to you,
but you will choose your genius; and let him who draws the first lot
have the first choice, and the life which he chooses shall be his
destiny. Virtue is free, and as a man honours or dishonours her he will
have more or less of her; the responsibility is with the chooser--God
is justified.'' When the Interpreter had thus spoken he scattered lots
indifferently among them all, and each of them took up the lot which
fell near him, all but Er himself (he was not allowed), and each as
he took his lot perceived the number which he had obtained. Then the
Interpreter placed on the ground before them the samples of lives; and
there were many more lives than the souls present, and they were of all
sorts. There were lives of every animal and of man in every condition.
And there were tyrannies among them, some lasting out the tyrant's life,
others which broke off in the middle and came to an end in poverty and
exile and beggary; and there were lives of famous men, some who were
famous for their form and beauty as well as for their strength and
success in games, or, again, for their birth and the qualities of their
ancestors; and some who were the reverse of famous for the opposite
qualities. And of women likewise; there was not, however, any definite
character in them, because the soul, when choosing a new life, must of
necessity become different. But there was every other quality, and
the all mingled with one another, and also with elements of wealth and
poverty, and disease and health; and there were mean states also. And
here, my dear Glaucon, is the supreme peril of our human state; and
therefore the utmost care should be taken. Let each one of us leave
every other kind of knowledge and seek and follow one thing only, if
peradventure he may be able to learn and may find some one who will make
him able to learn and discern between good and evil, and so to choose
always and everywhere the better life as he has opportunity. He should
consider the bearing of all these things which have been mentioned
severally and collectively upon virtue; he should know what the effect
of beauty is when combined with poverty or wealth in a particular soul,
and what are the good and evil consequences of noble and humble birth,
of private and public station, of strength and weakness, of cleverness
and dullness, and of all the natural and acquired gifts of the soul, and
the operation of them when conjoined; he will then look at the nature of
the soul, and from the consideration of all these qualities he will be
able to determine which is the better and which is the worse; and so
he will choose, giving the name of evil to the life which will make his
soul more unjust, and good to the life which will make his soul more
just; all else he will disregard. For we have seen and know that this is
the best choice both in life and after death. A man must take with him
into the world below an adamantine faith in truth and right, that there
too he may be undazzled by the desire of wealth or the other allurements
of evil, lest, coming upon tyrannies and similar villainies, he do
irremediable wrongs to others and suffer yet worse himself; but let him
know how to choose the mean and avoid the extremes on either side, as
far as possible, not only in this life but in all that which is to come.
For this is the way of happiness.

And according to the report of the messenger from the other world this
was what the prophet said at the time: ``Even for the last comer, if he
chooses wisely and will live diligently, there is appointed a happy and
not undesirable existence. Let not him who chooses first be careless,
and let not the last despair.'' And when he had spoken, he who had the
first choice came forward and in a moment chose the greatest tyranny;
his mind having been darkened by folly and sensuality, he had not
thought out the whole matter before he chose, and did not at first
sight perceive that he was fated, among other evils, to devour his own
children. But when he had time to reflect, and saw what was in the lot,
he began to beat his breast and lament over his choice, forgetting the
proclamation of the prophet; for, instead of throwing the blame of his
misfortune on himself, he accused chance and the gods, and everything
rather than himself. Now he was one of those who came from heaven, and
in a former life had dwelt in a well-ordered State, but his virtue was
a matter of habit only, and he had no philosophy. And it was true of
others who were similarly overtaken, that the greater number of them
came from heaven and therefore they had never been schooled by trial,
whereas the pilgrims who came from earth having themselves suffered and
seen others suffer, were not in a hurry to choose. And owing to this
inexperience of theirs, and also because the lot was a chance, many of
the souls exchanged a good destiny for an evil or an evil for a good.
For if a man had always on his arrival in this world dedicated himself
from the first to sound philosophy, and had been moderately fortunate
in the number of the lot, he might, as the messenger reported, be happy
here, and also his journey to another life and return to this, instead
of being rough and underground, would be smooth and heavenly. Most
curious, he said, was the spectacle--sad and laughable and strange; for
the choice of the souls was in most cases based on their experience of
a previous life. There he saw the soul which had once been Orpheus
choosing the life of a swan out of enmity to the race of women, hating
to be born of a woman because they had been his murderers; he beheld
also the soul of Thamyras choosing the life of a nightingale; birds, on
the other hand, like the swan and other musicians, wanting to be men.
The soul which obtained the twentieth lot chose the life of a lion, and
this was the soul of Ajax the son of Telamon, who would not be a man,
remembering the injustice which was done him in the judgment about the
arms. The next was Agamemnon, who took the life of an eagle, because,
like Ajax, he hated human nature by reason of his sufferings. About
the middle came the lot of Atalanta; she, seeing the great fame of
an athlete, was unable to resist the temptation: and after her there
followed the soul of Epeus the son of Panopeus passing into the nature
of a woman cunning in the arts; and far away among the last who chose,
the soul of the jester Thersites was putting on the form of a monkey.
There came also the soul of Odysseus having yet to make a choice, and
his lot happened to be the last of them all. Now the recollection of
former toils had disenchanted him of ambition, and he went about for
a considerable time in search of the life of a private man who had no
cares; he had some difficulty in finding this, which was lying about and
had been neglected by everybody else; and when he saw it, he said that
he would have done the same had his lot been first instead of last,
and that he was delighted to have it. And not only did men pass into
animals, but I must also mention that there were animals tame and wild
who changed into one another and into corresponding human natures--the
good into the gentle and the evil into the savage, in all sorts of
combinations.

All the souls had now chosen their lives, and they went in the order of
their choice to Lachesis, who sent with them the genius whom they had
severally chosen, to be the guardian of their lives and the fulfiller
of the choice: this genius led the souls first to Clotho, and drew
them within the revolution of the spindle impelled by her hand, thus
ratifying the destiny of each; and then, when they were fastened to
this, carried them to Atropos, who spun the threads and made them
irreversible, whence without turning round they passed beneath the
throne of Necessity; and when they had all passed, they marched on in a
scorching heat to the plain of Forgetfulness, which was a barren waste
destitute of trees and verdure; and then towards evening they encamped
by the river of Unmindfulness, whose water no vessel can hold; of this
they were all obliged to drink a certain quantity, and those who were
not saved by wisdom drank more than was necessary; and each one as he
drank forgot all things. Now after they had gone to rest, about the
middle of the night there was a thunderstorm and earthquake, and then
in an instant they were driven upwards in all manner of ways to their
birth, like stars shooting. He himself was hindered from drinking the
water. But in what manner or by what means he returned to the body he
could not say; only, in the morning, awaking suddenly, he found himself
lying on the pyre.

And thus, Glaucon, the tale has been saved and has not perished, and
will save us if we are obedient to the word spoken; and we shall pass
safely over the river of Forgetfulness and our soul will not be defiled.
Wherefore my counsel is, that we hold fast ever to the heavenly way and
follow after justice and virtue always, considering that the soul is
immortal and able to endure every sort of good and every sort of evil.
Thus shall we live dear to one another and to the gods, both while
remaining here and when, like conquerors in the games who go round to
gather gifts, we receive our reward. And it shall be well with us both
in this life and in the pilgrimage of a thousand years which we have
been describing.

% section book_x (end)


% chapter republic (end)
% \chapter{Parmenides} % (fold)
\label{cha:parmenides}


PARMENIDES

By Plato


Translated by Benjamin Jowett




INTRODUCTION AND ANALYSIS.

The awe with which Plato regarded the character of 'the great'
Parmenides has extended to the dialogue which he calls by his name. None
of the writings of Plato have been more copiously illustrated, both in
ancient and modern times, and in none of them have the interpreters
been more at variance with one another. Nor is this surprising. For the
Parmenides is more fragmentary and isolated than any other dialogue, and
the design of the writer is not expressly stated. The date is uncertain;
the relation to the other writings of Plato is also uncertain; the
connexion between the two parts is at first sight extremely obscure;
and in the latter of the two we are left in doubt as to whether Plato is
speaking his own sentiments by the lips of Parmenides, and overthrowing
him out of his own mouth, or whether he is propounding consequences
which would have been admitted by Zeno and Parmenides themselves. The
contradictions which follow from the hypotheses of the one and many have
been regarded by some as transcendental mysteries; by others as a mere
illustration, taken at random, of a new method. They seem to have been
inspired by a sort of dialectical frenzy, such as may be supposed to
have prevailed in the Megarian School (compare Cratylus, etc.). The
criticism on his own doctrine of Ideas has also been considered, not as
a real criticism, but as an exuberance of the metaphysical imagination
which enabled Plato to go beyond himself. To the latter part of the
dialogue we may certainly apply the words in which he himself describes
the earlier philosophers in the Sophist: 'They went on their way rather
regardless of whether we understood them or not.'

The Parmenides in point of style is one of the best of the Platonic
writings; the first portion of the dialogue is in no way defective in
ease and grace and dramatic interest; nor in the second part, where
there was no room for such qualities, is there any want of clearness or
precision. The latter half is an exquisite mosaic, of which the small
pieces are with the utmost fineness and regularity adapted to one
another. Like the Protagoras, Phaedo, and others, the whole is a
narrated dialogue, combining with the mere recital of the words spoken,
the observations of the reciter on the effect produced by them. Thus we
are informed by him that Zeno and Parmenides were not altogether pleased
at the request of Socrates that they would examine into the nature of
the one and many in the sphere of Ideas, although they received his
suggestion with approving smiles. And we are glad to be told that
Parmenides was 'aged but well-favoured,' and that Zeno was 'very
good-looking'; also that Parmenides affected to decline the great
argument, on which, as Zeno knew from experience, he was not unwilling
to enter. The character of Antiphon, the half-brother of Plato, who
had once been inclined to philosophy, but has now shown the hereditary
disposition for horses, is very naturally described. He is the sole
depositary of the famous dialogue; but, although he receives the
strangers like a courteous gentleman, he is impatient of the trouble of
reciting it. As they enter, he has been giving orders to a bridle-maker;
by this slight touch Plato verifies the previous description of him.
After a little persuasion he is induced to favour the Clazomenians, who
come from a distance, with a rehearsal. Respecting the visit of Zeno
and Parmenides to Athens, we may observe--first, that such a visit is
consistent with dates, and may possibly have occurred; secondly, that
Plato is very likely to have invented the meeting ('You, Socrates, can
easily invent Egyptian tales or anything else,' Phaedrus); thirdly, that
no reliance can be placed on the circumstance as determining the date
of Parmenides and Zeno; fourthly, that the same occasion appears to be
referred to by Plato in two other places (Theaet., Soph.).

Many interpreters have regarded the Parmenides as a 'reductio ad
absurdum' of the Eleatic philosophy. But would Plato have been likely to
place this in the mouth of the great Parmenides himself, who appeared
to him, in Homeric language, to be 'venerable and awful,' and to have
a 'glorious depth of mind'? (Theaet.). It may be admitted that he has
ascribed to an Eleatic stranger in the Sophist opinions which went
beyond the doctrines of the Eleatics. But the Eleatic stranger expressly
criticises the doctrines in which he had been brought up; he admits that
he is going to 'lay hands on his father Parmenides.' Nothing of this
kind is said of Zeno and Parmenides. How then, without a word of
explanation, could Plato assign to them the refutation of their own
tenets?

The conclusion at which we must arrive is that the Parmenides is not
a refutation of the Eleatic philosophy. Nor would such an explanation
afford any satisfactory connexion of the first and second parts of the
dialogue. And it is quite inconsistent with Plato's own relation to the
Eleatics. For of all the pre-Socratic philosophers, he speaks of them
with the greatest respect. But he could hardly have passed upon them a
more unmeaning slight than to ascribe to their great master tenets the
reverse of those which he actually held.

Two preliminary remarks may be made. First, that whatever latitude we
may allow to Plato in bringing together by a 'tour de force,' as in the
Phaedrus, dissimilar themes, yet he always in some way seeks to find
a connexion for them. Many threads join together in one the love and
dialectic of the Phaedrus. We cannot conceive that the great artist
would place in juxtaposition two absolutely divided and incoherent
subjects. And hence we are led to make a second remark: viz. that
no explanation of the Parmenides can be satisfactory which does not
indicate the connexion of the first and second parts. To suppose that
Plato would first go out of his way to make Parmenides attack the
Platonic Ideas, and then proceed to a similar but more fatal assault on
his own doctrine of Being, appears to be the height of absurdity.

Perhaps there is no passage in Plato showing greater metaphysical power
than that in which he assails his own theory of Ideas. The arguments are
nearly, if not quite, those of Aristotle; they are the objections which
naturally occur to a modern student of philosophy. Many persons will be
surprised to find Plato criticizing the very conceptions which have been
supposed in after ages to be peculiarly characteristic of him. How can
he have placed himself so completely without them? How can he have ever
persisted in them after seeing the fatal objections which might be urged
against them? The consideration of this difficulty has led a recent
critic (Ueberweg), who in general accepts the authorised canon of the
Platonic writings, to condemn the Parmenides as spurious. The accidental
want of external evidence, at first sight, seems to favour this opinion.

In answer, it might be sufficient to say, that no ancient writing of
equal length and excellence is known to be spurious. Nor is the silence
of Aristotle to be hastily assumed; there is at least a doubt whether
his use of the same arguments does not involve the inference that he
knew the work. And, if the Parmenides is spurious, like Ueberweg, we
are led on further than we originally intended, to pass a similar
condemnation on the Theaetetus and Sophist, and therefore on the
Politicus (compare Theaet., Soph.). But the objection is in reality
fanciful, and rests on the assumption that the doctrine of the Ideas was
held by Plato throughout his life in the same form. For the truth
is, that the Platonic Ideas were in constant process of growth and
transmutation; sometimes veiled in poetry and mythology, then again
emerging as fixed Ideas, in some passages regarded as absolute and
eternal, and in others as relative to the human mind, existing in
and derived from external objects as well as transcending them. The
anamnesis of the Ideas is chiefly insisted upon in the mythical portions
of the dialogues, and really occupies a very small space in the entire
works of Plato. Their transcendental existence is not asserted, and
is therefore implicitly denied in the Philebus; different forms
are ascribed to them in the Republic, and they are mentioned in the
Theaetetus, the Sophist, the Politicus, and the Laws, much as Universals
would be spoken of in modern books. Indeed, there are very faint traces
of the transcendental doctrine of Ideas, that is, of their existence
apart from the mind, in any of Plato's writings, with the exception of
the Meno, the Phaedrus, the Phaedo, and in portions of the Republic. The
stereotyped form which Aristotle has given to them is not found in Plato
(compare Essay on the Platonic Ideas in the Introduction to the Meno.)

The full discussion of this subject involves a comprehensive survey of
the philosophy of Plato, which would be out of place here. But, without
digressing further from the immediate subject of the Parmenides, we
may remark that Plato is quite serious in his objections to his own
doctrines: nor does Socrates attempt to offer any answer to them. The
perplexities which surround the one and many in the sphere of the Ideas
are also alluded to in the Philebus, and no answer is given to them. Nor
have they ever been answered, nor can they be answered by any one else
who separates the phenomenal from the real. To suppose that Plato, at a
later period of his life, reached a point of view from which he was able
to answer them, is a groundless assumption. The real progress of Plato's
own mind has been partly concealed from us by the dogmatic statements of
Aristotle, and also by the degeneracy of his own followers, with whom a
doctrine of numbers quickly superseded Ideas.

As a preparation for answering some of the difficulties which have
been suggested, we may begin by sketching the first portion of the
dialogue:--

Cephalus, of Clazomenae in Ionia, the birthplace of Anaxagoras, a
citizen of no mean city in the history of philosophy, who is the
narrator of the dialogue, describes himself as meeting Adeimantus and
Glaucon in the Agora at Athens. 'Welcome, Cephalus: can we do anything
for you in Athens?' 'Why, yes: I came to ask a favour of you. First,
tell me your half-brother's name, which I have forgotten--he was a mere
child when I was last here;--I know his father's, which is Pyrilampes.'
'Yes, and the name of our brother is Antiphon. But why do you ask?'
'Let me introduce to you some countrymen of mine, who are lovers of
philosophy; they have heard that Antiphon remembers a conversation of
Socrates with Parmenides and Zeno, of which the report came to him from
Pythodorus, Zeno's friend.' 'That is quite true.' 'And can they hear the
dialogue?' 'Nothing easier; in the days of his youth he made a careful
study of the piece; at present, his thoughts have another direction: he
takes after his grandfather, and has given up philosophy for horses.'

'We went to look for him, and found him giving instructions to a worker
in brass about a bridle. When he had done with him, and had learned
from his brothers the purpose of our visit, he saluted me as an old
acquaintance, and we asked him to repeat the dialogue. At first, he
complained of the trouble, but he soon consented. He told us that
Pythodorus had described to him the appearance of Parmenides and Zeno;
they had come to Athens at the great Panathenaea, the former being at
the time about sixty-five years old, aged but well-favoured--Zeno, who
was said to have been beloved of Parmenides in the days of his youth,
about forty, and very good-looking:--that they lodged with Pythodorus at
the Ceramicus outside the wall, whither Socrates, then a very young
man, came to see them: Zeno was reading one of his theses, which he
had nearly finished, when Pythodorus entered with Parmenides and
Aristoteles, who was afterwards one of the Thirty. When the recitation
was completed, Socrates requested that the first thesis of the treatise
might be read again.'

'You mean, Zeno,' said Socrates, 'to argue that being, if it is many,
must be both like and unlike, which is a contradiction; and each
division of your argument is intended to elicit a similar absurdity,
which may be supposed to follow from the assumption that being is many.'
'Such is my meaning.' 'I see,' said Socrates, turning to Parmenides,
'that Zeno is your second self in his writings too; you prove admirably
that the all is one: he gives proofs no less convincing that the many
are nought. To deceive the world by saying the same thing in entirely
different forms, is a strain of art beyond most of us.' 'Yes, Socrates,'
said Zeno; 'but though you are as keen as a Spartan hound, you do not
quite catch the motive of the piece, which was only intended to protect
Parmenides against ridicule by showing that the hypothesis of the
existence of the many involved greater absurdities than the hypothesis
of the one. The book was a youthful composition of mine, which was
stolen from me, and therefore I had no choice about the publication.' 'I
quite believe you,' said Socrates; 'but will you answer me a question? I
should like to know, whether you would assume an idea of likeness in the
abstract, which is the contradictory of unlikeness in the abstract, by
participation in either or both of which things are like or unlike
or partly both. For the same things may very well partake of like and
unlike in the concrete, though like and unlike in the abstract are
irreconcilable. Nor does there appear to me to be any absurdity in
maintaining that the same things may partake of the one and many, though
I should be indeed surprised to hear that the absolute one is also
many. For example, I, being many, that is to say, having many parts or
members, am yet also one, and partake of the one, being one of seven
who are here present (compare Philebus). This is not an absurdity, but
a truism. But I should be amazed if there were a similar entanglement in
the nature of the ideas themselves, nor can I believe that one and many,
like and unlike, rest and motion, in the abstract, are capable either of
admixture or of separation.'

Pythodorus said that in his opinion Parmenides and Zeno were not very
well pleased at the questions which were raised; nevertheless, they
looked at one another and smiled in seeming delight and admiration of
Socrates. 'Tell me,' said Parmenides, 'do you think that the abstract
ideas of likeness, unity, and the rest, exist apart from individuals
which partake of them? and is this your own distinction?' 'I think that
there are such ideas.' 'And would you make abstract ideas of the just,
the beautiful, the good?' 'Yes,' he said. 'And of human beings like
ourselves, of water, fire, and the like?' 'I am not certain.' 'And would
you be undecided also about ideas of which the mention will, perhaps,
appear laughable: of hair, mud, filth, and other things which are base
and vile?' 'No, Parmenides; visible things like these are, as I believe,
only what they appear to be: though I am sometimes disposed to imagine
that there is nothing without an idea; but I repress any such notion,
from a fear of falling into an abyss of nonsense.' 'You are young,
Socrates, and therefore naturally regard the opinions of men; the time
will come when philosophy will have a firmer hold of you, and you will
not despise even the meanest things. But tell me, is your meaning that
things become like by partaking of likeness, great by partaking of
greatness, just and beautiful by partaking of justice and beauty, and
so of other ideas?' 'Yes, that is my meaning.' 'And do you suppose the
individual to partake of the whole, or of the part?' 'Why not of the
whole?' said Socrates. 'Because,' said Parmenides, 'in that case the
whole, which is one, will become many.' 'Nay,' said Socrates, 'the whole
may be like the day, which is one and in many places: in this way
the ideas may be one and also many.' 'In the same sort of way,' said
Parmenides, 'as a sail, which is one, may be a cover to many--that is
your meaning?' 'Yes.' 'And would you say that each man is covered by the
whole sail, or by a part only?' 'By a part.' 'Then the ideas have parts,
and the objects partake of a part of them only?' 'That seems to follow.'
'And would you like to say that the ideas are really divisible and yet
remain one?' 'Certainly not.' 'Would you venture to affirm that great
objects have a portion only of greatness transferred to them; or that
small or equal objects are small or equal because they are only portions
of smallness or equality?' 'Impossible.' 'But how can individuals
participate in ideas, except in the ways which I have mentioned?' 'That
is not an easy question to answer.' 'I should imagine the conception of
ideas to arise as follows: you see great objects pervaded by a common
form or idea of greatness, which you abstract.' 'That is quite true.'
'And supposing you embrace in one view the idea of greatness thus gained
and the individuals which it comprises, a further idea of greatness
arises, which makes both great; and this may go on to infinity.'
Socrates replies that the ideas may be thoughts in the mind only; in
this case, the consequence would no longer follow. 'But must not the
thought be of something which is the same in all and is the idea? And
if the world partakes in the ideas, and the ideas are thoughts, must not
all things think? Or can thought be without thought?' 'I acknowledge the
unmeaningness of this,' says Socrates, 'and would rather have recourse
to the explanation that the ideas are types in nature, and that other
things partake of them by becoming like them.' 'But to become like them
is to be comprehended in the same idea; and the likeness of the idea and
the individuals implies another idea of likeness, and another without
end.' 'Quite true.' 'The theory, then, of participation by likeness
has to be given up. You have hardly yet, Socrates, found out the real
difficulty of maintaining abstract ideas.' 'What difficulty?' 'The
greatest of all perhaps is this: an opponent will argue that the ideas
are not within the range of human knowledge; and you cannot disprove the
assertion without a long and laborious demonstration, which he may be
unable or unwilling to follow. In the first place, neither you nor any
one who maintains the existence of absolute ideas will affirm that they
are subjective.' 'That would be a contradiction.' 'True; and therefore
any relation in these ideas is a relation which concerns themselves
only; and the objects which are named after them, are relative to one
another only, and have nothing to do with the ideas themselves.' 'How do
you mean?' said Socrates. 'I may illustrate my meaning in this way: one
of us has a slave; and the idea of a slave in the abstract is relative
to the idea of a master in the abstract; this correspondence of ideas,
however, has nothing to do with the particular relation of our slave to
us.--Do you see my meaning?' 'Perfectly.' 'And absolute knowledge in
the same way corresponds to absolute truth and being, and particular
knowledge to particular truth and being.' Clearly.' 'And there is a
subjective knowledge which is of subjective truth, having many kinds,
general and particular. But the ideas themselves are not subjective, and
therefore are not within our ken.' 'They are not.' 'Then the beautiful
and the good in their own nature are unknown to us?' 'It would seem so.'
'There is a worse consequence yet.' 'What is that?' 'I think we must
admit that absolute knowledge is the most exact knowledge, which we must
therefore attribute to God. But then see what follows: God, having
this exact knowledge, can have no knowledge of human things, as we
have divided the two spheres, and forbidden any passing from one to the
other:--the gods have knowledge and authority in their world only, as
we have in ours.' 'Yet, surely, to deprive God of knowledge is
monstrous.'--'These are some of the difficulties which are involved
in the assumption of absolute ideas; the learner will find them nearly
impossible to understand, and the teacher who has to impart them will
require superhuman ability; there will always be a suspicion, either
that they have no existence, or are beyond human knowledge.' 'There I
agree with you,' said Socrates. 'Yet if these difficulties induce you
to give up universal ideas, what becomes of the mind? and where are the
reasoning and reflecting powers? philosophy is at an end.' 'I certainly
do not see my way.' 'I think,' said Parmenides, 'that this arises out
of your attempting to define abstractions, such as the good and
the beautiful and the just, before you have had sufficient previous
training; I noticed your deficiency when you were talking with
Aristoteles, the day before yesterday. Your enthusiasm is a wonderful
gift; but I fear that unless you discipline yourself by dialectic
while you are young, truth will elude your grasp.' 'And what kind of
discipline would you recommend?' 'The training which you heard Zeno
practising; at the same time, I admire your saying to him that you did
not care to consider the difficulty in reference to visible objects,
but only in relation to ideas.' 'Yes; because I think that in visible
objects you may easily show any number of inconsistent consequences.'
'Yes; and you should consider, not only the consequences which follow
from a given hypothesis, but the consequences also which follow from the
denial of the hypothesis. For example, what follows from the assumption
of the existence of the many, and the counter-argument of what follows
from the denial of the existence of the many: and similarly of likeness
and unlikeness, motion, rest, generation, corruption, being and not
being. And the consequences must include consequences to the things
supposed and to other things, in themselves and in relation to one
another, to individuals whom you select, to the many, and to the all;
these must be drawn out both on the affirmative and on the negative
hypothesis,--that is, if you are to train yourself perfectly to the
intelligence of the truth.' 'What you are suggesting seems to be a
tremendous process, and one of which I do not quite understand the
nature,' said Socrates; 'will you give me an example?' 'You must not
impose such a task on a man of my years,' said Parmenides. 'Then will
you, Zeno?' 'Let us rather,' said Zeno, with a smile, 'ask Parmenides,
for the undertaking is a serious one, as he truly says; nor could I urge
him to make the attempt, except in a select audience of persons who will
understand him.' The whole party joined in the request.

Here we have, first of all, an unmistakable attack made by the youthful
Socrates on the paradoxes of Zeno. He perfectly understands their drift,
and Zeno himself is supposed to admit this. But they appear to him, as
he says in the Philebus also, to be rather truisms than paradoxes. For
every one must acknowledge the obvious fact, that the body being one
has many members, and that, in a thousand ways, the like partakes of
the unlike, the many of the one. The real difficulty begins with the
relations of ideas in themselves, whether of the one and many, or of
any other ideas, to one another and to the mind. But this was a problem
which the Eleatic philosophers had never considered; their thoughts had
not gone beyond the contradictions of matter, motion, space, and the
like.

It was no wonder that Parmenides and Zeno should hear the novel
speculations of Socrates with mixed feelings of admiration and
displeasure. He was going out of the received circle of disputation into
a region in which they could hardly follow him. From the crude idea of
Being in the abstract, he was about to proceed to universals or general
notions. There is no contradiction in material things partaking of the
ideas of one and many; neither is there any contradiction in the ideas
of one and many, like and unlike, in themselves. But the contradiction
arises when we attempt to conceive ideas in their connexion, or to
ascertain their relation to phenomena. Still he affirms the existence of
such ideas; and this is the position which is now in turn submitted to
the criticisms of Parmenides.

To appreciate truly the character of these criticisms, we must remember
the place held by Parmenides in the history of Greek philosophy. He
is the founder of idealism, and also of dialectic, or, in modern
phraseology, of metaphysics and logic (Theaet., Soph.). Like Plato,
he is struggling after something wider and deeper than satisfied the
contemporary Pythagoreans. And Plato with a true instinct recognizes
him as his spiritual father, whom he 'revered and honoured more than all
other philosophers together.' He may be supposed to have thought more
than he said, or was able to express. And, although he could not, as
a matter of fact, have criticized the ideas of Plato without an
anachronism, the criticism is appropriately placed in the mouth of the
founder of the ideal philosophy.

There was probably a time in the life of Plato when the ethical teaching
of Socrates came into conflict with the metaphysical theories of the
earlier philosophers, and he sought to supplement the one by the other.
The older philosophers were great and awful; and they had the charm of
antiquity. Something which found a response in his own mind seemed to
have been lost as well as gained in the Socratic dialectic. He felt no
incongruity in the veteran Parmenides correcting the youthful Socrates.
Two points in his criticism are especially deserving of notice. First
of all, Parmenides tries him by the test of consistency. Socrates is
willing to assume ideas or principles of the just, the beautiful, the
good, and to extend them to man (compare Phaedo); but he is reluctant to
admit that there are general ideas of hair, mud, filth, etc. There is an
ethical universal or idea, but is there also a universal of physics?--of
the meanest things in the world as well as of the greatest? Parmenides
rebukes this want of consistency in Socrates, which he attributes to his
youth. As he grows older, philosophy will take a firmer hold of him, and
then he will despise neither great things nor small, and he will think
less of the opinions of mankind (compare Soph.). Here is lightly touched
one of the most familiar principles of modern philosophy, that in the
meanest operations of nature, as well as in the noblest, in mud and
filth, as well as in the sun and stars, great truths are contained. At
the same time, we may note also the transition in the mind of Plato,
to which Aristotle alludes (Met.), when, as he says, he transferred the
Socratic universal of ethics to the whole of nature.

The other criticism of Parmenides on Socrates attributes to him a want
of practice in dialectic. He has observed this deficiency in him when
talking to Aristoteles on a previous occasion. Plato seems to imply
that there was something more in the dialectic of Zeno than in the mere
interrogation of Socrates. Here, again, he may perhaps be describing
the process which his own mind went through when he first became more
intimately acquainted, whether at Megara or elsewhere, with the Eleatic
and Megarian philosophers. Still, Parmenides does not deny to Socrates
the credit of having gone beyond them in seeking to apply the paradoxes
of Zeno to ideas; and this is the application which he himself makes of
them in the latter part of the dialogue. He then proceeds to explain
to him the sort of mental gymnastic which he should practise. He should
consider not only what would follow from a given hypothesis, but what
would follow from the denial of it, to that which is the subject of
the hypothesis, and to all other things. There is no trace in the
Memorabilia of Xenophon of any such method being attributed to
Socrates; nor is the dialectic here spoken of that 'favourite method' of
proceeding by regular divisions, which is described in the Phaedrus and
Philebus, and of which examples are given in the Politicus and in the
Sophist. It is expressly spoken of as the method which Socrates had
heard Zeno practise in the days of his youth (compare Soph.).

The discussion of Socrates with Parmenides is one of the most remarkable
passages in Plato. Few writers have ever been able to anticipate 'the
criticism of the morrow' on their favourite notions. But Plato may here
be said to anticipate the judgment not only of the morrow, but of
all after-ages on the Platonic Ideas. For in some points he touches
questions which have not yet received their solution in modern
philosophy.

The first difficulty which Parmenides raises respecting the Platonic
ideas relates to the manner in which individuals are connected with
them. Do they participate in the ideas, or do they merely resemble them?
Parmenides shows that objections may be urged against either of these
modes of conceiving the connection. Things are little by partaking of
littleness, great by partaking of greatness, and the like. But they
cannot partake of a part of greatness, for that will not make them
great, etc.; nor can each object monopolise the whole. The only answer
to this is, that 'partaking' is a figure of speech, really corresponding
to the processes which a later logic designates by the terms
'abstraction' and 'generalization.' When we have described accurately
the methods or forms which the mind employs, we cannot further criticize
them; at least we can only criticize them with reference to their
fitness as instruments of thought to express facts.

Socrates attempts to support his view of the ideas by the parallel of
the day, which is one and in many places; but he is easily driven from
his position by a counter illustration of Parmenides, who compares the
idea of greatness to a sail. He truly explains to Socrates that he has
attained the conception of ideas by a process of generalization. At
the same time, he points out a difficulty, which appears to be
involved--viz. that the process of generalization will go on to
infinity. Socrates meets the supposed difficulty by a flash of light,
which is indeed the true answer 'that the ideas are in our minds
only.' Neither realism is the truth, nor nominalism is the truth, but
conceptualism; and conceptualism or any other psychological theory falls
very far short of the infinite subtlety of language and thought.

But the realism of ancient philosophy will not admit of this answer,
which is repelled by Parmenides with another truth or half-truth of
later philosophy, 'Every subject or subjective must have an object.'
Here is the great though unconscious truth (shall we say?) or error,
which underlay the early Greek philosophy. 'Ideas must have a real
existence;' they are not mere forms or opinions, which may be changed
arbitrarily by individuals. But the early Greek philosopher never
clearly saw that true ideas were only universal facts, and that there
might be error in universals as well as in particulars.

Socrates makes one more attempt to defend the Platonic Ideas by
representing them as paradigms; this is again answered by the
'argumentum ad infinitum.' We may remark, in passing, that the process
which is thus described has no real existence. The mind, after having
obtained a general idea, does not really go on to form another which
includes that, and all the individuals contained under it, and another
and another without end. The difficulty belongs in fact to the Megarian
age of philosophy, and is due to their illogical logic, and to the
general ignorance of the ancients respecting the part played by language
in the process of thought. No such perplexity could ever trouble
a modern metaphysician, any more than the fallacy of 'calvus' or
'acervus,' or of 'Achilles and the tortoise.' These 'surds' of
metaphysics ought to occasion no more difficulty in speculation than a
perpetually recurring fraction in arithmetic.

It is otherwise with the objection which follows: How are we to bridge
the chasm between human truth and absolute truth, between gods and men?
This is the difficulty of philosophy in all ages: How can we get beyond
the circle of our own ideas, or how, remaining within them, can we have
any criterion of a truth beyond and independent of them? Parmenides
draws out this difficulty with great clearness. According to him, there
are not only one but two chasms: the first, between individuals and the
ideas which have a common name; the second, between the ideas in us and
the ideas absolute. The first of these two difficulties mankind, as
we may say, a little parodying the language of the Philebus, have long
agreed to treat as obsolete; the second remains a difficulty for us as
well as for the Greeks of the fourth century before Christ, and is the
stumbling-block of Kant's Kritik, and of the Hamiltonian adaptation
of Kant, as well as of the Platonic ideas. It has been said that 'you
cannot criticize Revelation.' 'Then how do you know what is Revelation,
or that there is one at all,' is the immediate rejoinder--'You know
nothing of things in themselves.' 'Then how do you know that there are
things in themselves?' In some respects, the difficulty pressed harder
upon the Greek than upon ourselves. For conceiving of God more under the
attribute of knowledge than we do, he was more under the necessity
of separating the divine from the human, as two spheres which had no
communication with one another.

It is remarkable that Plato, speaking by the mouth of Parmenides,
does not treat even this second class of difficulties as hopeless or
insoluble. He says only that they cannot be explained without a long and
laborious demonstration: 'The teacher will require superhuman ability,
and the learner will be hard of understanding.' But an attempt must be
made to find an answer to them; for, as Socrates and Parmenides both
admit, the denial of abstract ideas is the destruction of the mind. We
can easily imagine that among the Greek schools of philosophy in the
fourth century before Christ a panic might arise from the denial of
universals, similar to that which arose in the last century from Hume's
denial of our ideas of cause and effect. Men do not at first recognize
that thought, like digestion, will go on much the same, notwithstanding
any theories which may be entertained respecting the nature of the
process. Parmenides attributes the difficulties in which Socrates is
involved to a want of comprehensiveness in his mode of reasoning; he
should consider every question on the negative as well as the positive
hypothesis, with reference to the consequences which flow from the
denial as well as from the assertion of a given statement.

The argument which follows is the most singular in Plato. It appears
to be an imitation, or parody, of the Zenonian dialectic, just as the
speeches in the Phaedrus are an imitation of the style of Lysias, or as
the derivations in the Cratylus or the fallacies of the Euthydemus are
a parody of some contemporary Sophist. The interlocutor is not supposed,
as in most of the other Platonic dialogues, to take a living part in the
argument; he is only required to say 'Yes' and 'No' in the right places.
A hint has been already given that the paradoxes of Zeno admitted of a
higher application. This hint is the thread by which Plato connects the
two parts of the dialogue.

The paradoxes of Parmenides seem trivial to us, because the words to
which they relate have become trivial; their true nature as abstract
terms is perfectly understood by us, and we are inclined to regard the
treatment of them in Plato as a mere straw-splitting, or legerdemain of
words. Yet there was a power in them which fascinated the Neoplatonists
for centuries afterwards. Something that they found in them, or brought
to them--some echo or anticipation of a great truth or error, exercised
a wonderful influence over their minds. To do the Parmenides justice, we
should imagine similar aporiai raised on themes as sacred to us, as the
notions of One or Being were to an ancient Eleatic. 'If God is, what
follows? If God is not, what follows?' Or again: If God is or is not the
world; or if God is or is not many, or has or has not parts, or is or is
not in the world, or in time; or is or is not finite or infinite. Or if
the world is or is not; or has or has not a beginning or end; or is or
is not infinite, or infinitely divisible. Or again: if God is or is not
identical with his laws; or if man is or is not identical with the laws
of nature. We can easily see that here are many subjects for thought,
and that from these and similar hypotheses questions of great interest
might arise. And we also remark, that the conclusions derived from
either of the two alternative propositions might be equally impossible
and contradictory.

When we ask what is the object of these paradoxes, some have answered
that they are a mere logical puzzle, while others have seen in them an
Hegelian propaedeutic of the doctrine of Ideas. The first of these views
derives support from the manner in which Parmenides speaks of a similar
method being applied to all Ideas. Yet it is hard to suppose that Plato
would have furnished so elaborate an example, not of his own but of
the Eleatic dialectic, had he intended only to give an illustration of
method. The second view has been often overstated by those who, like
Hegel himself, have tended to confuse ancient with modern philosophy.
We need not deny that Plato, trained in the school of Cratylus and
Heracleitus, may have seen that a contradiction in terms is sometimes
the best expression of a truth higher than either (compare Soph.). But
his ideal theory is not based on antinomies. The correlation of Ideas
was the metaphysical difficulty of the age in which he lived; and the
Megarian and Cynic philosophy was a 'reductio ad absurdum' of their
isolation. To restore them to their natural connexion and to detect the
negative element in them is the aim of Plato in the Sophist. But his
view of their connexion falls very far short of the Hegelian identity
of Being and Not-being. The Being and Not-being of Plato never merge in
each other, though he is aware that 'determination is only negation.'

After criticizing the hypotheses of others, it may appear presumptuous
to add another guess to the many which have been already offered. May we
say, in Platonic language, that we still seem to see vestiges of a track
which has not yet been taken? It is quite possible that the obscurity
of the Parmenides would not have existed to a contemporary student of
philosophy, and, like the similar difficulty in the Philebus, is
really due to our ignorance of the mind of the age. There is an obscure
Megarian influence on Plato which cannot wholly be cleared up, and is
not much illustrated by the doubtful tradition of his retirement to
Megara after the death of Socrates. For Megara was within a walk of
Athens (Phaedr.), and Plato might have learned the Megarian doctrines
without settling there.

We may begin by remarking that the theses of Parmenides are expressly
said to follow the method of Zeno, and that the complex dilemma, though
declared to be capable of universal application, is applied in this
instance to Zeno's familiar question of the 'one and many.' Here, then,
is a double indication of the connexion of the Parmenides with the
Eristic school. The old Eleatics had asserted the existence of Being,
which they at first regarded as finite, then as infinite, then as
neither finite nor infinite, to which some of them had given what
Aristotle calls 'a form,' others had ascribed a material nature only.
The tendency of their philosophy was to deny to Being all predicates.
The Megarians, who succeeded them, like the Cynics, affirmed that no
predicate could be asserted of any subject; they also converted the
idea of Being into an abstraction of Good, perhaps with the view of
preserving a sort of neutrality or indifference between the mind and
things. As if they had said, in the language of modern philosophy:
'Being is not only neither finite nor infinite, neither at rest nor in
motion, but neither subjective nor objective.'

This is the track along which Plato is leading us. Zeno had attempted to
prove the existence of the one by disproving the existence of the many,
and Parmenides seems to aim at proving the existence of the subject
by showing the contradictions which follow from the assertion of any
predicates. Take the simplest of all notions, 'unity'; you cannot even
assert being or time of this without involving a contradiction. But is
the contradiction also the final conclusion? Probably no more than of
Zeno's denial of the many, or of Parmenides' assault upon the Ideas; no
more than of the earlier dialogues 'of search.' To us there seems to
be no residuum of this long piece of dialectics. But to the mind of
Parmenides and Plato, 'Gott-betrunkene Menschen,' there still remained
the idea of 'being' or 'good,' which could not be conceived, defined,
uttered, but could not be got rid of. Neither of them would have
imagined that their disputation ever touched the Divine Being (compare
Phil.). The same difficulties about Unity and Being are raised in the
Sophist; but there only as preliminary to their final solution.

If this view is correct, the real aim of the hypotheses of Parmenides
is to criticize the earlier Eleatic philosophy from the point of view of
Zeno or the Megarians. It is the same kind of criticism which Plato has
extended to his own doctrine of Ideas. Nor is there any want of poetical
consistency in attributing to the 'father Parmenides' the last review
of the Eleatic doctrines. The latest phases of all philosophies were
fathered upon the founder of the school.

Other critics have regarded the final conclusion of the Parmenides
either as sceptical or as Heracleitean. In the first case, they assume
that Plato means to show the impossibility of any truth. But this is not
the spirit of Plato, and could not with propriety be put into the mouth
of Parmenides, who, in this very dialogue, is urging Socrates, not to
doubt everything, but to discipline his mind with a view to the more
precise attainment of truth. The same remark applies to the second of
the two theories. Plato everywhere ridicules (perhaps unfairly) his
Heracleitean contemporaries: and if he had intended to support an
Heracleitean thesis, would hardly have chosen Parmenides, the condemner
of the 'undiscerning tribe who say that things both are and are not,'
to be the speaker. Nor, thirdly, can we easily persuade ourselves with
Zeller that by the 'one' he means the Idea; and that he is seeking to
prove indirectly the unity of the Idea in the multiplicity of phenomena.

We may now endeavour to thread the mazes of the labyrinth which
Parmenides knew so well, and trembled at the thought of them.

The argument has two divisions: There is the hypothesis that

     1.  One is.
     2.  One is not.
     If one is, it is nothing.
     If one is not, it is everything.

     But is and is not may be taken in two senses:
     Either one is one,
     Or, one has being,

     from which opposite consequences are deduced,
     1.a.  If one is one, it is nothing.
     1.b.  If one has being, it is all things.

     To which are appended two subordinate consequences:
     1.aa.  If one has being, all other things are.
     1.bb.  If one is one, all other things are not.

     The same distinction is then applied to the negative hypothesis:
     2.a.  If one is not one, it is all things.
     2.b.  If one has not being, it is nothing.

     Involving two parallel consequences respecting the other or remainder:
     2.aa.  If one is not one, other things are all.
     2.bb.  If one has not being, other things are not.


.....

'I cannot refuse,' said Parmenides, 'since, as Zeno remarks, we are
alone, though I may say with Ibycus, who in his old age fell in love, I,
like the old racehorse, tremble at the prospect of the course which I am
to run, and which I know so well. But as I must attempt this laborious
game, what shall be the subject? Suppose I take my own hypothesis of
the one.' 'By all means,' said Zeno. 'And who will answer me? Shall I
propose the youngest? he will be the most likely to say what he thinks,
and his answers will give me time to breathe.' 'I am the youngest,' said
Aristoteles, 'and at your service; proceed with your questions.'--The
result may be summed up as follows:--

1.a. One is not many, and therefore has no parts, and therefore is not
a whole, which is a sum of parts, and therefore has neither beginning,
middle, nor end, and is therefore unlimited, and therefore formless,
being neither round nor straight, for neither round nor straight can be
defined without assuming that they have parts; and therefore is not in
place, whether in another which would encircle and touch the one at
many points; or in itself, because that which is self-containing is also
contained, and therefore not one but two. This being premised, let us
consider whether one is capable either of motion or rest. For motion is
either change of substance, or motion on an axis, or from one place to
another. But the one is incapable of change of substance, which implies
that it ceases to be itself, or of motion on an axis, because there
would be parts around the axis; and any other motion involves change of
place. But existence in place has been already shown to be impossible;
and yet more impossible is coming into being in place, which implies
partial existence in two places at once, or entire existence neither
within nor without the same; and how can this be? And more impossible
still is the coming into being either as a whole or parts of that which
is neither a whole nor parts. The one, then, is incapable of motion.
But neither can the one be in anything, and therefore not in the same,
whether itself or some other, and is therefore incapable of rest.
Neither is one the same with itself or any other, or other than itself
or any other. For if other than itself, then other than one, and
therefore not one; and, if the same with other, it would be other, and
other than one. Neither can one while remaining one be other than other;
for other, and not one, is the other than other. But if not other by
virtue of being one, not by virtue of itself; and if not by virtue
of itself, not itself other, and if not itself other, not other than
anything. Neither will one be the same with itself. For the nature of
the same is not that of the one, but a thing which becomes the same with
anything does not become one; for example, that which becomes the same
with the many becomes many and not one. And therefore if the one is the
same with itself, the one is not one with itself; and therefore one and
not one. And therefore one is neither other than other, nor the same
with itself. Neither will the one be like or unlike itself or other;
for likeness is sameness of affections, and the one and the same are
different. And one having any affection which is other than being one
would be more than one. The one, then, cannot have the same affection
with and therefore cannot be like itself or other; nor can the one
have any other affection than its own, that is, be unlike itself or any
other, for this would imply that it was more than one. The one, then,
is neither like nor unlike itself or other. This being the case, neither
can the one be equal or unequal to itself or other. For equality implies
sameness of measure, as inequality implies a greater or less number
of measures. But the one, not having sameness, cannot have sameness of
measure; nor a greater or less number of measures, for that would imply
parts and multitude. Once more, can one be older or younger than itself
or other? or of the same age with itself or other? That would imply
likeness and unlikeness, equality and inequality. Therefore one cannot
be in time, because that which is in time is ever becoming older and
younger than itself, (for older and younger are relative terms, and he
who becomes older becomes younger,) and is also of the same age with
itself. None of which, or any other expressions of time, whether past,
future, or present, can be affirmed of one. One neither is, has been,
nor will be, nor becomes, nor has, nor will become. And, as these are
the only modes of being, one is not, and is not one. But to that which
is not, there is no attribute or relative, neither name nor word nor
idea nor science nor perception nor opinion appertaining. One, then, is
neither named, nor uttered, nor known, nor perceived, nor imagined. But
can all this be true? 'I think not.'

1.b. Let us, however, commence the inquiry again. We have to work out
all the consequences which follow on the assumption that the one is. If
one is, one partakes of being, which is not the same with one; the words
'being' and 'one' have different meanings. Observe the consequence: In
the one of being or the being of one are two parts, being and one, which
form one whole. And each of the two parts is also a whole, and involves
the other, and may be further subdivided into one and being, and is
therefore not one but two; and thus one is never one, and in this way
the one, if it is, becomes many and infinite. Again, let us conceive
of a one which by an effort of abstraction we separate from being: will
this abstract one be one or many? You say one only; let us see. In the
first place, the being of one is other than one; and one and being,
if different, are so because they both partake of the nature of other,
which is therefore neither one nor being; and whether we take being
and other, or being and one, or one and other, in any case we have two
things which separately are called either, and together both. And both
are two and either of two is severally one, and if one be added to any
of the pairs, the sum is three; and two is an even number, three an odd;
and two units exist twice, and therefore there are twice two; and three
units exist thrice, and therefore there are thrice three, and taken
together they give twice three and thrice two: we have even numbers
multiplied into even, and odd into even, and even into odd numbers. But
if one is, and both odd and even numbers are implied in one, must not
every number exist? And number is infinite, and therefore existence must
be infinite, for all and every number partakes of being; therefore
being has the greatest number of parts, and every part, however great or
however small, is equally one. But can one be in many places and yet be
a whole? If not a whole it must be divided into parts and represented
by a number corresponding to the number of the parts. And if so, we were
wrong in saying that being has the greatest number of parts; for being
is coequal and coextensive with one, and has no more parts than one; and
so the abstract one broken up into parts by being is many and infinite.
But the parts are parts of a whole, and the whole is their containing
limit, and the one is therefore limited as well as infinite in number;
and that which is a whole has beginning, middle, and end, and a middle
is equidistant from the extremes; and one is therefore of a certain
figure, round or straight, or a combination of the two, and being a
whole includes all the parts which are the whole, and is therefore
self-contained. But then, again, the whole is not in the parts, whether
all or some. Not in all, because, if in all, also in one; for, if
wanting in any one, how in all?--not in some, because the greater would
then be contained in the less. But if not in all, nor in any, nor in
some, either nowhere or in other. And if nowhere, nothing; therefore in
other. The one as a whole, then, is in another, but regarded as a sum
of parts is in itself; and is, therefore, both in itself and in another.
This being the case, the one is at once both at rest and in motion: at
rest, because resting in itself; in motion, because it is ever in other.
And if there is truth in what has preceded, one is the same and not the
same with itself and other. For everything in relation to every other
thing is either the same with it or other; or if neither the same nor
other, then in the relation of part to a whole or whole to a part. But
one cannot be a part or whole in relation to one, nor other than
one; and is therefore the same with one. Yet this sameness is again
contradicted by one being in another place from itself which is in the
same place; this follows from one being in itself and in another; one,
therefore, is other than itself. But if anything is other than anything,
will it not be other than other? And the not one is other than the one,
and the one than the not one; therefore one is other than all others.
But the same and the other exclude one another, and therefore the other
can never be in the same; nor can the other be in anything for ever so
short a time, as for that time the other will be in the same. And the
other, if never in the same, cannot be either in the one or in the not
one. And one is not other than not one, either by reason of other or
of itself; and therefore they are not other than one another at all.
Neither can the not one partake or be part of one, for in that case it
would be one; nor can the not one be number, for that also involves one.
And therefore, not being other than the one or related to the one as
a whole to parts or parts to a whole, not one is the same as one.
Wherefore the one is the same and also not the same with the others
and also with itself; and is therefore like and unlike itself and the
others, and just as different from the others as they are from the one,
neither more nor less. But if neither more nor less, equally different;
and therefore the one and the others have the same relations. This may
be illustrated by the case of names: when you repeat the same name twice
over, you mean the same thing; and when you say that the other is other
than the one, or the one other than the other, this very word other
(eteron), which is attributed to both, implies sameness. One, then, as
being other than others, and other as being other than one, are alike in
that they have the relation of otherness; and likeness is similarity
of relations. And everything as being other of everything is also like
everything. Again, same and other, like and unlike, are opposites: and
since in virtue of being other than the others the one is like them, in
virtue of being the same it must be unlike. Again, one, as having the
same relations, has no difference of relation, and is therefore not
unlike, and therefore like; or, as having different relations, is
different and unlike. Thus, one, as being the same and not the same with
itself and others--for both these reasons and for either of them--is
also like and unlike itself and the others. Again, how far can one touch
itself and the others? As existing in others, it touches the others; and
as existing in itself, touches only itself. But from another point of
view, that which touches another must be next in order of place; one,
therefore, must be next in order of place to itself, and would therefore
be two, and in two places. But one cannot be two, and therefore cannot
be in contact with itself. Nor again can one touch the other. Two
objects are required to make one contact; three objects make two
contacts; and all the objects in the world, if placed in a series, would
have as many contacts as there are objects, less one. But if one only
exists, and not two, there is no contact. And the others, being other
than one, have no part in one, and therefore none in number, and
therefore two has no existence, and therefore there is no contact.
For all which reasons, one has and has not contact with itself and the
others.

Once more, Is one equal and unequal to itself and the others? Suppose
one and the others to be greater or less than each other or equal to one
another, they will be greater or less or equal by reason of equality or
greatness or smallness inhering in them in addition to their own proper
nature. Let us begin by assuming smallness to be inherent in one: in
this case the inherence is either in the whole or in a part. If the
first, smallness is either coextensive with the whole one, or contains
the whole, and, if coextensive with the one, is equal to the one, or
if containing the one will be greater than the one. But smallness thus
performs the function of equality or of greatness, which is impossible.
Again, if the inherence be in a part, the same contradiction follows:
smallness will be equal to the part or greater than the part; therefore
smallness will not inhere in anything, and except the idea of smallness
there will be nothing small. Neither will greatness; for greatness will
have a greater;--and there will be no small in relation to which it is
great. And there will be no great or small in objects, but greatness
and smallness will be relative only to each other; therefore the others
cannot be greater or less than the one; also the one can neither exceed
nor be exceeded by the others, and they are therefore equal to one
another. And this will be true also of the one in relation to itself:
one will be equal to itself as well as to the others (talla). Yet one,
being in itself, must also be about itself, containing and contained,
and is therefore greater and less than itself. Further, there is nothing
beside the one and the others; and as these must be in something, they
must therefore be in one another; and as that in which a thing is is
greater than the thing, the inference is that they are both greater and
less than one another, because containing and contained in one another.
Therefore the one is equal to and greater and less than itself or other,
having also measures or parts or numbers equal to or greater or less
than itself or other.

But does one partake of time? This must be acknowledged, if the one
partakes of being. For 'to be' is the participation of being in present
time, 'to have been' in past, 'to be about to be' in future time. And
as time is ever moving forward, the one becomes older than itself; and
therefore younger than itself; and is older and also younger when in the
process of becoming it arrives at the present; and it is always older
and younger, for at any moment the one is, and therefore it becomes
and is not older and younger than itself but during an equal time with
itself, and is therefore contemporary with itself.

And what are the relations of the one to the others? Is it or does it
become older or younger than they? At any rate the others are more than
one, and one, being the least of all numbers, must be prior in time to
greater numbers. But on the other hand, one must come into being in a
manner accordant with its own nature. Now one has parts or others, and
has therefore a beginning, middle, and end, of which the beginning is
first and the end last. And the parts come into existence first; last of
all the whole, contemporaneously with the end, being therefore younger,
while the parts or others are older than the one. But, again, the one
comes into being in each of the parts as much as in the whole, and must
be of the same age with them. Therefore one is at once older and younger
than the parts or others, and also contemporaneous with them, for no
part can be a part which is not one. Is this true of becoming as well as
being? Thus much may be affirmed, that the same things which are older
or younger cannot become older or younger in a greater degree than they
were at first by the addition of equal times. But, on the other hand,
the one, if older than others, has come into being a longer time than
they have. And when equal time is added to a longer and shorter, the
relative difference between them is diminished. In this way that which
was older becomes younger, and that which was younger becomes older,
that is to say, younger and older than at first; and they ever become
and never have become, for then they would be. Thus the one and others
always are and are becoming and not becoming younger and also older than
one another. And one, partaking of time and also partaking of becoming
older and younger, admits of all time, present, past, and future--was,
is, shall be--was becoming, is becoming, will become. And there is
science of the one, and opinion and name and expression, as is already
implied in the fact of our inquiry.

Yet once more, if one be one and many, and neither one nor many, and
also participant of time, must there not be a time at which one as being
one partakes of being, and a time when one as not being one is deprived
of being? But these two contradictory states cannot be experienced
by the one both together: there must be a time of transition. And the
transition is a process of generation and destruction, into and from
being and not-being, the one and the others. For the generation of the
one is the destruction of the others, and the generation of the others
is the destruction of the one. There is also separation and aggregation,
assimilation and dissimilation, increase, diminution, equalization,
a passage from motion to rest, and from rest to motion in the one and
many. But when do all these changes take place? When does motion become
rest, or rest motion? The answer to this question will throw a light
upon all the others. Nothing can be in motion and at rest at the same
time; and therefore the change takes place 'in a moment'--which is a
strange expression, and seems to mean change in no time. Which is true
also of all the other changes, which likewise take place in no time.

1.aa. But if one is, what happens to the others, which in the first
place are not one, yet may partake of one in a certain way? The others
are other than the one because they have parts, for if they had no parts
they would be simply one, and parts imply a whole to which they belong;
otherwise each part would be a part of many, and being itself one of
them, of itself, and if a part of all, of each one of the other parts,
which is absurd. For a part, if not a part of one, must be a part of
all but this one, and if so not a part of each one; and if not a part
of each one, not a part of any one of many, and so not of one; and if of
none, how of all? Therefore a part is neither a part of many nor of
all, but of an absolute and perfect whole or one. And if the others have
parts, they must partake of the whole, and must be the whole of which
they are the parts. And each part, as the word 'each' implies, is also
an absolute one. And both the whole and the parts partake of one, for
the whole of which the parts are parts is one, and each part is one part
of the whole; and whole and parts as participating in one are other
than one, and as being other than one are many and infinite; and however
small a fraction you separate from them is many and not one. Yet the
fact of their being parts furnishes the others with a limit towards
other parts and towards the whole; they are finite and also infinite:
finite through participation in the one, infinite in their own nature.
And as being finite, they are alike; and as being infinite, they are
alike; but as being both finite and also infinite, they are in the
highest degree unlike. And all other opposites might without difficulty
be shown to unite in them.

1.bb. Once more, leaving all this: Is there not also an opposite series
of consequences which is equally true of the others, and may be deduced
from the existence of one? There is. One is distinct from the others,
and the others from one; for one and the others are all things, and
there is no third existence besides them. And the whole of one cannot
be in others nor parts of it, for it is separated from others and has
no parts, and therefore the others have no unity, nor plurality, nor
duality, nor any other number, nor any opposition or distinction, such
as likeness and unlikeness, some and other, generation and corruption,
odd and even. For if they had these they would partake either of one
opposite, and this would be a participation in one; or of two opposites,
and this would be a participation in two. Thus if one exists, one is all
things, and likewise nothing, in relation to one and to the others.

2.a. But, again, assume the opposite hypothesis, that the one is not,
and what is the consequence? In the first place, the proposition, that
one is not, is clearly opposed to the proposition, that not one is not.
The subject of any negative proposition implies at once knowledge and
difference. Thus 'one' in the proposition--'The one is not,' must be
something known, or the words would be unintelligible; and again this
'one which is not' is something different from other things. Moreover,
this and that, some and other, may be all attributed or related to
the one which is not, and which though non-existent may and must have
plurality, if the one only is non-existent and nothing else; but if all
is not-being there is nothing which can be spoken of. Also the one which
is not differs, and is different in kind from the others, and therefore
unlike them; and they being other than the one, are unlike the one,
which is therefore unlike them. But one, being unlike other, must be
like itself; for the unlikeness of one to itself is the destruction of
the hypothesis; and one cannot be equal to the others; for that would
suppose being in the one, and the others would be equal to one and like
one; both which are impossible, if one does not exist. The one which
is not, then, if not equal is unequal to the others, and in equality
implies great and small, and equality lies between great and small, and
therefore the one which is not partakes of equality. Further, the one
which is not has being; for that which is true is, and it is true that
the one is not. And so the one which is not, if remitting aught of the
being of non-existence, would become existent. For not being implies the
being of not-being, and being the not-being of not-being; or more truly
being partakes of the being of being and not of the being of not-being,
and not-being of the being of not-being and not of the not-being
of not-being. And therefore the one which is not has being and also
not-being. And the union of being and not-being involves change or
motion. But how can not-being, which is nowhere, move or change, either
from one place to another or in the same place? And whether it is or is
not, it would cease to be one if experiencing a change of substance. The
one which is not, then, is both in motion and at rest, is altered and
unaltered, and becomes and is destroyed, and does not become and is not
destroyed.

2.b. Once more, let us ask the question, If one is not, what happens in
regard to one? The expression 'is not' implies negation of being:--do we
mean by this to say that a thing, which is not, in a certain sense is?
or do we mean absolutely to deny being of it? The latter. Then the one
which is not can neither be nor become nor perish nor experience change
of substance or place. Neither can rest, or motion, or greatness, or
smallness, or equality, or unlikeness, or likeness either to itself or
other, or attribute or relation, or now or hereafter or formerly, or
knowledge or opinion or perception or name or anything else be asserted
of that which is not.

2.aa. Once more, if one is not, what becomes of the others? If we
speak of them they must be, and their very name implies difference, and
difference implies relation, not to the one, which is not, but to
one another. And they are others of each other not as units but
as infinities, the least of which is also infinity, and capable of
infinitesimal division. And they will have no unity or number, but only
a semblance of unity and number; and the least of them will appear large
and manifold in comparison with the infinitesimal fractions into which
it may be divided. Further, each particle will have the appearance of
being equal with the fractions. For in passing from the greater to the
less it must reach an intermediate point, which is equality. Moreover,
each particle although having a limit in relation to itself and to other
particles, yet it has neither beginning, middle, nor end; for there is
always a beginning before the beginning, and a middle within the middle,
and an end beyond the end, because the infinitesimal division is never
arrested by the one. Thus all being is one at a distance, and broken
up when near, and like at a distance and unlike when near; and also the
particles which compose being seem to be like and unlike, in rest and
motion, in generation and corruption, in contact and separation, if one
is not.

2.bb. Once more, let us inquire, If the one is not, and the others of
the one are, what follows? In the first place, the others will not be
the one, nor the many, for in that case the one would be contained in
them; neither will they appear to be one or many; because they have no
communion or participation in that which is not, nor semblance of that
which is not. If one is not, the others neither are, nor appear to be
one or many, like or unlike, in contact or separation. In short, if one
is not, nothing is.

The result of all which is, that whether one is or is not, one and the
others, in relation to themselves and to one another, are and are not,
and appear to be and appear not to be, in all manner of ways.

I. On the first hypothesis we may remark: first, That one is one is
an identical proposition, from which we might expect that no further
consequences could be deduced. The train of consequences which follows,
is inferred by altering the predicate into 'not many.' Yet, perhaps, if
a strict Eristic had been present, oios aner ei kai nun paren, he might
have affirmed that the not many presented a different aspect of the
conception from the one, and was therefore not identical with it. Such
a subtlety would be very much in character with the Zenonian dialectic.
Secondly, We may note, that the conclusion is really involved in the
premises. For one is conceived as one, in a sense which excludes all
predicates. When the meaning of one has been reduced to a point, there
is no use in saying that it has neither parts nor magnitude. Thirdly,
The conception of the same is, first of all, identified with the one;
and then by a further analysis distinguished from, and even opposed to
it. Fourthly, We may detect notions, which have reappeared in modern
philosophy, e.g. the bare abstraction of undefined unity, answering to
the Hegelian 'Seyn,' or the identity of contradictions 'that which is
older is also younger,' etc., or the Kantian conception of an a priori
synthetical proposition 'one is.'

II. In the first series of propositions the word 'is' is really the
copula; in the second, the verb of existence. As in the first series,
the negative consequence followed from one being affirmed to be
equivalent to the not many; so here the affirmative consequence is
deduced from one being equivalent to the many.

In the former case, nothing could be predicated of the one, but now
everything--multitude, relation, place, time, transition. One is
regarded in all the aspects of one, and with a reference to all the
consequences which flow, either from the combination or the separation
of them. The notion of transition involves the singular extra-temporal
conception of 'suddenness.' This idea of 'suddenness' is based upon the
contradiction which is involved in supposing that anything can be in two
places at once. It is a mere fiction; and we may observe that similar
antinomies have led modern philosophers to deny the reality of time and
space. It is not the infinitesimal of time, but the negative of time.
By the help of this invention the conception of change, which sorely
exercised the minds of early thinkers, seems to be, but is not really
at all explained. The difficulty arises out of the imperfection of
language, and should therefore be no longer regarded as a difficulty at
all. The only way of meeting it, if it exists, is to acknowledge that
this rather puzzling double conception is necessary to the expression
of the phenomena of motion or change, and that this and similar double
notions, instead of being anomalies, are among the higher and more
potent instruments of human thought.

The processes by which Parmenides obtains his remarkable results may be
summed up as follows: (1) Compound or correlative ideas which involve
each other, such as, being and not-being, one and many, are conceived
sometimes in a state of composition, and sometimes of division: (2) The
division or distinction is sometimes heightened into total opposition,
e.g. between one and same, one and other: or (3) The idea, which has
been already divided, is regarded, like a number, as capable of further
infinite subdivision: (4) The argument often proceeds 'a dicto secundum
quid ad dictum simpliciter' and conversely: (5) The analogy of opposites
is misused by him; he argues indiscriminately sometimes from what is
like, sometimes from what is unlike in them: (6) The idea of being or
not-being is identified with existence or non-existence in place
or time: (7) The same ideas are regarded sometimes as in process of
transition, sometimes as alternatives or opposites: (8) There are no
degrees or kinds of sameness, likeness, difference, nor any adequate
conception of motion or change: (9) One, being, time, like space in
Zeno's puzzle of Achilles and the tortoise, are regarded sometimes as
continuous and sometimes as discrete: (10) In some parts of the argument
the abstraction is so rarefied as to become not only fallacious, but
almost unintelligible, e.g. in the contradiction which is elicited out
of the relative terms older and younger: (11) The relation between two
terms is regarded under contradictory aspects, as for example when
the existence of the one and the non-existence of the one are equally
assumed to involve the existence of the many: (12) Words are used
through long chains of argument, sometimes loosely, sometimes with the
precision of numbers or of geometrical figures.

The argument is a very curious piece of work, unique in literature.
It seems to be an exposition or rather a 'reductio ad absurdum' of the
Megarian philosophy, but we are too imperfectly acquainted with this
last to speak with confidence about it. It would be safer to say that it
is an indication of the sceptical, hyperlogical fancies which prevailed
among the contemporaries of Socrates. It throws an indistinct light upon
Aristotle, and makes us aware of the debt which the world owes to him or
his school. It also bears a resemblance to some modern speculations, in
which an attempt is made to narrow language in such a manner that number
and figure may be made a calculus of thought. It exaggerates one side
of logic and forgets the rest. It has the appearance of a mathematical
process; the inventor of it delights, as mathematicians do, in eliciting
or discovering an unexpected result. It also helps to guard us against
some fallacies by showing the consequences which flow from them.

In the Parmenides we seem to breathe the spirit of the Megarian
philosophy, though we cannot compare the two in detail. But Plato also
goes beyond his Megarian contemporaries; he has split their straws over
again, and admitted more than they would have desired. He is indulging
the analytical tendencies of his age, which can divide but not combine.
And he does not stop to inquire whether the distinctions which he makes
are shadowy and fallacious, but 'whither the argument blows' he follows.

III. The negative series of propositions contains the first conception
of the negation of a negation. Two minus signs in arithmetic or algebra
make a plus. Two negatives destroy each other. This abstruse notion is
the foundation of the Hegelian logic. The mind must not only admit
that determination is negation, but must get through negation into
affirmation. Whether this process is real, or in any way an assistance
to thought, or, like some other logical forms, a mere figure of speech
transferred from the sphere of mathematics, may be doubted. That Plato
and the most subtle philosopher of the nineteenth century should have
lighted upon the same notion, is a singular coincidence of ancient and
modern thought.

IV. The one and the many or others are reduced to their strictest
arithmetical meaning. That one is three or three one, is a proposition
which has, perhaps, given rise to more controversy in the world than
any other. But no one has ever meant to say that three and one are to be
taken in the same sense. Whereas the one and many of the Parmenides have
precisely the same meaning; there is no notion of one personality or
substance having many attributes or qualities. The truth seems to
be rather the opposite of that which Socrates implies: There is no
contradiction in the concrete, but in the abstract; and the more
abstract the idea, the more palpable will be the contradiction. For just
as nothing can persuade us that the number one is the number three, so
neither can we be persuaded that any abstract idea is identical with
its opposite, although they may both inhere together in some external
object, or some more comprehensive conception. Ideas, persons, things
may be one in one sense and many in another, and may have various
degrees of unity and plurality. But in whatever sense and in whatever
degree they are one they cease to be many; and in whatever degree or
sense they are many they cease to be one.

Two points remain to be considered: 1st, the connexion between the first
and second parts of the dialogue; 2ndly, the relation of the Parmenides
to the other dialogues.

I. In both divisions of the dialogue the principal speaker is the same,
and the method pursued by him is also the same, being a criticism on
received opinions: first, on the doctrine of Ideas; secondly, of Being.
From the Platonic Ideas we naturally proceed to the Eleatic One or Being
which is the foundation of them. They are the same philosophy in two
forms, and the simpler form is the truer and deeper. For the Platonic
Ideas are mere numerical differences, and the moment we attempt to
distinguish between them, their transcendental character is lost; ideas
of justice, temperance, and good, are really distinguishable only with
reference to their application in the world. If we once ask how they
are related to individuals or to the ideas of the divine mind, they are
again merged in the aboriginal notion of Being. No one can answer the
questions which Parmenides asks of Socrates. And yet these questions are
asked with the express acknowledgment that the denial of ideas will be
the destruction of the human mind. The true answer to the difficulty
here thrown out is the establishment of a rational psychology; and
this is a work which is commenced in the Sophist. Plato, in urging the
difficulty of his own doctrine of Ideas, is far from denying that some
doctrine of Ideas is necessary, and for this he is paving the way.

In a similar spirit he criticizes the Eleatic doctrine of Being, not
intending to deny Ontology, but showing that the old Eleatic notion,
and the very name 'Being,' is unable to maintain itself against the
subtleties of the Megarians. He did not mean to say that Being or
Substance had no existence, but he is preparing for the development
of his later view, that ideas were capable of relation. The fact that
contradictory consequences follow from the existence or non-existence
of one or many, does not prove that they have or have not existence,
but rather that some different mode of conceiving them is required.
Parmenides may still have thought that 'Being was,' just as Kant would
have asserted the existence of 'things in themselves,' while denying the
transcendental use of the Categories.

Several lesser links also connect the first and second parts of the
dialogue: (1) The thesis is the same as that which Zeno has been already
discussing: (2) Parmenides has intimated in the first part, that the
method of Zeno should, as Socrates desired, be extended to Ideas: (3)
The difficulty of participating in greatness, smallness, equality is
urged against the Ideas as well as against the One.

II. The Parmenides is not only a criticism of the Eleatic notion of
Being, but also of the methods of reasoning then in existence, and
in this point of view, as well as in the other, may be regarded as an
introduction to the Sophist. Long ago, in the Euthydemus, the vulgar
application of the 'both and neither' Eristic had been subjected to a
similar criticism, which there takes the form of banter and irony, here
of illustration.

The attack upon the Ideas is resumed in the Philebus, and is followed
by a return to a more rational philosophy. The perplexity of the One and
Many is there confined to the region of Ideas, and replaced by a theory
of classification; the Good arranged in classes is also contrasted with
the barren abstraction of the Megarians. The war is carried on against
the Eristics in all the later dialogues, sometimes with a playful irony,
at other times with a sort of contempt. But there is no lengthened
refutation of them. The Parmenides belongs to that stage of the
dialogues of Plato in which he is partially under their influence, using
them as a sort of 'critics or diviners' of the truth of his own, and of
the Eleatic theories. In the Theaetetus a similar negative dialectic
is employed in the attempt to define science, which after every effort
remains undefined still. The same question is revived from the objective
side in the Sophist: Being and Not-being are no longer exhibited in
opposition, but are now reconciled; and the true nature of Not-being is
discovered and made the basis of the correlation of ideas. Some
links are probably missing which might have been supplied if we had
trustworthy accounts of Plato's oral teaching.

To sum up: the Parmenides of Plato is a critique, first, of the Platonic
Ideas, and secondly, of the Eleatic doctrine of Being. Neither are
absolutely denied. But certain difficulties and consequences are shown
in the assumption of either, which prove that the Platonic as well as
the Eleatic doctrine must be remodelled. The negation and contradiction
which are involved in the conception of the One and Many are preliminary
to their final adjustment. The Platonic Ideas are tested by the
interrogative method of Socrates; the Eleatic One or Being is tried by
the severer and perhaps impossible method of hypothetical consequences,
negative and affirmative. In the latter we have an example of the
Zenonian or Megarian dialectic, which proceeded, not 'by assailing
premises, but conclusions'; this is worked out and improved by Plato.
When primary abstractions are used in every conceivable sense, any or
every conclusion may be deduced from them. The words 'one,' 'other,'
'being,' 'like,' 'same,' 'whole,' and their opposites, have slightly
different meanings, as they are applied to objects of thought or
objects of sense--to number, time, place, and to the higher ideas of
the reason;--and out of their different meanings this 'feast' of
contradictions 'has been provided.'

...

The Parmenides of Plato belongs to a stage of philosophy which has
passed away. At first we read it with a purely antiquarian or historical
interest; and with difficulty throw ourselves back into a state of
the human mind in which Unity and Being occupied the attention of
philosophers. We admire the precision of the language, in which, as in
some curious puzzle, each word is exactly fitted into every other,
and long trains of argument are carried out with a sort of geometrical
accuracy. We doubt whether any abstract notion could stand the searching
cross-examination of Parmenides; and may at last perhaps arrive at the
conclusion that Plato has been using an imaginary method to work out an
unmeaning conclusion. But the truth is, that he is carrying on a process
which is not either useless or unnecessary in any age of philosophy.
We fail to understand him, because we do not realize that the questions
which he is discussing could have had any value or importance. We
suppose them to be like the speculations of some of the Schoolmen,
which end in nothing. But in truth he is trying to get rid of the
stumbling-blocks of thought which beset his contemporaries. Seeing that
the Megarians and Cynics were making knowledge impossible, he takes
their 'catch-words' and analyzes them from every conceivable point of
view. He is criticizing the simplest and most general of our ideas, in
which, as they are the most comprehensive, the danger of error is the
most serious; for, if they remain unexamined, as in a mathematical
demonstration, all that flows from them is affected, and the error
pervades knowledge far and wide. In the beginning of philosophy this
correction of human ideas was even more necessary than in our own
times, because they were more bound up with words; and words when once
presented to the mind exercised a greater power over thought. There is
a natural realism which says, 'Can there be a word devoid of meaning, or
an idea which is an idea of nothing?' In modern times mankind have often
given too great importance to a word or idea. The philosophy of the
ancients was still more in slavery to them, because they had not the
experience of error, which would have placed them above the illusion.

The method of the Parmenides may be compared with the process of
purgation, which Bacon sought to introduce into philosophy. Plato is
warning us against two sorts of 'Idols of the Den': first, his own
Ideas, which he himself having created is unable to connect in any way
with the external world; secondly, against two idols in particular,
'Unity' and 'Being,' which had grown up in the pre-Socratic philosophy,
and were still standing in the way of all progress and development of
thought. He does not say with Bacon, 'Let us make truth by experiment,'
or 'From these vague and inexact notions let us turn to facts.' The time
has not yet arrived for a purely inductive philosophy. The instruments
of thought must first be forged, that they may be used hereafter by
modern inquirers. How, while mankind were disputing about universals,
could they classify phenomena? How could they investigate causes, when
they had not as yet learned to distinguish between a cause and an end?
How could they make any progress in the sciences without first arranging
them? These are the deficiencies which Plato is seeking to supply in an
age when knowledge was a shadow of a name only. In the earlier dialogues
the Socratic conception of universals is illustrated by his genius; in
the Phaedrus the nature of division is explained; in the Republic the
law of contradiction and the unity of knowledge are asserted; in the
later dialogues he is constantly engaged both with the theory and
practice of classification. These were the 'new weapons,' as he terms
them in the Philebus, which he was preparing for the use of some who, in
after ages, would be found ready enough to disown their obligations
to the great master, or rather, perhaps, would be incapable of
understanding them.

Numberless fallacies, as we are often truly told, have originated in a
confusion of the 'copula,' and the 'verb of existence.' Would not the
distinction which Plato by the mouth of Parmenides makes between 'One
is one' and 'One has being' have saved us from this and many similar
confusions? We see again that a long period in the history of philosophy
was a barren tract, not uncultivated, but unfruitful, because there
was no inquiry into the relation of language and thought, and the
metaphysical imagination was incapable of supplying the missing link
between words and things. The famous dispute between Nominalists and
Realists would never have been heard of, if, instead of transferring the
Platonic Ideas into a crude Latin phraseology, the spirit of Plato had
been truly understood and appreciated. Upon the term substance at least
two celebrated theological controversies appear to hinge, which would
not have existed, or at least not in their present form, if we had
'interrogated' the word substance, as Plato has the notions of Unity and
Being. These weeds of philosophy have struck their roots deep into
the soil, and are always tending to reappear, sometimes in new-fangled
forms; while similar words, such as development, evolution, law, and
the like, are constantly put in the place of facts, even by writers who
profess to base truth entirely upon fact. In an unmetaphysical age there
is probably more metaphysics in the common sense (i.e. more a
priori assumption) than in any other, because there is more complete
unconsciousness that we are resting on our own ideas, while we please
ourselves with the conviction that we are resting on facts. We do
not consider how much metaphysics are required to place us above
metaphysics, or how difficult it is to prevent the forms of expression
which are ready made for our use from outrunning actual observation and
experiment.

In the last century the educated world were astonished to find that the
whole fabric of their ideas was falling to pieces, because Hume amused
himself by analyzing the word 'cause' into uniform sequence. Then arose
a philosophy which, equally regardless of the history of the mind,
sought to save mankind from scepticism by assigning to our notions
of 'cause and effect,' 'substance and accident,' 'whole and part,'
a necessary place in human thought. Without them we could have
no experience, and therefore they were supposed to be prior to
experience--to be incrusted on the 'I'; although in the phraseology of
Kant there could be no transcendental use of them, or, in other words,
they were only applicable within the range of our knowledge. But into
the origin of these ideas, which he obtains partly by an analysis of the
proposition, partly by development of the 'ego,' he never inquires--they
seem to him to have a necessary existence; nor does he attempt to
analyse the various senses in which the word 'cause' or 'substance' may
be employed.

The philosophy of Berkeley could never have had any meaning, even
to himself, if he had first analyzed from every point of view the
conception of 'matter.' This poor forgotten word (which was 'a very good
word' to describe the simplest generalization of external objects) is
now superseded in the vocabulary of physical philosophers by 'force,'
which seems to be accepted without any rigid examination of its meaning,
as if the general idea of 'force' in our minds furnished an explanation
of the infinite variety of forces which exist in the universe. A similar
ambiguity occurs in the use of the favourite word 'law,' which is
sometimes regarded as a mere abstraction, and then elevated into a real
power or entity, almost taking the place of God. Theology, again, is
full of undefined terms which have distracted the human mind for ages.
Mankind have reasoned from them, but not to them; they have drawn out
the conclusions without proving the premises; they have asserted the
premises without examining the terms. The passions of religious parties
have been roused to the utmost about words of which they could have
given no explanation, and which had really no distinct meaning. One sort
of them, faith, grace, justification, have been the symbols of one
class of disputes; as the words substance, nature, person, of another,
revelation, inspiration, and the like, of a third. All of them have been
the subject of endless reasonings and inferences; but a spell has hung
over the minds of theologians or philosophers which has prevented them
from examining the words themselves. Either the effort to rise above
and beyond their own first ideas was too great for them, or there might,
perhaps, have seemed to be an irreverence in doing so. About the Divine
Being Himself, in whom all true theological ideas live and move, men
have spoken and reasoned much, and have fancied that they instinctively
know Him. But they hardly suspect that under the name of God even
Christians have included two characters or natures as much opposed as
the good and evil principle of the Persians.

To have the true use of words we must compare them with things; in using
them we acknowledge that they seldom give a perfect representation of
our meaning. In like manner when we interrogate our ideas we find that
we are not using them always in the sense which we supposed. And Plato,
while he criticizes the inconsistency of his own doctrine of universals
and draws out the endless consequences which flow from the assertion
either that 'Being is' or that 'Being is not,' by no means intends
to deny the existence of universals or the unity under which they
are comprehended. There is nothing further from his thoughts than
scepticism. But before proceeding he must examine the foundations which
he and others have been laying; there is nothing true which is not from
some point of view untrue, nothing absolute which is not also relative
(compare Republic).

And so, in modern times, because we are called upon to analyze our ideas
and to come to a distinct understanding about the meaning of words;
because we know that the powers of language are very unequal to the
subtlety of nature or of mind, we do not therefore renounce the use of
them; but we replace them in their old connexion, having first tested
their meaning and quality, and having corrected the error which is
involved in them; or rather always remembering to make allowance for
the adulteration or alloy which they contain. We cannot call a new
metaphysical world into existence any more than we can frame a new
universal language; in thought as in speech, we are dependent on the
past. We know that the words 'cause' and 'effect' are very far from
representing to us the continuity or the complexity of nature or the
different modes or degrees in which phenomena are connected. Yet we
accept them as the best expression which we have of the correlation of
forces or objects. We see that the term 'law' is a mere abstraction,
under which laws of matter and of mind, the law of nature and the law of
the land are included, and some of these uses of the word are confusing,
because they introduce into one sphere of thought associations
which belong to another; for example, order or sequence is apt to be
confounded with external compulsion and the internal workings of the
mind with their material antecedents. Yet none of them can be dispensed
with; we can only be on our guard against the error or confusion which
arises out of them. Thus in the use of the word 'substance' we are far
from supposing that there is any mysterious substratum apart from the
objects which we see, and we acknowledge that the negative notion is
very likely to become a positive one. Still we retain the word as a
convenient generalization, though not without a double sense, substance,
and essence, derived from the two-fold translation of the Greek ousia.

So the human mind makes the reflection that God is not a person like
ourselves--is not a cause like the material causes in nature, nor even
an intelligent cause like a human agent--nor an individual, for He is
universal; and that every possible conception which we can form of Him
is limited by the human faculties. We cannot by any effort of thought
or exertion of faith be in and out of our own minds at the same instant.
How can we conceive Him under the forms of time and space, who is out of
time and space? How get rid of such forms and see Him as He is? How
can we imagine His relation to the world or to ourselves? Innumerable
contradictions follow from either of the two alternatives, that God is
or that He is not. Yet we are far from saying that we know nothing of
Him, because all that we know is subject to the conditions of human
thought. To the old belief in Him we return, but with corrections. He is
a person, but not like ourselves; a mind, but not a human mind; a cause,
but not a material cause, nor yet a maker or artificer. The words which
we use are imperfect expressions of His true nature; but we do not
therefore lose faith in what is best and highest in ourselves and in the
world.

'A little philosophy takes us away from God; a great deal brings us back
to Him.' When we begin to reflect, our first thoughts respecting Him and
ourselves are apt to be sceptical. For we can analyze our religious as
well as our other ideas; we can trace their history; we can criticize
their perversion; we see that they are relative to the human mind and
to one another. But when we have carried our criticism to the furthest
point, they still remain, a necessity of our moral nature, better known
and understood by us, and less liable to be shaken, because we are more
aware of their necessary imperfection. They come to us with 'better
opinion, better confirmation,' not merely as the inspirations either of
ourselves or of another, but deeply rooted in history and in the human
mind.




PARMENIDES


PERSONS OF THE DIALOGUE: Cephalus, Adeimantus, Glaucon, Antiphon,
Pythodorus, Socrates, Zeno, Parmenides, Aristoteles.

Cephalus rehearses a dialogue which is supposed to have been narrated in
his presence by Antiphon, the half-brother of Adeimantus and Glaucon, to
certain Clazomenians.


We had come from our home at Clazomenae to Athens, and met Adeimantus
and Glaucon in the Agora. Welcome, Cephalus, said Adeimantus, taking me
by the hand; is there anything which we can do for you in Athens?

Yes; that is why I am here; I wish to ask a favour of you.

What may that be? he said.

I want you to tell me the name of your half brother, which I have
forgotten; he was a mere child when I last came hither from Clazomenae,
but that was a long time ago; his father's name, if I remember rightly,
was Pyrilampes?

Yes, he said, and the name of our brother, Antiphon; but why do you ask?

Let me introduce some countrymen of mine, I said; they are lovers of
philosophy, and have heard that Antiphon was intimate with a certain
Pythodorus, a friend of Zeno, and remembers a conversation which took
place between Socrates, Zeno, and Parmenides many years ago, Pythodorus
having often recited it to him.

Quite true.

And could we hear it? I asked.

Nothing easier, he replied; when he was a youth he made a careful study
of the piece; at present his thoughts run in another direction; like his
grandfather Antiphon he is devoted to horses. But, if that is what you
want, let us go and look for him; he dwells at Melita, which is quite
near, and he has only just left us to go home.

Accordingly we went to look for him; he was at home, and in the act
of giving a bridle to a smith to be fitted. When he had done with the
smith, his brothers told him the purpose of our visit; and he saluted me
as an acquaintance whom he remembered from my former visit, and we
asked him to repeat the dialogue. At first he was not very willing, and
complained of the trouble, but at length he consented. He told us that
Pythodorus had described to him the appearance of Parmenides and Zeno;
they came to Athens, as he said, at the great Panathenaea; the former
was, at the time of his visit, about 65 years old, very white with age,
but well favoured. Zeno was nearly 40 years of age, tall and fair to
look upon; in the days of his youth he was reported to have been
beloved by Parmenides. He said that they lodged with Pythodorus in the
Ceramicus, outside the wall, whither Socrates, then a very young man,
came to see them, and many others with him; they wanted to hear the
writings of Zeno, which had been brought to Athens for the first time
on the occasion of their visit. These Zeno himself read to them in the
absence of Parmenides, and had very nearly finished when Pythodorus
entered, and with him Parmenides and Aristoteles who was afterwards
one of the Thirty, and heard the little that remained of the dialogue.
Pythodorus had heard Zeno repeat them before.

When the recitation was completed, Socrates requested that the first
thesis of the first argument might be read over again, and this having
been done, he said: What is your meaning, Zeno? Do you maintain that
if being is many, it must be both like and unlike, and that this is
impossible, for neither can the like be unlike, nor the unlike like--is
that your position?

Just so, said Zeno.

And if the unlike cannot be like, or the like unlike, then according to
you, being could not be many; for this would involve an impossibility.
In all that you say have you any other purpose except to disprove the
being of the many? and is not each division of your treatise intended to
furnish a separate proof of this, there being in all as many proofs of
the not-being of the many as you have composed arguments? Is that your
meaning, or have I misunderstood you?

No, said Zeno; you have correctly understood my general purpose.

I see, Parmenides, said Socrates, that Zeno would like to be not only
one with you in friendship but your second self in his writings too; he
puts what you say in another way, and would fain make believe that he is
telling us something which is new. For you, in your poems, say The All
is one, and of this you adduce excellent proofs; and he on the other
hand says There is no many; and on behalf of this he offers overwhelming
evidence. You affirm unity, he denies plurality. And so you deceive the
world into believing that you are saying different things when really
you are saying much the same. This is a strain of art beyond the reach
of most of us.

Yes, Socrates, said Zeno. But although you are as keen as a Spartan
hound in pursuing the track, you do not fully apprehend the true motive
of the composition, which is not really such an artificial work as you
imagine; for what you speak of was an accident; there was no pretence of
a great purpose; nor any serious intention of deceiving the world.
The truth is, that these writings of mine were meant to protect the
arguments of Parmenides against those who make fun of him and seek to
show the many ridiculous and contradictory results which they suppose
to follow from the affirmation of the one. My answer is addressed to the
partisans of the many, whose attack I return with interest by retorting
upon them that their hypothesis of the being of many, if carried out,
appears to be still more ridiculous than the hypothesis of the being
of one. Zeal for my master led me to write the book in the days of
my youth, but some one stole the copy; and therefore I had no choice
whether it should be published or not; the motive, however, of writing,
was not the ambition of an elder man, but the pugnacity of a young one.
This you do not seem to see, Socrates; though in other respects, as I
was saying, your notion is a very just one.

I understand, said Socrates, and quite accept your account. But tell
me, Zeno, do you not further think that there is an idea of likeness
in itself, and another idea of unlikeness, which is the opposite of
likeness, and that in these two, you and I and all other things to
which we apply the term many, participate--things which participate
in likeness become in that degree and manner like; and so far as they
participate in unlikeness become in that degree unlike, or both like and
unlike in the degree in which they participate in both? And may not all
things partake of both opposites, and be both like and unlike, by reason
of this participation?--Where is the wonder? Now if a person could prove
the absolute like to become unlike, or the absolute unlike to become
like, that, in my opinion, would indeed be a wonder; but there is
nothing extraordinary, Zeno, in showing that the things which only
partake of likeness and unlikeness experience both. Nor, again, if a
person were to show that all is one by partaking of one, and at the same
time many by partaking of many, would that be very astonishing. But if
he were to show me that the absolute one was many, or the absolute
many one, I should be truly amazed. And so of all the rest: I should
be surprised to hear that the natures or ideas themselves had these
opposite qualities; but not if a person wanted to prove of me that I was
many and also one. When he wanted to show that I was many he would say
that I have a right and a left side, and a front and a back, and an
upper and a lower half, for I cannot deny that I partake of multitude;
when, on the other hand, he wants to prove that I am one, he will say,
that we who are here assembled are seven, and that I am one and partake
of the one. In both instances he proves his case. So again, if a person
shows that such things as wood, stones, and the like, being many are
also one, we admit that he shows the coexistence of the one and many,
but he does not show that the many are one or the one many; he
is uttering not a paradox but a truism. If however, as I just now
suggested, some one were to abstract simple notions of like, unlike,
one, many, rest, motion, and similar ideas, and then to show that these
admit of admixture and separation in themselves, I should be very much
astonished. This part of the argument appears to be treated by you,
Zeno, in a very spirited manner; but, as I was saying, I should be
far more amazed if any one found in the ideas themselves which are
apprehended by reason, the same puzzle and entanglement which you have
shown to exist in visible objects.

While Socrates was speaking, Pythodorus thought that Parmenides and Zeno
were not altogether pleased at the successive steps of the argument; but
still they gave the closest attention, and often looked at one another,
and smiled as if in admiration of him. When he had finished, Parmenides
expressed their feelings in the following words:--

Socrates, he said, I admire the bent of your mind towards philosophy;
tell me now, was this your own distinction between ideas in themselves
and the things which partake of them? and do you think that there is an
idea of likeness apart from the likeness which we possess, and of the
one and many, and of the other things which Zeno mentioned?

I think that there are such ideas, said Socrates.

Parmenides proceeded: And would you also make absolute ideas of the just
and the beautiful and the good, and of all that class?

Yes, he said, I should.

And would you make an idea of man apart from us and from all other human
creatures, or of fire and water?

I am often undecided, Parmenides, as to whether I ought to include them
or not.

And would you feel equally undecided, Socrates, about things of which
the mention may provoke a smile?--I mean such things as hair, mud, dirt,
or anything else which is vile and paltry; would you suppose that each
of these has an idea distinct from the actual objects with which we come
into contact, or not?

Certainly not, said Socrates; visible things like these are such as
they appear to us, and I am afraid that there would be an absurdity in
assuming any idea of them, although I sometimes get disturbed, and begin
to think that there is nothing without an idea; but then again, when I
have taken up this position, I run away, because I am afraid that I may
fall into a bottomless pit of nonsense, and perish; and so I return to
the ideas of which I was just now speaking, and occupy myself with them.

Yes, Socrates, said Parmenides; that is because you are still young; the
time will come, if I am not mistaken, when philosophy will have a firmer
grasp of you, and then you will not despise even the meanest things; at
your age, you are too much disposed to regard the opinions of men. But
I should like to know whether you mean that there are certain ideas of
which all other things partake, and from which they derive their names;
that similars, for example, become similar, because they partake of
similarity; and great things become great, because they partake of
greatness; and that just and beautiful things become just and beautiful,
because they partake of justice and beauty?

Yes, certainly, said Socrates that is my meaning.

Then each individual partakes either of the whole of the idea or else of
a part of the idea? Can there be any other mode of participation?

There cannot be, he said.

Then do you think that the whole idea is one, and yet, being one, is in
each one of the many?

Why not, Parmenides? said Socrates.

Because one and the same thing will exist as a whole at the same time
in many separate individuals, and will therefore be in a state of
separation from itself.

Nay, but the idea may be like the day which is one and the same in many
places at once, and yet continuous with itself; in this way each idea
may be one and the same in all at the same time.

I like your way, Socrates, of making one in many places at once. You
mean to say, that if I were to spread out a sail and cover a number of
men, there would be one whole including many--is not that your meaning?

I think so.

And would you say that the whole sail includes each man, or a part of it
only, and different parts different men?

The latter.

Then, Socrates, the ideas themselves will be divisible, and things which
participate in them will have a part of them only and not the whole idea
existing in each of them?

That seems to follow.

Then would you like to say, Socrates, that the one idea is really
divisible and yet remains one?

Certainly not, he said.

Suppose that you divide absolute greatness, and that of the many great
things, each one is great in virtue of a portion of greatness less than
absolute greatness--is that conceivable?

No.

Or will each equal thing, if possessing some small portion of equality
less than absolute equality, be equal to some other thing by virtue of
that portion only?

Impossible.

Or suppose one of us to have a portion of smallness; this is but a part
of the small, and therefore the absolutely small is greater; if the
absolutely small be greater, that to which the part of the small is
added will be smaller and not greater than before.

How absurd!

Then in what way, Socrates, will all things participate in the ideas, if
they are unable to participate in them either as parts or wholes?

Indeed, he said, you have asked a question which is not easily answered.

Well, said Parmenides, and what do you say of another question?

What question?

I imagine that the way in which you are led to assume one idea of each
kind is as follows:--You see a number of great objects, and when you
look at them there seems to you to be one and the same idea (or nature)
in them all; hence you conceive of greatness as one.

Very true, said Socrates.

And if you go on and allow your mind in like manner to embrace in one
view the idea of greatness and of great things which are not the idea,
and to compare them, will not another greatness arise, which will appear
to be the source of all these?

It would seem so.

Then another idea of greatness now comes into view over and above
absolute greatness, and the individuals which partake of it; and then
another, over and above all these, by virtue of which they will all
be great, and so each idea instead of being one will be infinitely
multiplied.

But may not the ideas, asked Socrates, be thoughts only, and have no
proper existence except in our minds, Parmenides? For in that case each
idea may still be one, and not experience this infinite multiplication.

And can there be individual thoughts which are thoughts of nothing?

Impossible, he said.

The thought must be of something?

Yes.

Of something which is or which is not?

Of something which is.

Must it not be of a single something, which the thought recognizes as
attaching to all, being a single form or nature?

Yes.

And will not the something which is apprehended as one and the same in
all, be an idea?

From that, again, there is no escape.

Then, said Parmenides, if you say that everything else participates
in the ideas, must you not say either that everything is made up of
thoughts, and that all things think; or that they are thoughts but have
no thought?

The latter view, Parmenides, is no more rational than the previous one.
In my opinion, the ideas are, as it were, patterns fixed in nature, and
other things are like them, and resemblances of them--what is meant by
the participation of other things in the ideas, is really assimilation
to them.

But if, said he, the individual is like the idea, must not the idea also
be like the individual, in so far as the individual is a resemblance of
the idea? That which is like, cannot be conceived of as other than the
like of like.

Impossible.

And when two things are alike, must they not partake of the same idea?

They must.

And will not that of which the two partake, and which makes them alike,
be the idea itself?

Certainly.

Then the idea cannot be like the individual, or the individual like the
idea; for if they are alike, some further idea of likeness will always
be coming to light, and if that be like anything else, another; and new
ideas will be always arising, if the idea resembles that which partakes
of it?

Quite true.

The theory, then, that other things participate in the ideas by
resemblance, has to be given up, and some other mode of participation
devised?

It would seem so.

Do you see then, Socrates, how great is the difficulty of affirming the
ideas to be absolute?

Yes, indeed.

And, further, let me say that as yet you only understand a small part
of the difficulty which is involved if you make of each thing a single
idea, parting it off from other things.

What difficulty? he said.

There are many, but the greatest of all is this:--If an opponent argues
that these ideas, being such as we say they ought to be, must remain
unknown, no one can prove to him that he is wrong, unless he who denies
their existence be a man of great ability and knowledge, and is
willing to follow a long and laborious demonstration; he will remain
unconvinced, and still insist that they cannot be known.

What do you mean, Parmenides? said Socrates.

In the first place, I think, Socrates, that you, or any one who
maintains the existence of absolute essences, will admit that they
cannot exist in us.

No, said Socrates; for then they would be no longer absolute.

True, he said; and therefore when ideas are what they are in relation to
one another, their essence is determined by a relation among themselves,
and has nothing to do with the resemblances, or whatever they are to be
termed, which are in our sphere, and from which we receive this or that
name when we partake of them. And the things which are within our sphere
and have the same names with them, are likewise only relative to one
another, and not to the ideas which have the same names with them, but
belong to themselves and not to them.

What do you mean? said Socrates.

I may illustrate my meaning in this way, said Parmenides:--A master has
a slave; now there is nothing absolute in the relation between them,
which is simply a relation of one man to another. But there is also an
idea of mastership in the abstract, which is relative to the idea of
slavery in the abstract. These natures have nothing to do with us,
nor we with them; they are concerned with themselves only, and we with
ourselves. Do you see my meaning?

Yes, said Socrates, I quite see your meaning.

And will not knowledge--I mean absolute knowledge--answer to absolute
truth?

Certainly.

And each kind of absolute knowledge will answer to each kind of absolute
being?

Yes.

But the knowledge which we have, will answer to the truth which we have;
and again, each kind of knowledge which we have, will be a knowledge of
each kind of being which we have?

Certainly.

But the ideas themselves, as you admit, we have not, and cannot have?

No, we cannot.

And the absolute natures or kinds are known severally by the absolute
idea of knowledge?

Yes.

And we have not got the idea of knowledge?

No.

Then none of the ideas are known to us, because we have no share in
absolute knowledge?

I suppose not.

Then the nature of the beautiful in itself, and of the good in itself,
and all other ideas which we suppose to exist absolutely, are unknown to
us?

It would seem so.

I think that there is a stranger consequence still.

What is it?

Would you, or would you not say, that absolute knowledge, if there is
such a thing, must be a far more exact knowledge than our knowledge; and
the same of beauty and of the rest?

Yes.

And if there be such a thing as participation in absolute knowledge, no
one is more likely than God to have this most exact knowledge?

Certainly.

But then, will God, having absolute knowledge, have a knowledge of human
things?

Why not?

Because, Socrates, said Parmenides, we have admitted that the ideas are
not valid in relation to human things; nor human things in relation to
them; the relations of either are limited to their respective spheres.

Yes, that has been admitted.

And if God has this perfect authority, and perfect knowledge, his
authority cannot rule us, nor his knowledge know us, or any human thing;
just as our authority does not extend to the gods, nor our knowledge
know anything which is divine, so by parity of reason they, being gods,
are not our masters, neither do they know the things of men.

Yet, surely, said Socrates, to deprive God of knowledge is monstrous.

These, Socrates, said Parmenides, are a few, and only a few of the
difficulties in which we are involved if ideas really are and we
determine each one of them to be an absolute unity. He who hears what
may be said against them will deny the very existence of them--and even
if they do exist, he will say that they must of necessity be unknown
to man; and he will seem to have reason on his side, and as we were
remarking just now, will be very difficult to convince; a man must
be gifted with very considerable ability before he can learn that
everything has a class and an absolute essence; and still more
remarkable will he be who discovers all these things for himself, and
having thoroughly investigated them is able to teach them to others.

I agree with you, Parmenides, said Socrates; and what you say is very
much to my mind.

And yet, Socrates, said Parmenides, if a man, fixing his attention on
these and the like difficulties, does away with ideas of things and will
not admit that every individual thing has its own determinate idea which
is always one and the same, he will have nothing on which his mind can
rest; and so he will utterly destroy the power of reasoning, as you seem
to me to have particularly noted.

Very true, he said.

But, then, what is to become of philosophy? Whither shall we turn, if
the ideas are unknown?

I certainly do not see my way at present.

Yes, said Parmenides; and I think that this arises, Socrates, out of
your attempting to define the beautiful, the just, the good, and the
ideas generally, without sufficient previous training. I noticed your
deficiency, when I heard you talking here with your friend Aristoteles,
the day before yesterday. The impulse that carries you towards
philosophy is assuredly noble and divine; but there is an art which is
called by the vulgar idle talking, and which is often imagined to be
useless; in that you must train and exercise yourself, now that you are
young, or truth will elude your grasp.

And what is the nature of this exercise, Parmenides, which you would
recommend?

That which you heard Zeno practising; at the same time, I give you
credit for saying to him that you did not care to examine the perplexity
in reference to visible things, or to consider the question that way;
but only in reference to objects of thought, and to what may be called
ideas.

Why, yes, he said, there appears to me to be no difficulty in showing by
this method that visible things are like and unlike and may experience
anything.

Quite true, said Parmenides; but I think that you should go a step
further, and consider not only the consequences which flow from a
given hypothesis, but also the consequences which flow from denying the
hypothesis; and that will be still better training for you.

What do you mean? he said.

I mean, for example, that in the case of this very hypothesis of
Zeno's about the many, you should inquire not only what will be the
consequences to the many in relation to themselves and to the one, and
to the one in relation to itself and the many, on the hypothesis of the
being of the many, but also what will be the consequences to the one
and the many in their relation to themselves and to each other, on the
opposite hypothesis. Or, again, if likeness is or is not, what will
be the consequences in either of these cases to the subjects of the
hypothesis, and to other things, in relation both to themselves and to
one another, and so of unlikeness; and the same holds good of motion and
rest, of generation and destruction, and even of being and not-being.
In a word, when you suppose anything to be or not to be, or to be in any
way affected, you must look at the consequences in relation to the
thing itself, and to any other things which you choose,--to each of them
singly, to more than one, and to all; and so of other things, you must
look at them in relation to themselves and to anything else which you
suppose either to be or not to be, if you would train yourself perfectly
and see the real truth.

That, Parmenides, is a tremendous business of which you speak, and I do
not quite understand you; will you take some hypothesis and go through
the steps?--then I shall apprehend you better.

That, Socrates, is a serious task to impose on a man of my years.

Then will you, Zeno? said Socrates.

Zeno answered with a smile:--Let us make our petition to Parmenides
himself, who is quite right in saying that you are hardly aware of the
extent of the task which you are imposing on him; and if there were more
of us I should not ask him, for these are not subjects which any one,
especially at his age, can well speak of before a large audience; most
people are not aware that this roundabout progress through all things
is the only way in which the mind can attain truth and wisdom. And
therefore, Parmenides, I join in the request of Socrates, that I may
hear the process again which I have not heard for a long time.

When Zeno had thus spoken, Pythodorus, according to Antiphon's report
of him, said, that he himself and Aristoteles and the whole company
entreated Parmenides to give an example of the process. I cannot refuse,
said Parmenides; and yet I feel rather like Ibycus, who, when in his
old age, against his will, he fell in love, compared himself to an old
racehorse, who was about to run in a chariot race, shaking with fear at
the course he knew so well--this was his simile of himself. And I also
experience a trembling when I remember through what an ocean of words
I have to wade at my time of life. But I must indulge you, as Zeno says
that I ought, and we are alone. Where shall I begin? And what shall be
our first hypothesis, if I am to attempt this laborious pastime? Shall I
begin with myself, and take my own hypothesis the one? and consider the
consequences which follow on the supposition either of the being or of
the not-being of one?

By all means, said Zeno.

And who will answer me? he said. Shall I propose the youngest? He will
not make difficulties and will be the most likely to say what he thinks;
and his answers will give me time to breathe.

I am the one whom you mean, Parmenides, said Aristoteles; for I am the
youngest and at your service. Ask, and I will answer.

Parmenides proceeded: 1.a. If one is, he said, the one cannot be many?

Impossible.

Then the one cannot have parts, and cannot be a whole?

Why not?

Because every part is part of a whole; is it not?

Yes.

And what is a whole? would not that of which no part is wanting be a
whole?

Certainly.

Then, in either case, the one would be made up of parts; both as being a
whole, and also as having parts?

To be sure.

And in either case, the one would be many, and not one?

True.

But, surely, it ought to be one and not many?

It ought.

Then, if the one is to remain one, it will not be a whole, and will not
have parts?

No.

But if it has no parts, it will have neither beginning, middle, nor end;
for these would of course be parts of it.

Right.

But then, again, a beginning and an end are the limits of everything?

Certainly.

Then the one, having neither beginning nor end, is unlimited?

Yes, unlimited.

And therefore formless; for it cannot partake either of round or
straight.

But why?

Why, because the round is that of which all the extreme points are
equidistant from the centre?

Yes.

And the straight is that of which the centre intercepts the view of the
extremes?

True.

Then the one would have parts and would be many, if it partook either of
a straight or of a circular form?

Assuredly.

But having no parts, it will be neither straight nor round?

Right.

And, being of such a nature, it cannot be in any place, for it cannot be
either in another or in itself.

How so?

Because if it were in another, it would be encircled by that in which
it was, and would touch it at many places and with many parts; but that
which is one and indivisible, and does not partake of a circular nature,
cannot be touched all round in many places.

Certainly not.

But if, on the other hand, one were in itself, it would also be
contained by nothing else but itself; that is to say, if it were really
in itself; for nothing can be in anything which does not contain it.

Impossible.

But then, that which contains must be other than that which is
contained? for the same whole cannot do and suffer both at once; and if
so, one will be no longer one, but two?

True.

Then one cannot be anywhere, either in itself or in another?

No.

Further consider, whether that which is of such a nature can have either
rest or motion.

Why not?

Why, because the one, if it were moved, would be either moved in place
or changed in nature; for these are the only kinds of motion.

Yes.

And the one, when it changes and ceases to be itself, cannot be any
longer one.

It cannot.

It cannot therefore experience the sort of motion which is change of
nature?

Clearly not.

Then can the motion of the one be in place?

Perhaps.

But if the one moved in place, must it not either move round and round
in the same place, or from one place to another?

It must.

And that which moves in a circle must rest upon a centre; and that which
goes round upon a centre must have parts which are different from the
centre; but that which has no centre and no parts cannot possibly be
carried round upon a centre?

Impossible.

But perhaps the motion of the one consists in change of place?

Perhaps so, if it moves at all.

And have we not already shown that it cannot be in anything?

Yes.

Then its coming into being in anything is still more impossible; is it
not?

I do not see why.

Why, because anything which comes into being in anything, can neither
as yet be in that other thing while still coming into being, nor be
altogether out of it, if already coming into being in it.

Certainly not.

And therefore whatever comes into being in another must have parts, and
then one part may be in, and another part out of that other; but that
which has no parts can never be at one and the same time neither wholly
within nor wholly without anything.

True.

And is there not a still greater impossibility in that which has no
parts, and is not a whole, coming into being anywhere, since it cannot
come into being either as a part or as a whole?

Clearly.

Then it does not change place by revolving in the same spot, nor by
going somewhere and coming into being in something; nor again, by change
in itself?

Very true.

Then in respect of any kind of motion the one is immoveable?

Immoveable.

But neither can the one be in anything, as we affirm?

Yes, we said so.

Then it is never in the same?

Why not?

Because if it were in the same it would be in something.

Certainly.

And we said that it could not be in itself, and could not be in other?

True.

Then one is never in the same place?

It would seem not.

But that which is never in the same place is never quiet or at rest?

Never.

One then, as would seem, is neither at rest nor in motion?

It certainly appears so.

Neither will it be the same with itself or other; nor again, other than
itself or other.

How is that?

If other than itself it would be other than one, and would not be one.

True.

And if the same with other, it would be that other, and not itself; so
that upon this supposition too, it would not have the nature of one, but
would be other than one?

It would.

Then it will not be the same with other, or other than itself?

It will not.

Neither will it be other than other, while it remains one; for not one,
but only other, can be other than other, and nothing else.

True.

Then not by virtue of being one will it be other?

Certainly not.

But if not by virtue of being one, not by virtue of itself; and if not
by virtue of itself, not itself, and itself not being other at all, will
not be other than anything?

Right.

Neither will one be the same with itself.

How not?

Surely the nature of the one is not the nature of the same.

Why not?

It is not when anything becomes the same with anything that it becomes
one.

What of that?

Anything which becomes the same with the many, necessarily becomes many
and not one.

True.

But, if there were no difference between the one and the same, when a
thing became the same, it would always become one; and when it became
one, the same?

Certainly.

And, therefore, if one be the same with itself, it is not one with
itself, and will therefore be one and also not one.

Surely that is impossible.

And therefore the one can neither be other than other, nor the same with
itself.

Impossible.

And thus the one can neither be the same, nor other, either in relation
to itself or other?

No.

Neither will the one be like anything or unlike itself or other.

Why not?

Because likeness is sameness of affections.

Yes.

And sameness has been shown to be of a nature distinct from oneness?

That has been shown.

But if the one had any other affection than that of being one, it would
be affected in such a way as to be more than one; which is impossible.

True.

Then the one can never be so affected as to be the same either with
another or with itself?

Clearly not.

Then it cannot be like another, or like itself?

No.

Nor can it be affected so as to be other, for then it would be affected
in such a way as to be more than one.

It would.

That which is affected otherwise than itself or another, will be unlike
itself or another, for sameness of affections is likeness.

True.

But the one, as appears, never being affected otherwise, is never unlike
itself or other?

Never.

Then the one will never be either like or unlike itself or other?

Plainly not.

Again, being of this nature, it can neither be equal nor unequal either
to itself or to other.

How is that?

Why, because the one if equal must be of the same measures as that to
which it is equal.

True.

And if greater or less than things which are commensurable with it, the
one will have more measures than that which is less, and fewer than that
which is greater?

Yes.

And so of things which are not commensurate with it, the one will have
greater measures than that which is less and smaller than that which is
greater.

Certainly.

But how can that which does not partake of sameness, have either the
same measures or have anything else the same?

Impossible.

And not having the same measures, the one cannot be equal either with
itself or with another?

It appears so.

But again, whether it have fewer or more measures, it will have as many
parts as it has measures; and thus again the one will be no longer one
but will have as many parts as measures.

Right.

And if it were of one measure, it would be equal to that measure; yet it
has been shown to be incapable of equality.

It has.

Then it will neither partake of one measure, nor of many, nor of few,
nor of the same at all, nor be equal to itself or another; nor be
greater or less than itself, or other?

Certainly.

Well, and do we suppose that one can be older, or younger than anything,
or of the same age with it?

Why not?

Why, because that which is of the same age with itself or other, must
partake of equality or likeness of time; and we said that the one did
not partake either of equality or of likeness?

We did say so.

And we also said, that it did not partake of inequality or unlikeness.

Very true.

How then can one, being of this nature, be either older or younger than
anything, or have the same age with it?

In no way.

Then one cannot be older or younger, or of the same age, either with
itself or with another?

Clearly not.

Then the one, being of this nature, cannot be in time at all; for must
not that which is in time, be always growing older than itself?

Certainly.

And that which is older, must always be older than something which is
younger?

True.

Then, that which becomes older than itself, also becomes at the same
time younger than itself, if it is to have something to become older
than.

What do you mean?

I mean this:--A thing does not need to become different from another
thing which is already different; it IS different, and if its different
has become, it has become different; if its different will be, it will
be different; but of that which is becoming different, there cannot
have been, or be about to be, or yet be, a different--the only different
possible is one which is becoming.

That is inevitable.

But, surely, the elder is a difference relative to the younger, and to
nothing else.

True.

Then that which becomes older than itself must also, at the same time,
become younger than itself?

Yes.

But again, it is true that it cannot become for a longer or for a
shorter time than itself, but it must become, and be, and have become,
and be about to be, for the same time with itself?

That again is inevitable.

Then things which are in time, and partake of time, must in every case,
I suppose, be of the same age with themselves; and must also become at
once older and younger than themselves?

Yes.

But the one did not partake of those affections?

Not at all.

Then it does not partake of time, and is not in any time?

So the argument shows.

Well, but do not the expressions 'was,' and 'has become,' and 'was
becoming,' signify a participation of past time?

Certainly.

And do not 'will be,' 'will become,' 'will have become,' signify a
participation of future time?

Yes.

And 'is,' or 'becomes,' signifies a participation of present time?

Certainly.

And if the one is absolutely without participation in time, it never
had become, or was becoming, or was at any time, or is now become or
is becoming, or is, or will become, or will have become, or will be,
hereafter.

Most true.

But are there any modes of partaking of being other than these?

There are none.

Then the one cannot possibly partake of being?

That is the inference.

Then the one is not at all?

Clearly not.

Then the one does not exist in such way as to be one; for if it were
and partook of being, it would already be; but if the argument is to be
trusted, the one neither is nor is one?

True.

But that which is not admits of no attribute or relation?

Of course not.

Then there is no name, nor expression, nor perception, nor opinion, nor
knowledge of it?

Clearly not.

Then it is neither named, nor expressed, nor opined, nor known, nor does
anything that is perceive it.

So we must infer.

But can all this be true about the one?

I think not.

1.b. Suppose, now, that we return once more to the original hypothesis;
let us see whether, on a further review, any new aspect of the question
appears.

I shall be very happy to do so.

We say that we have to work out together all the consequences, whatever
they may be, which follow, if the one is?

Yes.

Then we will begin at the beginning:--If one is, can one be, and not
partake of being?

Impossible.

Then the one will have being, but its being will not be the same with
the one; for if the same, it would not be the being of the one; nor
would the one have participated in being, for the proposition that one
is would have been identical with the proposition that one is one;
but our hypothesis is not if one is one, what will follow, but if one
is:--am I not right?

Quite right.

We mean to say, that being has not the same significance as one?

Of course.

And when we put them together shortly, and say 'One is,' that is
equivalent to saying, 'partakes of being'?

Quite true.

Once more then let us ask, if one is what will follow. Does not this
hypothesis necessarily imply that one is of such a nature as to have
parts?

How so?

In this way:--If being is predicated of the one, if the one is, and one
of being, if being is one; and if being and one are not the same; and
since the one, which we have assumed, is, must not the whole, if it is
one, itself be, and have for its parts, one and being?

Certainly.

And is each of these parts--one and being--to be simply called a part,
or must the word 'part' be relative to the word 'whole'?

The latter.

Then that which is one is both a whole and has a part?

Certainly.

Again, of the parts of the one, if it is--I mean being and one--does
either fail to imply the other? is the one wanting to being, or being to
the one?

Impossible.

Thus, each of the parts also has in turn both one and being, and is at
the least made up of two parts; and the same principle goes on for ever,
and every part whatever has always these two parts; for being always
involves one, and one being; so that one is always disappearing, and
becoming two.

Certainly.

And so the one, if it is, must be infinite in multiplicity?

Clearly.

Let us take another direction.

What direction?

We say that the one partakes of being and therefore it is?

Yes.

And in this way, the one, if it has being, has turned out to be many?

True.

But now, let us abstract the one which, as we say, partakes of
being, and try to imagine it apart from that of which, as we say, it
partakes--will this abstract one be one only or many?

One, I think.

Let us see:--Must not the being of one be other than one? for the one is
not being, but, considered as one, only partook of being?

Certainly.

If being and the one be two different things, it is not because the one
is one that it is other than being; nor because being is being that it
is other than the one; but they differ from one another in virtue of
otherness and difference.

Certainly.

So that the other is not the same--either with the one or with being?

Certainly not.

And therefore whether we take being and the other, or being and the one,
or the one and the other, in every such case we take two things, which
may be rightly called both.

How so.

In this way--you may speak of being?

Yes.

And also of one?

Yes.

Then now we have spoken of either of them?

Yes.

Well, and when I speak of being and one, I speak of them both?

Certainly.

And if I speak of being and the other, or of the one and the other,--in
any such case do I not speak of both?

Yes.

And must not that which is correctly called both, be also two?

Undoubtedly.

And of two things how can either by any possibility not be one?

It cannot.

Then, if the individuals of the pair are together two, they must be
severally one?

Clearly.

And if each of them is one, then by the addition of any one to any pair,
the whole becomes three?

Yes.

And three are odd, and two are even?

Of course.

And if there are two there must also be twice, and if there are three
there must be thrice; that is, if twice one makes two, and thrice one
three?

Certainly.

There are two, and twice, and therefore there must be twice two; and
there are three, and there is thrice, and therefore there must be thrice
three?

Of course.

If there are three and twice, there is twice three; and if there are two
and thrice, there is thrice two?

Undoubtedly.

Here, then, we have even taken even times, and odd taken odd times, and
even taken odd times, and odd taken even times.

True.

And if this is so, does any number remain which has no necessity to be?

None whatever.

Then if one is, number must also be?

It must.

But if there is number, there must also be many, and infinite
multiplicity of being; for number is infinite in multiplicity, and
partakes also of being: am I not right?

Certainly.

And if all number participates in being, every part of number will also
participate?

Yes.

Then being is distributed over the whole multitude of things, and
nothing that is, however small or however great, is devoid of it? And,
indeed, the very supposition of this is absurd, for how can that which
is, be devoid of being?

In no way.

And it is divided into the greatest and into the smallest, and into
being of all sizes, and is broken up more than all things; the divisions
of it have no limit.

True.

Then it has the greatest number of parts?

Yes, the greatest number.

Is there any of these which is a part of being, and yet no part?

Impossible.

But if it is at all and so long as it is, it must be one, and cannot be
none?

Certainly.

Then the one attaches to every single part of being, and does not fail
in any part, whether great or small, or whatever may be the size of it?

True.

But reflect:--Can one, in its entirety, be in many places at the same
time?

No; I see the impossibility of that.

And if not in its entirety, then it is divided; for it cannot be present
with all the parts of being, unless divided.

True.

And that which has parts will be as many as the parts are?

Certainly.

Then we were wrong in saying just now, that being was distributed into
the greatest number of parts. For it is not distributed into parts more
than the one, into parts equal to the one; the one is never wanting
to being, or being to the one, but being two they are co-equal and
co-extensive.

Certainly that is true.

The one itself, then, having been broken up into parts by being, is many
and infinite?

True.

Then not only the one which has being is many, but the one itself
distributed by being, must also be many?

Certainly.

Further, inasmuch as the parts are parts of a whole, the one, as a
whole, will be limited; for are not the parts contained by the whole?

Certainly.

And that which contains, is a limit?

Of course.

Then the one if it has being is one and many, whole and parts, having
limits and yet unlimited in number?

Clearly.

And because having limits, also having extremes?

Certainly.

And if a whole, having beginning and middle and end. For can anything
be a whole without these three? And if any one of them is wanting to
anything, will that any longer be a whole?

No.

Then the one, as appears, will have beginning, middle, and end.

It will.

But, again, the middle will be equidistant from the extremes; or it
would not be in the middle?

Yes.

Then the one will partake of figure, either rectilinear or round, or a
union of the two?

True.

And if this is the case, it will be both in itself and in another too.

How?

Every part is in the whole, and none is outside the whole.

True.

And all the parts are contained by the whole?

Yes.

And the one is all its parts, and neither more nor less than all?

No.

And the one is the whole?

Of course.

But if all the parts are in the whole, and the one is all of them and
the whole, and they are all contained by the whole, the one will be
contained by the one; and thus the one will be in itself.

That is true.

But then, again, the whole is not in the parts--neither in all the
parts, nor in some one of them. For if it is in all, it must be in one;
for if there were any one in which it was not, it could not be in all
the parts; for the part in which it is wanting is one of all, and if the
whole is not in this, how can it be in them all?

It cannot.

Nor can the whole be in some of the parts; for if the whole were in some
of the parts, the greater would be in the less, which is impossible.

Yes, impossible.

But if the whole is neither in one, nor in more than one, nor in all of
the parts, it must be in something else, or cease to be anywhere at all?

Certainly.

If it were nowhere, it would be nothing; but being a whole, and not
being in itself, it must be in another.

Very true.

The one then, regarded as a whole, is in another, but regarded as being
all its parts, is in itself; and therefore the one must be itself in
itself and also in another.

Certainly.

The one then, being of this nature, is of necessity both at rest and in
motion?

How?

The one is at rest since it is in itself, for being in one, and not
passing out of this, it is in the same, which is itself.

True.

And that which is ever in the same, must be ever at rest?

Certainly.

Well, and must not that, on the contrary, which is ever in other, never
be in the same; and if never in the same, never at rest, and if not at
rest, in motion?

True.

Then the one being always itself in itself and other, must always be
both at rest and in motion?

Clearly.

And must be the same with itself, and other than itself; and also the
same with the others, and other than the others; this follows from its
previous affections.

How so?

Everything in relation to every other thing, is either the same or
other; or if neither the same nor other, then in the relation of a part
to a whole, or of a whole to a part.

Clearly.

And is the one a part of itself?

Certainly not.

Since it is not a part in relation to itself it cannot be related to
itself as whole to part?

It cannot.

But is the one other than one?

No.

And therefore not other than itself?

Certainly not.

If then it be neither other, nor a whole, nor a part in relation to
itself, must it not be the same with itself?

Certainly.

But then, again, a thing which is in another place from 'itself,' if
this 'itself' remains in the same place with itself, must be other than
'itself,' for it will be in another place?

True.

Then the one has been shown to be at once in itself and in another?

Yes.

Thus, then, as appears, the one will be other than itself?

True.

Well, then, if anything be other than anything, will it not be other
than that which is other?

Certainly.

And will not all things that are not one, be other than the one, and the
one other than the not-one?

Of course.

Then the one will be other than the others?

True.

But, consider:--Are not the absolute same, and the absolute other,
opposites to one another?

Of course.

Then will the same ever be in the other, or the other in the same?

They will not.

If then the other is never in the same, there is nothing in which
the other is during any space of time; for during that space of time,
however small, the other would be in the same. Is not that true?

Yes.

And since the other is never in the same, it can never be in anything
that is.

True.

Then the other will never be either in the not-one, or in the one?

Certainly not.

Then not by reason of otherness is the one other than the not-one, or
the not-one other than the one.

No.

Nor by reason of themselves will they be other than one another, if not
partaking of the other.

How can they be?

But if they are not other, either by reason of themselves or of the
other, will they not altogether escape being other than one another?

They will.

Again, the not-one cannot partake of the one; otherwise it would not
have been not-one, but would have been in some way one.

True.

Nor can the not-one be number; for having number, it would not have been
not-one at all.

It would not.

Again, is the not-one part of the one; or rather, would it not in that
case partake of the one?

It would.

If then, in every point of view, the one and the not-one are distinct,
then neither is the one part or whole of the not-one, nor is the not-one
part or whole of the one?

No.

But we said that things which are neither parts nor wholes of one
another, nor other than one another, will be the same with one
another:--so we said?

Yes.

Then shall we say that the one, being in this relation to the not-one,
is the same with it?

Let us say so.

Then it is the same with itself and the others, and also other than
itself and the others.

That appears to be the inference.

And it will also be like and unlike itself and the others?

Perhaps.

Since the one was shown to be other than the others, the others will
also be other than the one.

Yes.

And the one is other than the others in the same degree that the others
are other than it, and neither more nor less?

True.

And if neither more nor less, then in a like degree?

Yes.

In virtue of the affection by which the one is other than others and
others in like manner other than it, the one will be affected like the
others and the others like the one.

How do you mean?

I may take as an illustration the case of names: You give a name to a
thing?

Yes.

And you may say the name once or oftener?

Yes.

And when you say it once, you mention that of which it is the name? and
when more than once, is it something else which you mention? or must it
always be the same thing of which you speak, whether you utter the name
once or more than once?

Of course it is the same.

And is not 'other' a name given to a thing?

Certainly.

Whenever, then, you use the word 'other,' whether once or oftener, you
name that of which it is the name, and to no other do you give the name?

True.

Then when we say that the others are other than the one, and the one
other than the others, in repeating the word 'other' we speak of that
nature to which the name is applied, and of no other?

Quite true.

Then the one which is other than others, and the other which is other
than the one, in that the word 'other' is applied to both, will be in
the same condition; and that which is in the same condition is like?

Yes.

Then in virtue of the affection by which the one is other than the
others, every thing will be like every thing, for every thing is other
than every thing.

True.

Again, the like is opposed to the unlike?

Yes.

And the other to the same?

True again.

And the one was also shown to be the same with the others?

Yes.

And to be the same with the others is the opposite of being other than
the others?

Certainly.

And in that it was other it was shown to be like?

Yes.

But in that it was the same it will be unlike by virtue of the opposite
affection to that which made it like; and this was the affection of
otherness.

Yes.

The same then will make it unlike; otherwise it will not be the opposite
of the other.

True.

Then the one will be both like and unlike the others; like in so far as
it is other, and unlike in so far as it is the same.

Yes, that argument may be used.

And there is another argument.

What?

In so far as it is affected in the same way it is not affected
otherwise, and not being affected otherwise is not unlike, and not
being unlike, is like; but in so far as it is affected by other it is
otherwise, and being otherwise affected is unlike.

True.

Then because the one is the same with the others and other than the
others, on either of these two grounds, or on both of them, it will be
both like and unlike the others?

Certainly.

And in the same way as being other than itself and the same with itself,
on either of these two grounds and on both of them, it will be like and
unlike itself?

Of course.

Again, how far can the one touch or not touch itself and
others?--consider.

I am considering.

The one was shown to be in itself which was a whole?

True.

And also in other things?

Yes.

In so far as it is in other things it would touch other things, but in
so far as it is in itself it would be debarred from touching them, and
would touch itself only.

Clearly.

Then the inference is that it would touch both?

It would.

But what do you say to a new point of view? Must not that which is to
touch another be next to that which it is to touch, and occupy the place
nearest to that in which what it touches is situated?

True.

Then the one, if it is to touch itself, ought to be situated next to
itself, and occupy the place next to that in which itself is?

It ought.

And that would require that the one should be two, and be in two places
at once, and this, while it is one, will never happen.

No.

Then the one cannot touch itself any more than it can be two?

It cannot.

Neither can it touch others.

Why not?

The reason is, that whatever is to touch another must be in separation
from, and next to, that which it is to touch, and no third thing can be
between them.

True.

Two things, then, at the least are necessary to make contact possible?

They are.

And if to the two a third be added in due order, the number of terms
will be three, and the contacts two?

Yes.

And every additional term makes one additional contact, whence it
follows that the contacts are one less in number than the terms; the
first two terms exceeded the number of contacts by one, and the whole
number of terms exceeds the whole number of contacts by one in like
manner; and for every one which is afterwards added to the number of
terms, one contact is added to the contacts.

True.

Whatever is the whole number of things, the contacts will be always one
less.

True.

But if there be only one, and not two, there will be no contact?

How can there be?

And do we not say that the others being other than the one are not one
and have no part in the one?

True.

Then they have no number, if they have no one in them?

Of course not.

Then the others are neither one nor two, nor are they called by the name
of any number?

No.

One, then, alone is one, and two do not exist?

Clearly not.

And if there are not two, there is no contact?

There is not.

Then neither does the one touch the others, nor the others the one, if
there is no contact?

Certainly not.

For all which reasons the one touches and does not touch itself and the
others?

True.

Further--is the one equal and unequal to itself and others?

How do you mean?

If the one were greater or less than the others, or the others greater
or less than the one, they would not be greater or less than each other
in virtue of their being the one and the others; but, if in addition to
their being what they are they had equality, they would be equal to one
another, or if the one had smallness and the others greatness, or the
one had greatness and the others smallness--whichever kind had greatness
would be greater, and whichever had smallness would be smaller?

Certainly.

Then there are two such ideas as greatness and smallness; for if they
were not they could not be opposed to each other and be present in that
which is.

How could they?

If, then, smallness is present in the one it will be present either in
the whole or in a part of the whole?

Certainly.

Suppose the first; it will be either co-equal and co-extensive with the
whole one, or will contain the one?

Clearly.

If it be co-extensive with the one it will be co-equal with the one, or
if containing the one it will be greater than the one?

Of course.

But can smallness be equal to anything or greater than anything, and
have the functions of greatness and equality and not its own functions?

Impossible.

Then smallness cannot be in the whole of one, but, if at all, in a part
only?

Yes.

And surely not in all of a part, for then the difficulty of the whole
will recur; it will be equal to or greater than any part in which it is.

Certainly.

Then smallness will not be in anything, whether in a whole or in a part;
nor will there be anything small but actual smallness.

True.

Neither will greatness be in the one, for if greatness be in anything
there will be something greater other and besides greatness itself,
namely, that in which greatness is; and this too when the small itself
is not there, which the one, if it is great, must exceed; this, however,
is impossible, seeing that smallness is wholly absent.

True.

But absolute greatness is only greater than absolute smallness, and
smallness is only smaller than absolute greatness.

Very true.

Then other things not greater or less than the one, if they have neither
greatness nor smallness; nor have greatness or smallness any power of
exceeding or being exceeded in relation to the one, but only in relation
to one another; nor will the one be greater or less than them or others,
if it has neither greatness nor smallness.

Clearly not.

Then if the one is neither greater nor less than the others, it cannot
either exceed or be exceeded by them?

Certainly not.

And that which neither exceeds nor is exceeded, must be on an equality;
and being on an equality, must be equal.

Of course.

And this will be true also of the relation of the one to itself; having
neither greatness nor smallness in itself, it will neither exceed nor be
exceeded by itself, but will be on an equality with and equal to itself.

Certainly.

Then the one will be equal both to itself and the others?

Clearly so.

And yet the one, being itself in itself, will also surround and be
without itself; and, as containing itself, will be greater than itself;
and, as contained in itself, will be less; and will thus be greater and
less than itself.

It will.

Now there cannot possibly be anything which is not included in the one
and the others?

Of course not.

But, surely, that which is must always be somewhere?

Yes.

But that which is in anything will be less, and that in which it is will
be greater; in no other way can one thing be in another.

True.

And since there is nothing other or besides the one and the others, and
they must be in something, must they not be in one another, the one in
the others and the others in the one, if they are to be anywhere?

That is clear.

But inasmuch as the one is in the others, the others will be greater
than the one, because they contain the one, which will be less than the
others, because it is contained in them; and inasmuch as the others
are in the one, the one on the same principle will be greater than the
others, and the others less than the one.

True.

The one, then, will be equal to and greater and less than itself and the
others?

Clearly.

And if it be greater and less and equal, it will be of equal and more
and less measures or divisions than itself and the others, and if of
measures, also of parts?

Of course.

And if of equal and more and less measures or divisions, it will be in
number more or less than itself and the others, and likewise equal in
number to itself and to the others?

How is that?

It will be of more measures than those things which it exceeds, and of
as many parts as measures; and so with that to which it is equal, and
that than which it is less.

True.

And being greater and less than itself, and equal to itself, it will
be of equal measures with itself and of more and fewer measures than
itself; and if of measures then also of parts?

It will.

And being of equal parts with itself, it will be numerically equal to
itself; and being of more parts, more, and being of less, less than
itself?

Certainly.

And the same will hold of its relation to other things; inasmuch as it
is greater than them, it will be more in number than them; and inasmuch
as it is smaller, it will be less in number; and inasmuch as it is equal
in size to other things, it will be equal to them in number.

Certainly.

Once more, then, as would appear, the one will be in number both equal
to and more and less than both itself and all other things.

It will.

Does the one also partake of time? And is it and does it become older
and younger than itself and others, and again, neither younger nor older
than itself and others, by virtue of participation in time?

How do you mean?

If one is, being must be predicated of it?

Yes.

But to be (einai) is only participation of being in present time, and to
have been is the participation of being at a past time, and to be about
to be is the participation of being at a future time?

Very true.

Then the one, since it partakes of being, partakes of time?

Certainly.

And is not time always moving forward?

Yes.

Then the one is always becoming older than itself, since it moves
forward in time?

Certainly.

And do you remember that the older becomes older than that which becomes
younger?

I remember.

Then since the one becomes older than itself, it becomes younger at the
same time?

Certainly.

Thus, then, the one becomes older as well as younger than itself?

Yes.

And it is older (is it not?) when in becoming, it gets to the point of
time between 'was' and 'will be,' which is 'now': for surely in going
from the past to the future, it cannot skip the present?

No.

And when it arrives at the present it stops from becoming older, and
no longer becomes, but is older, for if it went on it would never be
reached by the present, for it is the nature of that which goes on,
to touch both the present and the future, letting go the present and
seizing the future, while in process of becoming between them.

True.

But that which is becoming cannot skip the present; when it reaches the
present it ceases to become, and is then whatever it may happen to be
becoming.

Clearly.

And so the one, when in becoming older it reaches the present, ceases to
become, and is then older.

Certainly.

And it is older than that than which it was becoming older, and it was
becoming older than itself.

Yes.

And that which is older is older than that which is younger?

True.

Then the one is younger than itself, when in becoming older it reaches
the present?

Certainly.

But the present is always present with the one during all its being; for
whenever it is it is always now.

Certainly.

Then the one always both is and becomes older and younger than itself?

Truly.

And is it or does it become a longer time than itself or an equal time
with itself?

An equal time.

But if it becomes or is for an equal time with itself, it is of the same
age with itself?

Of course.

And that which is of the same age, is neither older nor younger?

No.

The one, then, becoming and being the same time with itself, neither is
nor becomes older or younger than itself?

I should say not.

And what are its relations to other things? Is it or does it become
older or younger than they?

I cannot tell you.

You can at least tell me that others than the one are more than the
one--other would have been one, but the others have multitude, and are
more than one?

They will have multitude.

And a multitude implies a number larger than one?

Of course.

And shall we say that the lesser or the greater is the first to come or
to have come into existence?

The lesser.

Then the least is the first? And that is the one?

Yes.

Then the one of all things that have number is the first to come into
being; but all other things have also number, being plural and not
singular.

They have.

And since it came into being first it must be supposed to have come into
being prior to the others, and the others later; and the things which
came into being later, are younger than that which preceded them? And
so the other things will be younger than the one, and the one older than
other things?

True.

What would you say of another question? Can the one have come into being
contrary to its own nature, or is that impossible?

Impossible.

And yet, surely, the one was shown to have parts; and if parts, then a
beginning, middle and end?

Yes.

And a beginning, both of the one itself and of all other things, comes
into being first of all; and after the beginning, the others follow,
until you reach the end?

Certainly.

And all these others we shall affirm to be parts of the whole and of the
one, which, as soon as the end is reached, has become whole and one?

Yes; that is what we shall say.

But the end comes last, and the one is of such a nature as to come into
being with the last; and, since the one cannot come into being except in
accordance with its own nature, its nature will require that it should
come into being after the others, simultaneously with the end.

Clearly.

Then the one is younger than the others and the others older than the
one.

That also is clear in my judgment.

Well, and must not a beginning or any other part of the one or of
anything, if it be a part and not parts, being a part, be also of
necessity one?

Certainly.

And will not the one come into being together with each part--together
with the first part when that comes into being, and together with the
second part and with all the rest, and will not be wanting to any part,
which is added to any other part until it has reached the last and
become one whole; it will be wanting neither to the middle, nor to
the first, nor to the last, nor to any of them, while the process of
becoming is going on?

True.

Then the one is of the same age with all the others, so that if the one
itself does not contradict its own nature, it will be neither prior
nor posterior to the others, but simultaneous; and according to this
argument the one will be neither older nor younger than the others, nor
the others than the one, but according to the previous argument the one
will be older and younger than the others and the others than the one.

Certainly.

After this manner then the one is and has become. But as to its becoming
older and younger than the others, and the others than the one, and
neither older nor younger, what shall we say? Shall we say as of being
so also of becoming, or otherwise?

I cannot answer.

But I can venture to say, that even if one thing were older or younger
than another, it could not become older or younger in a greater degree
than it was at first; for equals added to unequals, whether to periods
of time or to anything else, leave the difference between them the same
as at first.

Of course.

Then that which is, cannot become older or younger than that which
is, since the difference of age is always the same; the one is and has
become older and the other younger; but they are no longer becoming so.

True.

And the one which is does not therefore become either older or younger
than the others which are.

No.

But consider whether they may not become older and younger in another
way.

In what way?

Just as the one was proven to be older than the others and the others
than the one.

And what of that?

If the one is older than the others, has come into being a longer time
than the others.

Yes.

But consider again; if we add equal time to a greater and a less time,
will the greater differ from the less time by an equal or by a smaller
portion than before?

By a smaller portion.

Then the difference between the age of the one and the age of the others
will not be afterwards so great as at first, but if an equal time be
added to both of them they will differ less and less in age?

Yes.

And that which differs in age from some other less than formerly, from
being older will become younger in relation to that other than which it
was older?

Yes, younger.

And if the one becomes younger the others aforesaid will become older
than they were before, in relation to the one.

Certainly.

Then that which had become younger becomes older relatively to that
which previously had become and was older; it never really is older, but
is always becoming, for the one is always growing on the side of youth
and the other on the side of age. And in like manner the older is always
in process of becoming younger than the younger; for as they are always
going in opposite directions they become in ways the opposite to one
another, the younger older than the older, and the older younger than
the younger. They cannot, however, have become; for if they had already
become they would be and not merely become. But that is impossible; for
they are always becoming both older and younger than one another: the
one becomes younger than the others because it was seen to be older and
prior, and the others become older than the one because they came into
being later; and in the same way the others are in the same relation to
the one, because they were seen to be older, and prior to the one.

That is clear.

Inasmuch then, one thing does not become older or younger than another,
in that they always differ from each other by an equal number, the one
cannot become older or younger than the others, nor the others than the
one; but inasmuch as that which came into being earlier and that which
came into being later must continually differ from each other by a
different portion--in this point of view the others must become older
and younger than the one, and the one than the others.

Certainly.

For all these reasons, then, the one is and becomes older and younger
than itself and the others, and neither is nor becomes older or younger
than itself or the others.

Certainly.

But since the one partakes of time, and partakes of becoming older and
younger, must it not also partake of the past, the present, and the
future?

Of course it must.

Then the one was and is and will be, and was becoming and is becoming
and will become?

Certainly.

And there is and was and will be something which is in relation to it
and belongs to it?

True.

And since we have at this moment opinion and knowledge and perception of
the one, there is opinion and knowledge and perception of it?

Quite right.

Then there is name and expression for it, and it is named and expressed,
and everything of this kind which appertains to other things appertains
to the one.

Certainly, that is true.

Yet once more and for the third time, let us consider: If the one is
both one and many, as we have described, and is neither one nor many,
and participates in time, must it not, in as far as it is one, at times
partake of being, and in as far as it is not one, at times not partake
of being?

Certainly.

But can it partake of being when not partaking of being, or not partake
of being when partaking of being?

Impossible.

Then the one partakes and does not partake of being at different times,
for that is the only way in which it can partake and not partake of the
same.

True.

And is there not also a time at which it assumes being and relinquishes
being--for how can it have and not have the same thing unless it
receives and also gives it up at some time?

Impossible.

And the assuming of being is what you would call becoming?

I should.

And the relinquishing of being you would call destruction?

I should.

The one then, as would appear, becomes and is destroyed by taking and
giving up being.

Certainly.

And being one and many and in process of becoming and being destroyed,
when it becomes one it ceases to be many, and when many, it ceases to be
one?

Certainly.

And as it becomes one and many, must it not inevitably experience
separation and aggregation?

Inevitably.

And whenever it becomes like and unlike it must be assimilated and
dissimilated?

Yes.

And when it becomes greater or less or equal it must grow or diminish or
be equalized?

True.

And when being in motion it rests, and when being at rest it changes to
motion, it can surely be in no time at all?

How can it?

But that a thing which is previously at rest should be afterwards
in motion, or previously in motion and afterwards at rest, without
experiencing change, is impossible.

Impossible.

And surely there cannot be a time in which a thing can be at once
neither in motion nor at rest?

There cannot.

But neither can it change without changing.

True.

When then does it change; for it cannot change either when at rest, or
when in motion, or when in time?

It cannot.

And does this strange thing in which it is at the time of changing
really exist?

What thing?

The moment. For the moment seems to imply a something out of which
change takes place into either of two states; for the change is not from
the state of rest as such, nor from the state of motion as such; but
there is this curious nature which we call the moment lying between rest
and motion, not being in any time; and into this and out of this what is
in motion changes into rest, and what is at rest into motion.

So it appears.

And the one then, since it is at rest and also in motion, will change
to either, for only in this way can it be in both. And in changing it
changes in a moment, and when it is changing it will be in no time, and
will not then be either in motion or at rest.

It will not.

And it will be in the same case in relation to the other changes, when
it passes from being into cessation of being, or from not-being into
becoming--then it passes between certain states of motion and rest, and
neither is nor is not, nor becomes nor is destroyed.

Very true.

And on the same principle, in the passage from one to many and from
many to one, the one is neither one nor many, neither separated nor
aggregated; and in the passage from like to unlike, and from unlike to
like, it is neither like nor unlike, neither in a state of assimilation
nor of dissimilation; and in the passage from small to great and equal
and back again, it will be neither small nor great, nor equal, nor in a
state of increase, or diminution, or equalization.

True.

All these, then, are the affections of the one, if the one has being.

Of course.

1.aa. But if one is, what will happen to the others--is not that also to
be considered?

Yes.

Let us show then, if one is, what will be the affections of the others
than the one.

Let us do so.

Inasmuch as there are things other than the one, the others are not the
one; for if they were they could not be other than the one.

Very true.

Nor are the others altogether without the one, but in a certain way they
participate in the one.

In what way?

Because the others are other than the one inasmuch as they have parts;
for if they had no parts they would be simply one.

Right.

And parts, as we affirm, have relation to a whole?

So we say.

And a whole must necessarily be one made up of many; and the parts will
be parts of the one, for each of the parts is not a part of many, but of
a whole.

How do you mean?

If anything were a part of many, being itself one of them, it will
surely be a part of itself, which is impossible, and it will be a part
of each one of the other parts, if of all; for if not a part of some
one, it will be a part of all the others but this one, and thus will not
be a part of each one; and if not a part of each, one it will not be a
part of any one of the many; and not being a part of any one, it cannot
be a part or anything else of all those things of none of which it is
anything.

Clearly not.

Then the part is not a part of the many, nor of all, but is of a certain
single form, which we call a whole, being one perfect unity framed out
of all--of this the part will be a part.

Certainly.

If, then, the others have parts, they will participate in the whole and
in the one.

True.

Then the others than the one must be one perfect whole, having parts.

Certainly.

And the same argument holds of each part, for the part must participate
in the one; for if each of the parts is a part, this means, I suppose,
that it is one separate from the rest and self-related; otherwise it is
not each.

True.

But when we speak of the part participating in the one, it must clearly
be other than one; for if not, it would not merely have participated,
but would have been one; whereas only the itself can be one.

Very true.

Both the whole and the part must participate in the one; for the whole
will be one whole, of which the parts will be parts; and each part will
be one part of the whole which is the whole of the part.

True.

And will not the things which participate in the one, be other than it?

Of course.

And the things which are other than the one will be many; for if the
things which are other than the one were neither one nor more than one,
they would be nothing.

True.

But, seeing that the things which participate in the one as a part, and
in the one as a whole, are more than one, must not those very things
which participate in the one be infinite in number?

How so?

Let us look at the matter thus:--Is it not a fact that in partaking of
the one they are not one, and do not partake of the one at the very time
when they are partaking of it?

Clearly.

They do so then as multitudes in which the one is not present?

Very true.

And if we were to abstract from them in idea the very smallest fraction,
must not that least fraction, if it does not partake of the one, be a
multitude and not one?

It must.

And if we continue to look at the other side of their nature, regarded
simply, and in itself, will not they, as far as we see them, be
unlimited in number?

Certainly.

And yet, when each several part becomes a part, then the parts have
a limit in relation to the whole and to each other, and the whole in
relation to the parts.

Just so.

The result to the others than the one is that the union of themselves
and the one appears to create a new element in them which gives to them
limitation in relation to one another; whereas in their own nature they
have no limit.

That is clear.

Then the others than the one, both as whole and parts, are infinite, and
also partake of limit.

Certainly.

Then they are both like and unlike one another and themselves.

How is that?

Inasmuch as they are unlimited in their own nature, they are all
affected in the same way.

True.

And inasmuch as they all partake of limit, they are all affected in the
same way.

Of course.

But inasmuch as their state is both limited and unlimited, they are
affected in opposite ways.

Yes.

And opposites are the most unlike of things.

Certainly.

Considered, then, in regard to either one of their affections, they will
be like themselves and one another; considered in reference to both of
them together, most opposed and most unlike.

That appears to be true.

Then the others are both like and unlike themselves and one another?

True.

And they are the same and also different from one another, and in motion
and at rest, and experience every sort of opposite affection, as may be
proved without difficulty of them, since they have been shown to have
experienced the affections aforesaid?

True.

1.bb. Suppose, now, that we leave the further discussion of these
matters as evident, and consider again upon the hypothesis that the
one is, whether opposite of all this is or is not equally true of the
others.

By all means.

Then let us begin again, and ask, If one is, what must be the affections
of the others?

Let us ask that question.

Must not the one be distinct from the others, and the others from the
one?

Why so?

Why, because there is nothing else beside them which is distinct from
both of them; for the expression 'one and the others' includes all
things.

Yes, all things.

Then we cannot suppose that there is anything different from them in
which both the one and the others might exist?

There is nothing.

Then the one and the others are never in the same?

True.

Then they are separated from each other?

Yes.

And we surely cannot say that what is truly one has parts?

Impossible.

Then the one will not be in the others as a whole, nor as part, if it be
separated from the others, and has no parts?

Impossible.

Then there is no way in which the others can partake of the one, if they
do not partake either in whole or in part?

It would seem not.

Then there is no way in which the others are one, or have in themselves
any unity?

There is not.

Nor are the others many; for if they were many, each part of them would
be a part of the whole; but now the others, not partaking in any way of
the one, are neither one nor many, nor whole, nor part.

True.

Then the others neither are nor contain two or three, if entirely
deprived of the one?

True.

Then the others are neither like nor unlike the one, nor is likeness
and unlikeness in them; for if they were like and unlike, or had in them
likeness and unlikeness, they would have two natures in them opposite to
one another.

That is clear.

But for that which partakes of nothing to partake of two things was held
by us to be impossible?

Impossible.

Then the others are neither like nor unlike nor both, for if they were
like or unlike they would partake of one of those two natures, which
would be one thing, and if they were both they would partake of
opposites which would be two things, and this has been shown to be
impossible.

True.

Therefore they are neither the same, nor other, nor in motion, nor at
rest, nor in a state of becoming, nor of being destroyed, nor greater,
nor less, nor equal, nor have they experienced anything else of the
sort; for, if they are capable of experiencing any such affection, they
will participate in one and two and three, and odd and even, and in
these, as has been proved, they do not participate, seeing that they are
altogether and in every way devoid of the one.

Very true.

Therefore if one is, the one is all things, and also nothing, both in
relation to itself and to other things.

Certainly.

2.a. Well, and ought we not to consider next what will be the
consequence if the one is not?

Yes; we ought.

What is the meaning of the hypothesis--If the one is not; is there any
difference between this and the hypothesis--If the not one is not?

There is a difference, certainly.

Is there a difference only, or rather are not the two expressions--if
the one is not, and if the not one is not, entirely opposed?

They are entirely opposed.

And suppose a person to say:--If greatness is not, if smallness is not,
or anything of that sort, does he not mean, whenever he uses such an
expression, that 'what is not' is other than other things?

To be sure.

And so when he says 'If one is not' he clearly means, that what 'is not'
is other than all others; we know what he means--do we not?

Yes, we do.

When he says 'one,' he says something which is known; and secondly
something which is other than all other things; it makes no difference
whether he predicate of one being or not-being, for that which is said
'not to be' is known to be something all the same, and is distinguished
from other things.

Certainly.

Then I will begin again, and ask: If one is not, what are the
consequences? In the first place, as would appear, there is a knowledge
of it, or the very meaning of the words, 'if one is not,' would not be
known.

True.

Secondly, the others differ from it, or it could not be described as
different from the others?

Certainly.

Difference, then, belongs to it as well as knowledge; for in speaking of
the one as different from the others, we do not speak of a difference in
the others, but in the one.

Clearly so.

Moreover, the one that is not is something and partakes of relation to
'that,' and 'this,' and 'these,' and the like, and is an attribute of
'this'; for the one, or the others than the one, could not have been
spoken of, nor could any attribute or relative of the one that is not
have been or been spoken of, nor could it have been said to be anything,
if it did not partake of 'some,' or of the other relations just now
mentioned.

True.

Being, then, cannot be ascribed to the one, since it is not; but the
one that is not may or rather must participate in many things, if it and
nothing else is not; if, however, neither the one nor the one that
is not is supposed not to be, and we are speaking of something of a
different nature, we can predicate nothing of it. But supposing that the
one that is not and nothing else is not, then it must participate in the
predicate 'that,' and in many others.

Certainly.

And it will have unlikeness in relation to the others, for the others
being different from the one will be of a different kind.

Certainly.

And are not things of a different kind also other in kind?

Of course.

And are not things other in kind unlike?

They are unlike.

And if they are unlike the one, that which they are unlike will clearly
be unlike them?

Clearly so.

Then the one will have unlikeness in respect of which the others are
unlike it?

That would seem to be true.

And if unlikeness to other things is attributed to it, it must have
likeness to itself.

How so?

If the one have unlikeness to one, something else must be meant; nor
will the hypothesis relate to one; but it will relate to something other
than one?

Quite so.

But that cannot be.

No.

Then the one must have likeness to itself?

It must.

Again, it is not equal to the others; for if it were equal, then it
would at once be and be like them in virtue of the equality; but if one
has no being, then it can neither be nor be like?

It cannot.

But since it is not equal to the others, neither can the others be equal
to it?

Certainly not.

And things that are not equal are unequal?

True.

And they are unequal to an unequal?

Of course.

Then the one partakes of inequality, and in respect of this the others
are unequal to it?

Very true.

And inequality implies greatness and smallness?

Yes.

Then the one, if of such a nature, has greatness and smallness?

That appears to be true.

And greatness and smallness always stand apart?

True.

Then there is always something between them?

There is.

And can you think of anything else which is between them other than
equality?

No, it is equality which lies between them.

Then that which has greatness and smallness also has equality, which
lies between them?

That is clear.

Then the one, which is not, partakes, as would appear, of greatness and
smallness and equality?

Clearly.

Further, it must surely in a sort partake of being?

How so?

It must be so, for if not, then we should not speak the truth in saying
that the one is not. But if we speak the truth, clearly we must say what
is. Am I not right?

Yes.

And since we affirm that we speak truly, we must also affirm that we say
what is?

Certainly.

Then, as would appear, the one, when it is not, is; for if it were
not to be when it is not, but (Or, 'to remit something of existence in
relation to not-being.') were to relinquish something of being, so as to
become not-being, it would at once be.

Quite true.

Then the one which is not, if it is to maintain itself, must have the
being of not-being as the bond of not-being, just as being must have as
a bond the not-being of not-being in order to perfect its own being;
for the truest assertion of the being of being and of the not-being of
not-being is when being partakes of the being of being, and not of the
being of not-being--that is, the perfection of being; and when not-being
does not partake of the not-being of not-being but of the being of
not-being--that is the perfection of not-being.

Most true.

Since then what is partakes of not-being, and what is not of being, must
not the one also partake of being in order not to be?

Certainly.

Then the one, if it is not, clearly has being?

Clearly.

And has not-being also, if it is not?

Of course.

But can anything which is in a certain state not be in that state
without changing?

Impossible.

Then everything which is and is not in a certain state, implies change?

Certainly.

And change is motion--we may say that?

Yes, motion.

And the one has been proved both to be and not to be?

Yes.

And therefore is and is not in the same state?

Yes.

Thus the one that is not has been shown to have motion also, because it
changes from being to not-being?

That appears to be true.

But surely if it is nowhere among what is, as is the fact, since it is
not, it cannot change from one place to another?

Impossible.

Then it cannot move by changing place?

No.

Nor can it turn on the same spot, for it nowhere touches the same, for
the same is, and that which is not cannot be reckoned among things that
are?

It cannot.

Then the one, if it is not, cannot turn in that in which it is not?

No.

Neither can the one, whether it is or is not, be altered into other
than itself, for if it altered and became different from itself, then we
could not be still speaking of the one, but of something else?

True.

But if the one neither suffers alteration, nor turns round in the same
place, nor changes place, can it still be capable of motion?

Impossible.

Now that which is unmoved must surely be at rest, and that which is at
rest must stand still?

Certainly.

Then the one that is not, stands still, and is also in motion?

That seems to be true.

But if it be in motion it must necessarily undergo alteration, for
anything which is moved, in so far as it is moved, is no longer in the
same state, but in another?

Yes.

Then the one, being moved, is altered?

Yes.

And, further, if not moved in any way, it will not be altered in any
way?

No.

Then, in so far as the one that is not is moved, it is altered, but in
so far as it is not moved, it is not altered?

Right.

Then the one that is not is altered and is not altered?

That is clear.

And must not that which is altered become other than it previously
was, and lose its former state and be destroyed; but that which is not
altered can neither come into being nor be destroyed?

Very true.

And the one that is not, being altered, becomes and is destroyed; and
not being altered, neither becomes nor is destroyed; and so the one that
is not becomes and is destroyed, and neither becomes nor is destroyed?

True.

2.b. And now, let us go back once more to the beginning, and see whether
these or some other consequences will follow.

Let us do as you say.

If one is not, we ask what will happen in respect of one? That is the
question.

Yes.

Do not the words 'is not' signify absence of being in that to which we
apply them?

Just so.

And when we say that a thing is not, do we mean that it is not in one
way but is in another? or do we mean, absolutely, that what is not has
in no sort or way or kind participation of being?

Quite absolutely.

Then, that which is not cannot be, or in any way participate in being?

It cannot.

And did we not mean by becoming, and being destroyed, the assumption of
being and the loss of being?

Nothing else.

And can that which has no participation in being, either assume or lose
being?

Impossible.

The one then, since it in no way is, cannot have or lose or assume being
in any way?

True.

Then the one that is not, since it in no way partakes of being, neither
perishes nor becomes?

No.

Then it is not altered at all; for if it were it would become and be
destroyed?

True.

But if it be not altered it cannot be moved?

Certainly not.

Nor can we say that it stands, if it is nowhere; for that which stands
must always be in one and the same spot?

Of course.

Then we must say that the one which is not never stands still and never
moves?

Neither.

Nor is there any existing thing which can be attributed to it; for if
there had been, it would partake of being?

That is clear.

And therefore neither smallness, nor greatness, nor equality, can be
attributed to it?

No.

Nor yet likeness nor difference, either in relation to itself or to
others?

Clearly not.

Well, and if nothing should be attributed to it, can other things be
attributed to it?

Certainly not.

And therefore other things can neither be like or unlike, the same, or
different in relation to it?

They cannot.

Nor can what is not, be anything, or be this thing, or be related to or
the attribute of this or that or other, or be past, present, or future.
Nor can knowledge, or opinion, or perception, or expression, or name, or
any other thing that is, have any concern with it?

No.

Then the one that is not has no condition of any kind?

Such appears to be the conclusion.

2.aa. Yet once more; if one is not, what becomes of the others? Let us
determine that.

Yes; let us determine that.

The others must surely be; for if they, like the one, were not, we could
not be now speaking of them.

True.

But to speak of the others implies difference--the terms 'other' and
'different' are synonymous?

True.

Other means other than other, and different, different from the
different?

Yes.

Then, if there are to be others, there is something than which they will
be other?

Certainly.

And what can that be?--for if the one is not, they will not be other
than the one.

They will not.

Then they will be other than each other; for the only remaining
alternative is that they are other than nothing.

True.

And they are each other than one another, as being plural and not
singular; for if one is not, they cannot be singular, but every particle
of them is infinite in number; and even if a person takes that which
appears to be the smallest fraction, this, which seemed one, in a moment
evanesces into many, as in a dream, and from being the smallest becomes
very great, in comparison with the fractions into which it is split up?

Very true.

And in such particles the others will be other than one another, if
others are, and the one is not?

Exactly.

And will there not be many particles, each appearing to be one, but not
being one, if one is not?

True.

And it would seem that number can be predicated of them if each of them
appears to be one, though it is really many?

It can.

And there will seem to be odd and even among them, which will also have
no reality, if one is not?

Yes.

And there will appear to be a least among them; and even this will seem
large and manifold in comparison with the many small fractions which are
contained in it?

Certainly.

And each particle will be imagined to be equal to the many and little;
for it could not have appeared to pass from the greater to the less
without having appeared to arrive at the middle; and thus would arise
the appearance of equality.

Yes.

And having neither beginning, middle, nor end, each separate particle
yet appears to have a limit in relation to itself and other.

How so?

Because, when a person conceives of any one of these as such, prior
to the beginning another beginning appears, and there is another end,
remaining after the end, and in the middle truer middles within but
smaller, because no unity can be conceived of any of them, since the one
is not.

Very true.

And so all being, whatever we think of, must be broken up into
fractions, for a particle will have to be conceived of without unity?

Certainly.

And such being when seen indistinctly and at a distance, appears to
be one; but when seen near and with keen intellect, every single thing
appears to be infinite, since it is deprived of the one, which is not?

Nothing more certain.

Then each of the others must appear to be infinite and finite, and one
and many, if others than the one exist and not the one.

They must.

Then will they not appear to be like and unlike?

In what way?

Just as in a picture things appear to be all one to a person standing at
a distance, and to be in the same state and alike?

True.

But when you approach them, they appear to be many and different; and
because of the appearance of the difference, different in kind from, and
unlike, themselves?

True.

And so must the particles appear to be like and unlike themselves and
each other.

Certainly.

And must they not be the same and yet different from one another, and in
contact with themselves, although they are separated, and having
every sort of motion, and every sort of rest, and becoming and being
destroyed, and in neither state, and the like, all which things may be
easily enumerated, if the one is not and the many are?

Most true.

2.bb. Once more, let us go back to the beginning, and ask if the one is
not, and the others of the one are, what will follow.

Let us ask that question.

In the first place, the others will not be one?

Impossible.

Nor will they be many; for if they were many one would be contained
in them. But if no one of them is one, all of them are nought, and
therefore they will not be many.

True.

If there be no one in the others, the others are neither many nor one.

They are not.

Nor do they appear either as one or many.

Why not?

Because the others have no sort or manner or way of communion with any
sort of not-being, nor can anything which is not, be connected with any
of the others; for that which is not has no parts.

True.

Nor is there an opinion or any appearance of not-being in connexion with
the others, nor is not-being ever in any way attributed to the others.

No.

Then if one is not, there is no conception of any of the others either
as one or many; for you cannot conceive the many without the one.

You cannot.

Then if one is not, the others neither are, nor can be conceived to be
either one or many?

It would seem not.

Nor as like or unlike?

No.

Nor as the same or different, nor in contact or separation, nor in any
of those states which we enumerated as appearing to be;--the others
neither are nor appear to be any of these, if one is not?

True.

Then may we not sum up the argument in a word and say truly: If one is
not, then nothing is?

Certainly.

Let thus much be said; and further let us affirm what seems to be the
truth, that, whether one is or is not, one and the others in relation to
themselves and one another, all of them, in every way, are and are not,
and appear to be and appear not to be.

Most true.




% chapter parmenides (end)
% \chapter{Categories} % (fold)
\label{cha:categories}





The Categories


By

Aristotle


Translated by E. M. Edghill



Section 1

Part 1

Things are said to be named 'equivocally' when, though they have a
common name, the definition corresponding with the name differs for
each. Thus, a real man and a figure in a picture can both lay claim to
the name 'animal'; yet these are equivocally so named, for, though they
have a common name, the definition corresponding with the name differs
for each. For should any one define in what sense each is an animal,
his definition in the one case will be appropriate to that case only.

On the other hand, things are said to be named 'univocally' which have
both the name and the definition answering to the name in common. A man
and an ox are both 'animal', and these are univocally so named,
inasmuch as not only the name, but also the definition, is the same in
both cases: for if a man should state in what sense each is an animal,
the statement in the one case would be identical with that in the other.

Things are said to be named 'derivatively', which derive their name
from some other name, but differ from it in termination. Thus the
grammarian derives his name from the word 'grammar', and the courageous
man from the word 'courage'.



Part 2

Forms of speech are either simple or composite. Examples of the latter
are such expressions as 'the man runs', 'the man wins'; of the former
'man', 'ox', 'runs', 'wins'.

Of things themselves some are predicable of a subject, and are never
present in a subject. Thus 'man' is predicable of the individual man,
and is never present in a subject.

By being 'present in a subject' I do not mean present as parts are
present in a whole, but being incapable of existence apart from the
said subject.

Some things, again, are present in a subject, but are never predicable
of a subject. For instance, a certain point of grammatical knowledge is
present in the mind, but is not predicable of any subject; or again, a
certain whiteness may be present in the body (for colour requires a
material basis), yet it is never predicable of anything.

Other things, again, are both predicable of a subject and present in a
subject. Thus while knowledge is present in the human mind, it is
predicable of grammar.

There is, lastly, a class of things which are neither present in a
subject nor predicable of a subject, such as the individual man or the
individual horse. But, to speak more generally, that which is
individual and has the character of a unit is never predicable of a
subject. Yet in some cases there is nothing to prevent such being
present in a subject. Thus a certain point of grammatical knowledge is
present in a subject.



Part 3

When one thing is predicated of another, all that which is predicable
of the predicate will be predicable also of the subject. Thus, 'man' is
predicated of the individual man; but 'animal' is predicated of 'man';
it will, therefore, be predicable of the individual man also: for the
individual man is both 'man' and 'animal'.

If genera are different and co-ordinate, their differentiae are
themselves different in kind. Take as an instance the genus 'animal'
and the genus 'knowledge'. 'With feet', 'two-footed', 'winged',
'aquatic', are differentiae of 'animal'; the species of knowledge are
not distinguished by the same differentiae. One species of knowledge
does not differ from another in being 'two-footed'.

But where one genus is subordinate to another, there is nothing to
prevent their having the same differentiae: for the greater class is
predicated of the lesser, so that all the differentiae of the predicate
will be differentiae also of the subject.



Part 4

Expressions which are in no way composite signify substance, quantity,
quality, relation, place, time, position, state, action, or affection.
To sketch my meaning roughly, examples of substance are 'man' or 'the
horse', of quantity, such terms as 'two cubits long' or 'three cubits
long', of quality, such attributes as 'white', 'grammatical'. 'Double',
'half', 'greater', fall under the category of relation; 'in a the
market place', 'in the Lyceum', under that of place; 'yesterday', 'last
year', under that of time. 'Lying', 'sitting', are terms indicating
position, 'shod', 'armed', state; 'to lance', 'to cauterize', action;
'to be lanced', 'to be cauterized', affection.

No one of these terms, in and by itself, involves an affirmation; it is
by the combination of such terms that positive or negative statements
arise. For every assertion must, as is admitted, be either true or
false, whereas expressions which are not in any way composite such as
'man', 'white', 'runs', 'wins', cannot be either true or false.



Part 5

Substance, in the truest and primary and most definite sense of the
word, is that which is neither predicable of a subject nor present in a
subject; for instance, the individual man or horse. But in a secondary
sense those things are called substances within which, as species, the
primary substances are included; also those which, as genera, include
the species. For instance, the individual man is included in the
species 'man', and the genus to which the species belongs is 'animal';
these, therefore-that is to say, the species 'man' and the genus
'animal,-are termed secondary substances.

It is plain from what has been said that both the name and the
definition of the predicate must be predicable of the subject. For
instance, 'man' is predicted of the individual man. Now in this case
the name of the species man' is applied to the individual, for we use
the term 'man' in describing the individual; and the definition of
'man' will also be predicated of the individual man, for the individual
man is both man and animal. Thus, both the name and the definition of
the species are predicable of the individual.

With regard, on the other hand, to those things which are present in a
subject, it is generally the case that neither their name nor their
definition is predicable of that in which they are present. Though,
however, the definition is never predicable, there is nothing in
certain cases to prevent the name being used. For instance, 'white'
being present in a body is predicated of that in which it is present,
for a body is called white: the definition, however, of the colour
white' is never predicable of the body.

Everything except primary substances is either predicable of a primary
substance or present in a primary substance. This becomes evident by
reference to particular instances which occur. 'Animal' is predicated
of the species 'man', therefore of the individual man, for if there
were no individual man of whom it could be predicated, it could not be
predicated of the species 'man' at all. Again, colour is present in
body, therefore in individual bodies, for if there were no individual
body in which it was present, it could not be present in body at all.
Thus everything except primary substances is either predicated of
primary substances, or is present in them, and if these last did not
exist, it would be impossible for anything else to exist.

Of secondary substances, the species is more truly substance than the
genus, being more nearly related to primary substance. For if any one
should render an account of what a primary substance is, he would
render a more instructive account, and one more proper to the subject,
by stating the species than by stating the genus. Thus, he would give a
more instructive account of an individual man by stating that he was
man than by stating that he was animal, for the former description is
peculiar to the individual in a greater degree, while the latter is too
general. Again, the man who gives an account of the nature of an
individual tree will give a more instructive account by mentioning the
species 'tree' than by mentioning the genus 'plant'.

Moreover, primary substances are most properly called substances in
virtue of the fact that they are the entities which underlie everything
else, and that everything else is either predicated of them or present
in them. Now the same relation which subsists between primary substance
and everything else subsists also between the species and the genus:
for the species is to the genus as subject is to predicate, since the
genus is predicated of the species, whereas the species cannot be
predicated of the genus. Thus we have a second ground for asserting
that the species is more truly substance than the genus.

Of species themselves, except in the case of such as are genera, no one
is more truly substance than another. We should not give a more
appropriate account of the individual man by stating the species to
which he belonged, than we should of an individual horse by adopting
the same method of definition. In the same way, of primary substances,
no one is more truly substance than another; an individual man is not
more truly substance than an individual ox.

It is, then, with good reason that of all that remains, when we exclude
primary substances, we concede to species and genera alone the name
'secondary substance', for these alone of all the predicates convey a
knowledge of primary substance. For it is by stating the species or the
genus that we appropriately define any individual man; and we shall
make our definition more exact by stating the former than by stating
the latter. All other things that we state, such as that he is white,
that he runs, and so on, are irrelevant to the definition. Thus it is
just that these alone, apart from primary substances, should be called
substances.

Further, primary substances are most properly so called, because they
underlie and are the subjects of everything else. Now the same relation
that subsists between primary substance and everything else subsists
also between the species and the genus to which the primary substance
belongs, on the one hand, and every attribute which is not included
within these, on the other. For these are the subjects of all such. If
we call an individual man 'skilled in grammar', the predicate is
applicable also to the species and to the genus to which he belongs.
This law holds good in all cases.

It is a common characteristic of all substance that it is never present
in a subject. For primary substance is neither present in a subject nor
predicated of a subject; while, with regard to secondary substances, it
is clear from the following arguments (apart from others) that they are
not present in a subject. For 'man' is predicated of the individual
man, but is not present in any subject: for manhood is not present in
the individual man. In the same way, 'animal' is also predicated of the
individual man, but is not present in him. Again, when a thing is
present in a subject, though the name may quite well be applied to that
in which it is present, the definition cannot be applied. Yet of
secondary substances, not only the name, but also the definition,
applies to the subject: we should use both the definition of the
species and that of the genus with reference to the individual man.
Thus substance cannot be present in a subject.

Yet this is not peculiar to substance, for it is also the case that
differentiae cannot be present in subjects. The characteristics
'terrestrial' and 'two-footed' are predicated of the species 'man', but
not present in it. For they are not in man. Moreover, the definition of
the differentia may be predicated of that of which the differentia
itself is predicated. For instance, if the characteristic 'terrestrial'
is predicated of the species 'man', the definition also of that
characteristic may be used to form the predicate of the species 'man':
for 'man' is terrestrial.

The fact that the parts of substances appear to be present in the
whole, as in a subject, should not make us apprehensive lest we should
have to admit that such parts are not substances: for in explaining the
phrase 'being present in a subject', we stated' that we meant
'otherwise than as parts in a whole'.

It is the mark of substances and of differentiae that, in all
propositions of which they form the predicate, they are predicated
univocally. For all such propositions have for their subject either the
individual or the species. It is true that, inasmuch as primary
substance is not predicable of anything, it can never form the
predicate of any proposition. But of secondary substances, the species
is predicated of the individual, the genus both of the species and of
the individual. Similarly the differentiae are predicated of the
species and of the individuals. Moreover, the definition of the species
and that of the genus are applicable to the primary substance, and that
of the genus to the species. For all that is predicated of the
predicate will be predicated also of the subject. Similarly, the
definition of the differentiae will be applicable to the species and to
the individuals. But it was stated above that the word 'univocal' was
applied to those things which had both name and definition in common.
It is, therefore, established that in every proposition, of which
either substance or a differentia forms the predicate, these are
predicated univocally.

All substance appears to signify that which is individual. In the case
of primary substance this is indisputably true, for the thing is a
unit. In the case of secondary substances, when we speak, for instance,
of 'man' or 'animal', our form of speech gives the impression that we
are here also indicating that which is individual, but the impression
is not strictly true; for a secondary substance is not an individual,
but a class with a certain qualification; for it is not one and single
as a primary substance is; the words 'man', 'animal', are predicable of
more than one subject.

Yet species and genus do not merely indicate quality, like the term
'white'; 'white' indicates quality and nothing further, but species and
genus determine the quality with reference to a substance: they signify
substance qualitatively differentiated. The determinate qualification
covers a larger field in the case of the genus that in that of the
species: he who uses the word 'animal' is herein using a word of wider
extension than he who uses the word 'man'.

Another mark of substance is that it has no contrary. What could be the
contrary of any primary substance, such as the individual man or
animal? It has none. Nor can the species or the genus have a contrary.
Yet this characteristic is not peculiar to substance, but is true of
many other things, such as quantity. There is nothing that forms the
contrary of 'two cubits long' or of 'three cubits long', or of 'ten',
or of any such term. A man may contend that 'much' is the contrary of
'little', or 'great' of 'small', but of definite quantitative terms no
contrary exists.

Substance, again, does not appear to admit of variation of degree. I do
not mean by this that one substance cannot be more or less truly
substance than another, for it has already been stated' that this is
the case; but that no single substance admits of varying degrees within
itself. For instance, one particular substance, 'man', cannot be more
or less man either than himself at some other time or than some other
man. One man cannot be more man than another, as that which is white
may be more or less white than some other white object, or as that
which is beautiful may be more or less beautiful than some other
beautiful object. The same quality, moreover, is said to subsist in a
thing in varying degrees at different times. A body, being white, is
said to be whiter at one time than it was before, or, being warm, is
said to be warmer or less warm than at some other time. But substance
is not said to be more or less that which it is: a man is not more
truly a man at one time than he was before, nor is anything, if it is
substance, more or less what it is. Substance, then, does not admit of
variation of degree.

The most distinctive mark of substance appears to be that, while
remaining numerically one and the same, it is capable of admitting
contrary qualities. From among things other than substance, we should
find ourselves unable to bring forward any which possessed this mark.
Thus, one and the same colour cannot be white and black. Nor can the
same one action be good and bad: this law holds good with everything
that is not substance. But one and the selfsame substance, while
retaining its identity, is yet capable of admitting contrary qualities.
The same individual person is at one time white, at another black, at
one time warm, at another cold, at one time good, at another bad. This
capacity is found nowhere else, though it might be maintained that a
statement or opinion was an exception to the rule. The same statement,
it is agreed, can be both true and false. For if the statement 'he is
sitting' is true, yet, when the person in question has risen, the same
statement will be false. The same applies to opinions. For if any one
thinks truly that a person is sitting, yet, when that person has risen,
this same opinion, if still held, will be false. Yet although this
exception may be allowed, there is, nevertheless, a difference in the
manner in which the thing takes place. It is by themselves changing
that substances admit contrary qualities. It is thus that that which
was hot becomes cold, for it has entered into a different state.
Similarly that which was white becomes black, and that which was bad
good, by a process of change; and in the same way in all other cases it
is by changing that substances are capable of admitting contrary
qualities. But statements and opinions themselves remain unaltered in
all respects: it is by the alteration in the facts of the case that the
contrary quality comes to be theirs. The statement 'he is sitting'
remains unaltered, but it is at one time true, at another false,
according to circumstances. What has been said of statements applies
also to opinions. Thus, in respect of the manner in which the thing
takes place, it is the peculiar mark of substance that it should be
capable of admitting contrary qualities; for it is by itself changing
that it does so.

If, then, a man should make this exception and contend that statements
and opinions are capable of admitting contrary qualities, his
contention is unsound. For statements and opinions are said to have
this capacity, not because they themselves undergo modification, but
because this modification occurs in the case of something else. The
truth or falsity of a statement depends on facts, and not on any power
on the part of the statement itself of admitting contrary qualities. In
short, there is nothing which can alter the nature of statements and
opinions. As, then, no change takes place in themselves, these cannot
be said to be capable of admitting contrary qualities.

But it is by reason of the modification which takes place within the
substance itself that a substance is said to be capable of admitting
contrary qualities; for a substance admits within itself either disease
or health, whiteness or blackness. It is in this sense that it is said
to be capable of admitting contrary qualities.

To sum up, it is a distinctive mark of substance, that, while remaining
numerically one and the same, it is capable of admitting contrary
qualities, the modification taking place through a change in the
substance itself.

Let these remarks suffice on the subject of substance.



Part 6

Quantity is either discrete or continuous. Moreover, some quantities
are such that each part of the whole has a relative position to the
other parts: others have within them no such relation of part to part.

Instances of discrete quantities are number and speech; of continuous,
lines, surfaces, solids, and, besides these, time and place.

In the case of the parts of a number, there is no common boundary at
which they join. For example: two fives make ten, but the two fives
have no common boundary, but are separate; the parts three and seven
also do not join at any boundary. Nor, to generalize, would it ever be
possible in the case of number that there should be a common boundary
among the parts; they are always separate. Number, therefore, is a
discrete quantity.

The same is true of speech. That speech is a quantity is evident: for
it is measured in long and short syllables. I mean here that speech
which is vocal. Moreover, it is a discrete quantity for its parts have
no common boundary. There is no common boundary at which the syllables
join, but each is separate and distinct from the rest.

A line, on the other hand, is a continuous quantity, for it is possible
to find a common boundary at which its parts join. In the case of the
line, this common boundary is the point; in the case of the plane, it
is the line: for the parts of the plane have also a common boundary.
Similarly you can find a common boundary in the case of the parts of a
solid, namely either a line or a plane.

Space and time also belong to this class of quantities. Time, past,
present, and future, forms a continuous whole. Space, likewise, is a
continuous quantity; for the parts of a solid occupy a certain space,
and these have a common boundary; it follows that the parts of space
also, which are occupied by the parts of the solid, have the same
common boundary as the parts of the solid. Thus, not only time, but
space also, is a continuous quantity, for its parts have a common
boundary.

Quantities consist either of parts which bear a relative position each
to each, or of parts which do not. The parts of a line bear a relative
position to each other, for each lies somewhere, and it would be
possible to distinguish each, and to state the position of each on the
plane and to explain to what sort of part among the rest each was
contiguous. Similarly the parts of a plane have position, for it could
similarly be stated what was the position of each and what sort of
parts were contiguous. The same is true with regard to the solid and to
space. But it would be impossible to show that the arts of a number had
a relative position each to each, or a particular position, or to state
what parts were contiguous. Nor could this be done in the case of time,
for none of the parts of time has an abiding existence, and that which
does not abide can hardly have position. It would be better to say that
such parts had a relative order, in virtue of one being prior to
another. Similarly with number: in counting, 'one' is prior to 'two',
and 'two' to 'three', and thus the parts of number may be said to
possess a relative order, though it would be impossible to discover any
distinct position for each. This holds good also in the case of speech.
None of its parts has an abiding existence: when once a syllable is
pronounced, it is not possible to retain it, so that, naturally, as the
parts do not abide, they cannot have position. Thus, some quantities
consist of parts which have position, and some of those which have not.

Strictly speaking, only the things which I have mentioned belong to the
category of quantity: everything else that is called quantitative is a
quantity in a secondary sense. It is because we have in mind some one
of these quantities, properly so called, that we apply quantitative
terms to other things. We speak of what is white as large, because the
surface over which the white extends is large; we speak of an action or
a process as lengthy, because the time covered is long; these things
cannot in their own right claim the quantitative epithet. For instance,
should any one explain how long an action was, his statement would be
made in terms of the time taken, to the effect that it lasted a year,
or something of that sort. In the same way, he would explain the size
of a white object in terms of surface, for he would state the area
which it covered. Thus the things already mentioned, and these alone,
are in their intrinsic nature quantities; nothing else can claim the
name in its own right, but, if at all, only in a secondary sense.

Quantities have no contraries. In the case of definite quantities this
is obvious; thus, there is nothing that is the contrary of 'two cubits
long' or of 'three cubits long', or of a surface, or of any such
quantities. A man might, indeed, argue that 'much' was the contrary of
'little', and 'great' of 'small'. But these are not quantitative, but
relative; things are not great or small absolutely, they are so called
rather as the result of an act of comparison. For instance, a mountain
is called small, a grain large, in virtue of the fact that the latter
is greater than others of its kind, the former less. Thus there is a
reference here to an external standard, for if the terms 'great' and
'small' were used absolutely, a mountain would never be called small or
a grain large. Again, we say that there are many people in a village,
and few in Athens, although those in the city are many times as
numerous as those in the village: or we say that a house has many in
it, and a theatre few, though those in the theatre far outnumber those
in the house. The terms 'two cubits long, 'three cubits long,' and so
on indicate quantity, the terms 'great' and 'small' indicate relation,
for they have reference to an external standard. It is, therefore,
plain that these are to be classed as relative.

Again, whether we define them as quantitative or not, they have no
contraries: for how can there be a contrary of an attribute which is
not to be apprehended in or by itself, but only by reference to
something external? Again, if 'great' and 'small' are contraries, it
will come about that the same subject can admit contrary qualities at
one and the same time, and that things will themselves be contrary to
themselves. For it happens at times that the same thing is both small
and great. For the same thing may be small in comparison with one
thing, and great in comparison with another, so that the same thing
comes to be both small and great at one and the same time, and is of
such a nature as to admit contrary qualities at one and the same
moment. Yet it was agreed, when substance was being discussed, that
nothing admits contrary qualities at one and the same moment. For
though substance is capable of admitting contrary qualities, yet no one
is at the same time both sick and healthy, nothing is at the same time
both white and black. Nor is there anything which is qualified in
contrary ways at one and the same time.

Moreover, if these were contraries, they would themselves be contrary
to themselves. For if 'great' is the contrary of 'small', and the same
thing is both great and small at the same time, then 'small' or 'great'
is the contrary of itself. But this is impossible. The term 'great',
therefore, is not the contrary of the term 'small', nor 'much' of
'little'. And even though a man should call these terms not relative
but quantitative, they would not have contraries.

It is in the case of space that quantity most plausibly appears to
admit of a contrary. For men define the term 'above' as the contrary of
'below', when it is the region at the centre they mean by 'below'; and
this is so, because nothing is farther from the extremities of the
universe than the region at the centre. Indeed, it seems that in
defining contraries of every kind men have recourse to a spatial
metaphor, for they say that those things are contraries which, within
the same class, are separated by the greatest possible distance.

Quantity does not, it appears, admit of variation of degree. One thing
cannot be two cubits long in a greater degree than another. Similarly
with regard to number: what is 'three' is not more truly three than
what is 'five' is five; nor is one set of three more truly three than
another set. Again, one period of time is not said to be more truly
time than another. Nor is there any other kind of quantity, of all that
have been mentioned, with regard to which variation of degree can be
predicated. The category of quantity, therefore, does not admit of
variation of degree.

The most distinctive mark of quantity is that equality and inequality
are predicated of it. Each of the aforesaid quantities is said to be
equal or unequal. For instance, one solid is said to be equal or
unequal to another; number, too, and time can have these terms applied
to them, indeed can all those kinds of quantity that have been
mentioned.

That which is not a quantity can by no means, it would seem, be termed
equal or unequal to anything else. One particular disposition or one
particular quality, such as whiteness, is by no means compared with
another in terms of equality and inequality but rather in terms of
similarity. Thus it is the distinctive mark of quantity that it can be
called equal and unequal.



Section 2


Part 7

Those things are called relative, which, being either said to be of
something else or related to something else, are explained by reference
to that other thing. For instance, the word 'superior' is explained by
reference to something else, for it is superiority over something else
that is meant. Similarly, the expression 'double' has this external
reference, for it is the double of something else that is meant. So it
is with everything else of this kind. There are, moreover, other
relatives, e.g. habit, disposition, perception, knowledge, and
attitude. The significance of all these is explained by a reference to
something else and in no other way. Thus, a habit is a habit of
something, knowledge is knowledge of something, attitude is the
attitude of something. So it is with all other relatives that have been
mentioned. Those terms, then, are called relative, the nature of which
is explained by reference to something else, the preposition 'of' or
some other preposition being used to indicate the relation. Thus, one
mountain is called great in comparison with son with another; for the
mountain claims this attribute by comparison with something. Again,
that which is called similar must be similar to something else, and all
other such attributes have this external reference. It is to be noted
that lying and standing and sitting are particular attitudes, but
attitude is itself a relative term. To lie, to stand, to be seated, are
not themselves attitudes, but take their name from the aforesaid
attitudes.

It is possible for relatives to have contraries. Thus virtue has a
contrary, vice, these both being relatives; knowledge, too, has a
contrary, ignorance. But this is not the mark of all relatives;
'double' and 'triple' have no contrary, nor indeed has any such term.

It also appears that relatives can admit of variation of degree. For
'like' and 'unlike', 'equal' and 'unequal', have the modifications
'more' and 'less' applied to them, and each of these is relative in
character: for the terms 'like' and 'unequal' bear 'unequal' bear a
reference to something external. Yet, again, it is not every relative
term that admits of variation of degree. No term such as 'double'
admits of this modification. All relatives have correlatives: by the
term 'slave' we mean the slave of a master, by the term 'master', the
master of a slave; by 'double', the double of its hall; by 'half', the
half of its double; by 'greater', greater than that which is less; by
'less,' less than that which is greater.

So it is with every other relative term; but the case we use to express
the correlation differs in some instances. Thus, by knowledge we mean
knowledge the knowable; by the knowable, that which is to be
apprehended by knowledge; by perception, perception of the perceptible;
by the perceptible, that which is apprehended by perception.

Sometimes, however, reciprocity of correlation does not appear to
exist. This comes about when a blunder is made, and that to which the
relative is related is not accurately stated. If a man states that a
wing is necessarily relative to a bird, the connexion between these two
will not be reciprocal, for it will not be possible to say that a bird
is a bird by reason of its wings. The reason is that the original
statement was inaccurate, for the wing is not said to be relative to
the bird qua bird, since many creatures besides birds have wings, but
qua winged creature. If, then, the statement is made accurate, the
connexion will be reciprocal, for we can speak of a wing, having
reference necessarily to a winged creature, and of a winged creature as
being such because of its wings.

Occasionally, perhaps, it is necessary to coin words, if no word exists
by which a correlation can adequately be explained. If we define a
rudder as necessarily having reference to a boat, our definition will
not be appropriate, for the rudder does not have this reference to a
boat qua boat, as there are boats which have no rudders. Thus we cannot
use the terms reciprocally, for the word 'boat' cannot be said to find
its explanation in the word 'rudder'. As there is no existing word, our
definition would perhaps be more accurate if we coined some word like
'ruddered' as the correlative of 'rudder'. If we express ourselves thus
accurately, at any rate the terms are reciprocally connected, for the
'ruddered' thing is 'ruddered' in virtue of its rudder. So it is in all
other cases. A head will be more accurately defined as the correlative
of that which is 'headed', than as that of an animal, for the animal
does not have a head qua animal, since many animals have no head.

Thus we may perhaps most easily comprehend that to which a thing is
related, when a name does not exist, if, from that which has a name, we
derive a new name, and apply it to that with which the first is
reciprocally connected, as in the aforesaid instances, when we derived
the word 'winged' from 'wing' and from 'rudder'.

All relatives, then, if properly defined, have a correlative. I add
this condition because, if that to which they are related is stated as
haphazard and not accurately, the two are not found to be
interdependent. Let me state what I mean more clearly. Even in the case
of acknowledged correlatives, and where names exist for each, there
will be no interdependence if one of the two is denoted, not by that
name which expresses the correlative notion, but by one of irrelevant
significance. The term 'slave,' if defined as related, not to a master,
but to a man, or a biped, or anything of that sort, is not reciprocally
connected with that in relation to which it is defined, for the
statement is not exact. Further, if one thing is said to be correlative
with another, and the terminology used is correct, then, though all
irrelevant attributes should be removed, and only that one attribute
left in virtue of which it was correctly stated to be correlative with
that other, the stated correlation will still exist. If the correlative
of 'the slave' is said to be 'the master', then, though all irrelevant
attributes of the said 'master', such as 'biped', 'receptive of
knowledge', 'human', should be removed, and the attribute 'master'
alone left, the stated correlation existing between him and the slave
will remain the same, for it is of a master that a slave is said to be
the slave. On the other hand, if, of two correlatives, one is not
correctly termed, then, when all other attributes are removed and that
alone is left in virtue of which it was stated to be correlative, the
stated correlation will be found to have disappeared.

For suppose the correlative of 'the slave' should be said to be 'the
man', or the correlative of 'the wing is the bird'; if the attribute
'master' be withdrawn from' the man', the correlation between 'the man'
and 'the slave' will cease to exist, for if the man is not a master,
the slave is not a slave. Similarly, if the attribute 'winged' be
withdrawn from 'the bird', 'the wing' will no longer be relative; for
if the so-called correlative is not winged, it follows that 'the wing'
has no correlative.

Thus it is essential that the correlated terms should be exactly
designated; if there is a name existing, the statement will be easy; if
not, it is doubtless our duty to construct names. When the terminology
is thus correct, it is evident that all correlatives are interdependent.

Correlatives are thought to come into existence simultaneously. This is
for the most part true, as in the case of the double and the half. The
existence of the half necessitates the existence of that of which it is
a half. Similarly the existence of a master necessitates the existence
of a slave, and that of a slave implies that of a master; these are
merely instances of a general rule. Moreover, they cancel one another;
for if there is no double it follows that there is no half, and vice
versa; this rule also applies to all such correlatives. Yet it does not
appear to be true in all cases that correlatives come into existence
simultaneously. The object of knowledge would appear to exist before
knowledge itself, for it is usually the case that we acquire knowledge
of objects already existing; it would be difficult, if not impossible,
to find a branch of knowledge the beginning of the existence of which
was contemporaneous with that of its object.

Again, while the object of knowledge, if it ceases to exist, cancels at
the same time the knowledge which was its correlative, the converse of
this is not true. It is true that if the object of knowledge does not
exist there can be no knowledge: for there will no longer be anything
to know. Yet it is equally true that, if knowledge of a certain object
does not exist, the object may nevertheless quite well exist. Thus, in
the case of the squaring of the circle, if indeed that process is an
object of knowledge, though it itself exists as an object of knowledge,
yet the knowledge of it has not yet come into existence. Again, if all
animals ceased to exist, there would be no knowledge, but there might
yet be many objects of knowledge.

This is likewise the case with regard to perception: for the object of
perception is, it appears, prior to the act of perception. If the
perceptible is annihilated, perception also will cease to exist; but
the annihilation of perception does not cancel the existence of the
perceptible. For perception implies a body perceived and a body in
which perception takes place. Now if that which is perceptible is
annihilated, it follows that the body is annihilated, for the body is a
perceptible thing; and if the body does not exist, it follows that
perception also ceases to exist. Thus the annihilation of the
perceptible involves that of perception.

But the annihilation of perception does not involve that of the
perceptible. For if the animal is annihilated, it follows that
perception also is annihilated, but perceptibles such as body, heat,
sweetness, bitterness, and so on, will remain.

Again, perception is generated at the same time as the perceiving
subject, for it comes into existence at the same time as the animal.
But the perceptible surely exists before perception; for fire and water
and such elements, out of which the animal is itself composed, exist
before the animal is an animal at all, and before perception. Thus it
would seem that the perceptible exists before perception.

It may be questioned whether it is true that no substance is relative,
as seems to be the case, or whether exception is to be made in the case
of certain secondary substances. With regard to primary substances, it
is quite true that there is no such possibility, for neither wholes nor
parts of primary substances are relative. The individual man or ox is
not defined with reference to something external. Similarly with the
parts: a particular hand or head is not defined as a particular hand or
head of a particular person, but as the hand or head of a particular
person. It is true also, for the most part at least, in the case of
secondary substances; the species 'man' and the species 'ox' are not
defined with reference to anything outside themselves. Wood, again, is
only relative in so far as it is some one's property, not in so far as
it is wood. It is plain, then, that in the cases mentioned substance is
not relative. But with regard to some secondary substances there is a
difference of opinion; thus, such terms as 'head' and 'hand' are
defined with reference to that of which the things indicated are a
part, and so it comes about that these appear to have a relative
character. Indeed, if our definition of that which is relative was
complete, it is very difficult, if not impossible, to prove that no
substance is relative. If, however, our definition was not complete, if
those things only are properly called relative in the case of which
relation to an external object is a necessary condition of existence,
perhaps some explanation of the dilemma may be found.

The former definition does indeed apply to all relatives, but the fact
that a thing is explained with reference to something else does not
make it essentially relative.

From this it is plain that, if a man definitely apprehends a relative
thing, he will also definitely apprehend that to which it is relative.
Indeed this is self-evident: for if a man knows that some particular
thing is relative, assuming that we call that a relative in the case of
which relation to something is a necessary condition of existence, he
knows that also to which it is related. For if he does not know at all
that to which it is related, he will not know whether or not it is
relative. This is clear, moreover, in particular instances. If a man
knows definitely that such and such a thing is 'double', he will also
forthwith know definitely that of which it is the double. For if there
is nothing definite of which he knows it to be the double, he does not
know at all that it is double. Again, if he knows that a thing is more
beautiful, it follows necessarily that he will forthwith definitely
know that also than which it is more beautiful. He will not merely know
indefinitely that it is more beautiful than something which is less
beautiful, for this would be supposition, not knowledge. For if he does
not know definitely that than which it is more beautiful, he can no
longer claim to know definitely that it is more beautiful than
something else which is less beautiful: for it might be that nothing
was less beautiful. It is, therefore, evident that if a man apprehends
some relative thing definitely, he necessarily knows that also
definitely to which it is related.

Now the head, the hand, and such things are substances, and it is
possible to know their essential character definitely, but it does not
necessarily follow that we should know that to which they are related.
It is not possible to know forthwith whose head or hand is meant. Thus
these are not relatives, and, this being the case, it would be true to
say that no substance is relative in character. It is perhaps a
difficult matter, in such cases, to make a positive statement without
more exhaustive examination, but to have raised questions with regard
to details is not without advantage.



Part 8

By 'quality' I mean that in virtue of which people are said to be such
and such.

Quality is a term that is used in many senses. One sort of quality let
us call 'habit' or 'disposition'. Habit differs from disposition in
being more lasting and more firmly established. The various kinds of
knowledge and of virtue are habits, for knowledge, even when acquired
only in a moderate degree, is, it is agreed, abiding in its character
and difficult to displace, unless some great mental upheaval takes
place, through disease or any such cause. The virtues, also, such as
justice, self-restraint, and so on, are not easily dislodged or
dismissed, so as to give place to vice.

By a disposition, on the other hand, we mean a condition that is easily
changed and quickly gives place to its opposite. Thus, heat, cold,
disease, health, and so on are dispositions. For a man is disposed in
one way or another with reference to these, but quickly changes,
becoming cold instead of warm, ill instead of well. So it is with all
other dispositions also, unless through lapse of time a disposition has
itself become inveterate and almost impossible to dislodge: in which
case we should perhaps go so far as to call it a habit.

It is evident that men incline to call those conditions habits which
are of a more or less permanent type and difficult to displace; for
those who are not retentive of knowledge, but volatile, are not said to
have such and such a 'habit' as regards knowledge, yet they are
disposed, we may say, either better or worse, towards knowledge. Thus
habit differs from disposition in this, that while the latter in
ephemeral, the former is permanent and difficult to alter.

Habits are at the same time dispositions, but dispositions are not
necessarily habits. For those who have some specific habit may be said
also, in virtue of that habit, to be thus or thus disposed; but those
who are disposed in some specific way have not in all cases the
corresponding habit.

Another sort of quality is that in virtue of which, for example, we
call men good boxers or runners, or healthy or sickly: in fact it
includes all those terms which refer to inborn capacity or incapacity.
Such things are not predicated of a person in virtue of his
disposition, but in virtue of his inborn capacity or incapacity to do
something with ease or to avoid defeat of any kind. Persons are called
good boxers or good runners, not in virtue of such and such a
disposition, but in virtue of an inborn capacity to accomplish
something with ease. Men are called healthy in virtue of the inborn
capacity of easy resistance to those unhealthy influences that may
ordinarily arise; unhealthy, in virtue of the lack of this capacity.
Similarly with regard to softness and hardness. Hardness is predicated
of a thing because it has that capacity of resistance which enables it
to withstand disintegration; softness, again, is predicated of a thing
by reason of the lack of that capacity.

A third class within this category is that of affective qualities and
affections. Sweetness, bitterness, sourness, are examples of this sort
of quality, together with all that is akin to these; heat, moreover,
and cold, whiteness, and blackness are affective qualities. It is
evident that these are qualities, for those things that possess them
are themselves said to be such and such by reason of their presence.
Honey is called sweet because it contains sweetness; the body is called
white because it contains whiteness; and so in all other cases.

The term 'affective quality' is not used as indicating that those
things which admit these qualities are affected in any way. Honey is
not called sweet because it is affected in a specific way, nor is this
what is meant in any other instance. Similarly heat and cold are called
affective qualities, not because those things which admit them are
affected. What is meant is that these said qualities are capable of
producing an 'affection' in the way of perception. For sweetness has
the power of affecting the sense of taste; heat, that of touch; and so
it is with the rest of these qualities.

Whiteness and blackness, however, and the other colours, are not said
to be affective qualities in this sense, but -because they themselves
are the results of an affection. It is plain that many changes of
colour take place because of affections. When a man is ashamed, he
blushes; when he is afraid, he becomes pale, and so on. So true is
this, that when a man is by nature liable to such affections, arising
from some concomitance of elements in his constitution, it is a
probable inference that he has the corresponding complexion of skin.
For the same disposition of bodily elements, which in the former
instance was momentarily present in the case of an access of shame,
might be a result of a man's natural temperament, so as to produce the
corresponding colouring also as a natural characteristic. All
conditions, therefore, of this kind, if caused by certain permanent and
lasting affections, are called affective qualities. For pallor and
duskiness of complexion are called qualities, inasmuch as we are said
to be such and such in virtue of them, not only if they originate in
natural constitution, but also if they come about through long disease
or sunburn, and are difficult to remove, or indeed remain throughout
life. For in the same way we are said to be such and such because of
these.

Those conditions, however, which arise from causes which may easily be
rendered ineffective or speedily removed, are called, not qualities,
but affections: for we are not said to be such virtue of them. The man
who blushes through shame is not said to be a constitutional blusher,
nor is the man who becomes pale through fear said to be
constitutionally pale. He is said rather to have been affected.

Thus such conditions are called affections, not qualities. In like
manner there are affective qualities and affections of the soul. That
temper with which a man is born and which has its origin in certain
deep-seated affections is called a quality. I mean such conditions as
insanity, irascibility, and so on: for people are said to be mad or
irascible in virtue of these. Similarly those abnormal psychic states
which are not inborn, but arise from the concomitance of certain other
elements, and are difficult to remove, or altogether permanent, are
called qualities, for in virtue of them men are said to be such and
such.

Those, however, which arise from causes easily rendered ineffective are
called affections, not qualities. Suppose that a man is irritable when
vexed: he is not even spoken of as a bad-tempered man, when in such
circumstances he loses his temper somewhat, but rather is said to be
affected. Such conditions are therefore termed, not qualities, but
affections.

The fourth sort of quality is figure and the shape that belongs to a
thing; and besides this, straightness and curvedness and any other
qualities of this type; each of these defines a thing as being such and
such. Because it is triangular or quadrangular a thing is said to have
a specific character, or again because it is straight or curved; in
fact a thing's shape in every case gives rise to a qualification of it.

Rarity and density, roughness and smoothness, seem to be terms
indicating quality: yet these, it would appear, really belong to a
class different from that of quality. For it is rather a certain
relative position of the parts composing the thing thus qualified
which, it appears, is indicated by each of these terms. A thing is
dense, owing to the fact that its parts are closely combined with one
another; rare, because there are interstices between the parts; smooth,
because its parts lie, so to speak, evenly; rough, because some parts
project beyond others.

There may be other sorts of quality, but those that are most properly
so called have, we may safely say, been enumerated.

These, then, are qualities, and the things that take their name from
them as derivatives, or are in some other way dependent on them, are
said to be qualified in some specific way. In most, indeed in almost
all cases, the name of that which is qualified is derived from that of
the quality. Thus the terms 'whiteness', 'grammar', 'justice', give us
the adjectives 'white', 'grammatical', 'just', and so on.

There are some cases, however, in which, as the quality under
consideration has no name, it is impossible that those possessed of it
should have a name that is derivative. For instance, the name given to
the runner or boxer, who is so called in virtue of an inborn capacity,
is not derived from that of any quality; for lob those capacities have
no name assigned to them. In this, the inborn capacity is distinct from
the science, with reference to which men are called, e.g. boxers or
wrestlers. Such a science is classed as a disposition; it has a name,
and is called 'boxing' or 'wrestling' as the case may be, and the name
given to those disposed in this way is derived from that of the
science. Sometimes, even though a name exists for the quality, that
which takes its character from the quality has a name that is not a
derivative. For instance, the upright man takes his character from the
possession of the quality of integrity, but the name given him is not
derived from the word 'integrity'. Yet this does not occur often.

We may therefore state that those things are said to be possessed of
some specific quality which have a name derived from that of the
aforesaid quality, or which are in some other way dependent on it.

One quality may be the contrary of another; thus justice is the
contrary of injustice, whiteness of blackness, and so on. The things,
also, which are said to be such and such in virtue of these qualities,
may be contrary the one to the other; for that which is unjust is
contrary to that which is just, that which is white to that which is
black. This, however, is not always the case. Red, yellow, and such
colours, though qualities, have no contraries.

If one of two contraries is a quality, the other will also be a
quality. This will be evident from particular instances, if we apply
the names used to denote the other categories; for instance, granted
that justice is the contrary of injustice and justice is a quality,
injustice will also be a quality: neither quantity, nor relation, nor
place, nor indeed any other category but that of quality, will be
applicable properly to injustice. So it is with all other contraries
falling under the category of quality.

Qualities admit of variation of degree. Whiteness is predicated of one
thing in a greater or less degree than of another. This is also the
case with reference to justice. Moreover, one and the same thing may
exhibit a quality in a greater degree than it did before: if a thing is
white, it may become whiter.

Though this is generally the case, there are exceptions. For if we
should say that justice admitted of variation of degree, difficulties
might ensue, and this is true with regard to all those qualities which
are dispositions. There are some, indeed, who dispute the possibility
of variation here. They maintain that justice and health cannot very
well admit of variation of degree themselves, but that people vary in
the degree in which they possess these qualities, and that this is the
case with grammatical learning and all those qualities which are
classed as dispositions. However that may be, it is an incontrovertible
fact that the things which in virtue of these qualities are said to be
what they are vary in the degree in which they possess them; for one
man is said to be better versed in grammar, or more healthy or just,
than another, and so on.

The qualities expressed by the terms 'triangular' and 'quadrangular' do
not appear to admit of variation of degree, nor indeed do any that have
to do with figure. For those things to which the definition of the
triangle or circle is applicable are all equally triangular or
circular. Those, on the other hand, to which the same definition is not
applicable, cannot be said to differ from one another in degree; the
square is no more a circle than the rectangle, for to neither is the
definition of the circle appropriate. In short, if the definition of
the term proposed is not applicable to both objects, they cannot be
compared. Thus it is not all qualities which admit of variation of
degree.

Whereas none of the characteristics I have mentioned are peculiar to
quality, the fact that likeness and unlikeness can be predicated with
reference to quality only, gives to that category its distinctive
feature. One thing is like another only with reference to that in
virtue of which it is such and such; thus this forms the peculiar mark
of quality.

We must not be disturbed because it may be argued that, though
proposing to discuss the category of quality, we have included in it
many relative terms. We did say that habits and dispositions were
relative. In practically all such cases the genus is relative, the
individual not. Thus knowledge, as a genus, is explained by reference
to something else, for we mean a knowledge of something. But particular
branches of knowledge are not thus explained. The knowledge of grammar
is not relative to anything external, nor is the knowledge of music,
but these, if relative at all, are relative only in virtue of their
genera; thus grammar is said be the knowledge of something, not the
grammar of something; similarly music is the knowledge of something,
not the music of something.

Thus individual branches of knowledge are not relative. And it is
because we possess these individual branches of knowledge that we are
said to be such and such. It is these that we actually possess: we are
called experts because we possess knowledge in some particular branch.
Those particular branches, therefore, of knowledge, in virtue of which
we are sometimes said to be such and such, are themselves qualities,
and are not relative. Further, if anything should happen to fall within
both the category of quality and that of relation, there would be
nothing extraordinary in classing it under both these heads.



Section 3


Part 9

Action and affection both admit of contraries and also of variation of
degree. Heating is the contrary of cooling, being heated of being
cooled, being glad of being vexed. Thus they admit of contraries. They
also admit of variation of degree: for it is possible to heat in a
greater or less degree; also to be heated in a greater or less degree.
Thus action and affection also admit of variation of degree. So much,
then, is stated with regard to these categories.

We spoke, moreover, of the category of position when we were dealing
with that of relation, and stated that such terms derived their names
from those of the corresponding attitudes.

As for the rest, time, place, state, since they are easily
intelligible, I say no more about them than was said at the beginning,
that in the category of state are included such states as 'shod',
'armed', in that of place 'in the Lyceum' and so on, as was explained
before.



Part 10

The proposed categories have, then, been adequately dealt with. We must
next explain the various senses in which the term 'opposite' is used.
Things are said to be opposed in four senses: (i) as correlatives to
one another, (ii) as contraries to one another, (iii) as privatives to
positives, (iv) as affirmatives to negatives.

Let me sketch my meaning in outline. An instance of the use of the word
'opposite' with reference to correlatives is afforded by the
expressions 'double' and 'half'; with reference to contraries by 'bad'
and 'good'. Opposites in the sense of 'privatives' and 'positives' are'
blindness' and 'sight'; in the sense of affirmatives and negatives, the
propositions 'he sits', 'he does not sit'.

(i) Pairs of opposites which fall under the category of relation are
explained by a reference of the one to the other, the reference being
indicated by the preposition 'of' or by some other preposition. Thus,
double is a relative term, for that which is double is explained as the
double of something. Knowledge, again, is the opposite of the thing
known, in the same sense; and the thing known also is explained by its
relation to its opposite, knowledge. For the thing known is explained
as that which is known by something, that is, by knowledge. Such
things, then, as are opposite the one to the other in the sense of
being correlatives are explained by a reference of the one to the other.

(ii) Pairs of opposites which are contraries are not in any way
interdependent, but are contrary the one to the other. The good is not
spoken of as the good of the bad, but as the contrary of the bad, nor
is white spoken of as the white of the black, but as the contrary of
the black. These two types of opposition are therefore distinct. Those
contraries which are such that the subjects in which they are naturally
present, or of which they are predicated, must necessarily contain
either the one or the other of them, have no intermediate, but those in
the case of which no such necessity obtains, always have an
intermediate. Thus disease and health are naturally present in the body
of an animal, and it is necessary that either the one or the other
should be present in the body of an animal. Odd and even, again, are
predicated of number, and it is necessary that the one or the other
should be present in numbers. Now there is no intermediate between the
terms of either of these two pairs. On the other hand, in those
contraries with regard to which no such necessity obtains, we find an
intermediate. Blackness and whiteness are naturally present in the
body, but it is not necessary that either the one or the other should
be present in the body, inasmuch as it is not true to say that
everybody must be white or black. Badness and goodness, again, are
predicated of man, and of many other things, but it is not necessary
that either the one quality or the other should be present in that of
which they are predicated: it is not true to say that everything that
may be good or bad must be either good or bad. These pairs of
contraries have intermediates: the intermediates between white and
black are grey, sallow, and all the other colours that come between;
the intermediate between good and bad is that which is neither the one
nor the other.

Some intermediate qualities have names, such as grey and sallow and all
the other colours that come between white and black; in other cases,
however, it is not easy to name the intermediate, but we must define it
as that which is not either extreme, as in the case of that which is
neither good nor bad, neither just nor unjust.

(iii) 'privatives' and 'Positives' have reference to the same subject.
Thus, sight and blindness have reference to the eye. It is a universal
rule that each of a pair of opposites of this type has reference to
that to which the particular 'positive' is natural. We say that that is
capable of some particular faculty or possession has suffered privation
when the faculty or possession in question is in no way present in that
in which, and at the time at which, it should naturally be present. We
do not call that toothless which has not teeth, or that blind which has
not sight, but rather that which has not teeth or sight at the time
when by nature it should. For there are some creatures which from birth
are without sight, or without teeth, but these are not called toothless
or blind.

To be without some faculty or to possess it is not the same as the
corresponding 'privative' or 'positive'. 'Sight' is a 'positive',
'blindness' a 'privative', but 'to possess sight' is not equivalent to
'sight', 'to be blind' is not equivalent to 'blindness'. Blindness is a
'privative', to be blind is to be in a state of privation, but is not a
'privative'. Moreover, if 'blindness' were equivalent to 'being blind',
both would be predicated of the same subject; but though a man is said
to be blind, he is by no means said to be blindness.

To be in a state of 'possession' is, it appears, the opposite of being
in a state of 'privation', just as 'positives' and 'privatives'
themselves are opposite. There is the same type of antithesis in both
cases; for just as blindness is opposed to sight, so is being blind
opposed to having sight.

That which is affirmed or denied is not itself affirmation or denial.
By 'affirmation' we mean an affirmative proposition, by 'denial' a
negative. Now, those facts which form the matter of the affirmation or
denial are not propositions; yet these two are said to be opposed in
the same sense as the affirmation and denial, for in this case also the
type of antithesis is the same. For as the affirmation is opposed to
the denial, as in the two propositions 'he sits', 'he does not sit', so
also the fact which constitutes the matter of the proposition in one
case is opposed to that in the other, his sitting, that is to say, to
his not sitting.

It is evident that 'positives' and 'privatives' are not opposed each to
each in the same sense as relatives. The one is not explained by
reference to the other; sight is not sight of blindness, nor is any
other preposition used to indicate the relation. Similarly blindness is
not said to be blindness of sight, but rather, privation of sight.
Relatives, moreover, reciprocate; if blindness, therefore, were a
relative, there would be a reciprocity of relation between it and that
with which it was correlative. But this is not the case. Sight is not
called the sight of blindness.

That those terms which fall under the heads of 'positives' and
'privatives' are not opposed each to each as contraries, either, is
plain from the following facts: Of a pair of contraries such that they
have no intermediate, one or the other must needs be present in the
subject in which they naturally subsist, or of which they are
predicated; for it is those, as we proved,' in the case of which this
necessity obtains, that have no intermediate. Moreover, we cited health
and disease, odd and even, as instances. But those contraries which
have an intermediate are not subject to any such necessity. It is not
necessary that every substance, receptive of such qualities, should be
either black or white, cold or hot, for something intermediate between
these contraries may very well be present in the subject. We proved,
moreover, that those contraries have an intermediate in the case of
which the said necessity does not obtain. Yet when one of the two
contraries is a constitutive property of the subject, as it is a
constitutive property of fire to be hot, of snow to be white, it is
necessary determinately that one of the two contraries, not one or the
other, should be present in the subject; for fire cannot be cold, or
snow black. Thus, it is not the case here that one of the two must
needs be present in every subject receptive of these qualities, but
only in that subject of which the one forms a constitutive property.
Moreover, in such cases it is one member of the pair determinately, and
not either the one or the other, which must be present.

In the case of 'positives' and 'privatives', on the other hand, neither
of the aforesaid statements holds good. For it is not necessary that a
subject receptive of the qualities should always have either the one or
the other; that which has not yet advanced to the state when sight is
natural is not said either to be blind or to see. Thus 'positives' and
'privatives' do not belong to that class of contraries which consists
of those which have no intermediate. On the other hand, they do not
belong either to that class which consists of contraries which have an
intermediate. For under certain conditions it is necessary that either
the one or the other should form part of the constitution of every
appropriate subject. For when a thing has reached the stage when it is
by nature capable of sight, it will be said either to see or to be
blind, and that in an indeterminate sense, signifying that the capacity
may be either present or absent; for it is not necessary either that it
should see or that it should be blind, but that it should be either in
the one state or in the other. Yet in the case of those contraries
which have an intermediate we found that it was never necessary that
either the one or the other should be present in every appropriate
subject, but only that in certain subjects one of the pair should be
present, and that in a determinate sense. It is, therefore, plain that
'positives' and 'privatives' are not opposed each to each in either of
the senses in which contraries are opposed.

Again, in the case of contraries, it is possible that there should be
changes from either into the other, while the subject retains its
identity, unless indeed one of the contraries is a constitutive
property of that subject, as heat is of fire. For it is possible that
that that which is healthy should become diseased, that which is white,
black, that which is cold, hot, that which is good, bad, that which is
bad, good. The bad man, if he is being brought into a better way of
life and thought, may make some advance, however slight, and if he
should once improve, even ever so little, it is plain that he might
change completely, or at any rate make very great progress; for a man
becomes more and more easily moved to virtue, however small the
improvement was at first. It is, therefore, natural to suppose that he
will make yet greater progress than he has made in the past; and as
this process goes on, it will change him completely and establish him
in the contrary state, provided he is not hindered by lack of time. In
the case of 'positives' and 'privatives', however, change in both
directions is impossible. There may be a change from possession to
privation, but not from privation to possession. The man who has become
blind does not regain his sight; the man who has become bald does not
regain his hair; the man who has lost his teeth does not grow a new
set. (iv) Statements opposed as affirmation and negation belong
manifestly to a class which is distinct, for in this case, and in this
case only, it is necessary for the one opposite to be true and the
other false.

Neither in the case of contraries, nor in the case of correlatives, nor
in the case of 'positives' and 'privatives', is it necessary for one to
be true and the other false. Health and disease are contraries: neither
of them is true or false. 'Double' and 'half' are opposed to each other
as correlatives: neither of them is true or false. The case is the
same, of course, with regard to 'positives' and 'privatives' such as
'sight' and 'blindness'. In short, where there is no sort of
combination of words, truth and falsity have no place, and all the
opposites we have mentioned so far consist of simple words.

At the same time, when the words which enter into opposed statements
are contraries, these, more than any other set of opposites, would seem
to claim this characteristic. 'Socrates is ill' is the contrary of
'Socrates is well', but not even of such composite expressions is it
true to say that one of the pair must always be true and the other
false. For if Socrates exists, one will be true and the other false,
but if he does not exist, both will be false; for neither 'Socrates is
ill' nor 'Socrates is well' is true, if Socrates does not exist at all.

In the case of 'positives' and 'privatives', if the subject does not
exist at all, neither proposition is true, but even if the subject
exists, it is not always the fact that one is true and the other false.
For 'Socrates has sight' is the opposite of 'Socrates is blind' in the
sense of the word 'opposite' which applies to possession and privation.
Now if Socrates exists, it is not necessary that one should be true and
the other false, for when he is not yet able to acquire the power of
vision, both are false, as also if Socrates is altogether non-existent.

But in the case of affirmation and negation, whether the subject exists
or not, one is always false and the other true. For manifestly, if
Socrates exists, one of the two propositions 'Socrates is ill',
'Socrates is not ill', is true, and the other false. This is likewise
the case if he does not exist; for if he does not exist, to say that he
is ill is false, to say that he is not ill is true. Thus it is in the
case of those opposites only, which are opposite in the sense in which
the term is used with reference to affirmation and negation, that the
rule holds good, that one of the pair must be true and the other false.



Part 11

That the contrary of a good is an evil is shown by induction: the
contrary of health is disease, of courage, cowardice, and so on. But
the contrary of an evil is sometimes a good, sometimes an evil. For
defect, which is an evil, has excess for its contrary, this also being
an evil, and the mean, which is a good, is equally the contrary of the
one and of the other. It is only in a few cases, however, that we see
instances of this: in most, the contrary of an evil is a good.

In the case of contraries, it is not always necessary that if one
exists the other should also exist: for if all become healthy there
will be health and no disease, and again, if everything turns white,
there will be white, but no black. Again, since the fact that Socrates
is ill is the contrary of the fact that Socrates is well, and two
contrary conditions cannot both obtain in one and the same individual
at the same time, both these contraries could not exist at once: for if
that Socrates was well was a fact, then that Socrates was ill could not
possibly be one.

It is plain that contrary attributes must needs be present in subjects
which belong to the same species or genus. Disease and health require
as their subject the body of an animal; white and black require a body,
without further qualification; justice and injustice require as their
subject the human soul.

Moreover, it is necessary that pairs of contraries should in all cases
either belong to the same genus or belong to contrary genera or be
themselves genera. White and black belong to the same genus, colour;
justice and injustice, to contrary genera, virtue and vice; while good
and evil do not belong to genera, but are themselves actual genera,
with terms under them.



Part 12

There are four senses in which one thing can be said to be 'prior' to
another. Primarily and most properly the term has reference to time: in
this sense the word is used to indicate that one thing is older or more
ancient than another, for the expressions 'older' and 'more ancient'
imply greater length of time.

Secondly, one thing is said to be 'prior' to another when the sequence
of their being cannot be reversed. In this sense 'one' is 'prior' to
'two'. For if 'two' exists, it follows directly that 'one' must exist,
but if 'one' exists, it does not follow necessarily that 'two' exists:
thus the sequence subsisting cannot be reversed. It is agreed, then,
that when the sequence of two things cannot be reversed, then that one
on which the other depends is called 'prior' to that other.

In the third place, the term 'prior' is used with reference to any
order, as in the case of science and of oratory. For in sciences which
use demonstration there is that which is prior and that which is
posterior in order; in geometry, the elements are prior to the
propositions; in reading and writing, the letters of the alphabet are
prior to the syllables. Similarly, in the case of speeches, the
exordium is prior in order to the narrative.

Besides these senses of the word, there is a fourth. That which is
better and more honourable is said to have a natural priority. In
common parlance men speak of those whom they honour and love as 'coming
first' with them. This sense of the word is perhaps the most
far-fetched.

Such, then, are the different senses in which the term 'prior' is used.

Yet it would seem that besides those mentioned there is yet another.
For in those things, the being of each of which implies that of the
other, that which is in any way the cause may reasonably be said to be
by nature 'prior' to the effect. It is plain that there are instances
of this. The fact of the being of a man carries with it the truth of
the proposition that he is, and the implication is reciprocal: for if a
man is, the proposition wherein we allege that he is true, and
conversely, if the proposition wherein we allege that he is true, then
he is. The true proposition, however, is in no way the cause of the
being of the man, but the fact of the man's being does seem somehow to
be the cause of the truth of the proposition, for the truth or falsity
of the proposition depends on the fact of the man's being or not being.

Thus the word 'prior' may be used in five senses.



Part 13

The term 'simultaneous' is primarily and most appropriately applied to
those things the genesis of the one of which is simultaneous with that
of the other; for in such cases neither is prior or posterior to the
other. Such things are said to be simultaneous in point of time. Those
things, again, are 'simultaneous' in point of nature, the being of each
of which involves that of the other, while at the same time neither is
the cause of the other's being. This is the case with regard to the
double and the half, for these are reciprocally dependent, since, if
there is a double, there is also a half, and if there is a half, there
is also a double, while at the same time neither is the cause of the
being of the other.

Again, those species which are distinguished one from another and
opposed one to another within the same genus are said to be
'simultaneous' in nature. I mean those species which are distinguished
each from each by one and the same method of division. Thus the
'winged' species is simultaneous with the 'terrestrial' and the 'water'
species. These are distinguished within the same genus, and are opposed
each to each, for the genus 'animal' has the 'winged', the
'terrestrial', and the 'water' species, and no one of these is prior or
posterior to another; on the contrary, all such things appear to be
'simultaneous' in nature. Each of these also, the terrestrial, the
winged, and the water species, can be divided again into subspecies.
Those species, then, also will be 'simultaneous' point of nature,
which, belonging to the same genus, are distinguished each from each by
one and the same method of differentiation.

But genera are prior to species, for the sequence of their being cannot
be reversed. If there is the species 'water-animal', there will be the
genus 'animal', but granted the being of the genus 'animal', it does
not follow necessarily that there will be the species 'water-animal'.

Those things, therefore, are said to be 'simultaneous' in nature, the
being of each of which involves that of the other, while at the same
time neither is in any way the cause of the other's being; those
species, also, which are distinguished each from each and opposed
within the same genus. Those things, moreover, are 'simultaneous' in
the unqualified sense of the word which come into being at the same
time.



Part 14

There are six sorts of movement: generation, destruction, increase,
diminution, alteration, and change of place.

It is evident in all but one case that all these sorts of movement are
distinct each from each. Generation is distinct from destruction,
increase and change of place from diminution, and so on. But in the
case of alteration it may be argued that the process necessarily
implies one or other of the other five sorts of motion. This is not
true, for we may say that all affections, or nearly all, produce in us
an alteration which is distinct from all other sorts of motion, for
that which is affected need not suffer either increase or diminution or
any of the other sorts of motion. Thus alteration is a distinct sort of
motion; for, if it were not, the thing altered would not only be
altered, but would forthwith necessarily suffer increase or diminution
or some one of the other sorts of motion in addition; which as a matter
of fact is not the case. Similarly that which was undergoing the
process of increase or was subject to some other sort of motion would,
if alteration were not a distinct form of motion, necessarily be
subject to alteration also. But there are some things which undergo
increase but yet not alteration. The square, for instance, if a gnomon
is applied to it, undergoes increase but not alteration, and so it is
with all other figures of this sort. Alteration and increase,
therefore, are distinct.

Speaking generally, rest is the contrary of motion. But the different
forms of motion have their own contraries in other forms; thus
destruction is the contrary of generation, diminution of increase, rest
in a place, of change of place. As for this last, change in the reverse
direction would seem to be most truly its contrary; thus motion upwards
is the contrary of motion downwards and vice versa.

In the case of that sort of motion which yet remains, of those that
have been enumerated, it is not easy to state what is its contrary. It
appears to have no contrary, unless one should define the contrary here
also either as 'rest in its quality' or as 'change in the direction of
the contrary quality', just as we defined the contrary of change of
place either as rest in a place or as change in the reverse direction.
For a thing is altered when change of quality takes place; therefore
either rest in its quality or change in the direction of the contrary
may be called the contrary of this qualitative form of motion. In this
way becoming white is the contrary of becoming black; there is
alteration in the contrary direction, since a change of a qualitative
nature takes place.



Part 15

The term 'to have' is used in various senses. In the first place it is
used with reference to habit or disposition or any other quality, for
we are said to 'have' a piece of knowledge or a virtue. Then, again, it
has reference to quantity, as, for instance, in the case of a man's
height; for he is said to 'have' a height of three or four cubits. It
is used, moreover, with regard to apparel, a man being said to 'have' a
coat or tunic; or in respect of something which we have on a part of
ourselves, as a ring on the hand: or in respect of something which is a
part of us, as hand or foot. The term refers also to content, as in the
case of a vessel and wheat, or of a jar and wine; a jar is said to
'have' wine, and a corn-measure wheat. The expression in such cases has
reference to content. Or it refers to that which has been acquired; we
are said to 'have' a house or a field. A man is also said to 'have' a
wife, and a wife a husband, and this appears to be the most remote
meaning of the term, for by the use of it we mean simply that the
husband lives with the wife.

Other senses of the word might perhaps be found, but the most ordinary
ones have all been enumerated.






% chapter categories (end)
% \chapter{Physics} % (fold)
\label{cha:physics}

Physics
By Aristotle


Translated by R. P. Hardie and R. K. Gaye

----------------------------------------------------------------------

BOOK I

Part 1 

When the objects of an inquiry, in any department, have principles,
conditions, or elements, it is through acquaintance with these that
knowledge, that is to say scientific knowledge, is attained. For we
do not think that we know a thing until we are acquainted with its
primary conditions or first principles, and have carried our analysis
as far as its simplest elements. Plainly therefore in the science
of Nature, as in other branches of study, our first task will be to
try to determine what relates to its principles. 

The natural way of doing this is to start from the things which are
more knowable and obvious to us and proceed towards those which are
clearer and more knowable by nature; for the same things are not 'knowable
relatively to us' and 'knowable' without qualification. So in the
present inquiry we must follow this method and advance from what is
more obscure by nature, but clearer to us, towards what is more clear
and more knowable by nature. 

Now what is to us plain and obvious at first is rather confused masses,
the elements and principles of which become known to us later by analysis.
Thus we must advance from generalities to particulars; for it is a
whole that is best known to sense-perception, and a generality is
a kind of whole, comprehending many things within it, like parts.
Much the same thing happens in the relation of the name to the formula.
A name, e.g. 'round', means vaguely a sort of whole: its definition
analyses this into its particular senses. Similarly a child begins
by calling all men 'father', and all women 'mother', but later on
distinguishes each of them. 

Part 2

The principles in question must be either (a) one or (b) more than
one. If (a) one, it must be either (i) motionless, as Parmenides and
Melissus assert, or (ii) in motion, as the physicists hold, some declaring
air to be the first principle, others water. If (b) more than one,
then either (i) a finite or (ii) an infinite plurality. If (i) finite
(but more than one), then either two or three or four or some other
number. If (ii) infinite, then either as Democritus believed one in
kind, but differing in shape or form; or different in kind and even
contrary. 

A similar inquiry is made by those who inquire into the number of
existents: for they inquire whether the ultimate constituents of existing
things are one or many, and if many, whether a finite or an infinite
plurality. So they too are inquiring whether the principle or element
is one or many. 

Now to investigate whether Being is one and motionless is not a contribution
to the science of Nature. For just as the geometer has nothing more
to say to one who denies the principles of his science-this being
a question for a different science or for or common to all-so a man
investigating principles cannot argue with one who denies their existence.
For if Being is just one, and one in the way mentioned, there is a
principle no longer, since a principle must be the principle of some
thing or things. 

To inquire therefore whether Being is one in this sense would be like
arguing against any other position maintained for the sake of argument
(such as the Heraclitean thesis, or such a thesis as that Being is
one man) or like refuting a merely contentious argument-a description
which applies to the arguments both of Melissus and of Parmenides:
their premisses are false and their conclusions do not follow. Or
rather the argument of Melissus is gross and palpable and offers no
difficulty at all: accept one ridiculous proposition and the rest
follows-a simple enough proceeding. 

We physicists, on the other hand, must take for granted that the things
that exist by nature are, either all or some of them, in motion which
is indeed made plain by induction. Moreover, no man of science is
bound to solve every kind of difficulty that may be raised, but only
as many as are drawn falsely from the principles of the science: it
is not our business to refute those that do not arise in this way:
just as it is the duty of the geometer to refute the squaring of the
circle by means of segments, but it is not his duty to refute Antiphon's
proof. At the same time the holders of the theory of which we are
speaking do incidentally raise physical questions, though Nature is
not their subject: so it will perhaps be as well to spend a few words
on them, especially as the inquiry is not without scientific interest.

The most pertinent question with which to begin will be this: In what
sense is it asserted that all things are one? For 'is' is used in
many senses. Do they mean that all things 'are' substance or quantities
or qualities? And, further, are all things one substance-one man,
one horse, or one soul-or quality and that one and the same-white
or hot or something of the kind? These are all very different doctrines
and all impossible to maintain. 

For if both substance and quantity and quality are, then, whether
these exist independently of each other or not, Being will be many.

If on the other hand it is asserted that all things are quality or
quantity, then, whether substance exists or not, an absurdity results,
if the impossible can properly be called absurd. For none of the others
can exist independently: substance alone is independent: for everything
is predicated of substance as subject. Now Melissus says that Being
is infinite. It is then a quantity. For the infinite is in the category
of quantity, whereas substance or quality or affection cannot be infinite
except through a concomitant attribute, that is, if at the same time
they are also quantities. For to define the infinite you must use
quantity in your formula, but not substance or quality. If then Being
is both substance and quantity, it is two, not one: if only substance,
it is not infinite and has no magnitude; for to have that it will
have to be a quantity. 

Again, 'one' itself, no less than 'being', is used in many senses,
so we must consider in what sense the word is used when it is said
that the All is one. 

Now we say that (a) the continuous is one or that (b) the indivisible
is one, or (c) things are said to be 'one', when their essence is
one and the same, as 'liquor' and 'drink'. 

If (a) their One is one in the sense of continuous, it is many, for
the continuous is divisible ad infinitum. 

There is, indeed, a difficulty about part and whole, perhaps not relevant
to the present argument, yet deserving consideration on its own account-namely,
whether the part and the whole are one or more than one, and how they
can be one or many, and, if they are more than one, in what sense
they are more than one. (Similarly with the parts of wholes which
are not continuous.) Further, if each of the two parts is indivisibly
one with the whole, the difficulty arises that they will be indivisibly
one with each other also. 

But to proceed: If (b) their One is one as indivisible, nothing will
have quantity or quality, and so the one will not be infinite, as
Melissus says-nor, indeed, limited, as Parmenides says, for though
the limit is indivisible, the limited is not. 

But if (c) all things are one in the sense of having the same definition,
like 'raiment' and 'dress', then it turns out that they are maintaining
the Heraclitean doctrine, for it will be the same thing 'to be good'
and 'to be bad', and 'to be good' and 'to be not good', and so the
same thing will be 'good' and 'not good', and man and horse; in fact,
their view will be, not that all things are one, but that they are
nothing; and that 'to be of such-and-such a quality' is the same as
'to be of such-and-such a size'. 

Even the more recent of the ancient thinkers were in a pother lest
the same thing should turn out in their hands both one and many. So
some, like Lycophron, were led to omit 'is', others to change the
mode of expression and say 'the man has been whitened' instead of
'is white', and 'walks' instead of 'is walking', for fear that if
they added the word 'is' they should be making the one to be many-as
if 'one' and 'being' were always used in one and the same sense. What
'is' may be many either in definition (for example 'to be white' is
one thing, 'to be musical' another, yet the same thing be both, so
the one is many) or by division, as the whole and its parts. On this
point, indeed, they were already getting into difficulties and admitted
that the one was many-as if there was any difficulty about the same
thing being both one and many, provided that these are not opposites;
for 'one' may mean either 'potentially one' or 'actually one'.

Part 3

If, then, we approach the thesis in this way it seems impossible for
all things to be one. Further, the arguments they use to prove their
position are not difficult to expose. For both of them reason contentiously-I
mean both Melissus and Parmenides. [Their premisses are false and
their conclusions do not follow. Or rather the argument of Melissus
is gross and palpable and offers no difficulty at all: admit one ridiculous
proposition and the rest follows-a simple enough proceeding.] The
fallacy of Melissus is obvious. For he supposes that the assumption
'what has come into being always has a beginning' justifies the assumption
'what has not come into being has no beginning'. Then this also is
absurd, that in every case there should be a beginning of the thing-not
of the time and not only in the case of coming to be in the full sense
but also in the case of coming to have a quality-as if change never
took place suddenly. Again, does it follow that Being, if one, is
motionless? Why should it not move, the whole of it within itself,
as parts of it do which are unities, e.g. this water? Again, why is
qualitative change impossible? But, further, Being cannot be one in
form, though it may be in what it is made of. (Even some of the physicists
hold it to be one in the latter way, though not in the former.) Man
obviously differs from horse in form, and contraries from each other.

The same kind of argument holds good against Parmenides also, besides
any that may apply specially to his view: the answer to him being
that 'this is not true' and 'that does not follow'. His assumption
that one is used in a single sense only is false, because it is used
in several. His conclusion does not follow, because if we take only
white things, and if 'white' has a single meaning, none the less what
is white will be many and not one. For what is white will not be one
either in the sense that it is continuous or in the sense that it
must be defined in only one way. 'Whiteness' will be different from
'what has whiteness'. Nor does this mean that there is anything that
can exist separately, over and above what is white. For 'whiteness'
and 'that which is white' differ in definition, not in the sense that
they are things which can exist apart from each other. But Parmenides
had not come in sight of this distinction. 

It is necessary for him, then, to assume not only that 'being' has
the same meaning, of whatever it is predicated, but further that it
means (1) what just is and (2) what is just one. 

It must be so, for (1) an attribute is predicated of some subject,
so that the subject to which 'being' is attributed will not be, as
it is something different from 'being'. Something, therefore, which
is not will be. Hence 'substance' will not be a predicate of anything
else. For the subject cannot be a being, unless 'being' means several
things, in such a way that each is something. But ex hypothesi 'being'
means only one thing. 

If, then, 'substance' is not attributed to anything, but other things
are attributed to it, how does 'substance' mean what is rather than
what is not? For suppose that 'substance' is also 'white'. Since the
definition of the latter is different (for being cannot even be attributed
to white, as nothing is which is not 'substance'), it follows that
'white' is not-being--and that not in the sense of a particular not-being,
but in the sense that it is not at all. Hence 'substance' is not;
for it is true to say that it is white, which we found to mean not-being.
If to avoid this we say that even 'white' means substance, it follows
that 'being' has more than one meaning. 

In particular, then, Being will not have magnitude, if it is substance.
For each of the two parts must he in a different sense. 

(2) Substance is plainly divisible into other substances, if we consider
the mere nature of a definition. For instance, if 'man' is a substance,
'animal' and 'biped' must also be substances. For if not substances,
they must be attributes-and if attributes, attributes either of (a)
man or of (b) some other subject. But neither is possible.

(a) An attribute is either that which may or may not belong to the
subject or that in whose definition the subject of which it is an
attribute is involved. Thus 'sitting' is an example of a separable
attribute, while 'snubness' contains the definition of 'nose', to
which we attribute snubness. Further, the definition of the whole
is not contained in the definitions of the contents or elements of
the definitory formula; that of 'man' for instance in 'biped', or
that of 'white man' in 'white'. If then this is so, and if 'biped'
is supposed to be an attribute of 'man', it must be either separable,
so that 'man' might possibly not be 'biped', or the definition of
'man' must come into the definition of 'biped'-which is impossible,
as the converse is the case. 

(b) If, on the other hand, we suppose that 'biped' and 'animal' are
attributes not of man but of something else, and are not each of them
a substance, then 'man' too will be an attribute of something else.
But we must assume that substance is not the attribute of anything,
that the subject of which both 'biped' and 'animal' and each separately
are predicated is the subject also of the complex 'biped animal'.

Are we then to say that the All is composed of indivisible substances?
Some thinkers did, in point of fact, give way to both arguments. To
the argument that all things are one if being means one thing, they
conceded that not-being is; to that from bisection, they yielded by
positing atomic magnitudes. But obviously it is not true that if being
means one thing, and cannot at the same time mean the contradictory
of this, there will be nothing which is not, for even if what is not
cannot be without qualification, there is no reason why it should
not be a particular not-being. To say that all things will be one,
if there is nothing besides Being itself, is absurd. For who understands
'being itself' to be anything but a particular substance? But if this
is so, there is nothing to prevent there being many beings, as has
been said. 

It is, then, clearly impossible for Being to be one in this sense.

Part 4

The physicists on the other hand have two modes of explanation.

The first set make the underlying body one either one of the three
or something else which is denser than fire and rarer than air then
generate everything else from this, and obtain multiplicity by condensation
and rarefaction. Now these are contraries, which may be generalized
into 'excess and defect'. (Compare Plato's 'Great and Small'-except
that he make these his matter, the one his form, while the others
treat the one which underlies as matter and the contraries as differentiae,
i.e. forms). 

The second set assert that the contrarieties are contained in the
one and emerge from it by segregation, for example Anaximander and
also all those who assert that 'what is' is one and many, like Empedocles
and Anaxagoras; for they too produce other things from their mixture
by segregation. These differ, however, from each other in that the
former imagines a cycle of such changes, the latter a single series.
Anaxagoras again made both his 'homceomerous' substances and his contraries
infinite in multitude, whereas Empedocles posits only the so-called
elements. 

The theory of Anaxagoras that the principles are infinite in multitude
was probably due to his acceptance of the common opinion of the physicists
that nothing comes into being from not-being. For this is the reason
why they use the phrase 'all things were together' and the coming
into being of such and such a kind of thing is reduced to change of
quality, while some spoke of combination and separation. Moreover,
the fact that the contraries proceed from each other led them to the
conclusion. The one, they reasoned, must have already existed in the
other; for since everything that comes into being must arise either
from what is or from what is not, and it is impossible for it to arise
from what is not (on this point all the physicists agree), they thought
that the truth of the alternative necessarily followed, namely that
things come into being out of existent things, i.e. out of things
already present, but imperceptible to our senses because of the smallness
of their bulk. So they assert that everything has been mixed in every.
thing, because they saw everything arising out of everything. But
things, as they say, appear different from one another and receive
different names according to the nature of the particles which are
numerically predominant among the innumerable constituents of the
mixture. For nothing, they say, is purely and entirely white or black
or sweet, bone or flesh, but the nature of a thing is held to be that
of which it contains the most. 

Now (1) the infinite qua infinite is unknowable, so that what is infinite
in multitude or size is unknowable in quantity, and what is infinite
in variety of kind is unknowable in quality. But the principles in
question are infinite both in multitude and in kind. Therefore it
is impossible to know things which are composed of them; for it is
when we know the nature and quantity of its components that we suppose
we know a complex. 

Further (2) if the parts of a whole may be of any size in the direction
either of greatness or of smallness (by 'parts' I mean components
into which a whole can be divided and which are actually present in
it), it is necessary that the whole thing itself may be of any size.
Clearly, therefore, since it is impossible for an animal or plant
to be indefinitely big or small, neither can its parts be such, or
the whole will be the same. But flesh, bone, and the like are the
parts of animals, and the fruits are the parts of plants. Hence it
is obvious that neither flesh, bone, nor any such thing can be of
indefinite size in the direction either of the greater or of the less.

Again (3) according to the theory all such things are already present
in one another and do not come into being but are constituents which
are separated out, and a thing receives its designation from its chief
constituent. Further, anything may come out of anything-water by segregation
from flesh and flesh from water. Hence, since every finite body is
exhausted by the repeated abstraction of a finite body, it seems obviously
to follow that everything cannot subsist in everything else. For let
flesh be extracted from water and again more flesh be produced from
the remainder by repeating the process of separation: then, even though
the quantity separated out will continually decrease, still it will
not fall below a certain magnitude. If, therefore, the process comes
to an end, everything will not be in everything else (for there will
be no flesh in the remaining water); if on the other hand it does
not, and further extraction is always possible, there will be an infinite
multitude of finite equal particles in a finite quantity-which is
impossible. Another proof may be added: Since every body must diminish
in size when something is taken from it, and flesh is quantitatively
definite in respect both of greatness and smallness, it is clear that
from the minimum quantity of flesh no body can be separated out; for
the flesh left would be less than the minimum of flesh. 

Lastly (4) in each of his infinite bodies there would be already present
infinite flesh and blood and brain- having a distinct existence, however,
from one another, and no less real than the infinite bodies, and each
infinite: which is contrary to reason. 

The statement that complete separation never will take place is correct
enough, though Anaxagoras is not fully aware of what it means. For
affections are indeed inseparable. If then colours and states had
entered into the mixture, and if separation took place, there would
be a 'white' or a 'healthy' which was nothing but white or healthy,
i.e. was not the predicate of a subject. So his 'Mind' is an absurd
person aiming at the impossible, if he is supposed to wish to separate
them, and it is impossible to do so, both in respect of quantity and
of quality- of quantity, because there is no minimum magnitude, and
of quality, because affections are inseparable. 

Nor is Anaxagoras right about the coming to be of homogeneous bodies.
It is true there is a sense in which clay is divided into pieces of
clay, but there is another in which it is not. Water and air are,
and are generated 'from' each other, but not in the way in which bricks
come 'from' a house and again a house 'from' bricks; and it is better
to assume a smaller and finite number of principles, as Empedocles
does. 

Part 5

All thinkers then agree in making the contraries principles, both
those who describe the All as one and unmoved (for even Parmenides
treats hot and cold as principles under the names of fire and earth)
and those too who use the rare and the dense. The same is true of
Democritus also, with his plenum and void, both of which exist, be
says, the one as being, the other as not-being. Again he speaks of
differences in position, shape, and order, and these are genera of
which the species are contraries, namely, of position, above and below,
before and behind; of shape, angular and angle-less, straight and
round. 

It is plain then that they all in one way or another identify the
contraries with the principles. And with good reason. For first principles
must not be derived from one another nor from anything else, while
everything has to be derived from them. But these conditions are fulfilled
by the primary contraries, which are not derived from anything else
because they are primary, nor from each other because they are contraries.

But we must see how this can be arrived at as a reasoned result, as
well as in the way just indicated. 

Our first presupposition must be that in nature nothing acts on, or
is acted on by, any other thing at random, nor may anything come from
anything else, unless we mean that it does so in virtue of a concomitant
attribute. For how could 'white' come from 'musical', unless 'musical'
happened to be an attribute of the not-white or of the black? No,
'white' comes from 'not-white'-and not from any 'not-white', but from
black or some intermediate colour. Similarly, 'musical' comes to be
from 'not-musical', but not from any thing other than musical, but
from 'unmusical' or any intermediate state there may be.

Nor again do things pass into the first chance thing; 'white' does
not pass into 'musical' (except, it may be, in virtue of a concomitant
attribute), but into 'not-white'-and not into any chance thing which
is not white, but into black or an intermediate colour; 'musical'
passes into 'not-musical'-and not into any chance thing other than
musical, but into 'unmusical' or any intermediate state there may
be. 

The same holds of other things also: even things which are not simple
but complex follow the same principle, but the opposite state has
not received a name, so we fail to notice the fact. What is in tune
must come from what is not in tune, and vice versa; the tuned passes
into untunedness-and not into any untunedness, but into the corresponding
opposite. It does not matter whether we take attunement, order, or
composition for our illustration; the principle is obviously the same
in all, and in fact applies equally to the production of a house,
a statue, or any other complex. A house comes from certain things
in a certain state of separation instead of conjunction, a statue
(or any other thing that has been shaped) from shapelessness-each
of these objects being partly order and partly composition.

If then this is true, everything that comes to be or passes away from,
or passes into, its contrary or an intermediate state. But the intermediates
are derived from the contraries-colours, for instance, from black
and white. Everything, therefore, that comes to be by a natural process
is either a contrary or a product of contraries. 

Up to this point we have practically had most of the other writers
on the subject with us, as I have said already: for all of them identify
their elements, and what they call their principles, with the contraries,
giving no reason indeed for the theory, but contrained as it were
by the truth itself. They differ, however, from one another in that
some assume contraries which are more primary, others contraries which
are less so: some those more knowable in the order of explanation,
others those more familiar to sense. For some make hot and cold, or
again moist and dry, the conditions of becoming; while others make
odd and even, or again Love and Strife; and these differ from each
other in the way mentioned. 

Hence their principles are in one sense the same, in another different;
different certainly, as indeed most people think, but the same inasmuch
as they are analogous; for all are taken from the same table of columns,
some of the pairs being wider, others narrower in extent. In this
way then their theories are both the same and different, some better,
some worse; some, as I have said, take as their contraries what is
more knowable in the order of explanation, others what is more familiar
to sense. (The universal is more knowable in the order of explanation,
the particular in the order of sense: for explanation has to do with
the universal, sense with the particular.) 'The great and the small',
for example, belong to the former class, 'the dense and the rare'
to the latter. 

It is clear then that our principles must be contraries.

Part 6

The next question is whether the principles are two or three or more
in number. 

One they cannot be, for there cannot be one contrary. Nor can they
be innumerable, because, if so, Being will not be knowable: and in
any one genus there is only one contrariety, and substance is one
genus: also a finite number is sufficient, and a finite number, such
as the principles of Empedocles, is better than an infinite multitude;
for Empedocles professes to obtain from his principles all that Anaxagoras
obtains from his innumerable principles. Lastly, some contraries are
more primary than others, and some arise from others-for example sweet
and bitter, white and black-whereas the principles must always remain
principles. 

This will suffice to show that the principles are neither one nor
innumerable. 

Granted, then, that they are a limited number, it is plausible to
suppose them more than two. For it is difficult to see how either
density should be of such a nature as to act in any way on rarity
or rarity on density. The same is true of any other pair of contraries;
for Love does not gather Strife together and make things out of it,
nor does Strife make anything out of Love, but both act on a third
thing different from both. Some indeed assume more than one such thing
from which they construct the world of nature. 

Other objections to the view that it is not necessary to assume a
third principle as a substratum may be added. (1) We do not find that
the contraries constitute the substance of any thing. But what is
a first principle ought not to be the predicate of any subject. If
it were, there would be a principle of the supposed principle: for
the subject is a principle, and prior presumably to what is predicated
of it. Again (2) we hold that a substance is not contrary to another
substance. How then can substance be derived from what are not substances?
Or how can non-substances be prior to substance? 

If then we accept both the former argument and this one, we must,
to preserve both, assume a third somewhat as the substratum of the
contraries, such as is spoken of by those who describe the All as
one nature-water or fire or what is intermediate between them. What
is intermediate seems preferable; for fire, earth, air, and water
are already involved with pairs of contraries. There is, therefore,
much to be said for those who make the underlying substance different
from these four; of the rest, the next best choice is air, as presenting
sensible differences in a less degree than the others; and after air,
water. All, however, agree in this, that they differentiate their
One by means of the contraries, such as density and rarity and more
and less, which may of course be generalized, as has already been
said into excess and defect. Indeed this doctrine too (that the One
and excess and defect are the principles of things) would appear to
be of old standing, though in different forms; for the early thinkers
made the two the active and the one the passive principle, whereas
some of the more recent maintain the reverse. 

To suppose then that the elements are three in number would seem,
from these and similar considerations, a plausible view, as I said
before. On the other hand, the view that they are more than three
in number would seem to be untenable. 

For the one substratum is sufficient to be acted on; but if we have
four contraries, there will be two contrarieties, and we shall have
to suppose an intermediate nature for each pair separately. If, on
the other hand, the contrarieties, being two, can generate from each
other, the second contrariety will be superfluous. Moreover, it is
impossible that there should be more than one primary contrariety.
For substance is a single genus of being, so that the principles can
differ only as prior and posterior, not in genus; in a single genus
there is always a single contrariety, all the other contrarieties
in it being held to be reducible to one. 

It is clear then that the number of elements is neither one nor more
than two or three; but whether two or three is, as I said, a question
of considerable difficulty. 

Part 7

We will now give our own account, approaching the question first with
reference to becoming in its widest sense: for we shall be following
the natural order of inquiry if we speak first of common characteristics,
and then investigate the characteristics of special cases.

We say that one thing comes to be from another thing, and one sort
of thing from another sort of thing, both in the case of simple and
of complex things. I mean the following. We can say (1) 'man becomes
musical', (2) what is 'not-musical becomes musical', or (3), the 'not-musical
man becomes a musical man'. Now what becomes in (1) and (2)-'man'
and 'not musical'-I call simple, and what each becomes-'musical'-simple
also. But when (3) we say the 'not-musical man becomes a musical man',
both what becomes and what it becomes are complex. 

As regards one of these simple 'things that become' we say not only
'this becomes so-and-so', but also 'from being this, comes to be so-and-so',
as 'from being not-musical comes to be musical'; as regards the other
we do not say this in all cases, as we do not say (1) 'from being
a man he came to be musical' but only 'the man became musical'.

When a 'simple' thing is said to become something, in one case (1)
it survives through the process, in the other (2) it does not. For
man remains a man and is such even when he becomes musical, whereas
what is not musical or is unmusical does not continue to exist, either
simply or combined with the subject. 

These distinctions drawn, one can gather from surveying the various
cases of becoming in the way we are describing that, as we say, there
must always be an underlying something, namely that which becomes,
and that this, though always one numerically, in form at least is
not one. (By that I mean that it can be described in different ways.)
For 'to be man' is not the same as 'to be unmusical'. One part survives,
the other does not: what is not an opposite survives (for 'man' survives),
but 'not-musical' or 'unmusical' does not survive, nor does the compound
of the two, namely 'unmusical man'. 

We speak of 'becoming that from this' instead of 'this becoming that'
more in the case of what does not survive the change-'becoming musical
from unmusical', not 'from man'-but there are exceptions, as we sometimes
use the latter form of expression even of what survives; we speak
of 'a statue coming to be from bronze', not of the 'bronze becoming
a statue'. The change, however, from an opposite which does not survive
is described indifferently in both ways, 'becoming that from this'
or 'this becoming that'. We say both that 'the unmusical becomes musical',
and that 'from unmusical he becomes musical'. And so both forms are
used of the complex, 'becoming a musical man from an unmusical man',
and unmusical man becoming a musical man'. 

But there are different senses of 'coming to be'. In some cases we
do not use the expression 'come to be', but 'come to be so-and-so'.
Only substances are said to 'come to be' in the unqualified sense.

Now in all cases other than substance it is plain that there must
be some subject, namely, that which becomes. For we know that when
a thing comes to be of such a quantity or quality or in such a relation,
time, or place, a subject is always presupposed, since substance alone
is not predicated of another subject, but everything else of substance.

But that substances too, and anything else that can be said 'to be'
without qualification, come to be from some substratum, will appear
on examination. For we find in every case something that underlies
from which proceeds that which comes to be; for instance, animals
and plants from seed. 

Generally things which come to be, come to be in different ways: (1)
by change of shape, as a statue; (2) by addition, as things which
grow; (3) by taking away, as the Hermes from the stone; (4) by putting
together, as a house; (5) by alteration, as things which 'turn' in
respect of their material substance. 

It is plain that these are all cases of coming to be from a substratum.

Thus, clearly, from what has been said, whatever comes to be is always
complex. There is, on the one hand, (a) something which comes into
existence, and again (b) something which becomes that-the latter (b)
in two senses, either the subject or the opposite. By the 'opposite'
I mean the 'unmusical', by the 'subject' 'man', and similarly I call
the absence of shape or form or order the 'opposite', and the bronze
or stone or gold the 'subject'. 

Plainly then, if there are conditions and principles which constitute
natural objects and from which they primarily are or have come to
be-have come to be, I mean, what each is said to be in its essential
nature, not what each is in respect of a concomitant attribute-plainly,
I say, everything comes to be from both subject and form. For 'musical
man' is composed (in a way) of 'man' and 'musical': you can analyse
it into the definitions of its elements. It is clear then that what
comes to be will come to be from these elements. 

Now the subject is one numerically, though it is two in form. (For
it is the man, the gold-the 'matter' generally-that is counted, for
it is more of the nature of a 'this', and what comes to be does not
come from it in virtue of a concomitant attribute; the privation,
on the other hand, and the contrary are incidental in the process.)
And the positive form is one-the order, the acquired art of music,
or any similar predicate. 

There is a sense, therefore, in which we must declare the principles
to be two, and a sense in which they are three; a sense in which the
contraries are the principles-say for example the musical and the
unmusical, the hot and the cold, the tuned and the untuned-and a sense
in which they are not, since it is impossible for the contraries to
be acted on by each other. But this difficulty also is solved by the
fact that the substratum is different from the contraries, for it
is itself not a contrary. The principles therefore are, in a way,
not more in number than the contraries, but as it were two, nor yet
precisely two, since there is a difference of essential nature, but
three. For 'to be man' is different from 'to be unmusical', and 'to
be unformed' from 'to be bronze'. 

We have now stated the number of the principles of natural objects
which are subject to generation, and how the number is reached: and
it is clear that there must be a substratum for the contraries, and
that the contraries must be two. (Yet in another way of putting it
this is not necessary, as one of the contraries will serve to effect
the change by its successive absence and presence.) 

The underlying nature is an object of scientific knowledge, by an
analogy. For as the bronze is to the statue, the wood to the bed,
or the matter and the formless before receiving form to any thing
which has form, so is the underlying nature to substance, i.e. the
'this' or existent. 

This then is one principle (though not one or existent in the same
sense as the 'this'), and the definition was one as we agreed; then
further there is its contrary, the privation. In what sense these
are two, and in what sense more, has been stated above. Briefly, we
explained first that only the contraries were principles, and later
that a substratum was indispensable, and that the principles were
three; our last statement has elucidated the difference between the
contraries, the mutual relation of the principles, and the nature
of the substratum. Whether the form or the substratum is the essential
nature of a physical object is not yet clear. But that the principles
are three, and in what sense, and the way in which each is a principle,
is clear. 

So much then for the question of the number and the nature of the
principles. 

Part 8

We will now proceed to show that the difficulty of the early thinkers,
as well as our own, is solved in this way alone. 

The first of those who studied science were misled in their search
for truth and the nature of things by their inexperience, which as
it were thrust them into another path. So they say that none of the
things that are either comes to be or passes out of existence, because
what comes to be must do so either from what is or from what is not,
both of which are impossible. For what is cannot come to be (because
it is already), and from what is not nothing could have come to be
(because something must be present as a substratum). So too they exaggerated
the consequence of this, and went so far as to deny even the existence
of a plurality of things, maintaining that only Being itself is. Such
then was their opinion, and such the reason for its adoption.

Our explanation on the other hand is that the phrases 'something comes
to be from what is or from what is not', 'what is not or what is does
something or has something done to it or becomes some particular thing',
are to be taken (in the first way of putting our explanation) in the
same sense as 'a doctor does something or has something done to him',
'is or becomes something from being a doctor.' These expressions may
be taken in two senses, and so too, clearly, may 'from being', and
'being acts or is acted on'. A doctor builds a house, not qua doctor,
but qua housebuilder, and turns gray, not qua doctor, but qua dark-haired.
On the other hand he doctors or fails to doctor qua doctor. But we
are using words most appropriately when we say that a doctor does
something or undergoes something, or becomes something from being
a doctor, if he does, undergoes, or becomes qua doctor. Clearly then
also 'to come to be so-and-so from not-being' means 'qua not-being'.

It was through failure to make this distinction that those thinkers
gave the matter up, and through this error that they went so much
farther astray as to suppose that nothing else comes to be or exists
apart from Being itself, thus doing away with all becoming.

We ourselves are in agreement with them in holding that nothing can
be said without qualification to come from what is not. But nevertheless
we maintain that a thing may 'come to be from what is not'-that is,
in a qualified sense. For a thing comes to be from the privation,
which in its own nature is not-being,-this not surviving as a constituent
of the result. Yet this causes surprise, and it is thought impossible
that something should come to be in the way described from what is
not. 

In the same way we maintain that nothing comes to be from being, and
that being does not come to be except in a qualified sense. In that
way, however, it does, just as animal might come to be from animal,
and an animal of a certain kind from an animal of a certain kind.
Thus, suppose a dog to come to be from a horse. The dog would then,
it is true, come to be from animal (as well as from an animal of a
certain kind) but not as animal, for that is already there. But if
anything is to become an animal, not in a qualified sense, it will
not be from animal: and if being, not from being-nor from not-being
either, for it has been explained that by 'from not being' we mean
from not-being qua not-being. 

Note further that we do not subvert the principle that everything
either is or is not. 

This then is one way of solving the difficulty. Another consists in
pointing out that the same things can be explained in terms of potentiality
and actuality. But this has been done with greater precision elsewhere.
So, as we said, the difficulties which constrain people to deny the
existence of some of the things we mentioned are now solved. For it
was this reason which also caused some of the earlier thinkers to
turn so far aside from the road which leads to coming to be and passing
away and change generally. If they had come in sight of this nature,
all their ignorance would have been dispelled. 

Part 9

Others, indeed, have apprehended the nature in question, but not adequately.

In the first place they allow that a thing may come to be without
qualification from not being, accepting on this point the statement
of Parmenides. Secondly, they think that if the substratum is one
numerically, it must have also only a single potentiality-which is
a very different thing. 

Now we distinguish matter and privation, and hold that one of these,
namely the matter, is not-being only in virtue of an attribute which
it has, while the privation in its own nature is not-being; and that
the matter is nearly, in a sense is, substance, while the privation
in no sense is. They, on the other hand, identify their Great and
Small alike with not being, and that whether they are taken together
as one or separately. Their triad is therefore of quite a different
kind from ours. For they got so far as to see that there must be some
underlying nature, but they make it one-for even if one philosopher
makes a dyad of it, which he calls Great and Small, the effect is
the same, for he overlooked the other nature. For the one which persists
is a joint cause, with the form, of what comes to be-a mother, as
it were. But the negative part of the contrariety may often seem,
if you concentrate your attention on it as an evil agent, not to exist
at all. 

For admitting with them that there is something divine, good, and
desirable, we hold that there are two other principles, the one contrary
to it, the other such as of its own nature to desire and yearn for
it. But the consequence of their view is that the contrary desires
its wtextinction. Yet the form cannot desire itself, for it is not
defective; nor can the contrary desire it, for contraries are mutually
destructive. The truth is that what desires the form is matter, as
the female desires the male and the ugly the beautiful-only the ugly
or the female not per se but per accidens. 

The matter comes to be and ceases to be in one sense, while in another
it does not. As that which contains the privation, it ceases to be
in its own nature, for what ceases to be-the privation-is contained
within it. But as potentiality it does not cease to be in its own
nature, but is necessarily outside the sphere of becoming and ceasing
to be. For if it came to be, something must have existed as a primary
substratum from which it should come and which should persist in it;
but this is its own special nature, so that it will be before coming
to be. (For my definition of matter is just this-the primary substratum
of each thing, from which it comes to be without qualification, and
which persists in the result.) And if it ceases to be it will pass
into that at the last, so it will have ceased to be before ceasing
to be. 

The accurate determination of the first principle in respect of form,
whether it is one or many and what it is or what they are, is the
province of the primary type of science; so these questions may stand
over till then. But of the natural, i.e. perishable, forms we shall
speak in the expositions which follow. 

The above, then, may be taken as sufficient to establish that there
are principles and what they are and how many there are. Now let us
make a fresh start and proceed. 

----------------------------------------------------------------------

BOOK II

Part 1 

Of things that exist, some exist by nature, some from other causes.

'By nature' the animals and their parts exist, and the plants and
the simple bodies (earth, fire, air, water)-for we say that these
and the like exist 'by nature'. 

All the things mentioned present a feature in which they differ from
things which are not constituted by nature. Each of them has within
itself a principle of motion and of stationariness (in respect of
place, or of growth and decrease, or by way of alteration). On the
other hand, a bed and a coat and anything else of that sort, qua receiving
these designations i.e. in so far as they are products of art-have
no innate impulse to change. But in so far as they happen to be composed
of stone or of earth or of a mixture of the two, they do have such
an impulse, and just to that extent which seems to indicate that nature
is a source or cause of being moved and of being at rest in that to
which it belongs primarily, in virtue of itself and not in virtue
of a concomitant attribute. 

I say 'not in virtue of a concomitant attribute', because (for instance)
a man who is a doctor might cure himself. Nevertheless it is not in
so far as he is a patient that he possesses the art of medicine: it
merely has happened that the same man is doctor and patient-and that
is why these attributes are not always found together. So it is with
all other artificial products. None of them has in itself the source
of its own production. But while in some cases (for instance houses
and the other products of manual labour) that principle is in something
else external to the thing, in others those which may cause a change
in themselves in virtue of a concomitant attribute-it lies in the
things themselves (but not in virtue of what they are). 

'Nature' then is what has been stated. Things 'have a nature'which
have a principle of this kind. Each of them is a substance; for it
is a subject, and nature always implies a subject in which it inheres.

The term 'according to nature' is applied to all these things and
also to the attributes which belong to them in virtue of what they
are, for instance the property of fire to be carried upwards-which
is not a 'nature' nor 'has a nature' but is 'by nature' or 'according
to nature'. 

What nature is, then, and the meaning of the terms 'by nature' and
'according to nature', has been stated. That nature exists, it would
be absurd to try to prove; for it is obvious that there are many things
of this kind, and to prove what is obvious by what is not is the mark
of a man who is unable to distinguish what is self-evident from what
is not. (This state of mind is clearly possible. A man blind from
birth might reason about colours. Presumably therefore such persons
must be talking about words without any thought to correspond.)

Some identify the nature or substance of a natural object with that
immediate constituent of it which taken by itself is without arrangement,
e.g. the wood is the 'nature' of the bed, and the bronze the 'nature'
of the statue. 

As an indication of this Antiphon points out that if you planted a
bed and the rotting wood acquired the power of sending up a shoot,
it would not be a bed that would come up, but wood-which shows that
the arrangement in accordance with the rules of the art is merely
an incidental attribute, whereas the real nature is the other, which,
further, persists continuously through the process of making.

But if the material of each of these objects has itself the same relation
to something else, say bronze (or gold) to water, bones (or wood)
to earth and so on, that (they say) would be their nature and essence.
Consequently some assert earth, others fire or air or water or some
or all of these, to be the nature of the things that are. For whatever
any one of them supposed to have this character-whether one thing
or more than one thing-this or these he declared to be the whole of
substance, all else being its affections, states, or dispositions.
Every such thing they held to be eternal (for it could not pass into
anything else), but other things to come into being and cease to be
times without number. 

This then is one account of 'nature', namely that it is the immediate
material substratum of things which have in themselves a principle
of motion or change. 

Another account is that 'nature' is the shape or form which is specified
in the definition of the thing. 

For the word 'nature' is applied to what is according to nature and
the natural in the same way as 'art' is applied to what is artistic
or a work of art. We should not say in the latter case that there
is anything artistic about a thing, if it is a bed only potentially,
not yet having the form of a bed; nor should we call it a work of
art. The same is true of natural compounds. What is potentially flesh
or bone has not yet its own 'nature', and does not exist until it
receives the form specified in the definition, which we name in defining
what flesh or bone is. Thus in the second sense of 'nature' it would
be the shape or form (not separable except in statement) of things
which have in themselves a source of motion. (The combination of the
two, e.g. man, is not 'nature' but 'by nature' or 'natural'.)

The form indeed is 'nature' rather than the matter; for a thing is
more properly said to be what it is when it has attained to fulfilment
than when it exists potentially. Again man is born from man, but not
bed from bed. That is why people say that the figure is not the nature
of a bed, but the wood is-if the bed sprouted not a bed but wood would
come up. But even if the figure is art, then on the same principle
the shape of man is his nature. For man is born from man.

We also speak of a thing's nature as being exhibited in the process
of growth by which its nature is attained. The 'nature' in this sense
is not like 'doctoring', which leads not to the art of doctoring but
to health. Doctoring must start from the art, not lead to it. But
it is not in this way that nature (in the one sense) is related to
nature (in the other). What grows qua growing grows from something
into something. Into what then does it grow? Not into that from which
it arose but into that to which it tends. The shape then is nature.

'Shape' and 'nature', it should be added, are in two senses. For the
privation too is in a way form. But whether in unqualified coming
to be there is privation, i.e. a contrary to what comes to be, we
must consider later. 

Part 2

We have distinguished, then, the different ways in which the term
'nature' is used. 

The next point to consider is how the mathematician differs from the
physicist. Obviously physical bodies contain surfaces and volumes,
lines and points, and these are the subject-matter of mathematics.

Further, is astronomy different from physics or a department of it?
It seems absurd that the physicist should be supposed to know the
nature of sun or moon, but not to know any of their essential attributes,
particularly as the writers on physics obviously do discuss their
shape also and whether the earth and the world are spherical or not.

Now the mathematician, though he too treats of these things, nevertheless
does not treat of them as the limits of a physical body; nor does
he consider the attributes indicated as the attributes of such bodies.
That is why he separates them; for in thought they are separable from
motion, and it makes no difference, nor does any falsity result, if
they are separated. The holders of the theory of Forms do the same,
though they are not aware of it; for they separate the objects of
physics, which are less separable than those of mathematics. This
becomes plain if one tries to state in each of the two cases the definitions
of the things and of their attributes. 'Odd' and 'even', 'straight'
and 'curved', and likewise 'number', 'line', and 'figure', do not
involve motion; not so 'flesh' and 'bone' and 'man'-these are defined
like 'snub nose', not like 'curved'. 

Similar evidence is supplied by the more physical of the branches
of mathematics, such as optics, harmonics, and astronomy. These are
in a way the converse of geometry. While geometry investigates physical
lines but not qua physical, optics investigates mathematical lines,
but qua physical, not qua mathematical. 

Since 'nature' has two senses, the form and the matter, we must investigate
its objects as we would the essence of snubness. That is, such things
are neither independent of matter nor can be defined in terms of matter
only. Here too indeed one might raise a difficulty. Since there are
two natures, with which is the physicist concerned? Or should he investigate
the combination of the two? But if the combination of the two, then
also each severally. Does it belong then to the same or to different
sciences to know each severally? 

If we look at the ancients, physics would to be concerned with the
matter. (It was only very slightly that Empedocles and Democritus
touched on the forms and the essence.) 

But if on the other hand art imitates nature, and it is the part of
the same discipline to know the form and the matter up to a point
(e.g. the doctor has a knowledge of health and also of bile and phlegm,
in which health is realized, and the builder both of the form of the
house and of the matter, namely that it is bricks and beams, and so
forth): if this is so, it would be the part of physics also to know
nature in both its senses. 

Again, 'that for the sake of which', or the end, belongs to the same
department of knowledge as the means. But the nature is the end or
'that for the sake of which'. For if a thing undergoes a continuous
change and there is a stage which is last, this stage is the end or
'that for the sake of which'. (That is why the poet was carried away
into making an absurd statement when he said 'he has the end for the
sake of which he was born'. For not every stage that is last claims
to be an end, but only that which is best.) 

For the arts make their material (some simply 'make' it, others make
it serviceable), and we use everything as if it was there for our
sake. (We also are in a sense an end. 'That for the sake of which'
has two senses: the distinction is made in our work On Philosophy.)
The arts, therefore, which govern the matter and have knowledge are
two, namely the art which uses the product and the art which directs
the production of it. That is why the using art also is in a sense
directive; but it differs in that it knows the form, whereas the art
which is directive as being concerned with production knows the matter.
For the helmsman knows and prescribes what sort of form a helm should
have, the other from what wood it should be made and by means of what
operations. In the products of art, however, we make the material
with a view to the function, whereas in the products of nature the
matter is there all along. 

Again, matter is a relative term: to each form there corresponds a
special matter. How far then must the physicist know the form or essence?
Up to a point, perhaps, as the doctor must know sinew or the smith
bronze (i.e. until he understands the purpose of each): and the physicist
is concerned only with things whose forms are separable indeed, but
do not exist apart from matter. Man is begotten by man and by the
sun as well. The mode of existence and essence of the separable it
is the business of the primary type of philosophy to define.

Part 3

Now that we have established these distinctions, we must proceed to
consider causes, their character and number. Knowledge is the object
of our inquiry, and men do not think they know a thing till they have
grasped the 'why' of (which is to grasp its primary cause). So clearly
we too must do this as regards both coming to be and passing away
and every kind of physical change, in order that, knowing their principles,
we may try to refer to these principles each of our problems.

In one sense, then, (1) that out of which a thing comes to be and
which persists, is called 'cause', e.g. the bronze of the statue,
the silver of the bowl, and the genera of which the bronze and the
silver are species. 

In another sense (2) the form or the archetype, i.e. the statement
of the essence, and its genera, are called 'causes' (e.g. of the octave
the relation of 2:1, and generally number), and the parts in the definition.

Again (3) the primary source of the change or coming to rest; e.g.
the man who gave advice is a cause, the father is cause of the child,
and generally what makes of what is made and what causes change of
what is changed. 

Again (4) in the sense of end or 'that for the sake of which' a thing
is done, e.g. health is the cause of walking about. ('Why is he walking
about?' we say. 'To be healthy', and, having said that, we think we
have assigned the cause.) The same is true also of all the intermediate
steps which are brought about through the action of something else
as means towards the end, e.g. reduction of flesh, purging, drugs,
or surgical instruments are means towards health. All these things
are 'for the sake of' the end, though they differ from one another
in that some are activities, others instruments. 

This then perhaps exhausts the number of ways in which the term 'cause'
is used. 

As the word has several senses, it follows that there are several
causes of the same thing not merely in virtue of a concomitant attribute),
e.g. both the art of the sculptor and the bronze are causes of the
statue. These are causes of the statue qua statue, not in virtue of
anything else that it may be-only not in the same way, the one being
the material cause, the other the cause whence the motion comes. Some
things cause each other reciprocally, e.g. hard work causes fitness
and vice versa, but again not in the same way, but the one as end,
the other as the origin of change. Further the same thing is the cause
of contrary results. For that which by its presence brings about one
result is sometimes blamed for bringing about the contrary by its
absence. Thus we ascribe the wreck of a ship to the absence of the
pilot whose presence was the cause of its safety. 

All the causes now mentioned fall into four familiar divisions. The
letters are the causes of syllables, the material of artificial products,
fire, &c., of bodies, the parts of the whole, and the premisses of
the conclusion, in the sense of 'that from which'. Of these pairs
the one set are causes in the sense of substratum, e.g. the parts,
the other set in the sense of essence-the whole and the combination
and the form. But the seed and the doctor and the adviser, and generally
the maker, are all sources whence the change or stationariness originates,
while the others are causes in the sense of the end or the good of
the rest; for 'that for the sake of which' means what is best and
the end of the things that lead up to it. (Whether we say the 'good
itself or the 'apparent good' makes no difference.) 

Such then is the number and nature of the kinds of cause.

Now the modes of causation are many, though when brought under heads
they too can be reduced in number. For 'cause' is used in many senses
and even within the same kind one may be prior to another (e.g. the
doctor and the expert are causes of health, the relation 2:1 and number
of the octave), and always what is inclusive to what is particular.
Another mode of causation is the incidental and its genera, e.g. in
one way 'Polyclitus', in another 'sculptor' is the cause of a statue,
because 'being Polyclitus' and 'sculptor' are incidentally conjoined.
Also the classes in which the incidental attribute is included; thus
'a man' could be said to be the cause of a statue or, generally, 'a
living creature'. An incidental attribute too may be more or less
remote, e.g. suppose that 'a pale man' or 'a musical man' were said
to be the cause of the statue. 

All causes, both proper and incidental, may be spoken of either as
potential or as actual; e.g. the cause of a house being built is either
'house-builder' or 'house-builder building'. 

Similar distinctions can be made in the things of which the causes
are causes, e.g. of 'this statue' or of 'statue' or of 'image' generally,
of 'this bronze' or of 'bronze' or of 'material' generally. So too
with the incidental attributes. Again we may use a complex expression
for either and say, e.g. neither 'Polyclitus' nor 'sculptor' but 'Polyclitus,
sculptor'. 

All these various uses, however, come to six in number, under each
of which again the usage is twofold. Cause means either what is particular
or a genus, or an incidental attribute or a genus of that, and these
either as a complex or each by itself; and all six either as actual
or as potential. The difference is this much, that causes which are
actually at work and particular exist and cease to exist simultaneously
with their effect, e.g. this healing person with this being-healed
person and that house-building man with that being-built house; but
this is not always true of potential causes--the house and the housebuilder
do not pass away simultaneously. 

In investigating the cause of each thing it is always necessary to
seek what is most precise (as also in other things): thus man builds
because he is a builder, and a builder builds in virtue of his art
of building. This last cause then is prior: and so generally.

Further, generic effects should be assigned to generic causes, particular
effects to particular causes, e.g. statue to sculptor, this statue
to this sculptor; and powers are relative to possible effects, actually
operating causes to things which are actually being effected.

This must suffice for our account of the number of causes and the
modes of causation. 

Part 4

But chance also and spontaneity are reckoned among causes: many things
are said both to be and to come to be as a result of chance and spontaneity.
We must inquire therefore in what manner chance and spontaneity are
present among the causes enumerated, and whether they are the same
or different, and generally what chance and spontaneity are.

Some people even question whether they are real or not. They say that
nothing happens by chance, but that everything which we ascribe to
chance or spontaneity has some definite cause, e.g. coming 'by chance'
into the market and finding there a man whom one wanted but did not
expect to meet is due to one's wish to go and buy in the market. Similarly
in other cases of chance it is always possible, they maintain, to
find something which is the cause; but not chance, for if chance were
real, it would seem strange indeed, and the question might be raised,
why on earth none of the wise men of old in speaking of the causes
of generation and decay took account of chance; whence it would seem
that they too did not believe that anything is by chance. But there
is a further circumstance that is surprising. Many things both come
to be and are by chance and spontaneity, and although know that each
of them can be ascribed to some cause (as the old argument said which
denied chance), nevertheless they speak of some of these things as
happening by chance and others not. For this reason also they ought
to have at least referred to the matter in some way or other.

Certainly the early physicists found no place for chance among the
causes which they recognized-love, strife, mind, fire, or the like.
This is strange, whether they supposed that there is no such thing
as chance or whether they thought there is but omitted to mention
it-and that too when they sometimes used it, as Empedocles does when
he says that the air is not always separated into the highest region,
but 'as it may chance'. At any rate he says in his cosmogony that
'it happened to run that way at that time, but it often ran otherwise.'
He tells us also that most of the parts of animals came to be by chance.

There are some too who ascribe this heavenly sphere and all the worlds
to spontaneity. They say that the vortex arose spontaneously, i.e.
the motion that separated and arranged in its present order all that
exists. This statement might well cause surprise. For they are asserting
that chance is not responsible for the existence or generation of
animals and plants, nature or mind or something of the kind being
the cause of them (for it is not any chance thing that comes from
a given seed but an olive from one kind and a man from another); and
yet at the same time they assert that the heavenly sphere and the
divinest of visible things arose spontaneously, having no such cause
as is assigned to animals and plants. Yet if this is so, it is a fact
which deserves to be dwelt upon, and something might well have been
said about it. For besides the other absurdities of the statement,
it is the more absurd that people should make it when they see nothing
coming to be spontaneously in the heavens, but much happening by chance
among the things which as they say are not due to chance; whereas
we should have expected exactly the opposite. 

Others there are who, indeed, believe that chance is a cause, but
that it is inscrutable to human intelligence, as being a divine thing
and full of mystery. 

Thus we must inquire what chance and spontaneity are, whether they
are the same or different, and how they fit into our division of causes.

Part 5

First then we observe that some things always come to pass in the
same way, and others for the most part. It is clearly of neither of
these that chance is said to be the cause, nor can the 'effect of
chance' be identified with any of the things that come to pass by
necessity and always, or for the most part. But as there is a third
class of events besides these two-events which all say are 'by chance'-it
is plain that there is such a thing as chance and spontaneity; for
we know that things of this kind are due to chance and that things
due to chance are of this kind. 

But, secondly, some events are for the sake of something, others not.
Again, some of the former class are in accordance with deliberate
intention, others not, but both are in the class of things which are
for the sake of something. Hence it is clear that even among the things
which are outside the necessary and the normal, there are some in
connexion withwhich the phrase 'for the sake of something' is applicable.
(Events that are for the sake of something include whatever may be
done as a result of thought or of nature.) Things of this kind, then,
when they come to pass incidental are said to be 'by chance'. For
just as a thing is something either in virtue of itself or incidentally,
so may it be a cause. For instance, the housebuilding faculty is in
virtue of itself the cause of a house, whereas the pale or the musical
is the incidental cause. That which is per se cause of the effect
is determinate, but the incidental cause is indeterminable, for the
possible attributes of an individual are innumerable. To resume then;
when a thing of this kind comes to pass among events which are for
the sake of something, it is said to be spontaneous or by chance.
(The distinction between the two must be made later-for the present
it is sufficient if it is plain that both are in the sphere of things
done for the sake of something.) 

Example: A man is engaged in collecting subscriptions for a feast.
He would have gone to such and such a place for the purpose of getting
the money, if he had known. He actually went there for another purpose
and it was only incidentally that he got his money by going there;
and this was not due to the fact that he went there as a rule or necessarily,
nor is the end effected (getting the money) a cause present in himself-it
belongs to the class of things that are intentional and the result
of intelligent deliberation. It is when these conditions are satisfied
that the man is said to have gone 'by chance'. If he had gone of deliberate
purpose and for the sake of this-if he always or normally went there
when he was collecting payments-he would not be said to have gone
'by chance'. 

It is clear then that chance is an incidental cause in the sphere
of those actions for the sake of something which involve purpose.
Intelligent reflection, then, and chance are in the same sphere, for
purpose implies intelligent reflection. 

It is necessary, no doubt, that the causes of what comes to pass by
chance be indefinite; and that is why chance is supposed to belong
to the class of the indefinite and to be inscrutable to man, and why
it might be thought that, in a way, nothing occurs by chance. For
all these statements are correct, because they are well grounded.
Things do, in a way, occur by chance, for they occur incidentally
and chance is an incidental cause. But strictly it is not the cause-without
qualification-of anything; for instance, a housebuilder is the cause
of a house; incidentally, a fluteplayer may be so. 

And the causes of the man's coming and getting the money (when he
did not come for the sake of that) are innumerable. He may have wished
to see somebody or been following somebody or avoiding somebody, or
may have gone to see a spectacle. Thus to say that chance is a thing
contrary to rule is correct. For 'rule' applies to what is always
true or true for the most part, whereas chance belongs to a third
type of event. Hence, to conclude, since causes of this kind are indefinite,
chance too is indefinite. (Yet in some cases one might raise the question
whether any incidental fact might be the cause of the chance occurrence,
e.g. of health the fresh air or the sun's heat may be the cause, but
having had one's hair cut cannot; for some incidental causes are more
relevant to the effect than others.) 

Chance or fortune is called 'good' when the result is good, 'evil'
when it is evil. The terms 'good fortune' and 'ill fortune' are used
when either result is of considerable magnitude. Thus one who comes
within an ace of some great evil or great good is said to be fortunate
or unfortunate. The mind affirms the essence of the attribute, ignoring
the hair's breadth of difference. Further, it is with reason that
good fortune is regarded as unstable; for chance is unstable, as none
of the things which result from it can be invariable or normal.

Both are then, as I have said, incidental causes-both chance and spontaneity-in
the sphere of things which are capable of coming to pass not necessarily,
nor normally, and with reference to such of these as might come to
pass for the sake of something. 

Part 6

They differ in that 'spontaneity' is the wider term. Every result
of chance is from what is spontaneous, but not everything that is
from what is spontaneous is from chance. 

Chance and what results from chance are appropriate to agents that
are capable of good fortune and of moral action generally. Therefore
necessarily chance is in the sphere of moral actions. This is indicated
by the fact that good fortune is thought to be the same, or nearly
the same, as happiness, and happiness to be a kind of moral action,
since it is well-doing. Hence what is not capable of moral action
cannot do anything by chance. Thus an inanimate thing or a lower animal
or a child cannot do anything by chance, because it is incapable of
deliberate intention; nor can 'good fortune' or 'ill fortune' be ascribed
to them, except metaphorically, as Protarchus, for example, said that
the stones of which altars are made are fortunate because they are
held in honour, while their fellows are trodden under foot. Even these
things, however, can in a way be affected by chance, when one who
is dealing with them does something to them by chance, but not otherwise.

The spontaneous on the other hand is found both in the lower animals
and in many inanimate objects. We say, for example, that the horse
came 'spontaneously', because, though his coming saved him, he did
not come for the sake of safety. Again, the tripod fell 'of itself',
because, though when it fell it stood on its feet so as to serve for
a seat, it did not fall for the sake of that. 

Hence it is clear that events which (1) belong to the general class
of things that may come to pass for the sake of something, (2) do
not come to pass for the sake of what actually results, and (3) have
an external cause, may be described by the phrase 'from spontaneity'.
These 'spontaneous' events are said to be 'from chance' if they have
the further characteristics of being the objects of deliberate intention
and due to agents capable of that mode of action. This is indicated
by the phrase 'in vain', which is used when A which is for the sake
of B, does not result in B. For instance, taking a walk is for the
sake of evacuation of the bowels; if this does not follow after walking,
we say that we have walked 'in vain' and that the walking was 'vain'.
This implies that what is naturally the means to an end is 'in vain',
when it does not effect the end towards which it was the natural means-for
it would be absurd for a man to say that he had bathed in vain because
the sun was not eclipsed, since the one was not done with a view to
the other. Thus the spontaneous is even according to its derivation
the case in which the thing itself happens in vain. The stone that
struck the man did not fall for the purpose of striking him; therefore
it fell spontaneously, because it might have fallen by the action
of an agent and for the purpose of striking. The difference between
spontaneity and what results by chance is greatest in things that
come to be by nature; for when anything comes to be contrary to nature,
we do not say that it came to be by chance, but by spontaneity. Yet
strictly this too is different from the spontaneous proper; for the
cause of the latter is external, that of the former internal.

We have now explained what chance is and what spontaneity is, and
in what they differ from each other. Both belong to the mode of causation
'source of change', for either some natural or some intelligent agent
is always the cause; but in this sort of causation the number of possible
causes is infinite. 

Spontaneity and chance are causes of effects which though they might
result from intelligence or nature, have in fact been caused by something
incidentally. Now since nothing which is incidental is prior to what
is per se, it is clear that no incidental cause can be prior to a
cause per se. Spontaneity and chance, therefore, are posterior to
intelligence and nature. Hence, however true it may be that the heavens
are due to spontaneity, it will still be true that intelligence and
nature will be prior causes of this All and of many things in it besides.

Part 7

It is clear then that there are causes, and that the number of them
is what we have stated. The number is the same as that of the things
comprehended under the question 'why'. The 'why' is referred ultimately
either (1), in things which do not involve motion, e.g. in mathematics,
to the 'what' (to the definition of 'straight line' or 'commensurable',
&c.), or (2) to what initiated a motion, e.g. 'why did they go to
war?-because there had been a raid'; or (3) we are inquiring 'for
the sake of what?'-'that they may rule'; or (4), in the case of things
that come into being, we are looking for the matter. The causes, therefore,
are these and so many in number. 

Now, the causes being four, it is the business of the physicist to
know about them all, and if he refers his problems back to all of
them, he will assign the 'why' in the way proper to his science-the
matter, the form, the mover, 'that for the sake of which'. The last
three often coincide; for the 'what' and 'that for the sake of which'
are one, while the primary source of motion is the same in species
as these (for man generates man), and so too, in general, are all
things which cause movement by being themselves moved; and such as
are not of this kind are no longer inside the province of physics,
for they cause motion not by possessing motion or a source of motion
in themselves, but being themselves incapable of motion. Hence there
are three branches of study, one of things which are incapable of
motion, the second of things in motion, but indestructible, the third
of destructible things. 

The question 'why', then, is answered by reference to the matter,
to the form, and to the primary moving cause. For in respect of coming
to be it is mostly in this last way that causes are investigated-'what
comes to be after what? what was the primary agent or patient?' and
so at each step of the series. 

Now the principles which cause motion in a physical way are two, of
which one is not physical, as it has no principle of motion in itself.
Of this kind is whatever causes movement, not being itself moved,
such as (1) that which is completely unchangeable, the primary reality,
and (2) the essence of that which is coming to be, i.e. the form;
for this is the end or 'that for the sake of which'. Hence since nature
is for the sake of something, we must know this cause also. We must
explain the 'why' in all the senses of the term, namely, (1) that
from this that will necessarily result ('from this' either without
qualification or in most cases); (2) that 'this must be so if that
is to be so' (as the conclusion presupposes the premisses); (3) that
this was the essence of the thing; and (4) because it is better thus
(not without qualification, but with reference to the essential nature
in each case). 

Part 8

We must explain then (1) that Nature belongs to the class of causes
which act for the sake of something; (2) about the necessary and its
place in physical problems, for all writers ascribe things to this
cause, arguing that since the hot and the cold, &c., are of such and
such a kind, therefore certain things necessarily are and come to
be-and if they mention any other cause (one his 'friendship and strife',
another his 'mind'), it is only to touch on it, and then good-bye
to it. 

A difficulty presents itself: why should not nature work, not for
the sake of something, nor because it is better so, but just as the
sky rains, not in order to make the corn grow, but of necessity? What
is drawn up must cool, and what has been cooled must become water
and descend, the result of this being that the corn grows. Similarly
if a man's crop is spoiled on the threshing-floor, the rain did not
fall for the sake of this-in order that the crop might be spoiled-but
that result just followed. Why then should it not be the same with
the parts in nature, e.g. that our teeth should come up of necessity-the
front teeth sharp, fitted for tearing, the molars broad and useful
for grinding down the food-since they did not arise for this end,
but it was merely a coincident result; and so with all other parts
in which we suppose that there is purpose? Wherever then all the parts
came about just what they would have been if they had come be for
an end, such things survived, being organized spontaneously in a fitting
way; whereas those which grew otherwise perished and continue to perish,
as Empedocles says his 'man-faced ox-progeny' did. 

Such are the arguments (and others of the kind) which may cause difficulty
on this point. Yet it is impossible that this should be the true view.
For teeth and all other natural things either invariably or normally
come about in a given way; but of not one of the results of chance
or spontaneity is this true. We do not ascribe to chance or mere coincidence
the frequency of rain in winter, but frequent rain in summer we do;
nor heat in the dog-days, but only if we have it in winter. If then,
it is agreed that things are either the result of coincidence or for
an end, and these cannot be the result of coincidence or spontaneity,
it follows that they must be for an end; and that such things are
all due to nature even the champions of the theory which is before
us would agree. Therefore action for an end is present in things which
come to be and are by nature. 

Further, where a series has a completion, all the preceding steps
are for the sake of that. Now surely as in intelligent action, so
in nature; and as in nature, so it is in each action, if nothing interferes.
Now intelligent action is for the sake of an end; therefore the nature
of things also is so. Thus if a house, e.g. had been a thing made
by nature, it would have been made in the same way as it is now by
art; and if things made by nature were made also by art, they would
come to be in the same way as by nature. Each step then in the series
is for the sake of the next; and generally art partly completes what
nature cannot bring to a finish, and partly imitates her. If, therefore,
artificial products are for the sake of an end, so clearly also are
natural products. The relation of the later to the earlier terms of
the series is the same in both. This is most obvious in the animals
other than man: they make things neither by art nor after inquiry
or deliberation. Wherefore people discuss whether it is by intelligence
or by some other faculty that these creatures work,spiders, ants,
and the like. By gradual advance in this direction we come to see
clearly that in plants too that is produced which is conducive to
the end-leaves, e.g. grow to provide shade for the fruit. If then
it is both by nature and for an end that the swallow makes its nest
and the spider its web, and plants grow leaves for the sake of the
fruit and send their roots down (not up) for the sake of nourishment,
it is plain that this kind of cause is operative in things which come
to be and are by nature. And since 'nature' means two things, the
matter and the form, of which the latter is the end, and since all
the rest is for the sake of the end, the form must be the cause in
the sense of 'that for the sake of which'. 

Now mistakes come to pass even in the operations of art: the grammarian
makes a mistake in writing and the doctor pours out the wrong dose.
Hence clearly mistakes are possible in the operations of nature also.
If then in art there are cases in which what is rightly produced serves
a purpose, and if where mistakes occur there was a purpose in what
was attempted, only it was not attained, so must it be also in natural
products, and monstrosities will be failures in the purposive effort.
Thus in the original combinations the 'ox-progeny' if they failed
to reach a determinate end must have arisen through the corruption
of some principle corresponding to what is now the seed.

Further, seed must have come into being first, and not straightway
the animals: the words 'whole-natured first...' must have meant seed.

Again, in plants too we find the relation of means to end, though
the degree of organization is less. Were there then in plants also
'olive-headed vine-progeny', like the 'man-headed ox-progeny', or
not? An absurd suggestion; yet there must have been, if there were
such things among animals. 

Moreover, among the seeds anything must have come to be at random.
But the person who asserts this entirely does away with 'nature' and
what exists 'by nature'. For those things are natural which, by a
continuous movement originated from an internal principle, arrive
at some completion: the same completion is not reached from every
principle; nor any chance completion, but always the tendency in each
is towards the same end, if there is no impediment. 

The end and the means towards it may come about by chance. We say,
for instance, that a stranger has come by chance, paid the ransom,
and gone away, when he does so as if he had come for that purpose,
though it was not for that that he came. This is incidental, for chance
is an incidental cause, as I remarked before. But when an event takes
place always or for the most part, it is not incidental or by chance.
In natural products the sequence is invariable, if there is no impediment.

It is absurd to suppose that purpose is not present because we do
not observe the agent deliberating. Art does not deliberate. If the
ship-building art were in the wood, it would produce the same results
by nature. If, therefore, purpose is present in art, it is present
also in nature. The best illustration is a doctor doctoring himself:
nature is like that. 

It is plain then that nature is a cause, a cause that operates for
a purpose. 

Part 9

As regards what is 'of necessity', we must ask whether the necessity
is 'hypothetical', or 'simple' as well. The current view places what
is of necessity in the process of production, just as if one were
to suppose that the wall of a house necessarily comes to be because
what is heavy is naturally carried downwards and what is light to
the top, wherefore the stones and foundations take the lowest place,
with earth above because it is lighter, and wood at the top of all
as being the lightest. Whereas, though the wall does not come to be
without these, it is not due to these, except as its material cause:
it comes to be for the sake of sheltering and guarding certain things.
Similarly in all other things which involve production for an end;
the product cannot come to be without things which have a necessary
nature, but it is not due to these (except as its material); it comes
to be for an end. For instance, why is a saw such as it is? To effect
so-and-so and for the sake of so-and-so. This end, however, cannot
be realized unless the saw is made of iron. It is, therefore, necessary
for it to be of iron, it we are to have a saw and perform the operation
of sawing. What is necessary then, is necessary on a hypothesis; it
is not a result necessarily determined by antecedents. Necessity is
in the matter, while 'that for the sake of which' is in the definition.

Necessity in mathematics is in a way similar to necessity in things
which come to be through the operation of nature. Since a straight
line is what it is, it is necessary that the angles of a triangle
should equal two right angles. But not conversely; though if the angles
are not equal to two right angles, then the straight line is not what
it is either. But in things which come to be for an end, the reverse
is true. If the end is to exist or does exist, that also which precedes
it will exist or does exist; otherwise just as there, if-the conclusion
is not true, the premiss will not be true, so here the end or 'that
for the sake of which' will not exist. For this too is itself a starting-point,
but of the reasoning, not of the action; while in mathematics the
starting-point is the starting-point of the reasoning only, as there
is no action. If then there is to be a house, such-and-such things
must be made or be there already or exist, or generally the matter
relative to the end, bricks and stones if it is a house. But the end
is not due to these except as the matter, nor will it come to exist
because of them. Yet if they do not exist at all, neither will the
house, or the saw-the former in the absence of stones, the latter
in the absence of iron-just as in the other case the premisses will
not be true, if the angles of the triangle are not equal to two right
angles. 

The necessary in nature, then, is plainly what we call by the name
of matter, and the changes in it. Both causes must be stated by the
physicist, but especially the end; for that is the cause of the matter,
not vice versa; and the end is 'that for the sake of which', and the
beginning starts from the definition or essence; as in artificial
products, since a house is of such-and-such a kind, certain things
must necessarily come to be or be there already, or since health is
this, these things must necessarily come to be or be there already.
Similarly if man is this, then these; if these, then those. Perhaps
the necessary is present also in the definition. For if one defines
the operation of sawing as being a certain kind of dividing, then
this cannot come about unless the saw has teeth of a certain kind;
and these cannot be unless it is of iron. For in the definition too
there are some parts that are, as it were, its matter. 

----------------------------------------------------------------------

BOOK III

Part 1 

Nature has been defined as a 'principle of motion and change', and
it is the subject of our inquiry. We must therefore see that we understand
the meaning of 'motion'; for if it were unknown, the meaning of 'nature'
too would be unknown. 

When we have determined the nature of motion, our next task will be
to attack in the same way the terms which are involved in it. Now
motion is supposed to belong to the class of things which are continuous;
and the infinite presents itself first in the continuous-that is how
it comes about that 'infinite' is often used in definitions of the
continuous ('what is infinitely divisible is continuous'). Besides
these, place, void, and time are thought to be necessary conditions
of motion. 

Clearly, then, for these reasons and also because the attributes mentioned
are common to, and coextensive with, all the objects of our science,
we must first take each of them in hand and discuss it. For the investigation
of special attributes comes after that of the common attributes.

To begin then, as we said, with motion. 
We may start by distinguishing (1) what exists in a state of fulfilment
only, (2) what exists as potential, (3) what exists as potential and
also in fulfilment-one being a 'this', another 'so much', a third
'such', and similarly in each of the other modes of the predication
of being. 

Further, the word 'relative' is used with reference to (1) excess
and defect, (2) agent and patient and generally what can move and
what can be moved. For 'what can cause movement' is relative to 'what
can be moved', and vice versa. 

Again, there is no such thing as motion over and above the things.
It is always with respect to substance or to quantity or to quality
or to place that what changes changes. But it is impossible, as we
assert, to find anything common to these which is neither 'this' nor
quantum nor quale nor any of the other predicates. Hence neither will
motion and change have reference to something over and above the things
mentioned, for there is nothing over and above them. 

Now each of these belongs to all its subjects in either of two ways:
namely (1) substance-the one is positive form, the other privation;
(2) in quality, white and black; (3) in quantity, complete and incomplete;
(4) in respect of locomotion, upwards and downwards or light and heavy.
Hence there are as many types of motion or change as there are meanings
of the word 'is'. 

We have now before us the distinctions in the various classes of being
between what is full real and what is potential. 

Def. The fulfilment of what exists potentially, in so far as it exists
potentially, is motion-namely, of what is alterable qua alterable,
alteration: of what can be increased and its opposite what can be
decreased (there is no common name), increase and decrease: of what
can come to be and can pass away, coming to he and passing away: of
what can be carried along, locomotion. 

Examples will elucidate this definition of motion. When the buildable,
in so far as it is just that, is fully real, it is being built, and
this is building. Similarly, learning, doctoring, rolling, leaping,
ripening, ageing. 

The same thing, if it is of a certain kind, can be both potential
and fully real, not indeed at the same time or not in the same respect,
but e.g. potentially hot and actually cold. Hence at once such things
will act and be acted on by one another in many ways: each of them
will be capable at the same time of causing alteration and of being
altered. Hence, too, what effects motion as a physical agent can be
moved: when a thing of this kind causes motion, it is itself also
moved. This, indeed, has led some people to suppose that every mover
is moved. But this question depends on another set of arguments, and
the truth will be made clear later. is possible for a thing to cause
motion, though it is itself incapable of being moved. 

It is the fulfilment of what is potential when it is already fully
real and operates not as itself but as movable, that is motion. What
I mean by 'as' is this: Bronze is potentially a statue. But it is
not the fulfilment of bronze as bronze which is motion. For 'to be
bronze' and 'to be a certain potentiality' are not the same.

If they were identical without qualification, i.e. in definition,
the fulfilment of bronze as bronze would have been motion. But they
are not the same, as has been said. (This is obvious in contraries.
'To be capable of health' and 'to be capable of illness' are not the
same, for if they were there would be no difference between being
ill and being well. Yet the subject both of health and of sickness-whether
it is humour or blood-is one and the same.) 

We can distinguish, then, between the two-just as, to give another
example, 'colour' and visible' are different-and clearly it is the
fulfilment of what is potential as potential that is motion. So this,
precisely, is motion. 

Further it is evident that motion is an attribute of a thing just
when it is fully real in this way, and neither before nor after. For
each thing of this kind is capable of being at one time actual, at
another not. Take for instance the buildable as buildable. The actuality
of the buildable as buildable is the process of building. For the
actuality of the buildable must be either this or the house. But when
there is a house, the buildable is no longer buildable. On the other
hand, it is the buildable which is being built. The process then of
being built must be the kind of actuality required But building is
a kind of motion, and the same account will apply to the other kinds
also. 

Part 2

The soundness of this definition is evident both when we consider
the accounts of motion that the others have given, and also from the
difficulty of defining it otherwise. 

One could not easily put motion and change in another genus-this is
plain if we consider where some people put it; they identify motion
with or 'inequality' or 'not being'; but such things are not necessarily
moved, whether they are 'different' or 'unequal' or 'non-existent';
Nor is change either to or from these rather than to or from their
opposites. 

The reason why they put motion into these genera is that it is thought
to be something indefinite, and the principles in the second column
are indefinite because they are privative: none of them is either
'this' or 'such' or comes under any of the other modes of predication.
The reason in turn why motion is thought to be indefinite is that
it cannot be classed simply as a potentiality or as an actuality-a
thing that is merely capable of having a certain size is not undergoing
change, nor yet a thing that is actually of a certain size, and motion
is thought to be a sort of actuality, but incomplete, the reason for
this view being that the potential whose actuality it is is incomplete.
This is why it is hard to grasp what motion is. It is necessary to
class it with privation or with potentiality or with sheer actuality,
yet none of these seems possible. There remains then the suggested
mode of definition, namely that it is a sort of actuality, or actuality
of the kind described, hard to grasp, but not incapable of existing.

The mover too is moved, as has been said-every mover, that is, which
is capable of motion, and whose immobility is rest-when a thing is
subject to motion its immobility is rest. For to act on the movable
as such is just to move it. But this it does by contact, so that at
the same time it is also acted on. Hence we can define motion as the
fulfilment of the movable qua movable, the cause of the attribute
being contact with what can move so that the mover is also acted on.
The mover or agent will always be the vehicle of a form, either a
'this' or 'such', which, when it acts, will be the source and cause
of the change, e.g. the full-formed man begets man from what is potentially
man. 

Part 3

The solution of the difficulty that is raised about the motion-whether
it is in the movable-is plain. It is the fulfilment of this potentiality,
and by the action of that which has the power of causing motion; and
the actuality of that which has the power of causing motion is not
other than the actuality of the movable, for it must be the fulfilment
of both. A thing is capable of causing motion because it can do this,
it is a mover because it actually does it. But it is on the movable
that it is capable of acting. Hence there is a single actuality of
both alike, just as one to two and two to one are the same interval,
and the steep ascent and the steep descent are one-for these are one
and the same, although they can be described in different ways. So
it is with the mover and the moved. 

This view has a dialectical difficulty. Perhaps it is necessary that
the actuality of the agent and that of the patient should not be the
same. The one is 'agency' and the other 'patiency'; and the outcome
and completion of the one is an 'action', that of the other a 'passion'.
Since then they are both motions, we may ask: in what are they, if
they are different? Either (a) both are in what is acted on and moved,
or (b) the agency is in the agent and the patiency in the patient.
(If we ought to call the latter also 'agency', the word would be used
in two senses.) 

Now, in alternative (b), the motion will be in the mover, for the
same statement will hold of 'mover' and 'moved'. Hence either every
mover will be moved, or, though having motion, it will not be moved.

If on the other hand (a) both are in what is moved and acted on-both
the agency and the patiency (e.g. both teaching and learning, though
they are two, in the learner), then, first, the actuality of each
will not be present in each, and, a second absurdity, a thing will
have two motions at the same time. How will there be two alterations
of quality in one subject towards one definite quality? The thing
is impossible: the actualization will be one. 

But (some one will say) it is contrary to reason to suppose that there
should be one identical actualization of two things which are different
in kind. Yet there will be, if teaching and learning are the same,
and agency and patiency. To teach will be the same as to learn, and
to act the same as to be acted on-the teacher will necessarily be
learning everything that he teaches, and the agent will be acted on.
One may reply: 

(1) It is not absurd that the actualization of one thing should be
in another. Teaching is the activity of a person who can teach, yet
the operation is performed on some patient-it is not cut adrift from
a subject, but is of A on B. 

(2) There is nothing to prevent two things having one and the same
actualization, provided the actualizations are not described in the
same way, but are related as what can act to what is acting.

(3) Nor is it necessary that the teacher should learn, even if to
act and to be acted on are one and the same, provided they are not
the same in definition (as 'raiment' and 'dress'), but are the same
merely in the sense in which the road from Thebes to Athens and the
road from Athens to Thebes are the same, as has been explained above.
For it is not things which are in a way the same that have all their
attributes the same, but only such as have the same definition. But
indeed it by no means follows from the fact that teaching is the same
as learning, that to learn is the same as to teach, any more than
it follows from the fact that there is one distance between two things
which are at a distance from each other, that the two vectors AB and
Ba, are one and the same. To generalize, teaching is not the same
as learning, or agency as patiency, in the full sense, though they
belong to the same subject, the motion; for the 'actualization of
X in Y' and the 'actualization of Y through the action of X' differ
in definition. 

What then Motion is, has been stated both generally and particularly.
It is not difficult to see how each of its types will be defined-alteration
is the fulfillment of the alterable qua alterable (or, more scientifically,
the fulfilment of what can act and what can be acted on, as such)-generally
and again in each particular case, building, healing, &c. A similar
definition will apply to each of the other kinds of motion.

Part 4

The science of nature is concerned with spatial magnitudes and motion
and time, and each of these at least is necessarily infinite or finite,
even if some things dealt with by the science are not, e.g. a quality
or a point-it is not necessary perhaps that such things should be
put under either head. Hence it is incumbent on the person who specializes
in physics to discuss the infinite and to inquire whether there is
such a thing or not, and, if there is, what it is. 

The appropriateness to the science of this problem is clearly indicated.
All who have touched on this kind of science in a way worth considering
have formulated views about the infinite, and indeed, to a man, make
it a principle of things. 

(1) Some, as the Pythagoreans and Plato, make the infinite a principle
in the sense of a self-subsistent substance, and not as a mere attribute
of some other thing. Only the Pythagoreans place the infinite among
the objects of sense (they do not regard number as separable from
these), and assert that what is outside the heaven is infinite. Plato,
on the other hand, holds that there is no body outside (the Forms
are not outside because they are nowhere),yet that the infinite is
present not only in the objects of sense but in the Forms also.

Further, the Pythagoreans identify the infinite with the even. For
this, they say, when it is cut off and shut in by the odd, provides
things with the element of infinity. An indication of this is what
happens with numbers. If the gnomons are placed round the one, and
without the one, in the one construction the figure that results is
always different, in the other it is always the same. But Plato has
two infinites, the Great and the Small. 

The physicists, on the other hand, all of them, always regard the
infinite as an attribute of a substance which is different from it
and belongs to the class of the so-called elements-water or air or
what is intermediate between them. Those who make them limited in
number never make them infinite in amount. But those who make the
elements infinite in number, as Anaxagoras and Democritus do, say
that the infinite is continuous by contact-compounded of the homogeneous
parts according to the one, of the seed-mass of the atomic shapes
according to the other. 

Further, Anaxagoras held that any part is a mixture in the same way
as the All, on the ground of the observed fact that anything comes
out of anything. For it is probably for this reason that he maintains
that once upon a time all things were together. (This flesh and this
bone were together, and so of any thing: therefore all things: and
at the same time too.) For there is a beginning of separation, not
only for each thing, but for all. Each thing that comes to be comes
from a similar body, and there is a coming to be of all things, though
not, it is true, at the same time. Hence there must also be an origin
of coming to be. One such source there is which he calls Mind, and
Mind begins its work of thinking from some starting-point. So necessarily
all things must have been together at a certain time, and must have
begun to be moved at a certain time. 

Democritus, for his part, asserts the contrary, namely that no element
arises from another element. Nevertheless for him the common body
is a source of all things, differing from part to part in size and
in shape. 

It is clear then from these considerations that the inquiry concerns
the physicist. Nor is it without reason that they all make it a principle
or source. We cannot say that the infinite has no effect, and the
only effectiveness which we can ascribe to it is that of a principle.
Everything is either a source or derived from a source. But there
cannot be a source of the infinite or limitless, for that would be
a limit of it. Further, as it is a beginning, it is both uncreatable
and indestructible. For there must be a point at which what has come
to be reaches completion, and also a termination of all passing away.
That is why, as we say, there is no principle of this, but it is this
which is held to be the principle of other things, and to encompass
all and to steer all, as those assert who do not recognize, alongside
the infinite, other causes, such as Mind or Friendship. Further they
identify it with the Divine, for it is 'deathless and imperishable'
as Anaximander says, with the majority of the physicists.

Belief in the existence of the infinite comes mainly from five considerations:

(1) From the nature of time-for it is infinite. 
(2) From the division of magnitudes-for the mathematicians also use
the notion of the infinite. 

(3) If coming to be and passing away do not give out, it is only because
that from which things come to be is infinite. 

(4) Because the limited always finds its limit in something, so that
there must be no limit, if everything is always limited by something
different from itself. 

(5) Most of all, a reason which is peculiarly appropriate and presents
the difficulty that is felt by everybody-not only number but also
mathematical magnitudes and what is outside the heaven are supposed
to be infinite because they never give out in our thought.

The last fact (that what is outside is infinite) leads people to suppose
that body also is infinite, and that there is an infinite number of
worlds. Why should there be body in one part of the void rather than
in another? Grant only that mass is anywhere and it follows that it
must be everywhere. Also, if void and place are infinite, there must
be infinite body too, for in the case of eternal things what may be
must be. But the problem of the infinite is difficult: many contradictions
result whether we suppose it to exist or not to exist. If it exists,
we have still to ask how it exists; as a substance or as the essential
attribute of some entity? Or in neither way, yet none the less is
there something which is infinite or some things which are infinitely
many? 

The problem, however, which specially belongs to the physicist is
to investigate whether there is a sensible magnitude which is infinite.

We must begin by distinguishing the various senses in which the term
'infinite' is used. 

(1) What is incapable of being gone through, because it is not in
its nature to be gone through (the sense in which the voice is 'invisible').

(2) What admits of being gone through, the process however having
no termination, or what scarcely admits of being gone through.

(3) What naturally admits of being gone through, but is not actually
gone through or does not actually reach an end. 

Further, everything that is infinite may be so in respect of addition
or division or both. 

Part 5

Now it is impossible that the infinite should be a thing which is
itself infinite, separable from sensible objects. If the infinite
is neither a magnitude nor an aggregate, but is itself a substance
and not an attribute, it will be indivisible; for the divisible must
be either a magnitude or an aggregate. But if indivisible, then not
infinite, except in the sense (1) in which the voice is 'invisible'.
But this is not the sense in which it is used by those who say that
the infinite exists, nor that in which we are investigating it, namely
as (2) 'that which cannot be gone through'. But if the infinite exists
as an attribute, it would not be, qua infinite an element in substances,
any more than the invisible would be an element of speech, though
the voice is invisible. 

Further, how can the infinite be itself any thing, unless both number
and magnitude, of which it is an essential attribute, exist in that
way? If they are not substances, a fortiori the infinite is not.

It is plain, too, that the infinite cannot be an actual thing and
a substance and principle. For any part of it that is taken will be
infinite, if it has parts: for 'to be infinite' and 'the infinite'
are the same, if it is a substance and not predicated of a subject.
Hence it will be either indivisible or divisible into infinites. But
the same thing cannot be many infinites. (Yet just as part of air
is air, so a part of the infinite would be infinite, if it is supposed
to be a substance and principle.) Therefore the infinite must be without
parts and indivisible. But this cannot be true of what is infinite
in full completion: for it must be a definite quantity. 

Suppose then that infinity belongs to substance as an attribute. But,
if so, it cannot, as we have said, be described as a principle, but
rather that of which it is an attribute-the air or the even number.

Thus the view of those who speak after the manner of the Pythagoreans
is absurd. With the same breath they treat the infinite as substance,
and divide it into parts. 

This discussion, however, involves the more general question whether
the infinite can be present in mathematical objects and things which
are intelligible and do not have extension, as well as among sensible
objects. Our inquiry (as physicists) is limited to its special subject-matter,
the objects of sense, and we have to ask whether there is or is not
among them a body which is infinite in the direction of increase.

We may begin with a dialectical argument and show as follows that
there is no such thing. If 'bounded by a surface' is the definition
of body there cannot be an infinite body either intelligible or sensible.
Nor can number taken in abstraction be infinite, for number or that
which has number is numerable. If then the numerable can be numbered,
it would also be possible to go through the infinite. 

If, on the other hand, we investigate the question more in accordance
with principles appropriate to physics, we are led as follows to the
same result. 

The infinite body must be either (1) compound, or (2) simple; yet
neither alternative is possible. 

(1) Compound the infinite body will not be, if the elements are finite
in number. For they must be more than one, and the contraries must
always balance, and no one of them can be infinite. If one of the
bodies falls in any degree short of the other in potency-suppose fire
is finite in amount while air is infinite and a given quantity of
fire exceeds in power the same amount of air in any ratio provided
it is numerically definite-the infinite body will obviously prevail
over and annihilate the finite body. On the other hand, it is impossible
that each should be infinite. 'Body' is what has extension in all
directions and the infinite is what is boundlessly extended, so that
the infinite body would be extended in all directions ad infinitum.

Nor (2) can the infinite body be one and simple, whether it is, as
some hold, a thing over and above the elements (from which they generate
the elements) or is not thus qualified. 

(a) We must consider the former alternative; for there are some people
who make this the infinite, and not air or water, in order that the
other elements may not be annihilated by the element which is infinite.
They have contrariety with each other-air is cold, water moist, fire
hot; if one were infinite, the others by now would have ceased to
be. As it is, they say, the infinite is different from them and is
their source. 

It is impossible, however, that there should be such a body; not because
it is infinite on that point a general proof can be given which applies
equally to all, air, water, or anything else-but simply because there
is, as a matter of fact, no such sensible body, alongside the so-called
elements. Everything can be resolved into the elements of which it
is composed. Hence the body in question would have been present in
our world here, alongside air and fire and earth and water: but nothing
of the kind is observed. 

(b) Nor can fire or any other of the elements be infinite. For generally,
and apart from the question of how any of them could be infinite,
the All, even if it were limited, cannot either be or become one of
them, as Heraclitus says that at some time all things become fire.
(The same argument applies also to the one which the physicists suppose
to exist alongside the elements: for everything changes from contrary
to contrary, e.g. from hot to cold). 

The preceding consideration of the various cases serves to show us
whether it is or is not possible that there should be an infinite
sensible body. The following arguments give a general demonstration
that it is not possible. 

It is the nature of every kind of sensible body to be somewhere, and
there is a place appropriate to each, the same for the part and for
the whole, e.g. for the whole earth and for a single clod, and for
fire and for a spark. 

Suppose (a) that the infinite sensible body is homogeneous. Then each
part will be either immovable or always being carried along. Yet neither
is possible. For why downwards rather than upwards or in any other
direction? I mean, e.g, if you take a clod, where will it be moved
or where will it be at rest? For ex hypothesi the place of the body
akin to it is infinite. Will it occupy the whole place, then? And
how? What then will be the nature of its rest and of its movement,
or where will they be? It will either be at home everywhere-then it
will not be moved; or it will be moved everywhere-then it will not
come to rest. 

But if (b) the All has dissimilar parts, the proper places of the
parts will be dissimilar also, and the body of the All will have no
unity except that of contact. Then, further, the parts will be either
finite or infinite in variety of kind. (i) Finite they cannot be,
for if the All is to be infinite, some of them would have to be infinite,
while the others were not, e.g. fire or water will be infinite. But,
as we have seen before, such an element would destroy what is contrary
to it. (This indeed is the reason why none of the physicists made
fire or earth the one infinite body, but either water or air or what
is intermediate between them, because the abode of each of the two
was plainly determinate, while the others have an ambiguous place
between up and down.) 

But (ii) if the parts are infinite in number and simple, their proper
places too will be infinite in number, and the same will be true of
the elements themselves. If that is impossible, and the places are
finite, the whole too must be finite; for the place and the body cannot
but fit each other. Neither is the whole place larger than what can
be filled by the body (and then the body would no longer be infinite),
nor is the body larger than the place; for either there would be an
empty space or a body whose nature it is to be nowhere. 

Anaxagoras gives an absurd account of why the infinite is at rest.
He says that the infinite itself is the cause of its being fixed.
This because it is in itself, since nothing else contains it-on the
assumption that wherever anything is, it is there by its own nature.
But this is not true: a thing could be somewhere by compulsion, and
not where it is its nature to be. 

Even if it is true as true can be that the whole is not moved (for
what is fixed by itself and is in itself must be immovable), yet we
must explain why it is not its nature to be moved. It is not enough
just to make this statement and then decamp. Anything else might be
in a state of rest, but there is no reason why it should not be its
nature to be moved. The earth is not carried along, and would not
be carried along if it were infinite, provided it is held together
by the centre. But it would not be because there was no other region
in which it could be carried along that it would remain at the centre,
but because this is its nature. Yet in this case also we may say that
it fixes itself. If then in the case of the earth, supposed to be
infinite, it is at rest, not because it is infinite, but because it
has weight and what is heavy rests at the centre and the earth is
at the centre, similarly the infinite also would rest in itself, not
because it is infinite and fixes itself, but owing to some other cause.

Another difficulty emerges at the same time. Any part of the infinite
body ought to remain at rest. Just as the infinite remains at rest
in itself because it fixes itself, so too any part of it you may take
will remain in itself. The appropriate places of the whole and of
the part are alike, e.g. of the whole earth and of a clod the appropriate
place is the lower region; of fire as a whole and of a spark, the
upper region. If, therefore, to be in itself is the place of the infinite,
that also will be appropriate to the part. Therefore it will remain
in itself. 

In general, the view that there is an infinite body is plainly incompatible
with the doctrine that there is necessarily a proper place for each
kind of body, if every sensible body has either weight or lightness,
and if a body has a natural locomotion towards the centre if it is
heavy, and upwards if it is light. This would need to be true of the
infinite also. But neither character can belong to it: it cannot be
either as a whole, nor can it be half the one and half the other.
For how should you divide it? or how can the infinite have the one
part up and the other down, or an extremity and a centre?

Further, every sensible body is in place, and the kinds or differences
of place are up-down, before-behind, right-left; and these distinctions
hold not only in relation to us and by arbitrary agreement, but also
in the whole itself. But in the infinite body they cannot exist. In
general, if it is impossible that there should be an infinite place,
and if every body is in place, there cannot be an infinite body.

Surely what is in a special place is in place, and what is in place
is in a special place. Just, then, as the infinite cannot be quantity-that
would imply that it has a particular quantity, e,g, two or three cubits;
quantity just means these-so a thing's being in place means that it
is somewhere, and that is either up or down or in some other of the
six differences of position: but each of these is a limit.

It is plain from these arguments that there is no body which is actually
infinite. 

Part 6

But on the other hand to suppose that the infinite does not exist
in any way leads obviously to many impossible consequences: there
will be a beginning and an end of time, a magnitude will not be divisible
into magnitudes, number will not be infinite. If, then, in view of
the above considerations, neither alternative seems possible, an arbiter
must be called in; and clearly there is a sense in which the infinite
exists and another in which it does not. 

We must keep in mind that the word 'is' means either what potentially
is or what fully is. Further, a thing is infinite either by addition
or by division. 

Now, as we have seen, magnitude is not actually infinite. But by division
it is infinite. (There is no difficulty in refuting the theory of
indivisible lines.) The alternative then remains that the infinite
has a potential existence. 

But the phrase 'potential existence' is ambiguous. When we speak of
the potential existence of a statue we mean that there will be an
actual statue. It is not so with the infinite. There will not be an
actual infinite. The word 'is' has many senses, and we say that the
infinite 'is' in the sense in which we say 'it is day' or 'it is the
games', because one thing after another is always coming into existence.
For of these things too the distinction between potential and actual
existence holds. We say that there are Olympic games, both in the
sense that they may occur and that they are actually occurring.

The infinite exhibits itself in different ways-in time, in the generations
of man, and in the division of magnitudes. For generally the infinite
has this mode of existence: one thing is always being taken after
another, and each thing that is taken is always finite, but always
different. Again, 'being' has more than one sense, so that we must
not regard the infinite as a 'this', such as a man or a horse, but
must suppose it to exist in the sense in which we speak of the day
or the games as existing things whose being has not come to them like
that of a substance, but consists in a process of coming to be or
passing away; definite if you like at each stage, yet always different.

But when this takes place in spatial magnitudes, what is taken perists,
while in the succession of time and of men it takes place by the passing
away of these in such a way that the source of supply never gives
out. 

In a way the infinite by addition is the same thing as the infinite
by division. In a finite magnitude, the infinite by addition comes
about in a way inverse to that of the other. For in proportion as
we see division going on, in the same proportion we see addition being
made to what is already marked off. For if we take a determinate part
of a finite magnitude and add another part determined by the same
ratio (not taking in the same amount of the original whole), and so
on, we shall not traverse the given magnitude. But if we increase
the ratio of the part, so as always to take in the same amount, we
shall traverse the magnitude, for every finite magnitude is exhausted
by means of any determinate quantity however small. 

The infinite, then, exists in no other way, but in this way it does
exist, potentially and by reduction. It exists fully in the sense
in which we say 'it is day' or 'it is the games'; and potentially
as matter exists, not independently as what is finite does.

By addition then, also, there is potentially an infinite, namely,
what we have described as being in a sense the same as the infinite
in respect of division. For it will always be possible to take something
ah extra. Yet the sum of the parts taken will not exceed every determinate
magnitude, just as in the direction of division every determinate
magnitude is surpassed in smallness and there will be a smaller part.

But in respect of addition there cannot be an infinite which even
potentially exceeds every assignable magnitude, unless it has the
attribute of being actually infinite, as the physicists hold to be
true of the body which is outside the world, whose essential nature
is air or something of the kind. But if there cannot be in this way
a sensible body which is infinite in the full sense, evidently there
can no more be a body which is potentially infinite in respect of
addition, except as the inverse of the infinite by division, as we
have said. It is for this reason that Plato also made the infinites
two in number, because it is supposed to be possible to exceed all
limits and to proceed ad infinitum in the direction both of increase
and of reduction. Yet though he makes the infinites two, he does not
use them. For in the numbers the infinite in the direction of reduction
is not present, as the monad is the smallest; nor is the infinite
in the direction of increase, for the parts number only up to the
decad. 

The infinite turns out to be the contrary of what it is said to be.
It is not what has nothing outside it that is infinite, but what always
has something outside it. This is indicated by the fact that rings
also that have no bezel are described as 'endless', because it is
always possible to take a part which is outside a given part. The
description depends on a certain similarity, but it is not true in
the full sense of the word. This condition alone is not sufficient:
it is necessary also that the next part which is taken should never
be the same. In the circle, the latter condition is not satisfied:
it is only the adjacent part from which the new part is different.

Our definition then is as follows: 
A quantity is infinite if it is such that we can always take a part
outside what has been already taken. On the other hand, what has nothing
outside it is complete and whole. For thus we define the whole-that
from which nothing is wanting, as a whole man or a whole box. What
is true of each particular is true of the whole as such-the whole
is that of which nothing is outside. On the other hand that from which
something is absent and outside, however small that may be, is not
'all'. 'Whole' and 'complete' are either quite identical or closely
akin. Nothing is complete (teleion) which has no end (telos); and
the end is a limit. 

Hence Parmenides must be thought to have spoken better than Melissus.
The latter says that the whole is infinite, but the former describes
it as limited, 'equally balanced from the middle'. For to connect
the infinite with the all and the whole is not like joining two pieces
of string; for it is from this they get the dignity they ascribe to
the infinite-its containing all things and holding the all in itself-from
its having a certain similarity to the whole. It is in fact the matter
of the completeness which belongs to size, and what is potentially
a whole, though not in the full sense. It is divisible both in the
direction of reduction and of the inverse addition. It is a whole
and limited; not, however, in virtue of its own nature, but in virtue
of what is other than it. It does not contain, but, in so far as it
is infinite, is contained. Consequently, also, it is unknowable, qua
infinite; for the matter has no form. (Hence it is plain that the
infinite stands in the relation of part rather than of whole. For
the matter is part of the whole, as the bronze is of the bronze statue.)
If it contains in the case of sensible things, in the case of intelligible
things the great and the small ought to contain them. But it is absurd
and impossible to suppose that the unknowable and indeterminate should
contain and determine. 

Part 7

It is reasonable that there should not be held to be an infinite in
respect of addition such as to surpass every magnitude, but that there
should be thought to be such an infinite in the direction of division.
For the matter and the infinite are contained inside what contains
them, while it is the form which contains. It is natural too to suppose
that in number there is a limit in the direction of the minimum, and
that in the other direction every assigned number is surpassed. In
magnitude, on the contrary, every assigned magnitude is surpassed
in the direction of smallness, while in the other direction there
is no infinite magnitude. The reason is that what is one is indivisible
whatever it may be, e.g. a man is one man, not many. Number on the
other hand is a plurality of 'ones' and a certain quantity of them.
Hence number must stop at the indivisible: for 'two' and 'three' are
merely derivative terms, and so with each of the other numbers. But
in the direction of largeness it is always possible to think of a
larger number: for the number of times a magnitude can be bisected
is infinite. Hence this infinite is potential, never actual: the number
of parts that can be taken always surpasses any assigned number. But
this number is not separable from the process of bisection, and its
infinity is not a permanent actuality but consists in a process of
coming to be, like time and the number of time. 

With magnitudes the contrary holds. What is continuous is divided
ad infinitum, but there is no infinite in the direction of increase.
For the size which it can potentially be, it can also actually be.
Hence since no sensible magnitude is infinite, it is impossible to
exceed every assigned magnitude; for if it were possible there would
be something bigger than the heavens. 

The infinite is not the same in magnitude and movement and time, in
the sense of a single nature, but its secondary sense depends on its
primary sense, i.e. movement is called infinite in virtue of the magnitude
covered by the movement (or alteration or growth), and time because
of the movement. (I use these terms for the moment. Later I shall
explain what each of them means, and also why every magnitude is divisible
into magnitudes.) 

Our account does not rob the mathematicians of their science, by disproving
the actual existence of the infinite in the direction of increase,
in the sense of the untraversable. In point of fact they do not need
the infinite and do not use it. They postulate only that the finite
straight line may be produced as far as they wish. It is possible
to have divided in the same ratio as the largest quantity another
magnitude of any size you like. Hence, for the purposes of proof,
it will make no difference to them to have such an infinite instead,
while its existence will be in the sphere of real magnitudes.

In the fourfold scheme of causes, it is plain that the infinite is
a cause in the sense of matter, and that its essence is privation,
the subject as such being what is continuous and sensible. All the
other thinkers, too, evidently treat the infinite as matter-that is
why it is inconsistent in them to make it what contains, and not what
is contained. 

Part 8

It remains to dispose of the arguments which are supposed to support
the view that the infinite exists not only potentially but as a separate
thing. Some have no cogency; others can be met by fresh objections
that are valid. 

(1) In order that coming to be should not fail, it is not necessary
that there should be a sensible body which is actually infinite. The
passing away of one thing may be the coming to be of another, the
All being limited. 

(2) There is a difference between touching and being limited. The
former is relative to something and is the touching of something (for
everything that touches touches something), and further is an attribute
of some one of the things which are limited. On the other hand, what
is limited is not limited in relation to anything. Again, contact
is not necessarily possible between any two things taken at random.

(3) To rely on mere thinking is absurd, for then the excess or defect
is not in the thing but in the thought. One might think that one of
us is bigger than he is and magnify him ad infinitum. But it does
not follow that he is bigger than the size we are, just because some
one thinks he is, but only because he is the size he is. The thought
is an accident. 

(a) Time indeed and movement are infinite, and also thinking, in the
sense that each part that is taken passes in succession out of existence.

(b) Magnitude is not infinite either in the way of reduction or of
magnification in thought. 

This concludes my account of the way in which the infinite exists,
and of the way in which it does not exist, and of what it is.

----------------------------------------------------------------------

BOOK IV

Part 1 

The physicist must have a knowledge of Place, too, as well as of
the infinite-namely, whether there is such a thing or not, and the
manner of its existence and what it is-both because all suppose that
things which exist are somewhere (the non-existent is nowhere--where
is the goat-stag or the sphinx?), and because 'motion' in its most
general and primary sense is change of place, which we call 'locomotion'.

The question, what is place? presents many difficulties. An examination
of all the relevant facts seems to lead to divergent conclusions.
Moreover, we have inherited nothing from previous thinkers, whether
in the way of a statement of difficulties or of a solution.

The existence of place is held to be obvious from the fact of mutual
replacement. Where water now is, there in turn, when the water has
gone out as from a vessel, air is present. When therefore another
body occupies this same place, the place is thought to be different
from all the bodies which come to be in it and replace one another.
What now contains air formerly contained water, so that clearly the
place or space into which and out of which they passed was something
different from both. 

Further, the typical locomotions of the elementary natural bodies-namely,
fire, earth, and the like-show not only that place is something, but
also that it exerts a certain influence. Each is carried to its own
place, if it is not hindered, the one up, the other down. Now these
are regions or kinds of place-up and down and the rest of the six
directions. Nor do such distinctions (up and down and right and left,
&c.) hold only in relation to us. To us they are not always the same
but change with the direction in which we are turned: that is why
the same thing may be both right and left, up and down, before and
behind. But in nature each is distinct, taken apart by itself. It
is not every chance direction which is 'up', but where fire and what
is light are carried; similarly, too, 'down' is not any chance direction
but where what has weight and what is made of earth are carried-the
implication being that these places do not differ merely in relative
position, but also as possessing distinct potencies. This is made
plain also by the objects studied by mathematics. Though they have
no real place, they nevertheless, in respect of their position relatively
to us, have a right and left as attributes ascribed to them only in
consequence of their relative position, not having by nature these
various characteristics. Again, the theory that the void exists involves
the existence of place: for one would define void as place bereft
of body. 

These considerations then would lead us to suppose that place is something
distinct from bodies, and that every sensible body is in place. Hesiod
too might be held to have given a correct account of it when he made
chaos first. At least he says: 

'First of all things came chaos to being, then broad-breasted earth,'
implying that things need to have space first, because he thought,
with most people, that everything is somewhere and in place. If this
is its nature, the potency of place must be a marvellous thing, and
take precedence of all other things. For that without which nothing
else can exist, while it can exist without the others, must needs
be first; for place does not pass out of existence when the things
in it are annihilated. 

True, but even if we suppose its existence settled, the question of
its nature presents difficulty-whether it is some sort of 'bulk' of
body or some entity other than that, for we must first determine its
genus. 

(1) Now it has three dimensions, length, breadth, depth, the dimensions
by which all body also is bounded. But the place cannot be body; for
if it were there would be two bodies in the same place. 

(2) Further, if body has a place and space, clearly so too have surface
and the other limits of body; for the same statement will apply to
them: where the bounding planes of the water were, there in turn will
be those of the air. But when we come to a point we cannot make a
distinction between it and its place. Hence if the place of a point
is not different from the point, no more will that of any of the others
be different, and place will not be something different from each
of them. 

(3) What in the world then are we to suppose place to be? If it has
the sort of nature described, it cannot be an element or composed
of elements, whether these be corporeal or incorporeal: for while
it has size, it has not body. But the elements of sensible bodies
are bodies, while nothing that has size results from a combination
of intelligible elements. 

(4) Also we may ask: of what in things is space the cause? None of
the four modes of causation can be ascribed to it. It is neither in
the sense of the matter of existents (for nothing is composed of it),
nor as the form and definition of things, nor as end, nor does it
move existents. 

(5) Further, too, if it is itself an existent, where will it be? Zeno's
difficulty demands an explanation: for if everything that exists has
a place, place too will have a place, and so on ad infinitum.

(6) Again, just as every body is in place, so, too, every place has
a body in it. What then shall we say about growing things? It follows
from these premisses that their place must grow with them, if their
place is neither less nor greater than they are. 

By asking these questions, then, we must raise the whole problem about
place-not only as to what it is, but even whether there is such a
thing. 

Part 2

We may distinguish generally between predicating B of A because it
(A) is itself, and because it is something else; and particularly
between place which is common and in which all bodies are, and the
special place occupied primarily by each. I mean, for instance, that
you are now in the heavens because you are in the air and it is in
the heavens; and you are in the air because you are on the earth;
and similarly on the earth because you are in this place which contains
no more than you. 

Now if place is what primarily contains each body, it would be a limit,
so that the place would be the form or shape of each body by which
the magnitude or the matter of the magnitude is defined: for this
is the limit of each body. 

If, then, we look at the question in this way the place of a thing
is its form. But, if we regard the place as the extension of the magnitude,
it is the matter. For this is different from the magnitude: it is
what is contained and defined by the form, as by a bounding plane.
Matter or the indeterminate is of this nature; when the boundary and
attributes of a sphere are taken away, nothing but the matter is left.

This is why Plato in the Timaeus says that matter and space are the
same; for the 'participant' and space are identical. (It is true,
indeed, that the account he gives there of the 'participant' is different
from what he says in his so-called 'unwritten teaching'. Nevertheless,
he did identify place and space.) I mention Plato because, while all
hold place to be something, he alone tried to say what it is.

In view of these facts we should naturally expect to find difficulty
in determining what place is, if indeed it is one of these two things,
matter or form. They demand a very close scrutiny, especially as it
is not easy to recognize them apart. 

But it is at any rate not difficult to see that place cannot be either
of them. The form and the matter are not separate from the thing,
whereas the place can be separated. As we pointed out, where air was,
water in turn comes to be, the one replacing the other; and similarly
with other bodies. Hence the place of a thing is neither a part nor
a state of it, but is separable from it. For place is supposed to
be something like a vessel-the vessel being a transportable place.
But the vessel is no part of the thing. 

In so far then as it is separable from the thing, it is not the form:
qua containing, it is different from the matter. 

Also it is held that what is anywhere is both itself something and
that there is a different thing outside it. (Plato of course, if we
may digress, ought to tell us why the form and the numbers are not
in place, if 'what participates' is place-whether what participates
is the Great and the Small or the matter, as he called it in writing
in the Timaeus.) 

Further, how could a body be carried to its own place, if place was
the matter or the form? It is impossible that what has no reference
to motion or the distinction of up and down can be place. So place
must be looked for among things which have these characteristics.

If the place is in the thing (it must be if it is either shape or
matter) place will have a place: for both the form and the indeterminate
undergo change and motion along with the thing, and are not always
in the same place, but are where the thing is. Hence the place will
have a place. 

Further, when water is produced from air, the place has been destroyed,
for the resulting body is not in the same place. What sort of destruction
then is that? 

This concludes my statement of the reasons why space must be something,
and again of the difficulties that may be raised about its essential
nature. 

Part 3

The next step we must take is to see in how many senses one thing
is said to be 'in' another. 

(1) As the finger is 'in' the hand and generally the part 'in' the
whole. 

(2) As the whole is 'in' the parts: for there is no whole over and
above the parts. 

(3) As man is 'in' animal and generally species 'in' genus.

(4) As the genus is 'in' the species and generally the part of the
specific form 'in' the definition of the specific form. 

(5) As health is 'in' the hot and the cold and generally the form
'in' the matter. 

(6) As the affairs of Greece centre 'in' the king, and generally events
centre 'in' their primary motive agent. 

(7) As the existence of a thing centres 'in its good and generally
'in' its end, i.e. in 'that for the sake of which' it exists.

(8) In the strictest sense of all, as a thing is 'in' a vessel, and
generally 'in' place. 

One might raise the question whether a thing can be in itself, or
whether nothing can be in itself-everything being either nowhere or
in something else. 

The question is ambiguous; we may mean the thing qua itself or qua
something else. 

When there are parts of a whole-the one that in which a thing is,
the other the thing which is in it-the whole will be described as
being in itself. For a thing is described in terms of its parts, as
well as in terms of the thing as a whole, e.g. a man is said to be
white because the visible surface of him is white, or to be scientific
because his thinking faculty has been trained. The jar then will not
be in itself and the wine will not be in itself. But the jar of wine
will: for the contents and the container are both parts of the same
whole. 

In this sense then, but not primarily, a thing can be in itself, namely,
as 'white' is in body (for the visible surface is in body), and science
is in the mind. 

It is from these, which are 'parts' (in the sense at least of being
'in' the man), that the man is called white, &c. But the jar and the
wine in separation are not parts of a whole, though together they
are. So when there are parts, a thing will be in itself, as 'white'
is in man because it is in body, and in body because it resides in
the visible surface. We cannot go further and say that it is in surface
in virtue of something other than itself. (Yet it is not in itself:
though these are in a way the same thing,) they differ in essence,
each having a special nature and capacity, 'surface' and 'white'.

Thus if we look at the matter inductively we do not find anything
to be 'in' itself in any of the senses that have been distinguished;
and it can be seen by argument that it is impossible. For each of
two things will have to be both, e.g. the jar will have to be both
vessel and wine, and the wine both wine and jar, if it is possible
for a thing to be in itself; so that, however true it might be that
they were in each other, the jar will receive the wine in virtue not
of its being wine but of the wine's being wine, and the wine will
be in the jar in virtue not of its being a jar but of the jar's being
a jar. Now that they are different in respect of their essence is
evident; for 'that in which something is' and 'that which is in it'
would be differently defined. 

Nor is it possible for a thing to be in itself even incidentally:
for two things would at the same time in the same thing. The jar would
be in itself-if a thing whose nature it is to receive can be in itself;
and that which it receives, namely (if wine) wine, will be in it.

Obviously then a thing cannot be in itself primarily. 
Zeno's problem-that if Place is something it must be in something-is
not difficult to solve. There is nothing to prevent the first place
from being 'in' something else-not indeed in that as 'in' place, but
as health is 'in' the hot as a positive determination of it or as
the hot is 'in' body as an affection. So we escape the infinite regress.

Another thing is plain: since the vessel is no part of what is in
it (what contains in the strict sense is different from what is contained),
place could not be either the matter or the form of the thing contained,
but must different-for the latter, both the matter and the shape,
are parts of what is contained. 

This then may serve as a critical statement of the difficulties involved.

Part 4

What then after all is place? The answer to this question may be elucidated
as follows. 

Let us take for granted about it the various characteristics which
are supposed correctly to belong to it essentially. We assume then-

(1) Place is what contains that of which it is the place.

(2) Place is no part of the thing. 
(3) The immediate place of a thing is neither less nor greater than
the thing. 

(4) Place can be left behind by the thing and is separable. In addition:

(5) All place admits of the distinction of up and down, and each of
the bodies is naturally carried to its appropriate place and rests
there, and this makes the place either up or down. 

Having laid these foundations, we must complete the theory. We ought
to try to make our investigation such as will render an account of
place, and will not only solve the difficulties connected with it,
but will also show that the attributes supposed to belong to it do
really belong to it, and further will make clear the cause of the
trouble and of the difficulties about it. Such is the most satisfactory
kind of exposition. 

First then we must understand that place would not have been thought
of, if there had not been a special kind of motion, namely that with
respect to place. It is chiefly for this reason that we suppose the
heaven also to be in place, because it is in constant movement. Of
this kind of change there are two species-locomotion on the one hand
and, on the other, increase and diminution. For these too involve
variation of place: what was then in this place has now in turn changed
to what is larger or smaller. 

Again, when we say a thing is 'moved', the predicate either (1) belongs
to it actually, in virtue of its own nature, or (2) in virtue of something
conjoined with it. In the latter case it may be either (a) something
which by its own nature is capable of being moved, e.g. the parts
of the body or the nail in the ship, or (b) something which is not
in itself capable of being moved, but is always moved through its
conjunction with something else, as 'whiteness' or 'science'. These
have changed their place only because the subjects to which they belong
do so. 

We say that a thing is in the world, in the sense of in place, because
it is in the air, and the air is in the world; and when we say it
is in the air, we do not mean it is in every part of the air, but
that it is in the air because of the outer surface of the air which
surrounds it; for if all the air were its place, the place of a thing
would not be equal to the thing-which it is supposed to be, and which
the primary place in which a thing is actually is. 

When what surrounds, then, is not separate from the thing, but is
in continuity with it, the thing is said to be in what surrounds it,
not in the sense of in place, but as a part in a whole. But when the
thing is separate and in contact, it is immediately 'in' the inner
surface of the surrounding body, and this surface is neither a part
of what is in it nor yet greater than its extension, but equal to
it; for the extremities of things which touch are coincident.

Further, if one body is in continuity with another, it is not moved
in that but with that. On the other hand it is moved in that if it
is separate. It makes no difference whether what contains is moved
or not. 

Again, when it is not separate it is described as a part in a whole,
as the pupil in the eye or the hand in the body: when it is separate,
as the water in the cask or the wine in the jar. For the hand is moved
with the body and the water in the cask. 

It will now be plain from these considerations what place is. There
are just four things of which place must be one-the shape, or the
matter, or some sort of extension between the bounding surfaces of
the containing body, or this boundary itself if it contains no extension
over and above the bulk of the body which comes to be in it.

Three of these it obviously cannot be: 
(1) The shape is supposed to be place because it surrounds, for the
extremities of what contains and of what is contained are coincident.
Both the shape and the place, it is true, are boundaries. But not
of the same thing: the form is the boundary of the thing, the place
is the boundary of the body which contains it. 

(2) The extension between the extremities is thought to be something,
because what is contained and separate may often be changed while
the container remains the same (as water may be poured from a vessel)-the
assumption being that the extension is something over and above the
body displaced. But there is no such extension. One of the bodies
which change places and are naturally capable of being in contact
with the container falls in whichever it may chance to be.

If there were an extension which were such as to exist independently
and be permanent, there would be an infinity of places in the same
thing. For when the water and the air change places, all the portions
of the two together will play the same part in the whole which was
previously played by all the water in the vessel; at the same time
the place too will be undergoing change; so that there will be another
place which is the place of the place, and many places will be coincident.
There is not a different place of the part, in which it is moved,
when the whole vessel changes its place: it is always the same: for
it is in the (proximate) place where they are that the air and the
water (or the parts of the water) succeed each other, not in that
place in which they come to be, which is part of the place which is
the place of the whole world. 

(3) The matter, too, might seem to be place, at least if we consider
it in what is at rest and is thus separate but in continuity. For
just as in change of quality there is something which was formerly
black and is now white, or formerly soft and now hard-this is just
why we say that the matter exists-so place, because it presents a
similar phenomenon, is thought to exist-only in the one case we say
so because what was air is now water, in the other because where air
formerly was there a is now water. But the matter, as we said before,
is neither separable from the thing nor contains it, whereas place
has both characteristics. 

Well, then, if place is none of the three-neither the form nor the
matter nor an extension which is always there, different from, and
over and above, the extension of the thing which is displaced-place
necessarily is the one of the four which is left, namely, the boundary
of the containing body at which it is in contact with the contained
body. (By the contained body is meant what can be moved by way of
locomotion.) 

Place is thought to be something important and hard to grasp, both
because the matter and the shape present themselves along with it,
and because the displacement of the body that is moved takes place
in a stationary container, for it seems possible that there should
be an interval which is other than the bodies which are moved. The
air, too, which is thought to be incorporeal, contributes something
to the belief: it is not only the boundaries of the vessel which seem
to be place, but also what is between them, regarded as empty. Just,
in fact, as the vessel is transportable place, so place is a non-portable
vessel. So when what is within a thing which is moved, is moved and
changes its place, as a boat on a river, what contains plays the part
of a vessel rather than that of place. Place on the other hand is
rather what is motionless: so it is rather the whole river that is
place, because as a whole it is motionless. 

Hence we conclude that the innermost motionless boundary of what contains
is place. 

This explains why the middle of the heaven and the surface which faces
us of the rotating system are held to be 'up' and 'down' in the strict
and fullest sense for all men: for the one is always at rest, while
the inner side of the rotating body remains always coincident with
itself. Hence since the light is what is naturally carried up, and
the heavy what is carried down, the boundary which contains in the
direction of the middle of the universe, and the middle itself, are
down, and that which contains in the direction of the outermost part
of the universe, and the outermost part itself, are up. 

For this reason, too, place is thought to be a kind of surface, and
as it were a vessel, i.e. a container of the thing. 

Further, place is coincident with the thing, for boundaries are coincident
with the bounded. 

Part 5

If then a body has another body outside it and containing it, it is
in place, and if not, not. That is why, even if there were to be water
which had not a container, the parts of it, on the one hand, will
be moved (for one part is contained in another), while, on the other
hand, the whole will be moved in one sense, but not in another. For
as a whole it does not simultaneously change its place, though it
will be moved in a circle: for this place is the place of its parts.
(Some things are moved, not up and down, but in a circle; others up
and down, such things namely as admit of condensation and rarefaction.)

As was explained, some things are potentially in place, others actually.
So, when you have a homogeneous substance which is continuous, the
parts are potentially in place: when the parts are separated, but
in contact, like a heap, they are actually in place. 

Again, (1) some things are per se in place, namely every body which
is movable either by way of locomotion or by way of increase is per
se somewhere, but the heaven, as has been said, is not anywhere as
a whole, nor in any place, if at least, as we must suppose, no body
contains it. On the line on which it is moved, its parts have place:
for each is contiguous the next. 

But (2) other things are in place indirectly, through something conjoined
with them, as the soul and the heaven. The latter is, in a way, in
place, for all its parts are: for on the orb one part contains another.
That is why the upper part is moved in a circle, while the All is
not anywhere. For what is somewhere is itself something, and there
must be alongside it some other thing wherein it is and which contains
it. But alongside the All or the Whole there is nothing outside the
All, and for this reason all things are in the heaven; for the heaven,
we may say, is the All. Yet their place is not the same as the heaven.
It is part of it, the innermost part of it, which is in contact with
the movable body; and for this reason the earth is in water, and this
in the air, and the air in the aether, and the aether in heaven, but
we cannot go on and say that the heaven is in anything else.

It is clear, too, from these considerations that all the problems
which were raised about place will be solved when it is explained
in this way: 

(1) There is no necessity that the place should grow with the body
in it, 

(2) Nor that a point should have a place, 
(3) Nor that two bodies should be in the same place, 
(4) Nor that place should be a corporeal interval: for what is between
the boundaries of the place is any body which may chance to be there,
not an interval in body. 

Further, (5) place is also somewhere, not in the sense of being in
a place, but as the limit is in the limited; for not everything that
is is in place, but only movable body. 

Also (6) it is reasonable that each kind of body should be carried
to its own place. For a body which is next in the series and in contact
(not by compulsion) is akin, and bodies which are united do not affect
each other, while those which are in contact interact on each other.

Nor (7) is it without reason that each should remain naturally in
its proper place. For this part has the same relation to its place,
as a separable part to its whole, as when one moves a part of water
or air: so, too, air is related to water, for the one is like matter,
the other form-water is the matter of air, air as it were the actuality
of water, for water is potentially air, while air is potentially water,
though in another way. 

These distinctions will be drawn more carefully later. On the present
occasion it was necessary to refer to them: what has now been stated
obscurely will then be made more clear. If the matter and the fulfilment
are the same thing (for water is both, the one potentially, the other
completely), water will be related to air in a way as part to whole.
That is why these have contact: it is organic union when both become
actually one. 

This concludes my account of place-both of its existence and of its
nature. 

Part 6

The investigation of similar questions about the void, also, must
be held to belong to the physicist-namely whether it exists or not,
and how it exists or what it is-just as about place. The views taken
of it involve arguments both for and against, in much the same sort
of way. For those who hold that the void exists regard it as a sort
of place or vessel which is supposed to be 'full' when it holds the
bulk which it is capable of containing, 'void' when it is deprived
of that-as if 'void' and 'full' and 'place' denoted the same thing,
though the essence of the three is different. 

We must begin the inquiry by putting down the account given by those
who say that it exists, then the account of those who say that it
does not exist, and third the current view on these questions.

Those who try to show that the void does not exist do not disprove
what people really mean by it, but only their erroneous way of speaking;
this is true of Anaxagoras and of those who refute the existence of
the void in this way. They merely give an ingenious demonstration
that air is something--by straining wine-skins and showing the resistance
of the air, and by cutting it off in clepsydras. But people really
mean that there is an empty interval in which there is no sensible
body. They hold that everything which is in body is body and say that
what has nothing in it at all is void (so what is full of air is void).
It is not then the existence of air that needs to be proved, but the
non-existence of an interval, different from the bodies, either separable
or actual-an interval which divides the whole body so as to break
its continuity, as Democritus and Leucippus hold, and many other physicists-or
even perhaps as something which is outside the whole body, which remains
continuous. 

These people, then, have not reached even the threshold of the problem,
but rather those who say that the void exists. 

(1) They argue, for one thing, that change in place (i.e. locomotion
and increase) would not be. For it is maintained that motion would
seem not to exist, if there were no void, since what is full cannot
contain anything more. If it could, and there were two bodies in the
same place, it would also be true that any number of bodies could
be together; for it is impossible to draw a line of division beyond
which the statement would become untrue. If this were possible, it
would follow also that the smallest body would contain the greatest;
for 'many a little makes a mickle': thus if many equal bodies can
be together, so also can many unequal bodies. 

Melissus, indeed, infers from these considerations that the All is
immovable; for if it were moved there must, he says, be void, but
void is not among the things that exist. 

This argument, then, is one way in which they show that there is a
void. 

(2) They reason from the fact that some things are observed to contract
and be compressed, as people say that a cask will hold the wine which
formerly filled it, along with the skins into which the wine has been
decanted, which implies that the compressed body contracts into the
voids present in it. 

Again (3) increase, too, is thought to take always by means of void,
for nutriment is body, and it is impossible for two bodies to be together.
A proof of this they find also in what happens to ashes, which absorb
as much water as the empty vessel. 

The Pythagoreans, too, (4) held that void exists and that it enters
the heaven itself, which as it were inhales it, from the infinite
air. Further it is the void which distinguishes the natures of things,
as if it were like what separates and distinguishes the terms of a
series. This holds primarily in the numbers, for the void distinguishes
their nature. 

These, then, and so many, are the main grounds on which people have
argued for and against the existence of the void. 

Part 7

As a step towards settling which view is true, we must determine the
meaning of the name. 

The void is thought to be place with nothing in it. The reason for
this is that people take what exists to be body, and hold that while
every body is in place, void is place in which there is no body, so
that where there is no body, there must be void. 

Every body, again, they suppose to be tangible; and of this nature
is whatever has weight or lightness. 

Hence, by a syllogism, what has nothing heavy or light in it, is void.

This result, then, as I have said, is reached by syllogism. It would
be absurd to suppose that the point is void; for the void must be
place which has in it an interval in tangible body. 

But at all events we observe then that in one way the void is described
as what is not full of body perceptible to touch; and what has heaviness
and lightness is perceptible to touch. So we would raise the question:
what would they say of an interval that has colour or sound-is it
void or not? Clearly they would reply that if it could receive what
is tangible it was void, and if not, not. 

In another way void is that in which there is no 'this' or corporeal
substance. So some say that the void is the matter of the body (they
identify the place, too, with this), and in this they speak incorrectly;
for the matter is not separable from the things, but they are inquiring
about the void as about something separable. 

Since we have determined the nature of place, and void must, if it
exists, be place deprived of body, and we have stated both in what
sense place exists and in what sense it does not, it is plain that
on this showing void does not exist, either unseparated or separated;
the void is meant to be, not body but rather an interval in body.
This is why the void is thought to be something, viz. because place
is, and for the same reasons. For the fact of motion in respect of
place comes to the aid both of those who maintain that place is something
over and above the bodies that come to occupy it, and of those who
maintain that the void is something. They state that the void is the
condition of movement in the sense of that in which movement takes
place; and this would be the kind of thing that some say place is.

But there is no necessity for there being a void if there is movement.
It is not in the least needed as a condition of movement in general,
for a reason which, incidentally, escaped Melissus; viz. that the
full can suffer qualitative change. 

But not even movement in respect of place involves a void; for bodies
may simultaneously make room for one another, though there is no interval
separate and apart from the bodies that are in movement. And this
is plain even in the rotation of continuous things, as in that of
liquids. 

And things can also be compressed not into a void but because they
squeeze out what is contained in them (as, for instance, when water
is compressed the air within it is squeezed out); and things can increase
in size not only by the entrance of something but also by qualitative
change; e.g. if water were to be transformed into air. 

In general, both the argument about increase of size and that about
water poured on to the ashes get in their own way. For either not
any and every part of the body is increased, or bodies may be increased
otherwise than by the addition of body, or there may be two bodies
in the same place (in which case they are claiming to solve a quite
general difficulty, but are not proving the existence of void), or
the whole body must be void, if it is increased in every part and
is increased by means of void. The same argument applies to the ashes.

It is evident, then, that it is easy to refute the arguments by which
they prove the existence of the void. 

Part 8

Let us explain again that there is no void existing separately, as
some maintain. If each of the simple bodies has a natural locomotion,
e.g. fire upward and earth downward and towards the middle of the
universe, it is clear that it cannot be the void that is the condition
of locomotion. What, then, will the void be the condition of? It is
thought to be the condition of movement in respect of place, and it
is not the condition of this. 

Again, if void is a sort of place deprived of body, when there is
a void where will a body placed in it move to? It certainly cannot
move into the whole of the void. The same argument applies as against
those who think that place is something separate, into which things
are carried; viz. how will what is placed in it move, or rest? Much
the same argument will apply to the void as to the 'up' and 'down'
in place, as is natural enough since those who maintain the existence
of the void make it a place. 

And in what way will things be present either in place-or in the void?
For the expected result does not take place when a body is placed
as a whole in a place conceived of as separate and permanent; for
a part of it, unless it be placed apart, will not be in a place but
in the whole. Further, if separate place does not exist, neither will
void. 

If people say that the void must exist, as being necessary if there
is to be movement, what rather turns out to be the case, if one the
matter, is the opposite, that not a single thing can be moved if there
is a void; for as with those who for a like reason say the earth is
at rest, so, too, in the void things must be at rest; for there is
no place to which things can move more or less than to another; since
the void in so far as it is void admits no difference. 

The second reason is this: all movement is either compulsory or according
to nature, and if there is compulsory movement there must also be
natural (for compulsory movement is contrary to nature, and movement
contrary to nature is posterior to that according to nature, so that
if each of the natural bodies has not a natural movement, none of
the other movements can exist); but how can there be natural movement
if there is no difference throughout the void or the infinite? For
in so far as it is infinite, there will be no up or down or middle,
and in so far as it is a void, up differs no whit from down; for as
there is no difference in what is nothing, there is none in the void
(for the void seems to be a non-existent and a privation of being),
but natural locomotion seems to be differentiated, so that the things
that exist by nature must be differentiated. Either, then, nothing
has a natural locomotion, or else there is no void. 

Further, in point of fact things that are thrown move though that
which gave them their impulse is not touching them, either by reason
of mutual replacement, as some maintain, or because the air that has
been pushed pushes them with a movement quicker than the natural locomotion
of the projectile wherewith it moves to its proper place. But in a
void none of these things can take place, nor can anything be moved
save as that which is carried is moved. 

Further, no one could say why a thing once set in motion should stop
anywhere; for why should it stop here rather than here? So that a
thing will either be at rest or must be moved ad infinitum, unless
something more powerful get in its way. 

Further, things are now thought to move into the void because it yields;
but in a void this quality is present equally everywhere, so that
things should move in all directions. 

Further, the truth of what we assert is plain from the following considerations.
We see the same weight or body moving faster than another for two
reasons, either because there is a difference in what it moves through,
as between water, air, and earth, or because, other things being equal,
the moving body differs from the other owing to excess of weight or
of lightness. 

Now the medium causes a difference because it impedes the moving thing,
most of all if it is moving in the opposite direction, but in a secondary
degree even if it is at rest; and especially a medium that is not
easily divided, i.e. a medium that is somewhat dense. A, then, will
move through B in time G, and through D, which is thinner, in time
E (if the length of B is egual to D), in proportion to the density
of the hindering body. For let B be water and D air; then by so much
as air is thinner and more incorporeal than water, A will move through
D faster than through B. Let the speed have the same ratio to the
speed, then, that air has to water. Then if air is twice as thin,
the body will traverse B in twice the time that it does D, and the
time G will be twice the time E. And always, by so much as the medium
is more incorporeal and less resistant and more easily divided, the
faster will be the movement. 

Now there is no ratio in which the void is exceeded by body, as there
is no ratio of 0 to a number. For if 4 exceeds 3 by 1, and 2 by more
than 1, and 1 by still more than it exceeds 2, still there is no ratio
by which it exceeds 0; for that which exceeds must be divisible into
the excess + that which is exceeded, so that will be what it exceeds
0 by + 0. For this reason, too, a line does not exceed a point unless
it is composed of points! Similarly the void can bear no ratio to
the full, and therefore neither can movement through the one to movement
through the other, but if a thing moves through the thickest medium
such and such a distance in such and such a time, it moves through
the void with a speed beyond any ratio. For let Z be void, equal in
magnitude to B and to D. Then if A is to traverse and move through
it in a certain time, H, a time less than E, however, the void will
bear this ratio to the full. But in a time equal to H, A will traverse
the part O of A. And it will surely also traverse in that time any
substance Z which exceeds air in thickness in the ratio which the
time E bears to the time H. For if the body Z be as much thinner than
D as E exceeds H, A, if it moves through Z, will traverse it in a
time inverse to the speed of the movement, i.e. in a time equal to
H. If, then, there is no body in Z, A will traverse Z still more quickly.
But we supposed that its traverse of Z when Z was void occupied the
time H. So that it will traverse Z in an equal time whether Z be full
or void. But this is impossible. It is plain, then, that if there
is a time in which it will move through any part of the void, this
impossible result will follow: it will be found to traverse a certain
distance, whether this be full or void, in an equal time; for there
will be some body which is in the same ratio to the other body as
the time is to the time. 

To sum the matter up, the cause of this result is obvious, viz. that
between any two movements there is a ratio (for they occupy time,
and there is a ratio between any two times, so long as both are finite),
but there is no ratio of void to full. 

These are the consequences that result from a difference in the media;
the following depend upon an excess of one moving body over another.
We see that bodies which have a greater impulse either of weight or
of lightness, if they are alike in other respects, move faster over
an equal space, and in the ratio which their magnitudes bear to each
other. Therefore they will also move through the void with this ratio
of speed. But that is impossible; for why should one move faster?
(In moving through plena it must be so; for the greater divides them
faster by its force. For a moving thing cleaves the medium either
by its shape, or by the impulse which the body that is carried along
or is projected possesses.) Therefore all will possess equal velocity.
But this is impossible. 

It is evident from what has been said, then, that, if there is a void,
a result follows which is the very opposite of the reason for which
those who believe in a void set it up. They think that if movement
in respect of place is to exist, the void cannot exist, separated
all by itself; but this is the same as to say that place is a separate
cavity; and this has already been stated to be impossible.

But even if we consider it on its own merits the so-called vacuum
will be found to be really vacuous. For as, if one puts a cube in
water, an amount of water equal to the cube will be displaced; so
too in air; but the effect is imperceptible to sense. And indeed always
in the case of any body that can be displaced, must, if it is not
compressed, be displaced in the direction in which it is its nature
to be displaced-always either down, if its locomotion is downwards
as in the case of earth, or up, if it is fire, or in both directions-whatever
be the nature of the inserted body. Now in the void this is impossible;
for it is not body; the void must have penetrated the cube to a distance
equal to that which this portion of void formerly occupied in the
void, just as if the water or air had not been displaced by the wooden
cube, but had penetrated right through it. 

But the cube also has a magnitude equal to that occupied by the void;
a magnitude which, if it is also hot or cold, or heavy or light, is
none the less different in essence from all its attributes, even if
it is not separable from them; I mean the volume of the wooden cube.
So that even if it were separated from everything else and were neither
heavy nor light, it will occupy an equal amount of void, and fill
the same place, as the part of place or of the void equal to itself.
How then will the body of the cube differ from the void or place that
is equal to it? And if there can be two such things, why cannot there
be any number coinciding? 

This, then, is one absurd and impossible implication of the theory.
It is also evident that the cube will have this same volume even if
it is displaced, which is an attribute possessed by all other bodies
also. Therefore if this differs in no respect from its place, why
need we assume a place for bodies over and above the volume of each,
if their volume be conceived of as free from attributes? It contributes
nothing to the situation if there is an equal interval attached to
it as well. [Further it ought to be clear by the study of moving things
what sort of thing void is. But in fact it is found nowhere in the
world. For air is something, though it does not seem to be so-nor,
for that matter, would water, if fishes were made of iron; for the
discrimination of the tangible is by touch.] 

It is clear, then, from these considerations that there is no separate
void. 

Part 9

There are some who think that the existence of rarity and density
shows that there is a void. If rarity and density do not exist, they
say, neither can things contract and be compressed. But if this were
not to take place, either there would be no movement at all, or the
universe would bulge, as Xuthus said, or air and water must always
change into equal amounts (e.g. if air has been made out of a cupful
of water, at the same time out of an equal amount of air a cupful
of water must have been made), or void must necessarily exist; for
compression and expansion cannot take place otherwise. 

Now, if they mean by the rare that which has many voids existing separately,
it is plain that if void cannot exist separate any more than a place
can exist with an extension all to itself, neither can the rare exist
in this sense. But if they mean that there is void, not separately
existent, but still present in the rare, this is less impossible,
yet, first, the void turns out not to be a condition of all movement,
but only of movement upwards (for the rare is light, which is the
reason why they say fire is rare); second, the void turns out to be
a condition of movement not as that in which it takes place, but in
that the void carries things up as skins by being carried up themselves
carry up what is continuous with them. Yet how can void have a local
movement or a place? For thus that into which void moves is till then
void of a void. 

Again, how will they explain, in the case of what is heavy, its movement
downwards? And it is plain that if the rarer and more void a thing
is the quicker it will move upwards, if it were completely void it
would move with a maximum speed! But perhaps even this is impossible,
that it should move at all; the same reason which showed that in the
void all things are incapable of moving shows that the void cannot
move, viz. the fact that the speeds are incomparable. 

Since we deny that a void exists, but for the rest the problem has
been truly stated, that either there will be no movement, if there
is not to be condensation and rarefaction, or the universe will bulge,
or a transformation of water into air will always be balanced by an
equal transformation of air into water (for it is clear that the air
produced from water is bulkier than the water): it is necessary therefore,
if compression does not exist, either that the next portion will be
pushed outwards and make the outermost part bulge, or that somewhere
else there must be an equal amount of water produced out of air, so
that the entire bulk of the whole may be equal, or that nothing moves.
For when anything is displaced this will always happen, unless it
comes round in a circle; but locomotion is not always circular, but
sometimes in a straight line. 

These then are the reasons for which they might say that there is
a void; our statement is based on the assumption that there is a single
matter for contraries, hot and cold and the other natural contrarieties,
and that what exists actually is produced from a potential existent,
and that matter is not separable from the contraries but its being
is different, and that a single matter may serve for colour and heat
and cold. 

The same matter also serves for both a large and a small body. This
is evident; for when air is produced from water, the same matter has
become something different, not by acquiring an addition to it, but
has become actually what it was potentially, and, again, water is
produced from air in the same way, the change being sometimes from
smallness to greatness, and sometimes from greatness to smallness.
Similarly, therefore, if air which is large in extent comes to have
a smaller volume, or becomes greater from being smaller, it is the
matter which is potentially both that comes to be each of the two.

For as the same matter becomes hot from being cold, and cold from
being hot, because it was potentially both, so too from hot it can
become more hot, though nothing in the matter has become hot that
was not hot when the thing was less hot; just as, if the arc or curve
of a greater circle becomes that of a smaller, whether it remains
the same or becomes a different curve, convexity has not come to exist
in anything that was not convex but straight (for differences of degree
do not depend on an intermission of the quality); nor can we get any
portion of a flame, in which both heat and whiteness are not present.
So too, then, is the earlier heat related to the later. So that the
greatness and smallness, also, of the sensible volume are extended,
not by the matter's acquiring anything new, but because the matter
is potentially matter for both states; so that the same thing is dense
and rare, and the two qualities have one matter. 

The dense is heavy, and the rare is light. [Again, as the arc of a
circle when contracted into a smaller space does not acquire a new
part which is convex, but what was there has been contracted; and
as any part of fire that one takes will be hot; so, too, it is all
a question of contraction and expansion of the same matter.] There
are two types in each case, both in the dense and in the rare; for
both the heavy and the hard are thought to be dense, and contrariwise
both the light and the soft are rare; and weight and hardness fail
to coincide in the case of lead and iron. 

From what has been said it is evident, then, that void does not exist
either separate (either absolutely separate or as a separate element
in the rare) or potentially, unless one is willing to call the condition
of movement void, whatever it may be. At that rate the matter of the
heavy and the light, qua matter of them, would be the void; for the
dense and the rare are productive of locomotion in virtue of this
contrariety, and in virtue of their hardness and softness productive
of passivity and impassivity, i.e. not of locomotion but rather of
qualitative change. 

So much, then, for the discussion of the void, and of the sense in
which it exists and the sense in which it does not exist.

Part 10

Next for discussion after the subjects mentioned is Time. The best
plan will be to begin by working out the difficulties connected with
it, making use of the current arguments. First, does it belong to
the class of things that exist or to that of things that do not exist?
Then secondly, what is its nature? To start, then: the following considerations
would make one suspect that it either does not exist at all or barely,
and in an obscure way. One part of it has been and is not, while the
other is going to be and is not yet. Yet time-both infinite time and
any time you like to take-is made up of these. One would naturally
suppose that what is made up of things which do not exist could have
no share in reality. 

Further, if a divisible thing is to exist, it is necessary that, when
it exists, all or some of its parts must exist. But of time some parts
have been, while others have to be, and no part of it is though it
is divisible. For what is 'now' is not a part: a part is a measure
of the whole, which must be made up of parts. Time, on the other hand,
is not held to be made up of 'nows'. 

Again, the 'now' which seems to bound the past and the future-does
it always remain one and the same or is it always other and other?
It is hard to say. 

(1) If it is always different and different, and if none of the parts
in time which are other and other are simultaneous (unless the one
contains and the other is contained, as the shorter time is by the
longer), and if the 'now' which is not, but formerly was, must have
ceased-to-be at some time, the 'nows' too cannot be simultaneous with
one another, but the prior 'now' must always have ceased-to-be. But
the prior 'now' cannot have ceased-to-be in itself (since it then
existed); yet it cannot have ceased-to-be in another 'now'. For we
may lay it down that one 'now' cannot be next to another, any more
than point to point. If then it did not cease-to-be in the next 'now'
but in another, it would exist simultaneously with the innumerable
'nows' between the two-which is impossible. 

Yes, but (2) neither is it possible for the 'now' to remain always
the same. No determinate divisible thing has a single termination,
whether it is continuously extended in one or in more than one dimension:
but the 'now' is a termination, and it is possible to cut off a determinate
time. Further, if coincidence in time (i.e. being neither prior nor
posterior) means to be 'in one and the same "now"', then, if both
what is before and what is after are in this same 'now', things which
happened ten thousand years ago would be simultaneous with what has
happened to-day, and nothing would be before or after anything else.

This may serve as a statement of the difficulties about the attributes
of time. 

As to what time is or what is its nature, the traditional accounts
give us as little light as the preliminary problems which we have
worked through. 

Some assert that it is (1) the movement of the whole, others that
it is (2) the sphere itself. 

(1) Yet part, too, of the revolution is a time, but it certainly is
not a revolution: for what is taken is part of a revolution, not a
revolution. Besides, if there were more heavens than one, the movement
of any of them equally would be time, so that there would be many
times at the same time. 

(2) Those who said that time is the sphere of the whole thought so,
no doubt, on the ground that all things are in time and all things
are in the sphere of the whole. The view is too naive for it to be
worth while to consider the impossibilities implied in it.

But as time is most usually supposed to be (3) motion and a kind of
change, we must consider this view. 

Now (a) the change or movement of each thing is only in the thing
which changes or where the thing itself which moves or changes may
chance to be. But time is present equally everywhere and with all
things. 

Again, (b) change is always faster or slower, whereas time is not:
for 'fast' and 'slow' are defined by time-'fast' is what moves much
in a short time, 'slow' what moves little in a long time; but time
is not defined by time, by being either a certain amount or a certain
kind of it. 

Clearly then it is not movement. (We need not distinguish at present
between 'movement' and 'change'.) 

Part 11

But neither does time exist without change; for when the state of
our own minds does not change at all, or we have not noticed its changing,
we do not realize that time has elapsed, any more than those who are
fabled to sleep among the heroes in Sardinia do when they are awakened;
for they connect the earlier 'now' with the later and make them one,
cutting out the interval because of their failure to notice it. So,
just as, if the 'now' were not different but one and the same, there
would not have been time, so too when its difference escapes our notice
the interval does not seem to be time. If, then, the non-realization
of the existence of time happens to us when we do not distinguish
any change, but the soul seems to stay in one indivisible state, and
when we perceive and distinguish we say time has elapsed, evidently
time is not independent of movement and change. It is evident, then,
that time is neither movement nor independent of movement.

We must take this as our starting-point and try to discover-since
we wish to know what time is-what exactly it has to do with movement.

Now we perceive movement and time together: for even when it is dark
and we are not being affected through the body, if any movement takes
place in the mind we at once suppose that some time also has elapsed;
and not only that but also, when some time is thought to have passed,
some movement also along with it seems to have taken place. Hence
time is either movement or something that belongs to movement. Since
then it is not movement, it must be the other. 

But what is moved is moved from something to something, and all magnitude
is continuous. Therefore the movement goes with the magnitude. Because
the magnitude is continuous, the movement too must be continuous,
and if the movement, then the time; for the time that has passed is
always thought to be in proportion to the movement. 

The distinction of 'before' and 'after' holds primarily, then, in
place; and there in virtue of relative position. Since then 'before'
and 'after' hold in magnitude, they must hold also in movement, these
corresponding to those. But also in time the distinction of 'before'
and 'after' must hold, for time and movement always correspond with
each other. The 'before' and 'after' in motion is identical in substratum
with motion yet differs from it in definition, and is not identical
with motion. 

But we apprehend time only when we have marked motion, marking it
by 'before' and 'after'; and it is only when we have perceived 'before'
and 'after' in motion that we say that time has elapsed. Now we mark
them by judging that A and B are different, and that some third thing
is intermediate to them. When we think of the extremes as different
from the middle and the mind pronounces that the 'nows' are two, one
before and one after, it is then that we say that there is time, and
this that we say is time. For what is bounded by the 'now' is thought
to be time-we may assume this. 

When, therefore, we perceive the 'now' one, and neither as before
and after in a motion nor as an identity but in relation to a 'before'
and an 'after', no time is thought to have elapsed, because there
has been no motion either. On the other hand, when we do perceive
a 'before' and an 'after', then we say that there is time. For time
is just this-number of motion in respect of 'before' and 'after'.

Hence time is not movement, but only movement in so far as it admits
of enumeration. A proof of this: we discriminate the more or the less
by number, but more or less movement by time. Time then is a kind
of number. (Number, we must note, is used in two senses-both of what
is counted or the countable and also of that with which we count.
Time obviously is what is counted, not that with which we count: there
are different kinds of thing.) Just as motion is a perpetual succession,
so also is time. But every simultaneous time is self-identical; for
the 'now' as a subject is an identity, but it accepts different attributes.
The 'now' measures time, in so far as time involves the 'before and
after'. 

The 'now' in one sense is the same, in another it is not the same.
In so far as it is in succession, it is different (which is just what
its being was supposed to mean), but its substratum is an identity:
for motion, as was said, goes with magnitude, and time, as we maintain,
with motion. Similarly, then, there corresponds to the point the body
which is carried along, and by which we are aware of the motion and
of the 'before and after' involved in it. This is an identical substratum
(whether a point or a stone or something else of the kind), but it
has different attributes as the sophists assume that Coriscus' being
in the Lyceum is a different thing from Coriscus' being in the market-place.
And the body which is carried along is different, in so far as it
is at one time here and at another there. But the 'now' corresponds
to the body that is carried along, as time corresponds to the motion.
For it is by means of the body that is carried along that we become
aware of the 'before and after' the motion, and if we regard these
as countable we get the 'now'. Hence in these also the 'now' as substratum
remains the same (for it is what is before and after in movement),
but what is predicated of it is different; for it is in so far as
the 'before and after' is numerable that we get the 'now'. This is
what is most knowable: for, similarly, motion is known because of
that which is moved, locomotion because of that which is carried.
what is carried is a real thing, the movement is not. Thus what is
called 'now' in one sense is always the same; in another it is not
the same: for this is true also of what is carried. 

Clearly, too, if there were no time, there would be no 'now', and
vice versa. just as the moving body and its locomotion involve each
other mutually, so too do the number of the moving body and the number
of its locomotion. For the number of the locomotion is time, while
the 'now' corresponds to the moving body, and is like the unit of
number. 

Time, then, also is both made continuous by the 'now' and divided
at it. For here too there is a correspondence with the locomotion
and the moving body. For the motion or locomotion is made one by the
thing which is moved, because it is one-not because it is one in its
own nature (for there might be pauses in the movement of such a thing)-but
because it is one in definition: for this determines the movement
as 'before' and 'after'. Here, too there is a correspondence with
the point; for the point also both connects and terminates the length-it
is the beginning of one and the end of another. But when you take
it in this way, using the one point as two, a pause is necessary,
if the same point is to be the beginning and the end. The 'now' on
the other hand, since the body carried is moving, is always different.

Hence time is not number in the sense in which there is 'number' of
the same point because it is beginning and end, but rather as the
extremities of a line form a number, and not as the parts of the line
do so, both for the reason given (for we can use the middle point
as two, so that on that analogy time might stand still), and further
because obviously the 'now' is no part of time nor the section any
part of the movement, any more than the points are parts of the line-for
it is two lines that are parts of one line. 

In so far then as the 'now' is a boundary, it is not time, but an
attribute of it; in so far as it numbers, it is number; for boundaries
belong only to that which they bound, but number (e.g. ten) is the
number of these horses, and belongs also elsewhere. 

It is clear, then, that time is 'number of movement in respect of
the before and after', and is continuous since it is an attribute
of what is continuous. 

Part 12

The smallest number, in the strict sense of the word 'number', is
two. But of number as concrete, sometimes there is a minimum, sometimes
not: e.g. of a 'line', the smallest in respect of multiplicity is
two (or, if you like, one), but in respect of size there is no minimum;
for every line is divided ad infinitum. Hence it is so with time.
In respect of number the minimum is one (or two); in point of extent
there is no minimum. 

It is clear, too, that time is not described as fast or slow, but
as many or few and as long or short. For as continuous it is long
or short and as a number many or few, but it is not fast or slow-any
more than any number with which we number is fast or slow.

Further, there is the same time everywhere at once, but not the same
time before and after, for while the present change is one, the change
which has happened and that which will happen are different. Time
is not number with which we count, but the number of things which
are counted, and this according as it occurs before or after is always
different, for the 'nows' are different. And the number of a hundred
horses and a hundred men is the same, but the things numbered are
different-the horses from the men. Further, as a movement can be one
and the same again and again, so too can time, e.g. a year or a spring
or an autumn. 

Not only do we measure the movement by the time, but also the time
by the movement, because they define each other. The time marks the
movement, since it is its number, and the movement the time. We describe
the time as much or little, measuring it by the movement, just as
we know the number by what is numbered, e.g. the number of the horses
by one horse as the unit. For we know how many horses there are by
the use of the number; and again by using the one horse as unit we
know the number of the horses itself. So it is with the time and the
movement; for we measure the movement by the time and vice versa.
It is natural that this should happen; for the movement goes with
the distance and the time with the movement, because they are quanta
and continuous and divisible. The movement has these attributes because
the distance is of this nature, and the time has them because of the
movement. And we measure both the distance by the movement and the
movement by the distance; for we say that the road is long, if the
journey is long, and that this is long, if the road is long-the time,
too, if the movement, and the movement, if the time. 

Time is a measure of motion and of being moved, and it measures the
motion by determining a motion which will measure exactly the whole
motion, as the cubit does the length by determining an amount which
will measure out the whole. Further 'to be in time' means for movement,
that both it and its essence are measured by time (for simultaneously
it measures both the movement and its essence, and this is what being
in time means for it, that its essence should be measured).

Clearly then 'to be in time' has the same meaning for other things
also, namely, that their being should be measured by time. 'To be
in time' is one of two things: (1) to exist when time exists, (2)
as we say of some things that they are 'in number'. The latter means
either what is a part or mode of number-in general, something which
belongs to number-or that things have a number. 

Now, since time is number, the 'now' and the 'before' and the like
are in time, just as 'unit' and 'odd' and 'even' are in number, i.e.
in the sense that the one set belongs to number, the other to time.
But things are in time as they are in number. If this is so, they
are contained by time as things in place are contained by place.

Plainly, too, to be in time does not mean to co-exist with time, any
more than to be in motion or in place means to co-exist with motion
or place. For if 'to be in something' is to mean this, then all things
will be in anything, and the heaven will be in a grain; for when the
grain is, then also is the heaven. But this is a merely incidental
conjunction, whereas the other is necessarily involved: that which
is in time necessarily involves that there is time when it is, and
that which is in motion that there is motion when it is.

Since what is 'in time' is so in the same sense as what is in number
is so, a time greater than everything in time can be found. So it
is necessary that all the things in time should be contained by time,
just like other things also which are 'in anything', e.g. the things
'in place' by place. 

A thing, then, will be affected by time, just as we are accustomed
to say that time wastes things away, and that all things grow old
through time, and that there is oblivion owing to the lapse of time,
but we do not say the same of getting to know or of becoming young
or fair. For time is by its nature the cause rather of decay, since
it is the number of change, and change removes what is. 

Hence, plainly, things which are always are not, as such, in time,
for they are not contained time, nor is their being measured by time.
A proof of this is that none of them is affected by time, which indicates
that they are not in time. 

Since time is the measure of motion, it will be the measure of rest
too-indirectly. For all rest is in time. For it does not follow that
what is in time is moved, though what is in motion is necessarily
moved. For time is not motion, but 'number of motion': and what is
at rest, also, can be in the number of motion. Not everything that
is not in motion can be said to be 'at rest'-but only that which can
be moved, though it actually is not moved, as was said above.

'To be in number' means that there is a number of the thing, and that
its being is measured by the number in which it is. Hence if a thing
is 'in time' it will be measured by time. But time will measure what
is moved and what is at rest, the one qua moved, the other qua at
rest; for it will measure their motion and rest respectively.

Hence what is moved will not be measurable by the time simply in so
far as it has quantity, but in so far as its motion has quantity.
Thus none of the things which are neither moved nor at rest are in
time: for 'to be in time' is 'to be measured by time', while time
is the measure of motion and rest. 

Plainly, then, neither will everything that does not exist be in time,
i.e. those non-existent things that cannot exist, as the diagonal
cannot be commensurate with the side. 

Generally, if time is directly the measure of motion and indirectly
of other things, it is clear that a thing whose existence is measured
by it will have its existence in rest or motion. Those things therefore
which are subject to perishing and becoming-generally, those which
at one time exist, at another do not-are necessarily in time: for
there is a greater time which will extend both beyond their existence
and beyond the time which measures their existence. Of things which
do not exist but are contained by time some were, e.g. Homer once
was, some will be, e.g. a future event; this depends on the direction
in which time contains them; if on both, they have both modes of existence.
As to such things as it does not contain in any way, they neither
were nor are nor will be. These are those nonexistents whose opposites
always are, as the incommensurability of the diagonal always is-and
this will not be in time. Nor will the commensurability, therefore;
hence this eternally is not, because it is contrary to what eternally
is. A thing whose contrary is not eternal can be and not be, and it
is of such things that there is coming to be and passing away.

Part 13

The 'now' is the link of time, as has been said (for it connects past
and future time), and it is a limit of time (for it is the beginning
of the one and the end of the other). But this is not obvious as it
is with the point, which is fixed. It divides potentially, and in
so far as it is dividing the 'now' is always different, but in so
far as it connects it is always the same, as it is with mathematical
lines. For the intellect it is not always one and the same point,
since it is other and other when one divides the line; but in so far
as it is one, it is the same in every respect. 

So the 'now' also is in one way a potential dividing of time, in another
the termination of both parts, and their unity. And the dividing and
the uniting are the same thing and in the same reference, but in essence
they are not the same. 

So one kind of 'now' is described in this way: another is when the
time is near this kind of 'now'. 'He will come now' because he will
come to-day; 'he has come now' because he came to-day. But the things
in the Iliad have not happened 'now', nor is the flood 'now'-not that
the time from now to them is not continuous, but because they are
not near. 

'At some time' means a time determined in relation to the first of
the two types of 'now', e.g. 'at some time' Troy was taken, and 'at
some time' there will be a flood; for it must be determined with reference
to the 'now'. There will thus be a determinate time from this 'now'
to that, and there was such in reference to the past event. But if
there be no time which is not 'sometime', every time will be determined.

Will time then fail? Surely not, if motion always exists. Is time
then always different or does the same time recur? Clearly time is,
in the same way as motion is. For if one and the same motion sometimes
recurs, it will be one and the same time, and if not, not.

Since the 'now' is an end and a beginning of time, not of the same
time however, but the end of that which is past and the beginning
of that which is to come, it follows that, as the circle has its convexity
and its concavity, in a sense, in the same thing, so time is always
at a beginning and at an end. And for this reason it seems to be always
different; for the 'now' is not the beginning and the end of the same
thing; if it were, it would be at the same time and in the same respect
two opposites. And time will not fail; for it is always at a beginning.

'Presently' or 'just' refers to the part of future time which is near
the indivisible present 'now' ('When do you walk? 'Presently', because
the time in which he is going to do so is near), and to the part of
past time which is not far from the 'now' ('When do you walk?' 'I
have just been walking'). But to say that Troy has just been taken-we
do not say that, because it is too far from the 'now'. 'Lately', too,
refers to the part of past time which is near the present 'now'. 'When
did you go?' 'Lately', if the time is near the existing now. 'Long
ago' refers to the distant past. 

'Suddenly' refers to what has departed from its former condition in
a time imperceptible because of its smallness; but it is the nature
of all change to alter things from their former condition. In time
all things come into being and pass away; for which reason some called
it the wisest of all things, but the Pythagorean Paron called it the
most stupid, because in it we also forget; and his was the truer view.
It is clear then that it must be in itself, as we said before, the
condition of destruction rather than of coming into being (for change,
in itself, makes things depart from their former condition), and only
incidentally of coming into being, and of being. A sufficient evidence
of this is that nothing comes into being without itself moving somehow
and acting, but a thing can be destroyed even if it does not move
at all. And this is what, as a rule, we chiefly mean by a thing's
being destroyed by time. Still, time does not work even this change;
even this sort of change takes place incidentally in time.

We have stated, then, that time exists and what it is, and in how
many senses we speak of the 'now', and what 'at some time', 'lately',
'presently' or 'just', 'long ago', and 'suddenly' mean. 

Part 14

These distinctions having been drawn, it is evident that every change
and everything that moves is in time; for the distinction of faster
and slower exists in reference to all change, since it is found in
every instance. In the phrase 'moving faster' I refer to that which
changes before another into the condition in question, when it moves
over the same interval and with a regular movement; e.g. in the case
of locomotion, if both things move along the circumference of a circle,
or both along a straight line; and similarly in all other cases. But
what is before is in time; for we say 'before' and 'after' with reference
to the distance from the 'now', and the 'now' is the boundary of the
past and the future; so that since 'nows' are in time, the before
and the after will be in time too; for in that in which the 'now'
is, the distance from the 'now' will also be. But 'before' is used
contrariwise with reference to past and to future time; for in the
past we call 'before' what is farther from the 'now', and 'after'
what is nearer, but in the future we call the nearer 'before' and
the farther 'after'. So that since the 'before' is in time, and every
movement involves a 'before', evidently every change and every movement
is in time. 

It is also worth considering how time can be related to the soul;
and why time is thought to be in everything, both in earth and in
sea and in heaven. Is because it is an attribute, or state, or movement
(since it is the number of movement) and all these things are movable
(for they are all in place), and time and movement are together, both
in respect of potentiality and in respect of actuality? 

Whether if soul did not exist time would exist or not, is a question
that may fairly be asked; for if there cannot be some one to count
there cannot be anything that can be counted, so that evidently there
cannot be number; for number is either what has been, or what can
be, counted. But if nothing but soul, or in soul reason, is qualified
to count, there would not be time unless there were soul, but only
that of which time is an attribute, i.e. if movement can exist without
soul, and the before and after are attributes of movement, and time
is these qua numerable. 

One might also raise the question what sort of movement time is the
number of. Must we not say 'of any kind'? For things both come into
being in time and pass away, and grow, and are altered in time, and
are moved locally; thus it is of each movement qua movement that time
is the number. And so it is simply the number of continuous movement,
not of any particular kind of it. 

But other things as well may have been moved now, and there would
be a number of each of the two movements. Is there another time, then,
and will there be two equal times at once? Surely not. For a time
that is both equal and simultaneous is one and the same time, and
even those that are not simultaneous are one in kind; for if there
were dogs, and horses, and seven of each, it would be the same number.
So, too, movements that have simultaneous limits have the same time,
yet the one may in fact be fast and the other not, and one may be
locomotion and the other alteration; still the time of the two changes
is the same if their number also is equal and simultaneous; and for
this reason, while the movements are different and separate, the time
is everywhere the same, because the number of equal and simultaneous
movements is everywhere one and the same. 

Now there is such a thing as locomotion, and in locomotion there is
included circular movement, and everything is measured by some one
thing homogeneous with it, units by a unit, horses by a horse, and
similarly times by some definite time, and, as we said, time is measured
by motion as well as motion by time (this being so because by a motion
definite in time the quantity both of the motion and of the time is
measured): if, then, what is first is the measure of everything homogeneous
with it, regular circular motion is above all else the measure, because
the number of this is the best known. Now neither alteration nor increase
nor coming into being can be regular, but locomotion can be. This
also is why time is thought to be the movement of the sphere, viz.
because the other movements are measured by this, and time by this
movement. 

This also explains the common saying that human affairs form a circle,
and that there is a circle in all other things that have a natural
movement and coming into being and passing away. This is because all
other things are discriminated by time, and end and begin as though
conforming to a cycle; for even time itself is thought to be a circle.
And this opinion again is held because time is the measure of this
kind of locomotion and is itself measured by such. So that to say
that the things that come into being form a circle is to say that
there is a circle of time; and this is to say that it is measured
by the circular movement; for apart from the measure nothing else
to be measured is observed; the whole is just a plurality of measures.

It is said rightly, too, that the number of the sheep and of the dogs
is the same number if the two numbers are equal, but not the same
decad or the same ten; just as the equilateral and the scalene are
not the same triangle, yet they are the same figure, because they
are both triangles. For things are called the same so-and-so if they
do not differ by a differentia of that thing, but not if they do;
e.g. triangle differs from triangle by a differentia of triangle,
therefore they are different triangles; but they do not differ by
a differentia of figure, but are in one and the same division of it.
For a figure of the one kind is a circle and a figure of another kind
of triangle, and a triangle of one kind is equilateral and a triangle
of another kind scalene. They are the same figure, then, that, triangle,
but not the same triangle. Therefore the number of two groups also-is
the same number (for their number does not differ by a differentia
of number), but it is not the same decad; for the things of which
it is asserted differ; one group are dogs, and the other horses.

We have now discussed time-both time itself and the matters appropriate
to the consideration of it. 

----------------------------------------------------------------------

BOOK V

Part 1 

Everything which changes does so in one of three senses. It may change
(1) accidentally, as for instance when we say that something musical
walks, that which walks being something in which aptitude for music
is an accident. Again (2) a thing is said without qualification to
change because something belonging to it changes, i.e. in statements
which refer to part of the thing in question: thus the body is restored
to health because the eye or the chest, that is to say a part of the
whole body, is restored to health. And above all there is (3) the
case of a thing which is in motion neither accidentally nor in respect
of something else belonging to it, but in virtue of being itself directly
in motion. Here we have a thing which is essentially movable: and
that which is so is a different thing according to the particular
variety of motion: for instance it may be a thing capable of alteration:
and within the sphere of alteration it is again a different thing
according as it is capable of being restored to health or capable
of being heated. And there are the same distinctions in the case of
the mover: (1) one thing causes motion accidentally, (2) another partially
(because something belonging to it causes motion), (3) another of
itself directly, as, for instance, the physician heals, the hand strikes.
We have, then, the following factors: (a) on the one hand that which
directly causes motion, and (b) on the other hand that which is in
motion: further, we have (c) that in which motion takes place, namely
time, and (distinct from these three, d) that from which and (e)
that to which it proceeds: for every motion proceeds from something
and to something, that which is directly in motion being distinct
from that to which it is in motion and that from which it is in motion:
for instance, we may take the three things 'wood', 'hot', and 'cold',
of which the first is that which is in motion, the second is that
to which the motion proceeds, and the third is that from which it
proceeds. This being so, it is clear that the motion is in the wood,
not in its form: for the motion is neither caused nor experienced
by the form or the place or the quantity. So we are left with a mover,
a moved, and a goal of motion. I do not include the starting-point
of motion: for it is the goal rather than the starting-point of motion
that gives its name to a particular process of change. Thus 'perishing'
is change to not-being, though it is also true that that that which
perishes changes from being: and 'becoming' is change to being, though
it is also change from not-being. 

Now a definition of motion has been given above, from which it will
be seen that every goal of motion, whether it be a form, an affection,
or a place, is immovable, as, for instance, knowledge and heat. Here,
however, a difficulty may be raised. Affections, it may be said, are
motions, and whiteness is an affection: thus there may be change to
a motion. To this we may reply that it is not whiteness but whitening
that is a motion. Here also the same distinctions are to be observed:
a goal of motion may be so accidentally, or partially and with reference
to something other than itself, or directly and with no reference
to anything else: for instance, a thing which is becoming white changes
accidentally to an object of thought, the colour being only accidentally
the object of thought; it changes to colour, because white is a part
of colour, or to Europe, because Athens is a part of Europe; but it
changes essentially to white colour. It is now clear in what sense
a thing is in motion essentially, accidentally, or in respect of something
other than itself, and in what sense the phrase 'itself directly'
is used in the case both of the mover and of the moved: and it is
also clear that the motion is not in the form but in that which is
in motion, that is to say 'the movable in activity'. Now accidental
change we may leave out of account: for it is to be found in everything,
at any time, and in any respect. Change which is not accidental on
the other hand is not to be found in everything, but only in contraries,
in things intermediate contraries, and in contradictories, as may
be proved by induction. An intermediate may be a starting-point of
change, since for the purposes of the change it serves as contrary
to either of two contraries: for the intermediate is in a sense the
extremes. Hence we speak of the intermediate as in a sense a contrary
relatively to the extremes and of either extreme as a contrary relatively
to the intermediate: for instance, the central note is low relatively-to
the highest and high relatively to the lowest, and grey is light relatively
to black and dark relatively to white. 

And since every change is from something to something-as the word
itself (metabole) indicates, implying something 'after' (meta) something
else, that is to say something earlier and something later-that which
changes must change in one of four ways: from subject to subject,
from subject to nonsubject, from non-subject to subject, or from non-subject
to non-subject, where by 'subject' I mean what is affirmatively expressed.
So it follows necessarily from what has been said above that there
are only three kinds of change, that from subject to subject, that
from subject to non-subject, and that from non-subject to subject:
for the fourth conceivable kind, that from non-subject to nonsubject,
is not change, as in that case there is no opposition either of contraries
or of contradictories. 

Now change from non-subject to subject, the relation being that of
contradiction, is 'coming to be'-'unqualified coming to be' when the
change takes place in an unqualified way, 'particular coming to be'
when the change is change in a particular character: for instance,
a change from not-white to white is a coming to be of the particular
thing, white, while change from unqualified not-being to being is
coming to be in an unqualified way, in respect of which we say that
a thing 'comes to be' without qualification, not that it 'comes to
be' some particular thing. Change from subject to non-subject is 'perishing'-'unqualified
perishing' when the change is from being to not-being, 'particular
perishing' when the change is to the opposite negation, the distinction
being the same as that made in the case of coming to be.

Now the expression 'not-being' is used in several senses: and there
can be motion neither of that which 'is not' in respect of the affirmation
or negation of a predicate, nor of that which 'is not' in the sense
that it only potentially 'is', that is to say the opposite of that
which actually 'is' in an unqualified sense: for although that which
is 'not-white' or 'not-good' may nevertheless he in motion accidentally
(for example that which is 'not-white' might be a man), yet that which
is without qualification 'not-so-and-so' cannot in any sense be in
motion: therefore it is impossible for that which is not to be in
motion. This being so, it follows that 'becoming' cannot be a motion:
for it is that which 'is not' that 'becomes'. For however true it
may be that it accidentally 'becomes', it is nevertheless correct
to say that it is that which 'is not' that in an unqualified sense
'becomes'. And similarly it is impossible for that which 'is not'
to be at rest. 

There are these difficulties, then, in the way of the assumption that
that which 'is not' can be in motion: and it may be further objected
that, whereas everything which is in motion is in space, that which
'is not' is not in space: for then it would be somewhere.

So, too, 'perishing' is not a motion: for a motion has for its contrary
either another motion or rest, whereas 'perishing' is the contrary
of 'becoming'. 

Since, then, every motion is a kind of change, and there are only
the three kinds of change mentioned above, and since of these three
those which take the form of 'becoming' and 'perishing', that is to
say those which imply a relation of contradiction, are not motions:
it necessarily follows that only change from subject to subject is
motion. And every such subject is either a contrary or an intermediate
(for a privation may be allowed to rank as a contrary) and can be
affirmatively expressed, as naked, toothless, or black. If, then,
the categories are severally distinguished as Being, Quality, Place,
Time, Relation, Quantity, and Activity or Passivity, it necessarily
follows that there are three kinds of motion-qualitative, quantitative,
and local. 

Part 2

In respect of Substance there is no motion, because Substance has
no contrary among things that are. Nor is there motion in respect
of Relation: for it may happen that when one correlative changes,
the other, although this does not itself change, is no longer applicable,
so that in these cases the motion is accidental. Nor is there motion
in respect of Agent and Patient-in fact there can never be motion
of mover and moved, because there cannot be motion of motion or becoming
of becoming or in general change of change. 

For in the first place there are two senses in which motion of motion
is conceivable. (1) The motion of which there is motion might be conceived
as subject; e.g. a man is in motion because he changes from fair to
dark. Can it be that in this sense motion grows hot or cold, or changes
place, or increases or decreases? Impossible: for change is not a
subject. Or (2) can there be motion of motion in the sense that some
other subject changes from a change to another mode of being, as e.g.
a man changes from falling ill to getting well? Even this is possible
only in an accidental sense. For, whatever the subject may be, movement
is change from one form to another. (And the same holds good of becoming
and perishing, except that in these processes we have a change to
a particular kind of opposite, while the other, motion, is a change
to a different kind.) So, if there is to be motion of motion, that
which is changing from health to sickness must simultaneously be changing
from this very change to another. It is clear, then, that by the time
that it has become sick, it must also have changed to whatever may
be the other change concerned (for that it should be at rest, though
logically possible, is excluded by the theory). Moreover this other
can never be any casual change, but must be a change from something
definite to some other definite thing. So in this case it must be
the opposite change, viz. convalescence. It is only accidentally that
there can be change of change, e.g. there is a change from remembering
to forgetting only because the subject of this change changes at one
time to knowledge, at another to ignorance. 

In the second place, if there is to be change of change and becoming
of becoming, we shall have an infinite regress. Thus if one of a series
of changes is to be a change of change, the preceding change must
also be so: e.g. if simple becoming was ever in process of becoming,
then that which was becoming simple becoming was also in process of
becoming, so that we should not yet have arrived at what was in process
of simple becoming but only at what was already in process of becoming
in process of becoming. And this again was sometime in process of
becoming, so that even then we should not have arrived at what was
in process of simple becoming. And since in an infinite series there
is no first term, here there will be no first stage and therefore
no following stage either. On this hypothesis, then, nothing can become
or be moved or change. 

Thirdly, if a thing is capable of any particular motion, it is also
capable of the corresponding contrary motion or the corresponding
coming to rest, and a thing that is capable of becoming is also capable
of perishing: consequently, if there be becoming of becoming, that
which is in process of becoming is in process of perishing at the
very moment when it has reached the stage of becoming: since it cannot
be in process of perishing when it is just beginning to become or
after it has ceased to become: for that which is in process of perishing
must be in existence. 

Fourthly, there must be a substrate underlying all processes of becoming
and changing. What can this be in the present case? It is either the
body or the soul that undergoes alteration: what is it that correspondingly
becomes motion or becoming? And again what is the goal of their motion?
It must be the motion or becoming of something from something to something
else. But in what sense can this be so? For the becoming of learning
cannot be learning: so neither can the becoming of becoming be becoming,
nor can the becoming of any process be that process. 

Finally, since there are three kinds of motion, the substratum and
the goal of motion must be one or other of these, e.g. locomotion
will have to be altered or to be locally moved. 

To sum up, then, since everything that is moved is moved in one of
three ways, either accidentally, or partially, or essentially, change
can change only accidentally, as e.g. when a man who is being restored
to health runs or learns: and accidental change we have long ago decided
to leave out of account. 

Since, then, motion can belong neither to Being nor to Relation nor
to Agent and Patient, it remains that there can be motion only in
respect of Quality, Quantity, and Place: for with each of these we
have a pair of contraries. Motion in respect of Quality let us call
alteration, a general designation that is used to include both contraries:
and by Quality I do not here mean a property of substance (in that
sense that which constitutes a specific distinction is a quality)
but a passive quality in virtue of which a thing is said to be acted
on or to be incapable of being acted on. Motion in respect of Quantity
has no name that includes both contraries, but it is called increase
or decrease according as one or the other is designated: that is to
say motion in the direction of complete magnitude is increase, motion
in the contrary direction is decrease. Motion in respect of Place
has no name either general or particular: but we may designate it
by the general name of locomotion, though strictly the term 'locomotion'
is applicable to things that change their place only when they have
not the power to come to a stand, and to things that do not move themselves
locally. 

Change within the same kind from a lesser to a greater or from a greater
to a lesser degree is alteration: for it is motion either from a contrary
or to a contrary, whether in an unqualified or in a qualified sense:
for change to a lesser degree of a quality will be called change to
the contrary of that quality, and change to a greater degree of a
quality will be regarded as change from the contrary of that quality
to the quality itself. It makes no difference whether the change be
qualified or unqualified, except that in the former case the contraries
will have to be contrary to one another only in a qualified sense:
and a thing's possessing a quality in a greater or in a lesser degree
means the presence or absence in it of more or less of the opposite
quality. It is now clear, then, that there are only these three kinds
of motion. 

The term 'immovable' we apply in the first place to that which is
absolutely incapable of being moved (just as we correspondingly apply
the term invisible to sound); in the second place to that which is
moved with difficulty after a long time or whose movement is slow
at the start-in fact, what we describe as hard to move; and in the
third place to that which is naturally designed for and capable of
motion, but is not in motion when, where, and as it naturally would
be so. This last is the only kind of immovable thing of which I use
the term 'being at rest': for rest is contrary to motion, so that
rest will be negation of motion in that which is capable of admitting
motion. 

The foregoing remarks are sufficient to explain the essential nature
of motion and rest, the number of kinds of change, and the different
varieties of motion. 

Part 3

Let us now proceed to define the terms 'together' and 'apart', 'in
contact', 'between', 'in succession', 'contiguous', and 'continuous',
and to show in what circumstances each of these terms is naturally
applicable. 

Things are said to be together in place when they are in one place
(in the strictest sense of the word 'place') and to be apart when
they are in different places. 

Things are said to be in contact when their extremities are together.

That which a changing thing, if it changes continuously in a natural
manner, naturally reaches before it reaches that to which it changes
last, is between. Thus 'between' implies the presence of at least
three things: for in a process of change it is the contrary that is
'last': and a thing is moved continuously if it leaves no gap or only
the smallest possible gap in the material-not in the time (for a gap
in the time does not prevent things having a 'between', while, on
the other hand, there is nothing to prevent the highest note sounding
immediately after the lowest) but in the material in which the motion
takes place. This is manifestly true not only in local changes but
in every other kind as well. (Now every change implies a pair of opposites,
and opposites may be either contraries or contradictories; since then
contradiction admits of no mean term, it is obvious that 'between'
must imply a pair of contraries) That is locally contrary which is
most distant in a straight line: for the shortest line is definitely
limited, and that which is definitely limited constitutes a measure.

A thing is 'in succession' when it is after the beginning in position
or in form or in some other respect in which it is definitely so regarded,
and when further there is nothing of the same kind as itself between
it and that to which it is in succession, e.g. a line or lines if
it is a line, a unit or units if it is a unit, a house if it is a
house (there is nothing to prevent something of a different kind being
between). For that which is in succession is in succession to a particular
thing, and is something posterior: for one is not 'in succession'
to two, nor is the first day of the month to be second: in each case
the latter is 'in succession' to the former. 

A thing that is in succession and touches is 'contiguous'. The 'continuous'
is a subdivision of the contiguous: things are called continuous when
the touching limits of each become one and the same and are, as the
word implies, contained in each other: continuity is impossible if
these extremities are two. This definition makes it plain that continuity
belongs to things that naturally in virtue of their mutual contact
form a unity. And in whatever way that which holds them together is
one, so too will the whole be one, e.g. by a rivet or glue or contact
or organic union. 

It is obvious that of these terms 'in succession' is first in order
of analysis: for that which touches is necessarily in succession,
but not everything that is in succession touches: and so succession
is a property of things prior in definition, e.g. numbers, while contact
is not. And if there is continuity there is necessarily contact, but
if there is contact, that alone does not imply continuity: for the
extremities of things may be 'together' without necessarily being
one: but they cannot be one without being necessarily together. So
natural junction is last in coming to be: for the extremities must
necessarily come into contact if they are to be naturally joined:
but things that are in contact are not all naturally joined, while
there is no contact clearly there is no natural junction either. Hence,
if as some say 'point' and 'unit' have an independent existence of
their own, it is impossible for the two to be identical: for points
can touch while units can only be in succession. Moreover, there can
always be something between points (for all lines are intermediate
between points), whereas it is not necessary that there should possibly
be anything between units: for there can be nothing between the numbers
one and two. 

We have now defined what is meant by 'together' and 'apart', 'contact',
'between' and 'in succession', 'contiguous' and 'continuous': and
we have shown in what circumstances each of these terms is applicable.

Part 4

There are many senses in which motion is said to be 'one': for we
use the term 'one' in many senses. 

Motion is one generically according to the different categories to
which it may be assigned: thus any locomotion is one generically with
any other locomotion, whereas alteration is different generically
from locomotion. 

Motion is one specifically when besides being one generically it also
takes place in a species incapable of subdivision: e.g. colour has
specific differences: therefore blackening and whitening differ specifically;
but at all events every whitening will be specifically the same with
every other whitening and every blackening with every other blackening.
But white is not further subdivided by specific differences: hence
any whitening is specifically one with any other whitening. Where
it happens that the genus is at the same time a species, it is clear
that the motion will then in a sense be one specifically though not
in an unqualified sense: learning is an example of this, knowledge
being on the one hand a species of apprehension and on the other hand
a genus including the various knowledges. A difficulty, however, may
be raised as to whether a motion is specifically one when the same
thing changes from the same to the same, e.g. when one point changes
again and again from a particular place to a particular place: if
this motion is specifically one, circular motion will be the same
as rectilinear motion, and rolling the same as walking. But is not
this difficulty removed by the principle already laid down that if
that in which the motion takes place is specifically different (as
in the present instance the circular path is specifically different
from the straight) the motion itself is also different? We have explained,
then, what is meant by saying that motion is one generically or one
specifically. 

Motion is one in an unqualified sense when it is one essentially or
numerically: and the following distinctions will make clear what this
kind of motion is. There are three classes of things in connexion
with which we speak of motion, the 'that which', the 'that in which',
and the 'that during which'. I mean that there must he something that
is in motion, e.g. a man or gold, and it must be in motion in something,
e.g. a place or an affection, and during something, for all motion
takes place during a time. Of these three it is the thing in which
the motion takes place that makes it one generically or specifically,
it is the thing moved that makes the motion one in subject, and it
is the time that makes it consecutive: but it is the three together
that make it one without qualification: to effect this, that in which
the motion takes place (the species) must be one and incapable of
subdivision, that during which it takes place (the time) must be one
and unintermittent, and that which is in motion must be one-not in
an accidental sense (i.e. it must be one as the white that blackens
is one or Coriscus who walks is one, not in the accidental sense in
which Coriscus and white may be one), nor merely in virtue of community
of nature (for there might be a case of two men being restored to
health at the same time in the same way, e.g. from inflammation of
the eye, yet this motion is not really one, but only specifically
one). 

Suppose, however, that Socrates undergoes an alteration specifically
the same but at one time and again at another: in this case if it
is possible for that which ceased to be again to come into being and
remain numerically the same, then this motion too will be one: otherwise
it will be the same but not one. And akin to this difficulty there
is another; viz. is health one? and generally are the states and affections
in bodies severally one in essence although (as is clear) the things
that contain them are obviously in motion and in flux? Thus if a person's
health at daybreak and at the present moment is one and the same,
why should not this health be numerically one with that which he recovers
after an interval? The same argument applies in each case. There is,
however, we may answer, this difference: that if the states are two
then it follows simply from this fact that the activities must also
in point of number be two (for only that which is numerically one
can give rise to an activity that is numerically one), but if the
state is one, this is not in itself enough to make us regard the activity
also as one: for when a man ceases walking, the walking no longer
is, but it will again be if he begins to walk again. But, be this
as it may, if in the above instance the health is one and the same,
then it must be possible for that which is one and the same to come
to be and to cease to be many times. However, these difficulties lie
outside our present inquiry. 

Since every motion is continuous, a motion that is one in an unqualified
sense must (since every motion is divisible) be continuous, and a
continuous motion must be one. There will not be continuity between
any motion and any other indiscriminately any more than there is between
any two things chosen at random in any other sphere: there can be
continuity only when the extremities of the two things are one. Now
some things have no extremities at all: and the extremities of others
differ specifically although we give them the same name of 'end':
how should e.g. the 'end' of a line and the 'end' of walking touch
or come to be one? Motions that are not the same either specifically
or generically may, it is true, be consecutive (e.g. a man may run
and then at once fall ill of a fever), and again, in the torch-race
we have consecutive but not continuous locomotion: for according to
our definition there can be continuity only when the ends of the two
things are one. Hence motions may be consecutive or successive in
virtue of the time being continuous, but there can be continuity only
in virtue of the motions themselves being continuous, that is when
the end of each is one with the end of the other. Motion, therefore,
that is in an unqualified sense continuous and one must be specifically
the same, of one thing, and in one time. Unity is required in respect
of time in order that there may be no interval of immobility, for
where there is intermission of motion there must be rest, and a motion
that includes intervals of rest will be not one but many, so that
a motion that is interrupted by stationariness is not one or continuous,
and it is so interrupted if there is an interval of time. And though
of a motion that is not specifically one (even if the time is unintermittent)
the time is one, the motion is specifically different, and so cannot
really be one, for motion that is one must be specifically one, though
motion that is specifically one is not necessarily one in an unqualified
sense. We have now explained what we mean when we call a motion one
without qualification. 

Further, a motion is also said to be one generically, specifically,
or essentially when it is complete, just as in other cases completeness
and wholeness are characteristics of what is one: and sometimes a
motion even if incomplete is said to be one, provided only that it
is continuous. 

And besides the cases already mentioned there is another in which
a motion is said to be one, viz. when it is regular: for in a sense
a motion that is irregular is not regarded as one, that title belonging
rather to that which is regular, as a straight line is regular, the
irregular being as such divisible. But the difference would seem to
be one of degree. In every kind of motion we may have regularity or
irregularity: thus there may be regular alteration, and locomotion
in a regular path, e.g. in a circle or on a straight line, and it
is the same with regard to increase and decrease. The difference that
makes a motion irregular is sometimes to be found in its path: thus
a motion cannot be regular if its path is an irregular magnitude,
e.g. a broken line, a spiral, or any other magnitude that is not such
that any part of it taken at random fits on to any other that may
be chosen. Sometimes it is found neither in the place nor in the time
nor in the goal but in the manner of the motion: for in some cases
the motion is differentiated by quickness and slowness: thus if its
velocity is uniform a motion is regular, if not it is irregular. So
quickness and slowness are not species of motion nor do they constitute
specific differences of motion, because this distinction occurs in
connexion with all the distinct species of motion. The same is true
of heaviness and lightness when they refer to the same thing: e.g.
they do not specifically distinguish earth from itself or fire from
itself. Irregular motion, therefore, while in virtue of being continuous
it is one, is so in a lesser degree, as is the case with locomotion
in a broken line: and a lesser degree of something always means an
admixture of its contrary. And since every motion that is one can
be both regular and irregular, motions that are consecutive but not
specifically the same cannot be one and continuous: for how should
a motion composed of alteration and locomotion be regular? If a motion
is to be regular its parts ought to fit one another. 

Part 5

We have further to determine what motions are contrary to each other,
and to determine similarly how it is with rest. And we have first
to decide whether contrary motions are motions respectively from and
to the same thing, e.g. a motion from health and a motion to health
(where the opposition, it would seem, is of the same kind as that
between coming to be and ceasing to be); or motions respectively from
contraries, e.g. a motion from health and a motion from disease; or
motions respectively to contraries, e.g. a motion to health and a
motion to disease; or motions respectively from a contrary and to
the opposite contrary, e.g. a motion from health and a motion to disease;
or motions respectively from a contrary to the opposite contrary and
from the latter to the former, e.g. a motion from health to disease
and a motion from disease to health: for motions must be contrary
to one another in one or more of these ways, as there is no other
way in which they can be opposed. 

Now motions respectively from a contrary and to the opposite contrary,
e.g. a motion from health and a motion to disease, are not contrary
motions: for they are one and the same. (Yet their essence is not
the same, just as changing from health is different from changing
to disease.) Nor are motion respectively from a contrary and from
the opposite contrary contrary motions, for a motion from a contrary
is at the same time a motion to a contrary or to an intermediate (of
this, however, we shall speak later), but changing to a contrary rather
than changing from a contrary would seem to be the cause of the contrariety
of motions, the latter being the loss, the former the gain, of contrariness.
Moreover, each several motion takes its name rather from the goal
than from the starting-point of change, e.g. motion to health we call
convalescence, motion to disease sickening. Thus we are left with
motions respectively to contraries, and motions respectively to contraries
from the opposite contraries. Now it would seem that motions to contraries
are at the same time motions from contraries (though their essence
may not be the same; 'to health' is distinct, I mean, from 'from disease',
and 'from health' from 'to disease'). 

Since then change differs from motion (motion being change from a
particular subject to a particular subject), it follows that contrary
motions are motions respectively from a contrary to the opposite contrary
and from the latter to the former, e.g. a motion from health to disease
and a motion from disease to health. Moreover, the consideration of
particular examples will also show what kinds of processes are generally
recognized as contrary: thus falling ill is regarded as contrary to
recovering one's health, these processes having contrary goals, and
being taught as contrary to being led into error by another, it being
possible to acquire error, like knowledge, either by one's own agency
or by that of another. Similarly we have upward locomotion and downward
locomotion, which are contrary lengthwise, locomotion to the right
and locomotion to the left, which are contrary breadthwise, and forward
locomotion and backward locomotion, which too are contraries. On the
other hand, a process simply to a contrary, e.g. that denoted by the
expression 'becoming white', where no starting-point is specified,
is a change but not a motion. And in all cases of a thing that has
no contrary we have as contraries change from and change to the same
thing. Thus coming to be is contrary to ceasing to be, and losing
to gaining. But these are changes and not motions. And wherever a
pair of contraries admit of an intermediate, motions to that intermediate
must be held to be in a sense motions to one or other of the contraries:
for the intermediate serves as a contrary for the purposes of the
motion, in whichever direction the change may be, e.g. grey in a motion
from grey to white takes the place of black as starting-point, in
a motion from white to grey it takes the place of black as goal, and
in a motion from black to grey it takes the place of white as goal:
for the middle is opposed in a sense to either of the extremes, as
has been said above. Thus we see that two motions are contrary to
each other only when one is a motion from a contrary to the opposite
contrary and the other is a motion from the latter to the former.

Part 6

But since a motion appears to have contrary to it not only another
motion but also a state of rest, we must determine how this is so.
A motion has for its contrary in the strict sense of the term another
motion, but it also has for an opposite a state of rest (for rest
is the privation of motion and the privation of anything may be called
its contrary), and motion of one kind has for its opposite rest of
that kind, e.g. local motion has local rest. This statement, however,
needs further qualification: there remains the question, is the opposite
of remaining at a particular place motion from or motion to that place?
It is surely clear that since there are two subjects between which
motion takes place, motion from one of these (A) to its contrary (B)
has for its opposite remaining in A while the reverse motion has for
its opposite remaining in B. At the same time these two are also contrary
to each other: for it would be absurd to suppose that there are contrary
motions and not opposite states of rest. States of rest in contraries
are opposed. To take an example, a state of rest in health is (1)
contrary to a state of rest in disease, and (2) the motion to which
it is contrary is that from health to disease. For (2) it would be
absurd that its contrary motion should be that from disease to health,
since motion to that in which a thing is at rest is rather a coming
to rest, the coming to rest being found to come into being simultaneously
with the motion; and one of these two motions it must be. And (1)
rest in whiteness is of course not contrary to rest in health.

Of all things that have no contraries there are opposite changes (viz.
change from the thing and change to the thing, e.g. change from being
and change to being), but no motion. So, too, of such things there
is no remaining though there is absence of change. Should there be
a particular subject, absence of change in its being will be contrary
to absence of change in its not-being. And here a difficulty may be
raised: if not-being is not a particular something, what is it, it
may be asked, that is contrary to absence of change in a thing's being?
and is this absence of change a state of rest? If it is, then either
it is not true that every state of rest is contrary to a motion or
else coming to be and ceasing to be are motion. It is clear then that,
since we exclude these from among motions, we must not say that this
absence of change is a state of rest: we must say that it is similar
to a state of rest and call it absence of change. And it will have
for its contrary either nothing or absence of change in the thing's
not-being, or the ceasing to be of the thing: for such ceasing to
be is change from it and the thing's coming to be is change to it.

Again, a further difficulty may be raised. How is it, it may be asked,
that whereas in local change both remaining and moving may be natural
or unnatural, in the other changes this is not so? e.g. alteration
is not now natural and now unnatural, for convalescence is no more
natural or unnatural than falling ill, whitening no more natural or
unnatural than blackening; so, too, with increase and decrease: these
are not contrary to each other in the sense that either of them is
natural while the other is unnatural, nor is one increase contrary
to another in this sense; and the same account may be given of becoming
and perishing: it is not true that becoming is natural and perishing
unnatural (for growing old is natural), nor do we observe one becoming
to be natural and another unnatural. We answer that if what happens
under violence is unnatural, then violent perishing is unnatural and
as such contrary to natural perishing. Are there then also some becomings
that are violent and not the result of natural necessity, and are
therefore contrary to natural becomings, and violent increases and
decreases, e.g. the rapid growth to maturity of profligates and the
rapid ripening of seeds even when not packed close in the earth? And
how is it with alterations? Surely just the same: we may say that
some alterations are violent while others are natural, e.g. patients
alter naturally or unnaturally according as they throw off fevers
on the critical days or not. But, it may be objected, then we shall
have perishings contrary to one another, not to becoming. Certainly:
and why should not this in a sense be so? Thus it is so if one perishing
is pleasant and another painful: and so one perishing will be contrary
to another not in an unqualified sense, but in so far as one has this
quality and the other that. 

Now motions and states of rest universally exhibit contrariety in
the manner described above, e.g. upward motion and rest above are
respectively contrary to downward motion and rest below, these being
instances of local contrariety; and upward locomotion belongs naturally
to fire and downward to earth, i.e. the locomotions of the two are
contrary to each other. And again, fire moves up naturally and down
unnaturally: and its natural motion is certainly contrary to its unnatural
motion. Similarly with remaining: remaining above is contrary to motion
from above downwards, and to earth this remaining comes unnaturally,
this motion naturally. So the unnatural remaining of a thing is contrary
to its natural motion, just as we find a similar contrariety in the
motion of the same thing: one of its motions, the upward or the downward,
will be natural, the other unnatural. 

Here, however, the question arises, has every state of rest that is
not permanent a becoming, and is this becoming a coming to a standstill?
If so, there must be a becoming of that which is at rest unnaturally,
e.g. of earth at rest above: and therefore this earth during the time
that it was being carried violently upward was coming to a standstill.
But whereas the velocity of that which comes to a standstill seems
always to increase, the velocity of that which is carried violently
seems always to decrease: so it will he in a state of rest without
having become so. Moreover 'coming to a standstill' is generally recognized
to be identical or at least concomitant with the locomotion of a thing
to its proper place. 

There is also another difficulty involved in the view that remaining
in a particular place is contrary to motion from that place. For when
a thing is moving from or discarding something, it still appears to
have that which is being discarded, so that if a state of rest is
itself contrary to the motion from the state of rest to its contrary,
the contraries rest and motion will be simultaneously predicable of
the same thing. May we not say, however, that in so far as the thing
is still stationary it is in a state of rest in a qualified sense?
For, in fact, whenever a thing is in motion, part of it is at the
starting-point while part is at the goal to which it is changing:
and consequently a motion finds its true contrary rather in another
motion than in a state of rest. 

With regard to motion and rest, then, we have now explained in what
sense each of them is one and under what conditions they exhibit contrariety.

[With regard to coming to a standstill the question may be raised
whether there is an opposite state of rest to unnatural as well as
to natural motions. It would be absurd if this were not the case:
for a thing may remain still merely under violence: thus we shall
have a thing being in a non-permanent state of rest without having
become so. But it is clear that it must be the case: for just as there
is unnatural motion, so, too, a thing may be in an unnatural state
of rest. Further, some things have a natural and an unnatural motion,
e.g. fire has a natural upward motion and an unnatural downward motion:
is it, then, this unnatural downward motion or is it the natural downward
motion of earth that is contrary to the natural upward motion? Surely
it is clear that both are contrary to it though not in the same sense:
the natural motion of earth is contrary inasmuch as the motion of
fire is also natural, whereas the upward motion of fire as being natural
is contrary to the downward motion of fire as being unnatural. The
same is true of the corresponding cases of remaining. But there would
seem to be a sense in which a state of rest and a motion are opposites.]

----------------------------------------------------------------------

BOOK VI

Part 1 

Now if the terms 'continuous', 'in contact', and 'in succession'
are understood as defined above things being 'continuous' if their
extremities are one, 'in contact' if their extremities are together,
and 'in succession' if there is nothing of their own kind intermediate
between them-nothing that is continuous can be composed 'of indivisibles':
e.g. a line cannot be composed of points, the line being continuous
and the point indivisible. For the extremities of two points can neither
be one (since of an indivisible there can be no extremity as distinct
from some other part) nor together (since that which has no parts
can have no extremity, the extremity and the thing of which it is
the extremity being distinct). 

Moreover, if that which is continuous is composed of points, these
points must be either continuous or in contact with one another: and
the same reasoning applies in the case of all indivisibles. Now for
the reason given above they cannot be continuous: and one thing can
be in contact with another only if whole is in contact with whole
or part with part or part with whole. But since indivisibles have
no parts, they must be in contact with one another as whole with whole.
And if they are in contact with one another as whole with whole, they
will not be continuous: for that which is continuous has distinct
parts: and these parts into which it is divisible are different in
this way, i.e. spatially separate. 

Nor, again, can a point be in succession to a point or a moment to
a moment in such a way that length can be composed of points or time
of moments: for things are in succession if there is nothing of their
own kind intermediate between them, whereas that which is intermediate
between points is always a line and that which is intermediate between
moments is always a period of time. 

Again, if length and time could thus be composed of indivisibles,
they could be divided into indivisibles, since each is divisible into
the parts of which it is composed. But, as we saw, no continuous thing
is divisible into things without parts. Nor can there be anything
of any other kind intermediate between the parts or between the moments:
for if there could be any such thing it is clear that it must be either
indivisible or divisible, and if it is divisible, it must be divisible
either into indivisibles or into divisibles that are infinitely divisible,
in which case it is continuous. 

Moreover, it is plain that everything continuous is divisible into
divisibles that are infinitely divisible: for if it were divisible
into indivisibles, we should have an indivisible in contact with an
indivisible, since the extremities of things that are continuous with
one another are one and are in contact. 

The same reasoning applies equally to magnitude, to time, and to motion:
either all of these are composed of indivisibles and are divisible
into indivisibles, or none. This may be made clear as follows. If
a magnitude is composed of indivisibles, the motion over that magnitude
must be composed of corresponding indivisible motions: e.g. if the
magnitude ABG is composed of the indivisibles A, B, G, each corresponding
part of the motion DEZ of O over ABG is indivisible. Therefore, since
where there is motion there must be something that is in motion, and
where there is something in motion there must be motion, therefore
the being-moved will also be composed of indivisibles. So O traversed
A when its motion was D, B when its motion was E, and G similarly
when its motion was Z. Now a thing that is in motion from one place
to another cannot at the moment when it was in motion both be in motion
and at the same time have completed its motion at the place to which
it was in motion: e.g. if a man is walking to Thebes, he cannot be
walking to Thebes and at the same time have completed his walk to
Thebes: and, as we saw, O traverses a the partless section A in virtue
of the presence of the motion D. Consequently, if O actually passed
through A after being in process of passing through, the motion must
be divisible: for at the time when O was passing through, it neither
was at rest nor had completed its passage but was in an intermediate
state: while if it is passing through and has completed its passage
at the same moment, then that which is walking will at the moment
when it is walking have completed its walk and will be in the place
to which it is walking; that is to say, it will have completed its
motion at the place to which it is in motion. And if a thing is in
motion over the whole Kbg and its motion is the three D, E, and Z,
and if it is not in motion at all over the partless section A but
has completed its motion over it, then the motion will consist not
of motions but of starts, and will take place by a thing's having
completed a motion without being in motion: for on this assumption
it has completed its passage through A without passing through it.
So it will be possible for a thing to have completed a walk without
ever walking: for on this assumption it has completed a walk over
a particular distance without walking over that distance. Since, then,
everything must be either at rest or in motion, and O is therefore
at rest in each of the sections A, B, and G, it follows that a thing
can be continuously at rest and at the same time in motion: for, as
we saw, O is in motion over the whole ABG and at rest in any part
(and consequently in the whole) of it. Moreover, if the indivisibles
composing DEZ are motions, it would be possible for a thing in spite
of the presence in it of motion to be not in motion but at rest, while
if they are not motions, it would be possible for motion to be composed
of something other than motions. 

And if length and motion are thus indivisible, it is neither more
nor less necessary that time also be similarly indivisible, that is
to say be composed of indivisible moments: for if the whole distance
is divisible and an equal velocity will cause a thing to pass through
less of it in less time, the time must also be divisible, and conversely,
if the time in which a thing is carried over the section A is divisible,
this section A must also be divisible. 

Part 2

And since every magnitude is divisible into magnitudes-for we have
shown that it is impossible for anything continuous to be composed
of indivisible parts, and every magnitude is continuous-it necessarily
follows that the quicker of two things traverses a greater magnitude
in an equal time, an equal magnitude in less time, and a greater magnitude
in less time, in conformity with the definition sometimes given of
'the quicker'. Suppose that A is quicker than B. Now since of two
things that which changes sooner is quicker, in the time ZH, in which
A has changed from G to D, B will not yet have arrived at D but will
be short of it: so that in an equal time the quicker will pass over
a greater magnitude. More than this, it will pass over a greater magnitude
in less time: for in the time in which A has arrived at D, B being
the slower has arrived, let us say, at E. Then since A has occupied
the whole time ZH in arriving at D, will have arrived at O in less
time than this, say ZK. Now the magnitude GO that A has passed over
is greater than the magnitude GE, and the time ZK is less than the
whole time ZH: so that the quicker will pass over a greater magnitude
in less time. And from this it is also clear that the quicker will
pass over an equal magnitude in less time than the slower. For since
it passes over the greater magnitude in less time than the slower,
and (regarded by itself) passes over LM the greater in more time than
LX the lesser, the time PRh in which it passes over LM will be more
than the time PS, which it passes over LX: so that, the time PRh being
less than the time PCh in which the slower passes over LX, the time
PS will also be less than the time PX: for it is less than the time
PRh, and that which is less than something else that is less than
a thing is also itself less than that thing. Hence it follows that
the quicker will traverse an equal magnitude in less time than the
slower. Again, since the motion of anything must always occupy either
an equal time or less or more time in comparison with that of another
thing, and since, whereas a thing is slower if its motion occupies
more time and of equal velocity if its motion occupies an equal time,
the quicker is neither of equal velocity nor slower, it follows that
the motion of the quicker can occupy neither an equal time nor more
time. It can only be, then, that it occupies less time, and thus we
get the necessary consequence that the quicker will pass over an equal
magnitude (as well as a greater) in less time than the slower.

And since every motion is in time and a motion may occupy any time,
and the motion of everything that is in motion may be either quicker
or slower, both quicker motion and slower motion may occupy any time:
and this being so, it necessarily follows that time also is continuous.
By continuous I mean that which is divisible into divisibles that
are infinitely divisible: and if we take this as the definition of
continuous, it follows necessarily that time is continuous. For since
it has been shown that the quicker will pass over an equal magnitude
in less time than the slower, suppose that A is quicker and B slower,
and that the slower has traversed the magnitude GD in the time ZH.
Now it is clear that the quicker will traverse the same magnitude
in less time than this: let us say in the time ZO. Again, since the
quicker has passed over the whole D in the time ZO, the slower will
in the same time pass over GK, say, which is less than GD. And since
B, the slower, has passed over GK in the time ZO, the quicker will
pass over it in less time: so that the time ZO will again be divided.
And if this is divided the magnitude GK will also be divided just
as GD was: and again, if the magnitude is divided, the time will also
be divided. And we can carry on this process for ever, taking the
slower after the quicker and the quicker after the slower alternately,
and using what has been demonstrated at each stage as a new point
of departure: for the quicker will divide the time and the slower
will divide the length. If, then, this alternation always holds good,
and at every turn involves a division, it is evident that all time
must be continuous. And at the same time it is clear that all magnitude
is also continuous; for the divisions of which time and magnitude
respectively are susceptible are the same and equal. 

Moreover, the current popular arguments make it plain that, if time
is continuous, magnitude is continuous also, inasmuch as a thing asses
over half a given magnitude in half the time taken to cover the whole:
in fact without qualification it passes over a less magnitude in less
time; for the divisions of time and of magnitude will be the same.
And if either is infinite, so is the other, and the one is so in the
same way as the other; i.e. if time is infinite in respect of its
extremities, length is also infinite in respect of its extremities:
if time is infinite in respect of divisibility, length is also infinite
in respect of divisibility: and if time is infinite in both respects,
magnitude is also infinite in both respects. 

Hence Zeno's argument makes a false assumption in asserting that it
is impossible for a thing to pass over or severally to come in contact
with infinite things in a finite time. For there are two senses in
which length and time and generally anything continuous are called
'infinite': they are called so either in respect of divisibility or
in respect of their extremities. So while a thing in a finite time
cannot come in contact with things quantitatively infinite, it can
come in contact with things infinite in respect of divisibility: for
in this sense the time itself is also infinite: and so we find that
the time occupied by the passage over the infinite is not a finite
but an infinite time, and the contact with the infinites is made by
means of moments not finite but infinite in number. 

The passage over the infinite, then, cannot occupy a finite time,
and the passage over the finite cannot occupy an infinite time: if
the time is infinite the magnitude must be infinite also, and if the
magnitude is infinite, so also is the time. This may be shown as follows.
Let AB be a finite magnitude, and let us suppose that it is traversed
in infinite time G, and let a finite period GD of the time be taken.
Now in this period the thing in motion will pass over a certain segment
of the magnitude: let BE be the segment that it has thus passed over.
(This will be either an exact measure of AB or less or greater than
an exact measure: it makes no difference which it is.) Then, since
a magnitude equal to BE will always be passed over in an equal time,
and BE measures the whole magnitude, the whole time occupied in passing
over AB will be finite: for it will be divisible into periods equal
in number to the segments into which the magnitude is divisible. Moreover,
if it is the case that infinite time is not occupied in passing over
every magnitude, but it is possible to ass over some magnitude, say
BE, in a finite time, and if this BE measures the whole of which it
is a part, and if an equal magnitude is passed over in an equal time,
then it follows that the time like the magnitude is finite. That infinite
time will not be occupied in passing over BE is evident if the time
be taken as limited in one direction: for as the part will be passed
over in less time than the whole, the time occupied in traversing
this part must be finite, the limit in one direction being given.
The same reasoning will also show the falsity of the assumption that
infinite length can be traversed in a finite time. It is evident,
then, from what has been said that neither a line nor a surface nor
in fact anything continuous can be indivisible. 

This conclusion follows not only from the present argument but from
the consideration that the opposite assumption implies the divisibility
of the indivisible. For since the distinction of quicker and slower
may apply to motions occupying any period of time and in an equal
time the quicker passes over a greater length, it may happen that
it will pass over a length twice, or one and a half times, as great
as that passed over by the slower: for their respective velocities
may stand to one another in this proportion. Suppose, then, that the
quicker has in the same time been carried over a length one and a
half times as great as that traversed by the slower, and that the
respective magnitudes are divided, that of the quicker, the magnitude
ABGD, into three indivisibles, and that of the slower into the two
indivisibles EZ, ZH. Then the time may also be divided into three
indivisibles, for an equal magnitude will be passed over in an equal
time. Suppose then that it is thus divided into KL, Lm, MN. Again,
since in the same time the slower has been carried over Ez, ZH, the
time may also be similarly divided into two. Thus the indivisible
will be divisible, and that which has no parts will be passed over
not in an indivisible but in a greater time. It is evident, therefore,
that nothing continuous is without parts. 

Part 3

The present also is necessarily indivisible-the present, that is,
not in the sense in which the word is applied to one thing in virtue
of another, but in its proper and primary sense; in which sense it
is inherent in all time. For the present is something that is an extremity
of the past (no part of the future being on this side of it) and also
of the future (no part of the past being on the other side of it):
it is, as we have said, a limit of both. And if it is once shown that
it is essentially of this character and one and the same, it will
at once be evident also that it is indivisible. 

Now the present that is the extremity of both times must be one and
the same: for if each extremity were different, the one could not
be in succession to the other, because nothing continuous can be composed
of things having no parts: and if the one is apart from the other,
there will be time intermediate between them, because everything continuous
is such that there is something intermediate between its limits and
described by the same name as itself. But if the intermediate thing
is time, it will be divisible: for all time has been shown to be divisible.
Thus on this assumption the present is divisible. But if the present
is divisible, there will be part of the past in the future and part
of the future in the past: for past time will be marked off from future
time at the actual point of division. Also the present will be a present
not in the proper sense but in virtue of something else: for the division
which yields it will not be a division proper. Furthermore, there
will be a part of the present that is past and a part that is future,
and it will not always be the same part that is past or future: in
fact one and the same present will not be simultaneous: for the time
may be divided at many points. If, therefore, the present cannot possibly
have these characteristics, it follows that it must be the same present
that belongs to each of the two times. But if this is so it is evident
that the present is also indivisible: for if it is divisible it will
be involved in the same implications as before. It is clear, then,
from what has been said that time contains something indivisible,
and this is what we call a present. 

We will now show that nothing can be in motion in a present. For if
this is possible, there can be both quicker and slower motion in the
present. Suppose then that in the present N the quicker has traversed
the distance AB. That being so, the slower will in the same present
traverse a distance less than AB, say AG. But since the slower will
have occupied the whole present in traversing AG, the quicker will
occupy less than this in traversing it. Thus we shall have a division
of the present, whereas we found it to be indivisible. It is impossible,
therefore, for anything to be in motion in a present. 

Nor can anything be at rest in a present: for, as we were saying,
only can be at rest which is naturally designed to be in motion but
is not in motion when, where, or as it would naturally be so: since,
therefore, nothing is naturally designed to be in motion in a present,
it is clear that nothing can be at rest in a present either.

Moreover, inasmuch as it is the same present that belongs to both
the times, and it is possible for a thing to be in motion throughout
one time and to be at rest throughout the other, and that which is
in motion or at rest for the whole of a time will be in motion or
at rest as the case may be in any part of it in which it is naturally
designed to be in motion or at rest: this being so, the assumption
that there can be motion or rest in a present will carry with it the
implication that the same thing can at the same time be at rest and
in motion: for both the times have the same extremity, viz. the present.

Again, when we say that a thing is at rest, we imply that its condition
in whole and in part is at the time of speaking uniform with what
it was previously: but the present contains no 'previously': consequently,
there can be no rest in it. 

It follows then that the motion of that which is in motion and the
rest of that which is at rest must occupy time. 

Part 4

Further, everything that changes must be divisible. For since every
change is from something to something, and when a thing is at the
goal of its change it is no longer changing, and when both it itself
and all its parts are at the starting-point of its change it is not
changing (for that which is in whole and in part in an unvarying condition
is not in a state of change); it follows, therefore, that part of
that which is changing must be at the starting-point and part at the
goal: for as a whole it cannot be in both or in neither. (Here by
'goal of change' I mean that which comes first in the process of change:
e.g. in a process of change from white the goal in question will be
grey, not black: for it is not necessary that that that which is changing
should be at either of the extremes.) It is evident, therefore, that
everything that changes must be divisible. 

Now motion is divisible in two senses. In the first place it is divisible
in virtue of the time that it occupies. In the second place it is
divisible according to the motions of the several parts of that which
is in motion: e.g. if the whole AG is in motion, there will be a motion
of AB and a motion of BG. That being so, let DE be the motion of the
part AB and EZ the motion of the part BG. Then the whole Dz must be
the motion of AG: for DZ must constitute the motion of AG inasmuch
as DE and EZ severally constitute the motions of each of its parts.
But the motion of a thing can never be constituted by the motion of
something else: consequently the whole motion is the motion of the
whole magnitude. 

Again, since every motion is a motion of something, and the whole
motion DZ is not the motion of either of the parts (for each of the
parts DE, EZ is the motion of one of the parts AB, BG) or of anything
else (for, the whole motion being the motion of a whole, the parts
of the motion are the motions of the parts of that whole: and the
parts of DZ are the motions of AB, BG and of nothing else: for, as
we saw, a motion that is one cannot be the motion of more things than
one): since this is so, the whole motion will be the motion of the
magnitude ABG. 

Again, if there is a motion of the whole other than DZ, say the the
of each of the arts may be subtracted from it: and these motions will
be equal to DE, EZ respectively: for the motion of that which is one
must be one. So if the whole motion OI may be divided into the motions
of the parts, OI will be equal to DZ: if on the other hand there is
any remainder, say KI, this will be a motion of nothing: for it can
be the motion neither of the whole nor of the parts (as the motion
of that which is one must be one) nor of anything else: for a motion
that is continuous must be the motion of things that are continuous.
And the same result follows if the division of OI reveals a surplus
on the side of the motions of the parts. Consequently, if this is
impossible, the whole motion must be the same as and equal to DZ.

This then is what is meant by the division of motion according to
the motions of the parts: and it must be applicable to everything
that is divisible into parts. 

Motion is also susceptible of another kind of division, that according
to time. For since all motion is in time and all time is divisible,
and in less time the motion is less, it follows that every motion
must be divisible according to time. And since everything that is
in motion is in motion in a certain sphere and for a certain time
and has a motion belonging to it, it follows that the time, the motion,
the being-in-motion, the thing that is in motion, and the sphere of
the motion must all be susceptible of the same divisions (though spheres
of motion are not all divisible in a like manner: thus quantity is
essentially, quality accidentally divisible). For suppose that A is
the time occupied by the motion B. Then if all the time has been occupied
by the whole motion, it will take less of the motion to occupy half
the time, less again to occupy a further subdivision of the time,
and so on to infinity. Again, the time will be divisible similarly
to the motion: for if the whole motion occupies all the time half
the motion will occupy half the time, and less of the motion again
will occupy less of the time. 

In the same way the being-in-motion will also be divisible. For let
G be the whole being-in-motion. Then the being-in-motion that corresponds
to half the motion will be less than the whole being-in-motion, that
which corresponds to a quarter of the motion will be less again, and
so on to infinity. Moreover by setting out successively the being-in-motion
corresponding to each of the two motions DG (say) and GE, we may argue
that the whole being-in-motion will correspond to the whole motion
(for if it were some other being-in-motion that corresponded to the
whole motion, there would be more than one being-in motion corresponding
to the same motion), the argument being the same as that whereby we
showed that the motion of a thing is divisible into the motions of
the parts of the thing: for if we take separately the being-in motion
corresponding to each of the two motions, we shall see that the whole
being-in motion is continuous. 

The same reasoning will show the divisibility of the length, and in
fact of everything that forms a sphere of change (though some of these
are only accidentally divisible because that which changes is so):
for the division of one term will involve the division of all. So,
too, in the matter of their being finite or infinite, they will all
alike be either the one or the other. And we now see that in most
cases the fact that all the terms are divisible or infinite is a direct
consequence of the fact that the thing that changes is divisible or
infinite: for the attributes 'divisible' and 'infinite' belong in
the first instance to the thing that changes. That divisibility does
so we have already shown: that infinity does so will be made clear
in what follows? 

Part 5

Since everything that changes changes from something to something,
that which has changed must at the moment when it has first changed
be in that to which it has changed. For that which changes retires
from or leaves that from which it changes: and leaving, if not identical
with changing, is at any rate a consequence of it. And if leaving
is a consequence of changing, having left is a consequence of having
changed: for there is a like relation between the two in each case.

One kind of change, then, being change in a relation of contradiction,
where a thing has changed from not-being to being it has left not-being.
Therefore it will be in being: for everything must either be or not
be. It is evident, then, that in contradictory change that which has
changed must be in that to which it has changed. And if this is true
in this kind of change, it will be true in all other kinds as well:
for in this matter what holds good in the case of one will hold good
likewise in the case of the rest. 

Moreover, if we take each kind of change separately, the truth of
our conclusion will be equally evident, on the ground that that that
which has changed must be somewhere or in something. For, since it
has left that from which it has changed and must be somewhere, it
must be either in that to which it has changed or in something else.
If, then, that which has changed to B is in something other than B,
say G, it must again be changing from G to B: for it cannot be assumed
that there is no interval between G and B, since change is continuous.
Thus we have the result that the thing that has changed, at the moment
when it has changed, is changing to that to which it has changed,
which is impossible: that which has changed, therefore, must be in
that to which it has changed. So it is evident likewise that that
that which has come to be, at the moment when it has come to be, will
be, and that which has ceased to be will not-be: for what we have
said applies universally to every kind of change, and its truth is
most obvious in the case of contradictory change. It is clear, then,
that that which has changed, at the moment when it has first changed,
is in that to which it has changed. 

We will now show that the 'primary when' in which that which has changed
effected the completion of its change must be indivisible, where by
'primary' I mean possessing the characteristics in question of itself
and not in virtue of the possession of them by something else belonging
to it. For let AG be divisible, and let it be divided at B. If then
the completion of change has been effected in AB or again in BG, AG
cannot be the primary thing in which the completion of change has
been effected. If, on the other hand, it has been changing in both
AB and BG (for it must either have changed or be changing in each
of them), it must have been changing in the whole AG: but our assumption
was that AG contains only the completion of the change. It is equally
impossible to suppose that one part of AG contains the process and
the other the completion of the change: for then we shall have something
prior to what is primary. So that in which the completion of change
has been effected must be indivisible. It is also evident, therefore,
that that that in which that which has ceased to be has ceased to
be and that in which that which has come to be has come to be are
indivisible. 

But there are two senses of the expression 'the primary when in which
something has changed'. On the one hand it may mean the primary when
containing the completion of the process of change- the moment when
it is correct to say 'it has changed': on the other hand it may mean
the primary when containing the beginning of the process of change.
Now the primary when that has reference to the end of the change is
something really existent: for a change may really be completed, and
there is such a thing as an end of change, which we have in fact shown
to be indivisible because it is a limit. But that which has reference
to the beginning is not existent at all: for there is no such thing
as a beginning of a process of change, and the time occupied by the
change does not contain any primary when in which the change began.
For suppose that AD is such a primary when. Then it cannot be indivisible:
for, if it were, the moment immediately preceding the change and the
moment in which the change begins would be consecutive (and moments
cannot be consecutive). Again, if the changing thing is at rest in
the whole preceding time GA (for we may suppose that it is at rest),
it is at rest in A also: so if AD is without parts, it will simultaneously
be at rest and have changed: for it is at rest in A and has changed
in D. Since then AD is not without parts, it must be divisible, and
the changing thing must have changed in every part of it (for if it
has changed in neither of the two parts into which AD is divided,
it has not changed in the whole either: if, on the other hand, it
is in process of change in both parts, it is likewise in process of
change in the whole: and if, again, it has changed in one of the two
parts, the whole is not the primary when in which it has changed:
it must therefore have changed in every part). It is evident, then,
that with reference to the beginning of change there is no primary
when in which change has been effected: for the divisions are infinite.

So, too, of that which has changed there is no primary part that has
changed. For suppose that of AE the primary part that has changed
is Az (everything that changes having been shown to be divisible):
and let OI be the time in which DZ has changed. If, then, in the whole
time DZ has changed, in half the time there will be a part that has
changed, less than and therefore prior to DZ: and again there will
be another part prior to this, and yet another, and so on to infinity.
Thus of that which changes there cannot be any primary part that has
changed. It is evident, then, from what has been said, that neither
of that which changes nor of the time in which it changes is there
any primary part. 

With regard, however, to the actual subject of change-that is to say
that in respect of which a thing changes-there is a difference to
be observed. For in a process of change we may distinguish three terms-that
which changes, that in which it changes, and the actual subject of
change, e.g. the man, the time, and the fair complexion. Of these
the man and the time are divisible: but with the fair complexion it
is otherwise (though they are all divisible accidentally, for that
in which the fair complexion or any other quality is an accident is
divisible). For of actual subjects of change it will be seen that
those which are classed as essentially, not accidentally, divisible
have no primary part. Take the case of magnitudes: let AB be a magnitude,
and suppose that it has moved from B to a primary 'where' G. Then
if BG is taken to be indivisible, two things without parts will have
to be contiguous (which is impossible): if on the other hand it is
taken to be divisible, there will be something prior to G to which
the magnitude has changed, and something else again prior to that,
and so on to infinity, because the process of division may be continued
without end. Thus there can be no primary 'where' to which a thing
has changed. And if we take the case of quantitative change, we shall
get a like result, for here too the change is in something continuous.
It is evident, then, that only in qualitative motion can there be
anything essentially indivisible. 

Part 6

Now everything that changes changes time, and that in two senses:
for the time in which a thing is said to change may be the primary
time, or on the other hand it may have an extended reference, as e.g.
when we say that a thing changes in a particular year because it changes
in a particular day. That being so, that which changes must be changing
in any part of the primary time in which it changes. This is clear
from our definition of 'primary', in which the word is said to express
just this: it may also, however, be made evident by the following
argument. Let ChRh be the primary time in which that which is in motion
is in motion: and (as all time is divisible) let it be divided at
K. Now in the time ChK it either is in motion or is not in motion,
and the same is likewise true of the time KRh. Then if it is in motion
in neither of the two parts, it will be at rest in the whole: for
it is impossible that it should be in motion in a time in no part
of which it is in motion. If on the other hand it is in motion in
only one of the two parts of the time, ChRh cannot be the primary
time in which it is in motion: for its motion will have reference
to a time other than ChRh. It must, then, have been in motion in any
part of ChRh. 

And now that this has been proved, it is evident that everything that
is in motion must have been in motion before. For if that which is
in motion has traversed the distance KL in the primary time ChRh,
in half the time a thing that is in motion with equal velocity and
began its motion at the same time will have traversed half the distance.
But if this second thing whose velocity is equal has traversed a certain
distance in a certain time, the original thing that is in motion must
have traversed the same distance in the same time. Hence that which
is in motion must have been in motion before. 

Again, if by taking the extreme moment of the time-for it is the moment
that defines the time, and time is that which is intermediate between
moments-we are enabled to say that motion has taken place in the whole
time ChRh or in fact in any period of it, motion may likewise be said
to have taken place in every other such period. But half the time
finds an extreme in the point of division. Therefore motion will have
taken place in half the time and in fact in any part of it: for as
soon as any division is made there is always a time defined by moments.
If, then, all time is divisible, and that which is intermediate between
moments is time, everything that is changing must have completed an
infinite number of changes. 

Again, since a thing that changes continuously and has not perished
or ceased from its change must either be changing or have changed
in any part of the time of its change, and since it cannot be changing
in a moment, it follows that it must have changed at every moment
in the time: consequently, since the moments are infinite in number,
everything that is changing must have completed an infinite number
of changes. 

And not only must that which is changing have changed, but that which
has changed must also previously have been changing, since everything
that has changed from something to something has changed in a period
of time. For suppose that a thing has changed from A to B in a moment.
Now the moment in which it has changed cannot be the same as that
in which it is at A (since in that case it would be in A and B at
once): for we have shown above that that that which has changed, when
it has changed, is not in that from which it has changed. If, on the
other hand, it is a different moment, there will be a period of time
intermediate between the two: for, as we saw, moments are not consecutive.
Since, then, it has changed in a period of time, and all time is divisible,
in half the time it will have completed another change, in a quarter
another, and so on to infinity: consequently when it has changed,
it must have previously been changing. 

Moreover, the truth of what has been said is more evident in the case
of magnitude, because the magnitude over which what is changing changes
is continuous. For suppose that a thing has changed from G to D. Then
if GD is indivisible, two things without parts will be consecutive.
But since this is impossible, that which is intermediate between them
must be a magnitude and divisible into an infinite number of segments:
consequently, before the change is completed, the thing changes to
those segments. Everything that has changed, therefore, must previously
have been changing: for the same proof also holds good of change with
respect to what is not continuous, changes, that is to say, between
contraries and between contradictories. In such cases we have only
to take the time in which a thing has changed and again apply the
same reasoning. So that which has changed must have been changing
and that which is changing must have changed, and a process of change
is preceded by a completion of change and a completion by a process:
and we can never take any stage and say that it is absolutely the
first. The reason of this is that no two things without parts can
be contiguous, and therefore in change the process of division is
infinite, just as lines may be infinitely divided so that one part
is continually increasing and the other continually decreasing.

So it is evident also that that that which has become must previously
have been in process of becoming, and that which is in process of
becoming must previously have become, everything (that is) that is
divisible and continuous: though it is not always the actual thing
that is in process of becoming of which this is true: sometimes it
is something else, that is to say, some part of the thing in question,
e.g. the foundation-stone of a house. So, too, in the case of that
which is perishing and that which has perished: for that which becomes
and that which perishes must contain an element of infiniteness as
an immediate consequence of the fact that they are continuous things:
and so a thing cannot be in process of becoming without having become
or have become without having been in process of becoming. So, too,
in the case of perishing and having perished: perishing must be preceded
by having perished, and having perished must be preceded by perishing.
It is evident, then, that that which has become must previously have
been in process of becoming, and that which is in process of becoming
must previously have become: for all magnitudes and all periods of
time are infinitely divisible. 

Consequently no absolutely first stage of change can be represented
by any particular part of space or time which the changing thing may
occupy. 

Part 7

Now since the motion of everything that is in motion occupies a period
of time, and a greater magnitude is traversed in a longer time, it
is impossible that a thing should undergo a finite motion in an infinite
time, if this is understood to mean not that the same motion or a
part of it is continually repeated, but that the whole infinite time
is occupied by the whole finite motion. In all cases where a thing
is in motion with uniform velocity it is clear that the finite magnitude
is traversed in a finite time. For if we take a part of the motion
which shall be a measure of the whole, the whole motion is completed
in as many equal periods of the time as there are parts of the motion.
Consequently, since these parts are finite, both in size individually
and in number collectively, the whole time must also be finite: for
it will be a multiple of the portion, equal to the time occupied in
completing the aforesaid part multiplied by the number of the parts.

But it makes no difference even if the velocity is not uniform. For
let us suppose that the line AB represents a finite stretch over which
a thing has been moved in the given time, and let GD be the infinite
time. Now if one part of the stretch must have been traversed before
another part (this is clear, that in the earlier and in the later
part of the time a different part of the stretch has been traversed:
for as the time lengthens a different part of the motion will always
be completed in it, whether the thing in motion changes with uniform
velocity or not: and whether the rate of motion increases or diminishes
or remains stationary this is none the less so), let us then take
AE a part of the whole stretch of motion AB which shall be a measure
of AB. Now this part of the motion occupies a certain period of the
infinite time: it cannot itself occupy an infinite time, for we are
assuming that that is occupied by the whole AB. And if again I take
another part equal to AE, that also must occupy a finite time in consequence
of the same assumption. And if I go on taking parts in this way, on
the one hand there is no part which will be a measure of the infinite
time (for the infinite cannot be composed of finite parts whether
equal or unequal, because there must be some unity which will be a
measure of things finite in multitude or in magnitude, which, whether
they are equal or unequal, are none the less limited in magnitude);
while on the other hand the finite stretch of motion AB is a certain
multiple of AE: consequently the motion AB must be accomplished in
a finite time. Moreover it is the same with coming to rest as with
motion. And so it is impossible for one and the same thing to be infinitely
in process of becoming or of perishing. The reasoning he will prove
that in a finite time there cannot be an infinite extent of motion
or of coming to rest, whether the motion is regular or irregular.
For if we take a part which shall be a measure of the whole time,
in this part a certain fraction, not the whole, of the magnitude will
be traversed, because we assume that the traversing of the whole occupies
all the time. Again, in another equal part of the time another part
of the magnitude will be traversed: and similarly in each part of
the time that we take, whether equal or unequal to the part originally
taken. It makes no difference whether the parts are equal or not,
if only each is finite: for it is clear that while the time is exhausted
by the subtraction of its parts, the infinite magnitude will not be
thus exhausted, since the process of subtraction is finite both in
respect of the quantity subtracted and of the number of times a subtraction
is made. Consequently the infinite magnitude will not be traversed
in finite time: and it makes no difference whether the magnitude is
infinite in only one direction or in both: for the same reasoning
will hold good. 

This having been proved, it is evident that neither can a finite magnitude
traverse an infinite magnitude in a finite time, the reason being
the same as that given above: in part of the time it will traverse
a finite magnitude and in each several part likewise, so that in the
whole time it will traverse a finite magnitude. 

And since a finite magnitude will not traverse an infinite in a finite
time, it is clear that neither will an infinite traverse a finite
in a finite time. For if the infinite could traverse the finite, the
finite could traverse the infinite; for it makes no difference which
of the two is the thing in motion; either case involves the traversing
of the infinite by the finite. For when the infinite magnitude A is
in motion a part of it, say GD, will occupy the finite and then another,
and then another, and so on to infinity. Thus the two results will
coincide: the infinite will have completed a motion over the finite
and the finite will have traversed the infinite: for it would seem
to be impossible for the motion of the infinite over the finite to
occur in any way other than by the finite traversing the infinite
either by locomotion over it or by measuring it. Therefore, since
this is impossible, the infinite cannot traverse the finite.

Nor again will the infinite traverse the infinite in a finite time.
Otherwise it would also traverse the finite, for the infinite includes
the finite. We can further prove this in the same way by taking the
time as our starting-point. 

Since, then, it is established that in a finite time neither will
the finite traverse the infinite, nor the infinite the finite, nor
the infinite the infinite, it is evident also that in a finite time
there cannot be infinite motion: for what difference does it make
whether we take the motion or the magnitude to be infinite? If either
of the two is infinite, the other must be so likewise: for all locomotion
is in space. 

Part 8

Since everything to which motion or rest is natural is in motion or
at rest in the natural time, place, and manner, that which is coming
to a stand, when it is coming to a stand, must be in motion: for if
it is not in motion it must be at rest: but that which is at rest
cannot be coming to rest. From this it evidently follows that coming
to a stand must occupy a period of time: for the motion of that which
is in motion occupies a period of time, and that which is coming to
a stand has been shown to be in motion: consequently coming to a stand
must occupy a period of time. 

Again, since the terms 'quicker' and 'slower' are used only of that
which occupies a period of time, and the process of coming to a stand
may be quicker or slower, the same conclusion follows. 

And that which is coming to a stand must be coming to a stand in any
part of the primary time in which it is coming to a stand. For if
it is coming to a stand in neither of two parts into which the time
may be divided, it cannot be coming to a stand in the whole time,
with the result that that that which is coming to a stand will not
be coming to a stand. If on the other hand it is coming to a stand
in only one of the two parts of the time, the whole cannot be the
primary time in which it is coming to a stand: for it is coming to
a stand in the whole time not primarily but in virtue of something
distinct from itself, the argument being the same as that which we
used above about things in motion. 

And just as there is no primary time in which that which is in motion
is in motion, so too there is no primary time in which that which
is coming to a stand is coming to a stand, there being no primary
stage either of being in motion or of coming to a stand. For let AB
be the primary time in which a thing is coming to a stand. Now AB
cannot be without parts: for there cannot be motion in that which
is without parts, because the moving thing would necessarily have
been already moved for part of the time of its movement: and that
which is coming to a stand has been shown to be in motion. But since
Ab is therefore divisible, the thing is coming to a stand in every
one of the parts of AB: for we have shown above that it is coming
to a stand in every one of the parts in which it is primarily coming
to a stand. Since then, that in which primarily a thing is coming
to a stand must be a period of time and not something indivisible,
and since all time is infinitely divisible, there cannot be anything
in which primarily it is coming to a stand. 

Nor again can there be a primary time at which the being at rest of
that which is at rest occurred: for it cannot have occurred in that
which has no parts, because there cannot be motion in that which is
indivisible, and that in which rest takes place is the same as that
in which motion takes place: for we defined a state of rest to be
the state of a thing to which motion is natural but which is not in
motion when (that is to say in that in which) motion would be natural
to it. Again, our use of the phrase 'being at rest' also implies that
the previous state of a thing is still unaltered, not one point only
but two at least being thus needed to determine its presence: consequently
that in which a thing is at rest cannot be without parts. Since, then
it is divisible, it must be a period of time, and the thing must be
at rest in every one of its parts, as may be shown by the same method
as that used above in similar demonstrations. 

So there can be no primary part of the time: and the reason is that
rest and motion are always in a period of time, and a period of time
has no primary part any more than a magnitude or in fact anything
continuous: for everything continuous is divisible into an infinite
number of parts. 

And since everything that is in motion is in motion in a period of
time and changes from something to something, when its motion is comprised
within a particular period of time essentially-that is to say when
it fills the whole and not merely a part of the time in question-it
is impossible that in that time that which is in motion should be
over against some particular thing primarily. For if a thing-itself
and each of its parts-occupies the same space for a definite period
of time, it is at rest: for it is in just these circumstances that
we use the term 'being at rest'-when at one moment after another it
can be said with truth that a thing, itself and its parts, occupies
the same space. So if this is being at rest it is impossible for that
which is changing to be as a whole, at the time when it is primarily
changing, over against any particular thing (for the whole period
of time is divisible), so that in one part of it after another it
will be true to say that the thing, itself and its parts, occupies
the same space. If this is not so and the aforesaid proposition is
true only at a single moment, then the thing will be over against
a particular thing not for any period of time but only at a moment
that limits the time. It is true that at any moment it is always over
against something stationary: but it is not at rest: for at a moment
it is not possible for anything to be either in motion or at rest.
So while it is true to say that that which is in motion is at a moment
not in motion and is opposite some particular thing, it cannot in
a period of time be over against that which is at rest: for that would
involve the conclusion that that which is in locomotion is at rest.

Part 9

Zeno's reasoning, however, is fallacious, when he says that if everything
when it occupies an equal space is at rest, and if that which is in
locomotion is always occupying such a space at any moment, the flying
arrow is therefore motionless. This is false, for time is not composed
of indivisible moments any more than any other magnitude is composed
of indivisibles. 

Zeno's arguments about motion, which cause so much disquietude to
those who try to solve the problems that they present, are four in
number. The first asserts the non-existence of motion on the ground
that that which is in locomotion must arrive at the half-way stage
before it arrives at the goal. This we have discussed above.

The second is the so-called 'Achilles', and it amounts to this, that
in a race the quickest runner can never overtake the slowest, since
the pursuer must first reach the point whence the pursued started,
so that the slower must always hold a lead. This argument is the same
in principle as that which depends on bisection, though it differs
from it in that the spaces with which we successively have to deal
are not divided into halves. The result of the argument is that the
slower is not overtaken: but it proceeds along the same lines as the
bisection-argument (for in both a division of the space in a certain
way leads to the result that the goal is not reached, though the 'Achilles'
goes further in that it affirms that even the quickest runner in legendary
tradition must fail in his pursuit of the slowest), so that the solution
must be the same. And the axiom that that which holds a lead is never
overtaken is false: it is not overtaken, it is true, while it holds
a lead: but it is overtaken nevertheless if it is granted that it
traverses the finite distance prescribed. These then are two of his
arguments. 

The third is that already given above, to the effect that the flying
arrow is at rest, which result follows from the assumption that time
is composed of moments: if this assumption is not granted, the conclusion
will not follow. 

The fourth argument is that concerning the two rows of bodies, each
row being composed of an equal number of bodies of equal size, passing
each other on a race-course as they proceed with equal velocity in
opposite directions, the one row originally occupying the space between
the goal and the middle point of the course and the other that between
the middle point and the starting-post. This, he thinks, involves
the conclusion that half a given time is equal to double that time.
The fallacy of the reasoning lies in the assumption that a body occupies
an equal time in passing with equal velocity a body that is in motion
and a body of equal size that is at rest; which is false. For instance
(so runs the argument), let A, A...be the stationary bodies of equal
size, B, B...the bodies, equal in number and in size to A, A...,originally
occupying the half of the course from the starting-post to the middle
of the A's, and G, G...those originally occupying the other half from
the goal to the middle of the A's, equal in number, size, and velocity
to B, B....Then three consequences follow: 

First, as the B's and the G's pass one another, the first B reaches
the last G at the same moment as the first G reaches the last B. Secondly
at this moment the first G has passed all the A's, whereas the first
B has passed only half the A's, and has consequently occupied only
half the time occupied by the first G, since each of the two occupies
an equal time in passing each A. Thirdly, at the same moment all the
B's have passed all the G's: for the first G and the first B will
simultaneously reach the opposite ends of the course, since (so says
Zeno) the time occupied by the first G in passing each of the B's
is equal to that occupied by it in passing each of the A's, because
an equal time is occupied by both the first B and the first G in passing
all the A's. This is the argument, but it presupposed the aforesaid
fallacious assumption. 

Nor in reference to contradictory change shall we find anything unanswerable
in the argument that if a thing is changing from not-white, say, to
white, and is in neither condition, then it will be neither white
nor not-white: for the fact that it is not wholly in either condition
will not preclude us from calling it white or not-white. We call a
thing white or not-white not necessarily because it is be one or the
other, but cause most of its parts or the most essential parts of
it are so: not being in a certain condition is different from not
being wholly in that condition. So, too, in the case of being and
not-being and all other conditions which stand in a contradictory
relation: while the changing thing must of necessity be in one of
the two opposites, it is never wholly in either. 

Again, in the case of circles and spheres and everything whose motion
is confined within the space that it occupies, it is not true to say
the motion can be nothing but rest, on the ground that such things
in motion, themselves and their parts, will occupy the same position
for a period of time, and that therefore they will be at once at rest
and in motion. For in the first place the parts do not occupy the
same position for any period of time: and in the second place the
whole also is always changing to a different position: for if we take
the orbit as described from a point A on a circumference, it will
not be the same as the orbit as described from B or G or any other
point on the same circumference except in an accidental sense, the
sense that is to say in which a musical man is the same as a man.
Thus one orbit is always changing into another, and the thing will
never be at rest. And it is the same with the sphere and everything
else whose motion is confined within the space that it occupies.

Part 10

Our next point is that that which is without parts cannot be in motion
except accidentally: i.e. it can be in motion only in so far as the
body or the magnitude is in motion and the partless is in motion by
inclusion therein, just as that which is in a boat may be in motion
in consequence of the locomotion of the boat, or a part may be in
motion in virtue of the motion of the whole. (It must be remembered,
however, that by 'that which is without parts' I mean that which is
quantitatively indivisible (and that the case of the motion of a part
is not exactly parallel): for parts have motions belonging essentially
and severally to themselves distinct from the motion of the whole.
The distinction may be seen most clearly in the case of a revolving
sphere, in which the velocities of the parts near the centre and of
those on the surface are different from one another and from that
of the whole; this implies that there is not one motion but many).
As we have said, then, that which is without parts can be in motion
in the sense in which a man sitting in a boat is in motion when the
boat is travelling, but it cannot be in motion of itself. For suppose
that it is changing from AB to BG-either from one magnitude to another,
or from one form to another, or from some state to its contradictory-and
let D be the primary time in which it undergoes the change. Then in
the time in which it is changing it must be either in AB or in BG
or partly in one and partly in the other: for this, as we saw, is
true of everything that is changing. Now it cannot be partly in each
of the two: for then it would be divisible into parts. Nor again can
it be in BG: for then it will have completed the change, whereas the
assumption is that the change is in process. It remains, then, that
in the time in which it is changing, it is in Ab. That being so, it
will be at rest: for, as we saw, to be in the same condition for a
period of time is to be at rest. So it is not possible for that which
has no parts to be in motion or to change in any way: for only one
condition could have made it possible for it to have motion, viz.
that time should be composed of moments, in which case at any moment
it would have completed a motion or a change, so that it would never
be in motion, but would always have been in motion. But this we have
already shown above to be impossible: time is not composed of moments,
just as a line is not composed of points, and motion is not composed
of starts: for this theory simply makes motion consist of indivisibles
in exactly the same way as time is made to consist of moments or a
length of points. 

Again, it may be shown in the following way that there can be no motion
of a point or of any other indivisible. That which is in motion can
never traverse a space greater than itself without first traversing
a space equal to or less than itself. That being so, it is evident
that the point also must first traverse a space equal to or less than
itself. But since it is indivisible, there can be no space less than
itself for it to traverse first: so it will have to traverse a distance
equal to itself. Thus the line will be composed of points, for the
point, as it continually traverses a distance equal to itself, will
be a measure of the whole line. But since this is impossible, it is
likewise impossible for the indivisible to be in motion.

Again, since motion is always in a period of time and never in a moment,
and all time is divisible, for everything that is in motion there
must be a time less than that in which it traverses a distance as
great as itself. For that in which it is in motion will be a time,
because all motion is in a period of time; and all time has been shown
above to be divisible. Therefore, if a point is in motion, there must
be a time less than that in which it has itself traversed any distance.
But this is impossible, for in less time it must traverse less distance,
and thus the indivisible will be divisible into something less than
itself, just as the time is so divisible: the fact being that the
only condition under which that which is without parts and indivisible
could be in motion would have been the possibility of the infinitely
small being in motion in a moment: for in the two questions-that of
motion in a moment and that of motion of something indivisible-the
same principle is involved. 

Our next point is that no process of change is infinite: for every
change, whether between contradictories or between contraries, is
a change from something to something. Thus in contradictory changes
the positive or the negative, as the case may be, is the limit, e.g.
being is the limit of coming to be and not-being is the limit of ceasing
to be: and in contrary changes the particular contraries are the limits,
since these are the extreme points of any such process of change,
and consequently of every process of alteration: for alteration is
always dependent upon some contraries. Similarly contraries are the
extreme points of processes of increase and decrease: the limit of
increase is to be found in the complete magnitude proper to the peculiar
nature of the thing that is increasing, while the limit of decrease
is the complete loss of such magnitude. Locomotion, it is true, we
cannot show to be finite in this way, since it is not always between
contraries. But since that which cannot be cut (in the sense that
it is inconceivable that it should be cut, the term 'cannot' being
used in several senses)-since it is inconceivable that that which
in this sense cannot be cut should be in process of being cut, and
generally that that which cannot come to be should be in process of
coming to be, it follows that it is inconceivable that that which
cannot complete a change should be in process of changing to that
to which it cannot complete a change. If, then, it is to be assumed
that that which is in locomotion is in process of changing, it must
be capable of completing the change. Consequently its motion is not
infinite, and it will not be in locomotion over an infinite distance,
for it cannot traverse such a distance. 

It is evident, then, that a process of change cannot be infinite in
the sense that it is not defined by limits. But it remains to be considered
whether it is possible in the sense that one and the same process
of change may be infinite in respect of the time which it occupies.
If it is not one process, it would seem that there is nothing to prevent
its being infinite in this sense; e.g. if a process of locomotion
be succeeded by a process of alteration and that by a process of increase
and that again by a process of coming to be: in this way there may
be motion for ever so far as the time is concerned, but it will not
be one motion, because all these motions do not compose one. If it
is to be one process, no motion can be infinite in respect of the
time that it occupies, with the single exception of rotatory locomotion.

----------------------------------------------------------------------

BOOK VII

Part 1 

Everything that is in motion must be moved by something. For if it
has not the source of its motion in itself it is evident that it is
moved by something other than itself, for there must be something
else that moves it. If on the other hand it has the source of its
motion in itself, let AB be taken to represent that which is in motion
essentially of itself and not in virtue of the fact that something
belonging to it is in motion. Now in the first place to assume that
Ab, because it is in motion as a whole and is not moved by anything
external to itself, is therefore moved by itself-this is just as if,
supposing that KL is moving LM and is also itself in motion, we were
to deny that KM is moved by anything on the ground that it is not
evident which is the part that is moving it and which the part that
is moved. In the second place that which is in motion without being
moved by anything does not necessarily cease from its motion because
something else is at rest, but a thing must be moved by something
if the fact of something else having ceased from its motion causes
it to be at rest. Thus, if this is accepted, everything that is in
motion must be moved by something. For AB, which has been taken to
represent that which is in motion, must be divisible since everything
that is in motion is divisible. Let it be divided, then, at G. Now
if GB is not in motion, then AB will not be in motion: for if it is,
it is clear that AG would be in motion while BG is at rest, and thus
AB cannot be in motion essentially and primarily. But ex hypothesi
AB is in motion essentially and primarily. Therefore if GB is not
in motion AB will be at rest. But we have agreed that that which is
at rest if something else is not in motion must be moved by something.
Consequently, everything that is in motion must be moved by something:
for that which is in motion will always be divisible, and if a part
of it is not in motion the whole must be at rest. 

Since everything that is in motion must be moved by something, let
us take the case in which a thing is in locomotion and is moved by
something that is itself in motion, and that again is moved by something
else that is in motion, and that by something else, and so on continually:
then the series cannot go on to infinity, but there must be some first
movent. For let us suppose that this is not so and take the series
to be infinite. Let A then be moved by B, B by G, G by D, and so on,
each member of the series being moved by that which comes next to
it. Then since ex hypothesi the movent while causing motion is also
itself in motion, and the motion of the moved and the motion of the
movent must proceed simultaneously (for the movent is causing motion
and the moved is being moved simultaneously) it is evident that the
respective motions of A, B, G, and each of the other moved movents
are simultaneous. Let us take the motion of each separately and let
E be the motion of A, Z of B, and H and O respectively the motions
of G and D: for though they are all moved severally one by another,
yet we may still take the motion of each as numerically one, since
every motion is from something to something and is not infinite in
respect of its extreme points. By a motion that is numerically one
I mean a motion that proceeds from something numerically one and the
same to something numerically one and the same in a period of time
numerically one and the same: for a motion may be the same generically,
specifically, or numerically: it is generically the same if it belongs
to the same category, e.g. substance or quality: it is specifically
the same if it proceeds from something specifically the same to something
specifically the same, e.g. from white to black or from good to bad,
which is not of a kind specifically distinct: it is numerically the
same if it proceeds from something numerically one to something numerically
one in the same period of time, e.g. from a particular white to a
particular black, or from a particular place to a particular place,
in a particular period of time: for if the period of time were not
one and the same, the motion would no longer be numerically one though
it would still be specifically one. 

We have dealt with this question above. Now let us further take the
time in which A has completed its motion, and let it be represented
by K. Then since the motion of A is finite the time will also be finite.
But since the movents and the things moved are infinite, the motion
EZHO, i.e. the motion that is composed of all the individual motions,
must be infinite. For the motions of A, B, and the others may be equal,
or the motions of the others may be greater: but assuming what is
conceivable, we find that whether they are equal or some are greater,
in both cases the whole motion is infinite. And since the motion of
A and that of each of the others are simultaneous, the whole motion
must occupy the same time as the motion of A: but the time occupied
by the motion of A is finite: consequently the motion will be infinite
in a finite time, which is impossible. 

It might be thought that what we set out to prove has thus been shown,
but our argument so far does not prove it, because it does not yet
prove that anything impossible results from the contrary supposition:
for in a finite time there may be an infinite motion, though not of
one thing, but of many: and in the case that we are considering this
is so: for each thing accomplishes its own motion, and there is no
impossibility in many things being in motion simultaneously. But if
(as we see to be universally the case) that which primarily is moved
locally and corporeally must be either in contact with or continuous
with that which moves it, the things moved and the movents must be
continuous or in contact with one another, so that together they all
form a single unity: whether this unity is finite or infinite makes
no difference to our present argument; for in any case since the things
in motion are infinite in number the whole motion will be infinite,
if, as is theoretically possible, each motion is either equal to or
greater than that which follows it in the series: for we shall take
as actual that which is theoretically possible. If, then, A, B, G,
D form an infinite magnitude that passes through the motion EZHO in
the finite time K, this involves the conclusion that an infinite motion
is passed through in a finite time: and whether the magnitude in question
is finite or infinite this is in either case impossible. Therefore
the series must come to an end, and there must be a first movent and
a first moved: for the fact that this impossibility results only from
the assumption of a particular case is immaterial, since the case
assumed is theoretically possible, and the assumption of a theoretically
possible case ought not to give rise to any impossible result.

Part 2

That which is the first movement of a thing-in the sense that it supplies
not 'that for the sake of which' but the source of the motion-is always
together with that which is moved by it by 'together' I mean that
there is nothing intermediate between them). This is universally true
wherever one thing is moved by another. And since there are three
kinds of motion, local, qualitative, and quantitative, there must
also be three kinds of movent, that which causes locomotion, that
which causes alteration, and that which causes increase or decrease.

Let us begin with locomotion, for this is the primary motion. Everything
that is in locomotion is moved either by itself or by something else.
In the case of things that are moved by themselves it is evident that
the moved and the movent are together: for they contain within themselves
their first movent, so that there is nothing in between. The motion
of things that are moved by something else must proceed in one of
four ways: for there are four kinds of locomotion caused by something
other than that which is in motion, viz. pulling, pushing, carrying,
and twirling. All forms of locomotion are reducible to these. Thus
pushing on is a form of pushing in which that which is causing motion
away from itself follows up that which it pushes and continues to
push it: pushing off occurs when the movent does not follow up the
thing that it has moved: throwing when the movent causes a motion
away from itself more violent than the natural locomotion of the thing
moved, which continues its course so long as it is controlled by the
motion imparted to it. Again, pushing apart and pushing together are
forms respectively of pushing off and pulling: pushing apart is pushing
off, which may be a motion either away from the pusher or away from
something else, while pushing together is pulling, which may be a
motion towards something else as well as the puller. We may similarly
classify all the varieties of these last two, e.g. packing and combing:
the former is a form of pushing together, the latter a form of pushing
apart. The same is true of the other processes of combination and
separation (they will all be found to be forms of pushing apart or
of pushing together), except such as are involved in the processes
of becoming and perishing. (At same time it is evident that there
is no other kind of motion but combination and separation: for they
may all be apportioned to one or other of those already mentioned.)
Again, inhaling is a form of pulling, exhaling a form of pushing:
and the same is true of spitting and of all other motions that proceed
through the body, whether secretive or assimilative, the assimilative
being forms of pulling, the secretive of pushing off. All other kinds
of locomotion must be similarly reduced, for they all fall under one
or other of our four heads. And again, of these four, carrying and
twirling are to pulling and pushing. For carrying always follows one
of the other three methods, for that which is carried is in motion
accidentally, because it is in or upon something that is in motion,
and that which carries it is in doing so being either pulled or pushed
or twirled; thus carrying belongs to all the other three kinds of
motion in common. And twirling is a compound of pulling and pushing,
for that which is twirling a thing must be pulling one part of the
thing and pushing another part, since it impels one part away from
itself and another part towards itself. If, therefore, it can be shown
that that which is pushing and that which is pushing and pulling are
adjacent respectively to that which is being pushed and that which
is being pulled, it will be evident that in all locomotion there is
nothing intermediate between moved and movent. But the former fact
is clear even from the definitions of pushing and pulling, for pushing
is motion to something else from oneself or from something else, and
pulling is motion from something else to oneself or to something else,
when the motion of that which is pulling is quicker than the motion
that would separate from one another the two things that are continuous:
for it is this that causes one thing to be pulled on along with the
other. (It might indeed be thought that there is a form of pulling
that arises in another way: that wood, e.g. pulls fire in a manner
different from that described above. But it makes no difference whether
that which pulls is in motion or is stationary when it is pulling:
in the latter case it pulls to the place where it is, while in the
former it pulls to the place where it was.) Now it is impossible to
move anything either from oneself to something else or something else
to oneself without being in contact with it: it is evident, therefore,
that in all locomotion there is nothing intermediate between moved
and movent. 

Nor again is there anything intermediate between that which undergoes
and that which causes alteration: this can be proved by induction:
for in every case we find that the respective extremities of that
which causes and that which undergoes alteration are adjacent. For
our assumption is that things that are undergoing alteration are altered
in virtue of their being affected in respect of their so-called affective
qualities, since that which is of a certain quality is altered in
so far as it is sensible, and the characteristics in which bodies
differ from one another are sensible characteristics: for every body
differs from another in possessing a greater or lesser number of sensible
characteristics or in possessing the same sensible characteristics
in a greater or lesser degree. But the alteration of that which undergoes
alteration is also caused by the above-mentioned characteristics,
which are affections of some particular underlying quality. Thus we
say that a thing is altered by becoming hot or sweet or thick or dry
or white: and we make these assertions alike of what is inanimate
and of what is animate, and further, where animate things are in question,
we make them both of the parts that have no power of sense-perception
and of the senses themselves. For in a way even the senses undergo
alteration, since the active sense is a motion through the body in
the course of which the sense is affected in a certain way. We see,
then, that the animate is capable of every kind of alteration of which
the inanimate is capable: but the inanimate is not capable of every
kind of alteration of which the animate is capable, since it is not
capable of alteration in respect of the senses: moreover the inanimate
is unconscious of being affected by alteration, whereas the animate
is conscious of it, though there is nothing to prevent the animate
also being unconscious of it when the process of the alteration does
not concern the senses. Since, then, the alteration of that which
undergoes alteration is caused by sensible things, in every case of
such alteration it is evident that the respective extremities of that
which causes and that which undergoes alteration are adjacent. Thus
the air is continuous with that which causes the alteration, and the
body that undergoes alteration is continuous with the air. Again,
the colour is continuous with the light and the light with the sight.
And the same is true of hearing and smelling: for the primary movent
in respect to the moved is the air. Similarly, in the case of tasting,
the flavour is adjacent to the sense of taste. And it is just the
same in the case of things that are inanimate and incapable of sense-perception.
Thus there can be nothing intermediate between that which undergoes
and that which causes alteration. 

Nor, again, can there be anything intermediate between that which
suffers and that which causes increase: for the part of the latter
that starts the increase does so by becoming attached in such a way
to the former that the whole becomes one. Again, the decrease of that
which suffers decrease is caused by a part of the thing becoming detached.
So that which causes increase and that which causes decrease must
be continuous with that which suffers increase and that which suffers
decrease respectively: and if two things are continuous with one another
there can be nothing intermediate between them. 

It is evident, therefore, that between the extremities of the moved
and the movent that are respectively first and last in reference to
the moved there is nothing intermediate. 

Part 3

Everything, we say, that undergoes alteration is altered by sensible
causes, and there is alteration only in things that are said to be
essentially affected by sensible things. The truth of this is to be
seen from the following considerations. Of all other things it would
be most natural to suppose that there is alteration in figures and
shapes, and in acquired states and in the processes of acquiring and
losing these: but as a matter of fact in neither of these two classes
of things is there alteration. 

In the first place, when a particular formation of a thing is completed,
we do not call it by the name of its material: e.g. we do not call
the statue 'bronze' or the pyramid 'wax' or the bed 'wood', but we
use a derived expression and call them 'of bronze', 'waxen', and 'wooden'
respectively. But when a thing has been affected and altered in any
way we still call it by the original name: thus we speak of the bronze
or the wax being dry or fluid or hard or hot. 

And not only so: we also speak of the particular fluid or hot substance
as being bronze, giving the material the same name as that which we
use to describe the affection. 

Since, therefore, having regard to the figure or shape of a thing
we no longer call that which has become of a certain figure by the
name of the material that exhibits the figure, whereas having regard
to a thing's affections or alterations we still call it by the name
of its material, it is evident that becomings of the former kind cannot
be alterations. 

Moreover it would seem absurd even to speak in this way, to speak,
that is to say, of a man or house or anything else that has come into
existence as having been altered. Though it may be true that every
such becoming is necessarily the result of something's being altered,
the result, e.g. of the material's being condensed or rarefied or
heated or cooled, nevertheless it is not the things that are coming
into existence that are altered, and their becoming is not an alteration.

Again, acquired states, whether of the body or of the soul, are not
alterations. For some are excellences and others are defects, and
neither excellence nor defect is an alteration: excellence is a perfection
(for when anything acquires its proper excellence we call it perfect,
since it is then if ever that we have a thing in its natural state:
e.g. we have a perfect circle when we have one as good as possible),
while defect is a perishing of or departure from this condition. So
as when speaking of a house we do not call its arrival at perfection
an alteration (for it would be absurd to suppose that the coping or
the tiling is an alteration or that in receiving its coping or its
tiling a house is altered and not perfected), the same also holds
good in the case of excellences and defects and of the persons or
things that possess or acquire them: for excellences are perfections
of a thing's nature and defects are departures from it: consequently
they are not alterations. 

Further, we say that all excellences depend upon particular relations.
Thus bodily excellences such as health and a good state of body we
regard as consisting in a blending of hot and cold elements within
the body in due proportion, in relation either to one another or to
the surrounding atmosphere: and in like manner we regard beauty, strength,
and all the other bodily excellences and defects. Each of them exists
in virtue of a particular relation and puts that which possesses it
in a good or bad condition with regard to its proper affections, where
by 'proper' affections I mean those influences that from the natural
constitution of a thing tend to promote or destroy its existence.
Since then, relatives are neither themselves alterations nor the subjects
of alteration or of becoming or in fact of any change whatever, it
is evident that neither states nor the processes of losing and acquiring
states are alterations, though it may be true that their becoming
or perishing is necessarily, like the becoming or perishing of a specific
character or form, the result of the alteration of certain other things,
e.g. hot and cold or dry and wet elements or the elements, whatever
they may be, on which the states primarily depend. For each several
bodily defect or excellence involves a relation with those things
from which the possessor of the defect or excellence is naturally
subject to alteration: thus excellence disposes its possessor to be
unaffected by these influences or to be affected by those of them
that ought to be admitted, while defect disposes its possessor to
be affected by them or to be unaffected by those of them that ought
to be admitted. 

And the case is similar in regard to the states of the soul, all of
which (like those of body) exist in virtue of particular relations,
the excellences being perfections of nature and the defects departures
from it: moreover, excellence puts its possessor in good condition,
while defect puts its possessor in a bad condition, to meet his proper
affections. Consequently these cannot any more than the bodily states
be alterations, nor can the processes of losing and acquiring them
be so, though their becoming is necessarily the result of an alteration
of the sensitive part of the soul, and this is altered by sensible
objects: for all moral excellence is concerned with bodily pleasures
and pains, which again depend either upon acting or upon remembering
or upon anticipating. Now those that depend upon action are determined
by sense-perception, i.e. they are stimulated by something sensible:
and those that depend upon memory or anticipation are likewise to
be traced to sense-perception, for in these cases pleasure is felt
either in remembering what one has experienced or in anticipating
what one is going to experience. Thus all pleasure of this kind must
be produced by sensible things: and since the presence in any one
of moral defect or excellence involves the presence in him of pleasure
or pain (with which moral excellence and defect are always concerned),
and these pleasures and pains are alterations of the sensitive part,
it is evident that the loss and acquisition of these states no less
than the loss and acquisition of the states of the body must be the
result of the alteration of something else. Consequently, though their
becoming is accompanied by an alteration, they are not themselves
alterations. 

Again, the states of the intellectual part of the soul are not alterations,
nor is there any becoming of them. In the first place it is much more
true of the possession of knowledge that it depends upon a particular
relation. And further, it is evident that there is no becoming of
these states. For that which is potentially possessed of knowledge
becomes actually possessed of it not by being set in motion at all
itself but by reason of the presence of something else: i.e. it is
when it meets with the particular object that it knows in a manner
the particular through its knowledge of the universal. (Again, there
is no becoming of the actual use and activity of these states, unless
it is thought that there is a becoming of vision and touching and
that the activity in question is similar to these.) And the original
acquisition of knowledge is not a becoming or an alteration: for the
terms 'knowing' and 'understanding' imply that the intellect has reached
a state of rest and come to a standstill, and there is no becoming
that leads to a state of rest, since, as we have said above, change
at all can have a becoming. Moreover, just as to say, when any one
has passed from a state of intoxication or sleep or disease to the
contrary state, that he has become possessed of knowledge again is
incorrect in spite of the fact that he was previously incapable of
using his knowledge, so, too, when any one originally acquires the
state, it is incorrect to say that he becomes possessed of knowledge:
for the possession of understanding and knowledge is produced by the
soul's settling down out of the restlessness natural to it. Hence,
too, in learning and in forming judgements on matters relating to
their sense-perceptions children are inferior to adults owing to the
great amount of restlessness and motion in their souls. Nature itself
causes the soul to settle down and come to a state of rest for the
performance of some of its functions, while for the performance of
others other things do so: but in either case the result is brought
about through the alteration of something in the body, as we see in
the case of the use and activity of the intellect arising from a man's
becoming sober or being awakened. It is evident, then, from the preceding
argument that alteration and being altered occur in sensible things
and in the sensitive part of the soul, and, except accidentally, in
nothing else. 

Part 4

A difficulty may be raised as to whether every motion is commensurable
with every other or not. Now if they are all commensurable and if
two things to have the same velocity must accomplish an equal motion
in an equal time, then we may have a circumference equal to a straight
line, or, of course, the one may be greater or less than the other.
Further, if one thing alters and another accomplishes a locomotion
in an equal time, we may have an alteration and a locomotion equal
to one another: thus an affection will be equal to a length, which
is impossible. But is it not only when an equal motion is accomplished
by two things in an equal time that the velocities of the two are
equal? Now an affection cannot be equal to a length. Therefore there
cannot be an alteration equal to or less than a locomotion: and consequently
it is not the case that every motion is commensurable with every other.

But how will our conclusion work out in the case of the circle and
the straight line? It would be absurd to suppose that the motion of
one in a circle and of another in a straight line cannot be similar,
but that the one must inevitably move more quickly or more slowly
than the other, just as if the course of one were downhill and of
the other uphill. Moreover it does not as a matter of fact make any
difference to the argument to say that the one motion must inevitably
be quicker or slower than the other: for then the circumference can
be greater or less than the straight line; and if so it is possible
for the two to be equal. For if in the time A the quicker (B) passes
over the distance B' and the slower (G) passes over the distance G',
B' will be greater than G': for this is what we took 'quicker' to
mean: and so quicker motion also implies that one thing traverses
an equal distance in less time than another: consequently there will
be a part of A in which B will pass over a part of the circle equal
to G', while G will occupy the whole of A in passing over G'. None
the less, if the two motions are commensurable, we are confronted
with the consequence stated above, viz. that there may be a straight
line equal to a circle. But these are not commensurable: and so the
corresponding motions are not commensurable either. 

But may we say that things are always commensurable if the same terms
are applied to them without equivocation? e.g. a pen, a wine, and
the highest note in a scale are not commensurable: we cannot say whether
any one of them is sharper than any other: and why is this? they are
incommensurable because it is only equivocally that the same term
'sharp' is applied to them: whereas the highest note in a scale is
commensurable with the leading-note, because the term 'sharp' has
the same meaning as applied to both. Can it be, then, that the term
'quick' has not the same meaning as applied to straight motion and
to circular motion respectively? If so, far less will it have the
same meaning as applied to alteration and to locomotion.

Or shall we in the first place deny that things are always commensurable
if the same terms are applied to them without equivocation? For the
term 'much' has the same meaning whether applied to water or to air,
yet water and air are not commensurable in respect of it: or, if this
illustration is not considered satisfactory, 'double' at any rate
would seem to have the same meaning as applied to each (denoting in
each case the proportion of two to one), yet water and air are not
commensurable in respect of it. But here again may we not take up
the same position and say that the term 'much' is equivocal? In fact
there are some terms of which even the definitions are equivocal;
e.g. if 'much' were defined as 'so much and more','so much' would
mean something different in different cases: 'equal' is similarly
equivocal; and 'one' again is perhaps inevitably an equivocal term;
and if 'one' is equivocal, so is 'two'. Otherwise why is it that some
things are commensurable while others are not, if the nature of the
attribute in the two cases is really one and the same? 

Can it be that the incommensurability of two things in respect of
any attribute is due to a difference in that which is primarily capable
of carrying the attribute? Thus horse and dog are so commensurable
that we may say which is the whiter, since that which primarily contains
the whiteness is the same in both, viz. the surface: and similarly
they are commensurable in respect of size. But water and speech are
not commensurable in respect of clearness, since that which primarily
contains the attribute is different in the two cases. It would seem,
however that we must reject this solution, since clearly we could
thus make all equivocal attributes univocal and say merely that that
contains each of them is different in different cases: thus 'equality',
'sweetness', and 'whiteness' will severally always be the same, though
that which contains them is different in different cases. Moreover,
it is not any casual thing that is capable of carrying any attribute:
each single attribute can be carried primarily only by one single
thing. 

Must we then say that, if two things are to be commensurable in respect
of any attribute, not only must the attribute in question be applicable
to both without equivocation, but there must also be no specific differences
either in the attribute itself or in that which contains the attribute-that
these, I mean, must not be divisible in the way in which colour is
divided into kinds? Thus in this respect one thing will not be commensurable
with another, i.e. we cannot say that one is more coloured than the
other where only colour in general and not any particular colour is
meant; but they are commensurable in respect of whiteness.

Similarly in the case of motion: two things are of the same velocity
if they occupy an equal time in accomplishing a certain equal amount
of motion. Suppose, then, that in a certain time an alteration is
undergone by one half of a body's length and a locomotion is accomplished
the other half: can be say that in this case the alteration is equal
to the locomotion and of the same velocity? That would be absurd,
and the reason is that there are different species of motion. And
if in consequence of this we must say that two things are of equal
velocity if they accomplish locomotion over an equal distance in an
equal time, we have to admit the equality of a straight line and a
circumference. What, then, is the reason of this? Is it that locomotion
is a genus or that line is a genus? (We may leave the time out of
account, since that is one and the same.) If the lines are specifically
different, the locomotions also differ specifically from one another:
for locomotion is specifically differentiated according to the specific
differentiation of that over which it takes place. (It is also similarly
differentiated, it would seem, accordingly as the instrument of the
locomotion is different: thus if feet are the instrument, it is walking,
if wings it is flying; but perhaps we should rather say that this
is not so, and that in this case the differences in the locomotion
are merely differences of posture in that which is in motion.) We
may say, therefore, that things are of equal velocity in an equal
time they traverse the same magnitude: and when I call it 'the same'
I mean that it contains no specific difference and therefore no difference
in the motion that takes place over it. So we have now to consider
how motion is differentiated: and this discussion serves to show that
the genus is not a unity but contains a plurality latent in it and
distinct from it, and that in the case of equivocal terms sometimes
the different senses in which they are used are far removed from one
another, while sometimes there is a certain likeness between them,
and sometimes again they are nearly related either generically or
analogically, with the result that they seem not to be equivocal though
they really are. 

When, then, is there a difference of species? Is an attribute specifically
different if the subject is different while the attribute is the same,
or must the attribute itself be different as well? And how are we
to define the limits of a species? What will enable us to decide that
particular instances of whiteness or sweetness are the same or different?
Is it enough that it appears different in one subject from what appears
in another? Or must there be no sameness at all? And further, where
alteration is in question, how is one alteration to be of equal velocity
with another? One person may be cured quickly and another slowly,
and cures may also be simultaneous: so that, recovery of health being
an alteration, we have here alterations of equal velocity, since each
alteration occupies an equal time. But what alteration? We cannot
here speak of an 'equal' alteration: what corresponds in the category
of quality to equality in the category of quantity is 'likeness'.
However, let us say that there is equal velocity where the same change
is accomplished in an equal time. Are we, then, to find the commensurability
in the subject of the affection or in the affection itself? In the
case that we have just been considering it is the fact that health
is one and the same that enables us to arrive at the conclusion that
the one alteration is neither more nor less than the other, but that
both are alike. If on the other hand the affection is different in
the two cases, e.g. when the alterations take the form of becoming
white and becoming healthy respectively, here there is no sameness
or equality or likeness inasmuch as the difference in the affections
at once makes the alterations specifically different, and there is
no unity of alteration any more than there would be unity of locomotion
under like conditions. So we must find out how many species there
are of alteration and of locomotion respectively. Now if the things
that are in motion-that is to say, the things to which the motions
belong essentially and not accidentally-differ specifically, then
their respective motions will also differ specifically: if on the
other hand they differ generically or numerically, the motions also
will differ generically or numerically as the case may be. But there
still remains the question whether, supposing that two alterations
are of equal velocity, we ought to look for this equality in the sameness
(or likeness) of the affections, or in the things altered, to see
e.g. whether a certain quantity of each has become white. Or ought
we not rather to look for it in both? That is to say, the alterations
are the same or different according as the affections are the same
or different, while they are equal or unequal according as the things
altered are equal or unequal. 

And now we must consider the same question in the case of becoming
and perishing: how is one becoming of equal velocity with another?
They are of equal velocity if in an equal time there are produced
two things that are the same and specifically inseparable, e.g. two
men (not merely generically inseparable as e.g. two animals). Similarly
one is quicker than the other if in an equal time the product is different
in the two cases. I state it thus because we have no pair of terms
that will convey this 'difference' in the way in which unlikeness
is conveyed. If we adopt the theory that it is number that constitutes
being, we may indeed speak of a 'greater number' and a 'lesser number'
within the same species, but there is no common term that will include
both relations, nor are there terms to express each of them separately
in the same way as we indicate a higher degree or preponderance of
an affection by 'more', of a quantity by 'greater.' 

Part 5

Now since wherever there is a movent, its motion always acts upon
something, is always in something, and always extends to something
(by 'is always in something' I mean that it occupies a time: and by
'extends to something' I mean that it involves the traversing of a
certain amount of distance: for at any moment when a thing is causing
motion, it also has caused motion, so that there must always be a
certain amount of distance that has been traversed and a certain amount
of time that has been occupied). then, A the movement have moved B
a distance G in a time D, then in the same time the same force A will
move 1/2B twice the distance G, and in 1/2D it will move 1/2B the
whole distance for G: thus the rules of proportion will be observed.
Again if a given force move a given weight a certain distance in a
certain time and half the distance in half the time, half the motive
power will move half the weight the same distance in the same time.
Let E represent half the motive power A and Z half the weight B: then
the ratio between the motive power and the weight in the one case
is similar and proportionate to the ratio in the other, so that each
force will cause the same distance to be traversed in the same time.
But if E move Z a distance G in a time D, it does not necessarily
follow that E can move twice Z half the distance G in the same time.
If, then, A move B a distance G in a time D, it does not follow that
E, being half of A, will in the time D or in any fraction of it cause
B to traverse a part of G the ratio between which and the whole of
G is proportionate to that between A and E (whatever fraction of AE
may be): in fact it might well be that it will cause no motion at
all; for it does not follow that, if a given motive power causes a
certain amount of motion, half that power will cause motion either
of any particular amount or in any length of time: otherwise one man
might move a ship, since both the motive power of the ship-haulers
and the distance that they all cause the ship to traverse are divisible
into as many parts as there are men. Hence Zeno's reasoning is false
when he argues that there is no part of the millet that does not make
a sound: for there is no reason why any such part should not in any
length of time fail to move the air that the whole bushel moves in
falling. In fact it does not of itself move even such a quantity of
the air as it would move if this part were by itself: for no part
even exists otherwise than potentially. 

If on the other hand we have two forces each of which separately moves
one of two weights a given distance in a given time, then the forces
in combination will move the combined weights an equal distance in
an equal time: for in this case the rules of proportion apply.

Then does this hold good of alteration and of increase also? Surely
it does, for in any given case we have a definite thing that cause
increase and a definite thing that suffers increase, and the one causes
and the other suffers a certain amount of increase in a certain amount
of time. Similarly we have a definite thing that causes alteration
and a definite thing that undergoes alteration, and a certain amount,
or rather degree, of alteration is completed in a certain amount of
time: thus in twice as much time twice as much alteration will be
completed and conversely twice as much alteration will occupy twice
as much time: and the alteration of half of its object will occupy
half as much time and in half as much time half of the object will
be altered: or again, in the same amount of time it will be altered
twice as much. 

On the other hand if that which causes alteration or increase causes
a certain amount of increase or alteration respectively in a certain
amount of time, it does not necessarily follow that half the force
will occupy twice the time in altering or increasing the object, or
that in twice the time the alteration or increase will be completed
by it: it may happen that there will be no alteration or increase
at all, the case being the same as with the weight. 

----------------------------------------------------------------------

BOOK VIII

Part 1 

It remains to consider the following question. Was there ever a becoming
of motion before which it had no being, and is it perishing again
so as to leave nothing in motion? Or are we to say that it never had
any becoming and is not perishing, but always was and always will
be? Is it in fact an immortal never-failing property of things that
are, a sort of life as it were to all naturally constituted things?

Now the existence of motion is asserted by all who have anything to
say about nature, because they all concern themselves with the construction
of the world and study the question of becoming and perishing, which
processes could not come about without the existence of motion. But
those who say that there is an infinite number of worlds, some of
which are in process of becoming while others are in process of perishing,
assert that there is always motion (for these processes of becoming
and perishing of the worlds necessarily involve motion), whereas those
who hold that there is only one world, whether everlasting or not,
make corresponding assumptions in regard to motion. If then it is
possible that at any time nothing should be in motion, this must come
about in one of two ways: either in the manner described by Anaxagoras,
who says that all things were together and at rest for an infinite
period of time, and that then Mind introduced motion and separated
them; or in the manner described by Empedocles, according to whom
the universe is alternately in motion and at rest-in motion, when
Love is making the one out of many, or Strife is making many out of
one, and at rest in the intermediate periods of time-his account being
as follows: 

'Since One hath learned to spring from Manifold, And One disjoined
makes manifold arise, Thus they Become, nor stable is their life:
But since their motion must alternate be, Thus have they ever Rest
upon their round': for we must suppose that he means by this that
they alternate from the one motion to the other. We must consider,
then, how this matter stands, for the discovery of the truth about
it is of importance, not only for the study of nature, but also for
the investigation of the First Principle. 

Let us take our start from what we have already laid down in our course
on Physics. Motion, we say, is the fulfilment of the movable in so
far as it is movable. Each kind of motion, therefore, necessarily
involves the presence of the things that are capable of that motion.
In fact, even apart from the definition of motion, every one would
admit that in each kind of motion it is that which is capable of that
motion that is in motion: thus it is that which is capable of alteration
that is altered, and that which is capable of local change that is
in locomotion: and so there must be something capable of being burned
before there can be a process of being burned, and something capable
of burning before there can be a process of burning. Moreover, these
things also must either have a beginning before which they had no
being, or they must be eternal. Now if there was a becoming of every
movable thing, it follows that before the motion in question another
change or motion must have taken place in which that which was capable
of being moved or of causing motion had its becoming. To suppose,
on the other hand, that these things were in being throughout all
previous time without there being any motion appears unreasonable
on a moment's thought, and still more unreasonable, we shall find,
on further consideration. For if we are to say that, while there are
on the one hand things that are movable, and on the other hand things
that are motive, there is a time when there is a first movent and
a first moved, and another time when there is no such thing but only
something that is at rest, then this thing that is at rest must previously
have been in process of change: for there must have been some cause
of its rest, rest being the privation of motion. Therefore, before
this first change there will be a previous change. For some things
cause motion in only one way, while others can produce either of two
contrary motions: thus fire causes heating but not cooling, whereas
it would seem that knowledge may be directed to two contrary ends
while remaining one and the same. Even in the former class, however,
there seems to be something similar, for a cold thing in a sense causes
heating by turning away and retiring, just as one possessed of knowledge
voluntarily makes an error when he uses his knowledge in the reverse
way. But at any rate all things that are capable respectively of affecting
and being affected, or of causing motion and being moved, are capable
of it not under all conditions, but only when they are in a particular
condition and approach one another: so it is on the approach of one
thing to another that the one causes motion and the other is moved,
and when they are present under such conditions as rendered the one
motive and the other movable. So if the motion was not always in process,
it is clear that they must have been in a condition not such as to
render them capable respectively of being moved and of causing motion,
and one or other of them must have been in process of change: for
in what is relative this is a necessary consequence: e.g. if one thing
is double another when before it was not so, one or other of them,
if not both, must have been in process of change. It follows then,
that there will be a process of change previous to the first.

(Further, how can there be any 'before' and 'after' without the existence
of time? Or how can there be any time without the existence of motion?
If, then, time is the number of motion or itself a kind of motion,
it follows that, if there is always time, motion must also be eternal.
But so far as time is concerned we see that all with one exception
are in agreement in saying that it is uncreated: in fact, it is just
this that enables Democritus to show that all things cannot have had
a becoming: for time, he says, is uncreated. Plato alone asserts the
creation of time, saying that it had a becoming together with the
universe, the universe according to him having had a becoming. Now
since time cannot exist and is unthinkable apart from the moment,
and the moment a kind of middle-point, uniting as it does in itself
both a beginning and an end, a beginning of future time and an end
of past time, it follows that there must always be time: for the extremity
of the last period of time that we take must be found in some moment,
since time contains no point of contact for us except the moment.
Therefore, since the moment is both a beginning and an end, there
must always be time on both sides of it. But if this is true of time,
it is evident that it must also be true of motion, time being a kind
of affection of motion.) 

The same reasoning will also serve to show the imperishability of
motion: just as a becoming of motion would involve, as we saw, the
existence of a process of change previous to the first, in the same
way a perishing of motion would involve the existence of a process
of change subsequent to the last: for when a thing ceases to be moved,
it does not therefore at the same time cease to be movable-e.g. the
cessation of the process of being burned does not involve the cessation
of the capacity of being burned, since a thing may be capable of being
burned without being in process of being burned-nor, when a thing
ceases to be movent, does it therefore at the same time cease to a
be motive. Again, the destructive agent will have to be destroyed,
after what it destroys has been destroyed, and then that which has
the capacity of destroying it will have to be destroyed afterwards,
(so that there will be a process of change subsequent to the last,)
for being destroyed also is a kind of change. If, then, view which
we are criticizing involves these impossible consequences, it is clear
that motion is eternal and cannot have existed at one time and not
at another: in fact such a view can hardly be described as anythling
else than fantastic. 

And much the same may be said of the view that such is the ordinance
of nature and that this must be regarded as a principle, as would
seem to be the view of Empedocles when he says that the constitution
of the world is of necessity such that Love and Strife alternately
predominate and cause motion, while in the intermediate period of
time there is a state of rest. Probably also those who like like Anaxagoras,
assert a single principle (of motion) would hold this view. But that
which is produced or directed by nature can never be anything disorderly:
for nature is everywhere the cause of order. Moreover, there is no
ratio in the relation of the infinite to the infinite, whereas order
always means ratio. But if we say that there is first a state of rest
for an infinite time, and then motion is started at some moment, and
that the fact that it is this rather than a previous moment is of
no importance, and involves no order, then we can no longer say that
it is nature's work: for if anything is of a certain character naturally,
it either is so invariably and is not sometimes of this and sometimes
of another character (e.g. fire, which travels upwards naturally,
does not sometimes do so and sometimes not) or there is a ratio in
the variation. It would be better, therefore, to say with Empedocles
and any one else who may have maintained such a theory as his that
the universe is alternately at rest and in motion: for in a system
of this kind we have at once a certain order. But even here the holder
of the theory ought not only to assert the fact: he ought to explain
the cause of it: i.e. he should not make any mere assumption or lay
down any gratuitous axiom, but should employ either inductive or demonstrative
reasoning. The Love and Strife postulated by Empedocles are not in
themselves causes of the fact in question, nor is it of the essence
of either that it should be so, the essential function of the former
being to unite, of the latter to separate. If he is to go on to explain
this alternate predominance, he should adduce cases where such a state
of things exists, as he points to the fact that among mankind we have
something that unites men, namely Love, while on the other hand enemies
avoid one another: thus from the observed fact that this occurs in
certain cases comes the assumption that it occurs also in the universe.
Then, again, some argument is needed to explain why the predominance
of each of the two forces lasts for an equal period of time. But it
is a wrong assumption to suppose universally that we have an adequate
first principle in virtue of the fact that something always is so
or always happens so. Thus Democritus reduces the causes that explain
nature to the fact that things happened in the past in the same way
as they happen now: but he does not think fit to seek for a first
principle to explain this 'always': so, while his theory is right
in so far as it is applied to certain individual cases, he is wrong
in making it of universal application. Thus, a triangle always has
its angles equal to two right angles, but there is nevertheless an
ulterior cause of the eternity of this truth, whereas first principles
are eternal and have no ulterior cause. Let this conclude what we
have to say in support of our contention that there never was a time
when there was not motion, and never will be a time when there will
not be motion. 

Part 2

The arguments that may be advanced against this position are not difficult
to dispose of. The chief considerations that might be thought to indicate
that motion may exist though at one time it had not existed at all
are the following: 

First, it may be said that no process of change is eternal: for the
nature of all change is such that it proceeds from something to something,
so that every process of change must be bounded by the contraries
that mark its course, and no motion can go on to infinity.

Secondly, we see that a thing that neither is in motion nor contains
any motion within itself can be set in motion; e.g. inanimate things
that are (whether the whole or some part is in question) not in motion
but at rest, are at some moment set in motion: whereas, if motion
cannot have a becoming before which it had no being, these things
ought to be either always or never in motion. 

Thirdly, the fact is evident above all in the case of animate beings:
for it sometimes happens that there is no motion in us and we are
quite still, and that nevertheless we are then at some moment set
in motion, that is to say it sometimes happens that we produce a beginning
of motion in ourselves spontaneously without anything having set us
in motion from without. We see nothing like this in the case of inanimate
things, which are always set in motion by something else from without:
the animal, on the other hand, we say, moves itself: therefore, if
an animal is ever in a state of absolute rest, we have a motionless
thing in which motion can be produced from the thing itself, and not
from without. Now if this can occur in an animal, why should not the
same be true also of the universe as a whole? If it can occur in a
small world it could also occur in a great one: and if it can occur
in the world, it could also occur in the infinite; that is, if the
infinite could as a whole possibly be in motion or at rest.

Of these objections, then, the first-mentioned motion to opposites
is not always the same and numerically one a correct statement; in
fact, this may be said to be a necessary conclusion, provided that
it is possible for the motion of that which is one and the same to
be not always one and the same. (I mean that e.g. we may question
whether the note given by a single string is one and the same, or
is different each time the string is struck, although the string is
in the same condition and is moved in the same way.) But still, however
this may be, there is nothing to prevent there being a motion that
is the same in virtue of being continuous and eternal: we shall have
something to say later that will make this point clearer.

As regards the second objection, no absurdity is involved in the fact
that something not in motion may be set in motion, that which caused
the motion from without being at one time present, and at another
absent. Nevertheless, how this can be so remains matter for inquiry;
how it comes about, I mean, that the same motive force at one time
causes a thing to be in motion, and at another does not do so: for
the difficulty raised by our objector really amounts to this-why is
it that some things are not always at rest, and the rest always in
motion? 

The third objection may be thought to present more difficulty than
the others, namely, that which alleges that motion arises in things
in which it did not exist before, and adduces in proof the case of
animate things: thus an animal is first at rest and afterwards walks,
not having been set in motion apparently by anything from without.
This, however, is false: for we observe that there is always some
part of the animal's organism in motion, and the cause of the motion
of this part is not the animal itself, but, it may be, its environment.
Moreover, we say that the animal itself originates not all of its
motions but its locomotion. So it may well be the case-or rather we
may perhaps say that it must necessarily be the case-that many motions
are produced in the body by its environment, and some of these set
in motion the intellect or the appetite, and this again then sets
the whole animal in motion: this is what happens when animals are
asleep: though there is then no perceptive motion in them, there is
some motion that causes them to wake up again. But we will leave this
point also to be elucidated at a later stage in our discussion.

Part 3

Our enquiry will resolve itself at the outset into a consideration
of the above-mentioned problem-what can be the reason why some things
in the world at one time are in motion and at another are at rest
again? Now one of three things must be true: either all things are
always at rest, or all things are always in motion, or some things
are in motion and others at rest: and in this last case again either
the things that are in motion are always in motion and the things
that are at rest are always at rest, or they are all constituted so
as to be capable alike of motion and of rest; or there is yet a third
possibility remaining-it may be that some things in the world are
always motionless, others always in motion, while others again admit
of both conditions. This last is the account of the matter that we
must give: for herein lies the solution of all the difficulties raised
and the conclusion of the investigation upon which we are engaged.

To maintain that all things are at rest, and to disregard sense-perception
in an attempt to show the theory to be reasonable, would be an instance
of intellectual weakness: it would call in question a whole system,
not a particular detail: moreover, it would be an attack not only
on the physicist but on almost all sciences and all received opinions,
since motion plays a part in all of them. Further, just as in arguments
about mathematics objections that involve first principles do not
affect the mathematician-and the other sciences are in similar case-so,
too, objections involving the point that we have just raised do not
affect the physicist: for it is a fundamental assumption with him
that motion is ultimately referable to nature herself. 

The assertion that all things are in motion we may fairly regard as
equally false, though it is less subversive of physical science: for
though in our course on physics it was laid down that rest no less
than motion is ultimately referable to nature herself, nevertheless
motion is the characteristic fact of nature: moreover, the view is
actually held by some that not merely some things but all things in
the world are in motion and always in motion, though we cannot apprehend
the fact by sense-perception. Although the supporters of this theory
do not state clearly what kind of motion they mean, or whether they
mean all kinds, it is no hard matter to reply to them: thus we may
point out that there cannot be a continuous process either of increase
or of decrease: that which comes between the two has to be included.
The theory resembles that about the stone being worn away by the drop
of water or split by plants growing out of it: if so much has been
extruded or removed by the drop, it does not follow that half the
amount has previously been extruded or removed in half the time: the
case of the hauled ship is exactly comparable: here we have so many
drops setting so much in motion, but a part of them will not set as
much in motion in any period of time. The amount removed is, it is
true, divisible into a number of parts, but no one of these was set
in motion separately: they were all set in motion together. It is
evident, then, that from the fact that the decrease is divisible into
an infinite number of parts it does not follow that some part must
always be passing away: it all passes away at a particular moment.
Similarly, too, in the case of any alteration whatever if that which
suffers alteration is infinitely divisible it does not follow from
this that the same is true of the alteration itself, which often occurs
all at once, as in freezing. Again, when any one has fallen ill, there
must follow a period of time in which his restoration to health is
in the future: the process of change cannot take place in an instant:
yet the change cannot be a change to anything else but health. The
assertion. therefore, that alteration is continuous is an extravagant
calling into question of the obvious: for alteration is a change from
one contrary to another. Moreover, we notice that a stone becomes
neither harder nor softer. Again, in the matter of locomotion, it
would be a strange thing if a stone could be falling or resting on
the ground without our being able to perceive the fact. Further, it
is a law of nature that earth and all other bodies should remain in
their proper places and be moved from them only by violence: from
the fact then that some of them are in their proper places it follows
that in respect of place also all things cannot be in motion. These
and other similar arguments, then, should convince us that it is impossible
either that all things are always in motion or that all things are
always at rest. 

Nor again can it be that some things are always at rest, others always
in motion, and nothing sometimes at rest and sometimes in motion.
This theory must be pronounced impossible on the same grounds as those
previously mentioned: viz. that we see the above-mentioned changes
occurring in the case of the same things. We may further point out
that the defender of this position is fighting against the obvious,
for on this theory there can be no such thing as increase: nor can
there be any such thing as compulsory motion, if it is impossible
that a thing can be at rest before being set in motion unnaturally.
This theory, then, does away with becoming and perishing. Moreover,
motion, it would seem, is generally thought to be a sort of becoming
and perishing, for that to which a thing changes comes to be, or occupancy
of it comes to be, and that from which a thing changes ceases to be,
or there ceases to be occupancy of it. It is clear, therefore, that
there are cases of occasional motion and occasional rest.

We have now to take the assertion that all things are sometimes at
rest and sometimes in motion and to confront it with the arguments
previously advanced. We must take our start as before from the possibilities
that we distinguished just above. Either all things are at rest, or
all things are in motion, or some things are at rest and others in
motion. And if some things are at rest and others in motion, then
it must be that either all things are sometimes at rest and sometimes
in motion, or some things are always at rest and the remainder always
in motion, or some of the things are always at rest and others always
in motion while others again are sometimes at rest and sometimes in
motion. Now we have said before that it is impossible that all things
should be at rest: nevertheless we may now repeat that assertion.
We may point out that, even if it is really the case, as certain persons
assert, that the existent is infinite and motionless, it certainly
does not appear to be so if we follow sense-perception: many things
that exist appear to be in motion. Now if there is such a thing as
false opinion or opinion at all, there is also motion; and similarly
if there is such a thing as imagination, or if it is the case that
anything seems to be different at different times: for imagination
and opinion are thought to be motions of a kind. But to investigate
this question at all-to seek a reasoned justification of a belief
with regard to which we are too well off to require reasoned justification-implies
bad judgement of what is better and what is worse, what commends itself
to belief and what does not, what is ultimate and what is not. It
is likewise impossible that all things should be in motion or that
some things should be always in motion and the remainder always at
rest. We have sufficient ground for rejecting all these theories in
the single fact that we see some things that are sometimes in motion
and sometimes at rest. It is evident, therefore, that it is no less
impossible that some things should be always in motion and the remainder
always at rest than that all things should be at rest or that all
things should be in motion continuously. It remains, then, to consider
whether all things are so constituted as to be capable both of being
in motion and of being at rest, or whether, while some things are
so constituted, some are always at rest and some are always in motion:
for it is this last view that we have to show to be true.

Part 4

Now of things that cause motion or suffer motion, to some the motion
is accidental, to others essential: thus it is accidental to what
merely belongs to or contains as a part a thing that causes motion
or suffers motion, essential to a thing that causes motion or suffers
motion not merely by belonging to such a thing or containing it as
a part. 

Of things to which the motion is essential some derive their motion
from themselves, others from something else: and in some cases their
motion is natural, in others violent and unnatural. Thus in things
that derive their motion from themselves, e.g. all animals, the motion
is natural (for when an animal is in motion its motion is derived
from itself): and whenever the source of the motion of a thing is
in the thing itself we say that the motion of that thing is natural.
Therefore the animal as a whole moves itself naturally: but the body
of the animal may be in motion unnaturally as well as naturally: it
depends upon the kind of motion that it may chance to be suffering
and the kind of element of which it is composed. And the motion of
things that derive their motion from something else is in some cases
natural, in other unnatural: e.g. upward motion of earthy things and
downward motion of fire are unnatural. Moreover the parts of animals
are often in motion in an unnatural way, their positions and the character
of the motion being abnormal. The fact that a thing that is in motion
derives its motion from something is most evident in things that are
in motion unnaturally, because in such cases it is clear that the
motion is derived from something other than the thing itself. Next
to things that are in motion unnaturally those whose motion while
natural is derived from themselves-e.g. animals-make this fact clear:
for here the uncertainty is not as to whether the motion is derived
from something but as to how we ought to distinguish in the thing
between the movent and the moved. It would seem that in animals, just
as in ships and things not naturally organized, that which causes
motion is separate from that which suffers motion, and that it is
only in this sense that the animal as a whole causes its own motion.

The greatest difficulty, however, is presented by the remaining case
of those that we last distinguished. Where things derive their motion
from something else we distinguished the cases in which the motion
is unnatural: we are left with those that are to be contrasted with
the others by reason of the fact that the motion is natural. It is
in these cases that difficulty would be experienced in deciding whence
the motion is derived, e.g. in the case of light and heavy things.
When these things are in motion to positions the reverse of those
they would properly occupy, their motion is violent: when they are
in motion to their proper positions-the light thing up and the heavy
thing down-their motion is natural; but in this latter case it is
no longer evident, as it is when the motion is unnatural, whence their
motion is derived. It is impossible to say that their motion is derived
from themselves: this is a characteristic of life and peculiar to
living things. Further, if it were, it would have been in their power
to stop themselves (I mean that if e.g. a thing can cause itself to
walk it can also cause itself not to walk), and so, since on this
supposition fire itself possesses the power of upward locomotion,
it is clear that it should also possess the power of downward locomotion.
Moreover if things move themselves, it would be unreasonable to suppose
that in only one kind of motion is their motion derived from themselves.
Again, how can anything of continuous and naturally connected substance
move itself? In so far as a thing is one and continuous not merely
in virtue of contact, it is impassive: it is only in so far as a thing
is divided that one part of it is by nature active and another passive.
Therefore none of the things that we are now considering move themselves
(for they are of naturally connected substance), nor does anything
else that is continuous: in each case the movent must be separate
from the moved, as we see to be the case with inanimate things when
an animate thing moves them. It is the fact that these things also
always derive their motion from something: what it is would become
evident if we were to distinguish the different kinds of cause.

The above-mentioned distinctions can also be made in the case of things
that cause motion: some of them are capable of causing motion unnaturally
(e.g. the lever is not naturally capable of moving the weight), others
naturally (e.g. what is actually hot is naturally capable of moving
what is potentially hot): and similarly in the case of all other things
of this kind. 

In the same way, too, what is potentially of a certain quality or
of a certain quantity in a certain place is naturally movable when
it contains the corresponding principle in itself and not accidentally
(for the same thing may be both of a certain quality and of a certain
quantity, but the one is an accidental, not an essential property
of the other). So when fire or earth is moved by something the motion
is violent when it is unnatural, and natural when it brings to actuality
the proper activities that they potentially possess. But the fact
that the term 'potentially' is used in more than one sense is the
reason why it is not evident whence such motions as the upward motion
of fire and the downward motion of earth are derived. One who is learning
a science potentially knows it in a different sense from one who while
already possessing the knowledge is not actually exercising it. Wherever
we have something capable of acting and something capable of being
correspondingly acted on, in the event of any such pair being in contact
what is potential becomes at times actual: e.g. the learner becomes
from one potential something another potential something: for one
who possesses knowledge of a science but is not actually exercising
it knows the science potentially in a sense, though not in the same
sense as he knew it potentially before he learnt it. And when he is
in this condition, if something does not prevent him, he actively
exercises his knowledge: otherwise he would be in the contradictory
state of not knowing. In regard to natural bodies also the case is
similar. Thus what is cold is potentially hot: then a change takes
place and it is fire, and it burns, unless something prevents and
hinders it. So, too, with heavy and light: light is generated from
heavy, e.g. air from water (for water is the first thing that is potentially
light), and air is actually light, and will at once realize its proper
activity as such unless something prevents it. The activity of lightness
consists in the light thing being in a certain situation, namely high
up: when it is in the contrary situation, it is being prevented from
rising. The case is similar also in regard to quantity and quality.
But, be it noted, this is the question we are trying to answer-how
can we account for the motion of light things and heavy things to
their proper situations? The reason for it is that they have a natural
tendency respectively towards a certain position: and this constitutes
the essence of lightness and heaviness, the former being determined
by an upward, the latter by a downward, tendency. As we have said,
a thing may be potentially light or heavy in more senses than one.
Thus not only when a thing is water is it in a sense potentially light,
but when it has become air it may be still potentially light: for
it may be that through some hindrance it does not occupy an upper
position, whereas, if what hinders it is removed, it realizes its
activity and continues to rise higher. The process whereby what is
of a certain quality changes to a condition of active existence is
similar: thus the exercise of knowledge follows at once upon the possession
of it unless something prevents it. So, too, what is of a certain
quantity extends itself over a certain space unless something prevents
it. The thing in a sense is and in a sense is not moved by one who
moves what is obstructing and preventing its motion (e.g. one who
pulls away a pillar from under a roof or one who removes a stone from
a wineskin in the water is the accidental cause of motion): and in
the same way the real cause of the motion of a ball rebounding from
a wall is not the wall but the thrower. So it is clear that in all
these cases the thing does not move itself, but it contains within
itself the source of motion-not of moving something or of causing
motion, but of suffering it. 

If then the motion of all things that are in motion is either natural
or unnatural and violent, and all things whose motion is violent and
unnatural are moved by something, and something other than themselves,
and again all things whose motion is natural are moved by something-both
those that are moved by themselves and those that are not moved by
themselves (e.g. light things and heavy things, which are moved either
by that which brought the thing into existence as such and made it
light and heavy, or by that which released what was hindering and
preventing it); then all things that are in motion must be moved by
something. 

Part 5

Now this may come about in either of two ways. Either the movent is
not itself responsible for the motion, which is to be referred to
something else which moves the movent, or the movent is itself responsible
for the motion. Further, in the latter case, either the movent immediately
precedes the last thing in the series, or there may be one or more
intermediate links: e.g. the stick moves the stone and is moved by
the hand, which again is moved by the man: in the man, however, we
have reached a movent that is not so in virtue of being moved by something
else. Now we say that the thing is moved both by the last and by the
first movent in the series, but more strictly by the first, since
the first movent moves the last, whereas the last does not move the
first, and the first will move the thing without the last, but the
last will not move it without the first: e.g. the stick will not move
anything unless it is itself moved by the man. If then everything
that is in motion must be moved by something, and the movent must
either itself be moved by something else or not, and in the former
case there must be some first movent that is not itself moved by anything
else, while in the case of the immediate movent being of this kind
there is no need of an intermediate movent that is also moved (for
it is impossible that there should be an infinite series of movents,
each of which is itself moved by something else, since in an infinite
series there is no first term)-if then everything that is in motion
is moved by something, and the first movent is moved but not by anything
else, it much be moved by itself. 

This same argument may also be stated in another way as follows. Every
movent moves something and moves it with something, either with itself
or with something else: e.g. a man moves a thing either himself or
with a stick, and a thing is knocked down either by the wind itself
or by a stone propelled by the wind. But it is impossible for that
with which a thing is moved to move it without being moved by that
which imparts motion by its own agency: on the other hand, if a thing
imparts motion by its own agency, it is not necessary that there should
be anything else with which it imparts motion, whereas if there is
a different thing with which it imparts motion, there must be something
that imparts motion not with something else but with itself, or else
there will be an infinite series. If, then, anything is a movent while
being itself moved, the series must stop somewhere and not be infinite.
Thus, if the stick moves something in virtue of being moved by the
hand, the hand moves the stick: and if something else moves with the
hand, the hand also is moved by something different from itself. So
when motion by means of an instrument is at each stage caused by something
different from the instrument, this must always be preceded by something
else which imparts motion with itself. Therefore, if this last movent
is in motion and there is nothing else that moves it, it must move
itself. So this reasoning also shows that when a thing is moved, if
it is not moved immediately by something that moves itself, the series
brings us at some time or other to a movent of this kind.

And if we consider the matter in yet a third wa Ly we shall get this
same result as follows. If everything that is in motion is moved by
something that is in motion, ether this being in motion is an accidental
attribute of the movents in question, so that each of them moves something
while being itself in motion, but not always because it is itself
in motion, or it is not accidental but an essential attribute. Let
us consider the former alternative. If then it is an accidental attribute,
it is not necessary that that is in motion should be in motion: and
if this is so it is clear that there may be a time when nothing that
exists is in motion, since the accidental is not necessary but contingent.
Now if we assume the existence of a possibility, any conclusion that
we thereby reach will not be an impossibility though it may be contrary
to fact. But the nonexistence of motion is an impossibility: for we
have shown above that there must always be motion. 

Moreover, the conclusion to which we have been led is a reasonable
one. For there must be three things-the moved, the movent, and the
instrument of motion. Now the moved must be in motion, but it need
not move anything else: the instrument of motion must both move something
else and be itself in motion (for it changes together with the moved,
with which it is in contact and continuous, as is clear in the case
of things that move other things locally, in which case the two things
must up to a certain point be in contact): and the movent-that is
to say, that which causes motion in such a manner that it is not merely
the instrument of motion-must be unmoved. Now we have visual experience
of the last term in this series, namely that which has the capacity
of being in motion, but does not contain a motive principle, and also
of that which is in motion but is moved by itself and not by anything
else: it is reasonable, therefore, not to say necessary, to suppose
the existence of the third term also, that which causes motion but
is itself unmoved. So, too, Anaxagoras is right when he says that
Mind is impassive and unmixed, since he makes it the principle of
motion: for it could cause motion in this sense only by being itself
unmoved, and have supreme control only by being unmixed.

We will now take the second alternative. If the movement is not accidentally
but necessarily in motion-so that, if it were not in motion, it would
not move anything-then the movent, in so far as it is in motion, must
be in motion in one of two ways: it is moved either as that is which
is moved with the same kind of motion, or with a different kind-either
that which is heating, I mean, is itself in process of becoming hot,
that which is making healthy in process of becoming healthy, and that
which is causing locomotion in process of locomotion, or else that
which is making healthy is, let us say, in process of locomotion,
and that which is causing locomotion in process of, say, increase.
But it is evident that this is impossible. For if we adopt the first
assumption we have to make it apply within each of the very lowest
species into which motion can be divided: e.g. we must say that if
some one is teaching some lesson in geometry, he is also in process
of being taught that same lesson in geometry, and that if he is throwing
he is in process of being thrown in just the same manner. Or if we
reject this assumption we must say that one kind of motion is derived
from another; e.g. that that which is causing locomotion is in process
of increase, that which is causing this increase is in process of
being altered by something else, and that which is causing this alteration
is in process of suffering some different kind of motion. But the
series must stop somewhere, since the kinds of motion are limited;
and if we say that the process is reversible, and that that which
is causing alteration is in process of locomotion, we do no more than
if we had said at the outset that that which is causing locomotion
is in process of locomotion, and that one who is teaching is in process
of being taught: for it is clear that everything that is moved is
moved by the movent that is further back in the series as well as
by that which immediately moves it: in fact the earlier movent is
that which more strictly moves it. But this is of course impossible:
for it involves the consequence that one who is teaching is in process
of learning what he is teaching, whereas teaching necessarily implies
possessing knowledge, and learning not possessing it. Still more unreasonable
is the consequence involved that, since everything that is moved is
moved by something that is itself moved by something else, everything
that has a capacity for causing motion has as such a corresponding
capacity for being moved: i.e. it will have a capacity for being moved
in the sense in which one might say that everything that has a capacity
for making healthy, and exercises that capacity, has as such a capacity
for being made healthy, and that which has a capacity for building
has as such a capacity for being built. It will have the capacity
for being thus moved either immediately or through one or more links
(as it will if, while everything that has a capacity for causing motion
has as such a capacity for being moved by something else, the motion
that it has the capacity for suffering is not that with which it affects
what is next to it, but a motion of a different kind; e.g. that which
has a capacity for making healthy might as such have a capacity for
learn. the series, however, could be traced back, as we said before,
until at some time or other we arrived at the same kind of motion).
Now the first alternative is impossible, and the second is fantastic:
it is absurd that that which has a capacity for causing alteration
should as such necessarily have a capacity, let us say, for increase.
It is not necessary, therefore, that that which is moved should always
be moved by something else that is itself moved by something else:
so there will be an end to the series. Consequently the first thing
that is in motion will derive its motion either from something that
is at rest or from itself. But if there were any need to consider
which of the two, that which moves itself or that which is moved by
something else, is the cause and principle of motion, every one would
decide the former: for that which is itself independently a cause
is always prior as a cause to that which is so only in virtue of being
itself dependent upon something else that makes it so. 

We must therefore make a fresh start and consider the question; if
a thing moves itself, in what sense and in what manner does it do
so? Now everything that is in motion must be infinitely divisible,
for it has been shown already in our general course on Physics, that
everything that is essentially in motion is continuous. Now it is
impossible that that which moves itself should in its entirety move
itself: for then, while being specifically one and indivisible, it
would as a Whole both undergo and cause the same locomotion or alteration:
thus it would at the same time be both teaching and being taught (the
same thing), or both restoring to and being restored to the same health.
Moreover, we have established the fact that it is the movable that
is moved; and this is potentially, not actually, in motion, but the
potential is in process to actuality, and motion is an incomplete
actuality of the movable. The movent on the other hand is already
in activity: e.g. it is that which is hot that produces heat: in fact,
that which produces the form is always something that possesses it.
Consequently (if a thing can move itself as a whole), the same thing
in respect of the same thing may be at the same time both hot and
not hot. So, too, in every other case where the movent must be described
by the same name in the same sense as the moved. Therefore when a
thing moves itself it is one part of it that is the movent and another
part that is moved. But it is not self-moving in the sense that each
of the two parts is moved by the other part: the following considerations
make this evident. In the first place, if each of the two parts is
to move the other, there will be no first movent. If a thing is moved
by a series of movents, that which is earlier in the series is more
the cause of its being moved than that which comes next, and will
be more truly the movent: for we found that there are two kinds of
movent, that which is itself moved by something else and that which
derives its motion from itself: and that which is further from the
thing that is moved is nearer to the principle of motion than that
which is intermediate. In the second place, there is no necessity
for the movent part to be moved by anything but itself: so it can
only be accidentally that the other part moves it in return. I take
then the possible case of its not moving it: then there will be a
part that is moved and a part that is an unmoved movent. In the third
place, there is no necessity for the movent to be moved in return:
on the contrary the necessity that there should always be motion makes
it necessary that there should be some movent that is either unmoved
or moved by itself. In the fourth place we should then have a thing
undergoing the same motion that it is causing-that which is producing
heat, therefore, being heated. But as a matter of fact that which
primarily moves itself cannot contain either a single part that moves
itself or a number of parts each of which moves itself. For, if the
whole is moved by itself, it must be moved either by some part of
itself or as a whole by itself as a whole. If, then, it is moved in
virtue of some part of it being moved by that part itself, it is this
part that will be the primary self-movent, since, if this part is
separated from the whole, the part will still move itself, but the
whole will do so no longer. If on the other hand the whole is moved
by itself as a whole, it must be accidentally that the parts move
themselves: and therefore, their self-motion not being necessary,
we may take the case of their not being moved by themselves. Therefore
in the whole of the thing we may distinguish that which imparts motion
without itself being moved and that which is moved: for only in this
way is it possible for a thing to be self-moved. Further, if the whole
moves itself we may distinguish in it that which imparts the motion
and that which is moved: so while we say that AB is moved by itself,
we may also say that it is moved by A. And since that which imparts
motion may be either a thing that is moved by something else or a
thing that is unmoved, and that which is moved may be either a thing
that imparts motion to something else or a thing that does not, that
which moves itself must be composed of something that is unmoved but
imparts motion and also of something that is moved but does not necessarily
impart motion but may or may not do so. Thus let A be something that
imparts motion but is unmoved, B something that is moved by A and
moves G, G something that is moved by B but moves nothing (granted
that we eventually arrive at G we may take it that there is only one
intermediate term, though there may be more). Then the whole ABG moves
itself. But if I take away G, AB will move itself, A imparting motion
and B being moved, whereas G will not move itself or in fact be moved
at all. Nor again will BG move itself apart from A: for B imparts
motion only through being moved by something else, not through being
moved by any part of itself. So only AB moves itself. That which moves
itself, therefore, must comprise something that imparts motion but
is unmoved and something that is moved but does not necessarily move
anything else: and each of these two things, or at any rate one of
them, must be in contact with the other. If, then, that which imparts
motion is a continuous substance-that which is moved must of course
be so-it is clear that it is not through some part of the whole being
of such a nature as to be capable of moving itself that the whole
moves itself: it moves itself as a whole, both being moved and imparting
motion through containing a part that imparts motion and a part that
is moved. It does not impart motion as a whole nor is it moved as
a whole: it is A alone that imparts motion and B alone that is moved.
It is not true, further, that G is moved by A, which is impossible.

Here a difficulty arises: if something is taken away from A (supposing
that that which imparts motion but is unmoved is a continuous substance),
or from B the part that is moved, will the remainder of A continue
to impart motion or the remainder of B continue to be moved? If so,
it will not be AB primarily that is moved by itself, since, when something
is taken away from AB, the remainder of AB will still continue to
move itself. Perhaps we may state the case thus: there is nothing
to prevent each of the two parts, or at any rate one of them, that
which is moved, being divisible though actually undivided, so that
if it is divided it will not continue in the possession of the same
capacity: and so there is nothing to prevent self-motion residing
primarily in things that are potentially divisible. 

From what has been said, then, it is evident that that which primarily
imparts motion is unmoved: for, whether the series is closed at once
by that which is in motion but moved by something else deriving its
motion directly from the first unmoved, or whether the motion is derived
from what is in motion but moves itself and stops its own motion,
on both suppositions we have the result that in all cases of things
being in motion that which primarily imparts motion is unmoved.

Part 6

Since there must always be motion without intermission, there must
necessarily be something, one thing or it may be a plurality, that
first imparts motion, and this first movent must be unmoved. Now the
question whether each of the things that are unmoved but impart motion
is eternal is irrelevant to our present argument: but the following
considerations will make it clear that there must necessarily be some
such thing, which, while it has the capacity of moving something else,
is itself unmoved and exempt from all change, which can affect it
neither in an unqualified nor in an accidental sense. Let us suppose,
if any one likes, that in the case of certain things it is possible
for them at different times to be and not to be, without any process
of becoming and perishing (in fact it would seem to be necessary,
if a thing that has not parts at one time is and at another time is
not, that any such thing should without undergoing any process of
change at one time be and at another time not be). And let us further
suppose it possible that some principles that are unmoved but capable
of imparting motion at one time are and at another time are not. Even
so, this cannot be true of all such principles, since there must clearly
be something that causes things that move themselves at one time to
be and at another not to be. For, since nothing that has not parts
can be in motion, that which moves itself must as a whole have magnitude,
though nothing that we have said makes this necessarily true of every
movent. So the fact that some things become and others perish, and
that this is so continuously, cannot be caused by any one of those
things that, though they are unmoved, do not always exist: nor again
can it be caused by any of those which move certain particular things,
while others move other things. The eternity and continuity of the
process cannot be caused either by any one of them singly or by the
sum of them, because this causal relation must be eternal and necessary,
whereas the sum of these movents is infinite and they do not all exist
together. It is clear, then, that though there may be countless instances
of the perishing of some principles that are unmoved but impart motion,
and though many things that move themselves perish and are succeeded
by others that come into being, and though one thing that is unmoved
moves one thing while another moves another, nevertheless there is
something that comprehends them all, and that as something apart from
each one of them, and this it is that is the cause of the fact that
some things are and others are not and of the continuous process of
change: and this causes the motion of the other movents, while they
are the causes of the motion of other things. Motion, then, being
eternal, the first movent, if there is but one, will be eternal also:
if there are more than one, there will be a plurality of such eternal
movents. We ought, however, to suppose that there is one rather than
many, and a finite rather than an infinite number. When the consequences
of either assumption are the same, we should always assume that things
are finite rather than infinite in number, since in things constituted
by nature that which is finite and that which is better ought, if
possible, to be present rather than the reverse: and here it is sufficient
to assume only one movent, the first of unmoved things, which being
eternal will be the principle of motion to everything else.

The following argument also makes it evident that the first movent
must be something that is one and eternal. We have shown that there
must always be motion. That being so, motion must also be continuous,
because what is always is continuous, whereas what is merely in succession
is not continuous. But further, if motion is continuous, it is one:
and it is one only if the movent and the moved that constitute it
are each of them one, since in the event of a thing's being moved
now by one thing and now by another the whole motion will not be continuous
but successive. 

Moreover a conviction that there is a first unmoved something may
be reached not only from the foregoing arguments, but also by considering
again the principles operative in movents. Now it is evident that
among existing things there are some that are sometimes in motion
and sometimes at rest. This fact has served above to make it clear
that it is not true either that all things are in motion or that all
things are at rest or that some things are always at rest and the
remainder always in motion: on this matter proof is supplied by things
that fluctuate between the two and have the capacity of being sometimes
in motion and sometimes at rest. The existence of things of this kind
is clear to all: but we wish to explain also the nature of each of
the other two kinds and show that there are some things that are always
unmoved and some things that are always in motion. In the course of
our argument directed to this end we established the fact that everything
that is in motion is moved by something, and that the movent is either
unmoved or in motion, and that, if it is in motion, it is moved either
by itself or by something else and so on throughout the series: and
so we proceeded to the position that the first principle that directly
causes things that are in motion to be moved is that which moves itself,
and the first principle of the whole series is the unmoved. Further
it is evident from actual observation that there are things that have
the characteristic of moving themselves, e.g. the animal kingdom and
the whole class of living things. This being so, then, the view was
suggested that perhaps it may be possible for motion to come to be
in a thing without having been in existence at all before, because
we see this actually occurring in animals: they are unmoved at one
time and then again they are in motion, as it seems. We must grasp
the fact, therefore, that animals move themselves only with one kind
of motion, and that this is not strictly originated by them. The cause
of it is not derived from the animal itself: it is connected with
other natural motions in animals, which they do not experience through
their own instrumentality, e.g. increase, decrease, and respiration:
these are experienced by every animal while it is at rest and not
in motion in respect of the motion set up by its own agency: here
the motion is caused by the atmosphere and by many things that enter
into the animal: thus in some cases the cause is nourishment: when
it is being digested animals sleep, and when it is being distributed
through the system they awake and move themselves, the first principle
of this motion being thus originally derived from outside. Therefore
animals are not always in continuous motion by their own agency: it
is something else that moves them, itself being in motion and changing
as it comes into relation with each several thing that moves itself.
(Moreover in all these self-moving things the first movent and cause
of their self-motion is itself moved by itself, though in an accidental
sense: that is to say, the body changes its place, so that that which
is in the body changes its place also and is a self-movent through
its exercise of leverage.) Hence we may confidently conclude that
if a thing belongs to the class of unmoved movents that are also themselves
moved accidentally, it is impossible that it should cause continuous
motion. So the necessity that there should be motion continuously
requires that there should be a first movent that is unmoved even
accidentally, if, as we have said, there is to be in the world of
things an unceasing and undying motion, and the world is to remain
permanently self-contained and within the same limits: for if the
first principle is permanent, the universe must also be permanent,
since it is continuous with the first principle. (We must distinguish,
however, between accidental motion of a thing by itself and such motion
by something else, the former being confined to perishable things,
whereas the latter belongs also to certain first principles of heavenly
bodies, of all those, that is to say, that experience more than one
locomotion.) 

And further, if there is always something of this nature, a movent
that is itself unmoved and eternal, then that which is first moved
by it must be eternal. Indeed this is clear also from the consideration
that there would otherwise be no becoming and perishing and no change
of any kind in other things, which require something that is in motion
to move them: for the motion imparted by the unmoved will always be
imparted in the same way and be one and the same, since the unmoved
does not itself change in relation to that which is moved by it. But
that which is moved by something that, though it is in motion, is
moved directly by the unmoved stands in varying relations to the things
that it moves, so that the motion that it causes will not be always
the same: by reason of the fact that it occupies contrary positions
or assumes contrary forms at different times it will produce contrary
motions in each several thing that it moves and will cause it to be
at one time at rest and at another time in motion. 

The foregoing argument, then, has served to clear up the point about
which we raised a difficulty at the outset-why is it that instead
of all things being either in motion or at rest, or some things being
always in motion and the remainder always at rest, there are things
that are sometimes in motion and sometimes not? The cause of this
is now plain: it is because, while some things are moved by an eternal
unmoved movent and are therefore always in motion, other things are
moved by a movent that is in motion and changing, so that they too
must change. But the unmoved movent, as has been said, since it remains
permanently simple and unvarying and in the same state, will cause
motion that is one and simple. 

Part 7

This matter will be made clearer, however, if we start afresh from
another point. We must consider whether it is or is not possible that
there should be a continuous motion, and, if it is possible, which
this motion is, and which is the primary motion: for it is plain that
if there must always be motion, and a particular motion is primary
and continuous, then it is this motion that is imparted by the first
movent, and so it is necessarily one and the same and continuous and
primary. 

Now of the three kinds of motion that there are-motion in respect
of magnitude, motion in respect of affection, and motion in respect
of place-it is this last, which we call locomotion, that must be primary.
This may be shown as follows. It is impossible that there should be
increase without the previous occurrence of alteration: for that which
is increased, although in a sense it is increased by what is like
itself, is in a sense increased by what is unlike itself: thus it
is said that contrary is nourishment to contrary: but growth is effected
only by things becoming like to like. There must be alteration, then,
in that there is this change from contrary to contrary. But the fact
that a thing is altered requires that there should be something that
alters it, something e.g. that makes the potentially hot into the
actually hot: so it is plain that the movent does not maintain a uniform
relation to it but is at one time nearer to and at another farther
from that which is altered: and we cannot have this without locomotion.
If, therefore, there must always be motion, there must also always
be locomotion as the primary motion, and, if there is a primary as
distinguished from a secondary form of locomotion, it must be the
primary form. Again, all affections have their origin in condensation
and rarefaction: thus heavy and light, soft and hard, hot and cold,
are considered to be forms of density and rarity. But condensation
and rarefaction are nothing more than combination and separation,
processes in accordance with which substances are said to become and
perish: and in being combined and separated things must change in
respect of place. And further, when a thing is increased or decreased
its magnitude changes in respect of place. 

Again, there is another point of view from which it will be clearly
seen that locomotion is primary. As in the case of other things so
too in the case of motion the word 'primary' may be used in several
senses. A thing is said to be prior to other things when, if it does
not exist, the others will not exist, whereas it can exist without
the others: and there is also priority in time and priority in perfection
of existence. Let us begin, then, with the first sense. Now there
must be motion continuously, and there may be continuously either
continuous motion or successive motion, the former, however, in a
higher degree than the latter: moreover it is better that it should
be continuous rather than successive motion, and we always assume
the presence in nature of the better, if it be possible: since, then,
continuous motion is possible (this will be proved later: for the
present let us take it for granted), and no other motion can be continuous
except locomotion, locomotion must be primary. For there is no necessity
for the subject of locomotion to be the subject either of increase
or of alteration, nor need it become or perish: on the other hand
there cannot be any one of these processes without the existence of
the continuous motion imparted by the first movent. 

Secondly, locomotion must be primary in time: for this is the only
motion possible for things. It is true indeed that, in the case of
any individual thing that has a becoming, locomotion must be the last
of its motions: for after its becoming it first experiences alteration
and increase, and locomotion is a motion that belongs to such things
only when they are perfected. But there must previously be something
else that is in process of locomotion to be the cause even of the
becoming of things that become, without itself being in process of
becoming, as e.g. the begotten is preceded by what begot it: otherwise
becoming might be thought to be the primary motion on the ground that
the thing must first become. But though this is so in the case of
any individual thing that becomes, nevertheless before anything becomes,
something else must be in motion, not itself becoming but being, and
before this there must again be something else. And since becoming
cannot be primary-for, if it were, everything that is in motion would
be perishable-it is plain that no one of the motions next in order
can be prior to locomotion. By the motions next in order I mean increase
and then alteration, decrease, and perishing. All these are posterior
to becoming: consequently, if not even becoming is prior to locomotion,
then no one of the other processes of change is so either.

Thirdly, that which is in process of becoming appears universally
as something imperfect and proceeding to a first principle: and so
what is posterior in the order of becoming is prior in the order of
nature. Now all things that go through the process of becoming acquire
locomotion last. It is this that accounts for the fact that some living
things, e.g. plants and many kinds of animals, owing to lack of the
requisite organ, are entirely without motion, whereas others acquire
it in the course of their being perfected. Therefore, if the degree
in which things possess locomotion corresponds to the degree in which
they have realized their natural development, then this motion must
be prior to all others in respect of perfection of existence: and
not only for this reason but also because a thing that is in motion
loses its essential character less in the process of locomotion than
in any other kind of motion: it is the only motion that does not involve
a change of being in the sense in which there is a change in quality
when a thing is altered and a change in quantity when a thing is increased
or decreased. Above all it is plain that this motion, motion in respect
of place, is what is in the strictest sense produced by that which
moves itself; but it is the self-movent that we declare to be the
first principle of things that are moved and impart motion and the
primary source to which things that are in motion are to be referred.

It is clear, then, from the foregoing arguments that locomotion is
the primary motion. We have now to show which kind of locomotion is
primary. The same process of reasoning will also make clear at the
same time the truth of the assumption we have made both now and at
a previous stage that it is possible that there should be a motion
that is continuous and eternal. Now it is clear from the following
considerations that no other than locomotion can be continuous. Every
other motion and change is from an opposite to an opposite: thus for
the processes of becoming and perishing the limits are the existent
and the non-existent, for alteration the various pairs of contrary
affections, and for increase and decrease either greatness and smallness
or perfection and imperfection of magnitude: and changes to the respective
contraries are contrary changes. Now a thing that is undergoing any
particular kind of motion, but though previously existent has not
always undergone it, must previously have been at rest so far as that
motion is concerned. It is clear, then, that for the changing thing
the contraries will be states of rest. And we have a similar result
in the case of changes that are not motions: for becoming and perishing,
whether regarded simply as such without qualification or as affecting
something in particular, are opposites: therefore provided it is impossible
for a thing to undergo opposite changes at the same time, the change
will not be continuous, but a period of time will intervene between
the opposite processes. The question whether these contradictory changes
are contraries or not makes no difference, provided only it is impossible
for them both to be present to the same thing at the same time: the
point is of no importance to the argument. Nor does it matter if the
thing need not rest in the contradictory state, or if there is no
state of rest as a contrary to the process of change: it may be true
that the non-existent is not at rest, and that perishing is a process
to the non-existent. All that matters is the intervention of a time:
it is this that prevents the change from being continuous: so, too,
in our previous instances the important thing was not the relation
of contrariety but the impossibility of the two processes being present
to a thing at the same time. And there is no need to be disturbed
by the fact that on this showing there may be more than one contrary
to the same thing, that a particular motion will be contrary both
to rest and to motion in the contrary direction. We have only to grasp
the fact that a particular motion is in a sense the opposite both
of a state of rest and of the contrary motion, in the same way as
that which is of equal or standard measure is the opposite both of
that which surpasses it and of that which it surpasses, and that it
is impossible for the opposite motions or changes to be present to
a thing at the same time. Furthermore, in the case of becoming and
perishing it would seem to be an utterly absurd thing if as soon as
anything has become it must necessarily perish and cannot continue
to exist for any time: and, if this is true of becoming and perishing,
we have fair grounds for inferring the same to be true of the other
kinds of change, since it would be in the natural order of things
that they should be uniform in this respect. 

Part 8

Let us now proceed to maintain that it is possible that there should
be an infinite motion that is single and continuous, and that this
motion is rotatory motion. The motion of everything that is in process
of locomotion is either rotatory or rectilinear or a compound of the
two: consequently, if one of the former two is not continuous, that
which is composed of them both cannot be continuous either. Now it
is plain that if the locomotion of a thing is rectilinear and finite
it is not continuous locomotion: for the thing must turn back, and
that which turns back in a straight line undergoes two contrary locomotions,
since, so far as motion in respect of place is concerned, upward motion
is the contrary of downward motion, forward motion of backward motion,
and motion to the left of motion to the right, these being the pairs
of contraries in the sphere of place. But we have already defined
single and continuous motion to be motion of a single thing in a single
period of time and operating within a sphere admitting of no further
specific differentiation (for we have three things to consider, first
that which is in motion, e.g. a man or a god, secondly the 'when'
of the motion, that is to say, the time, and thirdly the sphere within
which it operates, which may be either place or affection or essential
form or magnitude): and contraries are specifically not one and the
same but distinct: and within the sphere of place we have the above-mentioned
distinctions. Moreover we have an indication that motion from A to
B is the contrary of motion from B to A in the fact that, if they
occur at the same time, they arrest and stop each other. And the same
is true in the case of a circle: the motion from A towards B is the
contrary of the motion from A towards G: for even if they are continuous
and there is no turning back they arrest each other, because contraries
annihilate or obstruct one another. On the other hand lateral motion
is not the contrary of upward motion. But what shows most clearly
that rectilinear motion cannot be continuous is the fact that turning
back necessarily implies coming to a stand, not only when it is a
straight line that is traversed, but also in the case of locomotion
in a circle (which is not the same thing as rotatory locomotion: for,
when a thing merely traverses a circle, it may either proceed on its
course without a break or turn back again when it has reached the
same point from which it started). We may assure ourselves of the
necessity of this coming to a stand not only on the strength of observation,
but also on theoretical grounds. We may start as follows: we have
three points, starting-point, middle-point, and finishing-point, of
which the middle-point in virtue of the relations in which it stands
severally to the other two is both a starting-point and a finishing-point,
and though numerically one is theoretically two. We have further the
distinction between the potential and the actual. So in the straight
line in question any one of the points lying between the two extremes
is potentially a middle-point: but it is not actually so unless that
which is in motion divides the line by coming to a stand at that point
and beginning its motion again: thus the middle-point becomes both
a starting-point and a goal, the starting-point of the latter part
and the finishing-point of the first part of the motion. This is the
case e.g. when A in the course of its locomotion comes to a stand
at B and starts again towards G: but when its motion is continuous
A cannot either have come to be or have ceased to be at the point
B: it can only have been there at the moment of passing, its passage
not being contained within any period of time except the whole of
which the particular moment is a dividing-point. To maintain that
it has come to be and ceased to be there will involve the consequence
that A in the course of its locomotion will always be coming to a
stand: for it is impossible that A should simultaneously have come
to be at B and ceased to be there, so that the two things must have
happened at different points of time, and therefore there will be
the intervening period of time: consequently A will be in a state
of rest at B, and similarly at all other points, since the same reasoning
holds good in every case. When to A, that which is in process of locomotion,
B, the middle-point, serves both as a finishing-point and as a starting-point
for its motion, A must come to a stand at B, because it makes it two
just as one might do in thought. However, the point A is the real
starting-point at which the moving body has ceased to be, and it is
at G that it has really come to be when its course is finished and
it comes to a stand. So this is how we must meet the difficulty that
then arises, which is as follows. Suppose the line E is equal to the
line Z, that A proceeds in continuous locomotion from the extreme
point of E to G, and that, at the moment when A is at the point B,
D is proceeding in uniform locomotion and with the same velocity as
A from the extremity of Z to H: then, says the argument, D will have
reached H before A has reached G for that which makes an earlier start
and departure must make an earlier arrival: the reason, then, for
the late arrival of A is that it has not simultaneously come to be
and ceased to be at B: otherwise it will not arrive later: for this
to happen it will be necessary that it should come to a stand there.
Therefore we must not hold that there was a moment when A came to
be at B and that at the same moment D was in motion from the extremity
of Z: for the fact of A's having come to be at B will involve the
fact of its also ceasing to be there, and the two events will not
be simultaneous, whereas the truth is that A is at B at a sectional
point of time and does not occupy time there. In this case, therefore,
where the motion of a thing is continuous, it is impossible to use
this form of expression. On the other hand in the case of a thing
that turns back in its course we must do so. For suppose H in the
course of its locomotion proceeds to D and then turns back and proceeds
downwards again: then the extreme point D has served as finishing-point
and as starting-point for it, one point thus serving as two: therefore
H must have come to a stand there: it cannot have come to be at D
and departed from D simultaneously, for in that case it would simultaneously
be there and not be there at the same moment. And here we cannot apply
the argument used to solve the difficulty stated above: we cannot
argue that H is at D at a sectional point of time and has not come
to be or ceased to be there. For here the goal that is reached is
necessarily one that is actually, not potentially, existent. Now the
point in the middle is potential: but this one is actual, and regarded
from below it is a finishing-point, while regarded from above it is
a starting-point, so that it stands in these same two respective relations
to the two motions. Therefore that which turns back in traversing
a rectilinear course must in so doing come to a stand. Consequently
there cannot be a continuous rectilinear motion that is eternal.

The same method should also be adopted in replying to those who ask,
in the terms of Zeno's argument, whether we admit that before any
distance can be traversed half the distance must be traversed, that
these half-distances are infinite in number, and that it is impossible
to traverse distances infinite in number-or some on the lines of this
same argument put the questions in another form, and would have us
grant that in the time during which a motion is in progress it should
be possible to reckon a half-motion before the whole for every half-distance
that we get, so that we have the result that when the whole distance
is traversed we have reckoned an infinite number, which is admittedly
impossible. Now when we first discussed the question of motion we
put forward a solution of this difficulty turning on the fact that
the period of time occupied in traversing the distance contains within
itself an infinite number of units: there is no absurdity, we said,
in supposing the traversing of infinite distances in infinite time,
and the element of infinity is present in the time no less than in
the distance. But, although this solution is adequate as a reply to
the questioner (the question asked being whether it is possible in
a finite time to traverse or reckon an infinite number of units),
nevertheless as an account of the fact and explanation of its true
nature it is inadequate. For suppose the distance to be left out of
account and the question asked to be no longer whether it is possible
in a finite time to traverse an infinite number of distances, and
suppose that the inquiry is made to refer to the time taken by itself
(for the time contains an infinite number of divisions): then this
solution will no longer be adequate, and we must apply the truth that
we enunciated in our recent discussion, stating it in the following
way. In the act of dividing the continuous distance into two halves
one point is treated as two, since we make it a starting-point and
a finishing-point: and this same result is also produced by the act
of reckoning halves as well as by the act of dividing into halves.
But if divisions are made in this way, neither the distance nor the
motion will be continuous: for motion if it is to be continuous must
relate to what is continuous: and though what is continuous contains
an infinite number of halves, they are not actual but potential halves.
If the halves are made actual, we shall get not a continuous but an
intermittent motion. In the case of reckoning the halves, it is clear
that this result follows: for then one point must be reckoned as two:
it will be the finishing-point of the one half and the starting-point
of the other, if we reckon not the one continuous whole but the two
halves. Therefore to the question whether it is possible to pass through
an infinite number of units either of time or of distance we must
reply that in a sense it is and in a sense it is not. If the units
are actual, it is not possible: if they are potential, it is possible.
For in the course of a continuous motion the traveller has traversed
an infinite number of units in an accidental sense but not in an unqualified
sense: for though it is an accidental characteristic of the distance
to be an infinite number of half-distances, this is not its real and
essential character. It is also plain that unless we hold that the
point of time that divides earlier from later always belongs only
to the later so far as the thing is concerned, we shall be involved
in the consequence that the same thing is at the same moment existent
and not existent, and that a thing is not existent at the moment when
it has become. It is true that the point is common to both times,
the earlier as well as the later, and that, while numerically one
and the same, it is theoretically not so, being the finishing-point
of the one and the starting-point of the other: but so far as the
thing is concerned it belongs to the later stage of what happens to
it. Let us suppose a time ABG and a thing D, D being white in the
time A and not-white in the time B. Then D is at the moment G white
and not-white: for if we were right in saying that it is white during
the whole time A, it is true to call it white at any moment of A,
and not-white in B, and G is in both A and B. We must not allow, therefore,
that it is white in the whole of A, but must say that it is so in
all of it except the last moment G. G belongs already to the later
period, and if in the whole of A not-white was in process of becoming
and white of perishing, at G the process is complete. And so G is
the first moment at which it is true to call the thing white or not
white respectively. Otherwise a thing may be non-existent at the moment
when it has become and existent at the moment when it has perished:
or else it must be possible for a thing at the same time to be white
and not white and in fact to be existent and non-existent. Further,
if anything that exists after having been previously non-existent
must become existent and does not exist when it is becoming, time
cannot be divisible into time-atoms. For suppose that D was becoming
white in the time A and that at another time B, a time-atom consecutive
with the last atom of A, D has already become white and so is white
at that moment: then, inasmuch as in the time A it was becoming white
and so was not white and at the moment B it is white, there must have
been a becoming between A and B and therefore also a time in which
the becoming took place. On the other hand, those who deny atoms of
time (as we do) are not affected by this argument: according to them
D has become and so is white at the last point of the actual time
in which it was becoming white: and this point has no other point
consecutive with or in succession to it, whereas time-atoms are conceived
as successive. Moreover it is clear that if D was becoming white in
the whole time A, the time occupied by it in having become white in
addition to having been in process of becoming white is no more than
all that it occupied in the mere process of becoming white.

These and such-like, then, are the arguments for our conclusion that
derive cogency from the fact that they have a special bearing on the
point at issue. If we look at the question from the point of view
of general theory, the same result would also appear to be indicated
by the following arguments. Everything whose motion is continuous
must, on arriving at any point in the course of its locomotion, have
been previously also in process of locomotion to that point, if it
is not forced out of its path by anything: e.g. on arriving at B a
thing must also have been in process of locomotion to B, and that
not merely when it was near to B, but from the moment of its starting
on its course, since there can be, no reason for its being so at any
particular stage rather than at an earlier one. So, too, in the case
of the other kinds of motion. Now we are to suppose that a thing proceeds
in locomotion from A to G and that at the moment of its arrival at
G the continuity of its motion is unbroken and will remain so until
it has arrived back at A. Then when it is undergoing locomotion from
A to G it is at the same time undergoing also its locomotion to A
from G: consequently it is simultaneously undergoing two contrary
motions, since the two motions that follow the same straight line
are contrary to each other. With this consequence there also follows
another: we have a thing that is in process of change from a position
in which it has not yet been: so, inasmuch as this is impossible,
the thing must come to a stand at G. Therefore the motion is not a
single motion, since motion that is interrupted by stationariness
is not single. 

Further, the following argument will serve better to make this point
clear universally in respect of every kind of motion. If the motion
undergone by that which is in motion is always one of those already
enumerated, and the state of rest that it undergoes is one of those
that are the opposites of the motions (for we found that there could
be no other besides these), and moreover that which is undergoing
but does not always undergo a particular motion (by this I mean one
of the various specifically distinct motions, not some particular
part of the whole motion) must have been previously undergoing the
state of rest that is the opposite of the motion, the state of rest
being privation of motion; then, inasmuch as the two motions that
follow the same straight line are contrary motions, and it is impossible
for a thing to undergo simultaneously two contrary motions, that which
is undergoing locomotion from A to G cannot also simultaneously be
undergoing locomotion from G to A: and since the latter locomotion
is not simultaneous with the former but is still to be undergone,
before it is undergone there must occur a state of rest at G: for
this, as we found, is the state of rest that is the opposite of the
motion from G. The foregoing argument, then, makes it plain that the
motion in question is not continuous. 

Our next argument has a more special bearing than the foregoing on
the point at issue. We will suppose that there has occurred in something
simultaneously a perishing of not-white and a becoming of white. Then
if the alteration to white and from white is a continuous process
and the white does not remain any time, there must have occurred simultaneously
a perishing of not-white, a becoming of white, and a becoming of not-white:
for the time of the three will be the same. 

Again, from the continuity of the time in which the motion takes place
we cannot infer continuity in the motion, but only successiveness:
in fact, how could contraries, e.g. whiteness and blackness, meet
in the same extreme point? 

On the other hand, in motion on a circular line we shall find singleness
and continuity: for here we are met by no impossible consequence:
that which is in motion from A will in virtue of the same direction
of energy be simultaneously in motion to A (since it is in motion
to the point at which it will finally arrive), and yet will not be
undergoing two contrary or opposite motions: for a motion to a point
and a motion from that point are not always contraries or opposites:
they are contraries only if they are on the same straight line (for
then they are contrary to one another in respect of place, as e.g.
the two motions along the diameter of the circle, since the ends of
this are at the greatest possible distance from one another), and
they are opposites only if they are along the same line. Therefore
in the case we are now considering there is nothing to prevent the
motion being continuous and free from all intermission: for rotatory
motion is motion of a thing from its place to its place, whereas rectilinear
motion is motion from its place to another place. 

Moreover the progress of rotatory motion is never localized within
certain fixed limits, whereas that of rectilinear motion repeatedly
is so. Now a motion that is always shifting its ground from moment
to moment can be continuous: but a motion that is repeatedly localized
within certain fixed limits cannot be so, since then the same thing
would have to undergo simultaneously two opposite motions. So, too,
there cannot be continuous motion in a semicircle or in any other
arc of a circle, since here also the same ground must be traversed
repeatedly and two contrary processes of change must occur. The reason
is that in these motions the starting-point and the termination do
not coincide, whereas in motion over a circle they do coincide, and
so this is the only perfect motion. 

This differentiation also provides another means of showing that the
other kinds of motion cannot be continuous either: for in all of them
we find that there is the same ground to be traversed repeatedly;
thus in alteration there are the intermediate stages of the process,
and in quantitative change there are the intervening degrees of magnitude:
and in becoming and perishing the same thing is true. It makes no
difference whether we take the intermediate stages of the process
to be few or many, or whether we add or subtract one: for in either
case we find that there is still the same ground to be traversed repeatedly.
Moreover it is plain from what has been said that those physicists
who assert that all sensible things are always in motion are wrong:
for their motion must be one or other of the motions just mentioned:
in fact they mostly conceive it as alteration (things are always in
flux and decay, they say), and they go so far as to speak even of
becoming and perishing as a process of alteration. On the other hand,
our argument has enabled us to assert the fact, applying universally
to all motions, that no motion admits of continuity except rotatory
motion: consequently neither alteration nor increase admits of continuity.
We need now say no more in support of the position that there is no
process of change that admits of infinity or continuity except rotatory
locomotion. 

Part 9

It can now be shown plainly that rotation is the primary locomotion.
Every locomotion, as we said before, is either rotatory or rectilinear
or a compound of the two: and the two former must be prior to the
last, since they are the elements of which the latter consists. Moreover
rotatory locomotion is prior to rectilinear locomotion, because it
is more simple and complete, which may be shown as follows. The straight
line traversed in rectilinear motion cannot be infinite: for there
is no such thing as an infinite straight line; and even if there were,
it would not be traversed by anything in motion: for the impossible
does not happen and it is impossible to traverse an infinite distance.
On the other hand rectilinear motion on a finite straight line is
if it turns back a composite motion, in fact two motions, while if
it does not turn back it is incomplete and perishable: and in the
order of nature, of definition, and of time alike the complete is
prior to the incomplete and the imperishable to the perishable. Again,
a motion that admits of being eternal is prior to one that does not.
Now rotatory motion can be eternal: but no other motion, whether locomotion
or motion of any other kind, can be so, since in all of them rest
must occur and with the occurrence of rest the motion has perished.
Moreover the result at which we have arrived, that rotatory motion
is single and continuous, and rectilinear motion is not, is a reasonable
one. In rectilinear motion we have a definite starting-point, finishing-point,
middle-point, which all have their place in it in such a way that
there is a point from which that which is in motion can be said to
start and a point at which it can be said to finish its course (for
when anything is at the limits of its course, whether at the starting-point
or at the finishing-point, it must be in a state of rest). On the
other hand in circular motion there are no such definite points: for
why should any one point on the line be a limit rather than any other?
Any one point as much as any other is alike starting-point, middle-point,
and finishing-point, so that we can say of certain things both that
they are always and that they never are at a starting-point and at
a finishing-point (so that a revolving sphere, while it is in motion,
is also in a sense at rest, for it continues to occupy the same place).
The reason of this is that in this case all these characteristics
belong to the centre: that is to say, the centre is alike starting-point,
middle-point, and finishing-point of the space traversed; consequently
since this point is not a point on the circular line, there is no
point at which that which is in process of locomotion can be in a
state of rest as having traversed its course, because in its locomotion
it is proceeding always about a central point and not to an extreme
point: therefore it remains still, and the whole is in a sense always
at rest as well as continuously in motion. Our next point gives a
convertible result: on the one hand, because rotation is the measure
of motions it must be the primary motion (for all things are measured
by what is primary): on the other hand, because rotation is the primary
motion it is the measure of all other motions. Again, rotatory motion
is also the only motion that admits of being regular. In rectilinear
locomotion the motion of things in leaving the starting-point is not
uniform with their motion in approaching the finishing-point, since
the velocity of a thing always increases proportionately as it removes
itself farther from its position of rest: on the other hand rotatory
motion is the only motion whose course is naturally such that it has
no starting-point or finishing-point in itself but is determined from
elsewhere. 

As to locomotion being the primary motion, this is a truth that is
attested by all who have ever made mention of motion in their theories:
they all assign their first principles of motion to things that impart
motion of this kind. Thus 'separation' and 'combination' are motions
in respect of place, and the motion imparted by 'Love' and 'Strife'
takes these forms, the latter 'separating' and the former 'combining'.
Anaxagoras, too, says that 'Mind', his first movent, 'separates'.
Similarly those who assert no cause of this kind but say that 'void'
accounts for motion-they also hold that the motion of natural substance
is motion in respect of place: for their motion that is accounted
for by 'void' is locomotion, and its sphere of operation may be said
to be place. Moreover they are of opinion that the primary substances
are not subject to any of the other motions, though the things that
are compounds of these substances are so subject: the processes of
increase and decrease and alteration, they say, are effects of the
'combination' and 'separation' of atoms. It is the same, too, with
those who make out that the becoming or perishing of a thing is accounted
for by 'density' or 'rarity': for it is by 'combination' and 'separation'
that the place of these things in their systems is determined. Moreover
to these we may add those who make Soul the cause of motion: for they
say that things that undergo motion have as their first principle
'that which moves itself': and when animals and all living things
move themselves, the motion is motion in respect of place. Finally
it is to be noted that we say that a thing 'is in motion' in the strict
sense of the term only when its motion is motion in respect of place:
if a thing is in process of increase or decrease or is undergoing
some alteration while remaining at rest in the same place, we say
that it is in motion in some particular respect: we do not say that
it 'is in motion' without qualification. 

Our present position, then, is this: We have argued that there always
was motion and always will be motion throughout all time, and we have
explained what is the first principle of this eternal motion: we have
explained further which is the primary motion and which is the only
motion that can be eternal: and we have pronounced the first movent
to be unmoved. 

Part 10

We have now to assert that the first movent must be without parts
and without magnitude, beginning with the establishment of the premisses
on which this conclusion depends. 

One of these premisses is that nothing finite can cause motion during
an infinite time. We have three things, the movent, the moved, and
thirdly that in which the motion takes place, namely the time: and
these are either all infinite or all finite or partly-that is to say
two of them or one of them-finite and partly infinite. Let A be the
movement, B the moved, and G the infinite time. Now let us suppose
that D moves E, a part of B. Then the time occupied by this motion
cannot be equal to G: for the greater the amount moved, the longer
the time occupied. It follows that the time Z is not infinite. Now
we see that by continuing to add to D, I shall use up A and by continuing
to add to E, I shall use up B: but I shall not use up the time by
continually subtracting a corresponding amount from it, because it
is infinite. Consequently the duration of the part of G which is occupied
by all A in moving the whole of B, will be finite. Therefore a finite
thing cannot impart to anything an infinite motion. It is clear, then,
that it is impossible for the finite to cause motion during an infinite
time. 

It has now to be shown that in no case is it possible for an infinite
force to reside in a finite magnitude. This can be shown as follows:
we take it for granted that the greater force is always that which
in less time than another does an equal amount of work when engaged
in any activity-in heating, for example, or sweetening or throwing;
in fact, in causing any kind of motion. Then that on which the forces
act must be affected to some extent by our supposed finite magnitude
possessing an infinite force as well as by anything else, in fact
to a greater extent than by anything else, since the infinite force
is greater than any other. But then there cannot be any time in which
its action could take place. Suppose that A is the time occupied by
the infinite power in the performance of an act of heating or pushing,
and that AB is the time occupied by a finite power in the performance
of the same act: then by adding to the latter another finite power
and continually increasing the magnitude of the power so added I shall
at some time or other reach a point at which the finite power has
completed the motive act in the time A: for by continual addition
to a finite magnitude I must arrive at a magnitude that exceeds any
assigned limit, and in the same way by continual subtraction I must
arrive at one that falls short of any assigned limit. So we get the
result that the finite force will occupy the same amount of time in
performing the motive act as the infinite force. But this is impossible.
Therefore nothing finite can possess an infinite force. So it is also
impossible for a finite force to reside in an infinite magnitude.
It is true that a greater force can reside in a lesser magnitude:
but the superiority of any such greater force can be still greater
if the magnitude in which it resides is greater. Now let AB be an
infinite magnitude. Then BG possesses a certain force that occupies
a certain time, let us say the time Z in moving D. Now if I take a
magnitude twice as great at BG, the time occupied by this magnitude
in moving D will be half of EZ (assuming this to be the proportion):
so we may call this time ZH. That being so, by continually taking
a greater magnitude in this way I shall never arrive at the full AB,
whereas I shall always be getting a lesser fraction of the time given.
Therefore the force must be infinite, since it exceeds any finite
force. Moreover the time occupied by the action of any finite force
must also be finite: for if a given force moves something in a certain
time, a greater force will do so in a lesser time, but still a definite
time, in inverse proportion. But a force must always be infinite-just
as a number or a magnitude is-if it exceeds all definite limits. This
point may also be proved in another way-by taking a finite magnitude
in which there resides a force the same in kind as that which resides
in the infinite magnitude, so that this force will be a measure of
the finite force residing in the infinite magnitude. 

It is plain, then, from the foregoing arguments that it is impossible
for an infinite force to reside in a finite magnitude or for a finite
force to reside in an infinite magnitude. But before proceeding to
our conclusion it will be well to discuss a difficulty that arises
in connexion with locomotion. If everything that is in motion with
the exception of things that move themselves is moved by something
else, how is it that some things, e.g. things thrown, continue to
be in motion when their movent is no longer in contact with them?
If we say that the movent in such cases moves something else at the
same time, that the thrower e.g. also moves the air, and that this
in being moved is also a movent, then it would be no more possible
for this second thing than for the original thing to be in motion
when the original movent is not in contact with it or moving it: all
the things moved would have to be in motion simultaneously and also
to have ceased simultaneously to be in motion when the original movent
ceases to move them, even if, like the magnet, it makes that which
it has moved capable of being a movent. Therefore, while we must accept
this explanation to the extent of saying that the original movent
gives the power of being a movent either to air or to water or to
something else of the kind, naturally adapted for imparting and undergoing
motion, we must say further that this thing does not cease simultaneously
to impart motion and to undergo motion: it ceases to be in motion
at the moment when its movent ceases to move it, but it still remains
a movent, and so it causes something else consecutive with it to be
in motion, and of this again the same may be said. The motion begins
to cease when the motive force produced in one member of the consecutive
series is at each stage less than that possessed by the preceding
member, and it finally ceases when one member no longer causes the
next member to be a movent but only causes it to be in motion. The
motion of these last two-of the one as movent and of the other as
moved-must cease simultaneously, and with this the whole motion ceases.
Now the things in which this motion is produced are things that admit
of being sometimes in motion and sometimes at rest, and the motion
is not continuous but only appears so: for it is motion of things
that are either successive or in contact, there being not one movent
but a number of movents consecutive with one another: and so motion
of this kind takes place in air and water. Some say that it is 'mutual
replacement': but we must recognize that the difficulty raised cannot
be solved otherwise than in the way we have described. So far as they
are affected by 'mutual replacement', all the members of the series
are moved and impart motion simultaneously, so that their motions
also cease simultaneously: but our present problem concerns the appearance
of continuous motion in a single thing, and therefore, since it cannot
be moved throughout its motion by the same movent, the question is,
what moves it? 

Resuming our main argument, we proceed from the positions that there
must be continuous motion in the world of things, that this is a single
motion, that a single motion must be a motion of a magnitude (for
that which is without magnitude cannot be in motion), and that the
magnitude must be a single magnitude moved by a single movent (for
otherwise there will not be continuous motion but a consecutive series
of separate motions), and that if the movement is a single thing,
it is either itself in motion or itself unmoved: if, then, it is in
motion, it will have to be subject to the same conditions as that
which it moves, that is to say it will itself be in process of change
and in being so will also have to be moved by something: so we have
a series that must come to an end, and a point will be reached at
which motion is imparted by something that is unmoved. Thus we have
a movent that has no need to change along with that which it moves
but will be able to cause motion always (for the causing of motion
under these conditions involves no effort): and this motion alone
is regular, or at least it is so in a higher degree than any other,
since the movent is never subject to any change. So, too, in order
that the motion may continue to be of the same character, the moved
must not be subject to change in respect of its relation to the movent.
Moreover the movent must occupy either the centre or the circumference,
since these are the first principles from which a sphere is derived.
But the things nearest the movent are those whose motion is quickest,
and in this case it is the motion of the circumference that is the
quickest: therefore the movent occupies the circumference.

There is a further difficulty in supposing it to be possible for anything
that is in motion to cause motion continuously and not merely in the
way in which it is caused by something repeatedly pushing (in which
case the continuity amounts to no more than successiveness). Such
a movent must either itself continue to push or pull or perform both
these actions, or else the action must be taken up by something else
and be passed on from one movent to another (the process that we described
before as occurring in the case of things thrown, since the air or
the water, being divisible, is a movent only in virtue of the fact
that different parts of the air are moved one after another): and
in either case the motion cannot be a single motion, but only a consecutive
series of motions. The only continuous motion, then, is that which
is caused by the unmoved movent: and this motion is continuous because
the movent remains always invariable, so that its relation to that
which it moves remains also invariable and continuous. 

Now that these points are settled, it is clear that the first unmoved
movent cannot have any magnitude. For if it has magnitude, this must
be either a finite or an infinite magnitude. Now we have already'proved
in our course on Physics that there cannot be an infinite magnitude:
and we have now proved that it is impossible for a finite magnitude
to have an infinite force, and also that it is impossible for a thing
to be moved by a finite magnitude during an infinite time. But the
first movent causes a motion that is eternal and does cause it during
an infinite time. It is clear, therefore, that the first movent is
indivisible and is without parts and without magnitude. 

THE END

% chapter physics (end)
% \chapter{Metaphysics} % (fold)
\label{cha:metaphysics}


Metaphysics
By Aristotle


Translated by W. D. Ross

----------------------------------------------------------------------

BOOK I

Part 1 

"ALL men by nature desire to know. An indication of this is the delight
we take in our senses; for even apart from their usefulness they are
loved for themselves; and above all others the sense of sight. For
not only with a view to action, but even when we are not going to
do anything, we prefer seeing (one might say) to everything else.
The reason is that this, most of all the senses, makes us know and
brings to light many differences between things. 

"By nature animals are born with the faculty of sensation, and from
sensation memory is produced in some of them, though not in others.
And therefore the former are more intelligent and apt at learning
than those which cannot remember; those which are incapable of hearing
sounds are intelligent though they cannot be taught, e.g. the bee,
and any other race of animals that may be like it; and those which
besides memory have this sense of hearing can be taught.

"The animals other than man live by appearances and memories, and
have but little of connected experience; but the human race lives
also by art and reasonings. Now from memory experience is produced
in men; for the several memories of the same thing produce finally
the capacity for a single experience. And experience seems pretty
much like science and art, but really science and art come to men
through experience; for 'experience made art', as Polus says, 'but
inexperience luck.' Now art arises when from many notions gained by
experience one universal judgement about a class of objects is produced.
For to have a judgement that when Callias was ill of this disease
this did him good, and similarly in the case of Socrates and in many
individual cases, is a matter of experience; but to judge that it
has done good to all persons of a certain constitution, marked off
in one class, when they were ill of this disease, e.g. to phlegmatic
or bilious people when burning with fevers-this is a matter of art.

"With a view to action experience seems in no respect inferior to
art, and men of experience succeed even better than those who have
theory without experience. (The reason is that experience is knowledge
of individuals, art of universals, and actions and productions are
all concerned with the individual; for the physician does not cure
man, except in an incidental way, but Callias or Socrates or some
other called by some such individual name, who happens to be a man.
If, then, a man has the theory without the experience, and recognizes
the universal but does not know the individual included in this, he
will often fail to cure; for it is the individual that is to be cured.)
But yet we think that knowledge and understanding belong to art rather
than to experience, and we suppose artists to be wiser than men of
experience (which implies that Wisdom depends in all cases rather
on knowledge); and this because the former know the cause, but the
latter do not. For men of experience know that the thing is so, but
do not know why, while the others know the 'why' and the cause. Hence
we think also that the masterworkers in each craft are more honourable
and know in a truer sense and are wiser than the manual workers, because
they know the causes of the things that are done (we think the manual
workers are like certain lifeless things which act indeed, but act
without knowing what they do, as fire burns,-but while the lifeless
things perform each of their functions by a natural tendency, the
labourers perform them through habit); thus we view them as being
wiser not in virtue of being able to act, but of having the theory
for themselves and knowing the causes. And in general it is a sign
of the man who knows and of the man who does not know, that the former
can teach, and therefore we think art more truly knowledge than experience
is; for artists can teach, and men of mere experience cannot.

"Again, we do not regard any of the senses as Wisdom; yet surely these
give the most authoritative knowledge of particulars. But they do
not tell us the 'why' of anything-e.g. why fire is hot; they only
say that it is hot. 

"At first he who invented any art whatever that went beyond the common
perceptions of man was naturally admired by men, not only because
there was something useful in the inventions, but because he was thought
wise and superior to the rest. But as more arts were invented, and
some were directed to the necessities of life, others to recreation,
the inventors of the latter were naturally always regarded as wiser
than the inventors of the former, because their branches of knowledge
did not aim at utility. Hence when all such inventions were already
established, the sciences which do not aim at giving pleasure or at
the necessities of life were discovered, and first in the places where
men first began to have leisure. This is why the mathematical arts
were founded in Egypt; for there the priestly caste was allowed to
be at leisure. 

"We have said in the Ethics what the difference is between art and
science and the other kindred faculties; but the point of our present
discussion is this, that all men suppose what is called Wisdom to
deal with the first causes and the principles of things; so that,
as has been said before, the man of experience is thought to be wiser
than the possessors of any sense-perception whatever, the artist wiser
than the men of experience, the masterworker than the mechanic, and
the theoretical kinds of knowledge to be more of the nature of Wisdom
than the productive. Clearly then Wisdom is knowledge about certain
principles and causes. 

Part 2 "

"Since we are seeking this knowledge, we must inquire of what kind
are the causes and the principles, the knowledge of which is Wisdom.
If one were to take the notions we have about the wise man, this might
perhaps make the answer more evident. We suppose first, then, that
the wise man knows all things, as far as possible, although he has
not knowledge of each of them in detail; secondly, that he who can
learn things that are difficult, and not easy for man to know, is
wise (sense-perception is common to all, and therefore easy and no
mark of Wisdom); again, that he who is more exact and more capable
of teaching the causes is wiser, in every branch of knowledge; and
that of the sciences, also, that which is desirable on its own account
and for the sake of knowing it is more of the nature of Wisdom than
that which is desirable on account of its results, and the superior
science is more of the nature of Wisdom than the ancillary; for the
wise man must not be ordered but must order, and he must not obey
another, but the less wise must obey him. 

"Such and so many are the notions, then, which we have about Wisdom
and the wise. Now of these characteristics that of knowing all things
must belong to him who has in the highest degree universal knowledge;
for he knows in a sense all the instances that fall under the universal.
And these things, the most universal, are on the whole the hardest
for men to know; for they are farthest from the senses. And the most
exact of the sciences are those which deal most with first principles;
for those which involve fewer principles are more exact than those
which involve additional principles, e.g. arithmetic than geometry.
But the science which investigates causes is also instructive, in
a higher degree, for the people who instruct us are those who tell
the causes of each thing. And understanding and knowledge pursued
for their own sake are found most in the knowledge of that which is
most knowable (for he who chooses to know for the sake of knowing
will choose most readily that which is most truly knowledge, and such
is the knowledge of that which is most knowable); and the first principles
and the causes are most knowable; for by reason of these, and from
these, all other things come to be known, and not these by means of
the things subordinate to them. And the science which knows to what
end each thing must be done is the most authoritative of the sciences,
and more authoritative than any ancillary science; and this end is
the good of that thing, and in general the supreme good in the whole
of nature. Judged by all the tests we have mentioned, then, the name
in question falls to the same science; this must be a science that
investigates the first principles and causes; for the good, i.e. the
end, is one of the causes. 

"That it is not a science of production is clear even from the history
of the earliest philosophers. For it is owing to their wonder that
men both now begin and at first began to philosophize; they wondered
originally at the obvious difficulties, then advanced little by little
and stated difficulties about the greater matters, e.g. about the
phenomena of the moon and those of the sun and of the stars, and about
the genesis of the universe. And a man who is puzzled and wonders
thinks himself ignorant (whence even the lover of myth is in a sense
a lover of Wisdom, for the myth is composed of wonders); therefore
since they philosophized order to escape from ignorance, evidently
they were pursuing science in order to know, and not for any utilitarian
end. And this is confirmed by the facts; for it was when almost all
the necessities of life and the things that make for comfort and recreation
had been secured, that such knowledge began to be sought. Evidently
then we do not seek it for the sake of any other advantage; but as
the man is free, we say, who exists for his own sake and not for another's,
so we pursue this as the only free science, for it alone exists for
its own sake. 

"Hence also the possession of it might be justly regarded as beyond
human power; for in many ways human nature is in bondage, so that
according to Simonides 'God alone can have this privilege', and it
is unfitting that man should not be content to seek the knowledge
that is suited to him. If, then, there is something in what the poets
say, and jealousy is natural to the divine power, it would probably
occur in this case above all, and all who excelled in this knowledge
would be unfortunate. But the divine power cannot be jealous (nay,
according to the proverb, 'bards tell a lie'), nor should any other
science be thought more honourable than one of this sort. For the
most divine science is also most honourable; and this science alone
must be, in two ways, most divine. For the science which it would
be most meet for God to have is a divine science, and so is any science
that deals with divine objects; and this science alone has both these
qualities; for (1) God is thought to be among the causes of all things
and to be a first principle, and (2) such a science either God alone
can have, or God above all others. All the sciences, indeed, are more
necessary than this, but none is better. 

"Yet the acquisition of it must in a sense end in something which
is the opposite of our original inquiries. For all men begin, as we
said, by wondering that things are as they are, as they do about self-moving
marionettes, or about the solstices or the incommensurability of the
diagonal of a square with the side; for it seems wonderful to all
who have not yet seen the reason, that there is a thing which cannot
be measured even by the smallest unit. But we must end in the contrary
and, according to the proverb, the better state, as is the case in
these instances too when men learn the cause; for there is nothing
which would surprise a geometer so much as if the diagonal turned
out to be commensurable. 

"We have stated, then, what is the nature of the science we are searching
for, and what is the mark which our search and our whole investigation
must reach. 

Part 3 "

"Evidently we have to acquire knowledge of the original causes (for
we say we know each thing only when we think we recognize its first
cause), and causes are spoken of in four senses. In one of these we
mean the substance, i.e. the essence (for the 'why' is reducible finally
to the definition, and the ultimate 'why' is a cause and principle);
in another the matter or substratum, in a third the source of the
change, and in a fourth the cause opposed to this, the purpose and
the good (for this is the end of all generation and change). We have
studied these causes sufficiently in our work on nature, but yet let
us call to our aid those who have attacked the investigation of being
and philosophized about reality before us. For obviously they too
speak of certain principles and causes; to go over their views, then,
will be of profit to the present inquiry, for we shall either find
another kind of cause, or be more convinced of the correctness of
those which we now maintain. 

"Of the first philosophers, then, most thought the principles which
were of the nature of matter were the only principles of all things.
That of which all things that are consist, the first from which they
come to be, the last into which they are resolved (the substance remaining,
but changing in its modifications), this they say is the element and
this the principle of things, and therefore they think nothing is
either generated or destroyed, since this sort of entity is always
conserved, as we say Socrates neither comes to be absolutely when
he comes to be beautiful or musical, nor ceases to be when loses these
characteristics, because the substratum, Socrates himself remains.
just so they say nothing else comes to be or ceases to be; for there
must be some entity-either one or more than one-from which all other
things come to be, it being conserved. 

"Yet they do not all agree as to the number and the nature of these
principles. Thales, the founder of this type of philosophy, says the
principle is water (for which reason he declared that the earth rests
on water), getting the notion perhaps from seeing that the nutriment
of all things is moist, and that heat itself is generated from the
moist and kept alive by it (and that from which they come to be is
a principle of all things). He got his notion from this fact, and
from the fact that the seeds of all things have a moist nature, and
that water is the origin of the nature of moist things. 

"Some think that even the ancients who lived long before the present
generation, and first framed accounts of the gods, had a similar view
of nature; for they made Ocean and Tethys the parents of creation,
and described the oath of the gods as being by water, to which they
give the name of Styx; for what is oldest is most honourable, and
the most honourable thing is that by which one swears. It may perhaps
be uncertain whether this opinion about nature is primitive and ancient,
but Thales at any rate is said to have declared himself thus about
the first cause. Hippo no one would think fit to include among these
thinkers, because of the paltriness of his thought. 

"Anaximenes and Diogenes make air prior to water, and the most primary
of the simple bodies, while Hippasus of Metapontium and Heraclitus
of Ephesus say this of fire, and Empedocles says it of the four elements
(adding a fourth-earth-to those which have been named); for these,
he says, always remain and do not come to be, except that they come
to be more or fewer, being aggregated into one and segregated out
of one. 

"Anaxagoras of Clazomenae, who, though older than Empedocles, was
later in his philosophical activity, says the principles are infinite
in number; for he says almost all the things that are made of parts
like themselves, in the manner of water or fire, are generated and
destroyed in this way, only by aggregation and segregation, and are
not in any other sense generated or destroyed, but remain eternally.

"From these facts one might think that the only cause is the so-called
material cause; but as men thus advanced, the very facts opened the
way for them and joined in forcing them to investigate the subject.
However true it may be that all generation and destruction proceed
from some one or (for that matter) from more elements, why does this
happen and what is the cause? For at least the substratum itself does
not make itself change; e.g. neither the wood nor the bronze causes
the change of either of them, nor does the wood manufacture a bed
and the bronze a statue, but something else is the cause of the change.
And to seek this is to seek the second cause, as we should say,-that
from which comes the beginning of the movement. Now those who at the
very beginning set themselves to this kind of inquiry, and said the
substratum was one, were not at all dissatisfied with themselves;
but some at least of those who maintain it to be one-as though defeated
by this search for the second cause-say the one and nature as a whole
is unchangeable not only in respect of generation and destruction
(for this is a primitive belief, and all agreed in it), but also of
all other change; and this view is peculiar to them. Of those who
said the universe was one, then none succeeded in discovering a cause
of this sort, except perhaps Parmenides, and he only inasmuch as he
supposes that there is not only one but also in some sense two causes.
But for those who make more elements it is more possible to state
the second cause, e.g. for those who make hot and cold, or fire and
earth, the elements; for they treat fire as having a nature which
fits it to move things, and water and earth and such things they treat
in the contrary way. 

"When these men and the principles of this kind had had their day,
as the latter were found inadequate to generate the nature of things
men were again forced by the truth itself, as we said, to inquire
into the next kind of cause. For it is not likely either that fire
or earth or any such element should be the reason why things manifest
goodness and, beauty both in their being and in their coming to be,
or that those thinkers should have supposed it was; nor again could
it be right to entrust so great a matter to spontaneity and chance.
When one man said, then, that reason was present-as in animals, so
throughout nature-as the cause of order and of all arrangement, he
seemed like a sober man in contrast with the random talk of his predecessors.
We know that Anaxagoras certainly adopted these views, but Hermotimus
of Clazomenae is credited with expressing them earlier. Those who
thought thus stated that there is a principle of things which is at
the same time the cause of beauty, and that sort of cause from which
things acquire movement. 

Part 4 "

"One might suspect that Hesiod was the first to look for such a thing-or
some one else who put love or desire among existing things as a principle,
as Parmenides, too, does; for he, in constructing the genesis of the
universe, says:- "

"Love first of all the Gods she planned. "

"And Hesiod says:- "

"First of all things was chaos made, and then 

"Broad-breasted earth... 

"And love, 'mid all the gods pre-eminent, "

which implies that among existing things there must be from the first
a cause which will move things and bring them together. How these
thinkers should be arranged with regard to priority of discovery let
us be allowed to decide later; but since the contraries of the various
forms of good were also perceived to be present in nature-not only
order and the beautiful, but also disorder and the ugly, and bad things
in greater number than good, and ignoble things than beautiful-therefore
another thinker introduced friendship and strife, each of the two
the cause of one of these two sets of qualities. For if we were to
follow out the view of Empedocles, and interpret it according to its
meaning and not to its lisping expression, we should find that friendship
is the cause of good things, and strife of bad. Therefore, if we said
that Empedocles in a sense both mentions, and is the first to mention,
the bad and the good as principles, we should perhaps be right, since
the cause of all goods is the good itself. 

"These thinkers, as we say, evidently grasped, and to this extent,
two of the causes which we distinguished in our work on nature-the
matter and the source of the movement-vaguely, however, and with no
clearness, but as untrained men behave in fights; for they go round
their opponents and often strike fine blows, but they do not fight
on scientific principles, and so too these thinkers do not seem to
know what they say; for it is evident that, as a rule, they make no
use of their causes except to a small extent. For Anaxagoras uses
reason as a deus ex machina for the making of the world, and when
he is at a loss to tell from what cause something necessarily is,
then he drags reason in, but in all other cases ascribes events to
anything rather than to reason. And Empedocles, though he uses the
causes to a greater extent than this, neither does so sufficiently
nor attains consistency in their use. At least, in many cases he makes
love segregate things, and strife aggregate them. For whenever the
universe is dissolved into its elements by strife, fire is aggregated
into one, and so is each of the other elements; but whenever again
under the influence of love they come together into one, the parts
must again be segregated out of each element. 

"Empedocles, then, in contrast with his precessors, was the first
to introduce the dividing of this cause, not positing one source of
movement, but different and contrary sources. Again, he was the first
to speak of four material elements; yet he does not use four, but
treats them as two only; he treats fire by itself, and its opposite-earth,
air, and water-as one kind of thing. We may learn this by study of
his verses. 

"This philosopher then, as we say, has spoken of the principles in
this way, and made them of this number. Leucippus and his associate
Democritus say that the full and the empty are the elements, calling
the one being and the other non-being-the full and solid being being,
the empty non-being (whence they say being no more is than non-being,
because the solid no more is than the empty); and they make these
the material causes of things. And as those who make the underlying
substance one generate all other things by its modifications, supposing
the rare and the dense to be the sources of the modifications, in
the same way these philosophers say the differences in the elements
are the causes of all other qualities. These differences, they say,
are three-shape and order and position. For they say the real is differentiated
only by 'rhythm and 'inter-contact' and 'turning'; and of these rhythm
is shape, inter-contact is order, and turning is position; for A differs
from N in shape, AN from NA in order, M from W in position. The question
of movement-whence or how it is to belong to things-these thinkers,
like the others, lazily neglected. 

"Regarding the two causes, then, as we say, the inquiry seems to have
been pushed thus far by the early philosophers. 

Part 5 "

"Contemporaneously with these philosophers and before them, the so-called
Pythagoreans, who were the first to take up mathematics, not only
advanced this study, but also having been brought up in it they thought
its principles were the principles of all things. Since of these principles
numbers are by nature the first, and in numbers they seemed to see
many resemblances to the things that exist and come into being-more
than in fire and earth and water (such and such a modification of
numbers being justice, another being soul and reason, another being
opportunity-and similarly almost all other things being numerically
expressible); since, again, they saw that the modifications and the
ratios of the musical scales were expressible in numbers;-since, then,
all other things seemed in their whole nature to be modelled on numbers,
and numbers seemed to be the first things in the whole of nature,
they supposed the elements of numbers to be the elements of all things,
and the whole heaven to be a musical scale and a number. And all the
properties of numbers and scales which they could show to agree with
the attributes and parts and the whole arrangement of the heavens,
they collected and fitted into their scheme; and if there was a gap
anywhere, they readily made additions so as to make their whole theory
coherent. E.g. as the number 10 is thought to be perfect and to comprise
the whole nature of numbers, they say that the bodies which move through
the heavens are ten, but as the visible bodies are only nine, to meet
this they invent a tenth--the 'counter-earth'. We have discussed these
matters more exactly elsewhere. 

"But the object of our review is that we may learn from these philosophers
also what they suppose to be the principles and how these fall under
the causes we have named. Evidently, then, these thinkers also consider
that number is the principle both as matter for things and as forming
both their modifications and their permanent states, and hold that
the elements of number are the even and the odd, and that of these
the latter is limited, and the former unlimited; and that the One
proceeds from both of these (for it is both even and odd), and number
from the One; and that the whole heaven, as has been said, is numbers.

"Other members of this same school say there are ten principles, which
they arrange in two columns of cognates-limit and unlimited, odd and
even, one and plurality, right and left, male and female, resting
and moving, straight and curved, light and darkness, good and bad,
square and oblong. In this way Alcmaeon of Croton seems also to have
conceived the matter, and either he got this view from them or they
got it from him; for he expressed himself similarly to them. For he
says most human affairs go in pairs, meaning not definite contrarieties
such as the Pythagoreans speak of, but any chance contrarieties, e.g.
white and black, sweet and bitter, good and bad, great and small.
He threw out indefinite suggestions about the other contrarieties,
but the Pythagoreans declared both how many and which their contraricties
are. 

"From both these schools, then, we can learn this much, that the contraries
are the principles of things; and how many these principles are and
which they are, we can learn from one of the two schools. But how
these principles can be brought together under the causes we have
named has not been clearly and articulately stated by them; they seem,
however, to range the elements under the head of matter; for out of
these as immanent parts they say substance is composed and moulded.

"From these facts we may sufficiently perceive the meaning of the
ancients who said the elements of nature were more than one; but there
are some who spoke of the universe as if it were one entity, though
they were not all alike either in the excellence of their statement
or in its conformity to the facts of nature. The discussion of them
is in no way appropriate to our present investigation of causes, for.
they do not, like some of the natural philosophers, assume being to
be one and yet generate it out of the one as out of matter, but they
speak in another way; those others add change, since they generate
the universe, but these thinkers say the universe is unchangeable.
Yet this much is germane to the present inquiry: Parmenides seems
to fasten on that which is one in definition, Melissus on that which
is one in matter, for which reason the former says that it is limited,
the latter that it is unlimited; while Xenophanes, the first of these
partisans of the One (for Parmenides is said to have been his pupil),
gave no clear statement, nor does he seem to have grasped the nature
of either of these causes, but with reference to the whole material
universe he says the One is God. Now these thinkers, as we said, must
be neglected for the purposes of the present inquiry-two of them entirely,
as being a little too naive, viz. Xenophanes and Melissus; but Parmenides
seems in places to speak with more insight. For, claiming that, besides
the existent, nothing non-existent exists, he thinks that of necessity
one thing exists, viz. the existent and nothing else (on this we have
spoken more clearly in our work on nature), but being forced to follow
the observed facts, and supposing the existence of that which is one
in definition, but more than one according to our sensations, he now
posits two causes and two principles, calling them hot and cold, i.e.
fire and earth; and of these he ranges the hot with the existent,
and the other with the non-existent. 

"From what has been said, then, and from the wise men who have now
sat in council with us, we have got thus much-on the one hand from
the earliest philosophers, who regard the first principle as corporeal
(for water and fire and such things are bodies), and of whom some
suppose that there is one corporeal principle, others that there are
more than one, but both put these under the head of matter; and on
the other hand from some who posit both this cause and besides this
the source of movement, which we have got from some as single and
from others as twofold. 

"Down to the Italian school, then, and apart from it, philosophers
have treated these subjects rather obscurely, except that, as we said,
they have in fact used two kinds of cause, and one of these-the source
of movement-some treat as one and others as two. But the Pythagoreans
have said in the same way that there are two principles, but added
this much, which is peculiar to them, that they thought that finitude
and infinity were not attributes of certain other things, e.g. of
fire or earth or anything else of this kind, but that infinity itself
and unity itself were the substance of the things of which they are
predicated. This is why number was the substance of all things. On
this subject, then, they expressed themselves thus; and regarding
the question of essence they began to make statements and definitions,
but treated the matter too simply. For they both defined superficially
and thought that the first subject of which a given definition was
predicable was the substance of the thing defined, as if one supposed
that 'double' and '2' were the same, because 2 is the first thing
of which 'double' is predicable. But surely to be double and to be
2 are not the same; if they are, one thing will be many-a consequence
which they actually drew. From the earlier philosophers, then, and
from their successors we can learn thus much. 

Part 6 "

"After the systems we have named came the philosophy of Plato, which
in most respects followed these thinkers, but had pecullarities that
distinguished it from the philosophy of the Italians. For, having
in his youth first become familiar with Cratylus and with the Heraclitean
doctrines (that all sensible things are ever in a state of flux and
there is no knowledge about them), these views he held even in later
years. Socrates, however, was busying himself about ethical matters
and neglecting the world of nature as a whole but seeking the universal
in these ethical matters, and fixed thought for the first time on
definitions; Plato accepted his teaching, but held that the problem
applied not to sensible things but to entities of another kind-for
this reason, that the common definition could not be a definition
of any sensible thing, as they were always changing. Things of this
other sort, then, he called Ideas, and sensible things, he said, were
all named after these, and in virtue of a relation to these; for the
many existed by participation in the Ideas that have the same name
as they. Only the name 'participation' was new; for the Pythagoreans
say that things exist by 'imitation' of numbers, and Plato says they
exist by participation, changing the name. But what the participation
or the imitation of the Forms could be they left an open question.

"Further, besides sensible things and Forms he says there are the
objects of mathematics, which occupy an intermediate position, differing
from sensible things in being eternal and unchangeable, from Forms
in that there are many alike, while the Form itself is in each case
unique. 

"Since the Forms were the causes of all other things, he thought their
elements were the elements of all things. As matter, the great and
the small were principles; as essential reality, the One; for from
the great and the small, by participation in the One, come the Numbers.

"But he agreed with the Pythagoreans in saying that the One is substance
and not a predicate of something else; and in saying that the Numbers
are the causes of the reality of other things he agreed with them;
but positing a dyad and constructing the infinite out of great and
small, instead of treating the infinite as one, is peculiar to him;
and so is his view that the Numbers exist apart from sensible things,
while they say that the things themselves are Numbers, and do not
place the objects of mathematics between Forms and sensible things.
His divergence from the Pythagoreans in making the One and the Numbers
separate from things, and his introduction of the Forms, were due
to his inquiries in the region of definitions (for the earlier thinkers
had no tincture of dialectic), and his making the other entity besides
the One a dyad was due to the belief that the numbers, except those
which were prime, could be neatly produced out of the dyad as out
of some plastic material. Yet what happens is the contrary; the theory
is not a reasonable one. For they make many things out of the matter,
and the form generates only once, but what we observe is that one
table is made from one matter, while the man who applies the form,
though he is one, makes many tables. And the relation of the male
to the female is similar; for the latter is impregnated by one copulation,
but the male impregnates many females; yet these are analogues of
those first principles. 

"Plato, then, declared himself thus on the points in question; it
is evident from what has been said that he has used only two causes,
that of the essence and the material cause (for the Forms are the
causes of the essence of all other things, and the One is the cause
of the essence of the Forms); and it is evident what the underlying
matter is, of which the Forms are predicated in the case of sensible
things, and the One in the case of Forms, viz. that this is a dyad,
the great and the small. Further, he has assigned the cause of good
and that of evil to the elements, one to each of the two, as we say
some of his predecessors sought to do, e.g. Empedocles and Anaxagoras.

Part 7 "

"Our review of those who have spoken about first principles and reality
and of the way in which they have spoken, has been concise and summary;
but yet we have learnt this much from them, that of those who speak
about 'principle' and 'cause' no one has mentioned any principle except
those which have been distinguished in our work on nature, but all
evidently have some inkling of them, though only vaguely. For some
speak of the first principle as matter, whether they suppose one or
more first principles, and whether they suppose this to be a body
or to be incorporeal; e.g. Plato spoke of the great and the small,
the Italians of the infinite, Empedocles of fire, earth, water, and
air, Anaxagoras of the infinity of things composed of similar parts.
These, then, have all had a notion of this kind of cause, and so have
all who speak of air or fire or water, or something denser than fire
and rarer than air; for some have said the prime element is of this
kind. 

"These thinkers grasped this cause only; but certain others have mentioned
the source of movement, e.g. those who make friendship and strife,
or reason, or love, a principle. 

"The essence, i.e. the substantial reality, no one has expressed distinctly.
It is hinted at chiefly by those who believe in the Forms; for they
do not suppose either that the Forms are the matter of sensible things,
and the One the matter of the Forms, or that they are the source of
movement (for they say these are causes rather of immobility and of
being at rest), but they furnish the Forms as the essence of every
other thing, and the One as the essence of the Forms. 

"That for whose sake actions and changes and movements take place,
they assert to be a cause in a way, but not in this way, i.e. not
in the way in which it is its nature to be a cause. For those who
speak of reason or friendship class these causes as goods; they do
not speak, however, as if anything that exists either existed or came
into being for the sake of these, but as if movements started from
these. In the same way those who say the One or the existent is the
good, say that it is the cause of substance, but not that substance
either is or comes to be for the sake of this. Therefore it turns
out that in a sense they both say and do not say the good is a cause;
for they do not call it a cause qua good but only incidentally.

"All these thinkers then, as they cannot pitch on another cause, seem
to testify that we have determined rightly both how many and of what
sort the causes are. Besides this it is plain that when the causes
are being looked for, either all four must be sought thus or they
must be sought in one of these four ways. Let us next discuss the
possible difficulties with regard to the way in which each of these
thinkers has spoken, and with regard to his situation relatively to
the first principles. 

Part 8 "

"Those, then, who say the universe is one and posit one kind of thing
as matter, and as corporeal matter which has spatial magnitude, evidently
go astray in many ways. For they posit the elements of bodies only,
not of incorporeal things, though there are also incorporeal things.
And in trying to state the causes of generation and destruction, and
in giving a physical account of all things, they do away with the
cause of movement. Further, they err in not positing the substance,
i.e. the essence, as the cause of anything, and besides this in lightly
calling any of the simple bodies except earth the first principle,
without inquiring how they are produced out of one anothers-I mean
fire, water, earth, and air. For some things are produced out of each
other by combination, others by separation, and this makes the greatest
difference to their priority and posteriority. For (1) in a way the
property of being most elementary of all would seem to belong to the
first thing from which they are produced by combination, and this
property would belong to the most fine-grained and subtle of bodies.
For this reason those who make fire the principle would be most in
agreement with this argument. But each of the other thinkers agrees
that the element of corporeal things is of this sort. At least none
of those who named one element claimed that earth was the element,
evidently because of the coarseness of its grain. (Of the other three
elements each has found some judge on its side; for some maintain
that fire, others that water, others that air is the element. Yet
why, after all, do they not name earth also, as most men do? For people
say all things are earth Hesiod says earth was produced first of corporeal
things; so primitive and popular has the opinion been.) According
to this argument, then, no one would be right who either says the
first principle is any of the elements other than fire, or supposes
it to be denser than air but rarer than water. But (2) if that which
is later in generation is prior in nature, and that which is concocted
and compounded is later in generation, the contrary of what we have
been saying must be true,-water must be prior to air, and earth to
water. 

"So much, then, for those who posit one cause such as we mentioned;
but the same is true if one supposes more of these, as Empedocles
says matter of things is four bodies. For he too is confronted by
consequences some of which are the same as have been mentioned, while
others are peculiar to him. For we see these bodies produced from
one another, which implies that the same body does not always remain
fire or earth (we have spoken about this in our works on nature);
and regarding the cause of movement and the question whether we must
posit one or two, he must be thought to have spoken neither correctly
nor altogether plausibly. And in general, change of quality is necessarily
done away with for those who speak thus, for on their view cold will
not come from hot nor hot from cold. For if it did there would be
something that accepted the contraries themselves, and there would
be some one entity that became fire and water, which Empedocles denies.

"As regards Anaxagoras, if one were to suppose that he said there
were two elements, the supposition would accord thoroughly with an
argument which Anaxagoras himself did not state articulately, but
which he must have accepted if any one had led him on to it. True,
to say that in the beginning all things were mixed is absurd both
on other grounds and because it follows that they must have existed
before in an unmixed form, and because nature does not allow any chance
thing to be mixed with any chance thing, and also because on this
view modifications and accidents could be separated from substances
(for the same things which are mixed can be separated); yet if one
were to follow him up, piecing together what he means, he would perhaps
be seen to be somewhat modern in his views. For when nothing was separated
out, evidently nothing could be truly asserted of the substance that
then existed. I mean, e.g. that it was neither white nor black, nor
grey nor any other colour, but of necessity colourless; for if it
had been coloured, it would have had one of these colours. And similarly,
by this same argument, it was flavourless, nor had it any similar
attribute; for it could not be either of any quality or of any size,
nor could it be any definite kind of thing. For if it were, one of
the particular forms would have belonged to it, and this is impossible,
since all were mixed together; for the particular form would necessarily
have been already separated out, but he all were mixed except reason,
and this alone was unmixed and pure. From this it follows, then, that
he must say the principles are the One (for this is simple and unmixed)
and the Other, which is of such a nature as we suppose the indefinite
to be before it is defined and partakes of some form. Therefore, while
expressing himself neither rightly nor clearly, he means something
like what the later thinkers say and what is now more clearly seen
to be the case. 

"But these thinkers are, after all, at home only in arguments about
generation and destruction and movement; for it is practically only
of this sort of substance that they seek the principles and the causes.
But those who extend their vision to all things that exist, and of
existing things suppose some to be perceptible and others not perceptible,
evidently study both classes, which is all the more reason why one
should devote some time to seeing what is good in their views and
what bad from the standpoint of the inquiry we have now before us.

"The 'Pythagoreans' treat of principles and elements stranger than
those of the physical philosophers (the reason is that they got the
principles from non-sensible things, for the objects of mathematics,
except those of astronomy, are of the class of things without movement);
yet their discussions and investigations are all about nature; for
they generate the heavens, and with regard to their parts and attributes
and functions they observe the phenomena, and use up the principles
and the causes in explaining these, which implies that they agree
with the others, the physical philosophers, that the real is just
all that which is perceptible and contained by the so-called 'heavens'.
But the causes and the principles which they mention are, as we said,
sufficient to act as steps even up to the higher realms of reality,
and are more suited to these than to theories about nature. They do
not tell us at all, however, how there can be movement if limit and
unlimited and odd and even are the only things assumed, or how without
movement and change there can be generation and destruction, or the
bodies that move through the heavens can do what they do.

"Further, if one either granted them that spatial magnitude consists
of these elements, or this were proved, still how would some bodies
be light and others have weight? To judge from what they assume and
maintain they are speaking no more of mathematical bodies than of
perceptible; hence they have said nothing whatever about fire or earth
or the other bodies of this sort, I suppose because they have nothing
to say which applies peculiarly to perceptible things. 

"Further, how are we to combine the beliefs that the attributes of
number, and number itself, are causes of what exists and happens in
the heavens both from the beginning and now, and that there is no
other number than this number out of which the world is composed?
When in one particular region they place opinion and opportunity,
and, a little above or below, injustice and decision or mixture, and
allege, as proof, that each of these is a number, and that there happens
to be already in this place a plurality of the extended bodies composed
of numbers, because these attributes of number attach to the various
places,-this being so, is this number, which we must suppose each
of these abstractions to be, the same number which is exhibited in
the material universe, or is it another than this? Plato says it is
different; yet even he thinks that both these bodies and their causes
are numbers, but that the intelligible numbers are causes, while the
others are sensible. 

Part 9 "

"Let us leave the Pythagoreans for the present; for it is enough to
have touched on them as much as we have done. But as for those who
posit the Ideas as causes, firstly, in seeking to grasp the causes
of the things around us, they introduced others equal in number to
these, as if a man who wanted to count things thought he would not
be able to do it while they were few, but tried to count them when
he had added to their number. For the Forms are practically equal
to-or not fewer than-the things, in trying to explain which these
thinkers proceeded from them to the Forms. For to each thing there
answers an entity which has the same name and exists apart from the
substances, and so also in the case of all other groups there is a
one over many, whether the many are in this world or are eternal.

"Further, of the ways in which we prove that the Forms exist, none
is convincing; for from some no inference necessarily follows, and
from some arise Forms even of things of which we think there are no
Forms. For according to the arguments from the existence of the sciences
there will be Forms of all things of which there are sciences and
according to the 'one over many' argument there will be Forms even
of negations, and according to the argument that there is an object
for thought even when the thing has perished, there will be Forms
of perishable things; for we have an image of these. Further, of the
more accurate arguments, some lead to Ideas of relations, of which
we say there is no independent class, and others introduce the 'third
man'. 

"And in general the arguments for the Forms destroy the things for
whose existence we are more zealous than for the existence of the
Ideas; for it follows that not the dyad but number is first, i.e.
that the relative is prior to the absolute,-besides all the other
points on which certain people by following out the opinions held
about the Ideas have come into conflict with the principles of the
theory. 

"Further, according to the assumption on which our belief in the Ideas
rests, there will be Forms not only of substances but also of many
other things (for the concept is single not only in the case of substances
but also in the other cases, and there are sciences not only of substance
but also of other things, and a thousand other such difficulties confront
them). But according to the necessities of the case and the opinions
held about the Forms, if Forms can be shared in there must be Ideas
of substances only. For they are not shared in incidentally, but a
thing must share in its Form as in something not predicated of a subject
(by 'being shared in incidentally' I mean that e.g. if a thing shares
in 'double itself', it shares also in 'eternal', but incidentally;
for 'eternal' happens to be predicable of the 'double'). Therefore
the Forms will be substance; but the same terms indicate substance
in this and in the ideal world (or what will be the meaning of saying
that there is something apart from the particulars-the one over many?).
And if the Ideas and the particulars that share in them have the same
form, there will be something common to these; for why should '2'
be one and the same in the perishable 2's or in those which are many
but eternal, and not the same in the '2' itself' as in the particular
2? But if they have not the same form, they must have only the name
in common, and it is as if one were to call both Callias and a wooden
image a 'man', without observing any community between them.

"Above all one might discuss the question what on earth the Forms
contribute to sensible things, either to those that are eternal or
to those that come into being and cease to be. For they cause neither
movement nor any change in them. But again they help in no wise either
towards the knowledge of the other things (for they are not even the
substance of these, else they would have been in them), or towards
their being, if they are not in the particulars which share in them;
though if they were, they might be thought to be causes, as white
causes whiteness in a white object by entering into its composition.
But this argument, which first Anaxagoras and later Eudoxus and certain
others used, is very easily upset; for it is not difficult to collect
many insuperable objections to such a view. 

"But, further, all other things cannot come from the Forms in any
of the usual senses of 'from'. And to say that they are patterns and
the other things share in them is to use empty words and poetical
metaphors. For what is it that works, looking to the Ideas? And anything
can either be, or become, like another without being copied from it,
so that whether Socrates or not a man Socrates like might come to
be; and evidently this might be so even if Socrates were eternal.
And there will be several patterns of the same thing, and therefore
several Forms; e.g. 'animal' and 'two-footed' and also 'man himself'
will be Forms of man. Again, the Forms are patterns not only sensible
things, but of Forms themselves also; i.e. the genus, as genus of
various species, will be so; therefore the same thing will be pattern
and copy. 

"Again, it would seem impossible that the substance and that of which
it is the substance should exist apart; how, therefore, could the
Ideas, being the substances of things, exist apart? In the Phaedo'
the case is stated in this way-that the Forms are causes both of being
and of becoming; yet when the Forms exist, still the things that share
in them do not come into being, unless there is something to originate
movement; and many other things come into being (e.g. a house or a
ring) of which we say there are no Forms. Clearly, therefore, even
the other things can both be and come into being owing to such causes
as produce the things just mentioned. 

"Again, if the Forms are numbers, how can they be causes? Is it because
existing things are other numbers, e.g. one number is man, another
is Socrates, another Callias? Why then are the one set of numbers
causes of the other set? It will not make any difference even if the
former are eternal and the latter are not. But if it is because things
in this sensible world (e.g. harmony) are ratios of numbers, evidently
the things between which they are ratios are some one class of things.
If, then, this--the matter--is some definite thing, evidently the
numbers themselves too will be ratios of something to something else.
E.g. if Callias is a numerical ratio between fire and earth and water
and air, his Idea also will be a number of certain other underlying
things; and man himself, whether it is a number in a sense or not,
will still be a numerical ratio of certain things and not a number
proper, nor will it be a of number merely because it is a numerical
ratio. 

"Again, from many numbers one number is produced, but how can one
Form come from many Forms? And if the number comes not from the many
numbers themselves but from the units in them, e.g. in 10,000, how
is it with the units? If they are specifically alike, numerous absurdities
will follow, and also if they are not alike (neither the units in
one number being themselves like one another nor those in other numbers
being all like to all); for in what will they differ, as they are
without quality? This is not a plausible view, nor is it consistent
with our thought on the matter. 

"Further, they must set up a second kind of number (with which arithmetic
deals), and all the objects which are called 'intermediate' by some
thinkers; and how do these exist or from what principles do they proceed?
Or why must they be intermediate between the things in this sensible
world and the things-themselves? 

"Further, the units in must each come from a prior but this is impossible.

"Further, why is a number, when taken all together, one?

"Again, besides what has been said, if the units are diverse the Platonists
should have spoken like those who say there are four, or two, elements;
for each of these thinkers gives the name of element not to that which
is common, e.g. to body, but to fire and earth, whether there is something
common to them, viz. body, or not. But in fact the Platonists speak
as if the One were homogeneous like fire or water; and if this is
so, the numbers will not be substances. Evidently, if there is a One
itself and this is a first principle, 'one' is being used in more
than one sense; for otherwise the theory is impossible. 

"When we wish to reduce substances to their principles, we state that
lines come from the short and long (i.e. from a kind of small and
great), and the plane from the broad and narrow, and body from the
deep and shallow. Yet how then can either the plane contain a line,
or the solid a line or a plane? For the broad and narrow is a different
class from the deep and shallow. Therefore, just as number is not
present in these, because the many and few are different from these,
evidently no other of the higher classes will be present in the lower.
But again the broad is not a genus which includes the deep, for then
the solid would have been a species of plane. Further, from what principle
will the presence of the points in the line be derived? Plato even
used to object to this class of things as being a geometrical fiction.
He gave the name of principle of the line-and this he often posited-to
the indivisible lines. Yet these must have a limit; therefore the
argument from which the existence of the line follows proves also
the existence of the point. 

"In general, though philosophy seeks the cause of perceptible things,
we have given this up (for we say nothing of the cause from which
change takes its start), but while we fancy we are stating the substance
of perceptible things, we assert the existence of a second class of
substances, while our account of the way in which they are the substances
of perceptible things is empty talk; for 'sharing', as we said before,
means nothing. 

"Nor have the Forms any connexion with what we see to be the cause
in the case of the arts, that for whose sake both all mind and the
whole of nature are operative,-with this cause which we assert to
be one of the first principles; but mathematics has come to be identical
with philosophy for modern thinkers, though they say that it should
be studied for the sake of other things. Further, one might suppose
that the substance which according to them underlies as matter is
too mathematical, and is a predicate and differentia of the substance,
ie. of the matter, rather than matter itself; i.e. the great and the
small are like the rare and the dense which the physical philosophers
speak of, calling these the primary differentiae of the substratum;
for these are a kind of excess and defect. And regarding movement,
if the great and the small are to he movement, evidently the Forms
will be moved; but if they are not to be movement, whence did movement
come? The whole study of nature has been annihilated. 

"And what is thought to be easy-to show that all things are one-is
not done; for what is proved by the method of setting out instances
is not that all things are one but that there is a One itself,-if
we grant all the assumptions. And not even this follows, if we do
not grant that the universal is a genus; and this in some cases it
cannot be. 

"Nor can it be explained either how the lines and planes and solids
that come after the numbers exist or can exist, or what significance
they have; for these can neither be Forms (for they are not numbers),
nor the intermediates (for those are the objects of mathematics),
nor the perishable things. This is evidently a distinct fourth class.

"In general, if we search for the elements of existing things without
distinguishing the many senses in which things are said to exist,
we cannot find them, especially if the search for the elements of
which things are made is conducted in this manner. For it is surely
impossible to discover what 'acting' or 'being acted on', or 'the
straight', is made of, but if elements can be discovered at all, it
is only the elements of substances; therefore either to seek the elements
of all existing things or to think one has them is incorrect.

"And how could we learn the elements of all things? Evidently we cannot
start by knowing anything before. For as he who is learning geometry,
though he may know other things before, knows none of the things with
which the science deals and about which he is to learn, so is it in
all other cases. Therefore if there is a science of all things, such
as some assert to exist, he who is learning this will know nothing
before. Yet all learning is by means of premisses which are (either
all or some of them) known before,-whether the learning be by demonstration
or by definitions; for the elements of the definition must be known
before and be familiar; and learning by induction proceeds similarly.
But again, if the science were actually innate, it were strange that
we are unaware of our possession of the greatest of sciences.

"Again, how is one to come to know what all things are made of, and
how is this to be made evident? This also affords a difficulty; for
there might be a conflict of opinion, as there is about certain syllables;
some say za is made out of s and d and a, while others say it is a
distinct sound and none of those that are familiar. 

"Further, how could we know the objects of sense without having the
sense in question? Yet we ought to, if the elements of which all things
consist, as complex sounds consist of the clements proper to sound,
are the same. 

Part 10 "

"It is evident, then, even from what we have said before, that all
men seem to seek the causes named in the Physics, and that we cannot
name any beyond these; but they seek these vaguely; and though in
a sense they have all been described before, in a sense they have
not been described at all. For the earliest philosophy is, on all
subjects, like one who lisps, since it is young and in its beginnings.
For even Empedocles says bone exists by virtue of the ratio in it.
Now this is the essence and the substance of the thing. But it is
similarly necessary that flesh and each of the other tissues should
be the ratio of its elements, or that not one of them should; for
it is on account of this that both flesh and bone and everything else
will exist, and not on account of the matter, which he names,-fire
and earth and water and air. But while he would necessarily have agreed
if another had said this, he has not said it clearly. 

"On these questions our views have been expressed before; but let
us return to enumerate the difficulties that might be raised on these
same points; for perhaps we may get from them some help towards our
later difficulties. 

----------------------------------------------------------------------

BOOK II

Part 1 

"

"THE investigation of the truth is in one way hard, in another easy.
An indication of this is found in the fact that no one is able to
attain the truth adequately, while, on the other hand, we do not collectively
fail, but every one says something true about the nature of things,
and while individually we contribute little or nothing to the truth,
by the union of all a considerable amount is amassed. Therefore, since
the truth seems to be like the proverbial door, which no one can fail
to hit, in this respect it must be easy, but the fact that we can
have a whole truth and not the particular part we aim at shows the
difficulty of it. 

"Perhaps, too, as difficulties are of two kinds, the cause of the
present difficulty is not in the facts but in us. For as the eyes
of bats are to the blaze of day, so is the reason in our soul to the
things which are by nature most evident of all. 

"It is just that we should be grateful, not only to those with whose
views we may agree, but also to those who have expressed more superficial
views; for these also contributed something, by developing before
us the powers of thought. It is true that if there had been no Timotheus
we should have been without much of our lyric poetry; but if there
had been no Phrynis there would have been no Timotheus. The same holds
good of those who have expressed views about the truth; for from some
thinkers we have inherited certain opinions, while the others have
been responsible for the appearance of the former. 

"It is right also that philosophy should be called knowledge of the
truth. For the end of theoretical knowledge is truth, while that of
practical knowledge is action (for even if they consider how things
are, practical men do not study the eternal, but what is relative
and in the present). Now we do not know a truth without its cause;
and a thing has a quality in a higher degree than other things if
in virtue of it the similar quality belongs to the other things as
well (e.g. fire is the hottest of things; for it is the cause of the
heat of all other things); so that that causes derivative truths to
be true is most true. Hence the principles of eternal things must
be always most true (for they are not merely sometimes true, nor is
there any cause of their being, but they themselves are the cause
of the being of other things), so that as each thing is in respect
of being, so is it in respect of truth. 

Part 2 "

"But evidently there is a first principle, and the causes of things
are neither an infinite series nor infinitely various in kind. For
neither can one thing proceed from another, as from matter, ad infinitum
(e.g. flesh from earth, earth from air, air from fire, and so on without
stopping), nor can the sources of movement form an endless series
(man for instance being acted on by air, air by the sun, the sun by
Strife, and so on without limit). Similarly the final causes cannot
go on ad infinitum,-walking being for the sake of health, this for
the sake of happiness, happiness for the sake of something else, and
so one thing always for the sake of another. And the case of the essence
is similar. For in the case of intermediates, which have a last term
and a term prior to them, the prior must be the cause of the later
terms. For if we had to say which of the three is the cause, we should
say the first; surely not the last, for the final term is the cause
of none; nor even the intermediate, for it is the cause only of one.
(It makes no difference whether there is one intermediate or more,
nor whether they are infinite or finite in number.) But of series
which are infinite in this way, and of the infinite in general, all
the parts down to that now present are alike intermediates; so that
if there is no first there is no cause at all. 

"Nor can there be an infinite process downwards, with a beginning
in the upward direction, so that water should proceed from fire, earth
from water, and so always some other kind should be produced. For
one thing comes from another in two ways-not in the sense in which
'from' means 'after' (as we say 'from the Isthmian games come the
Olympian'), but either (i) as the man comes from the boy, by the boy's
changing, or (ii) as air comes from water. By 'as the man comes from
the boy' we mean 'as that which has come to be from that which is
coming to be' or 'as that which is finished from that which is being
achieved' (for as becoming is between being and not being, so that
which is becoming is always between that which is and that which is
not; for the learner is a man of science in the making, and this is
what is meant when we say that from a learner a man of science is
being made); on the other hand, coming from another thing as water
comes from air implies the destruction of the other thing. This is
why changes of the former kind are not reversible, and the boy does
not come from the man (for it is not that which comes to be something
that comes to be as a result of coming to be, but that which exists
after the coming to be; for it is thus that the day, too, comes from
the morning-in the sense that it comes after the morning; which is
the reason why the morning cannot come from the day); but changes
of the other kind are reversible. But in both cases it is impossible
that the number of terms should be infinite. For terms of the former
kind, being intermediates, must have an end, and terms of the latter
kind change back into one another, for the destruction of either is
the generation of the other. 

"At the same time it is impossible that the first cause, being eternal,
should be destroyed; for since the process of becoming is not infinite
in the upward direction, that which is the first thing by whose destruction
something came to be must be non-eternal. 

"Further, the final cause is an end, and that sort of end which is
not for the sake of something else, but for whose sake everything
else is; so that if there is to be a last term of this sort, the process
will not be infinite; but if there is no such term, there will be
no final cause, but those who maintain the infinite series eliminate
the Good without knowing it (yet no one would try to do anything if
he were not going to come to a limit); nor would there be reason in
the world; the reasonable man, at least, always acts for a purpose,
and this is a limit; for the end is a limit. 

"But the essence, also, cannot be reduced to another definition which
is fuller in expression. For the original definition is always more
of a definition, and not the later one; and in a series in which the
first term has not the required character, the next has not it either.
Further, those who speak thus destroy science; for it is not possible
to have this till one comes to the unanalysable terms. And knowledge
becomes impossible; for how can one apprehend things that are infinite
in this way? For this is not like the case of the line, to whose divisibility
there is no stop, but which we cannot think if we do not make a stop
(for which reason one who is tracing the infinitely divisible line
cannot be counting the possibilities of section), but the whole line
also must be apprehended by something in us that does not move from
part to part.-Again, nothing infinite can exist; and if it could,
at least the notion of infinity is not infinite. 

"But if the kinds of causes had been infinite in number, then also
knowledge would have been impossible; for we think we know, only when
we have ascertained the causes, that but that which is infinite by
addition cannot be gone through in a finite time. 

Part 3 "

"The effect which lectures produce on a hearer depends on his habits;
for we demand the language we are accustomed to, and that which is
different from this seems not in keeping but somewhat unintelligible
and foreign because of its unwontedness. For it is the customary that
is intelligible. The force of habit is shown by the laws, in which
the legendary and childish elements prevail over our knowledge about
them, owing to habit. Thus some people do not listen to a speaker
unless he speaks mathematically, others unless he gives instances,
while others expect him to cite a poet as witness. And some want to
have everything done accurately, while others are annoyed by accuracy,
either because they cannot follow the connexion of thought or because
they regard it as pettifoggery. For accuracy has something of this
character, so that as in trade so in argument some people think it
mean. Hence one must be already trained to know how to take each sort
of argument, since it is absurd to seek at the same time knowledge
and the way of attaining knowledge; and it is not easy to get even
one of the two. 

"The minute accuracy of mathematics is not to be demanded in all cases,
but only in the case of things which have no matter. Hence method
is not that of natural science; for presumably the whole of nature
has matter. Hence we must inquire first what nature is: for thus we
shall also see what natural science treats of (and whether it belongs
to one science or to more to investigate the causes and the principles
of things). 

----------------------------------------------------------------------

BOOK III

Part 1 

"

"WE must, with a view to the science which we are seeking, first recount
the subjects that should be first discussed. These include both the
other opinions that some have held on the first principles, and any
point besides these that happens to have been overlooked. For those
who wish to get clear of difficulties it is advantageous to discuss
the difficulties well; for the subsequent free play of thought implies
the solution of the previous difficulties, and it is not possible
to untie a knot of which one does not know. But the difficulty of
our thinking points to a 'knot' in the object; for in so far as our
thought is in difficulties, it is in like case with those who are
bound; for in either case it is impossible to go forward. Hence one
should have surveyed all the difficulties beforehand, both for the
purposes we have stated and because people who inquire without first
stating the difficulties are like those who do not know where they
have to go; besides, a man does not otherwise know even whether he
has at any given time found what he is looking for or not; for the
end is not clear to such a man, while to him who has first discussed
the difficulties it is clear. Further, he who has heard all the contending
arguments, as if they were the parties to a case, must be in a better
position for judging. 

"The first problem concerns the subject which we discussed in our
prefatory remarks. It is this-(1) whether the investigation of the
causes belongs to one or to more sciences, and (2) whether such a
science should survey only the first principles of substance, or also
the principles on which all men base their proofs, e.g. whether it
is possible at the same time to assert and deny one and the same thing
or not, and all other such questions; and (3) if the science in question
deals with substance, whether one science deals with all substances,
or more than one, and if more, whether all are akin, or some of them
must be called forms of Wisdom and the others something else. And
(4) this itself is also one of the things that must be discussed-whether
sensible substances alone should be said to exist or others also besides
them, and whether these others are of one kind or there are several
classes of substances, as is supposed by those who believe both in
Forms and in mathematical objects intermediate between these and sensible
things. Into these questions, then, as we say, we must inquire, and
also (5) whether our investigation is concerned only with substances
or also with the essential attributes of substances. Further, with
regard to the same and other and like and unlike and contrariety,
and with regard to prior and posterior and all other such terms about
which the dialecticians try to inquire, starting their investigation
from probable premises only,-whose business is it to inquire into
all these? Further, we must discuss the essential attributes of these
themselves; and we must ask not only what each of these is, but also
whether one thing always has one contrary. Again (6), are the principles
and elements of things the genera, or the parts present in each thing,
into which it is divided; and (7) if they are the genera, are they
the genera that are predicated proximately of the individuals, or
the highest genera, e.g. is animal or man the first principle and
the more independent of the individual instance? And (8) we must inquire
and discuss especially whether there is, besides the matter, any thing
that is a cause in itself or not, and whether this can exist apart
or not, and whether it is one or more in number, and whether there
is something apart from the concrete thing (by the concrete thing
I mean the matter with something already predicated of it), or there
is nothing apart, or there is something in some cases though not in
others, and what sort of cases these are. Again (9) we ask whether
the principles are limited in number or in kind, both those in the
definitions and those in the substratum; and (10) whether the principles
of perishable and of imperishable things are the same or different;
and whether they are all imperishable or those of perishable things
are perishable. Further (11) there is the question which is hardest
of all and most perplexing, whether unity and being, as the Pythagoreans
and Plato said, are not attributes of something else but the substance
of existing things, or this is not the case, but the substratum is
something else,-as Empedocles says, love; as some one else says, fire;
while another says water or air. Again (12) we ask whether the principles
are universal or like individual things, and (13) whether they exist
potentially or actually, and further, whether they are potential or
actual in any other sense than in reference to movement; for these
questions also would present much difficulty. Further (14), are numbers
and lines and figures and points a kind of substance or not, and if
they are substances are they separate from sensible things or present
in them? With regard to all these matters not only is it hard to get
possession of the truth, but it is not easy even to think out the
difficulties well. 

Part 2 "

"(1) First then with regard to what we mentioned first, does it belong
to one or to more sciences to investigate all the kinds of causes?
How could it belong to one science to recognize the principles if
these are not contrary? 

"Further, there are many things to which not all the principles pertain.
For how can a principle of change or the nature of the good exist
for unchangeable things, since everything that in itself and by its
own nature is good is an end, and a cause in the sense that for its
sake the other things both come to be and are, and since an end or
purpose is the end of some action, and all actions imply change? So
in the case of unchangeable things this principle could not exist,
nor could there be a good itself. This is why in mathematics nothing
is proved by means of this kind of cause, nor is there any demonstration
of this kind-'because it is better, or worse'; indeed no one even
mentions anything of the kind. And so for this reason some of the
Sophists, e.g. Aristippus, used to ridicule mathematics; for in the
arts (he maintained), even in the industrial arts, e.g. in carpentry
and cobbling, the reason always given is 'because it is better, or
worse,' but the mathematical sciences take no account of goods and
evils. 

"But if there are several sciences of the causes, and a different
science for each different principle, which of these sciences should
be said to be that which we seek, or which of the people who possess
them has the most scientific knowledge of the object in question?
The same thing may have all the kinds of causes, e.g. the moving cause
of a house is the art or the builder, the final cause is the function
it fulfils, the matter is earth and stones, and the form is the definition.
To judge from our previous discussion of the question which of the
sciences should be called Wisdom, there is reason for applying the
name to each of them. For inasmuch as it is most architectonic and
authoritative and the other sciences, like slavewomen, may not even
contradict it, the science of the end and of the good is of the nature
of Wisdom (for the other things are for the sake of the end). But
inasmuch as it was described' as dealing with the first causes and
that which is in the highest sense object of knowledge, the science
of substance must be of the nature of Wisdom. For since men may know
the same thing in many ways, we say that he who recognizes what a
thing is by its being so and so knows more fully than he who recognizes
it by its not being so and so, and in the former class itself one
knows more fully than another, and he knows most fully who knows what
a thing is, not he who knows its quantity or quality or what it can
by nature do or have done to it. And further in all cases also we
think that the knowledge of each even of the things of which demonstration
is possible is present only when we know what the thing is, e.g. what
squaring a rectangle is, viz. that it is the finding of a mean; and
similarly in all other cases. And we know about becomings and actions
and about every change when we know the source of the movement; and
this is other than and opposed to the end. Therefore it would seem
to belong to different sciences to investigate these causes severally.

"But (2), taking the starting-points of demonstration as well as the
causes, it is a disputable question whether they are the object of
one science or of more (by the starting-points of demonstration I
mean the common beliefs, on which all men base their proofs); e.g.
that everything must be either affirmed or denied, and that a thing
cannot at the same time be and not be, and all other such premisses:-the
question is whether the same science deals with them as with substance,
or a different science, and if it is not one science, which of the
two must be identified with that which we now seek.-It is not reasonable
that these topics should be the object of one science; for why should
it be peculiarly appropriate to geometry or to any other science to
understand these matters? If then it belongs to every science alike,
and cannot belong to all, it is not peculiar to the science which
investigates substances, any more than to any other science, to know
about these topics.-And, at the same time, in what way can there be
a science of the first principles? For we are aware even now what
each of them in fact is (at least even other sciences use them as
familiar); but if there is a demonstrative science which deals with
them, there will have to be an underlying kind, and some of them must
be demonstrable attributes and others must be axioms (for it is impossible
that there should be demonstration about all of them); for the demonstration
must start from certain premisses and be about a certain subject and
prove certain attributes. Therefore it follows that all attributes
that are proved must belong to a single class; for all demonstrative
sciences use the axioms. 

"But if the science of substance and the science which deals with
the axioms are different, which of them is by nature more authoritative
and prior? The axioms are most universal and are principles of all
things. And if it is not the business of the philosopher, to whom
else will it belong to inquire what is true and what is untrue about
them? 

"(3) In general, do all substances fall under one science or under
more than one? If the latter, to what sort of substance is the present
science to be assigned?-On the other hand, it is not reasonable that
one science should deal with all. For then there would be one demonstrative
science dealing with all attributes. For ever demonstrative science
investigates with regard to some subject its essential attributes,
starting from the common beliefs. Therefore to investigate the essential
attributes of one class of things, starting from one set of beliefs,
is the business of one science. For the subject belongs to one science,
and the premisses belong to one, whether to the same or to another;
so that the attributes do so too, whether they are investigated by
these sciences or by one compounded out of them. 

"(5) Further, does our investigation deal with substances alone or
also with their attributes? I mean for instance, if the solid is a
substance and so are lines and planes, is it the business of the same
science to know these and to know the attributes of each of these
classes (the attributes about which the mathematical sciences offer
proofs), or of a different science? If of the same, the science of
substance also must be a demonstrative science, but it is thought
that there is no demonstration of the essence of things. And if of
another, what will be the science that investigates the attributes
of substance? This is a very difficult question. 

"(4) Further, must we say that sensible substances alone exist, or
that there are others besides these? And are substances of one kind
or are there in fact several kinds of substances, as those say who
assert the existence both of the Forms and of the intermediates, with
which they say the mathematical sciences deal?-The sense in which
we say the Forms are both causes and self-dependent substances has
been explained in our first remarks about them; while the theory presents
difficulties in many ways, the most paradoxical thing of all is the
statement that there are certain things besides those in the material
universe, and that these are the same as sensible things except that
they are eternal while the latter are perishable. For they say there
is a man-himself and a horse-itself and health-itself, with no further
qualification,-a procedure like that of the people who said there
are gods, but in human form. For they were positing nothing but eternal
men, nor are the Platonists making the Forms anything other than eternal
sensible things. 

"Further, if we are to posit besides the Forms and the sensibles the
intermediates between them, we shall have many difficulties. For clearly
on the same principle there will be lines besides the lines-themselves
and the sensible lines, and so with each of the other classes of things;
so that since astronomy is one of these mathematical sciences there
will also be a heaven besides the sensible heaven, and a sun and a
moon (and so with the other heavenly bodies) besides the sensible.
Yet how are we to believe in these things? It is not reasonable even
to suppose such a body immovable, but to suppose it moving is quite
impossible.-And similarly with the things of which optics and mathematical
harmonics treat; for these also cannot exist apart from the sensible
things, for the same reasons. For if there are sensible things and
sensations intermediate between Form and individual, evidently there
will also be animals intermediate between animals-themselves and the
perishable animals.-We might also raise the question, with reference
to which kind of existing things we must look for these sciences of
intermediates. If geometry is to differ from mensuration only in this,
that the latter deals with things that we perceive, and the former
with things that are not perceptible, evidently there will also be
a science other than medicine, intermediate between medical-science-itself
and this individual medical science, and so with each of the other
sciences. Yet how is this possible? There would have to be also healthy
things besides the perceptible healthy things and the healthy-itself.--And
at the same time not even this is true, that mensuration deals with
perceptible and perishable magnitudes; for then it would have perished
when they perished. 

"But on the other hand astronomy cannot be dealing with perceptible
magnitudes nor with this heaven above us. For neither are perceptible
lines such lines as the geometer speaks of (for no perceptible thing
is straight or round in the way in which he defines 'straight' and
'round'; for a hoop touches a straight edge not at a point, but as
Protagoras used to say it did, in his refutation of the geometers),
nor are the movements and spiral orbits in the heavens like those
of which astronomy treats, nor have geometrical points the same nature
as the actual stars.-Now there are some who say that these so-called
intermediates between the Forms and the perceptible things exist,
not apart from the perceptible things, however, but in these; the
impossible results of this view would take too long to enumerate,
but it is enough to consider even such points as the following:-It
is not reasonable that this should be so only in the case of these
intermediates, but clearly the Forms also might be in the perceptible
things; for both statements are parts of the same theory. Further,
it follows from this theory that there are two solids in the same
place, and that the intermediates are not immovable, since they are
in the moving perceptible things. And in general to what purpose would
one suppose them to exist indeed, but to exist in perceptible things?
For the same paradoxical results will follow which we have already
mentioned; there will be a heaven besides the heaven, only it will
be not apart but in the same place; which is still more impossible.

Part 3 "

"(6) Apart from the great difficulty of stating the case truly with
regard to these matters, it is very hard to say, with regard to the
first principles, whether it is the genera that should be taken as
elements and principles, or rather the primary constituents of a thing;
e.g. it is the primary parts of which articulate sounds consist that
are thought to be elements and principles of articulate sound, not
the common genus-articulate sound; and we give the name of 'elements'
to those geometrical propositions, the proofs of which are implied
in the proofs of the others, either of all or of most. Further, both
those who say there are several elements of corporeal things and those
who say there is one, say the parts of which bodies are compounded
and consist are principles; e.g. Empedocles says fire and water and
the rest are the constituent elements of things, but does not describe
these as genera of existing things. Besides this, if we want to examine
the nature of anything else, we examine the parts of which, e.g. a
bed consists and how they are put together, and then we know its nature.

"To judge from these arguments, then, the principles of things would
not be the genera; but if we know each thing by its definition, and
the genera are the principles or starting-points of definitions, the
genera must also be the principles of definable things. And if to
get the knowledge of the species according to which things are named
is to get the knowledge of things, the genera are at least starting-points
of the species. And some also of those who say unity or being, or
the great and the small, are elements of things, seem to treat them
as genera. 

"But, again, it is not possible to describe the principles in both
ways. For the formula of the essence is one; but definition by genera
will be different from that which states the constituent parts of
a thing. 

"(7) Besides this, even if the genera are in the highest degree principles,
should one regard the first of the genera as principles, or those
which are predicated directly of the individuals? This also admits
of dispute. For if the universals are always more of the nature of
principles, evidently the uppermost of the genera are the principles;
for these are predicated of all things. There will, then, be as many
principles of things as there are primary genera, so that both being
and unity will be principles and substances; for these are most of
all predicated of all existing things. But it is not possible that
either unity or being should be a single genus of things; for the
differentiae of any genus must each of them both have being and be
one, but it is not possible for the genus taken apart from its species
(any more than for the species of the genus) to be predicated of its
proper differentiae; so that if unity or being is a genus, no differentia
will either have being or be one. But if unity and being are not genera,
neither will they be principles, if the genera are the principles.
Again, the intermediate kinds, in whose nature the differentiae are
included, will on this theory be genera, down to the indivisible species;
but as it is, some are thought to be genera and others are not thought
to be so. Besides this, the differentiae are principles even more
than the genera; and if these also are principles, there comes to
be practically an infinite number of principles, especially if we
suppose the highest genus to be a principle.-But again, if unity is
more of the nature of a principle, and the indivisible is one, and
everything indivisible is so either in quantity or in species, and
that which is so in species is the prior, and genera are divisible
into species for man is not the genus of individual men), that which
is predicated directly of the individuals will have more unity.-Further,
in the case of things in which the distinction of prior and posterior
is present, that which is predicable of these things cannot be something
apart from them (e.g. if two is the first of numbers, there will not
be a Number apart from the kinds of numbers; and similarly there will
not be a Figure apart from the kinds of figures; and if the genera
of these things do not exist apart from the species, the genera of
other things will scarcely do so; for genera of these things are thought
to exist if any do). But among the individuals one is not prior and
another posterior. Further, where one thing is better and another
worse, the better is always prior; so that of these also no genus
can exist. From these considerations, then, the species predicated
of individuals seem to be principles rather than the genera. But again,
it is not easy to say in what sense these are to be taken as principles.
For the principle or cause must exist alongside of the things of which
it is the principle, and must be capable of existing in separation
from them; but for what reason should we suppose any such thing to
exist alongside of the individual, except that it is predicated universally
and of all? But if this is the reason, the things that are more universal
must be supposed to be more of the nature of principles; so that the
highest genera would be the principles. 

Part 4 "

"(8) There is a difficulty connected with these, the hardest of all
and the most necessary to examine, and of this the discussion now
awaits us. If, on the one hand, there is nothing apart from individual
things, and the individuals are infinite in number, how then is it
possible to get knowledge of the infinite individuals? For all things
that we come to know, we come to know in so far as they have some
unity and identity, and in so far as some attribute belongs to them
universally. 

"But if this is necessary, and there must be something apart from
the individuals, it will be necessary that the genera exist apart
from the individuals, either the lowest or the highest genera; but
we found by discussion just now that this is impossible.

"Further, if we admit in the fullest sense that something exists apart
from the concrete thing, whenever something is predicated of the matter,
must there, if there is something apart, be something apart from each
set of individuals, or from some and not from others, or from none?
(A) If there is nothing apart from individuals, there will be no object
of thought, but all things will be objects of sense, and there will
not be knowledge of anything, unless we say that sensation is knowledge.
Further, nothing will be eternal or unmovable; for all perceptible
things perish and are in movement. But if there is nothing eternal,
neither can there be a process of coming to be; for there must be
something that comes to be, i.e. from which something comes to be,
and the ultimate term in this series cannot have come to be, since
the series has a limit and since nothing can come to be out of that
which is not. Further, if generation and movement exist there must
also be a limit; for no movement is infinite, but every movement has
an end, and that which is incapable of completing its coming to be
cannot be in process of coming to be; and that which has completed
its coming to be must he as soon as it has come to be. Further, since
the matter exists, because it is ungenerated, it is a fortiori reasonable
that the substance or essence, that which the matter is at any time
coming to be, should exist; for if neither essence nor matter is to
be, nothing will be at all, and since this is impossible there must
be something besides the concrete thing, viz. the shape or form.

"But again (B) if we are to suppose this, it is hard to say in which
cases we are to suppose it and in which not. For evidently it is not
possible to suppose it in all cases; we could not suppose that there
is a house besides the particular houses.-Besides this, will the substance
of all the individuals, e.g. of all men, be one? This is paradoxical,
for all the things whose substance is one are one. But are the substances
many and different? This also is unreasonable.-At the same time, how
does the matter become each of the individuals, and how is the concrete
thing these two elements? 

"(9) Again, one might ask the following question also about the first
principles. If they are one in kind only, nothing will be numerically
one, not even unity-itself and being-itself; and how will knowing
exist, if there is not to be something common to a whole set of individuals?

"But if there is a common element which is numerically one, and each
of the principles is one, and the principles are not as in the case
of perceptible things different for different things (e.g. since this
particular syllable is the same in kind whenever it occurs, the elements
it are also the same in kind; only in kind, for these also, like the
syllable, are numerically different in different contexts),-if it
is not like this but the principles of things are numerically one,
there will be nothing else besides the elements (for there is no difference
of meaning between 'numerically one' and 'individual'; for this is
just what we mean by the individual-the numerically one, and by the
universal we mean that which is predicable of the individuals). Therefore
it will be just as if the elements of articulate sound were limited
in number; all the language in the world would be confined to the
ABC, since there could not be two or more letters of the same kind.

"(10) One difficulty which is as great as any has been neglected both
by modern philosophers and by their predecessors-whether the principles
of perishable and those of imperishable things are the same or different.
If they are the same, how are some things perishable and others imperishable,
and for what reason? The school of Hesiod and all the theologians
thought only of what was plausible to themselves, and had no regard
to us. For, asserting the first principles to be gods and born of
gods, they say that the beings which did not taste of nectar and ambrosia
became mortal; and clearly they are using words which are familiar
to themselves, yet what they have said about the very application
of these causes is above our comprehension. For if the gods taste
of nectar and ambrosia for their pleasure, these are in no wise the
causes of their existence; and if they taste them to maintain their
existence, how can gods who need food be eternal?-But into the subtleties
of the mythologists it is not worth our while to inquire seriously;
those, however, who use the language of proof we must cross-examine
and ask why, after all, things which consist of the same elements
are, some of them, eternal in nature, while others perish. Since these
philosophers mention no cause, and it is unreasonable that things
should be as they say, evidently the principles or causes of things
cannot be the same. Even the man whom one might suppose to speak most
consistently-Empedocles, even he has made the same mistake; for he
maintains that strife is a principle that causes destruction, but
even strife would seem no less to produce everything, except the One;
for all things excepting God proceed from strife. At least he says:-
"

"From which all that was and is and will be hereafter- 

"Trees, and men and women, took their growth, 

"And beasts and birds and water-nourished fish, 

"And long-aged gods. "

"The implication is evident even apart from these words; for if strife
had not been present in things, all things would have been one, according
to him; for when they have come together, 'then strife stood outermost.'
Hence it also follows on his theory that God most blessed is less
wise than all others; for he does not know all the elements; for he
has in him no strife, and knowledge is of the like by the like. 'For
by earth,' he says, "

"we see earth, by water water, 

"By ether godlike ether, by fire wasting fire, 

"Love by love, and strife by gloomy strife. "

But-and this is the point we started from this at least is evident,
that on his theory it follows that strife is as much the cause of
existence as of destruction. And similarly love is not specially the
cause of existence; for in collecting things into the One it destroys
all other things. And at the same time Empedocles mentions no cause
of the change itself, except that things are so by nature.

"But when strife at last waxed great in the limbs of the

"Sphere, 

"And sprang to assert its rights as the time was fulfilled

"Which is fixed for them in turn by a mighty oath. "

"This implies that change was necessary; but he shows no cause of
the necessity. But yet so far at least he alone speaks consistently;
for he does not make some things perishable and others imperishable,
but makes all perishable except the elements. The difficulty we are
speaking of now is, why some things are perishable and others are
not, if they consist of the same principles. 

"Let this suffice as proof of the fact that the principles cannot
be the same. But if there are different principles, one difficulty
is whether these also will be imperishable or perishable. For if they
are perishable, evidently these also must consist of certain elements
(for all things that perish, perish by being resolved into the elements
of which they consist); so that it follows that prior to the principles
there are other principles. But this is impossible, whether the process
has a limit or proceeds to infinity. Further, how will perishable
things exist, if their principles are to be annulled? But if the principles
are imperishable, why will things composed of some imperishable principles
be perishable, while those composed of the others are imperishable?
This is not probable, but is either impossible or needs much proof.
Further, no one has even tried to maintain different principles; they
maintain the same principles for all things. But they swallow the
difficulty we stated first as if they took it to be something trifling.

"(11) The inquiry that is both the hardest of all and the most necessary
for knowledge of the truth is whether being and unity are the substances
of things, and whether each of them, without being anything else,
is being or unity respectively, or we must inquire what being and
unity are, with the implication that they have some other underlying
nature. For some people think they are of the former, others think
they are of the latter character. Plato and the Pythagoreans thought
being and unity were nothing else, but this was their nature, their
essence being just unity and being. But the natural philosophers take
a different line; e.g. Empedocles-as though reducing to something
more intelligible-says what unity is; for he would seem to say it
is love: at least, this is for all things the cause of their being
one. Others say this unity and being, of which things consist and
have been made, is fire, and others say it is air. A similar view
is expressed by those who make the elements more than one; for these
also must say that unity and being are precisely all the things which
they say are principles. 

"(A) If we do not suppose unity and being to be substances, it follows
that none of the other universals is a substance; for these are most
universal of all, and if there is no unity itself or being-itself,
there will scarcely be in any other case anything apart from what
are called the individuals. Further, if unity is not a substance,
evidently number also will not exist as an entity separate from the
individual things; for number is units, and the unit is precisely
a certain kind of one. 

"But (B) if there is a unity-itself and a being itself, unity and
being must be their substance; for it is not something else that is
predicated universally of the things that are and are one, but just
unity and being. But if there is to be a being-itself and a unity-itself,
there is much difficulty in seeing how there will be anything else
besides these,-I mean, how things will be more than one in number.
For what is different from being does not exist, so that it necessarily
follows, according to the argument of Parmenides, that all things
that are are one and this is being. 

"There are objections to both views. For whether unity is not a substance
or there is a unity-itself, number cannot be a substance. We have
already said why this result follows if unity is not a substance;
and if it is, the same difficulty arises as arose with regard to being.
For whence is there to be another one besides unity-itself? It must
be not-one; but all things are either one or many, and of the many
each is one. 

"Further, if unity-itself is indivisible, according to Zeno's postulate
it will be nothing. For that which neither when added makes a thing
greater nor when subtracted makes it less, he asserts to have no being,
evidently assuming that whatever has being is a spatial magnitude.
And if it is a magnitude, it is corporeal; for the corporeal has being
in every dimension, while the other objects of mathematics, e.g. a
plane or a line, added in one way will increase what they are added
to, but in another way will not do so, and a point or a unit does
so in no way. But, since his theory is of a low order, and an indivisible
thing can exist in such a way as to have a defence even against him
(for the indivisible when added will make the number, though not the
size, greater),-yet how can a magnitude proceed from one such indivisible
or from many? It is like saying that the line is made out of points.

"But even if ore supposes the case to be such that, as some say, number
proceeds from unity-itself and something else which is not one, none
the less we must inquire why and how the product will be sometimes
a number and sometimes a magnitude, if the not-one was inequality
and was the same principle in either case. For it is not evident how
magnitudes could proceed either from the one and this principle, or
from some number and this principle. 

Part 5 "

"(14) A question connected with these is whether numbers and bodies
and planes and points are substances of a kind, or not. If they are
not, it baffles us to say what being is and what the substances of
things are. For modifications and movements and relations and dispositions
and ratios do not seem to indicate the substance of anything; for
all are predicated of a subject, and none is a 'this'. And as to the
things which might seem most of all to indicate substance, water and
earth and fire and air, of which composite bodies consist, heat and
cold and the like are modifications of these, not substances, and
the body which is thus modified alone persists as something real and
as a substance. But, on the other hand, the body is surely less of
a substance than the surface, and the surface than the line, and the
line than the unit and the point. For the body is bounded by these;
and they are thought to be capable of existing without body, but body
incapable of existing without these. This is why, while most of the
philosophers and the earlier among them thought that substance and
being were identical with body, and that all other things were modifications
of this, so that the first principles of the bodies were the first
principles of being, the more recent and those who were held to be
wiser thought numbers were the first principles. As we said, then,
if these are not substance, there is no substance and no being at
all; for the accidents of these it cannot be right to call beings.

"But if this is admitted, that lines and points are substance more
than bodies, but we do not see to what sort of bodies these could
belong (for they cannot be in perceptible bodies), there can be no
substance.-Further, these are all evidently divisions of body,-one
in breadth, another in depth, another in length. Besides this, no
sort of shape is present in the solid more than any other; so that
if the Hermes is not in the stone, neither is the half of the cube
in the cube as something determinate; therefore the surface is not
in it either; for if any sort of surface were in it, the surface which
marks off the half of the cube would be in it too. And the same account
applies to the line and to the point and the unit. Therefore, if on
the one hand body is in the highest degree substance, and on the other
hand these things are so more than body, but these are not even instances
of substance, it baffles us to say what being is and what the substance
of things is.-For besides what has been said, the questions of generation
and instruction confront us with further paradoxes. For if substance,
not having existed before, now exists, or having existed before, afterwards
does not exist, this change is thought to be accompanied by a process
of becoming or perishing; but points and lines and surfaces cannot
be in process either of becoming or of perishing, when they at one
time exist and at another do not. For when bodies come into contact
or are divided, their boundaries simultaneously become one in the
one case when they touch, and two in the other-when they are divided;
so that when they have been put together one boundary does not exist
but has perished, and when they have been divided the boundaries exist
which before did not exist (for it cannot be said that the point,
which is indivisible, was divided into two). And if the boundaries
come into being and cease to be, from what do they come into being?
A similar account may also be given of the 'now' in time; for this
also cannot be in process of coming into being or of ceasing to be,
but yet seems to be always different, which shows that it is not a
substance. And evidently the same is true of points and lines and
planes; for the same argument applies, since they are all alike either
limits or divisions. 

Part 6 "

"In general one might raise the question why after all, besides perceptible
things and the intermediates, we have to look for another class of
things, i.e. the Forms which we posit. If it is for this reason, because
the objects of mathematics, while they differ from the things in this
world in some other respect, differ not at all in that there are many
of the same kind, so that their first principles cannot be limited
in number (just as the elements of all the language in this sensible
world are not limited in number, but in kind, unless one takes the
elements of this individual syllable or of this individual articulate
sound-whose elements will be limited even in number; so is it also
in the case of the intermediates; for there also the members of the
same kind are infinite in number), so that if there are not-besides
perceptible and mathematical objects-others such as some maintain
the Forms to be, there will be no substance which is one in number,
but only in kind, nor will the first principles of things be determinate
in number, but only in kind:-if then this must be so, the Forms also
must therefore be held to exist. Even if those who support this view
do not express it articulately, still this is what they mean, and
they must be maintaining the Forms just because each of the Forms
is a substance and none is by accident. 

"But if we are to suppose both that the Forms exist and that the principles
are one in number, not in kind, we have mentioned the impossible results
that necessarily follow. 

"(13) Closely connected with this is the question whether the elements
exist potentially or in some other manner. If in some other way, there
will be something else prior to the first principles; for the potency
is prior to the actual cause, and it is not necessary for everything
potential to be actual.-But if the elements exist potentially, it
is possible that everything that is should not be. For even that which
is not yet is capable of being; for that which is not comes to be,
but nothing that is incapable of being comes to be. 

"(12) We must not only raise these questions about the first principles,
but also ask whether they are universal or what we call individuals.
If they are universal, they will not be substances; for everything
that is common indicates not a 'this' but a 'such', but substance
is a 'this'. And if we are to be allowed to lay it down that a common
predicate is a 'this' and a single thing, Socrates will be several
animals-himself and 'man' and 'animal', if each of these indicates
a 'this' and a single thing. 

"If, then, the principles are universals, these universal. Therefore
if there is to be results follow; if they are not universals but of
knowledge of the principles there must be the nature of individuals,
they will not be other principles prior to them, namely those knowable;
for the knowledge of anything is that are universally predicated of
them. 

----------------------------------------------------------------------

BOOK IV

Part 1 

"

"THERE is a science which investigates being as being and the attributes
which belong to this in virtue of its own nature. Now this is not
the same as any of the so-called special sciences; for none of these
others treats universally of being as being. They cut off a part of
being and investigate the attribute of this part; this is what the
mathematical sciences for instance do. Now since we are seeking the
first principles and the highest causes, clearly there must be some
thing to which these belong in virtue of its own nature. If then those
who sought the elements of existing things were seeking these same
principles, it is necessary that the elements must be elements of
being not by accident but just because it is being. Therefore it is
of being as being that we also must grasp the first causes.
"

Part 2 

"There are many senses in which a thing may be said to 'be', but all
that 'is' is related to one central point, one definite kind of thing,
and is not said to 'be' by a mere ambiguity. Everything which is healthy
is related to health, one thing in the sense that it preserves health,
another in the sense that it produces it, another in the sense that
it is a symptom of health, another because it is capable of it. And
that which is medical is relative to the medical art, one thing being
called medical because it possesses it, another because it is naturally
adapted to it, another because it is a function of the medical art.
And we shall find other words used similarly to these. So, too, there
are many senses in which a thing is said to be, but all refer to one
starting-point; some things are said to be because they are substances,
others because they are affections of substance, others because they
are a process towards substance, or destructions or privations or
qualities of substance, or productive or generative of substance,
or of things which are relative to substance, or negations of one
of these thing of substance itself. It is for this reason that we
say even of non-being that it is nonbeing. As, then, there is one
science which deals with all healthy things, the same applies in the
other cases also. For not only in the case of things which have one
common notion does the investigation belong to one science, but also
in the case of things which are related to one common nature; for
even these in a sense have one common notion. It is clear then that
it is the work of one science also to study the things that are, qua
being.-But everywhere science deals chiefly with that which is primary,
and on which the other things depend, and in virtue of which they
get their names. If, then, this is substance, it will be of substances
that the philosopher must grasp the principles and the causes.

"Now for each one class of things, as there is one perception, so
there is one science, as for instance grammar, being one science,
investigates all articulate sounds. Hence to investigate all the species
of being qua being is the work of a science which is generically one,
and to investigate the several species is the work of the specific
parts of the science. 

"If, now, being and unity are the same and are one thing in the sense
that they are implied in one another as principle and cause are, not
in the sense that they are explained by the same definition (though
it makes no difference even if we suppose them to be like that-in
fact this would even strengthen our case); for 'one man' and 'man'
are the same thing, and so are 'existent man' and 'man', and the doubling
of the words in 'one man and one existent man' does not express anything
different (it is clear that the two things are not separated either
in coming to be or in ceasing to be); and similarly 'one existent
man' adds nothing to 'existent man', and that it is obvious that the
addition in these cases means the same thing, and unity is nothing
apart from being; and if, further, the substance of each thing is
one in no merely accidental way, and similarly is from its very nature
something that is:-all this being so, there must be exactly as many
species of being as of unity. And to investigate the essence of these
is the work of a science which is generically one-I mean, for instance,
the discussion of the same and the similar and the other concepts
of this sort; and nearly all contraries may be referred to this origin;
let us take them as having been investigated in the 'Selection of
Contraries'. 

"And there are as many parts of philosophy as there are kinds of substance,
so that there must necessarily be among them a first philosophy and
one which follows this. For being falls immediately into genera; for
which reason the sciences too will correspond to these genera. For
the philosopher is like the mathematician, as that word is used; for
mathematics also has parts, and there is a first and a second science
and other successive ones within the sphere of mathematics.

"Now since it is the work of one science to investigate opposites,
and plurality is opposed to unity-and it belongs to one science to
investigate the negation and the privation because in both cases we
are really investigating the one thing of which the negation or the
privation is a negation or privation (for we either say simply that
that thing is not present, or that it is not present in some particular
class; in the latter case difference is present over and above what
is implied in negation; for negation means just the absence of the
thing in question, while in privation there is also employed an underlying
nature of which the privation is asserted):-in view of all these facts,
the contraries of the concepts we named above, the other and the dissimilar
and the unequal, and everything else which is derived either from
these or from plurality and unity, must fall within the province of
the science above named. And contrariety is one of these concepts;
for contrariety is a kind of difference, and difference is a kind
of otherness. Therefore, since there are many senses in which a thing
is said to be one, these terms also will have many senses, but yet
it belongs to one science to know them all; for a term belongs to
different sciences not if it has different senses, but if it has not
one meaning and its definitions cannot be referred to one central
meaning. And since all things are referred to that which is primary,
as for instance all things which are called one are referred to the
primary one, we must say that this holds good also of the same and
the other and of contraries in general; so that after distinguishing
the various senses of each, we must then explain by reference to what
is primary in the case of each of the predicates in question, saying
how they are related to it; for some will be called what they are
called because they possess it, others because they produce it, and
others in other such ways. 

"It is evident, then, that it belongs to one science to be able to
give an account of these concepts as well as of substance (this was
one of the questions in our book of problems), and that it is the
function of the philosopher to be able to investigate all things.
For if it is not the function of the philosopher, who is it who will
inquire whether Socrates and Socrates seated are the same thing, or
whether one thing has one contrary, or what contrariety is, or how
many meanings it has? And similarly with all other such questions.
Since, then, these are essential modifications of unity qua unity
and of being qua being, not qua numbers or lines or fire, it is clear
that it belongs to this science to investigate both the essence of
these concepts and their properties. And those who study these properties
err not by leaving the sphere of philosophy, but by forgetting that
substance, of which they have no correct idea, is prior to these other
things. For number qua number has peculiar attributes, such as oddness
and evenness, commensurability and equality, excess and defect, and
these belong to numbers either in themselves or in relation to one
another. And similarly the solid and the motionless and that which
is in motion and the weightless and that which has weight have other
peculiar properties. So too there are certain properties peculiar
to being as such, and it is about these that the philosopher has to
investigate the truth.-An indication of this may be mentioned: dialecticians
and sophists assume the same guise as the philosopher, for sophistic
is Wisdom which exists only in semblance, and dialecticians embrace
all things in their dialectic, and being is common to all things;
but evidently their dialectic embraces these subjects because these
are proper to philosophy.-For sophistic and dialectic turn on the
same class of things as philosophy, but this differs from dialectic
in the nature of the faculty required and from sophistic in respect
of the purpose of the philosophic life. Dialectic is merely critical
where philosophy claims to know, and sophistic is what appears to
be philosophy but is not. 

"Again, in the list of contraries one of the two columns is privative,
and all contraries are reducible to being and non-being, and to unity
and plurality, as for instance rest belongs to unity and movement
to plurality. And nearly all thinkers agree that being and substance
are composed of contraries; at least all name contraries as their
first principles-some name odd and even, some hot and cold, some limit
and the unlimited, some love and strife. And all the others as well
are evidently reducible to unity and plurality (this reduction we
must take for granted), and the principles stated by other thinkers
fall entirely under these as their genera. It is obvious then from
these considerations too that it belongs to one science to examine
being qua being. For all things are either contraries or composed
of contraries, and unity and plurality are the starting-points of
all contraries. And these belong to one science, whether they have
or have not one single meaning. Probably the truth is that they have
not; yet even if 'one' has several meanings, the other meanings will
be related to the primary meaning (and similarly in the case of the
contraries), even if being or unity is not a universal and the same
in every instance or is not separable from the particular instances
(as in fact it probably is not; the unity is in some cases that of
common reference, in some cases that of serial succession). And for
this reason it does not belong to the geometer to inquire what is
contrariety or completeness or unity or being or the same or the other,
but only to presuppose these concepts and reason from this starting-point.--Obviously
then it is the work of one science to examine being qua being, and
the attributes which belong to it qua being, and the same science
will examine not only substances but also their attributes, both those
above named and the concepts 'prior' and 'posterior', 'genus' and
'species', 'whole' and 'part', and the others of this sort.

Part 3 "

"We must state whether it belongs to one or to different sciences
to inquire into the truths which are in mathematics called axioms,
and into substance. Evidently, the inquiry into these also belongs
to one science, and that the science of the philosopher; for these
truths hold good for everything that is, and not for some special
genus apart from others. And all men use them, because they are true
of being qua being and each genus has being. But men use them just
so far as to satisfy their purposes; that is, as far as the genus
to which their demonstrations refer extends. Therefore since these
truths clearly hold good for all things qua being (for this is what
is common to them), to him who studies being qua being belongs the
inquiry into these as well. And for this reason no one who is conducting
a special inquiry tries to say anything about their truth or falsity,-neither
the geometer nor the arithmetician. Some natural philosophers indeed
have done so, and their procedure was intelligible enough; for they
thought that they alone were inquiring about the whole of nature and
about being. But since there is one kind of thinker who is above even
the natural philosopher (for nature is only one particular genus of
being), the discussion of these truths also will belong to him whose
inquiry is universal and deals with primary substance. Physics also
is a kind of Wisdom, but it is not the first kind.-And the attempts
of some of those who discuss the terms on which truth should be accepted,
are due to a want of training in logic; for they should know these
things already when they come to a special study, and not be inquiring
into them while they are listening to lectures on it. 

"Evidently then it belongs to the philosopher, i.e. to him who is
studying the nature of all substance, to inquire also into the principles
of syllogism. But he who knows best about each genus must be able
to state the most certain principles of his subject, so that he whose
subject is existing things qua existing must be able to state the
most certain principles of all things. This is the philosopher, and
the most certain principle of all is that regarding which it is impossible
to be mistaken; for such a principle must be both the best known (for
all men may be mistaken about things which they do not know), and
non-hypothetical. For a principle which every one must have who understands
anything that is, is not a hypothesis; and that which every one must
know who knows anything, he must already have when he comes to a special
study. Evidently then such a principle is the most certain of all;
which principle this is, let us proceed to say. It is, that the same
attribute cannot at the same time belong and not belong to the same
subject and in the same respect; we must presuppose, to guard against
dialectical objections, any further qualifications which might be
added. This, then, is the most certain of all principles, since it
answers to the definition given above. For it is impossible for any
one to believe the same thing to be and not to be, as some think Heraclitus
says. For what a man says, he does not necessarily believe; and if
it is impossible that contrary attributes should belong at the same
time to the same subject (the usual qualifications must be presupposed
in this premiss too), and if an opinion which contradicts another
is contrary to it, obviously it is impossible for the same man at
the same time to believe the same thing to be and not to be; for if
a man were mistaken on this point he would have contrary opinions
at the same time. It is for this reason that all who are carrying
out a demonstration reduce it to this as an ultimate belief; for this
is naturally the starting-point even for all the other axioms.

Part 4 "

"There are some who, as we said, both themselves assert that it is
possible for the same thing to be and not to be, and say that people
can judge this to be the case. And among others many writers about
nature use this language. But we have now posited that it is impossible
for anything at the same time to be and not to be, and by this means
have shown that this is the most indisputable of all principles.-Some
indeed demand that even this shall be demonstrated, but this they
do through want of education, for not to know of what things one should
demand demonstration, and of what one should not, argues want of education.
For it is impossible that there should be demonstration of absolutely
everything (there would be an infinite regress, so that there would
still be no demonstration); but if there are things of which one should
not demand demonstration, these persons could not say what principle
they maintain to be more self-evident than the present one.

"We can, however, demonstrate negatively even that this view is impossible,
if our opponent will only say something; and if he says nothing, it
is absurd to seek to give an account of our views to one who cannot
give an account of anything, in so far as he cannot do so. For such
a man, as such, is from the start no better than a vegetable. Now
negative demonstration I distinguish from demonstration proper, because
in a demonstration one might be thought to be begging the question,
but if another person is responsible for the assumption we shall have
negative proof, not demonstration. The starting-point for all such
arguments is not the demand that our opponent shall say that something
either is or is not (for this one might perhaps take to be a begging
of the question), but that he shall say something which is significant
both for himself and for another; for this is necessary, if he really
is to say anything. For, if he means nothing, such a man will not
be capable of reasoning, either with himself or with another. But
if any one grants this, demonstration will be possible; for we shall
already have something definite. The person responsible for the proof,
however, is not he who demonstrates but he who listens; for while
disowning reason he listens to reason. And again he who admits this
has admitted that something is true apart from demonstration (so that
not everything will be 'so and not so'). 

"First then this at least is obviously true, that the word 'be' or
'not be' has a definite meaning, so that not everything will be 'so
and not so'. Again, if 'man' has one meaning, let this be 'two-footed
animal'; by having one meaning I understand this:-if 'man' means 'X',
then if A is a man 'X' will be what 'being a man' means for him. (It
makes no difference even if one were to say a word has several meanings,
if only they are limited in number; for to each definition there might
be assigned a different word. For instance, we might say that 'man'
has not one meaning but several, one of which would have one definition,
viz. 'two-footed animal', while there might be also several other
definitions if only they were limited in number; for a peculiar name
might be assigned to each of the definitions. If, however, they were
not limited but one were to say that the word has an infinite number
of meanings, obviously reasoning would be impossible; for not to have
one meaning is to have no meaning, and if words have no meaning our
reasoning with one another, and indeed with ourselves, has been annihilated;
for it is impossible to think of anything if we do not think of one
thing; but if this is possible, one name might be assigned to this
thing.) 

"Let it be assumed then, as was said at the beginning, that the name
has a meaning and has one meaning; it is impossible, then, that 'being
a man' should mean precisely 'not being a man', if 'man' not only
signifies something about one subject but also has one significance
(for we do not identify 'having one significance' with 'signifying
something about one subject', since on that assumption even 'musical'
and 'white' and 'man' would have had one significance, so that all
things would have been one; for they would all have had the same significance).

"And it will not be possible to be and not to be the same thing, except
in virtue of an ambiguity, just as if one whom we call 'man', others
were to call 'not-man'; but the point in question is not this, whether
the same thing can at the same time be and not be a man in name, but
whether it can in fact. Now if 'man' and 'not-man' mean nothing different,
obviously 'not being a man' will mean nothing different from 'being
a man'; so that 'being a man' will be 'not being a man'; for they
will be one. For being one means this-being related as 'raiment' and
'dress' are, if their definition is one. And if 'being a man' and
'being a not-man' are to be one, they must mean one thing. But it
was shown earlier' that they mean different things.-Therefore, if
it is true to say of anything that it is a man, it must be a two-footed
animal (for this was what 'man' meant); and if this is necessary,
it is impossible that the same thing should not at that time be a
two-footed animal; for this is what 'being necessary' means-that it
is impossible for the thing not to be. It is, then, impossible that
it should be at the same time true to say the same thing is a man
and is not a man. 

"The same account holds good with regard to 'not being a man', for
'being a man' and 'being a not-man' mean different things, since even
'being white' and 'being a man' are different; for the former terms
are much more different so that they must a fortiori mean different
things. And if any one says that 'white' means one and the same thing
as 'man', again we shall say the same as what was said before, that
it would follow that all things are one, and not only opposites. But
if this is impossible, then what we have maintained will follow, if
our opponent will only answer our question. 

"And if, when one asks the question simply, he adds the contradictories,
he is not answering the question. For there is nothing to prevent
the same thing from being both a man and white and countless other
things: but still, if one asks whether it is or is not true to say
that this is a man, our opponent must give an answer which means one
thing, and not add that 'it is also white and large'. For, besides
other reasons, it is impossible to enumerate its accidental attributes,
which are infinite in number; let him, then, enumerate either all
or none. Similarly, therefore, even if the same thing is a thousand
times a man and a not-man, he must not, in answering the question
whether this is a man, add that it is also at the same time a not-man,
unless he is bound to add also all the other accidents, all that the
subject is or is not; and if he does this, he is not observing the
rules of argument. 

"And in general those who say this do away with substance and essence.
For they must say that all attributes are accidents, and that there
is no such thing as 'being essentially a man' or 'an animal'. For
if there is to be any such thing as 'being essentially a man' this
will not be 'being a not-man' or 'not being a man' (yet these are
negations of it); for there was one thing which it meant, and this
was the substance of something. And denoting the substance of a thing
means that the essence of the thing is nothing else. But if its being
essentially a man is to be the same as either being essentially a
not-man or essentially not being a man, then its essence will be something
else. Therefore our opponents must say that there cannot be such a
definition of anything, but that all attributes are accidental; for
this is the distinction between substance and accident-'white' is
accidental to man, because though he is white, whiteness is not his
essence. But if all statements are accidental, there will be nothing
primary about which they are made, if the accidental always implies
predication about a subject. The predication, then, must go on ad
infinitum. But this is impossible; for not even more than two terms
can be combined in accidental predication. For (1) an accident is
not an accident of an accident, unless it be because both are accidents
of the same subject. I mean, for instance, that the white is musical
and the latter is white, only because both are accidental to man.
But (2) Socrates is musical, not in this sense, that both terms are
accidental to something else. Since then some predicates are accidental
in this and some in that sense, (a) those which are accidental in
the latter sense, in which white is accidental to Socrates, cannot
form an infinite series in the upward direction; e.g. Socrates the
white has not yet another accident; for no unity can be got out of
such a sum. Nor again (b) will 'white' have another term accidental
to it, e.g. 'musical'. For this is no more accidental to that than
that is to this; and at the same time we have drawn the distinction,
that while some predicates are accidental in this sense, others are
so in the sense in which 'musical' is accidental to Socrates; and
the accident is an accident of an accident not in cases of the latter
kind, but only in cases of the other kind, so that not all terms will
be accidental. There must, then, even so be something which denotes
substance. And if this is so, it has been shown that contradictories
cannot be predicated at the same time. 

"Again, if all contradictory statements are true of the same subject
at the same time, evidently all things will be one. For the same thing
will be a trireme, a wall, and a man, if of everything it is possible
either to affirm or to deny anything (and this premiss must be accepted
by those who share the views of Protagoras). For if any one thinks
that the man is not a trireme, evidently he is not a trireme; so that
he also is a trireme, if, as they say, contradictory statements are
both true. And we thus get the doctrine of Anaxagoras, that all things
are mixed together; so that nothing really exists. They seem, then,
to be speaking of the indeterminate, and, while fancying themselves
to be speaking of being, they are speaking about non-being; for it
is that which exists potentially and not in complete reality that
is indeterminate. But they must predicate of every subject the affirmation
or the negation of every attribute. For it is absurd if of each subject
its own negation is to be predicable, while the negation of something
else which cannot be predicated of it is not to be predicable of it;
for instance, if it is true to say of a man that he is not a man,
evidently it is also true to say that he is either a trireme or not
a trireme. If, then, the affirmative can be predicated, the negative
must be predicable too; and if the affirmative is not predicable,
the negative, at least, will be more predicable than the negative
of the subject itself. If, then, even the latter negative is predicable,
the negative of 'trireme' will be also predicable; and, if this is
predicable, the affirmative will be so too. 

"Those, then, who maintain this view are driven to this conclusion,
and to the further conclusion that it is not necessary either to assert
or to deny. For if it is true that a thing is a man and a not-man,
evidently also it will be neither a man nor a not-man. For to the
two assertions there answer two negations, and if the former is treated
as a single proposition compounded out of two, the latter also is
a single proposition opposite to the former. 

"Again, either the theory is true in all cases, and a thing is both
white and not-white, and existent and non-existent, and all other
assertions and negations are similarly compatible or the theory is
true of some statements and not of others. And if not of all, the
exceptions will be contradictories of which admittedly only one is
true; but if of all, again either the negation will be true wherever
the assertion is, and the assertion true wherever the negation is,
or the negation will be true where the assertion is, but the assertion
not always true where the negation is. And (a) in the latter case
there will be something which fixedly is not, and this will be an
indisputable belief; and if non-being is something indisputable and
knowable, the opposite assertion will be more knowable. But (b) if
it is equally possible also to assert all that it is possible to deny,
one must either be saying what is true when one separates the predicates
(and says, for instance, that a thing is white, and again that it
is not-white), or not. And if (i) it is not true to apply the predicates
separately, our opponent is not saying what he professes to say, and
also nothing at all exists; but how could non-existent things speak
or walk, as he does? Also all things would on this view be one, as
has been already said, and man and God and trireme and their contradictories
will be the same. For if contradictories can be predicated alike of
each subject, one thing will in no wise differ from another; for if
it differ, this difference will be something true and peculiar to
it. And (ii) if one may with truth apply the predicates separately,
the above-mentioned result follows none the less, and, further, it
follows that all would then be right and all would be in error, and
our opponent himself confesses himself to be in error.-And at the
same time our discussion with him is evidently about nothing at all;
for he says nothing. For he says neither 'yes' nor 'no', but 'yes
and no'; and again he denies both of these and says 'neither yes nor
no'; for otherwise there would already be something definite.

"Again if when the assertion is true, the negation is false, and when
this is true, the affirmation is false, it will not be possible to
assert and deny the same thing truly at the same time. But perhaps
they might say this was the very question at issue. 

"Again, is he in error who judges either that the thing is so or that
it is not so, and is he right who judges both? If he is right, what
can they mean by saying that the nature of existing things is of this
kind? And if he is not right, but more right than he who judges in
the other way, being will already be of a definite nature, and this
will be true, and not at the same time also not true. But if all are
alike both wrong and right, one who is in this condition will not
be able either to speak or to say anything intelligible; for he says
at the same time both 'yes' and 'no.' And if he makes no judgement
but 'thinks' and 'does not think', indifferently, what difference
will there be between him and a vegetable?-Thus, then, it is in the
highest degree evident that neither any one of those who maintain
this view nor any one else is really in this position. For why does
a man walk to Megara and not stay at home, when he thinks he ought
to be walking there? Why does he not walk early some morning into
a well or over a precipice, if one happens to be in his way? Why do
we observe him guarding against this, evidently because he does not
think that falling in is alike good and not good? Evidently, then,
he judges one thing to be better and another worse. And if this is
so, he must also judge one thing to be a man and another to be not-a-man,
one thing to be sweet and another to be not-sweet. For he does not
aim at and judge all things alike, when, thinking it desirable to
drink water or to see a man, he proceeds to aim at these things; yet
he ought, if the same thing were alike a man and not-a-man. But, as
was said, there is no one who does not obviously avoid some things
and not others. Therefore, as it seems, all men make unqualified judgements,
if not about all things, still about what is better and worse. And
if this is not knowledge but opinion, they should be all the more
anxious about the truth, as a sick man should be more anxious about
his health than one who is healthy; for he who has opinions is, in
comparison with the man who knows, not in a healthy state as far as
the truth is concerned. 

"Again, however much all things may be 'so and not so', still there
is a more and a less in the nature of things; for we should not say
that two and three are equally even, nor is he who thinks four things
are five equally wrong with him who thinks they are a thousand. If
then they are not equally wrong, obviously one is less wrong and therefore
more right. If then that which has more of any quality is nearer the
norm, there must be some truth to which the more true is nearer. And
even if there is not, still there is already something better founded
and liker the truth, and we shall have got rid of the unqualified
doctrine which would prevent us from determining anything in our thought.

Part 5 "

"From the same opinion proceeds the doctrine of Protagoras, and both
doctrines must be alike true or alike untrue. For on the one hand,
if all opinions and appearances are true, all statements must be at
the same time true and false. For many men hold beliefs in which they
conflict with one another, and think those mistaken who have not the
same opinions as themselves; so that the same thing must both be and
not be. And on the other hand, if this is so, all opinions must be
true; for those who are mistaken and those who are right are opposed
to one another in their opinions; if, then, reality is such as the
view in question supposes, all will be right in their beliefs.

"Evidently, then, both doctrines proceed from the same way of thinking.
But the same method of discussion must not be used with all opponents;
for some need persuasion, and others compulsion. Those who have been
driven to this position by difficulties in their thinking can easily
be cured of their ignorance; for it is not their expressed argument
but their thought that one has to meet. But those who argue for the
sake of argument can be cured only by refuting the argument as expressed
in speech and in words. 

"Those who really feel the difficulties have been led to this opinion
by observation of the sensible world. (1) They think that contradictories
or contraries are true at the same time, because they see contraries
coming into existence out of the same thing. If, then, that which
is not cannot come to be, the thing must have existed before as both
contraries alike, as Anaxagoras says all is mixed in all, and Democritus
too; for he says the void and the full exist alike in every part,
and yet one of these is being, and the other non-being. To those,
then, whose belief rests on these grounds, we shall say that in a
sense they speak rightly and in a sense they err. For 'that which
is' has two meanings, so that in some sense a thing can come to be
out of that which is not, while in some sense it cannot, and the same
thing can at the same time be in being and not in being-but not in
the same respect. For the same thing can be potentially at the same
time two contraries, but it cannot actually. And again we shall ask
them to believe that among existing things there is also another kind
of substance to which neither movement nor destruction nor generation
at all belongs. 

"And (2) similarly some have inferred from observation of the sensible
world the truth of appearances. For they think that the truth should
not be determined by the large or small number of those who hold a
belief, and that the same thing is thought sweet by some when they
taste it, and bitter by others, so that if all were ill or all were
mad, and only two or three were well or sane, these would be thought
ill and mad, and not the others. 

"And again, they say that many of the other animals receive impressions
contrary to ours; and that even to the senses of each individual,
things do not always seem the same. Which, then, of these impressions
are true and which are false is not obvious; for the one set is no
more true than the other, but both are alike. And this is why Democritus,
at any rate, says that either there is no truth or to us at least
it is not evident. 

"And in general it is because these thinkers suppose knowledge to
be sensation, and this to be a physical alteration, that they say
that what appears to our senses must be true; for it is for these
reasons that both Empedocles and Democritus and, one may almost say,
all the others have fallen victims to opinions of this sort. For Empedocles
says that when men change their condition they change their knowledge;
"

"For wisdom increases in men according to what is before them.
"

"And elsewhere he says that:- "

"So far as their nature changed, so far to them always 

"Came changed thoughts into mind. "

"And Parmenides also expresses himself in the same way: "

"For as at each time the much-bent limbs are composed, 

"So is the mind of men; for in each and all men 

"'Tis one thing thinks-the substance of their limbs: 

"For that of which there is more is thought. "

"A saying of Anaxagoras to some of his friends is also related,-that
things would be for them such as they supposed them to be. And they
say that Homer also evidently had this opinion, because he made Hector,
when he was unconscious from the blow, lie 'thinking other thoughts',-which
implies that even those who are bereft of thought have thoughts, though
not the same thoughts. Evidently, then, if both are forms of knowledge,
the real things also are at the same time 'both so and not so'. And
it is in this direction that the consequences are most difficult.
For if those who have seen most of such truth as is possible for us
(and these are those who seek and love it most)-if these have such
opinions and express these views about the truth, is it not natural
that beginners in philosophy should lose heart? For to seek the truth
would be to follow flying game. 

"But the reason why these thinkers held this opinion is that while
they were inquiring into the truth of that which is, they thought,
'that which is' was identical with the sensible world; in this, however,
there is largely present the nature of the indeterminate-of that which
exists in the peculiar sense which we have explained; and therefore,
while they speak plausibly, they do not say what is true (for it is
fitting to put the matter so rather than as Epicharmus put it against
Xenophanes). And again, because they saw that all this world of nature
is in movement and that about that which changes no true statement
can be made, they said that of course, regarding that which everywhere
in every respect is changing, nothing could truly be affirmed. It
was this belief that blossomed into the most extreme of the views
above mentioned, that of the professed Heracliteans, such as was held
by Cratylus, who finally did not think it right to say anything but
only moved his finger, and criticized Heraclitus for saying that it
is impossible to step twice into the same river; for he thought one
could not do it even once. 

"But we shall say in answer to this argument also that while there
is some justification for their thinking that the changing, when it
is changing, does not exist, yet it is after all disputable; for that
which is losing a quality has something of that which is being lost,
and of that which is coming to be, something must already be. And
in general if a thing is perishing, will be present something that
exists; and if a thing is coming to be, there must be something from
which it comes to be and something by which it is generated, and this
process cannot go on ad infinitum.-But, leaving these arguments, let
us insist on this, that it is not the same thing to change in quantity
and in quality. Grant that in quantity a thing is not constant; still
it is in respect of its form that we know each thing.-And again, it
would be fair to criticize those who hold this view for asserting
about the whole material universe what they saw only in a minority
even of sensible things. For only that region of the sensible world
which immediately surrounds us is always in process of destruction
and generation; but this is-so to speak-not even a fraction of the
whole, so that it would have been juster to acquit this part of the
world because of the other part, than to condemn the other because
of this.-And again, obviously we shall make to them also the same
reply that we made long ago; we must show them and persuade them that
there is something whose nature is changeless. Indeed, those who say
that things at the same time are and are not, should in consequence
say that all things are at rest rather than that they are in movement;
for there is nothing into which they can change, since all attributes
belong already to all subjects. 

"Regarding the nature of truth, we must maintain that not everything
which appears is true; firstly, because even if sensation-at least
of the object peculiar to the sense in question-is not false, still
appearance is not the same as sensation.-Again, it is fair to express
surprise at our opponents' raising the question whether magnitudes
are as great, and colours are of such a nature, as they appear to
people at a distance, or as they appear to those close at hand, and
whether they are such as they appear to the healthy or to the sick,
and whether those things are heavy which appear so to the weak or
those which appear so to the strong, and those things true which appear
to the slee ing or to the waking. For obviously they do not think
these to be open questions; no one, at least, if when he is in Libya
he has fancied one night that he is in Athens, starts for the concert
hall.-And again with regard to the future, as Plato says, surely the
opinion of the physician and that of the ignorant man are not equally
weighty, for instance, on the question whether a man will get well
or not.-And again, among sensations themselves the sensation of a
foreign object and that of the appropriate object, or that of a kindred
object and that of the object of the sense in question, are not equally
authoritative, but in the case of colour sight, not taste, has the
authority, and in the case of flavour taste, not sight; each of which
senses never says at the same time of the same object that it simultaneously
is 'so and not so'.-But not even at different times does one sense
disagree about the quality, but only about that to which the quality
belongs. I mean, for instance, that the same wine might seem, if either
it or one's body changed, at one time sweet and at another time not
sweet; but at least the sweet, such as it is when it exists, has never
yet changed, but one is always right about it, and that which is to
be sweet is of necessity of such and such a nature. Yet all these
views destroy this necessity, leaving nothing to be of necessity,
as they leave no essence of anything; for the necessary cannot be
in this way and also in that, so that if anything is of necessity,
it will not be 'both so and not so'. 

"And, in general, if only the sensible exists, there would be nothing
if animate things were not; for there would be no faculty of sense.
Now the view that neither the sensible qualities nor the sensations
would exist is doubtless true (for they are affections of the perceiver),
but that the substrata which cause the sensation should not exist
even apart from sensation is impossible. For sensation is surely not
the sensation of itself, but there is something beyond the sensation,
which must be prior to the sensation; for that which moves is prior
in nature to that which is moved, and if they are correlative terms,
this is no less the case. 

Part 6 "

"There are, both among those who have these convictions and among
those who merely profess these views, some who raise a difficulty
by asking, who is to be the judge of the healthy man, and in general
who is likely to judge rightly on each class of questions. But such
inquiries are like puzzling over the question whether we are now asleep
or awake. And all such questions have the same meaning. These people
demand that a reason shall be given for everything; for they seek
a starting-point, and they seek to get this by demonstration, while
it is obvious from their actions that they have no conviction. But
their mistake is what we have stated it to be; they seek a reason
for things for which no reason can be given; for the starting-point
of demonstration is not demonstration. 

"These, then, might be easily persuaded of this truth, for it is not
difficult to grasp; but those who seek merely compulsion in argument
seek what is impossible; for they demand to be allowed to contradict
themselves-a claim which contradicts itself from the very first.-But
if not all things are relative, but some are self-existent, not everything
that appears will be true; for that which appears is apparent to some
one; so that he who says all things that appear are true, makes all
things relative. And, therefore, those who ask for an irresistible
argument, and at the same time demand to be called to account for
their views, must guard themselves by saying that the truth is not
that what appears exists, but that what appears exists for him to
whom it appears, and when, and to the sense to which, and under the
conditions under which it appears. And if they give an account of
their view, but do not give it in this way, they will soon find themselves
contradicting themselves. For it is possible that the same thing may
appear to be honey to the sight, but not to the taste, and that, since
we have two eyes, things may not appear the same to each, if their
sight is unlike. For to those who for the reasons named some time
ago say that what appears is true, and therefore that all things are
alike false and true, for things do not appear either the same to
all men or always the same to the same man, but often have contrary
appearances at the same time (for touch says there are two objects
when we cross our fingers, while sight says there is one)-to these
we shall say 'yes, but not to the same sense and in the same part
of it and under the same conditions and at the same time', so that
what appears will be with these qualifications true. But perhaps for
this reason those who argue thus not because they feel a difficulty
but for the sake of argument, should say that this is not true, but
true for this man. And as has been said before, they must make everything
relative-relative to opinion and perception, so that nothing either
has come to be or will be without some one's first thinking so. But
if things have come to be or will be, evidently not all things will
be relative to opinion.-Again, if a thing is one, it is in relation
to one thing or to a definite number of things; and if the same thing
is both half and equal, it is not to the double that the equal is
correlative. If, then, in relation to that which thinks, man and that
which is thought are the same, man will not be that which thinks,
but only that which is thought. And if each thing is to be relative
to that which thinks, that which thinks will be relative to an infinity
of specifically different things. 

"Let this, then, suffice to show (1) that the most indisputable of
all beliefs is that contradictory statements are not at the same time
true, and (2) what consequences follow from the assertion that they
are, and (3) why people do assert this. Now since it is impossible
that contradictories should be at the same time true of the same thing,
obviously contraries also cannot belong at the same time to the same
thing. For of contraries, one is a privation no less than it is a
contrary-and a privation of the essential nature; and privation is
the denial of a predicate to a determinate genus. If, then, it is
impossible to affirm and deny truly at the same time, it is also impossible
that contraries should belong to a subject at the same time, unless
both belong to it in particular relations, or one in a particular
relation and one without qualification. 

Part 7 "

"But on the other hand there cannot be an intermediate between contradictories,
but of one subject we must either affirm or deny any one predicate.
This is clear, in the first place, if we define what the true and
the false are. To say of what is that it is not, or of what is not
that it is, is false, while to say of what is that it is, and of what
is not that it is not, is true; so that he who says of anything that
it is, or that it is not, will say either what is true or what is
false; but neither what is nor what is not is said to be or not to
be.-Again, the intermediate between the contradictories will be so
either in the way in which grey is between black and white, or as
that which is neither man nor horse is between man and horse. (a)
If it were of the latter kind, it could not change into the extremes
(for change is from not-good to good, or from good to not-good), but
as a matter of fact when there is an intermediate it is always observed
to change into the extremes. For there is no change except to opposites
and to their intermediates. (b) But if it is really intermediate,
in this way too there would have to be a change to white, which was
not from not-white; but as it is, this is never seen.-Again, every
object of understanding or reason the understanding either affirms
or denies-this is obvious from the definition-whenever it says what
is true or false. When it connects in one way by assertion or negation,
it says what is true, and when it does so in another way, what is
false.-Again, there must be an intermediate between all contradictories,
if one is not arguing merely for the sake of argument; so that it
will be possible for a man to say what is neither true nor untrue,
and there will be a middle between that which is and that which is
not, so that there will also be a kind of change intermediate between
generation and destruction.-Again, in all classes in which the negation
of an attribute involves the assertion of its contrary, even in these
there will be an intermediate; for instance, in the sphere of numbers
there will be number which is neither odd nor not-odd. But this is
impossible, as is obvious from the definition.-Again, the process
will go on ad infinitum, and the number of realities will be not only
half as great again, but even greater. For again it will be possible
to deny this intermediate with reference both to its assertion and
to its negation, and this new term will be some definite thing; for
its essence is something different.-Again, when a man, on being asked
whether a thing is white, says 'no', he has denied nothing except
that it is; and its not being is a negation. 

"Some people have acquired this opinion as other paradoxical opinions
have been acquired; when men cannot refute eristical arguments, they
give in to the argument and agree that the conclusion is true. This,
then, is why some express this view; others do so because they demand
a reason for everything. And the starting-point in dealing with all
such people is definition. Now the definition rests on the necessity
of their meaning something; for the form of words of which the word
is a sign will be its definition.-While the doctrine of Heraclitus,
that all things are and are not, seems to make everything true, that
of Anaxagoras, that there is an intermediate between the terms of
a contradiction, seems to make everything false; for when things are
mixed, the mixture is neither good nor not-good, so that one cannot
say anything that is true. 

Part 8 "

"In view of these distinctions it is obvious that the one-sided theories
which some people express about all things cannot be valid-on the
one hand the theory that nothing is true (for, say they, there is
nothing to prevent every statement from being like the statement 'the
diagonal of a square is commensurate with the side'), on the other
hand the theory that everything is true. These views are practically
the same as that of Heraclitus; for he who says that all things are
true and all are false also makes each of these statements separately,
so that since they are impossible, the double statement must be impossible
too.-Again, there are obviously contradictories which cannot be at
the same time true-nor on the other hand can all statements be false;
yet this would seem more possible in the light of what has been said.-But
against all such views we must postulate, as we said above,' not that
something is or is not, but that something has a meaning, so that
we must argue from a definition, viz. by assuming what falsity or
truth means. If that which it is true to affirm is nothing other than
that which it is false to deny, it is impossible that all statements
should be false; for one side of the contradiction must be true. Again,
if it is necessary with regard to everything either to assert or to
deny it, it is impossible that both should be false; for it is one
side of the contradiction that is false.-Therefore all such views
are also exposed to the often expressed objection, that they destroy
themselves. For he who says that everything is true makes even the
statement contrary to his own true, and therefore his own not true
(for the contrary statement denies that it is true), while he who
says everything is false makes himself also false.-And if the former
person excepts the contrary statement, saying it alone is not true,
while the latter excepts his own as being not false, none the less
they are driven to postulate the truth or falsity of an infinite number
of statements; for that which says the true statement is true is true,
and this process will go on to infinity. 

"Evidently, again, those who say all things are at rest are not right,
nor are those who say all things are in movement. For if all things
are at rest, the same statements will always be true and the same
always false,-but this obviously changes; for he who makes a statement,
himself at one time was not and again will not be. And if all things
are in motion, nothing will be true; everything therefore will be
false. But it has been shown that this is impossible. Again, it must
be that which is that changes; for change is from something to something.
But again it is not the case that all things are at rest or in motion
sometimes, and nothing for ever; for there is something which always
moves the things that are in motion, and the first mover is itself
unmoved. 

----------------------------------------------------------------------

BOOK V

Part 1 

"

"'BEGINNING' means (1) that part of a thing from which one would start
first, e.g a line or a road has a beginning in either of the contrary
directions. (2) That from which each thing would best be originated,
e.g. even in learning we must sometimes begin not from the first point
and the beginning of the subject, but from the point from which we
should learn most easily. (4) That from which, as an immanent part,
a thing first comes to be, e,g, as the keel of a ship and the foundation
of a house, while in animals some suppose the heart, others the brain,
others some other part, to be of this nature. (4) That from which,
not as an immanent part, a thing first comes to be, and from which
the movement or the change naturally first begins, as a child comes
from its father and its mother, and a fight from abusive language.
(5) That at whose will that which is moved is moved and that which
changes changes, e.g. the magistracies in cities, and oligarchies
and monarchies and tyrannies, are called arhchai, and so are the arts,
and of these especially the architectonic arts. (6) That from which
a thing can first be known,-this also is called the beginning of the
thing, e.g. the hypotheses are the beginnings of demonstrations. (Causes
are spoken of in an equal number of senses; for all causes are beginnings.)
It is common, then, to all beginnings to be the first point from which
a thing either is or comes to be or is known; but of these some are
immanent in the thing and others are outside. Hence the nature of
a thing is a beginning, and so is the element of a thing, and thought
and will, and essence, and the final cause-for the good and the beautiful
are the beginning both of the knowledge and of the movement of many
things. 

Part 2 "

"'Cause' means (1) that from which, as immanent material, a thing
comes into being, e.g. the bronze is the cause of the statue and the
silver of the saucer, and so are the classes which include these.
(2) The form or pattern, i.e. the definition of the essence, and the
classes which include this (e.g. the ratio 2:1 and number in general
are causes of the octave), and the parts included in the definition.
(3) That from which the change or the resting from change first begins;
e.g. the adviser is a cause of the action, and the father a cause
of the child, and in general the maker a cause of the thing made and
the change-producing of the changing. (4) The end, i.e. that for the
sake of which a thing is; e.g. health is the cause of walking. For
'Why does one walk?' we say; 'that one may be healthy'; and in speaking
thus we think we have given the cause. The same is true of all the
means that intervene before the end, when something else has put the
process in motion, as e.g. thinning or purging or drugs or instruments
intervene before health is reached; for all these are for the sake
of the end, though they differ from one another in that some are instruments
and others are actions. 

"These, then, are practically all the senses in which causes are spoken
of, and as they are spoken of in several senses it follows both that
there are several causes of the same thing, and in no accidental sense
(e.g. both the art of sculpture and the bronze are causes of the statue
not in respect of anything else but qua statue; not, however, in the
same way, but the one as matter and the other as source of the movement),
and that things can be causes of one another (e.g. exercise of good
condition, and the latter of exercise; not, however, in the same way,
but the one as end and the other as source of movement).-Again, the
same thing is the cause of contraries; for that which when present
causes a particular thing, we sometimes charge, when absent, with
the contrary, e.g. we impute the shipwreck to the absence of the steersman,
whose presence was the cause of safety; and both-the presence and
the privation-are causes as sources of movement. 

"All the causes now mentioned fall under four senses which are the
most obvious. For the letters are the cause of syllables, and the
material is the cause of manufactured things, and fire and earth and
all such things are the causes of bodies, and the parts are causes
of the whole, and the hypotheses are causes of the conclusion, in
the sense that they are that out of which these respectively are made;
but of these some are cause as the substratum (e.g. the parts), others
as the essence (the whole, the synthesis, and the form). The semen,
the physician, the adviser, and in general the agent, are all sources
of change or of rest. The remainder are causes as the end and the
good of the other things; for that for the sake of which other things
are tends to be the best and the end of the other things; let us take
it as making no difference whether we call it good or apparent good.

"These, then, are the causes, and this is the number of their kinds,
but the varieties of causes are many in number, though when summarized
these also are comparatively few. Causes are spoken of in many senses,
and even of those which are of the same kind some are causes in a
prior and others in a posterior sense, e.g. both 'the physician' and
'the professional man' are causes of health, and both 'the ratio 2:1'
and 'number' are causes of the octave, and the classes that include
any particular cause are always causes of the particular effect. Again,
there are accidental causes and the classes which include these; e.g.
while in one sense 'the sculptor' causes the statue, in another sense
'Polyclitus' causes it, because the sculptor happens to be Polyclitus;
and the classes that include the accidental cause are also causes,
e.g. 'man'-or in general 'animal'-is the cause of the statue, because
Polyclitus is a man, and man is an animal. Of accidental causes also
some are more remote or nearer than others, as, for instance, if 'the
white' and 'the musical' were called causes of the statue, and not
only 'Polyclitus' or 'man'. But besides all these varieties of causes,
whether proper or accidental, some are called causes as being able
to act, others as acting; e.g. the cause of the house's being built
is a builder, or a builder who is building.-The same variety of language
will be found with regard to the effects of causes; e.g. a thing may
be called the cause of this statue or of a statue or in general of
an image, and of this bronze or of bronze or of matter in general;
and similarly in the case of accidental effects. Again, both accidental
and proper causes may be spoken of in combination; e.g. we may say
not 'Polyclitus' nor 'the sculptor' but 'Polyclitus the sculptor'.
Yet all these are but six in number, while each is spoken of in two
ways; for (A) they are causes either as the individual, or as the
genus, or as the accidental, or as the genus that includes the accidental,
and these either as combined, or as taken simply; and (B) all may
be taken as acting or as having a capacity. But they differ inasmuch
as the acting causes, i.e. the individuals, exist, or do not exist,
simultaneously with the things of which they are causes, e.g. this
particular man who is healing, with this particular man who is recovering
health, and this particular builder with this particular thing that
is being built; but the potential causes are not always in this case;
for the house does not perish at the same time as the builder.

Part 3 "

"'Element' means (1) the primary component immanent in a thing, and
indivisible in kind into other kinds; e.g. the elements of speech
are the parts of which speech consists and into which it is ultimately
divided, while they are no longer divided into other forms of speech
different in kind from them. If they are divided, their parts are
of the same kind, as a part of water is water (while a part of the
syllable is not a syllable). Similarly those who speak of the elements
of bodies mean the things into which bodies are ultimately divided,
while they are no longer divided into other things differing in kind;
and whether the things of this sort are one or more, they call these
elements. The so-called elements of geometrical proofs, and in general
the elements of demonstrations, have a similar character; for the
primary demonstrations, each of which is implied in many demonstrations,
are called elements of demonstrations; and the primary syllogisms,
which have three terms and proceed by means of one middle, are of
this nature. 

"(2) People also transfer the word 'element' from this meaning and
apply it to that which, being one and small, is useful for many purposes;
for which reason what is small and simple and indivisible is called
an element. Hence come the facts that the most universal things are
elements (because each of them being one and simple is present in
a plurality of things, either in all or in as many as possible), and
that unity and the point are thought by some to be first principles.
Now, since the so-called genera are universal and indivisible (for
there is no definition of them), some say the genera are elements,
and more so than the differentia, because the genus is more universal;
for where the differentia is present, the genus accompanies it, but
where the genus is present, the differentia is not always so. It is
common to all the meanings that the element of each thing is the first
component immanent in each. 

Part 4 "

"'Nature' means (1) the genesis of growing things-the meaning which
would be suggested if one were to pronounce the 'u' in phusis long.
(2) That immanent part of a growing thing, from which its growth first
proceeds. (3) The source from which the primary movement in each natural
object is present in it in virtue of its own essence. Those things
are said to grow which derive increase from something else by contact
and either by organic unity, or by organic adhesion as in the case
of embryos. Organic unity differs from contact; for in the latter
case there need not be anything besides the contact, but in organic
unities there is something identical in both parts, which makes them
grow together instead of merely touching, and be one in respect of
continuity and quantity, though not of quality.-(4) 'Nature' means
the primary material of which any natural object consists or out of
which it is made, which is relatively unshaped and cannot be changed
from its own potency, as e.g. bronze is said to be the nature of a
statue and of bronze utensils, and wood the nature of wooden things;
and so in all other cases; for when a product is made out of these
materials, the first matter is preserved throughout. For it is in
this way that people call the elements of natural objects also their
nature, some naming fire, others earth, others air, others water,
others something else of the sort, and some naming more than one of
these, and others all of them.-(5) 'Nature' means the essence of natural
objects, as with those who say the nature is the primary mode of composition,
or as Empedocles says:- "

"Nothing that is has a nature, 

"But only mixing and parting of the mixed, 

"And nature is but a name given them by men. "

Hence as regards the things that are or come to be by nature, though
that from which they naturally come to be or are is already present,
we say they have not their nature yet, unless they have their form
or shape. That which comprises both of these exists by nature, e.g.
the animals and their parts; and not only is the first matter nature
(and this in two senses, either the first, counting from the thing,
or the first in general; e.g. in the case of works in bronze, bronze
is first with reference to them, but in general perhaps water is first,
if all things that can be melted are water), but also the form or
essence, which is the end of the process of becoming.-(6) By an extension
of meaning from this sense of 'nature' every essence in general has
come to be called a 'nature', because the nature of a thing is one
kind of essence. 

"From what has been said, then, it is plain that nature in the primary
and strict sense is the essence of things which have in themselves,
as such, a source of movement; for the matter is called the nature
because it is qualified to receive this, and processes of becoming
and growing are called nature because they are movements proceeding
from this. And nature in this sense is the source of the movement
of natural objects, being present in them somehow, either potentially
or in complete reality. 

Part 5 "

"We call 'necessary' (1, a) that without which, as a condition, a
thing cannot live; e.g. breathing and food are necessary for an animal;
for it is incapable of existing without these; (b) the conditions
without which good cannot be or come to be, or without which we cannot
get rid or be freed of evil; e.g. drinking the medicine is necessary
in order that we may be cured of disease, and a man's sailing to Aegina
is necessary in order that he may get his money.-(2) The compulsory
and compulsion, i.e. that which impedes and tends to hinder, contrary
to impulse and purpose. For the compulsory is called necessary (whence
the necessary is painful, as Evenus says: 'For every necessary thing
is ever irksome'), and compulsion is a form of necessity, as Sophocles
says: 'But force necessitates me to this act'. And necessity is held
to be something that cannot be persuaded-and rightly, for it is contrary
to the movement which accords with purpose and with reasoning.-(3)
We say that that which cannot be otherwise is necessarily as it is.
And from this sense of 'necessary' all the others are somehow derived;
for a thing is said to do or suffer what is necessary in the sense
of compulsory, only when it cannot act according to its impulse because
of the compelling forces-which implies that necessity is that because
of which a thing cannot be otherwise; and similarly as regards the
conditions of life and of good; for when in the one case good, in
the other life and being, are not possible without certain conditions,
these are necessary, and this kind of cause is a sort of necessity.
Again, demonstration is a necessary thing because the conclusion cannot
be otherwise, if there has been demonstration in the unqualified sense;
and the causes of this necessity are the first premisses, i.e. the
fact that the propositions from which the syllogism proceeds cannot
be otherwise. 

"Now some things owe their necessity to something other than themselves;
others do not, but are themselves the source of necessity in other
things. Therefore the necessary in the primary and strict sense is
the simple; for this does not admit of more states than one, so that
it cannot even be in one state and also in another; for if it did
it would already be in more than one. If, then, there are any things
that are eternal and unmovable, nothing compulsory or against their
nature attaches to them. 

Part 6 "

"'One' means (1) that which is one by accident, (2) that which is
one by its own nature. (1) Instances of the accidentally one are 'Coriscus
and what is musical', and 'musical Coriscus' (for it is the same thing
to say 'Coriscus and what is musical', and 'musical Coriscus'), and
'what is musical and what is just', and 'musical Coriscus and just
Coriscus'. For all of these are called one by virtue of an accident,
'what is just and what is musical' because they are accidents of one
substance, 'what is musical and Coriscus' because the one is an accident
of the other; and similarly in a sense 'musical Coriscus' is one with
'Coriscus' because one of the parts of the phrase is an accident of
the other, i.e. 'musical' is an accident of Coriscus; and 'musical
Coriscus' is one with 'just Coriscus' because one part of each is
an accident of one and the same subject. The case is similar if the
accident is predicated of a genus or of any universal name, e.g. if
one says that man is the same as 'musical man'; for this is either
because 'musical' is an accident of man, which is one substance, or
because both are accidents of some individual, e.g. Coriscus. Both,
however, do not belong to him in the same way, but one presumably
as genus and included in his substance, the other as a state or affection
of the substance. 

"The things, then, that are called one in virtue of an accident, are
called so in this way. (2) Of things that are called one in virtue
of their own nature some (a) are so called because they are continuous,
e.g. a bundle is made one by a band, and pieces of wood are made one
by glue; and a line, even if it is bent, is called one if it is continuous,
as each part of the body is, e.g. the leg or the arm. Of these themselves,
the continuous by nature are more one than the continuous by art.
A thing is called continuous which has by its own nature one movement
and cannot have any other; and the movement is one when it is indivisible,
and it is indivisible in respect of time. Those things are continuous
by their own nature which are one not merely by contact; for if you
put pieces of wood touching one another, you will not say these are
one piece of wood or one body or one continuum of any other sort.
Things, then, that are continuous in any way called one, even if they
admit of being bent, and still more those which cannot be bent; e.g.
the shin or the thigh is more one than the leg, because the movement
of the leg need not be one. And the straight line is more one than
the bent; but that which is bent and has an angle we call both one
and not one, because its movement may be either simultaneous or not
simultaneous; but that of the straight line is always simultaneous,
and no part of it which has magnitude rests while another moves, as
in the bent line. 

"(b, i) Things are called one in another sense because their substratum
does not differ in kind; it does not differ in the case of things
whose kind is indivisible to sense. The substratum meant is either
the nearest to, or the farthest from, the final state. For, one the
one hand, wine is said to be one and water is said to be one, qua
indivisible in kind; and, on the other hand, all juices, e.g. oil
and wine, are said to be one, and so are all things that can be melted,
because the ultimate substratum of all is the same; for all of these
are water or air. 

"(ii) Those things also are called one whose genus is one though distinguished
by opposite differentiae-these too are all called one because the
genus which underlies the differentiae is one (e.g. horse, man, and
dog form a unity, because all are animals), and indeed in a way similar
to that in which the matter is one. These are sometimes called one
in this way, but sometimes it is the higher genus that is said to
be the same (if they are infimae species of their genus)-the genus
above the proximate genera; e.g. the isosceles and the equilateral
are one and the same figure because both are triangles; but they are
not the same triangles. 

"(c) Two things are called one, when the definition which states the
essence of one is indivisible from another definition which shows
us the other (though in itself every definition is divisible). Thus
even that which has increased or is diminishing is one, because its
definition is one, as, in the case of plane figures, is the definition
of their form. In general those things the thought of whose essence
is indivisible, and cannot separate them either in time or in place
or in definition, are most of all one, and of these especially those
which are substances. For in general those things that do not admit
of division are called one in so far as they do not admit of it; e.g.
if two things are indistinguishable qua man, they are one kind of
man; if qua animal, one kind of animal; if qua magnitude, one kind
of magnitude.-Now most things are called one because they either do
or have or suffer or are related to something else that is one, but
the things that are primarily called one are those whose substance
is one,-and one either in continuity or in form or in definition;
for we count as more than one either things that are not continuous,
or those whose form is not one, or those whose definition is not one.

"While in a sense we call anything one if it is a quantity and continuous,
in a sense we do not unless it is a whole, i.e. unless it has unity
of form; e.g. if we saw the parts of a shoe put together anyhow we
should not call them one all the same (unless because of their continuity);
we do this only if they are put together so as to be a shoe and to
have already a certain single form. This is why the circle is of all
lines most truly one, because it is whole and complete. 

"(3) The essence of what is one is to be some kind of beginning of
number; for the first measure is the beginning, since that by which
we first know each class is the first measure of the class; the one,
then, is the beginning of the knowable regarding each class. But the
one is not the same in all classes. For here it is a quarter-tone,
and there it is the vowel or the consonant; and there is another unit
of weight and another of movement. But everywhere the one is indivisible
either in quantity or in kind. Now that which is indivisible in quantity
is called a unit if it is not divisible in any dimension and is without
position, a point if it is not divisible in any dimension and has
position, a line if it is divisible in one dimension, a plane if in
two, a body if divisible in quantity in all--i.e. in three--dimensions.
And, reversing the order, that which is divisible in two dimensions
is a plane, that which is divisible in one a line, that which is in
no way divisible in quantity is a point or a unit,-that which has
not position a unit, that which has position a point. 

"Again, some things are one in number, others in species, others in
genus, others by analogy; in number those whose matter is one, in
species those whose definition is one, in genus those to which the
same figure of predication applies, by analogy those which are related
as a third thing is to a fourth. The latter kinds of unity are always
found when the former are; e.g. things that are one in number are
also one in species, while things that are one in species are not
all one in number; but things that are one in species are all one
in genus, while things that are so in genus are not all one in species
but are all one by analogy; while things that are one by analogy are
not all one in genus. 

"Evidently 'many' will have meanings opposite to those of 'one'; some
things are many because they are not continuous, others because their
matter-either the proximate matter or the ultimate-is divisible in
kind, others because the definitions which state their essence are
more than one. 

Part 7 "

"Things are said to 'be' (1) in an accidental sense, (2) by their
own nature. 

"(1) In an accidental sense, e.g. we say 'the righteous doer is musical',
and 'the man is musical', and 'the musician is a man', just as we
say 'the musician builds', because the builder happens to be musical
or the musician to be a builder; for here 'one thing is another' means
'one is an accident of another'. So in the cases we have mentioned;
for when we say 'the man is musical' and 'the musician is a man',
or 'he who is pale is musical' or 'the musician is pale', the last
two mean that both attributes are accidents of the same thing; the
first that the attribute is an accident of that which is, while 'the
musical is a man' means that 'musical' is an accident of a man. (In
this sense, too, the not-pale is said to be, because that of which
it is an accident is.) Thus when one thing is said in an accidental
sense to be another, this is either because both belong to the same
thing, and this is, or because that to which the attribute belongs
is, or because the subject which has as an attribute that of which
it is itself predicated, itself is. 

"(2) The kinds of essential being are precisely those that are indicated
by the figures of predication; for the senses of 'being' are just
as many as these figures. Since, then, some predicates indicate what
the subject is, others its quality, others quantity, others relation,
others activity or passivity, others its 'where', others its 'when',
'being' has a meaning answering to each of these. For there is no
difference between 'the man is recovering' and 'the man recovers',
nor between 'the man is walking or cutting' and 'the man walks' or
'cuts'; and similarly in all other cases. 

"(3) Again, 'being' and 'is' mean that a statement is true, 'not being'
that it is not true but falses-and this alike in the case of affirmation
and of negation; e.g. 'Socrates is musical' means that this is true,
or 'Socrates is not-pale' means that this is true; but 'the diagonal
of the square is not commensurate with the side' means that it is
false to say it is. 

"(4) Again, 'being' and 'that which is' mean that some of the things
we have mentioned 'are' potentially, others in complete reality. For
we say both of that which sees potentially and of that which sees
actually, that it is 'seeing', and both of that which can actualize
its knowledge and of that which is actualizing it, that it knows,
and both of that to which rest is already present and of that which
can rest, that it rests. And similarly in the case of substances;
we say the Hermes is in the stone, and the half of the line is in
the line, and we say of that which is not yet ripe that it is corn.
When a thing is potential and when it is not yet potential must be
explained elsewhere. 

Part 8 "

"We call 'substance' (1) the simple bodies, i.e. earth and fire and
water and everything of the sort, and in general bodies and the things
composed of them, both animals and divine beings, and the parts of
these. All these are called substance because they are not predicated
of a subject but everything else is predicated of them.-(2) That which,
being present in such things as are not predicated of a subject, is
the cause of their being, as the soul is of the being of an animal.-(3)
The parts which are present in such things, limiting them and marking
them as individuals, and by whose destruction the whole is destroyed,
as the body is by the destruction of the plane, as some say, and the
plane by the destruction of the line; and in general number is thought
by some to be of this nature; for if it is destroyed, they say, nothing
exists, and it limits all things.-(4) The essence, the formula of
which is a definition, is also called the substance of each thing.

"It follows, then, that 'substance' has two senses, (A) ultimate substratum,
which is no longer predicated of anything else, and (B) that which,
being a 'this', is also separable and of this nature is the shape
or form of each thing. 

Part 9 "

"'The same' means (1) that which is the same in an accidental sense,
e.g. 'the pale' and 'the musical' are the same because they are accidents
of the same thing, and 'a man' and 'musical' because the one is an
accident of the other; and 'the musical' is 'a man' because it is
an accident of the man. (The complex entity is the same as either
of the simple ones and each of these is the same as it; for both 'the
man' and 'the musical' are said to be the same as 'the musical man',
and this the same as they.) This is why all of these statements are
made not universally; for it is not true to say that every man is
the same as 'the musical' (for universal attributes belong to things
in virtue of their own nature, but accidents do not belong to them
in virtue of their own nature); but of the individuals the statements
are made without qualification. For 'Socrates' and 'musical Socrates'
are thought to be the same; but 'Socrates' is not predicable of more
than one subject, and therefore we do not say 'every Socrates' as
we say 'every man'. 

"Some things are said to be the same in this sense, others (2) are
the same by their own nature, in as many senses as that which is one
by its own nature is so; for both the things whose matter is one either
in kind or in number, and those whose essence is one, are said to
be the same. Clearly, therefore, sameness is a unity of the being
either of more than one thing or of one thing when it is treated as
more than one, ie. when we say a thing is the same as itself; for
we treat it as two. 

"Things are called 'other' if either their kinds or their matters
or the definitions of their essence are more than one; and in general
'other' has meanings opposite to those of 'the same'. 

"'Different' is applied (1) to those things which though other are
the same in some respect, only not in number but either in species
or in genus or by analogy; (2) to those whose genus is other, and
to contraries, and to an things that have their otherness in their
essence. 

"Those things are called 'like' which have the same attributes in
every respect, and those which have more attributes the same than
different, and those whose quality is one; and that which shares with
another thing the greater number or the more important of the attributes
(each of them one of two contraries) in respect of which things are
capable of altering, is like that other thing. The senses of 'unlike'
are opposite to those of 'like'. 

Part 10 "

"The term 'opposite' is applied to contradictories, and to contraries,
and to relative terms, and to privation and possession, and to the
extremes from which and into which generation and dissolution take
place; and the attributes that cannot be present at the same time
in that which is receptive of both, are said to be opposed,-either
themselves of their constituents. Grey and white colour do not belong
at the same time to the same thing; hence their constituents are opposed.

"The term 'contrary' is applied (1) to those attributes differing
in genus which cannot belong at the same time to the same subject,
(2) to the most different of the things in the same genus, (3) to
the most different of the attributes in the same recipient subject,
(4) to the most different of the things that fall under the same faculty,
(5) to the things whose difference is greatest either absolutely or
in genus or in species. The other things that are called contrary
are so called, some because they possess contraries of the above kind,
some because they are receptive of such, some because they are productive
of or susceptible to such, or are producing or suffering them, or
are losses or acquisitions, or possessions or privations, of such.
Since 'one' and 'being' have many senses, the other terms which are
derived from these, and therefore 'same', 'other', and 'contrary',
must correspond, so that they must be different for each category.

"The term 'other in species' is applied to things which being of the
same genus are not subordinate the one to the other, or which being
in the same genus have a difference, or which have a contrariety in
their substance; and contraries are other than one another in species
(either all contraries or those which are so called in the primary
sense), and so are those things whose definitions differ in the infima
species of the genus (e.g. man and horse are indivisible in genus,
but their definitions are different), and those which being in the
same substance have a difference. 'The same in species' has the various
meanings opposite to these. 

Part 11 "

"The words 'prior' and 'posterior' are applied (1) to some things
(on the assumption that there is a first, i.e. a beginning, in each
class) because they are nearer some beginning determined either absolutely
and by nature, or by reference to something or in some place or by
certain people; e.g. things are prior in place because they are nearer
either to some place determined by nature (e.g. the middle or the
last place), or to some chance object; and that which is farther is
posterior.-Other things are prior in time; some by being farther from
the present, i.e. in the case of past events (for the Trojan war is
prior to the Persian, because it is farther from the present), others
by being nearer the present, i.e. in the case of future events (for
the Nemean games are prior to the Pythian, if we treat the present
as beginning and first point, because they are nearer the present).-Other
things are prior in movement; for that which is nearer the first mover
is prior (e.g. the boy is prior to the man); and the prime mover also
is a beginning absolutely.-Others are prior in power; for that which
exceeds in power, i.e. the more powerful, is prior; and such is that
according to whose will the other-i.e. the posterior-must follow,
so that if the prior does not set it in motion the other does not
move, and if it sets it in motion it does move; and here will is a
beginning.-Others are prior in arrangement; these are the things that
are placed at intervals in reference to some one definite thing according
to some rule, e.g. in the chorus the second man is prior to the third,
and in the lyre the second lowest string is prior to the lowest; for
in the one case the leader and in the other the middle string is the
beginning. 

"These, then, are called prior in this sense, but (2) in another sense
that which is prior for knowledge is treated as also absolutely prior;
of these, the things that are prior in definition do not coincide
with those that are prior in relation to perception. For in definition
universals are prior, in relation to perception individuals. And in
definition also the accident is prior to the whole, e.g. 'musical'
to 'musical man', for the definition cannot exist as a whole without
the part; yet musicalness cannot exist unless there is some one who
is musical. 

"(3) The attributes of prior things are called prior, e.g. straightness
is prior to smoothness; for one is an attribute of a line as such,
and the other of a surface. 

"Some things then are called prior and posterior in this sense, others
(4) in respect of nature and substance, i.e. those which can be without
other things, while the others cannot be without them,-a distinction
which Plato used. (If we consider the various senses of 'being', firstly
the subject is prior, so that substance is prior; secondly, according
as potency or complete reality is taken into account, different things
are prior, for some things are prior in respect of potency, others
in respect of complete reality, e.g. in potency the half line is prior
to the whole line, and the part to the whole, and the matter to the
concrete substance, but in complete reality these are posterior; for
it is only when the whole has been dissolved that they will exist
in complete reality.) In a sense, therefore, all things that are called
prior and posterior are so called with reference to this fourth sense;
for some things can exist without others in respect of generation,
e.g. the whole without the parts, and others in respect of dissolution,
e.g. the part without the whole. And the same is true in all other
cases. 

Part 12 "

"'Potency' means (1) a source of movement or change, which is in another
thing than the thing moved or in the same thing qua other; e.g. the
art of building is a potency which is not in the thing built, while
the art of healing, which is a potency, may be in the man healed,
but not in him qua healed. 'Potency' then means the source, in general,
of change or movement in another thing or in the same thing qua other,
and also (2) the source of a thing's being moved by another thing
or by itself qua other. For in virtue of that principle, in virtue
of which a patient suffers anything, we call it 'capable' of suffering;
and this we do sometimes if it suffers anything at all, sometimes
not in respect of everything it suffers, but only if it suffers a
change for the better--(3) The capacity of performing this well or
according to intention; for sometimes we say of those who merely can
walk or speak but not well or not as they intend, that they cannot
speak or walk. So too (4) in the case of passivity--(5) The states
in virtue of which things are absolutely impassive or unchangeable,
or not easily changed for the worse, are called potencies; for things
are broken and crushed and bent and in general destroyed not by having
a potency but by not having one and by lacking something, and things
are impassive with respect to such processes if they are scarcely
and slightly affected by them, because of a 'potency' and because
they 'can' do something and are in some positive state. 

"'Potency' having this variety of meanings, so too the 'potent' or
'capable' in one sense will mean that which can begin a movement (or
a change in general, for even that which can bring things to rest
is a 'potent' thing) in another thing or in itself qua other; and
in one sense that over which something else has such a potency; and
in one sense that which has a potency of changing into something,
whether for the worse or for the better (for even that which perishes
is thought to be 'capable' of perishing, for it would not have perished
if it had not been capable of it; but, as a matter of fact, it has
a certain disposition and cause and principle which fits it to suffer
this; sometimes it is thought to be of this sort because it has something,
sometimes because it is deprived of something; but if privation is
in a sense 'having' or 'habit', everything will be capable by having
something, so that things are capable both by having a positive habit
and principle, and by having the privation of this, if it is possible
to have a privation; and if privation is not in a sense 'habit', 'capable'
is used in two distinct senses); and a thing is capable in another
sense because neither any other thing, nor itself qua other, has a
potency or principle which can destroy it. Again, all of these are
capable either merely because the thing might chance to happen or
not to happen, or because it might do so well. This sort of potency
is found even in lifeless things, e.g. in instruments; for we say
one lyre can speak, and another cannot speak at all, if it has not
a good tone. 

"Incapacity is privation of capacity-i.e. of such a principle as has
been described either in general or in the case of something that
would naturally have the capacity, or even at the time when it would
naturally already have it; for the senses in which we should call
a boy and a man and a eunuch 'incapable of begetting' are distinct.-Again,
to either kind of capacity there is an opposite incapacity-both to
that which only can produce movement and to that which can produce
it well. 

"Some things, then, are called adunata in virtue of this kind of incapacity,
while others are so in another sense; i.e. both dunaton and adunaton
are used as follows. The impossible is that of which the contrary
is of necessity true, e.g. that the diagonal of a square is commensurate
with the side is impossible, because such a statement is a falsity
of which the contrary is not only true but also necessary; that it
is commensurate, then, is not only false but also of necessity false.
The contrary of this, the possible, is found when it is not necessary
that the contrary is false, e.g. that a man should be seated is possible;
for that he is not seated is not of necessity false. The possible,
then, in one sense, as has been said, means that which is not of necessity
false; in one, that which is true; in one, that which may be true.-A
'potency' or 'power' in geometry is so called by a change of meaning.-These
senses of 'capable' or 'possible' involve no reference to potency.
But the senses which involve a reference to potency all refer to the
primary kind of potency; and this is a source of change in another
thing or in the same thing qua other. For other things are called
'capable', some because something else has such a potency over them,
some because it has not, some because it has it in a particular way.
The same is true of the things that are incapable. Therefore the proper
definition of the primary kind of potency will be 'a source of change
in another thing or in the same thing qua other'. 

Part 13 "

"'Quantum' means that which is divisible into two or more constituent
parts of which each is by nature a 'one' and a 'this'. A quantum is
a plurality if it is numerable, a magnitude if it is a measurable.
'Plurality' means that which is divisible potentially into non-continuous
parts, 'magnitude' that which is divisible into continuous parts;
of magnitude, that which is continuous in one dimension is length;
in two breadth, in three depth. Of these, limited plurality is number,
limited length is a line, breadth a surface, depth a solid.

"Again, some things are called quanta in virtue of their own nature,
others incidentally; e.g. the line is a quantum by its own nature,
the musical is one incidentally. Of the things that are quanta by
their own nature some are so as substances, e.g. the line is a quantum
(for 'a certain kind of quantum' is present in the definition which
states what it is), and others are modifications and states of this
kind of substance, e.g. much and little, long and short, broad and
narrow, deep and shallow, heavy and light, and all other such attributes.
And also great and small, and greater and smaller, both in themselves
and when taken relatively to each other, are by their own nature attributes
of what is quantitative; but these names are transferred to other
things also. Of things that are quanta incidentally, some are so called
in the sense in which it was said that the musical and the white were
quanta, viz. because that to which musicalness and whiteness belong
is a quantum, and some are quanta in the way in which movement and
time are so; for these also are called quanta of a sort and continuous
because the things of which these are attributes are divisible. I
mean not that which is moved, but the space through which it is moved;
for because that is a quantum movement also is a quantum, and because
this is a quantum time is one. 

Part 14 "

"'Quality' means (1) the differentia of the essence, e.g. man is an
animal of a certain quality because he is two-footed, and the horse
is so because it is four-footed; and a circle is a figure of particular
quality because it is without angles,-which shows that the essential
differentia is a quality.-This, then, is one meaning of quality-the
differentia of the essence, but (2) there is another sense in which
it applies to the unmovable objects of mathematics, the sense in which
the numbers have a certain quality, e.g. the composite numbers which
are not in one dimension only, but of which the plane and the solid
are copies (these are those which have two or three factors); and
in general that which exists in the essence of numbers besides quantity
is quality; for the essence of each is what it is once, e.g. that
of is not what it is twice or thrice, but what it is once; for 6 is
once 6. 

"(3) All the modifications of substances that move (e.g. heat and
cold, whiteness and blackness, heaviness and lightness, and the others
of the sort) in virtue of which, when they change, bodies are said
to alter. (4) Quality in respect of virtue and vice, and in general,
of evil and good. 

"Quality, then, seems to have practically two meanings, and one of
these is the more proper. The primary quality is the differentia of
the essence, and of this the quality in numbers is a part; for it
is a differentia of essences, but either not of things that move or
not of them qua moving. Secondly, there are the modifications of things
that move, qua moving, and the differentiae of movements. Virtue and
vice fall among these modifications; for they indicate differentiae
of the movement or activity, according to which the things in motion
act or are acted on well or badly; for that which can be moved or
act in one way is good, and that which can do so in another--the contrary--way
is vicious. Good and evil indicate quality especially in living things,
and among these especially in those which have purpose. 

Part 15 

"Things are 'relative' (1) as double to half, and treble to a third,
and in general that which contains something else many times to that
which is contained many times in something else, and that which exceeds
to that which is exceeded; (2) as that which can heat to that which
can be heated, and that which can cut to that which can be cut, and
in general the active to the passive; (3) as the measurable to the
measure, and the knowable to knowledge, and the perceptible to perception.

"(1) Relative terms of the first kind are numerically related either
indefinitely or definitely, to numbers themselves or to 1. E.g. the
double is in a definite numerical relation to 1, and that which is
'many times as great' is in a numerical, but not a definite, relation
to 1, i.e. not in this or in that numerical relation to it; the relation
of that which is half as big again as something else to that something
is a definite numerical relation to a number; that which is n+I/n
times something else is in an indefinite relation to that something,
as that which is 'many times as great' is in an indefinite relation
to 1; the relation of that which exceeds to that which is exceeded
is numerically quite indefinite; for number is always commensurate,
and 'number' is not predicated of that which is not commensurate,
but that which exceeds is, in relation to that which is exceeded,
so much and something more; and this something is indefinite; for
it can, indifferently, be either equal or not equal to that which
is exceeded.-All these relations, then, are numerically expressed
and are determinations of number, and so in another way are the equal
and the like and the same. For all refer to unity. Those things are
the same whose substance is one; those are like whose quality is one;
those are equal whose quantity is one; and 1 is the beginning and
measure of number, so that all these relations imply number, though
not in the same way. 

"(2) Things that are active or passive imply an active or a passive
potency and the actualizations of the potencies; e.g. that which is
capable of heating is related to that which is capable of being heated,
because it can heat it, and, again, that which heats is related to
that which is heated and that which cuts to that which is cut, in
the sense that they actually do these things. But numerical relations
are not actualized except in the sense which has been elsewhere stated;
actualizations in the sense of movement they have not. Of relations
which imply potency some further imply particular periods of time,
e.g. that which has made is relative to that which has been made,
and that which will make to that which will be made. For it is in
this way that a father is called the father of his son; for the one
has acted and the other has been acted on in a certain way. Further,
some relative terms imply privation of potency, i.e. 'incapable' and
terms of this sort, e.g. 'invisible'. 

"Relative terms which imply number or potency, therefore, are all
relative because their very essence includes in its nature a reference
to something else, not because something else involves a reference
to it; but (3) that which is measurable or knowable or thinkable is
called relative because something else involves a reference to it.
For 'that which is thinkable' implies that the thought of it is possible,
but the thought is not relative to 'that of which it is the thought';
for we should then have said the same thing twice. Similarly sight
is the sight of something, not 'of that of which it is the sight'
(though of course it is true to say this); in fact it is relative
to colour or to something else of the sort. But according to the other
way of speaking the same thing would be said twice,-'the sight is
of that of which it is.' 

"Things that are by their own nature called relative are called so
sometimes in these senses, sometimes if the classes that include them
are of this sort; e.g. medicine is a relative term because its genus,
science, is thought to be a relative term. Further, there are the
properties in virtue of which the things that have them are called
relative, e.g. equality is relative because the equal is, and likeness
because the like is. Other things are relative by accident; e.g. a
man is relative because he happens to be double of something and double
is a relative term; or the white is relative, if the same thing happens
to be double and white. 

Part 16 "

"What is called 'complete' is (1) that outside which it is not possible
to find any, even one, of its parts; e.g. the complete time of each
thing is that outside which it is not possible to find any time which
is a part proper to it.-(2) That which in respect of excellence and
goodness cannot be excelled in its kind; e.g. we have a complete doctor
or a complete flute-player, when they lack nothing in respect of the
form of their proper excellence. And thus, transferring the word to
bad things, we speak of a complete scandal-monger and a complete thief;
indeed we even call them good, i.e. a good thief and a good scandal-monger.
And excellence is a completion; for each thing is complete and every
substance is complete, when in respect of the form of its proper excellence
it lacks no part of its natural magnitude.-(3) The things which have
attained their end, this being good, are called complete; for things
are complete in virtue of having attained their end. Therefore, since
the end is something ultimate, we transfer the word to bad things
and say a thing has been completely spoilt, and completely destroyed,
when it in no wise falls short of destruction and badness, but is
at its last point. This is why death, too, is by a figure of speech
called the end, because both are last things. But the ultimate purpose
is also an end.-Things, then, that are called complete in virtue of
their own nature are so called in all these senses, some because in
respect of goodness they lack nothing and cannot be excelled and no
part proper to them can be found outside them, others in general because
they cannot be exceeded in their several classes and no part proper
to them is outside them; the others presuppose these first two kinds,
and are called complete because they either make or have something
of the sort or are adapted to it or in some way or other involve a
reference to the things that are called complete in the primary sense.

Part 17 "

"'Limit' means (1) the last point of each thing, i.e. the first point
beyond which it is not possible to find any part, and the first point
within which every part is; (2) the form, whatever it may be, of a
spatial magnitude or of a thing that has magnitude; (3) the end of
each thing (and of this nature is that towards which the movement
and the action are, not that from which they are-though sometimes
it is both, that from which and that to which the movement is, i.e.
the final cause); (4) the substance of each thing, and the essence
of each; for this is the limit of knowledge; and if of knowledge,
of the object also. Evidently, therefore, 'limit' has as many senses
as 'beginning', and yet more; for the beginning is a limit, but not
every limit is a beginning. 

Part 18 "

"'That in virtue of which' has several meanings:-(1) the form or substance
of each thing, e.g. that in virtue of which a man is good is the good
itself, (2) the proximate subject in which it is the nature of an
attribute to be found, e.g. colour in a surface. 'That in virtue of
which', then, in the primary sense is the form, and in a secondary
sense the matter of each thing and the proximate substratum of each.-In
general 'that in virtue of which' will found in the same number of
senses as 'cause'; for we say indifferently (3) in virtue of what
has he come?' or 'for what end has he come?'; and (4) in virtue of
what has he inferred wrongly, or inferred?' or 'what is the cause
of the inference, or of the wrong inference?'-Further (5) Kath' d
is used in reference to position, e.g. 'at which he stands' or 'along
which he walks; for all such phrases indicate place and position.

"Therefore 'in virtue of itself' must likewise have several meanings.
The following belong to a thing in virtue of itself:-(1) the essence
of each thing, e.g. Callias is in virtue of himself Callias and what
it was to be Callias;-(2) whatever is present in the 'what', e.g.
Callias is in virtue of himself an animal. For 'animal' is present
in his definition; Callias is a particular animal.-(3) Whatever attribute
a thing receives in itself directly or in one of its parts; e.g. a
surface is white in virtue of itself, and a man is alive in virtue
of himself; for the soul, in which life directly resides, is a part
of the man.-(4) That which has no cause other than itself; man has
more than one cause--animal, two-footed--but yet man is man in virtue
of himself.-(5) Whatever attributes belong to a thing alone, and in
so far as they belong to it merely by virtue of itself considered
apart by itself. 

Part 19 "

"'Disposition' means the arrangement of that which has parts, in respect
either of place or of potency or of kind; for there must be a certain
position, as even the word 'disposition' shows. 

Part 20 "

"'Having' means (1) a kind of activity of the haver and of what he
has-something like an action or movement. For when one thing makes
and one is made, between them there is a making; so too between him
who has a garment and the garment which he has there is a having.
This sort of having, then, evidently we cannot have; for the process
will go on to infinity, if it is to be possible to have the having
of what we have.-(2) 'Having' or 'habit' means a disposition according
to which that which is disposed is either well or ill disposed, and
either in itself or with reference to something else; e.g. health
is a 'habit'; for it is such a disposition.-(3) We speak of a 'habit'
if there is a portion of such a disposition; and so even the excellence
of the parts is a 'habit' of the whole thing. 

Part 21 "

"'Affection' means (1) a quality in respect of which a thing can be
altered, e.g. white and black, sweet and bitter, heaviness and lightness,
and all others of the kind.-(2) The actualization of these-the already
accomplished alterations.-(3) Especially, injurious alterations and
movements, and, above all painful injuries.-(4) Misfortunes and painful
experiences when on a large scale are called affections.

Part 22 "

"We speak of 'privation' (1) if something has not one of the attributes
which a thing might naturally have, even if this thing itself would
not naturally have it; e.g. a plant is said to be 'deprived' of eyes.-(2)
If, though either the thing itself or its genus would naturally have
an attribute, it has it not; e.g. a blind man and a mole are in different
senses 'deprived' of sight; the latter in contrast with its genus,
the former in contrast with his own normal nature.-(3) If, though
it would naturally have the attribute, and when it would naturally
have it, it has it not; for blindness is a privation, but one is not
'blind' at any and every age, but only if one has not sight at the
age at which one would naturally have it. Similarly a thing is called
blind if it has not sight in the medium in which, and in respect of
the organ in respect of which, and with reference to the object with
reference to which, and in the circumstances in which, it would naturally
have it.-(4) The violent taking away of anything is called privation.

"Indeed there are just as many kinds of privations as there are of
words with negative prefixes; for a thing is called unequal because
it has not equality though it would naturally have it, and invisible
either because it has no colour at all or because it has a poor colour,
and apodous either because it has no feet at all or because it has
imperfect feet. Again, a privative term may be used because the thing
has little of the attribute (and this means having it in a sense imperfectly),
e.g. 'kernel-less'; or because it has it not easily or not well (e.g.
we call a thing uncuttable not only if it cannot be cut but also if
it cannot be cut easily or well); or because it has not the attribute
at all; for it is not the one-eyed man but he who is sightless in
both eyes that is called blind. This is why not every man is 'good'
or 'bad', 'just' or 'unjust', but there is also an intermediate state.

Part 23 "

"To 'have' or 'hold' means many things:-(1) to treat a thing according
to one's own nature or according to one's own impulse; so that fever
is said to have a man, and tyrants to have their cities, and people
to have the clothes they wear.-(2) That in which a thing is present
as in something receptive of it is said to have the thing; e.g. the
bronze has the form of the statue, and the body has the disease.-(3)
As that which contains holds the things contained; for a thing is
said to be held by that in which it is as in a container; e.g. we
say that the vessel holds the liquid and the city holds men and the
ship sailors; and so too that the whole holds the parts.-(4) That
which hinders a thing from moving or acting according to its own impulse
is said to hold it, as pillars hold the incumbent weights, and as
the poets make Atlas hold the heavens, implying that otherwise they
would collapse on the earth, as some of the natural philosophers also
say. In this way also that which holds things together is said to
hold the things it holds together, since they would otherwise separate,
each according to its own impulse. 

"'Being in something' has similar and corresponding meanings to 'holding'
or 'having'. 

Part 24 "

"'To come from something' means (1) to come from something as from
matter, and this in two senses, either in respect of the highest genus
or in respect of the lowest species; e.g. in a sense all things that
can be melted come from water, but in a sense the statue comes from
bronze.-(2) As from the first moving principle; e.g. 'what did the
fight come from?' From abusive language, because this was the origin
of the fight.-(3) From the compound of matter and shape, as the parts
come from the whole, and the verse from the Iliad, and the stones
from the house; (in every such case the whole is a compound of matter
and shape,) for the shape is the end, and only that which attains
an end is complete.-(4) As the form from its part, e.g. man from 'two-footed'and
syllable from 'letter'; for this is a different sense from that in
which the statue comes from bronze; for the composite substance comes
from the sensible matter, but the form also comes from the matter
of the form.-Some things, then, are said to come from something else
in these senses; but (5) others are so described if one of these senses
is applicable to a part of that other thing; e.g. the child comes
from its father and mother, and plants come from the earth, because
they come from a part of those things.-(6) It means coming after a
thing in time, e.g. night comes from day and storm from fine weather,
because the one comes after the other. Of these things some are so
described because they admit of change into one another, as in the
cases now mentioned; some merely because they are successive in time,
e.g. the voyage took place 'from' the equinox, because it took place
after the equinox, and the festival of the Thargelia comes 'from'
the Dionysia, because after the Dionysia. 

Part 25 "

"'Part' means (1, a) that into which a quantum can in any way be
divided; for that which is taken from a quantum qua quantum is always
called a part of it, e.g. two is called in a sense a part of three.
It means (b), of the parts in the first sense, only those which measure
the whole; this is why two, though in one sense it is, in another
is not, called a part of three.-(2) The elements into which a kind
might be divided apart from the quantity are also called parts of
it; for which reason we say the species are parts of the genus.-(3)
The elements into which a whole is divided, or of which it consists-the
'whole' meaning either the form or that which has the form; e.g. of
the bronze sphere or of the bronze cube both the bronze-i.e. the matter
in which the form is-and the characteristic angle are parts.-(4) The
elements in the definition which explains a thing are also parts of
the whole; this is why the genus is called a part of the species,
though in another sense the species is part of the genus.

Part 26 "

"'A whole' means (1) that from which is absent none of the parts of
which it is said to be naturally a whole, and (2) that which so contains
the things it contains that they form a unity; and this in two senses-either
as being each severally one single thing, or as making up the unity
between them. For (a) that which is true of a whole class and is said
to hold good as a whole (which implies that it is a kind whole) is
true of a whole in the sense that it contains many things by being
predicated of each, and by all of them, e.g. man, horse, god, being
severally one single thing, because all are living things. But (b)
the continuous and limited is a whole, when it is a unity consisting
of several parts, especially if they are present only potentially,
but, failing this, even if they are present actually. Of these things
themselves, those which are so by nature are wholes in a higher degree
than those which are so by art, as we said in the case of unity also,
wholeness being in fact a sort of oneness. 

"Again (3) of quanta that have a beginning and a middle and an end,
those to which the position does not make a difference are called
totals, and those to which it does, wholes. Those which admit of both
descriptions are both wholes and totals. These are the things whose
nature remains the same after transposition, but whose form does not,
e.g. wax or a coat; they are called both wholes and totals; for they
have both characteristics. Water and all liquids and number are called
totals, but 'the whole number' or 'the whole water' one does not speak
of, except by an extension of meaning. To things, to which qua one
the term 'total' is applied, the term 'all' is applied when they are
treated as separate; 'this total number,' 'all these units.'

Part 27 "

"It is not any chance quantitative thing that can be said to be 'mutilated';
it must be a whole as well as divisible. For not only is two not 'mutilated'
if one of the two ones is taken away (for the part removed by mutilation
is never equal to the remainder), but in general no number is thus
mutilated; for it is also necessary that the essence remain; if a
cup is mutilated, it must still be a cup; but the number is no longer
the same. Further, even if things consist of unlike parts, not even
these things can all be said to be mutilated, for in a sense a number
has unlike parts (e.g. two and three) as well as like; but in general
of the things to which their position makes no difference, e.g. water
or fire, none can be mutilated; to be mutilated, things must be such
as in virtue of their essence have a certain position. Again, they
must be continuous; for a musical scale consists of unlike parts and
has position, but cannot become mutilated. Besides, not even the things
that are wholes are mutilated by the privation of any part. For the
parts removed must be neither those which determine the essence nor
any chance parts, irrespective of their position; e.g. a cup is not
mutilated if it is bored through, but only if the handle or a projecting
part is removed, and a man is mutilated not if the flesh or the spleen
is removed, but if an extremity is, and that not every extremity but
one which when completely removed cannot grow again. Therefore baldness
is not a mutilation. 

Part 28 "

"The term 'race' or 'genus' is used (1) if generation of things which
have the same form is continuous, e.g. 'while the race of men lasts'
means 'while the generation of them goes on continuously'.-(2) It
is used with reference to that which first brought things into existence;
for it is thus that some are called Hellenes by race and others Ionians,
because the former proceed from Hellen and the latter from Ion as
their first begetter. And the word is used in reference to the begetter
more than to the matter, though people also get a race-name from the
female, e.g. 'the descendants of Pyrrha'.-(3) There is genus in the
sense in which 'plane' is the genus of plane figures and solid' of
solids; for each of the figures is in the one case a plane of such
and such a kind, and in the other a solid of such and such a kind;
and this is what underlies the differentiae. Again (4) in definitions
the first constituent element, which is included in the 'what', is
the genus, whose differentiae the qualities are said to be 'Genus'
then is used in all these ways, (1) in reference to continuous generation
of the same kind, (2) in reference to the first mover which is of
the same kind as the things it moves, (3) as matter; for that to which
the differentia or quality belongs is the substratum, which we call
matter. 

"Those things are said to be 'other in genus' whose proximate substratum
is different, and which are not analysed the one into the other nor
both into the same thing (e.g. form and matter are different in genus);
and things which belong to different categories of being (for some
of the things that are said to 'be' signify essence, others a quality,
others the other categories we have before distinguished); these also
are not analysed either into one another or into some one thing.

Part 29 "

"'The false' means (1) that which is false as a thing, and that (a)
because it is not put together or cannot be put together, e.g. 'that
the diagonal of a square is commensurate with the side' or 'that you
are sitting'; for one of these is false always, and the other sometimes;
it is in these two senses that they are non-existent. (b) There are
things which exist, but whose nature it is to appear either not to
be such as they are or to be things that do not exist, e.g. a sketch
or a dream; for these are something, but are not the things the appearance
of which they produce in us. We call things false in this way, then,-either
because they themselves do not exist, or because the appearance which
results from them is that of something that does not exist.

"(2) A false account is the account of non-existent objects, in so
far as it is false. Hence every account is false when applied to something
other than that of which it is true; e.g. the account of a circle
is false when applied to a triangle. In a sense there is one account
of each thing, i.e. the account of its essence, but in a sense there
are many, since the thing itself and the thing itself with an attribute
are in a sense the same, e.g. Socrates and musical Socrates (a false
account is not the account of anything, except in a qualified sense).
Hence Antisthenes was too simple-minded when he claimed that nothing
could be described except by the account proper to it,-one predicate
to one subject; from which the conclusion used to be drawn that there
could be no contradiction, and almost that there could be no error.
But it is possible to describe each thing not only by the account
of itself, but also by that of something else. This may be done altogether
falsely indeed, but there is also a way in which it may be done truly;
e.g. eight may be described as a double number by the use of the definition
of two. "

"These things, then, are called false in these senses, but (3) a false
man is one who is ready at and fond of such accounts, not for any
other reason but for their own sake, and one who is good at impressing
such accounts on other people, just as we say things are which produce
a false appearance. This is why the proof in the Hippias that the
same man is false and true is misleading. For it assumes that he is
false who can deceive (i.e. the man who knows and is wise); and further
that he who is willingly bad is better. This is a false result of
induction-for a man who limps willingly is better than one who does
so unwillingly-by 'limping' Plato means 'mimicking a limp', for if
the man were lame willingly, he would presumably be worse in this
case as in the corresponding case of moral character. 

Part 30 "

"'Accident' means (1) that which attaches to something and can be
truly asserted, but neither of necessity nor usually, e.g. if some
one in digging a hole for a plant has found treasure. This-the finding
of treasure-is for the man who dug the hole an accident; for neither
does the one come of necessity from the other or after the other,
nor, if a man plants, does he usually find treasure. And a musical
man might be pale; but since this does not happen of necessity nor
usually, we call it an accident. Therefore since there are attributes
and they attach to subjects, and some of them attach to these only
in a particular place and at a particular time, whatever attaches
to a subject, but not because it was this subject, or the time this
time, or the place this place, will be an accident. Therefore, too,
there is no definite cause for an accident, but a chance cause, i.e.
an indefinite one. Going to Aegina was an accident for a man, if he
went not in order to get there, but because he was carried out of
his way by a storm or captured by pirates. The accident has happened
or exists,-not in virtue of the subject's nature, however, but of
something else; for the storm was the cause of his coming to a place
for which he was not sailing, and this was Aegina. 

"'Accident' has also (2) another meaning, i.e. all that attaches to
each thing in virtue of itself but is not in its essence, as having
its angles equal to two right angles attaches to the triangle. And
accidents of this sort may be eternal, but no accident of the other
sort is. This is explained elsewhere. 

----------------------------------------------------------------------

BOOK VI

Part 1 

"

"WE are seeking the principles and the causes of the things that are,
and obviously of them qua being. For, while there is a cause of health
and of good condition, and the objects of mathematics have first principles
and elements and causes, and in general every science which is ratiocinative
or at all involves reasoning deals with causes and principles, more
or less precise, all these sciences mark off some particular being-some
genus, and inquire into this, but not into being simply nor qua being,
nor do they offer any discussion of the essence of the things of which
they treat; but starting from the essence-some making it plain to
the senses, others assuming it as a hypothesis-they then demonstrate,
more or less cogently, the essential attributes of the genus with
which they deal. It is obvious, therefore, that such an induction
yields no demonstration of substance or of the essence, but some other
way of exhibiting it. And similarly the sciences omit the question
whether the genus with which they deal exists or does not exist, because
it belongs to the same kind of thinking to show what it is and that
it is. 

"And since natural science, like other sciences, is in fact about
one class of being, i.e. to that sort of substance which has the principle
of its movement and rest present in itself, evidently it is neither
practical nor productive. For in the case of things made the principle
is in the maker-it is either reason or art or some faculty, while
in the case of things done it is in the doer-viz. will, for that which
is done and that which is willed are the same. Therefore, if all thought
is either practical or productive or theoretical, physics must be
a theoretical science, but it will theorize about such being as admits
of being moved, and about substance-as-defined for the most part only
as not separable from matter. Now, we must not fail to notice the
mode of being of the essence and of its definition, for, without this,
inquiry is but idle. Of things defined, i.e. of 'whats', some are
like 'snub', and some like 'concave'. And these differ because 'snub'
is bound up with matter (for what is snub is a concave nose), while
concavity is independent of perceptible matter. If then all natural
things are a analogous to the snub in their nature; e.g. nose, eye,
face, flesh, bone, and, in general, animal; leaf, root, bark, and,
in general, plant (for none of these can be defined without reference
to movement-they always have matter), it is clear how we must seek
and define the 'what' in the case of natural objects, and also that
it belongs to the student of nature to study even soul in a certain
sense, i.e. so much of it as is not independent of matter.

"That physics, then, is a theoretical science, is plain from these
considerations. Mathematics also, however, is theoretical; but whether
its objects are immovable and separable from matter, is not at present
clear; still, it is clear that some mathematical theorems consider
them qua immovable and qua separable from matter. But if there is
something which is eternal and immovable and separable, clearly the
knowledge of it belongs to a theoretical science,-not, however, to
physics (for physics deals with certain movable things) nor to mathematics,
but to a science prior to both. For physics deals with things which
exist separately but are not immovable, and some parts of mathematics
deal with things which are immovable but presumably do not exist separately,
but as embodied in matter; while the first science deals with things
which both exist separately and are immovable. Now all causes must
be eternal, but especially these; for they are the causes that operate
on so much of the divine as appears to us. There must, then, be three
theoretical philosophies, mathematics, physics, and what we may call
theology, since it is obvious that if the divine is present anywhere,
it is present in things of this sort. And the highest science must
deal with the highest genus. Thus, while the theoretical sciences
are more to be desired than the other sciences, this is more to be
desired than the other theoretical sciences. For one might raise the
question whether first philosophy is universal, or deals with one
genus, i.e. some one kind of being; for not even the mathematical
sciences are all alike in this respect,-geometry and astronomy deal
with a certain particular kind of thing, while universal mathematics
applies alike to all. We answer that if there is no substance other
than those which are formed by nature, natural science will be the
first science; but if there is an immovable substance, the science
of this must be prior and must be first philosophy, and universal
in this way, because it is first. And it will belong to this to consider
being qua being-both what it is and the attributes which belong to
it qua being. 

Part 2 "

"But since the unqualified term 'being' has several meanings, of which
one was seen' to be the accidental, and another the true ('non-being'
being the false), while besides these there are the figures of predication
(e.g. the 'what', quality, quantity, place, time, and any similar
meanings which 'being' may have), and again besides all these there
is that which 'is' potentially or actually:-since 'being' has many
meanings, we must say regarding the accidental, that there can be
no scientific treatment of it. This is confirmed by the fact that
no science practical, productive, or theoretical troubles itself about
it. For on the one hand he who produces a house does not produce all
the attributes that come into being along with the house; for these
are innumerable; the house that has been made may quite well be pleasant
for some people, hurtful for some, and useful to others, and different-to
put it shortly from all things that are; and the science of building
does not aim at producing any of these attributes. And in the same
way the geometer does not consider the attributes which attach thus
to figures, nor whether 'triangle' is different from 'triangle whose
angles are equal to two right angles'.-And this happens naturally
enough; for the accidental is practically a mere name. And so Plato
was in a sense not wrong in ranking sophistic as dealing with that
which is not. For the arguments of the sophists deal, we may say,
above all with the accidental; e.g. the question whether 'musical'
and 'lettered' are different or the same, and whether 'musical Coriscus'
and 'Coriscus' are the same, and whether 'everything which is, but
is not eternal, has come to be', with the paradoxical conclusion that
if one who was musical has come to be lettered, he must also have
been lettered and have come to be musical, and all the other arguments
of this sort; the accidental is obviously akin to non-being. And this
is clear also from arguments such as the following: things which are
in another sense come into being and pass out of being by a process,
but things which are accidentally do not. But still we must, as far
as we can, say further, regarding the accidental, what its nature
is and from what cause it proceeds; for it will perhaps at the same
time become clear why there is no science of it. 

"Since, among things which are, some are always in the same state
and are of necessity (not necessity in the sense of compulsion but
that which we assert of things because they cannot be otherwise),
and some are not of necessity nor always, but for the most part, this
is the principle and this the cause of the existence of the accidental;
for that which is neither always nor for the most part, we call accidental.
For instance, if in the dog-days there is wintry and cold weather,
we say this is an accident, but not if there is sultry heat, because
the latter is always or for the most part so, but not the former.
And it is an accident that a man is pale (for this is neither always
nor for the most part so), but it is not by accident that he is an
animal. And that the builder produces health is an accident, because
it is the nature not of the builder but of the doctor to do this,-but
the builder happened to be a doctor. Again, a confectioner, aiming
at giving pleasure, may make something wholesome, but not in virtue
of the confectioner's art; and therefore we say 'it was an accident',
and while there is a sense in which he makes it, in the unqualified
sense he does not. For to other things answer faculties productive
of them, but to accidental results there corresponds no determinate
art nor faculty; for of things which are or come to be by accident,
the cause also is accidental. Therefore, since not all things either
are or come to be of necessity and always, but, the majority of things
are for the most part, the accidental must exist; for instance a pale
man is not always nor for the most part musical, but since this sometimes
happens, it must be accidental (if not, everything will be of necessity).
The matter, therefore, which is capable of being otherwise than as
it usually is, must be the cause of the accidental. And we must take
as our starting-point the question whether there is nothing that is
neither always nor for the most part. Surely this is impossible. There
is, then, besides these something which is fortuitous and accidental.
But while the usual exists, can nothing be said to be always, or are
there eternal things? This must be considered later,' but that there
is no science of the accidental is obvious; for all science is either
of that which is always or of that which is for the most part. (For
how else is one to learn or to teach another? The thing must be determined
as occurring either always or for the most part, e.g. that honey-water
is useful for a patient in a fever is true for the most part.) But
that which is contrary to the usual law science will be unable to
state, i.e. when the thing does not happen, e.g.'on the day of new
moon'; for even that which happens on the day of new moon happens
then either always or for the most part; but the accidental is contrary
to such laws. We have stated, then, what the accidental is, and from
what cause it arises, and that there is no science which deals with
it. 

Part 3 "

"That there are principles and causes which are generable and destructible
without ever being in course of being generated or destroyed, is obvious.
For otherwise all things will be of necessity, since that which is
being generated or destroyed must have a cause which is not accidentally
its cause. Will A exist or not? It will if B happens; and if not,
not. And B will exist if C happens. And thus if time is constantly
subtracted from a limited extent of time, one will obviously come
to the present. This man, then, will die by violence, if he goes out;
and he will do this if he gets thirsty; and he will get thirsty if
something else happens; and thus we shall come to that which is now
present, or to some past event. For instance, he will go out if he
gets thirsty; and he will get thirsty if he is eating pungent food;
and this is either the case or not; so that he will of necessity die,
or of necessity not die. And similarly if one jumps over to past events,
the same account will hold good; for this-I mean the past condition-is
already present in something. Everything, therefore, that will be,
will be of necessity; e.g. it is necessary that he who lives shall
one day die; for already some condition has come into existence, e.g.
the presence of contraries in the same body. But whether he is to
die by disease or by violence is not yet determined, but depends on
the happening of something else. Clearly then the process goes back
to a certain starting-point, but this no longer points to something
further. This then will be the starting-point for the fortuitous,
and will have nothing else as cause of its coming to be. But to what
sort of starting-point and what sort of cause we thus refer the fortuitous-whether
to matter or to the purpose or to the motive power, must be carefully
considered. 

Part 4 "

"Let us dismiss accidental being; for we have sufficiently determined
its nature. But since that which is in the sense of being true, or
is not in the sense of being false, depends on combination and separation,
and truth and falsity together depend on the allocation of a pair
of contradictory judgements (for the true judgement affirms where
the subject and predicate really are combined, and denies where they
are separated, while the false judgement has the opposite of this
allocation; it is another question, how it happens that we think things
together or apart; by 'together' and 'apart' I mean thinking them
so that there is no succession in the thoughts but they become a unity);
for falsity and truth are not in things-it is not as if the good were
true, and the bad were in itself false-but in thought; while with
regard to simple concepts and 'whats' falsity and truth do not exist
even in thought--this being so, we must consider later what has to
be discussed with regard to that which is or is not in this sense.
But since the combination and the separation are in thought and not
in the things, and that which is in this sense is a different sort
of 'being' from the things that are in the full sense (for the thought
attaches or removes either the subject's 'what' or its having a certain
quality or quantity or something else), that which is accidentally
and that which is in the sense of being true must be dismissed. For
the cause of the former is indeterminate, and that of the latter is
some affection of the thought, and both are related to the remaining
genus of being, and do not indicate the existence of any separate
class of being. Therefore let these be dismissed, and let us consider
the causes and the principles of being itself, qua being. (It was
clear in our discussion of the various meanings of terms, that 'being'
has several meanings.) 

----------------------------------------------------------------------

BOOK VII

Part 1 

"

"THERE are several senses in which a thing may be said to 'be', as
we pointed out previously in our book on the various senses of words;'
for in one sense the 'being' meant is 'what a thing is' or a 'this',
and in another sense it means a quality or quantity or one of the
other things that are predicated as these are. While 'being' has all
these senses, obviously that which 'is' primarily is the 'what', which
indicates the substance of the thing. For when we say of what quality
a thing is, we say that it is good or bad, not that it is three cubits
long or that it is a man; but when we say what it is, we do not say
'white' or 'hot' or 'three cubits long', but 'a man' or 'a 'god'.
And all other things are said to be because they are, some of them,
quantities of that which is in this primary sense, others qualities
of it, others affections of it, and others some other determination
of it. And so one might even raise the question whether the words
'to walk', 'to be healthy', 'to sit' imply that each of these things
is existent, and similarly in any other case of this sort; for none
of them is either self-subsistent or capable of being separated from
substance, but rather, if anything, it is that which walks or sits
or is healthy that is an existent thing. Now these are seen to be
more real because there is something definite which underlies them
(i.e. the substance or individual), which is implied in such a predicate;
for we never use the word 'good' or 'sitting' without implying this.
Clearly then it is in virtue of this category that each of the others
also is. Therefore that which is primarily, i.e. not in a qualified
sense but without qualification, must be substance. 

"Now there are several senses in which a thing is said to be first;
yet substance is first in every sense-(1) in definition, (2) in order
of knowledge, (3) in time. For (3) of the other categories none can
exist independently, but only substance. And (1) in definition also
this is first; for in the definition of each term the definition of
its substance must be present. And (2) we think we know each thing
most fully, when we know what it is, e.g. what man is or what fire
is, rather than when we know its quality, its quantity, or its place;
since we know each of these predicates also, only when we know what
the quantity or the quality is. 

"And indeed the question which was raised of old and is raised now
and always, and is always the subject of doubt, viz. what being is,
is just the question, what is substance? For it is this that some
assert to be one, others more than one, and that some assert to be
limited in number, others unlimited. And so we also must consider
chiefly and primarily and almost exclusively what that is which is
in this sense. 

Part 2 "

"Substance is thought to belong most obviously to bodies; and so we
say that not only animals and plants and their parts are substances,
but also natural bodies such as fire and water and earth and everything
of the sort, and all things that are either parts of these or composed
of these (either of parts or of the whole bodies), e.g. the physical
universe and its parts, stars and moon and sun. But whether these
alone are substances, or there are also others, or only some of these,
or others as well, or none of these but only some other things, are
substances, must be considered. Some think the limits of body, i.e.
surface, line, point, and unit, are substances, and more so than body
or the solid. 

"Further, some do not think there is anything substantial besides
sensible things, but others think there are eternal substances which
are more in number and more real; e.g. Plato posited two kinds of
substance-the Forms and objects of mathematics-as well as a third
kind, viz. the substance of sensible bodies. And Speusippus made still
more kinds of substance, beginning with the One, and assuming principles
for each kind of substance, one for numbers, another for spatial magnitudes,
and then another for the soul; and by going on in this way he multiplies
the kinds of substance. And some say Forms and numbers have the same
nature, and the other things come after them-lines and planes-until
we come to the substance of the material universe and to sensible
bodies. 

"Regarding these matters, then, we must inquire which of the common
statements are right and which are not right, and what substances
there are, and whether there are or are not any besides sensible substances,
and how sensible substances exist, and whether there is a substance
capable of separate existence (and if so why and how) or no such substance,
apart from sensible substances; and we must first sketch the nature
of substance. 

Part 3 "

"The word 'substance' is applied, if not in more senses, still at
least to four main objects; for both the essence and the universal
and the genus, are thought to be the substance of each thing, and
fourthly the substratum. Now the substratum is that of which everything
else is predicated, while it is itself not predicated of anything
else. And so we must first determine the nature of this; for that
which underlies a thing primarily is thought to be in the truest sense
its substance. And in one sense matter is said to be of the nature
of substratum, in another, shape, and in a third, the compound of
these. (By the matter I mean, for instance, the bronze, by the shape
the pattern of its form, and by the compound of these the statue,
the concrete whole.) Therefore if the form is prior to the matter
and more real, it will be prior also to the compound of both, for
the same reason. 

"We have now outlined the nature of substance, showing that it is
that which is not predicated of a stratum, but of which all else is
predicated. But we must not merely state the matter thus; for this
is not enough. The statement itself is obscure, and further, on this
view, matter becomes substance. For if this is not substance, it baffles
us to say what else is. When all else is stripped off evidently nothing
but matter remains. For while the rest are affections, products, and
potencies of bodies, length, breadth, and depth are quantities and
not substances (for a quantity is not a substance), but the substance
is rather that to which these belong primarily. But when length and
breadth and depth are taken away we see nothing left unless there
is something that is bounded by these; so that to those who consider
the question thus matter alone must seem to be substance. By matter
I mean that which in itself is neither a particular thing nor of a
certain quantity nor assigned to any other of the categories by which
being is determined. For there is something of which each of these
is predicated, whose being is different from that of each of the predicates
(for the predicates other than substance are predicated of substance,
while substance is predicated of matter). Therefore the ultimate substratum
is of itself neither a particular thing nor of a particular quantity
nor otherwise positively characterized; nor yet is it the negations
of these, for negations also will belong to it only by accident.

"If we adopt this point of view, then, it follows that matter is substance.
But this is impossible; for both separability and 'thisness' are thought
to belong chiefly to substance. And so form and the compound of form
and matter would be thought to be substance, rather than matter. The
substance compounded of both, i.e. of matter and shape, may be dismissed;
for it is posterior and its nature is obvious. And matter also is
in a sense manifest. But we must inquire into the third kind of substance;
for this is the most perplexing. 

"Some of the sensible substances are generally admitted to be substances,
so that we must look first among these. For it is an advantage to
advance to that which is more knowable. For learning proceeds for
all in this way-through that which is less knowable by nature to that
which is more knowable; and just as in conduct our task is to start
from what is good for each and make what is without qualification
good good for each, so it is our task to start from what is more knowable
to oneself and make what is knowable by nature knowable to oneself.
Now what is knowable and primary for particular sets of people is
often knowable to a very small extent, and has little or nothing of
reality. But yet one must start from that which is barely knowable
but knowable to oneself, and try to know what is knowable without
qualification, passing, as has been said, by way of those very things
which one does know. 

Part 4 "

"Since at the start we distinguished the various marks by which we
determine substance, and one of these was thought to be the essence,
we must investigate this. And first let us make some linguistic remarks
about it. The essence of each thing is what it is said to be propter
se. For being you is not being musical, since you are not by your
very nature musical. What, then, you are by your very nature is your
essence. 

"Nor yet is the whole of this the essence of a thing; not that which
is propter se as white is to a surface, because being a surface is
not identical with being white. But again the combination of both-'being
a white surface'-is not the essence of surface, because 'surface'
itself is added. The formula, therefore, in which the term itself
is not present but its meaning is expressed, this is the formula of
the essence of each thing. Therefore if to be a white surface is to
be a smooth surface, to be white and to be smooth are one and the
same. 

"But since there are also compounds answering to the other categories
(for there is a substratum for each category, e.g. for quality, quantity,
time, place, and motion), we must inquire whether there is a formula
of the essence of each of them, i.e. whether to these compounds also
there belongs an essence, e.g. 'white man'. Let the compound be denoted
by 'cloak'. What is the essence of cloak? But, it may be said, this
also is not a propter se expression. We reply that there are just
two ways in which a predicate may fail to be true of a subject propter
se, and one of these results from the addition, and the other from
the omission, of a determinant. One kind of predicate is not propter
se because the term that is being defined is combined with another
determinant, e.g. if in defining the essence of white one were to
state the formula of white man; the other because in the subject another
determinant is combined with that which is expressed in the formula,
e.g. if 'cloak' meant 'white man', and one were to define cloak as
white; white man is white indeed, but its essence is not to be white.

"But is being-a-cloak an essence at all? Probably not. For the essence
is precisely what something is; but when an attribute is asserted
of a subject other than itself, the complex is not precisely what
some 'this' is, e.g. white man is not precisely what some 'this' is,
since thisness belongs only to substances. Therefore there is an essence
only of those things whose formula is a definition. But we have a
definition not where we have a word and a formula identical in meaning
(for in that case all formulae or sets of words would be definitions;
for there will be some name for any set of words whatever, so that
even the Iliad will be a definition), but where there is a formula
of something primary; and primary things are those which do not imply
the predication of one element in them of another element. Nothing,
then, which is not a species of a genus will have an essence-only
species will have it, for these are thought to imply not merely that
the subject participates in the attribute and has it as an affection,
or has it by accident; but for ever thing else as well, if it has
a name, there be a formula of its meaning-viz. that this attribute
belongs to this subject; or instead of a simple formula we shall be
able to give a more accurate one; but there will be no definition
nor essence. 

"Or has 'definition', like 'what a thing is', several meanings? 'What
a thing is' in one sense means substance and the 'this', in another
one or other of the predicates, quantity, quality, and the like. For
as 'is' belongs to all things, not however in the same sense, but
to one sort of thing primarily and to others in a secondary way, so
too 'what a thing is' belongs in the simple sense to substance, but
in a limited sense to the other categories. For even of a quality
we might ask what it is, so that quality also is a 'what a thing is',-not
in the simple sense, however, but just as, in the case of that which
is not, some say, emphasizing the linguistic form, that that is which
is not is-not is simply, but is non-existent; so too with quality.

"We must no doubt inquire how we should express ourselves on each
point, but certainly not more than how the facts actually stand. And
so now also, since it is evident what language we use, essence will
belong, just as 'what a thing is' does, primarily and in the simple
sense to substance, and in a secondary way to the other categories
also,-not essence in the simple sense, but the essence of a quality
or of a quantity. For it must be either by an equivocation that we
say these are, or by adding to and taking from the meaning of 'are'
(in the way in which that which is not known may be said to be known),-the
truth being that we use the word neither ambiguously nor in the same
sense, but just as we apply the word 'medical' by virtue of a reference
to one and the same thing, not meaning one and the same thing, nor
yet speaking ambiguously; for a patient and an operation and an instrument
are called medical neither by an ambiguity nor with a single meaning,
but with reference to a common end. But it does not matter at all
in which of the two ways one likes to describe the facts; this is
evident, that definition and essence in the primary and simple sense
belong to substances. Still they belong to other things as well, only
not in the primary sense. For if we suppose this it does not follow
that there is a definition of every word which means the same as any
formula; it must mean the same as a particular kind of formula; and
this condition is satisfied if it is a formula of something which
is one, not by continuity like the Iliad or the things that are one
by being bound together, but in one of the main senses of 'one', which
answer to the senses of 'is'; now 'that which is' in one sense denotes
a 'this', in another a quantity, in another a quality. And so there
can be a formula or definition even of white man, but not in the sense
in which there is a definition either of white or of a substance.

Part 5 "

"It is a difficult question, if one denies that a formula with an
added determinant is a definition, whether any of the terms that are
not simple but coupled will be definable. For we must explain them
by adding a determinant. E.g. there is the nose, and concavity, and
snubness, which is compounded out of the two by the presence of the
one in the other, and it is not by accident that the nose has the
attribute either of concavity or of snubness, but in virtue of its
nature; nor do they attach to it as whiteness does to Callias, or
to man (because Callias, who happens to be a man, is white), but as
'male' attaches to animal and 'equal' to quantity, and as all so-called
'attributes propter se' attach to their subjects. And such attributes
are those in which is involved either the formula or the name of the
subject of the particular attribute, and which cannot be explained
without this; e.g. white can be explained apart from man, but not
female apart from animal. Therefore there is either no essence and
definition of any of these things, or if there is, it is in another
sense, as we have said. 

"But there is also a second difficulty about them. For if snub nose
and concave nose are the same thing, snub and concave will be the
thing; but if snub and concave are not the same (because it is impossible
to speak of snubness apart from the thing of which it is an attribute
propter se, for snubness is concavity-in-a-nose), either it is impossible
to say 'snub nose' or the same thing will have been said twice, concave-nose
nose; for snub nose will be concave-nose nose. And so it is absurd
that such things should have an essence; if they have, there will
be an infinite regress; for in snub-nose nose yet another 'nose' will
be involved. 

"Clearly, then, only substance is definable. For if the other categories
also are definable, it must be by addition of a determinant, e.g.
the qualitative is defined thus, and so is the odd, for it cannot
be defined apart from number; nor can female be defined apart from
animal. (When I say 'by addition' I mean the expressions in which
it turns out that we are saying the same thing twice, as in these
instances.) And if this is true, coupled terms also, like 'odd number',
will not be definable (but this escapes our notice because our formulae
are not accurate.). But if these also are definable, either it is
in some other way or, as we definition and essence must be said to
have more than one sense. Therefore in one sense nothing will have
a definition and nothing will have an essence, except substances,
but in another sense other things will have them. Clearly, then, definition
is the formula of the essence, and essence belongs to substances either
alone or chiefly and primarily and in the unqualified sense.

Part 6 "

"We must inquire whether each thing and its essence are the same or
different. This is of some use for the inquiry concerning substance;
for each thing is thought to be not different from its substance,
and the essence is said to be the substance of each thing.

"Now in the case of accidental unities the two would be generally
thought to be different, e.g. white man would be thought to be different
from the essence of white man. For if they are the same, the essence
of man and that of white man are also the same; for a man and a white
man are the same thing, as people say, so that the essence of white
man and that of man would be also the same. But perhaps it does not
follow that the essence of accidental unities should be the same as
that of the simple terms. For the extreme terms are not in the same
way identical with the middle term. But perhaps this might be thought
to follow, that the extreme terms, the accidents, should turn out
to be the same, e.g. the essence of white and that of musical; but
this is not actually thought to be the case. 

"But in the case of so-called self-subsistent things, is a thing necessarily
the same as its essence? E.g. if there are some substances which have
no other substances nor entities prior to them-substances such as
some assert the Ideas to be?-If the essence of good is to be different
from good-itself, and the essence of animal from animal-itself, and
the essence of being from being-itself, there will, firstly, be other
substances and entities and Ideas besides those which are asserted,
and, secondly, these others will be prior substances, if essence is
substance. And if the posterior substances and the prior are severed
from each other, (a) there will be no knowledge of the former, and
(b) the latter will have no being. (By 'severed' I mean, if the good-itself
has not the essence of good, and the latter has not the property of
being good.) For (a) there is knowledge of each thing only when we
know its essence. And (b) the case is the same for other things as
for the good; so that if the essence of good is not good, neither
is the essence of reality real, nor the essence of unity one. And
all essences alike exist or none of them does; so that if the essence
of reality is not real, neither is any of the others. Again, that
to which the essence of good does not belong is not good.-The good,
then, must be one with the essence of good, and the beautiful with
the essence of beauty, and so with all things which do not depend
on something else but are self-subsistent and primary. For it is enough
if they are this, even if they are not Forms; or rather, perhaps,
even if they are Forms. (At the same time it is clear that if there
are Ideas such as some people say there are, it will not be substratum
that is substance; for these must be substances, but not predicable
of a substratum; for if they were they would exist only by being participated
in.) 

"Each thing itself, then, and its essence are one and the same in
no merely accidental way, as is evident both from the preceding arguments
and because to know each thing, at least, is just to know its essence,
so that even by the exhibition of instances it becomes clear that
both must be one. 

"(But of an accidental term, e.g.'the musical' or 'the white', since
it has two meanings, it is not true to say that it itself is identical
with its essence; for both that to which the accidental quality belongs,
and the accidental quality, are white, so that in a sense the accident
and its essence are the same, and in a sense they are not; for the
essence of white is not the same as the man or the white man, but
it is the same as the attribute white.) 

"The absurdity of the separation would appear also if one were to
assign a name to each of the essences; for there would be yet another
essence besides the original one, e.g. to the essence of horse there
will belong a second essence. Yet why should not some things be their
essences from the start, since essence is substance? But indeed not
only are a thing and its essence one, but the formula of them is also
the same, as is clear even from what has been said; for it is not
by accident that the essence of one, and the one, are one. Further,
if they are to be different, the process will go on to infinity; for
we shall have (1) the essence of one, and (2) the one, so that to
terms of the former kind the same argument will be applicable.

"Clearly, then, each primary and self-subsistent thing is one and
the same as its essence. The sophistical objections to this position,
and the question whether Socrates and to be Socrates are the same
thing, are obviously answered by the same solution; for there is no
difference either in the standpoint from which the question would
be asked, or in that from which one could answer it successfully.
We have explained, then, in what sense each thing is the same as its
essence and in what sense it is not. 

Part 7 "

"Of things that come to be, some come to be by nature, some by art,
some spontaneously. Now everything that comes to be comes to be by
the agency of something and from something and comes to be something.
And the something which I say it comes to be may be found in any category;
it may come to be either a 'this' or of some size or of some quality
or somewhere. 

"Now natural comings to be are the comings to be of those things which
come to be by nature; and that out of which they come to be is what
we call matter; and that by which they come to be is something which
exists naturally; and the something which they come to be is a man
or a plant or one of the things of this kind, which we say are substances
if anything is-all things produced either by nature or by art have
matter; for each of them is capable both of being and of not being,
and this capacity is the matter in each-and, in general, both that
from which they are produced is nature, and the type according to
which they are produced is nature (for that which is produced, e.g.
a plant or an animal, has a nature), and so is that by which they
are produced--the so-called 'formal' nature, which is specifically
the same (though this is in another individual); for man begets man.

"Thus, then, are natural products produced; all other productions
are called 'makings'. And all makings proceed either from art or from
a faculty or from thought. Some of them happen also spontaneously
or by luck just as natural products sometimes do; for there also the
same things sometimes are produced without seed as well as from seed.
Concerning these cases, then, we must inquire later, but from art
proceed the things of which the form is in the soul of the artist.
(By form I mean the essence of each thing and its primary substance.)
For even contraries have in a sense the same form; for the substance
of a privation is the opposite substance, e.g. health is the substance
of disease (for disease is the absence of health); and health is the
formula in the soul or the knowledge of it. The healthy subject is
produced as the result of the following train of thought:-since this
is health, if the subject is to be healthy this must first be present,
e.g. a uniform state of body, and if this is to be present, there
must be heat; and the physician goes on thinking thus until he reduces
the matter to a final something which he himself can produce. Then
the process from this point onward, i.e. the process towards health,
is called a 'making'. Therefore it follows that in a sense health
comes from health and house from house, that with matter from that
without matter; for the medical art and the building art are the form
of health and of the house, and when I speak of substance without
matter I mean the essence. 

"Of the productions or processes one part is called thinking and the
other making,-that which proceeds from the starting-point and the
form is thinking, and that which proceeds from the final step of the
thinking is making. And each of the other, intermediate, things is
produced in the same way. I mean, for instance, if the subject is
to be healthy his bodily state must be made uniform. What then does
being made uniform imply? This or that. And this depends on his being
made warm. What does this imply? Something else. And this something
is present potentially; and what is present potentially is already
in the physician's power. 

"The active principle then and the starting point for the process
of becoming healthy is, if it happens by art, the form in the soul,
and if spontaneously, it is that, whatever it is, which starts the
making, for the man who makes by art, as in healing the starting-point
is perhaps the production of warmth (and this the physician produces
by rubbing). Warmth in the body, then, is either a part of health
or is followed (either directly or through several intermediate steps)
by something similar which is a part of health; and this, viz. that
which produces the part of health, is the limiting-point--and so too
with a house (the stones are the limiting-point here) and in all other
cases. Therefore, as the saying goes, it is impossible that anything
should be produced if there were nothing existing before. Obviously
then some part of the result will pre-exist of necessity; for the
matter is a part; for this is present in the process and it is this
that becomes something. But is the matter an element even in the formula?
We certainly describe in both ways what brazen circles are; we describe
both the matter by saying it is brass, and the form by saying that
it is such and such a figure; and figure is the proximate genus in
which it is placed. The brazen circle, then, has its matter in its
formula. 

"As for that out of which as matter they are produced, some things
are said, when they have been produced, to be not that but 'thaten';
e.g. the statue is not gold but golden. And a healthy man is not said
to be that from which he has come. The reason is that though a thing
comes both from its privation and from its substratum, which we call
its matter (e.g. what becomes healthy is both a man and an invalid),
it is said to come rather from its privation (e.g. it is from an invalid
rather than from a man that a healthy subject is produced). And so
the healthy subject is not said to he an invalid, but to be a man,
and the man is said to be healthy. But as for the things whose privation
is obscure and nameless, e.g. in brass the privation of a particular
shape or in bricks and timber the privation of arrangement as a house,
the thing is thought to be produced from these materials, as in the
former case the healthy man is produced from an invalid. And so, as
there also a thing is not said to be that from which it comes, here
the statue is not said to be wood but is said by a verbal change to
be wooden, not brass but brazen, not gold but golden, and the house
is said to be not bricks but bricken (though we should not say without
qualification, if we looked at the matter carefully, even that a statue
is produced from wood or a house from bricks, because coming to be
implies change in that from which a thing comes to be, and not permanence).
It is for this reason, then, that we use this way of speaking.

Part 8 "

"Since anything which is produced is produced by something (and this
I call the starting-point of the production), and from something (and
let this be taken to be not the privation but the matter; for the
meaning we attach to this has already been explained), and since something
is produced (and this is either a sphere or a circle or whatever else
it may chance to be), just as we do not make the substratum (the brass),
so we do not make the sphere, except incidentally, because the brazen
sphere is a sphere and we make the forme. For to make a 'this' is
to make a 'this' out of the substratum in the full sense of the word.
(I mean that to make the brass round is not to make the round or the
sphere, but something else, i.e. to produce this form in something
different from itself. For if we make the form, we must make it out
of something else; for this was assumed. E.g. we make a brazen sphere;
and that in the sense that out of this, which is brass, we make this
other, which is a sphere.) If, then, we also make the substratum itself,
clearly we shall make it in the same way, and the processes of making
will regress to infinity. Obviously then the form also, or whatever
we ought to call the shape present in the sensible thing, is not produced,
nor is there any production of it, nor is the essence produced; for
this is that which is made to be in something else either by art or
by nature or by some faculty. But that there is a brazen sphere, this
we make. For we make it out of brass and the sphere; we bring the
form into this particular matter, and the result is a brazen sphere.
But if the essence of sphere in general is to be produced, something
must be produced out of something. For the product will always have
to be divisible, and one part must be this and another that; I mean
the one must be matter and the other form. If, then, a sphere is 'the
figure whose circumference is at all points equidistant from the centre',
part of this will be the medium in which the thing made will be, and
part will be in that medium, and the whole will be the thing produced,
which corresponds to the brazen sphere. It is obvious, then, from
what has been said, that that which is spoken of as form or substance
is not produced, but the concrete thing which gets its name from this
is produced, and that in everything which is generated matter is present,
and one part of the thing is matter and the other form. 

"Is there, then, a sphere apart from the individual spheres or a house
apart from the bricks? Rather we may say that no 'this' would ever
have been coming to be, if this had been so, but that the 'form' means
the 'such', and is not a 'this'-a definite thing; but the artist makes,
or the father begets, a 'such' out of a 'this'; and when it has been
begotten, it is a 'this such'. And the whole 'this', Callias or Socrates,
is analogous to 'this brazen sphere', but man and animal to 'brazen
sphere' in general. Obviously, then, the cause which consists of the
Forms (taken in the sense in which some maintain the existence of
the Forms, i.e. if they are something apart from the individuals)
is useless, at least with regard to comings-to-be and to substances;
and the Forms need not, for this reason at least, be self-subsistent
substances. In some cases indeed it is even obvious that the begetter
is of the same kind as the begotten (not, however, the same nor one
in number, but in form), i.e. in the case of natural products (for
man begets man), unless something happens contrary to nature, e.g.
the production of a mule by a horse. (And even these cases are similar;
for that which would be found to be common to horse and ass, the genus
next above them, has not received a name, but it would doubtless be
both in fact something like a mule.) Obviously, therefore, it is quite
unnecessary to set up a Form as a pattern (for we should have looked
for Forms in these cases if in any; for these are substances if anything
is so); the begetter is adequate to the making of the product and
to the causing of the form in the matter. And when we have the whole,
such and such a form in this flesh and in these bones, this is Callias
or Socrates; and they are different in virtue of their matter (for
that is different), but the same in form; for their form is indivisible.

Part 9 "

"The question might be raised, why some things are produced spontaneously
as well as by art, e.g. health, while others are not, e.g. a house.
The reason is that in some cases the matter which governs the production
in the making and producing of any work of art, and in which a part
of the product is present,-some matter is such as to be set in motion
by itself and some is not of this nature, and of the former kind some
can move itself in the particular way required, while other matter
is incapable of this; for many things can be set in motion by themselves
but not in some particular way, e.g. that of dancing. The things,
then, whose matter is of this sort, e.g. stones, cannot be moved in
the particular way required, except by something else, but in another
way they can move themselves-and so it is with fire. Therefore some
things will not exist apart from some one who has the art of making
them, while others will; for motion will be started by these things
which have not the art but can themselves be moved by other things
which have not the art or with a motion starting from a part of the
product. 

"And it is clear also from what has been said that in a sense every
product of art is produced from a thing which shares its name (as
natural products are produced), or from a part of itself which shares
its name (e.g. the house is produced from a house, qua produced by
reason; for the art of building is the form of the house), or from
something which contains a art of it,-if we exclude things produced
by accident; for the cause of the thing's producing the product directly
per se is a part of the product. The heat in the movement caused heat
in the body, and this is either health, or a part of health, or is
followed by a part of health or by health itself. And so it is said
to cause health, because it causes that to which health attaches as
a consequence. 

"Therefore, as in syllogisms, substance is the starting-point of everything.
It is from 'what a thing is' that syllogisms start; and from it also
we now find processes of production to start. 

"Things which are formed by nature are in the same case as these products
of art. For the seed is productive in the same way as the things that
work by art; for it has the form potentially, and that from which
the seed comes has in a sense the same name as the offspring only
in a sense, for we must not expect parent and offspring always to
have exactly the same name, as in the production of 'human being'
from 'human' for a 'woman' also can be produced by a 'man'-unless
the offspring be an imperfect form; which is the reason why the parent
of a mule is not a mule. The natural things which (like the artificial
objects previously considered) can be produced spontaneously are those
whose matter can be moved even by itself in the way in which the seed
usually moves it; those things which have not such matter cannot be
produced except from the parent animals themselves. 

"But not only regarding substance does our argument prove that its
form does not come to be, but the argument applies to all the primary
classes alike, i.e. quantity, quality, and the other categories. For
as the brazen sphere comes to be, but not the sphere nor the brass,
and so too in the case of brass itself, if it comes to be, it is its
concrete unity that comes to be (for the matter and the form must
always exist before), so is it both in the case of substance and in
that of quality and quantity and the other categories likewise; for
the quality does not come to be, but the wood of that quality, and
the quantity does not come to be, but the wood or the animal of that
size. But we may learn from these instances a peculiarity of substance,
that there must exist beforehand in complete reality another substance
which produces it, e.g. an animal if an animal is produced; but it
is not necessary that a quality or quantity should pre-exist otherwise
than potentially. 

Part 10 "

"Since a definition is a formula, and every formula has parts, and
as the formula is to the thing, so is the part of the formula to the
part of the thing, the question is already being asked whether the
formula of the parts must be present in the formula of the whole or
not. For in some cases the formulae of the parts are seen to be present,
and in some not. The formula of the circle does not include that of
the segments, but that of the syllable includes that of the letters;
yet the circle is divided into segments as the syllable is into letters.-And
further if the parts are prior to the whole, and the acute angle is
a part of the right angle and the finger a part of the animal, the
acute angle will be prior to the right angle and finger to the man.
But the latter are thought to be prior; for in formula the parts are
explained by reference to them, and in respect also of the power of
existing apart from each other the wholes are prior to the parts.

"Perhaps we should rather say that 'part' is used in several senses.
One of these is 'that which measures another thing in respect of quantity'.
But let this sense be set aside; let us inquire about the parts of
which substance consists. If then matter is one thing, form another,
the compound of these a third, and both the matter and the form and
the compound are substance even the matter is in a sense called part
of a thing, while in a sense it is not, but only the elements of which
the formula of the form consists. E.g. of concavity flesh (for this
is the matter in which it is produced) is not a part, but of snubness
it is a part; and the bronze is a part of the concrete statue, but
not of the statue when this is spoken of in the sense of the form.
(For the form, or the thing as having form, should be said to be the
thing, but the material element by itself must never be said to be
so.) And so the formula of the circle does not include that of the
segments, but the formula of the syllable includes that of the letters;
for the letters are parts of the formula of the form, and not matter,
but the segments are parts in the sense of matter on which the form
supervenes; yet they are nearer the form than the bronze is when roundness
is produced in bronze. But in a sense not even every kind of letter
will be present in the formula of the syllable, e.g. particular waxen
letters or the letters as movements in the air; for in these also
we have already something that is part of the syllable only in the
sense that it is its perceptible matter. For even if the line when
divided passes away into its halves, or the man into bones and muscles
and flesh, it does not follow that they are composed of these as parts
of their essence, but rather as matter; and these are parts of the
concrete thing, but not also of the form, i.e. of that to which the
formula refers; wherefore also they are not present in the formulae.
In one kind of formula, then, the formula of such parts will be present,
but in another it must not be present, where the formula does not
refer to the concrete object. For it is for this reason that some
things have as their constituent principles parts into which they
pass away, while some have not. Those things which are the form and
the matter taken together, e.g. the snub, or the bronze circle, pass
away into these materials, and the matter is a part of them; but those
things which do not involve matter but are without matter, and whose
formulae are formulae of the form only, do not pass away,-either not
at all or at any rate not in this way. Therefore these materials are
principles and parts of the concrete things, while of the form they
are neither parts nor principles. And therefore the clay statue is
resolved into clay and the ball into bronze and Callias into flesh
and bones, and again the circle into its segments; for there is a
sense of 'circle' in which involves matter. For 'circle' is used ambiguously,
meaning both the circle, unqualified, and the individual circle, because
there is no name peculiar to the individuals. 

"The truth has indeed now been stated, but still let us state it yet
more clearly, taking up the question again. The parts of the formula,
into which the formula is divided, are prior to it, either all or
some of them. The formula of the right angle, however, does not include
the formula of the acute, but the formula of the acute includes that
of the right angle; for he who defines the acute uses the right angle;
for the acute is 'less than a right angle'. The circle and the semicircle
also are in a like relation; for the semicircle is defined by the
circle; and so is the finger by the whole body, for a finger is 'such
and such a part of a man'. Therefore the parts which are of the nature
of matter, and into which as its matter a thing is divided, are posterior;
but those which are of the nature of parts of the formula, and of
the substance according to its formula, are prior, either all or some
of them. And since the soul of animals (for this is the substance
of a living being) is their substance according to the formula, i.e.
the form and the essence of a body of a certain kind (at least we
shall define each part, if we define it well, not without reference
to its function, and this cannot belong to it without perception),
so that the parts of soul are prior, either all or some of them, to
the concrete 'animal', and so too with each individual animal; and
the body and parts are posterior to this, the essential substance,
and it is not the substance but the concrete thing that is divided
into these parts as its matter:-this being so, to the concrete thing
these are in a sense prior, but in a sense they are not. For they
cannot even exist if severed from the whole; for it is not a finger
in any and every state that is the finger of a living thing, but a
dead finger is a finger only in name. Some parts are neither prior
nor posterior to the whole, i.e. those which are dominant and in which
the formula, i.e. the essential substance, is immediately present,
e.g. perhaps the heart or the brain; for it does not matter in the
least which of the two has this quality. But man and horse and terms
which are thus applied to individuals, but universally, are not substance
but something composed of this particular formula and this particular
matter treated as universal; and as regards the individual, Socrates
already includes in him ultimate individual matter; and similarly
in all other cases. 'A part' may be a part either of the form (i.e.
of the essence), or of the compound of the form and the matter, or
of the matter itself. But only the parts of the form are parts of
the formula, and the formula is of the universal; for 'being a circle'
is the same as the circle, and 'being a soul' the same as the soul.
But when we come to the concrete thing, e.g. this circle, i.e. one
of the individual circles, whether perceptible or intelligible (I
mean by intelligible circles the mathematical, and by perceptible
circles those of bronze and of wood),-of these there is no definition,
but they are known by the aid of intuitive thinking or of perception;
and when they pass out of this complete realization it is not clear
whether they exist or not; but they are always stated and recognized
by means of the universal formula. But matter is unknowable in itself.
And some matter is perceptible and some intelligible, perceptible
matter being for instance bronze and wood and all matter that is changeable,
and intelligible matter being that which is present in perceptible
things not qua perceptible, i.e. the objects of mathematics.

"We have stated, then, how matters stand with regard to whole and
part, and their priority and posteriority. But when any one asks whether
the right angle and the circle and the animal are prior, or the things
into which they are divided and of which they consist, i.e. the parts,
we must meet the inquiry by saying that the question cannot be answered
simply. For if even bare soul is the animal or the living thing, or
the soul of each individual is the individual itself, and 'being a
circle' is the circle, and 'being a right angle' and the essence of
the right angle is the right angle, then the whole in one sense must
be called posterior to the art in one sense, i.e. to the parts included
in the formula and to the parts of the individual right angle (for
both the material right angle which is made of bronze, and that which
is formed by individual lines, are posterior to their parts); while
the immaterial right angle is posterior to the parts included in the
formula, but prior to those included in the particular instance, and
the question must not be answered simply. If, however, the soul is
something different and is not identical with the animal, even so
some parts must, as we have maintained, be called prior and others
must not. 

Part 11 "

"Another question is naturally raised, viz. what sort of parts belong
to the form and what sort not to the form, but to the concrete thing.
Yet if this is not plain it is not possible to define any thing; for
definition is of the universal and of the form. If then it is not
evident what sort of parts are of the nature of matter and what sort
are not, neither will the formula of the thing be evident. In the
case of things which are found to occur in specifically different
materials, as a circle may exist in bronze or stone or wood, it seems
plain that these, the bronze or the stone, are no part of the essence
of the circle, since it is found apart from them. Of things which
are not seen to exist apart, there is no reason why the same may not
be true, just as if all circles that had ever been seen were of bronze;
for none the less the bronze would be no part of the form; but it
is hard to eliminate it in thought. E.g. the form of man is always
found in flesh and bones and parts of this kind; are these then also
parts of the form and the formula? No, they are matter; but because
man is not found also in other matters we are unable to perform the
abstraction. 

"Since this is thought to be possible, but it is not clear when it
is the case, some people already raise the question even in the case
of the circle and the triangle, thinking that it is not right to define
these by reference to lines and to the continuous, but that all these
are to the circle or the triangle as flesh and bones are to man, and
bronze or stone to the statue; and they reduce all things to numbers,
and they say the formula of 'line' is that of 'two'. And of those
who assert the Ideas some make 'two' the line-itself, and others make
it the Form of the line; for in some cases they say the Form and that
of which it is the Form are the same, e.g. 'two' and the Form of two;
but in the case of 'line' they say this is no longer so.

"It follows then that there is one Form for many things whose form
is evidently different (a conclusion which confronted the Pythagoreans
also); and it is possible to make one thing the Form-itself of all,
and to hold that the others are not Forms; but thus all things will
be one. 

"We have pointed out, then, that the question of definitions contains
some difficulty, and why this is so. And so to reduce all things thus
to Forms and to eliminate the matter is useless labour; for some things
surely are a particular form in a particular matter, or particular
things in a particular state. And the comparison which Socrates the
younger used to make in the case of 'animal' is not sound; for it
leads away from the truth, and makes one suppose that man can possibly
exist without his parts, as the circle can without the bronze. But
the case is not similar; for an animal is something perceptible, and
it is not possible to define it without reference to movement-nor,
therefore, without reference to the parts' being in a certain state.
For it is not a hand in any and every state that is a part of man,
but only when it can fulfil its work, and therefore only when it is
alive; if it is not alive it is not a part. 

"Regarding the objects of mathematics, why are the formulae of the
parts not parts of the formulae of the wholes; e.g. why are not the
semicircles included in the formula of the circle? It cannot be said,
'because these parts are perceptible things'; for they are not. But
perhaps this makes no difference; for even some things which are not
perceptible must have matter; indeed there is some matter in everything
which is not an essence and a bare form but a 'this'. The semicircles,
then, will not be parts of the universal circle, but will be parts
of the individual circles, as has been said before; for while one
kind of matter is perceptible, there is another which is intelligible.

"It is clear also that the soul is the primary substance and the body
is matter, and man or animal is the compound of both taken universally;
and 'Socrates' or 'Coriscus', if even the soul of Socrates may be
called Socrates, has two meanings (for some mean by such a term the
soul, and others mean the concrete thing), but if 'Socrates' or 'Coriscus'
means simply this particular soul and this particular body, the individual
is analogous to the universal in its composition. 

"Whether there is, apart from the matter of such substances, another
kind of matter, and one should look for some substance other than
these, e.g. numbers or something of the sort, must be considered later.
For it is for the sake of this that we are trying to determine the
nature of perceptible substances as well, since in a sense the inquiry
about perceptible substances is the work of physics, i.e. of second
philosophy; for the physicist must come to know not only about the
matter, but also about the substance expressed in the formula, and
even more than about the other. And in the case of definitions, how
the elements in the formula are parts of the definition, and why the
definition is one formula (for clearly the thing is one, but in virtue
of what is the thing one, although it has parts?),-this must be considered
later. 

"What the essence is and in what sense it is independent, has been
stated universally in a way which is true of every case, and also
why the formula of the essence of some things contains the parts of
the thing defined, while that of others does not. And we have stated
that in the formula of the substance the material parts will not be
present (for they are not even parts of the substance in that sense,
but of the concrete substance; but of this there is in a sense a formula,
and in a sense there is not; for there is no formula of it with its
matter, for this is indefinite, but there is a formula of it with
reference to its primary substance-e.g. in the case of man the formula
of the soul-, for the substance is the indwelling form, from which
and the matter the so-called concrete substance is derived; e.g. concavity
is a form of this sort, for from this and the nose arise 'snub nose'
and 'snubness'); but in the concrete substance, e.g. a snub nose or
Callias, the matter also will be present. And we have stated that
the essence and the thing itself are in some cases the same; ie. in
the case of primary substances, e.g. curvature and the essence of
curvature if this is primary. (By a 'primary' substance I mean one
which does not imply the presence of something in something else,
i.e. in something that underlies it which acts as matter.) But things
which are of the nature of matter, or of wholes that include matter,
are not the same as their essences, nor are accidental unities like
that of 'Socrates' and 'musical'; for these are the same only by accident.

Part 12 "

"Now let us treat first of definition, in so far as we have not treated
of it in the Analytics; for the problem stated in them is useful for
our inquiries concerning substance. I mean this problem:-wherein can
consist the unity of that, the formula of which we call a definition,
as for instance, in the case of man, 'two-footed animal'; for let
this be the formula of man. Why, then, is this one, and not many,
viz. 'animal' and 'two-footed'? For in the case of 'man' and 'pale'
there is a plurality when one term does not belong to the other, but
a unity when it does belong and the subject, man, has a certain attribute;
for then a unity is produced and we have 'the pale man'. In the present
case, on the other hand, one does not share in the other; the genus
is not thought to share in its differentiae (for then the same thing
would share in contraries; for the differentiae by which the genus
is divided are contrary). And even if the genus does share in them,
the same argument applies, since the differentiae present in man are
many, e.g. endowed with feet, two-footed, featherless. Why are these
one and not many? Not because they are present in one thing; for on
this principle a unity can be made out of all the attributes of a
thing. But surely all the attributes in the definition must be one;
for the definition is a single formula and a formula of substance,
so that it must be a formula of some one thing; for substance means
a 'one' and a 'this', as we maintain. 

"We must first inquire about definitions reached by the method of
divisions. There is nothing in the definition except the first-named
and the differentiae. The other genera are the first genus and along
with this the differentiae that are taken with it, e.g. the first
may be 'animal', the next 'animal which is two-footed', and again
'animal which is two-footed and featherless', and similarly if the
definition includes more terms. And in general it makes no difference
whether it includes many or few terms,-nor, therefore, whether it
includes few or simply two; and of the two the one is differentia
and the other genus; e.g. in 'two-footed animal' 'animal' is genus,
and the other is differentia. 

"If then the genus absolutely does not exist apart from the species-of-a-genus,
or if it exists but exists as matter (for the voice is genus and matter,
but its differentiae make the species, i.e. the letters, out of it),
clearly the definition is the formula which comprises the differentiae.

"But it is also necessary that the division be by the differentia
of the diferentia; e.g. 'endowed with feet' is a differentia of 'animal';
again the differentia of 'animal endowed with feet' must be of it
qua endowed with feet. Therefore we must not say, if we are to speak
rightly, that of that which is endowed with feet one part has feathers
and one is featherless (if we do this we do it through incapacity);
we must divide it only into cloven-footed and not cloven; for these
are differentiae in the foot; cloven-footedness is a form of footedness.
And the process wants always to go on so till it reaches the species
that contain no differences. And then there will be as many kinds
of foot as there are differentiae, and the kinds of animals endowed
with feet will be equal in number to the differentiae. If then this
is so, clearly the last differentia will be the substance of the thing
and its definition, since it is not right to state the same things
more than once in our definitions; for it is superfluous. And this
does happen; for when we say 'animal endowed with feet and two-footed'
we have said nothing other than 'animal having feet, having two feet';
and if we divide this by the proper division, we shall be saying the
same thing more than once-as many times as there are differentiae.

"If then a differentia of a differentia be taken at each step, one
differentia-the last-will be the form and the substance; but if we
divide according to accidental qualities, e.g. if we were to divide
that which is endowed with feet into the white and the black, there
will be as many differentiae as there are cuts. Therefore it is plain
that the definition is the formula which contains the differentiae,
or, according to the right method, the last of these. This would be
evident, if we were to change the order of such definitions, e.g.
of that of man, saying 'animal which is two-footed and endowed with
feet'; for 'endowed with feet' is superfluous when 'two-footed' has
been said. But there is no order in the substance; for how are we
to think the one element posterior and the other prior? Regarding
the definitions, then, which are reached by the method of divisions,
let this suffice as our first attempt at stating their nature.

Part 13 "

"Let us return to the subject of our inquiry, which is substance.
As the substratum and the essence and the compound of these are called
substance, so also is the universal. About two of these we have spoken;
both about the essence and about the substratum, of which we have
said that it underlies in two senses, either being a 'this'-which
is the way in which an animal underlies its attributes-or as the matter
underlies the complete reality. The universal also is thought by some
to be in the fullest sense a cause, and a principle; therefore let
us attack the discussion of this point also. For it seems impossible
that any universal term should be the name of a substance. For firstly
the substance of each thing is that which is peculiar to it, which
does not belong to anything else; but the universal is common, since
that is called universal which is such as to belong to more than one
thing. Of which individual then will this be the substance? Either
of all or of none; but it cannot be the substance of all. And if it
is to be the substance of one, this one will be the others also; for
things whose substance is one and whose essence is one are themselves
also one. 

"Further, substance means that which is not predicable of a subject,
but the universal is predicable of some subject always. 

"But perhaps the universal, while it cannot be substance in the way
in which the essence is so, can be present in this; e.g. 'animal'
can be present in 'man' and 'horse'. Then clearly it is a formula
of the essence. And it makes no difference even if it is not a formula
of everything that is in the substance; for none the less the universal
will be the substance of something, as 'man' is the substance of the
individual man in whom it is present, so that the same result will
follow once more; for the universal, e.g. 'animal', will be the substance
of that in which it is present as something peculiar to it. And further
it is impossible and absurd that the 'this', i.e. the substance, if
it consists of parts, should not consist of substances nor of what
is a 'this', but of quality; for that which is not substance, i.e.
the quality, will then be prior to substance and to the 'this'. Which
is impossible; for neither in formula nor in time nor in coming to
be can the modifications be prior to the substance; for then they
will also be separable from it. Further, Socrates will contain a substance
present in a substance, so that this will be the substance of two
things. And in general it follows, if man and such things are substance,
that none of the elements in their formulae is the substance of anything,
nor does it exist apart from the species or in anything else; I mean,
for instance, that no 'animal' exists apart from the particular kinds
of animal, nor does any other of the elements present in formulae
exist apart. 

"If, then, we view the matter from these standpoints, it is plain
that no universal attribute is a substance, and this is plain also
from the fact that no common predicate indicates a 'this', but rather
a 'such'. If not, many difficulties follow and especially the 'third
man'. 

"The conclusion is evident also from the following consideration.
A substance cannot consist of substances present in it in complete
reality; for things that are thus in complete reality two are never
in complete reality one, though if they are potentially two, they
can be one (e.g. the double line consists of two halves-potentially;
for the complete realization of the halves divides them from one another);
therefore if the substance is one, it will not consist of substances
present in it and present in this way, which Democritus describes
rightly; he says one thing cannot be made out of two nor two out of
one; for he identifies substances with his indivisible magnitudes.
It is clear therefore that the same will hold good of number, if number
is a synthesis of units, as is said by some; for two is either not
one, or there is no unit present in it in complete reality. But our
result involves a difficulty. If no substance can consist of universals
because a universal indicates a 'such', not a 'this', and if no substance
can be composed of substances existing in complete reality, every
substance would be incomposite, so that there would not even be a
formula of any substance. But it is thought by all and was stated
long ago that it is either only, or primarily, substance that can
defined; yet now it seems that not even substance can. There cannot,
then, be a definition of anything; or in a sense there can be, and
in a sense there cannot. And what we are saying will be plainer from
what follows. 

Part 14 "

"It is clear also from these very facts what consequence confronts
those who say the Ideas are substances capable of separate existence,
and at the same time make the Form consist of the genus and the differentiae.
For if the Forms exist and 'animal' is present in 'man' and 'horse',
it is either one and the same in number, or different. (In formula
it is clearly one; for he who states the formula will go through the
formula in either case.) If then there is a 'man-in-himself' who is
a 'this' and exists apart, the parts also of which he consists, e.g.
'animal' and 'two-footed', must indicate 'thises', and be capable
of separate existence, and substances; therefore 'animal', as well
as 'man', must be of this sort. 

"Now (1) if the 'animal' in 'the horse' and in 'man' is one and the
same, as you are with yourself, (a) how will the one in things that
exist apart be one, and how will this 'animal' escape being divided
even from itself? 

"Further, (b) if it is to share in 'two-footed' and 'many-footed',
an impossible conclusion follows; for contrary attributes will belong
at the same time to it although it is one and a 'this'. If it is not
to share in them, what is the relation implied when one says the animal
is two-footed or possessed of feet? But perhaps the two things are
'put together' and are 'in contact', or are 'mixed'. Yet all these
expressions are absurd. 

"But (2) suppose the Form to be different in each species. Then there
will be practically an infinite number of things whose substance is
animal'; for it is not by accident that 'man' has 'animal' for one
of its elements. Further, many things will be 'animal-itself'. For
(i) the 'animal' in each species will be the substance of the species;
for it is after nothing else that the species is called; if it were,
that other would be an element in 'man', i.e. would be the genus of
man. And further, (ii) all the elements of which 'man' is composed
will be Ideas. None of them, then, will be the Idea of one thing and
the substance of another; this is impossible. The 'animal', then,
present in each species of animals will be animal-itself. Further,
from what is this 'animal' in each species derived, and how will it
be derived from animal-itself? Or how can this 'animal', whose essence
is simply animality, exist apart from animal-itself? 

"Further, (3)in the case of sensible things both these consequences
and others still more absurd follow. If, then, these consequences
are impossible, clearly there are not Forms of sensible things in
the sense in which some maintain their existence. 

Part 15 "

"Since substance is of two kinds, the concrete thing and the formula
(I mean that one kind of substance is the formula taken with the matter,
while another kind is the formula in its generality), substances in
the former sense are capable of destruction (for they are capable
also of generation), but there is no destruction of the formula in
the sense that it is ever in course of being destroyed (for there
is no generation of it either; the being of house is not generated,
but only the being of this house), but without generation and destruction
formulae are and are not; for it has been shown that no one begets
nor makes these. For this reason, also, there is neither definition
of nor demonstration about sensible individual substances, because
they have matter whose nature is such that they are capable both of
being and of not being; for which reason all the individual instances
of them are destructible. If then demonstration is of necessary truths
and definition is a scientific process, and if, just as knowledge
cannot be sometimes knowledge and sometimes ignorance, but the state
which varies thus is opinion, so too demonstration and definition
cannot vary thus, but it is opinion that deals with that which can
be otherwise than as it is, clearly there can neither be definition
of nor demonstration about sensible individuals. For perishing things
are obscure to those who have the relevant knowledge, when they have
passed from our perception; and though the formulae remain in the
soul unchanged, there will no longer be either definition or demonstration.
And so when one of the definition-mongers defines any individual,
he must recognize that his definition may always be overthrown; for
it is not possible to define such things. 

"Nor is it possible to define any Idea. For the Idea is, as its supporters
say, an individual, and can exist apart; and the formula must consist
of words; and he who defines must not invent a word (for it would
be unknown), but the established words are common to all the members
of a class; these then must apply to something besides the thing defined;
e.g. if one were defining you, he would say 'an animal which is lean'
or 'pale', or something else which will apply also to some one other
than you. If any one were to say that perhaps all the attributes taken
apart may belong to many subjects, but together they belong only to
this one, we must reply first that they belong also to both the elements;
e.g. 'two-footed animal' belongs to animal and to the two-footed.
(And in the case of eternal entities this is even necessary, since
the elements are prior to and parts of the compound; nay more, they
can also exist apart, if 'man' can exist apart. For either neither
or both can. If, then, neither can, the genus will not exist apart
from the various species; but if it does, the differentia will also.)
Secondly, we must reply that 'animal' and 'two-footed' are prior in
being to 'two-footed animal'; and things which are prior to others
are not destroyed when the others are. 

"Again, if the Ideas consist of Ideas (as they must, since elements
are simpler than the compound), it will be further necessary that
the elements also of which the Idea consists, e.g. 'animal' and 'two-footed',
should be predicated of many subjects. If not, how will they come
to be known? For there will then be an Idea which cannot be predicated
of more subjects than one. But this is not thought possible-every
Idea is thought to be capable of being shared. 

"As has been said, then, the impossibility of defining individuals
escapes notice in the case of eternal things, especially those which
are unique, like the sun or the moon. For people err not only by adding
attributes whose removal the sun would survive, e.g. 'going round
the earth' or 'night-hidden' (for from their view it follows that
if it stands still or is visible, it will no longer be the sun; but
it is strange if this is so; for 'the sun' means a certain substance);
but also by the mention of attributes which can belong to another
subject; e.g. if another thing with the stated attributes comes into
existence, clearly it will be a sun; the formula therefore is general.
But the sun was supposed to be an individual, like Cleon or Socrates.
After all, why does not one of the supporters of the Ideas produce
a definition of an Idea? It would become clear, if they tried, that
what has now been said is true. 

Part 16 "

"Evidently even of the things that are thought to be substances, most
are only potencies,-both the parts of animals (for none of them exists
separately; and when they are separated, then too they exist, all
of them, merely as matter) and earth and fire and air; for none of
them is a unity, but as it were a mere heap, till they are worked
up and some unity is made out of them. One might most readily suppose
the parts of living things and the parts of the soul nearly related
to them to turn out to be both, i.e. existent in complete reality
as well as in potency, because they have sources of movement in something
in their joints; for which reason some animals live when divided.
Yet all the parts must exist only potentially, when they are one and
continuous by nature,-not by force or by growing into one, for such
a phenomenon is an abnormality. 

"Since the term 'unity' is used like the term 'being', and the substance
of that which is one is one, and things whose substance is numerically
one are numerically one, evidently neither unity nor being can be
the substance of things, just as being an element or a principle cannot
be the substance, but we ask what, then, the principle is, that we
may reduce the thing to something more knowable. Now of these concepts
'being' and 'unity' are more substantial than 'principle' or 'element'
or 'cause', but not even the former are substance, since in general
nothing that is common is substance; for substance does not belong
to anything but to itself and to that which has it, of which it is
the substance. Further, that which is one cannot be in many places
at the same time, but that which is common is present in many places
at the same time; so that clearly no universal exists apart from its
individuals. 

"But those who say the Forms exist, in one respect are right, in giving
the Forms separate existence, if they are substances; but in another
respect they are not right, because they say the one over many is
a Form. The reason for their doing this is that they cannot declare
what are the substances of this sort, the imperishable substances
which exist apart from the individual and sensible substances. They
make them, then, the same in kind as the perishable things (for this
kind of substance we know)--'man-himself' and 'horse-itself', adding
to the sensible things the word 'itself'. Yet even if we had not seen
the stars, none the less, I suppose, would they have been eternal
substances apart from those which we knew; so that now also if we
do not know what non-sensible substances there are, yet it is doubtless
necessary that there should he some.-Clearly, then, no universal term
is the name of a substance, and no substance is composed of substances.

Part 17 "

"Let us state what, i.e. what kind of thing, substance should be said
to be, taking once more another starting-point; for perhaps from this
we shall get a clear view also of that substance which exists apart
from sensible substances. Since, then, substance is a principle and
a cause, let us pursue it from this starting-point. The 'why' is always
sought in this form--'why does one thing attach to some other?' For
to inquire why the musical man is a musical man, is either to inquire--as
we have said why the man is musical, or it is something else. Now
'why a thing is itself' is a meaningless inquiry (for (to give meaning
to the question 'why') the fact or the existence of the thing must
already be evident-e.g. that the moon is eclipsed-but the fact that
a thing is itself is the single reason and the single cause to be
given in answer to all such questions as why the man is man, or the
musician musical', unless one were to answer 'because each thing is
inseparable from itself, and its being one just meant this'; this,
however, is common to all things and is a short and easy way with
the question). But we can inquire why man is an animal of such and
such a nature. This, then, is plain, that we are not inquiring why
he who is a man is a man. We are inquiring, then, why something is
predicable of something (that it is predicable must be clear; for
if not, the inquiry is an inquiry into nothing). E.g. why does it
thunder? This is the same as 'why is sound produced in the clouds?'
Thus the inquiry is about the predication of one thing of another.
And why are these things, i.e. bricks and stones, a house? Plainly
we are seeking the cause. And this is the essence (to speak abstractly),
which in some cases is the end, e.g. perhaps in the case of a house
or a bed, and in some cases is the first mover; for this also is a
cause. But while the efficient cause is sought in the case of genesis
and destruction, the final cause is sought in the case of being also.

"The object of the inquiry is most easily overlooked where one term
is not expressly predicated of another (e.g. when we inquire 'what
man is'), because we do not distinguish and do not say definitely
that certain elements make up a certain whole. But we must articulate
our meaning before we begin to inquire; if not, the inquiry is on
the border-line between being a search for something and a search
for nothing. Since we must have the existence of the thing as something
given, clearly the question is why the matter is some definite thing;
e.g. why are these materials a house? Because that which was the essence
of a house is present. And why is this individual thing, or this body
having this form, a man? Therefore what we seek is the cause, i.e.
the form, by reason of which the matter is some definite thing; and
this is the substance of the thing. Evidently, then, in the case of
simple terms no inquiry nor teaching is possible; our attitude towards
such things is other than that of inquiry. 

"Since that which is compounded out of something so that the whole
is one, not like a heap but like a syllable-now the syllable is not
its elements, ba is not the same as b and a, nor is flesh fire and
earth (for when these are separated the wholes, i.e. the flesh and
the syllable, no longer exist, but the elements of the syllable exist,
and so do fire and earth); the syllable, then, is something-not only
its elements (the vowel and the consonant) but also something else,
and the flesh is not only fire and earth or the hot and the cold,
but also something else:-if, then, that something must itself be either
an element or composed of elements, (1) if it is an element the same
argument will again apply; for flesh will consist of this and fire
and earth and something still further, so that the process will go
on to infinity. But (2) if it is a compound, clearly it will be a
compound not of one but of more than one (or else that one will be
the thing itself), so that again in this case we can use the same
argument as in the case of flesh or of the syllable. But it would
seem that this 'other' is something, and not an element, and that
it is the cause which makes this thing flesh and that a syllable.
And similarly in all other cases. And this is the substance of each
thing (for this is the primary cause of its being); and since, while
some things are not substances, as many as are substances are formed
in accordance with a nature of their own and by a process of nature,
their substance would seem to be this kind of 'nature', which is not
an element but a principle. An element, on the other hand, is that
into which a thing is divided and which is present in it as matter;
e.g. a and b are the elements of the syllable. 

----------------------------------------------------------------------

BOOK VIII

Part 1 

"

"WE must reckon up the results arising from what has been said, and
compute the sum of them, and put the finishing touch to our inquiry.
We have said that the causes, principles, and elements of substances
are the object of our search. And some substances are recognized by
every one, but some have been advocated by particular schools. Those
generally recognized are the natural substances, i.e. fire, earth,
water, air, &c., the simple bodies; second plants and their parts,
and animals and the parts of animals; and finally the physical universe
and its parts; while some particular schools say that Forms and the
objects of mathematics are substances. But there are arguments which
lead to the conclusion that there are other substances, the essence
and the substratum. Again, in another way the genus seems more substantial
than the various spccies, and the universal than the particulars.
And with the universal and the genus the Ideas are connected; it is
in virtue of the same argument that they are thought to be substances.
And since the essence is substance, and the definition is a formula
of the essence, for this reason we have discussed definition and essential
predication. Since the definition is a formula, and a formula has
parts, we had to consider also with respect to the notion of 'part',
what are parts of the substance and what are not, and whether the
parts of the substance are also parts of the definition. Further,
too, neither the universal nor the genus is a substance; we must inquire
later into the Ideas and the objects of mathematics; for some say
these are substances as well as the sensible substances.

"But now let us resume the discussion of the generally recognized
substances. These are the sensible substances, and sensible substances
all have matter. The substratum is substance, and this is in one sense
the matter (and by matter I mean that which, not being a 'this' actually,
is potentially a 'this'), and in another sense the formula or shape
(that which being a 'this' can be separately formulated), and thirdly
the complex of these two, which alone is generated and destroyed,
and is, without qualification, capable of separate existence; for
of substances completely expressible in a formula some are separable
and some are separable and some are not. 

"But clearly matter also is substance; for in all the opposite changes
that occur there is something which underlies the changes, e.g. in
respect of place that which is now here and again elsewhere, and in
respect of increase that which is now of one size and again less or
greater, and in respect of alteration that which is now healthy and
again diseased; and similarly in respect of substance there is something
that is now being generated and again being destroyed, and now underlies
the process as a 'this' and again underlies it in respect of a privation
of positive character. And in this change the others are involved.
But in either one or two of the others this is not involved; for it
is not necessary if a thing has matter for change of place that it
should also have matter for generation and destruction. 

"The difference between becoming in the full sense and becoming in
a qualified sense has been stated in our physical works.

Part 2 "

"Since the substance which exists as underlying and as matter is generally
recognized, and this that which exists potentially, it remains for
us to say what is the substance, in the sense of actuality, of sensible
things. Democritus seems to think there are three kinds of difference
between things; the underlying body, the matter, is one and the same,
but they differ either in rhythm, i.e. shape, or in turning, i.e.
position, or in inter-contact, i.e. order. But evidently there are
many differences; for instance, some things are characterized by the
mode of composition of their matter, e.g. the things formed by blending,
such as honey-water; and others by being bound together, e.g. bundle;
and others by being glued together, e.g. a book; and others by being
nailed together, e.g. a casket; and others in more than one of these
ways; and others by position, e.g. threshold and lintel (for these
differ by being placed in a certain way); and others by time, e.g.
dinner and breakfast; and others by place, e.g. the winds; and others
by the affections proper to sensible things, e.g. hardness and softness,
density and rarity, dryness and wetness; and some things by some of
these qualities, others by them all, and in general some by excess
and some by defect. Clearly, then, the word 'is' has just as many
meanings; a thing is a threshold because it lies in such and such
a position, and its being means its lying in that position, while
being ice means having been solidified in such and such a way. And
the being of some things will be defined by all these qualities, because
some parts of them are mixed, others are blended, others are bound
together, others are solidified, and others use the other differentiae;
e.g. the hand or the foot requires such complex definition. We must
grasp, then, the kinds of differentiae (for these will be the principles
of the being of things), e.g. the things characterized by the more
and the less, or by the dense and the rare, and by other such qualities;
for all these are forms of excess and defect. And anything that is
characterized by shape or by smoothness and roughness is characterized
by the straight and the curved. And for other things their being will
mean their being mixed, and their not being will mean the opposite.

"It is clear, then, from these facts that, since its substance is
the cause of each thing's being, we must seek in these differentiae
what is the cause of the being of each of these things. Now none of
these differentiae is substance, even when coupled with matter, yet
it is what is analogous to substance in each case; and as in substances
that which is predicated of the matter is the actuality itself, in
all other definitions also it is what most resembles full actuality.
E.g. if we had to define a threshold, we should say 'wood or stone
in such and such a position', and a house we should define as 'bricks
and timbers in such and such a position',(or a purpose may exist as
well in some cases), and if we had to define ice we should say 'water
frozen or solidified in such and such a way', and harmony is 'such
and such a blending of high and low'; and similarly in all other cases.

"Obviously, then, the actuality or the formula is different when the
matter is different; for in some cases it is the composition, in others
the mixing, and in others some other of the attributes we have named.
And so, of the people who go in for defining, those who define a house
as stones, bricks, and timbers are speaking of the potential house,
for these are the matter; but those who propose 'a receptacle to shelter
chattels and living beings', or something of the sort, speak of the
actuality. Those who combine both of these speak of the third kind
of substance, which is composed of matter and form (for the formula
that gives the differentiae seems to be an account of the form or
actuality, while that which gives the components is rather an account
of the matter); and the same is true of the kind of definitions which
Archytas used to accept; they are accounts of the combined form and
matter. E.g. what is still weather? Absence of motion in a large expanse
of air; air is the matter, and absence of motion is the actuality
and substance. What is a calm? Smoothness of sea; the material substratum
is the sea, and the actuality or shape is smoothness. It is obvious
then, from what has been said, what sensible substance is and how
it exists-one kind of it as matter, another as form or actuality,
while the third kind is that which is composed of these two.

Part 3 "

"We must not fail to notice that sometimes it is not clear whether
a name means the composite substance, or the actuality or form, e.g.
whether 'house' is a sign for the composite thing, 'a covering consisting
of bricks and stones laid thus and thus', or for the actuality or
form, 'a covering', and whether a line is 'twoness in length' or 'twoness',
and whether an animal is soul in a body' or 'a soul'; for soul is
the substance or actuality of some body. 'Animal' might even be applied
to both, not as something definable by one formula, but as related
to a single thing. But this question, while important for another
purpose, is of no importance for the inquiry into sensible substance;
for the essence certainly attaches to the form and the actuality.
For 'soul' and 'to be soul' are the same, but 'to be man' and 'man'
are not the same, unless even the bare soul is to be called man; and
thus on one interpretation the thing is the same as its essence, and
on another it is not. 

"If we examine we find that the syllable does not consist of the letters
+ juxtaposition, nor is the house bricks + juxtaposition. And this
is right; for the juxtaposition or mixing does not consist of those
things of which it is the juxtaposition or mixing. And the same is
true in all other cases; e.g. if the threshold is characterized by
its position, the position is not constituted by the threshold, but
rather the latter is constituted by the former. Nor is man animal
+ biped, but there must be something besides these, if these are matter,-something
which is neither an element in the whole nor a compound, but is the
substance; but this people eliminate, and state only the matter. If,
then, this is the cause of the thing's being, and if the cause of
its being is its substance, they will not be stating the substance
itself. 

"(This, then, must either be eternal or it must be destructible without
being ever in course of being destroyed, and must have come to be
without ever being in course of coming to be. But it has been proved
and explained elsewhere that no one makes or begets the form, but
it is the individual that is made, i.e. the complex of form and matter
that is generated. Whether the substances of destructible things can
exist apart, is not yet at all clear; except that obviously this is
impossible in some cases-in the case of things which cannot exist
apart from the individual instances, e.g. house or utensil. Perhaps,
indeed, neither these things themselves, nor any of the other things
which are not formed by nature, are substances at all; for one might
say that the nature in natural objects is the only substance to be
found in destructible things.) 

"Therefore the difficulty which used to be raised by the school of
Antisthenes and other such uneducated people has a certain timeliness.
They said that the 'what' cannot be defined (for the definition so
called is a 'long rigmarole') but of what sort a thing, e.g. silver,
is, they thought it possible actually to explain, not saying what
it is, but that it is like tin. Therefore one kind of substance can
be defined and formulated, i.e. the composite kind, whether it be
perceptible or intelligible; but the primary parts of which this consists
cannot be defined, since a definitory formula predicates something
of something, and one part of the definition must play the part of
matter and the other that of form. 

"It is also obvious that, if substances are in a sense numbers, they
are so in this sense and not, as some say, as numbers of units. For
a definition is a sort of number; for (1) it is divisible, and into
indivisible parts (for definitory formulae are not infinite), and
number also is of this nature. And (2) as, when one of the parts of
which a number consists has been taken from or added to the number,
it is no longer the same number, but a different one, even if it is
the very smallest part that has been taken away or added, so the definition
and the essence will no longer remain when anything has been taken
away or added. And (3) the number must be something in virtue of which
it is one, and this these thinkers cannot state, what makes it one,
if it is one (for either it is not one but a sort of heap, or if it
is, we ought to say what it is that makes one out of many); and the
definition is one, but similarly they cannot say what makes it one.
And this is a natural result; for the same reason is applicable, and
substance is one in the sense which we have explained, and not, as
some say, by being a sort of unit or point; each is a complete reality
and a definite nature. And (4) as number does not admit of the more
and the less, neither does substance, in the sense of form, but if
any substance does, it is only the substance which involves matter.
Let this, then, suffice for an account of the generation and destruction
of so-called substances in what sense it is possible and in what sense
impossible--and of the reduction of things to number. 

Part 4 "

"Regarding material substance we must not forget that even if all
things come from the same first cause or have the same things for
their first causes, and if the same matter serves as starting-point
for their generation, yet there is a matter proper to each, e.g. for
phlegm the sweet or the fat, and for bile the bitter, or something
else; though perhaps these come from the same original matter. And
there come to be several matters for the same thing, when the one
matter is matter for the other; e.g. phlegm comes from the fat and
from the sweet, if the fat comes from the sweet; and it comes from
bile by analysis of the bile into its ultimate matter. For one thing
comes from another in two senses, either because it will be found
at a later stage, or because it is produced if the other is analysed
into its original constituents. When the matter is one, different
things may be produced owing to difference in the moving cause; e.g.
from wood may be made both a chest and a bed. But some different things
must have their matter different; e.g. a saw could not be made of
wood, nor is this in the power of the moving cause; for it could not
make a saw of wool or of wood. But if, as a matter of fact, the same
thing can be made of different material, clearly the art, i.e. the
moving principle, is the same; for if both the matter and the moving
cause were different, the product would be so too. 

"When one inquires into the cause of something, one should, since
'causes' are spoken of in several senses, state all the possible causes.
what is the material cause of man? Shall we say 'the menstrual fluid'?
What is moving cause? Shall we say 'the seed'? The formal cause? His
essence. The final cause? His end. But perhaps the latter two are
the same.-It is the proximate causes we must state. What is the material
cause? We must name not fire or earth, but the matter peculiar to
the thing. 

"Regarding the substances that are natural and generable, if the causes
are really these and of this number and we have to learn the causes,
we must inquire thus, if we are to inquire rightly. But in the case
of natural but eternal substances another account must be given. For
perhaps some have no matter, or not matter of this sort but only such
as can be moved in respect of place. Nor does matter belong to those
things which exist by nature but are not substances; their substratum
is the substance. E.g what is the cause of eclipse? What is its matter?
There is none; the moon is that which suffers eclipse. What is the
moving cause which extinguished the light? The earth. The final cause
perhaps does not exist. The formal principle is the definitory formula,
but this is obscure if it does not include the cause. E.g. what is
eclipse? Deprivation of light. But if we add 'by the earth's coming
in between', this is the formula which includes the cause. In the
case of sleep it is not clear what it is that proximately has this
affection. Shall we say that it is the animal? Yes, but the animal
in virtue of what, i.e. what is the proximate subject? The heart or
some other part. Next, by what is it produced? Next, what is the affection-that
of the proximate subject, not of the whole animal? Shall we say that
it is immobility of such and such a kind? Yes, but to what process
in the proximate subject is this due? 

Part 5 "

"Since some things are and are not, without coming to be and ceasing
to be, e.g. points, if they can be said to be, and in general forms
(for it is not 'white' comes to be, but the wood comes to be white,
if everything that comes to be comes from something and comes to be
something), not all contraries can come from one another, but it is
in different senses that a pale man comes from a dark man, and pale
comes from dark. Nor has everything matter, but only those things
which come to be and change into one another. Those things which,
without ever being in course of changing, are or are not, have no
matter. 

"There is difficulty in the question how the matter of each thing
is related to its contrary states. E.g. if the body is potentially
healthy, and disease is contrary to health, is it potentially both
healthy and diseased? And is water potentially wine and vinegar? We
answer that it is the matter of one in virtue of its positive state
and its form, and of the other in virtue of the privation of its positive
state and the corruption of it contrary to its nature. It is also
hard to say why wine is not said to be the matter of vinegar nor potentially
vinegar (though vinegar is produced from it), and why a living man
is not said to be potentially dead. In fact they are not, but the
corruptions in question are accidental, and it is the matter of the
animal that is itself in virtue of its corruption the potency and
matter of a corpse, and it is water that is the matter of vinegar.
For the corpse comes from the animal, and vinegar from wine, as night
from day. And all the things which change thus into one another must
go back to their matter; e.g. if from a corpse is produced an animal,
the corpse first goes back to its matter, and only then becomes an
animal; and vinegar first goes back to water, and only then becomes
wine. 

Part 6 "

"To return to the difficulty which has been stated with respect both
to definitions and to numbers, what is the cause of their unity? In
the case of all things which have several parts and in which the totality
is not, as it were, a mere heap, but the whole is something beside
the parts, there is a cause; for even in bodies contact is the cause
of unity in some cases, and in others viscosity or some other such
quality. And a definition is a set of words which is one not by being
connected together, like the Iliad, but by dealing with one object.-What
then, is it that makes man one; why is he one and not many, e.g. animal
+ biped, especially if there are, as some say, an animal-itself and
a biped-itself? Why are not those Forms themselves the man, so that
men would exist by participation not in man, nor in-one Form, but
in two, animal and biped, and in general man would be not one but
more than one thing, animal and biped? 

"Clearly, then, if people proceed thus in their usual manner of definition
and speech, they cannot explain and solve the difficulty. But if,
as we say, one element is matter and another is form, and one is potentially
and the other actually, the question will no longer be thought a difficulty.
For this difficulty is the same as would arise if 'round bronze' were
the definition of 'cloak'; for this word would be a sign of the definitory
formula, so that the question is, what is the cause of the unity of
'round' and 'bronze'? The difficulty disappears, because the one is
matter, the other form. What, then, causes this-that which was potentially
to be actually-except, in the case of things which are generated,
the agent? For there is no other cause of the potential sphere's becoming
actually a sphere, but this was the essence of either. Of matter some
is intelligible, some perceptible, and in a formula there is always
an element of matter as well as one of actuality; e.g. the circle
is 'a plane figure'. But of the things which have no matter, either
intelligible or perceptible, each is by its nature essentially a kind
of unity, as it is essentially a kind of being-individual substance,
quality, or quantity (and so neither 'existent' nor 'one' is present
in their definitions), and the essence of each of them is by its very
nature a kind of unity as it is a kind of being-and so none of these
has any reason outside itself, for being one, nor for being a kind
of being; for each is by its nature a kind of being and a kind of
unity, not as being in the genus 'being' or 'one' nor in the sense
that being and unity can exist apart from particulars. 

"Owing to the difficulty about unity some speak of 'participation',
and raise the question, what is the cause of participation and what
is it to participate; and others speak of 'communion', as Lycophron
says knowledge is a communion of knowing with the soul; and others
say life is a 'composition' or 'connexion' of soul with body. Yet
the same account applies to all cases; for being healthy, too, will
on this showing be either a 'communion' or a 'connexion' or a 'composition'
of soul and health, and the fact that the bronze is a triangle will
be a 'composition' of bronze and triangle, and the fact that a thing
is white will be a 'composition' of surface and whiteness. The reason
is that people look for a unifying formula, and a difference, between
potency and complete reality. But, as has been said, the proximate
matter and the form are one and the same thing, the one potentially,
and the other actually. Therefore it is like asking what in general
is the cause of unity and of a thing's being one; for each thing is
a unity, and the potential and the actual are somehow one. Therefore
there is no other cause here unless there is something which caused
the movement from potency into actuality. And all things which have
no matter are without qualification essentially unities.

----------------------------------------------------------------------

BOOK IX

Part 1 

"

"WE have treated of that which is primarily and to which all the other
categories of being are referred-i.e. of substance. For it is in virtue
of the concept of substance that the others also are said to be-quantity
and quality and the like; for all will be found to involve the concept
of substance, as we said in the first part of our work. And since
'being' is in one way divided into individual thing, quality, and
quantity, and is in another way distinguished in respect of potency
and complete reality, and of function, let us now add a discussion
of potency and complete reality. And first let us explain potency
in the strictest sense, which is, however, not the most useful for
our present purpose. For potency and actuality extend beyond the cases
that involve a reference to motion. But when we have spoken of this
first kind, we shall in our discussions of actuality' explain the
other kinds of potency as well. 

"We have pointed out elsewhere that 'potency' and the word 'can' have
several senses. Of these we may neglect all the potencies that are
so called by an equivocation. For some are called so by analogy, as
in geometry we say one thing is or is not a 'power' of another by
virtue of the presence or absence of some relation between them. But
all potencies that conform to the same type are originative sources
of some kind, and are called potencies in reference to one primary
kind of potency, which is an originative source of change in another
thing or in the thing itself qua other. For one kind is a potency
of being acted on, i.e. the originative source, in the very thing
acted on, of its being passively changed by another thing or by itself
qua other; and another kind is a state of insusceptibility to change
for the worse and to destruction by another thing or by the thing
itself qua other by virtue of an originative source of change. In
all these definitions is implied the formula if potency in the primary
sense.-And again these so-called potencies are potencies either of
merely acting or being acted on, or of acting or being acted on well,
so that even in the formulae of the latter the formulae of the prior
kinds of potency are somehow implied. 

"Obviously, then, in a sense the potency of acting and of being acted
on is one (for a thing may be 'capable' either because it can itself
be acted on or because something else can be acted on by it), but
in a sense the potencies are different. For the one is in the thing
acted on; it is because it contains a certain originative source,
and because even the matter is an originative source, that the thing
acted on is acted on, and one thing by one, another by another; for
that which is oily can be burnt, and that which yields in a particular
way can be crushed; and similarly in all other cases. But the other
potency is in the agent, e.g. heat and the art of building are present,
one in that which can produce heat and the other in the man who can
build. And so, in so far as a thing is an organic unity, it cannot
be acted on by itself; for it is one and not two different things.
And 'impotence'and 'impotent' stand for the privation which is contrary
to potency of this sort, so that every potency belongs to the same
subject and refers to the same process as a corresponding impotence.
Privation has several senses; for it means (1) that which has not
a certain quality and (2) that which might naturally have it but has
not it, either (a) in general or (b) when it might naturally have
it, and either (a) in some particular way, e.g. when it has not it
completely, or (b) when it has not it at all. And in certain cases
if things which naturally have a quality lose it by violence, we say
they have suffered privation. 

Part 2 "

"Since some such originative sources are present in soulless things,
and others in things possessed of soul, and in soul, and in the rational
part of the soul, clearly some potencies will, be non-rational and
some will be non-rational and some will be accompanied by a rational
formula. This is why all arts, i.e. all productive forms of knowledge,
are potencies; they are originative sources of change in another thing
or in the artist himself considered as other. 

"And each of those which are accompanied by a rational formula is
alike capable of contrary effects, but one non-rational power produces
one effect; e.g. the hot is capable only of heating, but the medical
art can produce both disease and health. The reason is that science
is a rational formula, and the same rational formula explains a thing
and its privation, only not in the same way; and in a sense it applies
to both, but in a sense it applies rather to the positive fact. Therefore
such sciences must deal with contraries, but with one in virtue of
their own nature and with the other not in virtue of their nature;
for the rational formula applies to one object in virtue of that object's
nature, and to the other, in a sense, accidentally. For it is by denial
and removal that it exhibits the contrary; for the contrary is the
primary privation, and this is the removal of the positive term. Now
since contraries do not occur in the same thing, but science is a
potency which depends on the possession of a rational formula, and
the soul possesses an originative source of movement; therefore, while
the wholesome produces only health and the calorific only heat and
the frigorific only cold, the scientific man produces both the contrary
effects. For the rational formula is one which applies to both, though
not in the same way, and it is in a soul which possesses an originative
source of movement; so that the soul will start both processes from
the same originative source, having linked them up with the same thing.
And so the things whose potency is according to a rational formula
act contrariwise to the things whose potency is non-rational; for
the products of the former are included under one originative source,
the rational formula. 

"It is obvious also that the potency of merely doing a thing or having
it done to one is implied in that of doing it or having it done well,
but the latter is not always implied in the former: for he who does
a thing well must also do it, but he who does it merely need not also
do it well. 

Part 3 "

"There are some who say, as the Megaric school does, that a thing
'can' act only when it is acting, and when it is not acting it 'cannot'
act, e.g. that he who is not building cannot build, but only he who
is building, when he is building; and so in all other cases. It is
not hard to see the absurdities that attend this view. 

"For it is clear that on this view a man will not be a builder unless
he is building (for to be a builder is to be able to build), and so
with the other arts. If, then, it is impossible to have such arts
if one has not at some time learnt and acquired them, and it is then
impossible not to have them if one has not sometime lost them (either
by forgetfulness or by some accident or by time; for it cannot be
by the destruction of the object, for that lasts for ever), a man
will not have the art when he has ceased to use it, and yet he may
immediately build again; how then will he have got the art? And similarly
with regard to lifeless things; nothing will be either cold or hot
or sweet or perceptible at all if people are not perceiving it; so
that the upholders of this view will have to maintain the doctrine
of Protagoras. But, indeed, nothing will even have perception if it
is not perceiving, i.e. exercising its perception. If, then, that
is blind which has not sight though it would naturally have it, when
it would naturally have it and when it still exists, the same people
will be blind many times in the day-and deaf too. 

"Again, if that which is deprived of potency is incapable, that which
is not happening will be incapable of happening; but he who says of
that which is incapable of happening either that it is or that it
will be will say what is untrue; for this is what incapacity meant.
Therefore these views do away with both movement and becoming. For
that which stands will always stand, and that which sits will always
sit, since if it is sitting it will not get up; for that which, as
we are told, cannot get up will be incapable of getting up. But we
cannot say this, so that evidently potency and actuality are different
(but these views make potency and actuality the same, and so it is
no small thing they are seeking to annihilate), so that it is possible
that a thing may be capable of being and not he, and capable of not
being and yet he, and similarly with the other kinds of predicate;
it may be capable of walking and yet not walk, or capable of not walking
and yet walk. And a thing is capable of doing something if there will
be nothing impossible in its having the actuality of that of which
it is said to have the capacity. I mean, for instance, if a thing
is capable of sitting and it is open to it to sit, there will be nothing
impossible in its actually sitting; and similarly if it is capable
of being moved or moving, or of standing or making to stand, or of
being or coming to be, or of not being or not coming to be.

"The word 'actuality', which we connect with 'complete reality', has,
in the main, been extended from movements to other things; for actuality
in the strict sense is thought to be identical with movement. And
so people do not assign movement to non-existent things, though they
do assign some other predicates. E.g. they say that non-existent things
are objects of thought and desire, but not that they are moved; and
this because, while ex hypothesi they do not actually exist, they
would have to exist actually if they were moved. For of non-existent
things some exist potentially; but they do not exist, because they
do not exist in complete reality. 

Part 4 "

"If what we have described is identical with the capable or convertible
with it, evidently it cannot be true to say 'this is capable of being
but will not be', which would imply that the things incapable of being
would on this showing vanish. Suppose, for instance, that a man-one
who did not take account of that which is incapable of being-were
to say that the diagonal of the square is capable of being measured
but will not be measured, because a thing may well be capable of being
or coming to be, and yet not be or be about to be. But from the premisses
this necessarily follows, that if we actually supposed that which
is not, but is capable of being, to be or to have come to be, there
will be nothing impossible in this; but the result will be impossible,
for the measuring of the diagonal is impossible. For the false and
the impossible are not the same; that you are standing now is false,
but that you should be standing is not impossible. 

"At the same time it is clear that if, when A is real, B must be real,
then, when A is possible, B also must be possible. For if B need not
be possible, there is nothing to prevent its not being possible. Now
let A be supposed possible. Then, when A was possible, we agreed that
nothing impossible followed if A were supposed to be real; and then
B must of course be real. But we supposed B to be impossible. Let
it be impossible then. If, then, B is impossible, A also must be so.
But the first was supposed impossible; therefore the second also is
impossible. If, then, A is possible, B also will be possible, if they
were so related that if A,is real, B must be real. If, then, A and
B being thus related, B is not possible on this condition, and B will
not be related as was supposed. And if when A is possible, B must
be possible, then if A is real, B also must be real. For to say that
B must be possible, if A is possible, means this, that if A is real
both at the time when and in the way in which it was supposed capable
of being real, B also must then and in that way be real.

Part 5 "

"As all potencies are either innate, like the senses, or come by practice,
like the power of playing the flute, or by learning, like artistic
power, those which come by practice or by rational formula we must
acquire by previous exercise but this is not necessary with those
which are not of this nature and which imply passivity. 

"Since that which is 'capable' is capable of something and at some
time in some way (with all the other qualifications which must be
present in the definition), and since some things can produce change
according to a rational formula and their potencies involve such a
formula, while other things are nonrational and their potencies are
non-rational, and the former potencies must be in a living thing,
while the latter can be both in the living and in the lifeless; as
regards potencies of the latter kind, when the agent and the patient
meet in the way appropriate to the potency in question, the one must
act and the other be acted on, but with the former kind of potency
this is not necessary. For the nonrational potencies are all productive
of one effect each, but the rational produce contrary effects, so
that if they produced their effects necessarily they would produce
contrary effects at the same time; but this is impossible. There must,
then, be something else that decides; I mean by this, desire or will.
For whichever of two things the animal desires decisively, it will
do, when it is present, and meets the passive object, in the way appropriate
to the potency in question. Therefore everything which has a rational
potency, when it desires that for which it has a potency and in the
circumstances in which it has the potency, must do this. And it has
the potency in question when the passive object is present and is
in a certain state; if not it will not be able to act. (To add the
qualification 'if nothing external prevents it' is not further necessary;
for it has the potency on the terms on which this is a potency of
acting, and it is this not in all circumstances but on certain conditions,
among which will be the exclusion of external hindrances; for these
are barred by some of the positive qualifications.) And so even if
one has a rational wish, or an appetite, to do two things or contrary
things at the same time, one will not do them; for it is not on these
terms that one has the potency for them, nor is it a potency of doing
both at the same time, since one will do the things which it is a
potency of doing, on the terms on which one has the potency.

Part 6 "

"Since we have treated of the kind of potency which is related to
movement, let us discuss actuality-what, and what kind of thing, actuality
is. For in the course of our analysis it will also become clear, with
regard to the potential, that we not only ascribe potency to that
whose nature it is to move something else, or to be moved by something
else, either without qualification or in some particular way, but
also use the word in another sense, which is the reason of the inquiry
in the course of which we have discussed these previous senses also.
Actuality, then, is the existence of a thing not in the way which
we express by 'potentially'; we say that potentially, for instance,
a statue of Hermes is in the block of wood and the half-line is in
the whole, because it might be separated out, and we call even the
man who is not studying a man of science, if he is capable of studying;
the thing that stands in contrast to each of these exists actually.
Our meaning can be seen in the particular cases by induction, and
we must not seek a definition of everything but be content to grasp
the analogy, that it is as that which is building is to that which
is capable of building, and the waking to the sleeping, and that which
is seeing to that which has its eyes shut but has sight, and that
which has been shaped out of the matter to the matter, and that which
has been wrought up to the unwrought. Let actuality be defined by
one member of this antithesis, and the potential by the other. But
all things are not said in the same sense to exist actually, but only
by analogy-as A is in B or to B, C is in D or to D; for some are as
movement to potency, and the others as substance to some sort of matter.

"But also the infinite and the void and all similar things are said
to exist potentially and actually in a different sense from that which
applies to many other things, e.g. to that which sees or walks or
is seen. For of the latter class these predicates can at some time
be also truly asserted without qualification; for the seen is so called
sometimes because it is being seen, sometimes because it is capable
of being seen. But the infinite does not exist potentially in the
sense that it will ever actually have separate existence; it exists
potentially only for knowledge. For the fact that the process of dividing
never comes to an end ensures that this activity exists potentially,
but not that the infinite exists separately. 

"Since of the actions which have a limit none is an end but all are
relative to the end, e.g. the removing of fat, or fat-removal, and
the bodily parts themselves when one is making them thin are in movement
in this way (i.e. without being already that at which the movement
aims), this is not an action or at least not a complete one (for it
is not an end); but that movement in which the end is present is an
action. E.g. at the same time we are seeing and have seen, are understanding
and have understood, are thinking and have thought (while it is not
true that at the same time we are learning and have learnt, or are
being cured and have been cured). At the same time we are living well
and have lived well, and are happy and have been happy. If not, the
process would have had sometime to cease, as the process of making
thin ceases: but, as things are, it does not cease; we are living
and have lived. Of these processes, then, we must call the one set
movements, and the other actualities. For every movement is incomplete-making
thin, learning, walking, building; these are movements, and incomplete
at that. For it is not true that at the same time a thing is walking
and has walked, or is building and has built, or is coming to be and
has come to be, or is being moved and has been moved, but what is
being moved is different from what has been moved, and what is moving
from what has moved. But it is the same thing that at the same time
has seen and is seeing, seeing, or is thinking and has thought. The
latter sort of process, then, I call an actuality, and the former
a movement. 

Part 7 "

"What, and what kind of thing, the actual is, may be taken as explained
by these and similar considerations. But we must distinguish when
a thing exists potentially and when it does not; for it is not at
any and every time. E.g. is earth potentially a man? No-but rather
when it has already become seed, and perhaps not even then. It is
just as it is with being healed; not everything can be healed by the
medical art or by luck, but there is a certain kind of thing which
is capable of it, and only this is potentially healthy. And (1) the
delimiting mark of that which as a result of thought comes to exist
in complete reality from having existed potentially is that if the
agent has willed it it comes to pass if nothing external hinders,
while the condition on the other side-viz. in that which is healed-is
that nothing in it hinders the result. It is on similar terms that
we have what is potentially a house; if nothing in the thing acted
on-i.e. in the matter-prevents it from becoming a house, and if there
is nothing which must be added or taken away or changed, this is potentially
a house; and the same is true of all other things the source of whose
becoming is external. And (2) in the cases in which the source of
the becoming is in the very thing which comes to be, a thing is potentially
all those things which it will be of itself if nothing external hinders
it. E.g. the seed is not yet potentially a man; for it must be deposited
in something other than itself and undergo a change. But when through
its own motive principle it has already got such and such attributes,
in this state it is already potentially a man; while in the former
state it needs another motive principle, just as earth is not yet
potentially a statue (for it must first change in order to become
brass.) 

"It seems that when we call a thing not something else but 'thaten'-e.g.
a casket is not 'wood' but 'wooden', and wood is not 'earth' but 'earthen',
and again earth will illustrate our point if it is similarly not something
else but 'thaten'-that other thing is always potentially (in the full
sense of that word) the thing which comes after it in this series.
E.g. a casket is not 'earthen' nor 'earth', but 'wooden'; for this
is potentially a casket and this is the matter of a casket, wood in
general of a casket in general, and this particular wood of this particular
casket. And if there is a first thing, which is no longer, in reference
to something else, called 'thaten', this is prime matter; e.g. if
earth is 'airy' and air is not 'fire' but 'fiery', fire is prime matter,
which is not a 'this'. For the subject or substratum is differentiated
by being a 'this' or not being one; i.e. the substratum of modifications
is, e.g. a man, i.e. a body and a soul, while the modification is
'musical' or 'pale'. (The subject is called, when music comes to be
present in it, not 'music' but 'musical', and the man is not 'paleness'
but 'pale', and not 'ambulation' or 'movement' but 'walking' or 'moving',-which
is akin to the 'thaten'.) Wherever this is so, then, the ultimate
subject is a substance; but when this is not so but the predicate
is a form and a 'this', the ultimate subject is matter and material
substance. And it is only right that 'thaten' should be used with
reference both to the matter and to the accidents; for both are indeterminates.

"We have stated, then, when a thing is to be said to exist potentially
and when it is not. 

Part 8 "

"From our discussion of the various senses of 'prior', it is clear
that actuality is prior to potency. And I mean by potency not only
that definite kind which is said to be a principle of change in another
thing or in the thing itself regarded as other, but in general every
principle of movement or of rest. For nature also is in the same genus
as potency; for it is a principle of movement-not, however, in something
else but in the thing itself qua itself. To all such potency, then,
actuality is prior both in formula and in substantiality; and in time
it is prior in one sense, and in another not. 

"(1) Clearly it is prior in formula; for that which is in the primary
sense potential is potential because it is possible for it to become
active; e.g. I mean by 'capable of building' that which can build,
and by 'capable of seeing' that which can see, and by 'visible' that
which can be seen. And the same account applies to all other cases,
so that the formula and the knowledge of the one must precede the
knowledge of the other. 

"(2) In time it is prior in this sense: the actual which is identical
in species though not in number with a potentially existing thing
is to it. I mean that to this particular man who now exists actually
and to the corn and to the seeing subject the matter and the seed
and that which is capable of seeing, which are potentially a man and
corn and seeing, but not yet actually so, are prior in time; but prior
in time to these are other actually existing things, from which they
were produced. For from the potentially existing the actually existing
is always produced by an actually existing thing, e.g. man from man,
musician by musician; there is always a first mover, and the mover
already exists actually. We have said in our account of substance
that everything that is produced is something produced from something
and by something, and that the same in species as it. 

"This is why it is thought impossible to be a builder if one has built
nothing or a harper if one has never played the harp; for he who learns
to play the harp learns to play it by playing it, and all other learners
do similarly. And thence arose the sophistical quibble, that one who
does not possess a science will be doing that which is the object
of the science; for he who is learning it does not possess it. But
since, of that which is coming to be, some part must have come to
be, and, of that which, in general, is changing, some part must have
changed (this is shown in the treatise on movement), he who is learning
must, it would seem, possess some part of the science. But here too,
then, it is clear that actuality is in this sense also, viz. in order
of generation and of time, prior to potency. 

"But (3) it is also prior in substantiality; firstly, (a) because
the things that are posterior in becoming are prior in form and in
substantiality (e.g. man is prior to boy and human being to seed;
for the one already has its form, and the other has not), and because
everything that comes to be moves towards a principle, i.e. an end
(for that for the sake of which a thing is, is its principle, and
the becoming is for the sake of the end), and the actuality is the
end, and it is for the sake of this that the potency is acquired.
For animals do not see in order that they may have sight, but they
have sight that they may see. And similarly men have the art of building
that they may build, and theoretical science that they may theorize;
but they do not theorize that they may have theoretical science, except
those who are learning by practice; and these do not theorize except
in a limited sense, or because they have no need to theorize. Further,
matter exists in a potential state, just because it may come to its
form; and when it exists actually, then it is in its form. And the
same holds good in all cases, even those in which the end is a movement.
And so, as teachers think they have achieved their end when they have
exhibited the pupil at work, nature does likewise. For if this is
not the case, we shall have Pauson's Hermes over again, since it will
be hard to say about the knowledge, as about the figure in the picture,
whether it is within or without. For the action is the end, and the
actuality is the action. And so even the word 'actuality' is derived
from 'action', and points to the complete reality. 

"And while in some cases the exercise is the ultimate thing (e.g.
in sight the ultimate thing is seeing, and no other product besides
this results from sight), but from some things a product follows (e.g.
from the art of building there results a house as well as the act
of building), yet none the less the act is in the former case the
end and in the latter more of an end than the potency is. For the
act of building is realized in the thing that is being built, and
comes to be, and is, at the same time as the house. 

"Where, then, the result is something apart from the exercise, the
actuality is in the thing that is being made, e.g. the act of building
is in the thing that is being built and that of weaving in the thing
that is being woven, and similarly in all other cases, and in general
the movement is in the thing that is being moved; but where there
is no product apart from the actuality, the actuality is present in
the agents, e.g. the act of seeing is in the seeing subject and that
of theorizing in the theorizing subject and the life is in the soul
(and therefore well-being also; for it is a certain kind of life).

"Obviously, therefore, the substance or form is actuality. According
to this argument, then, it is obvious that actuality is prior in substantial
being to potency; and as we have said, one actuality always precedes
another in time right back to the actuality of the eternal prime mover.

"But (b) actuality is prior in a stricter sense also; for eternal
things are prior in substance to perishable things, and no eternal
thing exists potentially. The reason is this. Every potency is at
one and the same time a potency of the opposite; for, while that which
is not capable of being present in a subject cannot be present, everything
that is capable of being may possibly not be actual. That, then, which
is capable of being may either be or not be; the same thing, then,
is capable both of being and of not being. And that which is capable
of not being may possibly not be; and that which may possibly not
be is perishable, either in the full sense, or in the precise sense
in which it is said that it possibly may not be, i.e. in respect either
of place or of quantity or quality; 'in the full sense' means 'in
respect of substance'. Nothing, then, which is in the full sense imperishable
is in the full sense potentially existent (though there is nothing
to prevent its being so in some respect, e.g. potentially of a certain
quality or in a certain place); all imperishable things, then, exist
actually. Nor can anything which is of necessity exist potentially;
yet these things are primary; for if these did not exist, nothing
would exist. Nor does eternal movement, if there be such, exist potentially;
and, if there is an eternal mobile, it is not in motion in virtue
of a potentiality, except in respect of 'whence' and 'whither' (there
is nothing to prevent its having matter which makes it capable of
movement in various directions). And so the sun and the stars and
the whole heaven are ever active, and there is no fear that they may
sometime stand still, as the natural philosophers fear they may. Nor
do they tire in this activity; for movement is not for them, as it
is for perishable things, connected with the potentiality for opposites,
so that the continuity of the movement should be laborious; for it
is that kind of substance which is matter and potency, not actuality,
that causes this. 

"Imperishable things are imitated by those that are involved in change,
e.g. earth and fire. For these also are ever active; for they have
their movement of themselves and in themselves. But the other potencies,
according to our previous discussion, are all potencies for opposites;
for that which can move another in this way can also move it not in
this way, i.e. if it acts according to a rational formula; and the
same non-rational potencies will produce opposite results by their
presence or absence. 

"If, then, there are any entities or substances such as the dialecticians
say the Ideas are, there must be something much more scientific than
science-itself and something more mobile than movement-itself; for
these will be more of the nature of actualities, while science-itself
and movement-itself are potencies for these. 

"Obviously, then, actuality is prior both to potency and to every
principle of change. 

Part 9 "

"That the actuality is also better and more valuable than the good
potency is evident from the following argument. Everything of which
we say that it can do something, is alike capable of contraries, e.g.
that of which we say that it can be well is the same as that which
can be ill, and has both potencies at once; for the same potency is
a potency of health and illness, of rest and motion, of building and
throwing down, of being built and being thrown down. The capacity
for contraries, then, is present at the same time; but contraries
cannot be present at the same time, and the actualities also cannot
be present at the same time, e.g. health and illness. Therefore, while
the good must be one of them, the capacity is both alike, or neither;
the actuality, then, is better. Also in the case of bad things the
end or actuality must be worse than the potency; for that which 'can'
is both contraries alike. Clearly, then, the bad does not exist apart
from bad things; for the bad is in its nature posterior to the potency.
And therefore we may also say that in the things which are from the
beginning, i.e. in eternal things, there is nothing bad, nothing defective,
nothing perverted (for perversion is something bad). 

"It is an activity also that geometrical constructions are discovered;
for we find them by dividing. If the figures had been already divided,
the constructions would have been obvious; but as it is they are present
only potentially. Why are the angles of the triangle equal to two
right angles? Because the angles about one point are equal to two
right angles. If, then, the line parallel to the side had been already
drawn upwards, the reason would have been evident to any one as soon
as he saw the figure. Why is the angle in a semicircle in all cases
a right angle? If three lines are equal the two which form the base,
and the perpendicular from the centre-the conclusion is evident at
a glance to one who knows the former proposition. Obviously, therefore,
the potentially existing constructions are discovered by being brought
to actuality; the reason is that the geometer's thinking is an actuality;
so that the potency proceeds from an actuality; and therefore it is
by making constructions that people come to know them (though the
single actuality is later in generation than the corresponding potency).
(See diagram.) 

Part 10 "

"The terms 'being' and 'non-being' are employed firstly with reference
to the categories, and secondly with reference to the potency or actuality
of these or their non-potency or nonactuality, and thirdly in the
sense of true and false. This depends, on the side of the objects,
on their being combined or separated, so that he who thinks the separated
to be separated and the combined to be combined has the truth, while
he whose thought is in a state contrary to that of the objects is
in error. This being so, when is what is called truth or falsity present,
and when is it not? We must consider what we mean by these terms.
It is not because we think truly that you are pale, that you are pale,
but because you are pale we who say this have the truth. If, then,
some things are always combined and cannot be separated, and others
are always separated and cannot be combined, while others are capable
either of combination or of separation, 'being' is being combined
and one, and 'not being' is being not combined but more than one.
Regarding contingent facts, then, the same opinion or the same statement
comes to be false and true, and it is possible for it to be at one
time correct and at another erroneous; but regarding things that cannot
be otherwise opinions are not at one time true and at another false,
but the same opinions are always true or always false. 

"But with regard to incomposites, what is being or not being, and
truth or falsity? A thing of this sort is not composite, so as to
'be' when it is compounded, and not to 'be' if it is separated, like
'that the wood is white' or 'that the diagonal is incommensurable';
nor will truth and falsity be still present in the same way as in
the previous cases. In fact, as truth is not the same in these cases,
so also being is not the same; but (a) truth or falsity is as follows--contact
and assertion are truth (assertion not being the same as affirmation),
and ignorance is non-contact. For it is not possible to be in error
regarding the question what a thing is, save in an accidental sense;
and the same holds good regarding non-composite substances (for it
is not possible to be in error about them). And they all exist actually,
not potentially; for otherwise they would have come to be and ceased
to be; but, as it is, being itself does not come to be (nor cease
to be); for if it had done so it would have had to come out of something.
About the things, then, which are essences and actualities, it is
not possible to be in error, but only to know them or not to know
them. But we do inquire what they are, viz. whether they are of such
and such a nature or not. 

"(b) As regards the 'being' that answers to truth and the 'non-being'
that answers to falsity, in one case there is truth if the subject
and the attribute are really combined, and falsity if they are not
combined; in the other case, if the object is existent it exists in
a particular way, and if it does not exist in this way does not exist
at all. And truth means knowing these objects, and falsity does not
exist, nor error, but only ignorance-and not an ignorance which is
like blindness; for blindness is akin to a total absence of the faculty
of thinking. 

"It is evident also that about unchangeable things there can be no
error in respect of time, if we assume them to be unchangeable. E.g.
if we suppose that the triangle does not change, we shall not suppose
that at one time its angles are equal to two right angles while at
another time they are not (for that would imply change). It is possible,
however, to suppose that one member of such a class has a certain
attribute and another has not; e.g. while we may suppose that no even
number is prime, we may suppose that some are and some are not. But
regarding a numerically single number not even this form of error
is possible; for we cannot in this case suppose that one instance
has an attribute and another has not, but whether our judgement be
true or false, it is implied that the fact is eternal. 

----------------------------------------------------------------------

BOOK X

Part 1 

"

"WE have said previously, in our distinction of the various meanings
of words, that 'one' has several meanings; the things that are directly
and of their own nature and not accidentally called one may be summarized
under four heads, though the word is used in more senses. (1) There
is the continuous, either in general, or especially that which is
continuous by nature and not by contact nor by being together; and
of these, that has more unity and is prior, whose movement is more
indivisible and simpler. (2) That which is a whole and has a certain
shape and form is one in a still higher degree; and especially if
a thing is of this sort by nature, and not by force like the things
which are unified by glue or nails or by being tied together, i.e.
if it has in itself the cause of its continuity. A thing is of this
sort because its movement is one and indivisible in place and time;
so that evidently if a thing has by nature a principle of movement
that is of the first kind (i.e. local movement) and the first in that
kind (i.e. circular movement), this is in the primary sense one extended
thing. Some things, then, are one in this way, qua continuous or whole,
and the other things that are one are those whose definition is one.
Of this sort are the things the thought of which is one, i.e. those
the thought of which is indivisible; and it is indivisible if the
thing is indivisible in kind or in number. (3) In number, then, the
individual is indivisible, and (4) in kind, that which in intelligibility
and in knowledge is indivisible, so that that which causes substances
to be one must be one in the primary sense. 'One', then, has all these
meanings-the naturally continuous and the whole, and the individual
and the universal. And all these are one because in some cases the
movement, in others the thought or the definition is indivisible.

"But it must be observed that the questions, what sort of things are
said to be one, and what it is to be one and what is the definition
of it, should not be assumed to be the same. 'One' has all these meanings,
and each of the things to which one of these kinds of unity belongs
will be one; but 'to be one' will sometimes mean being one of these
things, and sometimes being something else which is even nearer to
the meaning of the word 'one' while these other things approximate
to its application. This is also true of 'element' or 'cause', if
one had both to specify the things of which it is predicable and to
render the definition of the word. For in a sense fire is an element
(and doubtless also 'the indefinite' or something else of the sort
is by its own nature the element), but in a sense it is not; for it
is not the same thing to be fire and to be an element, but while as
a particular thing with a nature of its own fire is an element, the
name 'element' means that it has this attribute, that there is something
which is made of it as a primary constituent. And so with 'cause'
and 'one' and all such terms. For this reason, too, 'to be one' means
'to be indivisible, being essentially one means a "this" and capable
of being isolated either in place, or in form or thought'; or perhaps
'to be whole and indivisible'; but it means especially 'to be the
first measure of a kind', and most strictly of quantity; for it is
from this that it has been extended to the other categories. For measure
is that by which quantity is known; and quantity qua quantity is known
either by a 'one' or by a number, and all number is known by a 'one'.
Therefore all quantity qua quantity is known by the one, and that
by which quantities are primarily known is the one itself; and so
the one is the starting-point of number qua number. And hence in the
other classes too 'measure' means that by which each is first known,
and the measure of each is a unit-in length, in breadth, in depth,
in weight, in speed. (The words 'weight' and 'speed' are common to
both contraries; for each of them has two meanings-'weight' means
both that which has any amount of gravity and that which has an excess
of gravity, and 'speed' both that which has any amount of movement
and that which has an excess of movement; for even the slow has a
certain speed and the comparatively light a certain weight.)

"In all these, then, the measure and starting-point is something one
and indivisible, since even in lines we treat as indivisible the line
a foot long. For everywhere we seek as the measure something one and
indivisible; and this is that which is simple either in quality or
in quantity. Now where it is thought impossible to take away or to
add, there the measure is exact (hence that of number is most exact;
for we posit the unit as indivisible in every respect); but in all
other cases we imitate this sort of measure. For in the case of a
furlong or a talent or of anything comparatively large any addition
or subtraction might more easily escape our notice than in the case
of something smaller; so that the first thing from which, as far as
our perception goes, nothing can be subtracted, all men make the measure,
whether of liquids or of solids, whether of weight or of size; and
they think they know the quantity when they know it by means of this
measure. And indeed they know movement too by the simple movement
and the quickest; for this occupies least time. And so in astronomy
a 'one' of this sort is the starting-point and measure (for they assume
the movement of the heavens to be uniform and the quickest, and judge
the others by reference to it), and in music the quarter-tone (because
it is the least interval), and in speech the letter. And all these
are ones in this sense--not that 'one' is something predicable in
the same sense of all of these, but in the sense we have mentioned.

"But the measure is not always one in number--sometimes there are
several; e.g. the quarter-tones (not to the ear, but as determined
by the ratios) are two, and the articulate sounds by which we measure
are more than one, and the diagonal of the square and its side are
measured by two quantities, and all spatial magnitudes reveal similar
varieties of unit. Thus, then, the one is the measure of all things,
because we come to know the elements in the substance by dividing
the things either in respect of quantity or in respect of kind. And
the one is indivisible just because the first of each class of things
is indivisible. But it is not in the same way that every 'one' is
indivisible e.g. a foot and a unit; the latter is indivisible in every
respect, while the former must be placed among things which are undivided
to perception, as has been said already-only to perception, for doubtless
every continuous thing is divisible. 

"The measure is always homogeneous with the thing measured; the measure
of spatial magnitudes is a spatial magnitude, and in particular that
of length is a length, that of breadth a breadth, that of articulate
sound an articulate sound, that of weight a weight, that of units
a unit. (For we must state the matter so, and not say that the measure
of numbers is a number; we ought indeed to say this if we were to
use the corresponding form of words, but the claim does not really
correspond-it is as if one claimed that the measure of units is units
and not a unit; number is a plurality of units.) 

"Knowledge, also, and perception, we call the measure of things for
the same reason, because we come to know something by them-while as
a matter of fact they are measured rather than measure other things.
But it is with us as if some one else measured us and we came to know
how big we are by seeing that he applied the cubit-measure to such
and such a fraction of us. But Protagoras says 'man is the measure
of all things', as if he had said 'the man who knows' or 'the man
who perceives'; and these because they have respectively knowledge
and perception, which we say are the measures of objects. Such thinkers
are saying nothing, then, while they appear to be saying something
remarkable. 

"Evidently, then, unity in the strictest sense, if we define it according
to the meaning of the word, is a measure, and most properly of quantity,
and secondly of quality. And some things will be one if they are indivisible
in quantity, and others if they are indivisible in quality; and so
that which is one is indivisible, either absolutely or qua one.

Part 2 "

"With regard to the substance and nature of the one we must ask in
which of two ways it exists. This is the very question that we reviewed
in our discussion of problems, viz. what the one is and how we must
conceive of it, whether we must take the one itself as being a substance
(as both the Pythagoreans say in earlier and Plato in later times),
or there is, rather, an underlying nature and the one should be described
more intelligibly and more in the manner of the physical philosophers,
of whom one says the one is love, another says it is air, and another
the indefinite. 

"If, then, no universal can be a substance, as has been said our discussion
of substance and being, and if being itself cannot be a substance
in the sense of a one apart from the many (for it is common to the
many), but is only a predicate, clearly unity also cannot be a substance;
for being and unity are the most universal of all predicates. Therefore,
on the one hand, genera are not certain entities and substances separable
from other things; and on the other hand the one cannot be a genus,
for the same reasons for which being and substance cannot be genera.

"Further, the position must be similar in all the kinds of unity.
Now 'unity' has just as many meanings as 'being'; so that since in
the sphere of qualities the one is something definite-some particular
kind of thing-and similarly in the sphere of quantities, clearly we
must in every category ask what the one is, as we must ask what the
existent is, since it is not enough to say that its nature is just
to be one or existent. But in colours the one is a colour, e.g. white,
and then the other colours are observed to be produced out of this
and black, and black is the privation of white, as darkness of light.
Therefore if all existent things were colours, existent things would
have been a number, indeed, but of what? Clearly of colours; and the
'one' would have been a particular 'one', i.e. white. And similarly
if all existing things were tunes, they would have been a number,
but a number of quarter-tones, and their essence would not have been
number; and the one would have been something whose substance was
not to be one but to be the quarter-tone. And similarly if all existent
things had been articulate sounds, they would have been a number of
letters, and the one would have been a vowel. And if all existent
things were rectilinear figures, they would have been a number of
figures, and the one would have been the triangle. And the same argument
applies to all other classes. Since, therefore, while there are numbers
and a one both in affections and in qualities and in quantities and
in movement, in all cases the number is a number of particular things
and the one is one something, and its substance is not just to be
one, the same must be true of substances also; for it is true of all
cases alike. 

"That the one, then, in every class is a definite thing, and in no
case is its nature just this, unity, is evident; but as in colours
the one-itself which we must seek is one colour, so too in substance
the one-itself is one substance. That in a sense unity means the same
as being is clear from the facts that its meanings correspond to the
categories one to one, and it is not comprised within any category
(e.g. it is comprised neither in 'what a thing is' nor in quality,
but is related to them just as being is); that in 'one man' nothing
more is predicated than in 'man' (just as being is nothing apart from
substance or quality or quantity); and that to be one is just to be
a particular thing. 

Part 3 "

"The one and the many are opposed in several ways, of which one is
the opposition of the one and plurality as indivisible and divisible;
for that which is either divided or divisible is called a plurality,
and that which is indivisible or not divided is called one. Now since
opposition is of four kinds, and one of these two terms is privative
in meaning, they must be contraries, and neither contradictory nor
correlative in meaning. And the one derives its name and its explanation
from its contrary, the indivisible from the divisible, because plurality
and the divisible is more perceptible than the indivisible, so that
in definition plurality is prior to the indivisible, because of the
conditions of perception. 

"To the one belong, as we indicated graphically in our distinction
of the contraries, the same and the like and the equal, and to plurality
belong the other and the unlike and the unequal. 'The same' has several
meanings; (1) we sometimes mean 'the same numerically'; again, (2)
we call a thing the same if it is one both in definition and in number,
e.g. you are one with yourself both in form and in matter; and again,
(3) if the definition of its primary essence is one; e.g. equal straight
lines are the same, and so are equal and equal-angled quadrilaterals;
there are many such, but in these equality constitutes unity.

"Things are like if, not being absolutely the same, nor without difference
in respect of their concrete substance, they are the same in form;
e.g. the larger square is like the smaller, and unequal straight lines
are like; they are like, but not absolutely the same. Other things
are like, if, having the same form, and being things in which difference
of degree is possible, they have no difference of degree. Other things,
if they have a quality that is in form one and same-e.g. whiteness-in
a greater or less degree, are called like because their form is one.
Other things are called like if the qualities they have in common
are more numerous than those in which they differ-either the qualities
in general or the prominent qualities; e.g. tin is like silver, qua
white, and gold is like fire, qua yellow and red. 

"Evidently, then, 'other' and 'unlike' also have several meanings.
And the other in one sense is the opposite of the same (so that everything
is either the same as or other than everything else). In another sense
things are other unless both their matter and their definition are
one (so that you are other than your neighbour). The other in the
third sense is exemplified in the objects of mathematics. 'Other or
the same' can therefore be predicated of everything with regard to
everything else-but only if the things are one and existent, for 'other'
is not the contradictory of 'the same'; which is why it is not predicated
of non-existent things (while 'not the same' is so predicated). It
is predicated of all existing things; for everything that is existent
and one is by its very nature either one or not one with anything
else. 

"The other, then, and the same are thus opposed. But difference is
not the same as otherness. For the other and that which it is other
than need not be other in some definite respect (for everything that
is existent is either other or the same), but that which is different
is different from some particular thing in some particular respect,
so that there must be something identical whereby they differ. And
this identical thing is genus or species; for everything that differs
differs either in genus or in species, in genus if the things have
not their matter in common and are not generated out of each other
(i.e. if they belong to different figures of predication), and in
species if they have the same genus ('genus' meaning that identical
thing which is essentially predicated of both the different things).

"Contraries are different, and contrariety is a kind of difference.
That we are right in this supposition is shown by induction. For all
of these too are seen to be different; they are not merely other,
but some are other in genus, and others are in the same line of predication,
and therefore in the same genus, and the same in genus. We have distinguished
elsewhere what sort of things are the same or other in genus.

Part 4 "

"Since things which differ may differ from one another more or less,
there is also a greatest difference, and this I call contrariety.
That contrariety is the greatest difference is made clear by induction.
For things which differ in genus have no way to one another, but are
too far distant and are not comparable; and for things that differ
in species the extremes from which generation takes place are the
contraries, and the distance between extremes-and therefore that between
the contraries-is the greatest. 

"But surely that which is greatest in each class is complete. For
that is greatest which cannot be exceeded, and that is complete beyond
which nothing can be found. For the complete difference marks the
end of a series (just as the other things which are called complete
are so called because they have attained an end), and beyond the end
there is nothing; for in everything it is the extreme and includes
all else, and therefore there is nothing beyond the end, and the complete
needs nothing further. From this, then, it is clear that contrariety
is complete difference; and as contraries are so called in several
senses, their modes of completeness will answer to the various modes
of contrariety which attach to the contraries. 

"This being so, it is clear that one thing have more than one contrary
(for neither can there be anything more extreme than the extreme,
nor can there be more than two extremes for the one interval), and,
to put the matter generally, this is clear if contrariety is a difference,
and if difference, and therefore also the complete difference, must
be between two things. 

"And the other commonly accepted definitions of contraries are also
necessarily true. For not only is (1) the complete difference the
greatest difference (for we can get no difference beyond it of things
differing either in genus or in species; for it has been shown that
there is no 'difference' between anything and the things outside its
genus, and among the things which differ in species the complete difference
is the greatest); but also (2) the things in the same genus which
differ most are contrary (for the complete difference is the greatest
difference between species of the same genus); and (3) the things
in the same receptive material which differ most are contrary (for
the matter is the same for contraries); and (4) of the things which
fall under the same faculty the most different are contrary (for one
science deals with one class of things, and in these the complete
difference is the greatest). 

"The primary contrariety is that between positive state and privation-not
every privation, however (for 'privation' has several meanings), but
that which is complete. And the other contraries must be called so
with reference to these, some because they possess these, others because
they produce or tend to produce them, others because they are acquisitions
or losses of these or of other contraries. Now if the kinds of opposition
are contradiction and privation and contrariety and relation, and
of these the first is contradiction, and contradiction admits of no
intermediate, while contraries admit of one, clearly contradiction
and contrariety are not the same. But privation is a kind of contradiction;
for what suffers privation, either in general or in some determinate
way, either that which is quite incapable of having some attribute
or that which, being of such a nature as to have it, has it not; here
we have already a variety of meanings, which have been distinguished
elsewhere. Privation, therefore, is a contradiction or incapacity
which is determinate or taken along with the receptive material. This
is the reason why, while contradiction does not admit of an intermediate,
privation sometimes does; for everything is equal or not equal, but
not everything is equal or unequal, or if it is, it is only within
the sphere of that which is receptive of equality. If, then, the comings-to-be
which happen to the matter start from the contraries, and proceed
either from the form and the possession of the form or from a privation
of the form or shape, clearly all contrariety must be privation, but
presumably not all privation is contrariety (the reason being that
that has suffered privation may have suffered it in several ways);
for it is only the extremes from which changes proceed that are contraries.

"And this is obvious also by induction. For every contrariety involves,
as one of its terms, a privation, but not all cases are alike; inequality
is the privation of equality and unlikeness of likeness, and on the
other hand vice is the privation of virtue. But the cases differ in
a way already described; in one case we mean simply that the thing
has suffered privation, in another case that it has done so either
at a certain time or in a certain part (e.g. at a certain age or in
the dominant part), or throughout. This is why in some cases there
is a mean (there are men who are neither good nor bad), and in others
there is not (a number must be either odd or even). Further, some
contraries have their subject defined, others have not. Therefore
it is evident that one of the contraries is always privative; but
it is enough if this is true of the first-i.e. the generic-contraries,
e.g. the one and the many; for the others can be reduced to these.

Part 5 "

"Since one thing has one contrary, we might raise the question how
the one is opposed to the many, and the equal to the great and the
small. For if we used the word 'whether' only in an antithesis such
as 'whether it is white or black', or 'whether it is white or not
white' (we do not ask 'whether it is a man or white'), unless we are
proceeding on a prior assumption and asking something such as 'whether
it was Cleon or Socrates that came' as this is not a necessary disjunction
in any class of things; yet even this is an extension from the case
of opposites; for opposites alone cannot be present together; and
we assume this incompatibility here too in asking which of the two
came; for if they might both have come, the question would have been
absurd; but if they might, even so this falls just as much into an
antithesis, that of the 'one or many', i.e. 'whether both came or
one of the two':-if, then, the question 'whether' is always concerned
with opposites, and we can ask 'whether it is greater or less or equal',
what is the opposition of the equal to the other two? It is not contrary
either to one alone or to both; for why should it be contrary to the
greater rather than to the less? Further, the equal is contrary to
the unequal. Therefore if it is contrary to the greater and the less,
it will be contrary to more things than one. But if the unequal means
the same as both the greater and the less together, the equal will
be opposite to both (and the difficulty supports those who say the
unequal is a 'two'), but it follows that one thing is contrary to
two others, which is impossible. Again, the equal is evidently intermediate
between the great and the small, but no contrariety is either observed
to be intermediate, or, from its definition, can be so; for it would
not be complete if it were intermediate between any two things, but
rather it always has something intermediate between its own terms.

"It remains, then, that it is opposed either as negation or as privation.
It cannot be the negation or privation of one of the two; for why
of the great rather than of the small? It is, then, the privative
negation of both. This is why 'whether' is said with reference to
both, not to one of the two (e.g. 'whether it is greater or equal'
or 'whether it is equal or less'); there are always three cases. But
it is not a necessary privation; for not everything which is not greater
or less is equal, but only the things which are of such a nature as
to have these attributes. 

"The equal, then, is that which is neither great nor small but is
naturally fitted to be either great or small; and it is opposed to
both as a privative negation (and therefore is also intermediate).
And that which is neither good nor bad is opposed to both, but has
no name; for each of these has several meanings and the recipient
subject is not one; but that which is neither white nor black has
more claim to unity. Yet even this has not one name, though the colours
of which this negation is privatively predicated are in a way limited;
for they must be either grey or yellow or something else of the kind.
Therefore it is an incorrect criticism that is passed by those who
think that all such phrases are used in the same way, so that that
which is neither a shoe nor a hand would be intermediate between a
shoe and a hand, since that which is neither good nor bad is intermediate
between the good and the bad-as if there must be an intermediate in
all cases. But this does not necessarily follow. For the one phrase
is a joint denial of opposites between which there is an intermediate
and a certain natural interval; but between the other two there is
no 'difference'; for the things, the denials of which are combined,
belong to different classes, so that the substratum is not one.

Part 6 "

"We might raise similar questions about the one and the many. For
if the many are absolutely opposed to the one, certain impossible
results follow. One will then be few, whether few be treated here
as singular or plural; for the many are opposed also to the few. Further,
two will be many, since the double is multiple and 'double' derives
its meaning from 'two'; therefore one will be few; for what is that
in comparison with which two are many, except one, which must therefore
be few? For there is nothing fewer. Further, if the much and the little
are in plurality what the long and the short are in length, and whatever
is much is also many, and the many are much (unless, indeed, there
is a difference in the case of an easily-bounded continuum), the little
(or few) will be a plurality. Therefore one is a plurality if it is
few; and this it must be, if two are many. But perhaps, while the
'many' are in a sense said to be also 'much', it is with a difference;
e.g. water is much but not many. But 'many' is applied to the things
that are divisible; in the one sense it means a plurality which is
excessive either absolutely or relatively (while 'few' is similarly
a plurality which is deficient), and in another sense it means number,
in which sense alone it is opposed to the one. For we say 'one or
many', just as if one were to say 'one and ones' or 'white thing and
white things', or to compare the things that have been measured with
the measure. It is in this sense also that multiples are so called.
For each number is said to be many because it consists of ones and
because each number is measurable by one; and it is 'many' as that
which is opposed to one, not to the few. In this sense, then, even
two is many-not, however, in the sense of a plurality which is excessive
either relatively or absolutely; it is the first plurality. But without
qualification two is few; for it is first plurality which is deficient
(for this reason Anaxagoras was not right in leaving the subject with
the statement that 'all things were together, boundless both in plurality
and in smallness'-where for 'and in smallness' he should have said
'and in fewness'; for they could not have been boundless in fewness),
since it is not one, as some say, but two, that make a few.

"The one is opposed then to the many in numbers as measure to thing
measurable; and these are opposed as are the relatives which are not
from their very nature relatives. We have distinguished elsewhere
the two senses in which relatives are so called:-(1) as contraries;
(2) as knowledge to thing known, a term being called relative because
another is relative to it. There is nothing to prevent one from being
fewer than something, e.g. than two; for if one is fewer, it is not
therefore few. Plurality is as it were the class to which number belongs;
for number is plurality measurable by one, and one and number are
in a sense opposed, not as contrary, but as we have said some relative
terms are opposed; for inasmuch as one is measure and the other measurable,
they are opposed. This is why not everything that is one is a number;
i.e. if the thing is indivisible it is not a number. But though knowledge
is similarly spoken of as relative to the knowable, the relation does
not work out similarly; for while knowledge might be thought to be
the measure, and the knowable the thing measured, the fact that all
knowledge is knowable, but not all that is knowable is knowledge,
because in a sense knowledge is measured by the knowable.-Plurality
is contrary neither to the few (the many being contrary to this as
excessive plurality to plurality exceeded), nor to the one in every
sense; but in the one sense these are contrary, as has been said,
because the former is divisible and the latter indivisible, while
in another sense they are relative as knowledge is to knowable, if
plurality is number and the one is a measure. 

Part 7 "

"Since contraries admit of an intermediate and in some cases have
it, intermediates must be composed of the contraries. For (1) all
intermediates are in the same genus as the things between which they
stand. For we call those things intermediates, into which that which
changes must change first; e.g. if we were to pass from the highest
string to the lowest by the smallest intervals, we should come sooner
to the intermediate notes, and in colours if we were to pass from
white to black, we should come sooner to crimson and grey than to
black; and similarly in all other cases. But to change from one genus
to another genus is not possible except in an incidental way, as from
colour to figure. Intermediates, then, must be in the same genus both
as one another and as the things they stand between. 

"But (2) all intermediates stand between opposites of some kind; for
only between these can change take place in virtue of their own nature
(so that an intermediate is impossible between things which are not
opposite; for then there would be change which was not from one opposite
towards the other). Of opposites, contradictories admit of no middle
term; for this is what contradiction is-an opposition, one or other
side of which must attach to anything whatever, i.e. which has no
intermediate. Of other opposites, some are relative, others privative,
others contrary. Of relative terms, those which are not contrary have
no intermediate; the reason is that they are not in the same genus.
For what intermediate could there be between knowledge and knowable?
But between great and small there is one. 

"(3) If intermediates are in the same genus, as has been shown, and
stand between contraries, they must be composed of these contraries.
For either there will be a genus including the contraries or there
will be none. And if (a) there is to be a genus in such a way that
it is something prior to the contraries, the differentiae which constituted
the contrary species-of-a-genus will be contraries prior to the species;
for species are composed of the genus and the differentiae. (E.g.
if white and black are contraries, and one is a piercing colour and
the other a compressing colour, these differentiae-'piercing' and
'compressing'-are prior; so that these are prior contraries of one
another.) But, again, the species which differ contrariwise are the
more truly contrary species. And the other.species, i.e. the intermediates,
must be composed of their genus and their differentiae. (E.g. all
colours which are between white and black must be said to be composed
of the genus, i.e. colour, and certain differentiae. But these differentiae
will not be the primary contraries; otherwise every colour would be
either white or black. They are different, then, from the primary
contraries; and therefore they will be between the primary contraries;
the primary differentiae are 'piercing' and 'compressing'.)

"Therefore it is (b) with regard to these contraries which do not
fall within a genus that we must first ask of what their intermediates
are composed. (For things which are in the same genus must be composed
of terms in which the genus is not an element, or else be themselves
incomposite.) Now contraries do not involve one another in their composition,
and are therefore first principles; but the intermediates are either
all incomposite, or none of them. But there is something compounded
out of the contraries, so that there can be a change from a contrary
to it sooner than to the other contrary; for it will have less of
the quality in question than the one contrary and more than the other.
This also, then, will come between the contraries. All the other intermediates
also, therefore, are composite; for that which has more of a quality
than one thing and less than another is compounded somehow out of
the things than which it is said to have more and less respectively
of the quality. And since there are no other things prior to the contraries
and homogeneous with the intermediates, all intermediates must be
compounded out of the contraries. Therefore also all the inferior
classes, both the contraries and their intermediates, will be compounded
out of the primary contraries. Clearly, then, intermediates are (1)
all in the same genus and (2) intermediate between contraries, and
(3) all compounded out of the contraries. 

Part 8 "

"That which is other in species is other than something in something,
and this must belong to both; e.g. if it is an animal other in species,
both are animals. The things, then, which are other in species must
be in the same genus. For by genus I mean that one identical thing
which is predicated of both and is differentiated in no merely accidental
way, whether conceived as matter or otherwise. For not only must the
common nature attach to the different things, e.g. not only must both
be animals, but this very animality must also be different for each
(e.g. in the one case equinity, in the other humanity), and so this
common nature is specifically different for each from what it is for
the other. One, then, will be in virtue of its own nature one sort
of animal, and the other another, e.g. one a horse and the other a
man. This difference, then, must be an otherness of the genus. For
I give the name of 'difference in the genus' an otherness which makes
the genus itself other. 

"This, then, will be a contrariety (as can be shown also by induction).
For all things are divided by opposites, and it has been proved that
contraries are in the same genus. For contrariety was seen to be complete
difference; and all difference in species is a difference from something
in something; so that this is the same for both and is their genus.
(Hence also all contraries which are different in species and not
in genus are in the same line of predication, and other than one another
in the highest degree-for the difference is complete-, and cannot
be present along with one another.) The difference, then, is a contrariety.

"This, then, is what it is to be 'other in species'-to have a contrariety,
being in the same genus and being indivisible (and those things are
the same in species which have no contrariety, being indivisible);
we say 'being indivisible', for in the process of division contrarieties
arise in the intermediate stages before we come to the indivisibles.
Evidently, therefore, with reference to that which is called the genus,
none of the species-of-a-genus is either the same as it or other than
it in species (and this is fitting; for the matter is indicated by
negation, and the genus is the matter of that of which it is called
the genus, not in the sense in which we speak of the genus or family
of the Heraclidae, but in that in which the genus is an element in
a thing's nature), nor is it so with reference to things which are
not in the same genus, but it will differ in genus from them, and
in species from things in the same genus. For a thing's difference
from that from which it differs in species must be a contrariety;
and this belongs only to things in the same genus. 

Part 9 "

"One might raise the question, why woman does not differ from man
in species, when female and male are contrary and their difference
is a contrariety; and why a female and a male animal are not different
in species, though this difference belongs to animal in virtue of
its own nature, and not as paleness or darkness does; both 'female'
and 'male' belong to it qua animal. This question is almost the same
as the other, why one contrariety makes things different in species
and another does not, e.g. 'with feet' and 'with wings' do, but paleness
and darkness do not. Perhaps it is because the former are modifications
peculiar to the genus, and the latter are less so. And since one element
is definition and one is matter, contrarieties which are in the definition
make a difference in species, but those which are in the thing taken
as including its matter do not make one. And so paleness in a man,
or darkness, does not make one, nor is there a difference in species
between the pale man and the dark man, not even if each of them be
denoted by one word. For man is here being considered on his material
side, and matter does not create a difference; for it does not make
individual men species of man, though the flesh and the bones of which
this man and that man consist are other. The concrete thing is other,
but not other in species, because in the definition there is no contrariety.
This is the ultimate indivisible kind. Callias is definition + matter,
the pale man, then, is so also, because it is the individual Callias
that is pale; man, then, is pale only incidentally. Neither do a brazen
and a wooden circle, then, differ in species; and if a brazen triangle
and a wooden circle differ in species, it is not because of the matter,
but because there is a contrariety in the definition. But does the
matter not make things other in species, when it is other in a certain
way, or is there a sense in which it does? For why is this horse other
than this man in species, although their matter is included with their
definitions? Doubtless because there is a contrariety in the definition.
For while there is a contrariety also between pale man and dark horse,
and it is a contrariety in species, it does not depend on the paleness
of the one and the darkness of the other, since even if both had been
pale, yet they would have been other in species. But male and female,
while they are modifications peculiar to 'animal', are so not in virtue
of its essence but in the matter, ie. the body. This is why the same
seed becomes female or male by being acted on in a certain way. We
have stated, then, what it is to be other in species, and why some
things differ in species and others do not. 

Part 10 "

"Since contraries are other in form, and the perishable and the imperishable
are contraries (for privation is a determinate incapacity), the perishable
and the imperishable must be different in kind. 

"Now so far we have spoken of the general terms themselves, so that
it might be thought not to be necessary that every imperishable thing
should be different from every perishable thing in form, just as not
every pale thing is different in form from every dark thing. For the
same thing can be both, and even at the same time if it is a universal
(e.g. man can be both pale and dark), and if it is an individual it
can still be both; for the same man can be, though not at the same
time, pale and dark. Yet pale is contrary to dark. 

"But while some contraries belong to certain things by accident (e.g.
both those now mentioned and many others), others cannot, and among
these are 'perishable' and 'imperishable'. For nothing is by accident
perishable. For what is accidental is capable of not being present,
but perishableness is one of the attributes that belong of necessity
to the things to which they belong; or else one and the same thing
may be perishable and imperishable, if perishableness is capable of
not belonging to it. Perishableness then must either be the essence
or be present in the essence of each perishable thing. The same account
holds good for imperishableness also; for both are attributes which
are present of necessity. The characteristics, then, in respect of
which and in direct consequence of which one thing is perishable and
another imperishable, are opposite, so that the things must be different
in kind. 

"Evidently, then, there cannot be Forms such as some maintain, for
then one man would be perishable and another imperishable. Yet the
Forms are said to be the same in form with the individuals and not
merely to have the same name; but things which differ in kind are
farther apart than those which differ in form. 

----------------------------------------------------------------------

BOOK XI

Part 1 

"

"THAT Wisdom is a science of first principles is evident from the
introductory chapters, in which we have raised objections to the statements
of others about the first principles; but one might ask the question
whether Wisdom is to be conceived as one science or as several. If
as one, it may be objected that one science always deals with contraries,
but the first principles are not contrary. If it is not one, what
sort of sciences are those with which it is to be identified?

"Further, is it the business of one science, or of more than one,
to examine the first principles of demonstration? If of one, why of
this rather than of any other? If of more, what sort of sciences must
these be said to be? 

"Further, does Wisdom investigate all substances or not? If not all,
it is hard to say which; but if, being one, it investigates them all,
it is doubtful how the same science can embrace several subject-matters.

"Further, does it deal with substances only or also with their attributes?
If in the case of attributes demonstration is possible, in that of
substances it is not. But if the two sciences are different, what
is each of them and which is Wisdom? If we think of it as demonstrative,
the science of the attributes is Wisdom, but if as dealing with what
is primary, the science of substances claims the tide. 

"But again the science we are looking for must not be supposed to
deal with the causes which have been mentioned in the Physics. For
(A) it does not deal with the final cause (for that is the nature
of the good, and this is found in the field of action and movement;
and it is the first mover-for that is the nature of the end-but in
the case of things unmovable there is nothing that moved them first),
and (B) in general it is hard to say whether perchance the science
we are now looking for deals with perceptible substances or not with
them, but with certain others. If with others, it must deal either
with the Forms or with the objects of mathematics. Now (a) evidently
the Forms do not exist. (But it is hard to say, even if one suppose
them to exist, why in the world the same is not true of the other
things of which there are Forms, as of the objects of mathematics.
I mean that these thinkers place the objects of mathematics between
the Forms and perceptible things, as a kind of third set of things
apart both from the Forms and from the things in this world; but there
is not a third man or horse besides the ideal and the individuals.
If on the other hand it is not as they say, with what sort of things
must the mathematician be supposed to deal? Certainly not with the
things in this world; for none of these is the sort of thing which
the mathematical sciences demand.) Nor (b) does the science which
we are now seeking treat of the objects of mathematics; for none of
them can exist separately. But again it does not deal with perceptible
substances; for they are perishable. 

"In general one might raise the question, to what kind of science
it belongs to discuss the difficulties about the matter of the objects
of mathematics. Neither to physics (because the whole inquiry of the
physicist is about the things that have in themselves a principle.
of movement and rest), nor yet to the science which inquires into
demonstration and science; for this is just the subject which it investigates.
It remains then that it is the philosophy which we have set before
ourselves that treats of those subjects. 

"One might discuss the question whether the science we are seeking
should be said to deal with the principles which are by some called
elements; all men suppose these to be present in composite things.
But it might be thought that the science we seek should treat rather
of universals; for every definition and every science is of universals
and not of infimae species, so that as far as this goes it would deal
with the highest genera. These would turn out to be being and unity;
for these might most of all be supposed to contain all things that
are, and to be most like principles because they are by nature; for
if they perish all other things are destroyed with them; for everything
is and is one. But inasmuch as, if one is to suppose them to be genera,
they must be predicable of their differentiae, and no genus is predicable
of any of its differentiae, in this way it would seem that we should
not make them genera nor principles. Further, if the simpler is more
of a principle than the less simple, and the ultimate members of the
genus are simpler than the genera (for they are indivisible, but the
genera are divided into many and differing species), the species might
seem to be the principles, rather than the genera. But inasmuch as
the species are involved in the destruction of the genera, the genera
are more like principles; for that which involves another in its destruction
is a principle of it. These and others of the kind are the subjects
that involve difficulties. 

Part 2 "

"Further, must we suppose something apart from individual things,
or is it these that the science we are seeking treats of? But these
are infinite in number. Yet the things that are apart from the individuals
are genera or species; but the science we now seek treats of neither
of these. The reason why this is impossible has been stated. Indeed,
it is in general hard to say whether one must assume that there is
a separable substance besides the sensible substances (i.e. the substances
in this world), or that these are the real things and Wisdom is concerned
with them. For we seem to seek another kind of substance, and this
is our problem, i.e. to see if there is something which can exist
apart by itself and belongs to no sensible thing.-Further, if there
is another substance apart from and corresponding to sensible substances,
which kinds of sensible substance must be supposed to have this corresponding
to them? Why should one suppose men or horses to have it, more than
either the other animals or even all lifeless things? On the other
hand to set up other and eternal substances equal in number to the
sensible and perishable substances would seem to fall beyond the bounds
of probability.-But if the principle we now seek is not separable
from corporeal things, what has a better claim to the name matter?
This, however, does not exist in actuality, but exists in potency.
And it would seem rather that the form or shape is a more important
principle than this; but the form is perishable, so that there is
no eternal substance at all which can exist apart and independent.
But this is paradoxical; for such a principle and substance seems
to exist and is sought by nearly all the most refined thinkers as
something that exists; for how is there to be order unless there is
something eternal and independent and permanent? 

"Further, if there is a substance or principle of such a nature as
that which we are now seeking, and if this is one for all things,
and the same for eternal and for perishable things, it is hard to
say why in the world, if there is the same principle, some of the
things that fall under the principle are eternal, and others are not
eternal; this is paradoxical. But if there is one principle of perishable
and another of eternal things, we shall be in a like difficulty if
the principle of perishable things, as well as that of eternal, is
eternal; for why, if the principle is eternal, are not the things
that fall under the principle also eternal? But if it is perishable
another principle is involved to account for it, and another to account
for that, and this will go on to infinity. 

"If on the other hand we are to set up what are thought to be the
most unchangeable principles, being and unity, firstly, if each of
these does not indicate a 'this' or substance, how will they be separable
and independent? Yet we expect the eternal and primary principles
to be so. But if each of them does signify a 'this' or substance,
all things that are are substances; for being is predicated of all
things (and unity also of some); but that all things that are are
substance is false. Further, how can they be right who say that the
first principle is unity and this is substance, and generate number
as the first product from unity and from matter, assert that number
is substance? How are we to think of 'two', and each of the other
numbers composed of units, as one? On this point neither do they say
anything nor is it easy to say anything. But if we are to suppose
lines or what comes after these (I mean the primary surfaces) to be
principles, these at least are not separable substances, but sections
and divisions-the former of surfaces, the latter of bodies (while
points are sections and divisions of lines); and further they are
limits of these same things; and all these are in other things and
none is separable. Further, how are we to suppose that there is a
substance of unity and the point? Every substance comes into being
by a gradual process, but a point does not; for the point is a division.

"A further difficulty is raised by the fact that all knowledge is
of universals and of the 'such', but substance is not a universal,
but is rather a 'this'-a separable thing, so that if there is knowledge
about the first principles, the question arises, how are we to suppose
the first principle to be substance? 

"Further, is there anything apart from the concrete thing (by which
I mean the matter and that which is joined with it), or not? If not,
we are met by the objection that all things that are in matter are
perishable. But if there is something, it must be the form or shape.
Now it is hard to determine in which cases this exists apart and in
which it does not; for in some cases the form is evidently not separable,
e.g. in the case of a house. 

"Further, are the principles the same in kind or in number? If they
are one in number, all things will be the same. 

Part 3 "

"Since the science of the philosopher treats of being qua being universally
and not in respect of a part of it, and 'being' has many senses and
is not used in one only, it follows that if the word is used equivocally
and in virtue of nothing common to its various uses, being does not
fall under one science (for the meanings of an equivocal term do not
form one genus); but if the word is used in virtue of something common,
being will fall under one science. The term seems to be used in the
way we have mentioned, like 'medical' and 'healthy'. For each of these
also we use in many senses. Terms are used in this way by virtue of
some kind of reference, in the one case to medical science, in the
other to health, in others to something else, but in each case to
one identical concept. For a discussion and a knife are called medical
because the former proceeds from medical science, and the latter is
useful to it. And a thing is called healthy in a similar way; one
thing because it is indicative of health, another because it is productive
of it. And the same is true in the other cases. Everything that is,
then, is said to 'be' in this same way; each thing that is is said
to 'be' because it is a modification of being qua being or a permanent
or a transient state or a movement of it, or something else of the
sort. And since everything that is may be referred to something single
and common, each of the contrarieties also may be referred to the
first differences and contrarieties of being, whether the first differences
of being are plurality and unity, or likeness and unlikeness, or some
other differences; let these be taken as already discussed. It makes
no difference whether that which is be referred to being or to unity.
For even if they are not the same but different, at least they are
convertible; for that which is one is also somehow being, and that
which is being is one. 

"But since every pair of contraries falls to be examined by one and
the same science, and in each pair one term is the privative of the
other though one might regarding some contraries raise the question,
how they can be privately related, viz. those which have an intermediate,
e.g. unjust and just-in all such cases one must maintain that the
privation is not of the whole definition, but of the infima species.
if the just man is 'by virtue of some permanent disposition obedient
to the laws', the unjust man will not in every case have the whole
definition denied of him, but may be merely 'in some respect deficient
in obedience to the laws', and in this respect the privation will
attach to him; and similarly in all other cases. 

"As the mathematician investigates abstractions (for before beginning
his investigation he strips off all the sensible qualities, e.g. weight
and lightness, hardness and its contrary, and also heat and cold and
the other sensible contrarieties, and leaves only the quantitative
and continuous, sometimes in one, sometimes in two, sometimes in three
dimensions, and the attributes of these qua quantitative and continuous,
and does not consider them in any other respect, and examines the
relative positions of some and the attributes of these, and the commensurabilities
and incommensurabilities of others, and the ratios of others; but
yet we posit one and the same science of all these things--geometry)--the
same is true with regard to being. For the attributes of this in so
far as it is being, and the contrarieties in it qua being, it is the
business of no other science than philosophy to investigate; for to
physics one would assign the study of things not qua being, but rather
qua sharing in movement; while dialectic and sophistic deal with the
attributes of things that are, but not of things qua being, and not
with being itself in so far as it is being; therefore it remains that
it is the philosopher who studies the things we have named, in so
far as they are being. Since all that is is to 'be' in virtue of something
single and common, though the term has many meanings, and contraries
are in the same case (for they are referred to the first contrarieties
and differences of being), and things of this sort can fall under
one science, the difficulty we stated at the beginning appears to
be solved,-I mean the question how there can be a single science of
things which are many and different in genus. 

Part 4 "

"Since even the mathematician uses the common axioms only in a special
application, it must be the business of first philosophy to examine
the principles of mathematics also. That when equals are taken from
equals the remainders are equal, is common to all quantities, but
mathematics studies a part of its proper matter which it has detached,
e.g. lines or angles or numbers or some other kind of quantity-not,
however, qua being but in so far as each of them is continuous in
one or two or three dimensions; but philosophy does not inquire about
particular subjects in so far as each of them has some attribute or
other, but speculates about being, in so far as each particular thing
is.-Physics is in the same position as mathematics; for physics studies
the attributes and the principles of the things that are, qua moving
and not qua being (whereas the primary science, we have said, deals
with these, only in so far as the underlying subjects are existent,
and not in virtue of any other character); and so both physics and
mathematics must be classed as parts of Wisdom. 

Part 5 "

"There is a principle in things, about which we cannot be deceived,
but must always, on the contrary recognize the truth,-viz. that the
same thing cannot at one and the same time be and not be, or admit
any other similar pair of opposites. About such matters there is no
proof in the full sense, though there is proof ad hominem. For it
is not possible to infer this truth itself from a more certain principle,
yet this is necessary if there is to be completed proof of it in the
full sense. But he who wants to prove to the asserter of opposites
that he is wrong must get from him an admission which shall be identical
with the principle that the same thing cannot be and not be at one
and the same time, but shall not seem to be identical; for thus alone
can his thesis be demonstrated to the man who asserts that opposite
statements can be truly made about the same subject. Those, then,
who are to join in argument with one another must to some extent understand
one another; for if this does not happen how are they to join in argument
with one another? Therefore every word must be intelligible and indicate
something, and not many things but only one; and if it signifies more
than one thing, it must be made plain to which of these the word is
being applied. He, then, who says 'this is and is not' denies what
he affirms, so that what the word signifies, he says it does not signify;
and this is impossible. Therefore if 'this is' signifies something,
one cannot truly assert its contradictory. 

"Further, if the word signifies something and this is asserted truly,
this connexion must be necessary; and it is not possible that that
which necessarily is should ever not be; it is not possible therefore
to make the opposed affirmations and negations truly of the same subject.
Further, if the affirmation is no more true than the negation, he
who says 'man' will be no more right than he who says 'not-man'. It
would seem also that in saying the man is not a horse one would be
either more or not less right than in saying he is not a man, so that
one will also be right in saying that the same person is a horse;
for it was assumed to be possible to make opposite statements equally
truly. It follows then that the same person is a man and a horse,
or any other animal. 

"While, then, there is no proof of these things in the full sense,
there is a proof which may suffice against one who will make these
suppositions. And perhaps if one had questioned Heraclitus himself
in this way one might have forced him to confess that opposite statements
can never be true of the same subjects. But, as it is, he adopted
this opinion without understanding what his statement involves. But
in any case if what is said by him is true, not even this itself will
be true-viz. that the same thing can at one and the same time both
be and not be. For as, when the statements are separated, the affirmation
is no more true than the negation, in the same way-the combined and
complex statement being like a single affirmation-the whole taken
as an affirmation will be no more true than the negation. Further,
if it is not possible to affirm anything truly, this itself will be
false-the assertion that there is no true affirmation. But if a true
affirmation exists, this appears to refute what is said by those who
raise such objections and utterly destroy rational discourse.

Part 6 "

"The saying of Protagoras is like the views we have mentioned; he
said that man is the measure of all things, meaning simply that that
which seems to each man also assuredly is. If this is so, it follows
that the same thing both is and is not, and is bad and good, and that
the contents of all other opposite statements are true, because often
a particular thing appears beautiful to some and the contrary of beautiful
to others, and that which appears to each man is the measure. This
difficulty may be solved by considering the source of this opinion.
It seems to have arisen in some cases from the doctrine of the natural
philosophers, and in others from the fact that all men have not the
same views about the same things, but a particular thing appears pleasant
to some and the contrary of pleasant to others. 

"That nothing comes to be out of that which is not, but everything
out of that which is, is a dogma common to nearly all the natural
philosophers. Since, then, white cannot come to be if the perfectly
white and in no respect not-white existed before, that which becomes
white must come from that which is not white; so that it must come
to be out of that which is not (so they argue), unless the same thing
was at the beginning white and not-white. But it is not hard to solve
this difficulty; for we have said in our works on physics in what
sense things that come to be come to be from that which is not, and
in what sense from that which is. 

"But to attend equally to the opinions and the fancies of disputing
parties is childish; for clearly one of them must be mistaken. And
this is evident from what happens in respect of sensation; for the
same thing never appears sweet to some and the contrary of sweet to
others, unless in the one case the sense-organ which discriminates
the aforesaid flavours has been perverted and injured. And if this
is so the one party must be taken to be the measure, and the other
must not. And say the same of good and bad, and beautiful and ugly,
and all other such qualities. For to maintain the view we are opposing
is just like maintaining that the things that appear to people who
put their finger under their eye and make the object appear two instead
of one must be two (because they appear to be of that number) and
again one (for to those who do not interfere with their eye the one
object appears one). 

"In general, it is absurd to make the fact that the things of this
earth are observed to change and never to remain in the same state,
the basis of our judgement about the truth. For in pursuing the truth
one must start from the things that are always in the same state and
suffer no change. Such are the heavenly bodies; for these do not appear
to be now of one nature and again of another, but are manifestly always
the same and share in no change. 

"Further, if there is movement, there is also something moved, and
everything is moved out of something and into something; it follows
that that that which is moved must first be in that out of which it
is to be moved, and then not be in it, and move into the other and
come to be in it, and that the contradictory statements are not true
at the same time, as these thinkers assert they are. 

"And if the things of this earth continuously flow and move in respect
of quantity-if one were to suppose this, although it is not true-why
should they not endure in respect of quality? For the assertion of
contradictory statements about the same thing seems to have arisen
largely from the belief that the quantity of bodies does not endure,
which, our opponents hold, justifies them in saying that the same
thing both is and is not four cubits long. But essence depends on
quality, and this is of determinate nature, though quantity is of
indeterminate. 

"Further, when the doctor orders people to take some particular food,
why do they take it? In what respect is 'this is bread' truer than
'this is not bread'? And so it would make no difference whether one
ate or not. But as a matter of fact they take the food which is ordered,
assuming that they know the truth about it and that it is bread. Yet
they should not, if there were no fixed constant nature in sensible
things, but all natures moved and flowed for ever. 

"Again, if we are always changing and never remain the same, what
wonder is it if to us, as to the sick, things never appear the same?
(For to them also, because they are not in the same condition as when
they were well, sensible qualities do not appear alike; yet, for all
that, the sensible things themselves need not share in any change,
though they produce different, and not identical, sensations in the
sick. And the same must surely happen to the healthy if the afore-said
change takes place.) But if we do not change but remain the same,
there will be something that endures. 

"As for those to whom the difficulties mentioned are suggested by
reasoning, it is not easy to solve the difficulties to their satisfaction,
unless they will posit something and no longer demand a reason for
it; for it is only thus that all reasoning and all proof is accomplished;
if they posit nothing, they destroy discussion and all reasoning.
Therefore with such men there is no reasoning. But as for those who
are perplexed by the traditional difficulties, it is easy to meet
them and to dissipate the causes of their perplexity. This is evident
from what has been said. 

"It is manifest, therefore, from these arguments that contradictory
statements cannot be truly made about the same subject at one time,
nor can contrary statements, because every contrariety depends on
privation. This is evident if we reduce the definitions of contraries
to their principle. 

"Similarly, no intermediate between contraries can be predicated of
one and the same subject, of which one of the contraries is predicated.
If the subject is white we shall be wrong in saying it is neither
black nor white, for then it follows that it is and is not white;
for the second of the two terms we have put together is true of it,
and this is the contradictory of white. 

"We could not be right, then, in accepting the views either of Heraclitus
or of Anaxagoras. If we were, it would follow that contraries would
be predicated of the same subject; for when Anaxagoras says that in
everything there is a part of everything, he says nothing is sweet
any more than it is bitter, and so with any other pair of contraries,
since in everything everything is present not potentially only, but
actually and separately. And similarly all statements cannot be false
nor all true, both because of many other difficulties which might
be adduced as arising from this position, and because if all are false
it will not be true to say even this, and if all are true it will
not be false to say all are false. 

Part 7 "

"Every science seeks certain principles and causes for each of its
objects-e.g. medicine and gymnastics and each of the other sciences,
whether productive or mathematical. For each of these marks off a
certain class of things for itself and busies itself about this as
about something existing and real,-not however qua real; the science
that does this is another distinct from these. Of the sciences mentioned
each gets somehow the 'what' in some class of things and tries to
prove the other truths, with more or less precision. Some get the
'what' through perception, others by hypothesis; so that it is clear
from an induction of this sort that there is no demonstration. of
the substance or 'what'. 

"There is a science of nature, and evidently it must be different
both from practical and from productive science. For in the case of
productive science the principle of movement is in the producer and
not in the product, and is either an art or some other faculty. And
similarly in practical science the movement is not in the thing done,
but rather in the doers. But the science of the natural philosopher
deals with the things that have in themselves a principle of movement.
It is clear from these facts, then, that natural science must be neither
practical nor productive, but theoretical (for it must fall into some
one of these classes). And since each of the sciences must somehow
know the 'what' and use this as a principle, we must not fall to observe
how the natural philosopher should define things and how he should
state the definition of the essence-whether as akin to 'snub' or rather
to 'concave'. For of these the definition of 'snub' includes the matter
of the thing, but that of 'concave' is independent of the matter;
for snubness is found in a nose, so that we look for its definition
without eliminating the nose, for what is snub is a concave nose.
Evidently then the definition of flesh also and of the eye and of
the other parts must always be stated without eliminating the matter.

"Since there is a science of being qua being and capable of existing
apart, we must consider whether this is to be regarded as the same
as physics or rather as different. Physics deals with the things that
have a principle of movement in themselves; mathematics is theoretical,
and is a science that deals with things that are at rest, but its
subjects cannot exist apart. Therefore about that which can exist
apart and is unmovable there is a science different from both of these,
if there is a substance of this nature (I mean separable and unmovable),
as we shall try to prove there is. And if there is such a kind of
thing in the world, here must surely be the divine, and this must
be the first and most dominant principle. Evidently, then, there are
three kinds of theoretical sciences-physics, mathematics, theology.
The class of theoretical sciences is the best, and of these themselves
the last named is best; for it deals with the highest of existing
things, and each science is called better or worse in virtue of its
proper object. 

"One might raise the question whether the science of being qua being
is to be regarded as universal or not. Each of the mathematical sciences
deals with some one determinate class of things, but universal mathematics
applies alike to all. Now if natural substances are the first of existing
things, physics must be the first of sciences; but if there is another
entity and substance, separable and unmovable, the knowledge of it
must be different and prior to physics and universal because it is
prior. 

Part 8 "

"Since 'being' in general has several senses, of which one is 'being
by accident', we must consider first that which 'is' in this sense.
Evidently none of the traditional sciences busies itself about the
accidental. For neither does architecture consider what will happen
to those who are to use the house (e.g. whether they have a painful
life in it or not), nor does weaving, or shoemaking, or the confectioner's
art, do the like; but each of these sciences considers only what is
peculiar to it, i.e. its proper end. And as for the argument that
'when he who is musical becomes lettered he'll be both at once, not
having been both before; and that which is, not always having been,
must have come to be; therefore he must have at once become musical
and lettered',-this none of the recognized sciences considers, but
only sophistic; for this alone busies itself about the accidental,
so that Plato is not far wrong when he says that the sophist spends
his time on non-being. 

"That a science of the accidental is not even possible will be evident
if we try to see what the accidental really is. We say that everything
either is always and of necessity (necessity not in the sense of violence,
but that which we appeal to in demonstrations), or is for the most
part, or is neither for the most part, nor always and of necessity,
but merely as it chances; e.g. there might be cold in the dogdays,
but this occurs neither always and of necessity, nor for the most
part, though it might happen sometimes. The accidental, then, is what
occurs, but not always nor of necessity, nor for the most part. Now
we have said what the accidental is, and it is obvious why there is
no science of such a thing; for all science is of that which is always
or for the most part, but the accidental is in neither of these classes.

"Evidently there are not causes and principles of the accidental,
of the same kind as there are of the essential; for if there were,
everything would be of necessity. If A is when B is, and B is when
C is, and if C exists not by chance but of necessity, that also of
which C was cause will exist of necessity, down to the last causatum
as it is called (but this was supposed to be accidental). Therefore
all things will be of necessity, and chance and the possibility of
a thing's either occurring or not occurring are removed entirely from
the range of events. And if the cause be supposed not to exist but
to be coming to be, the same results will follow; everything will
occur of necessity. For to-morrow's eclipse will occur if A occurs,
and A if B occurs, and B if C occurs; and in this way if we subtract
time from the limited time between now and to-morrow we shall come
sometime to the already existing condition. Therefore since this exists,
everything after this will occur of necessity, so that all things
occur of necessity. 

"As to that which 'is' in the sense of being true or of being by accident,
the former depends on a combination in thought and is an affection
of thought (which is the reason why it is the principles, not of that
which 'is' in this sense, but of that which is outside and can exist
apart, that are sought); and the latter is not necessary but indeterminate
(I mean the accidental); and of such a thing the causes are unordered
and indefinite. 

"Adaptation to an end is found in events that happen by nature or
as the result of thought. It is 'luck' when one of these events happens
by accident. For as a thing may exist, so it may be a cause, either
by its own nature or by accident. Luck is an accidental cause at work
in such events adapted to an end as are usually effected in accordance
with purpose. And so luck and thought are concerned with the same
sphere; for purpose cannot exist without thought. The causes from
which lucky results might happen are indeterminate; and so luck is
obscure to human calculation and is a cause by accident, but in the
unqualified sense a cause of nothing. It is good or bad luck when
the result is good or evil; and prosperity or misfortune when the
scale of the results is large. 

"Since nothing accidental is prior to the essential, neither are accidental
causes prior. If, then, luck or spontaneity is a cause of the material
universe, reason and nature are causes before it. 

Part 9 "

"Some things are only actually, some potentially, some potentially
and actually, what they are, viz. in one case a particular reality,
in another, characterized by a particular quantity, or the like. There
is no movement apart from things; for change is always according to
the categories of being, and there is nothing common to these and
in no one category. But each of the categories belongs to all its
subjects in either of two ways (e.g. 'this-ness'-for one kind of it
is 'positive form', and the other is 'privation'; and as regards quality
one kind is 'white' and the other 'black', and as regards quantity
one kind is 'complete' and the other 'incomplete', and as regards
spatial movement one is 'upwards' and the other 'downwards', or one
thing is 'light' and another 'heavy'); so that there are as many kinds
of movement and change as of being. There being a distinction in each
class of things between the potential and the completely real, I call
the actuality of the potential as such, movement. That what we say
is true, is plain from the following facts. When the 'buildable',
in so far as it is what we mean by 'buildable', exists actually, it
is being built, and this is the process of building. Similarly with
learning, healing, walking, leaping, ageing, ripening. Movement takes
when the complete reality itself exists, and neither earlier nor later.
The complete reality, then, of that which exists potentially, when
it is completely real and actual, not qua itself, but qua movable,
is movement. By qua I mean this: bronze is potentially a statue; but
yet it is not the complete reality of bronze qua bronze that is movement.
For it is not the same thing to be bronze and to be a certain potency.
If it were absolutely the same in its definition, the complete reality
of bronze would have been a movement. But it is not the same. (This
is evident in the case of contraries; for to be capable of being well
and to be capable of being ill are not the same-for if they were,
being well and being ill would have been the same-it is that which
underlies and is healthy or diseased, whether it is moisture or blood,
that is one and the same.) And since it is not. the same, as colour
and the visible are not the same, it is the complete reality of the
potential, and as potential, that is movement. That it is this, and
that movement takes place when the complete reality itself exists,
and neither earlier nor later, is evident. For each thing is capable
of being sometimes actual, sometimes not, e.g. the buildable qua buildable;
and the actuality of the buildable qua buildable is building. For
the actuality is either this-the act of building-or the house. But
when the house exists, it is no longer buildable; the buildable is
what is being built. The actuality, then, must be the act of building,
and this is a movement. And the same account applies to all other
movements. 

"That what we have said is right is evident from what all others say
about movement, and from the fact that it is not easy to define it
otherwise. For firstly one cannot put it in any class. This is evident
from what people say. Some call it otherness and inequality and the
unreal; none of these, however, is necessarily moved, and further,
change is not either to these or from these any more than from their
opposites. The reason why people put movement in these classes is
that it is thought to be something indefinite, and the principles
in one of the two 'columns of contraries' are indefinite because they
are privative, for none of them is either a 'this' or a 'such' or
in any of the other categories. And the reason why movement is thought
to be indefinite is that it cannot be classed either with the potency
of things or with their actuality; for neither that which is capable
of being of a certain quantity, nor that which is actually of a certain
quantity, is of necessity moved, and movement is thought to be an
actuality, but incomplete; the reason is that the potential, whose
actuality it is, is incomplete. And therefore it is hard to grasp
what movement is; for it must be classed either under privation or
under potency or under absolute actuality, but evidently none of these
is possible. Therefore what remains is that it must be what we said-both
actuality and the actuality we have described-which is hard to detect
but capable of existing. 

"And evidently movement is in the movable; for it is the complete
realization of this by that which is capable of causing movement.
And the actuality of that which is capable of causing movement is
no other than that of the movable. For it must be the complete reality
of both. For while a thing is capable of causing movement because
it can do this, it is a mover because it is active; but it is on the
movable that it is capable of acting, so that the actuality of both
is one, just as there is the same interval from one to two as from
two to one, and as the steep ascent and the steep descent are one,
but the being of them is not one; the case of the mover and the moved
is similar. 

Part 10 "

"The infinite is either that which is incapable of being traversed
because it is not its nature to be traversed (this corresponds to
the sense in which the voice is 'invisible'), or that which admits
only of incomplete traverse or scarcely admits of traverse, or that
which, though it naturally admits of traverse, is not traversed or
limited; further, a thing may be infinite in respect of addition or
of subtraction, or both. The infinite cannot be a separate, independent
thing. For if it is neither a spatial magnitude nor a plurality, but
infinity itself is its substance and not an accident of it, it will
be indivisible; for the divisible is either magnitude or plurality.
But if indivisible, it is not infinite, except as the voice is invisible;
but people do not mean this, nor are we examining this sort of infinite,
but the infinite as untraversable. Further, how can an infinite exist
by itself, unless number and magnitude also exist by themselvess-since
infinity is an attribute of these? Further, if the infinite is an
accident of something else, it cannot be qua infinite an element in
things, as the invisible is not an element in speech, though the voice
is invisible. And evidently the infinite cannot exist actually. For
then any part of it that might be taken would be infinite (for 'to
be infinite' and 'the infinite' are the same, if the infinite is substance
and not predicated of a subject). Therefore it is either indivisible,
or if it is partible, it is divisible into infinites; but the same
thing cannot be many infinites (as a part of air is air, so a part
of the infinite would be infinite, if the infinite is substance and
a principle). Therefore it must be impartible and indivisible. But
the actually infinite cannot be indivisible; for it must be of a certain
quantity. Therefore infinity belongs to its subject incidentally.
But if so, then (as we have said) it cannot be it that is a principle,
but that of which it is an accident-the air or the even number.

"This inquiry is universal; but that the infinite is not among sensible
things, is evident from the following argument. If the definition
of a body is 'that which is bounded by planes', there cannot be an
infinite body either sensible or intelligible; nor a separate and
infinite number, for number or that which has a number is numerable.
Concretely, the truth is evident from the following argument. The
infinite can neither be composite nor simple. For (a) it cannot be
a composite body, since the elements are limited in multitude. For
the contraries must be equal and no one of them must be infinite;
for if one of the two bodies falls at all short of the other in potency,
the finite will be destroyed by the infinite. And that each should
be infinite is impossible. For body is that which has extension in
all directions, and the infinite is the boundlessly extended, so that
if the infinite is a body it will be infinite in every direction.
Nor (b) can the infinite body be one and simple-neither, as some say,
something apart from the elements, from which they generate these
(for there is no such body apart from the elements; for everything
can be resolved into that of which it consists, but no such product
of analysis is observed except the simple bodies), nor fire nor any
other of the elements. For apart from the question how any of them
could be infinite, the All, even if it is finite, cannot either be
or become any one of them, as Heraclitus says all things sometime
become fire. The same argument applies to this as to the One which
the natural philosophers posit besides the elements. For everything
changes from contrary to contrary, e.g. from hot to cold.

"Further, a sensible body is somewhere, and whole and part have the
same proper place, e.g. the whole earth and part of the earth. Therefore
if (a) the infinite body is homogeneous, it will be unmovable or it
will be always moving. But this is impossible; for why should it rather
rest, or move, down, up, or anywhere, rather than anywhere else? E.g.
if there were a clod which were part of an infinite body, where will
this move or rest? The proper place of the body which is homogeneous
with it is infinite. Will the clod occupy the whole place, then? And
how? (This is impossible.) What then is its rest or its movement?
It will either rest everywhere, and then it cannot move; or it will
move everywhere, and then it cannot be still. But (b) if the All has
unlike parts, the proper places of the parts are unlike also, and,
firstly, the body of the All is not one except by contact, and, secondly,
the parts will be either finite or infinite in variety of kind. Finite
they cannot be; for then those of one kind will be infinite in quantity
and those of another will not (if the All is infinite), e.g. fire
or water would be infinite, but such an infinite element would be
destruction to the contrary elements. But if the parts are infinite
and simple, their places also are infinite and there will be an infinite
number of elements; and if this is impossible, and the places are
finite, the All also must be limited. 

"In general, there cannot be an infinite body and also a proper place
for bodies, if every sensible body has either weight or lightness.
For it must move either towards the middle or upwards, and the infinite
either the whole or the half of it-cannot do either; for how will
you divide it? Or how will part of the infinite be down and part up,
or part extreme and part middle? Further, every sensible body is in
a place, and there are six kinds of place, but these cannot exist
in an infinite body. In general, if there cannot be an infinite place,
there cannot be an infinite body; (and there cannot be an infinite
place,) for that which is in a place is somewhere, and this means
either up or down or in one of the other directions, and each of these
is a limit. 

"The infinite is not the same in the sense that it is a single thing
whether exhibited in distance or in movement or in time, but the posterior
among these is called infinite in virtue of its relation to the prior;
i.e. a movement is called infinite in virtue of the distance covered
by the spatial movement or alteration or growth, and a time is called
infinite because of the movement which occupies it. 

Part 11 "

"Of things which change, some change in an accidental sense, like
that in which 'the musical' may be said to walk, and others are said,
without qualification, to change, because something in them changes,
i.e. the things that change in parts; the body becomes healthy, because
the eye does. But there is something which is by its own nature moved
directly, and this is the essentially movable. The same distinction
is found in the case of the mover; for it causes movement either in
an accidental sense or in respect of a part of itself or essentially.
There is something that directly causes movement; and there is something
that is moved, also the time in which it is moved, and that from which
and that into which it is moved. But the forms and the affections
and the place, which are the terminals of the movement of moving things,
are unmovable, e.g. knowledge or heat; it is not heat that is a movement,
but heating. Change which is not accidental is found not in all things,
but between contraries, and their intermediates, and between contradictories.
We may convince ourselves of this by induction. 

"That which changes changes either from positive into positive, or
from negative into negative, or from positive into negative, or from
negative into positive. (By positive I mean that which is expressed
by an affirmative term.) Therefore there must be three changes; that
from negative into negative is not change, because (since the terms
are neither contraries nor contradictories) there is no opposition.
The change from the negative into the positive which is its contradictory
is generation-absolute change absolute generation, and partial change
partial generation; and the change from positive to negative is destruction-absolute
change absolute destruction, and partial change partial destruction.
If, then, 'that which is not' has several senses, and movement can
attach neither to that which implies putting together or separating,
nor to that which implies potency and is opposed to that which is
in the full sense (true, the not-white or not-good can be moved incidentally,
for the not-white might be a man; but that which is not a particular
thing at all can in no wise be moved), that which is not cannot be
moved (and if this is so, generation cannot be movement; for that
which is not is generated; for even if we admit to the full that its
generation is accidental, yet it is true to say that 'not-being' is
predicable of that which is generated absolutely). Similarly rest
cannot be long to that which is not. These consequences, then, turn
out to be awkward, and also this, that everything that is moved is
in a place, but that which is not is not in a place; for then it would
be somewhere. Nor is destruction movement; for the contrary of movement
is rest, but the contrary of destruction is generation. Since every
movement is a change, and the kinds of change are the three named
above, and of these those in the way of generation and destruction
are not movements, and these are the changes from a thing to its contradictory,
it follows that only the change from positive into positive is movement.
And the positives are either contrary or intermediate (for even privation
must be regarded as contrary), and are expressed by an affirmative
term, e.g. 'naked' or 'toothless' or 'black'. 

Part 12 "

"If the categories are classified as substance, quality, place, acting
or being acted on, relation, quantity, there must be three kinds of
movement-of quality, of quantity, of place. There is no movement in
respect of substance (because there is nothing contrary to substance),
nor of relation (for it is possible that if one of two things in relation
changes, the relative term which was true of the other thing ceases
to be true, though this other does not change at all,-so that their
movement is accidental), nor of agent and patient, or mover and moved,
because there is no movement of movement nor generation of generation,
nor, in general, change of change. For there might be movement of
movement in two senses; (1) movement might be the subject moved, as
a man is moved because he changes from pale to dark,-so that on this
showing movement, too, may be either heated or cooled or change its
place or increase. But this is impossible; for change is not a subject.
Or (2) some other subject might change from change into some other
form of existence (e.g. a man from disease into health). But this
also is not possible except incidentally. For every movement is change
from something into something. (And so are generation and destruction;
only, these are changes into things opposed in certain ways while
the other, movement, is into things opposed in another way.) A thing
changes, then, at the same time from health into illness, and from
this change itself into another. Clearly, then, if it has become ill,
it will have changed into whatever may be the other change concerned
(though it may be at rest), and, further, into a determinate change
each time; and that new change will be from something definite into
some other definite thing; therefore it will be the opposite change,
that of growing well. We answer that this happens only incidentally;
e.g. there is a change from the process of recollection to that of
forgetting, only because that to which the process attaches is changing,
now into a state of knowledge, now into one of ignorance.

"Further, the process will go on to infinity, if there is to be change
of change and coming to be of coming to be. What is true of the later,
then, must be true of the earlier; e.g. if the simple coming to be
was once coming to be, that which comes to be something was also once
coming to be; therefore that which simply comes to be something was
not yet in existence, but something which was coming to be coming
to be something was already in existence. And this was once coming
to be, so that at that time it was not yet coming to be something
else. Now since of an infinite number of terms there is not a first,
the first in this series will not exist, and therefore no following
term exist. Nothing, then, can either come term wi to be or move or
change. Further, that which is capable of a movement is also capable
of the contrary movement and rest, and that which comes to be also
ceases to be. Therefore that which is coming to be is ceasing to be
when it has come to be coming to be; for it cannot cease to be as
soon as it is coming to be coming to be, nor after it has come to
be; for that which is ceasing to be must be. Further, there must be
a matter underlying that which comes to be and changes. What will
this be, then,-what is it that becomes movement or becoming, as body
or soul is that which suffers alteration? And; again, what is it that
they move into? For it must be the movement or becoming of something
from something into something. How, then, can this condition be fulfilled?
There can be no learning of learning, and therefore no becoming of
becoming. Since there is not movement either of substance or of relation
or of activity and passivity, it remains that movement is in respect
of quality and quantity and place; for each of these admits of contrariety.
By quality I mean not that which is in the substance (for even the
differentia is a quality), but the passive quality, in virtue of which
a thing is said to be acted on or to be incapable of being acted on.
The immobile is either that which is wholly incapable of being moved,
or that which is moved with difficulty in a long time or begins slowly,
or that which is of a nature to be moved and can be moved but is not
moved when and where and as it would naturally be moved. This alone
among immobiles I describe as being at rest; for rest is contrary
to movement, so that it must be a privation in that which is receptive
of movement. 

"Things which are in one proximate place are together in place, and
things which are in different places are apart: things whose extremes
are together touch: that at which a changing thing, if it changes
continuously according to its nature, naturally arrives before it
arrives at the extreme into which it is changing, is between. That
which is most distant in a straight line is contrary in place. That
is successive which is after the beginning (the order being determined
by position or form or in some other way) and has nothing of the same
class between it and that which it succeeds, e.g. lines in the case
of a line, units in that of a unit, or a house in that of a house.
(There is nothing to prevent a thing of some other class from being
between.) For the successive succeeds something and is something later;
'one' does not succeed 'two', nor the first day of the month the second.
That which, being successive, touches, is contiguous. (Since all change
is between opposites, and these are either contraries or contradictories,
and there is no middle term for contradictories, clearly that which
is between is between contraries.) The continuous is a species of
the contiguous. I call two things continuous when the limits of each,
with which they touch and by which they are kept together, become
one and the same, so that plainly the continuous is found in the things
out of which a unity naturally arises in virtue of their contact.
And plainly the successive is the first of these concepts (for the
successive does not necessarily touch, but that which touches is successive;
and if a thing is continuous, it touches, but if it touches, it is
not necessarily continuous; and in things in which there is no touching,
there is no organic unity); therefore a point is not the same as a
unit; for contact belongs to points, but not to units, which have
only succession; and there is something between two of the former,
but not between two of the latter. 

----------------------------------------------------------------------

BOOK XII

Part 1 

"

"The subject of our inquiry is substance; for the principles and the
causes we are seeking are those of substances. For if the universe
is of the nature of a whole, substance is its first part; and if it
coheres merely by virtue of serial succession, on this view also substance
is first, and is succeeded by quality, and then by quantity. At the
same time these latter are not even being in the full sense, but are
qualities and movements of it,-or else even the not-white and the
not-straight would be being; at least we say even these are, e.g.
'there is a not-white'. Further, none of the categories other than
substance can exist apart. And the early philosophers also in practice
testify to the primacy of substance; for it was of substance that
they sought the principles and elements and causes. The thinkers of
the present day tend to rank universals as substances (for genera
are universals, and these they tend to describe as principles and
substances, owing to the abstract nature of their inquiry); but the
thinkers of old ranked particular things as substances, e.g. fire
and earth, not what is common to both, body. 

"There are three kinds of substance-one that is sensible (of which
one subdivision is eternal and another is perishable; the latter is
recognized by all men, and includes e.g. plants and animals), of which
we must grasp the elements, whether one or many; and another that
is immovable, and this certain thinkers assert to be capable of existing
apart, some dividing it into two, others identifying the Forms and
the objects of mathematics, and others positing, of these two, only
the objects of mathematics. The former two kinds of substance are
the subject of physics (for they imply movement); but the third kind
belongs to another science, if there is no principle common to it
and to the other kinds. 

Part 2 "

"Sensible substance is changeable. Now if change proceeds from opposites
or from intermediates, and not from all opposites (for the voice is
not-white, (but it does not therefore change to white)), but from
the contrary, there must be something underlying which changes into
the contrary state; for the contraries do not change. Further, something
persists, but the contrary does not persist; there is, then, some
third thing besides the contraries, viz. the matter. Now since changes
are of four kinds-either in respect of the 'what' or of the quality
or of the quantity or of the place, and change in respect of 'thisness'
is simple generation and destruction, and change in quantity is increase
and diminution, and change in respect of an affection is alteration,
and change of place is motion, changes will be from given states into
those contrary to them in these several respects. The matter, then,
which changes must be capable of both states. And since that which
'is' has two senses, we must say that everything changes from that
which is potentially to that which is actually, e.g. from potentially
white to actually white, and similarly in the case of increase and
diminution. Therefore not only can a thing come to be, incidentally,
out of that which is not, but also all things come to be out of that
which is, but is potentially, and is not actually. And this is the
'One' of Anaxagoras; for instead of 'all things were together'-and
the 'Mixture' of Empedocles and Anaximander and the account given
by Democritus-it is better to say 'all things were together potentially
but not actually'. Therefore these thinkers seem to have had some
notion of matter. Now all things that change have matter, but different
matter; and of eternal things those which are not generable but are
movable in space have matter-not matter for generation, however, but
for motion from one place to another. 

"One might raise the question from what sort of non-being generation
proceeds; for 'non-being' has three senses. If, then, one form of
non-being exists potentially, still it is not by virtue of a potentiality
for any and every thing, but different things come from different
things; nor is it satisfactory to say that 'all things were together';
for they differ in their matter, since otherwise why did an infinity
of things come to be, and not one thing? For 'reason' is one, so that
if matter also were one, that must have come to be in actuality which
the matter was in potency. The causes and the principles, then, are
three, two being the pair of contraries of which one is definition
and form and the other is privation, and the third being the matter.

Part 3 "

"Note, next, that neither the matter nor the form comes to be-and
I mean the last matter and form. For everything that changes is something
and is changed by something and into something. That by which it is
changed is the immediate mover; that which is changed, the matter;
that into which it is changed, the form. The process, then, will go
on to infinity, if not only the bronze comes to be round but also
the round or the bronze comes to be; therefore there must be a stop.

"Note, next, that each substance comes into being out of something
that shares its name. (Natural objects and other things both rank
as substances.) For things come into being either by art or by nature
or by luck or by spontaneity. Now art is a principle of movement in
something other than the thing moved, nature is a principle in the
thing itself (for man begets man), and the other causes are privations
of these two. 

"There are three kinds of substance-the matter, which is a 'this'
in appearance (for all things that are characterized by contact and
not, by organic unity are matter and substratum, e.g. fire, flesh,
head; for these are all matter, and the last matter is the matter
of that which is in the full sense substance); the nature, which is
a 'this' or positive state towards which movement takes place; and
again, thirdly, the particular substance which is composed of these
two, e.g. Socrates or Callias. Now in some cases the 'this' does not
exist apart from the composite substance, e.g. the form of house does
not so exist, unless the art of building exists apart (nor is there
generation and destruction of these forms, but it is in another way
that the house apart from its matter, and health, and all ideals of
art, exist and do not exist); but if the 'this' exists apart from
the concrete thing, it is only in the case of natural objects. And
so Plato was not far wrong when he said that there are as many Forms
as there are kinds of natural object (if there are Forms distinct
from the things of this earth). The moving causes exist as things
preceding the effects, but causes in the sense of definitions are
simultaneous with their effects. For when a man is healthy, then health
also exists; and the shape of a bronze sphere exists at the same time
as the bronze sphere. (But we must examine whether any form also survives
afterwards. For in some cases there is nothing to prevent this; e.g.
the soul may be of this sort-not all soul but the reason; for presumably
it is impossible that all soul should survive.) Evidently then there
is no necessity, on this ground at least, for the existence of the
Ideas. For man is begotten by man, a given man by an individual father;
and similarly in the arts; for the medical art is the formal cause
of health. 

Part 4 "

"The causes and the principles of different things are in a sense
different, but in a sense, if one speaks universally and analogically,
they are the same for all. For one might raise the question whether
the principles and elements are different or the same for substances
and for relative terms, and similarly in the case of each of the categories.
But it would be paradoxical if they were the same for all. For then
from the same elements will proceed relative terms and substances.
What then will this common element be? For (1, a) there is nothing
common to and distinct from substance and the other categories, viz.
those which are predicated; but an element is prior to the things
of which it is an element. But again (b) substance is not an element
in relative terms, nor is any of these an element in substance. Further,
(2) how can all things have the same elements? For none of the elements
can be the same as that which is composed of elements, e.g. b or a
cannot be the same as ba. (None, therefore, of the intelligibles,
e.g. being or unity, is an element; for these are predicable of each
of the compounds as well.) None of the elements, then, will be either
a substance or a relative term; but it must be one or other. All things,
then, have not the same elements. 

"Or, as we are wont to put it, in a sense they have and in a sense
they have not; e.g. perhaps the elements of perceptible bodies are,
as form, the hot, and in another sense the cold, which is the privation;
and, as matter, that which directly and of itself potentially has
these attributes; and substances comprise both these and the things
composed of these, of which these are the principles, or any unity
which is produced out of the hot and the cold, e.g. flesh or bone;
for the product must be different from the elements. These things
then have the same elements and principles (though specifically different
things have specifically different elements); but all things have
not the same elements in this sense, but only analogically; i.e. one
might say that there are three principles-the form, the privation,
and the matter. But each of these is different for each class; e.g.
in colour they are white, black, and surface, and in day and night
they are light, darkness, and air. 

"Since not only the elements present in a thing are causes, but also
something external, i.e. the moving cause, clearly while 'principle'
and 'element' are different both are causes, and 'principle' is divided
into these two kinds; and that which acts as producing movement or
rest is a principle and a substance. Therefore analogically there
are three elements, and four causes and principles; but the elements
are different in different things, and the proximate moving cause
is different for different things. Health, disease, body; the moving
cause is the medical art. Form, disorder of a particular kind, bricks;
the moving cause is the building art. And since the moving cause in
the case of natural things is-for man, for instance, man, and in the
products of thought the form or its contrary, there will be in a sense
three causes, while in a sense there are four. For the medical art
is in some sense health, and the building art is the form of the house,
and man begets man; further, besides these there is that which as
first of all things moves all things. 

Part 5 "

"Some things can exist apart and some cannot, and it is the former
that are substances. And therefore all things have the same causes,
because, without substances, modifications and movements do not exist.
Further, these causes will probably be soul and body, or reason and
desire and body. 

"And in yet another way, analogically identical things are principles,
i.e. actuality and potency; but these also are not only different
for different things but also apply in different ways to them. For
in some cases the same thing exists at one time actually and at another
potentially, e.g. wine or flesh or man does so. (And these too fall
under the above-named causes. For the form exists actually, if it
can exist apart, and so does the complex of form and matter, and the
privation, e.g. darkness or disease; but the matter exists potentially;
for this is that which can become qualified either by the form or
by the privation.) But the distinction of actuality and potentiality
applies in another way to cases where the matter of cause and of effect
is not the same, in some of which cases the form is not the same but
different; e.g. the cause of man is (1) the elements in man (viz.
fire and earth as matter, and the peculiar form), and further (2)
something else outside, i.e. the father, and (3) besides these the
sun and its oblique course, which are neither matter nor form nor
privation of man nor of the same species with him, but moving causes.

"Further, one must observe that some causes can be expressed in universal
terms, and some cannot. The proximate principles of all things are
the 'this' which is proximate in actuality, and another which is proximate
in potentiality. The universal causes, then, of which we spoke do
not exist. For it is the individual that is the originative principle
of the individuals. For while man is the originative principle of
man universally, there is no universal man, but Peleus is the originative
principle of Achilles, and your father of you, and this particular
b of this particular ba, though b in general is the originative principle
of ba taken without qualification. 

"Further, if the causes of substances are the causes of all things,
yet different things have different causes and elements, as was said;
the causes of things that are not in the same class, e.g. of colours
and sounds, of substances and quantities, are different except in
an analogical sense; and those of things in the same species are different,
not in species, but in the sense that the causes of different individuals
are different, your matter and form and moving cause being different
from mine, while in their universal definition they are the same.
And if we inquire what are the principles or elements of substances
and relations and qualities-whether they are the same or different-clearly
when the names of the causes are used in several senses the causes
of each are the same, but when the senses are distinguished the causes
are not the same but different, except that in the following senses
the causes of all are the same. They are (1) the same or analogous
in this sense, that matter, form, privation, and the moving cause
are common to all things; and (2) the causes of substances may be
treated as causes of all things in this sense, that when substances
are removed all things are removed; further, (3) that which is first
in respect of complete reality is the cause of all things. But in
another sense there are different first causes, viz. all the contraries
which are neither generic nor ambiguous terms; and, further, the matters
of different things are different. We have stated, then, what are
the principles of sensible things and how many they are, and in what
sense they are the same and in what sense different. 

Part 6 "

"Since there were three kinds of substance, two of them physical and
one unmovable, regarding the latter we must assert that it is necessary
that there should be an eternal unmovable substance. For substances
are the first of existing things, and if they are all destructible,
all things are destructible. But it is impossible that movement should
either have come into being or cease to be (for it must always have
existed), or that time should. For there could not be a before and
an after if time did not exist. Movement also is continuous, then,
in the sense in which time is; for time is either the same thing as
movement or an attribute of movement. And there is no continuous movement
except movement in place, and of this only that which is circular
is continuous. 

"But if there is something which is capable of moving things or acting
on them, but is not actually doing so, there will not necessarily
be movement; for that which has a potency need not exercise it. Nothing,
then, is gained even if we suppose eternal substances, as the believers
in the Forms do, unless there is to be in them some principle which
can cause change; nay, even this is not enough, nor is another substance
besides the Forms enough; for if it is not to act, there will be no
movement. Further even if it acts, this will not be enough, if its
essence is potency; for there will not be eternal movement, since
that which is potentially may possibly not be. There must, then, be
such a principle, whose very essence is actuality. Further, then,
these substances must be without matter; for they must be eternal,
if anything is eternal. Therefore they must be actuality.

"Yet there is a difficulty; for it is thought that everything that
acts is able to act, but that not everything that is able to act acts,
so that the potency is prior. But if this is so, nothing that is need
be; for it is possible for all things to be capable of existing but
not yet to exist. 

"Yet if we follow the theologians who generate the world from night,
or the natural philosophers who say that 'all things were together',
the same impossible result ensues. For how will there be movement,
if there is no actually existing cause? Wood will surely not move
itself-the carpenter's art must act on it; nor will the menstrual
blood nor the earth set themselves in motion, but the seeds must act
on the earth and the semen on the menstrual blood. 

"This is why some suppose eternal actuality-e.g. Leucippus and Plato;
for they say there is always movement. But why and what this movement
is they do say, nor, if the world moves in this way or that, do they
tell us the cause of its doing so. Now nothing is moved at random,
but there must always be something present to move it; e.g. as a matter
of fact a thing moves in one way by nature, and in another by force
or through the influence of reason or something else. (Further, what
sort of movement is primary? This makes a vast difference.) But again
for Plato, at least, it is not permissible to name here that which
he sometimes supposes to be the source of movement-that which moves
itself; for the soul is later, and coeval with the heavens, according
to his account. To suppose potency prior to actuality, then, is in
a sense right, and in a sense not; and we have specified these senses.
That actuality is prior is testified by Anaxagoras (for his 'reason'
is actuality) and by Empedocles in his doctrine of love and strife,
and by those who say that there is always movement, e.g. Leucippus.
Therefore chaos or night did not exist for an infinite time, but the
same things have always existed (either passing through a cycle of
changes or obeying some other law), since actuality is prior to potency.
If, then, there is a constant cycle, something must always remain,
acting in the same way. And if there is to be generation and destruction,
there must be something else which is always acting in different ways.
This must, then, act in one way in virtue of itself, and in another
in virtue of something else-either of a third agent, therefore, or
of the first. Now it must be in virtue of the first. For otherwise
this again causes the motion both of the second agent and of the third.
Therefore it is better to say 'the first'. For it was the cause of
eternal uniformity; and something else is the cause of variety, and
evidently both together are the cause of eternal variety. This, accordingly,
is the character which the motions actually exhibit. What need then
is there to seek for other principles? 

Part 7 "

"Since (1) this is a possible account of the matter, and (2) if it
were not true, the world would have proceeded out of night and 'all
things together' and out of non-being, these difficulties may be taken
as solved. There is, then, something which is always moved with an
unceasing motion, which is motion in a circle; and this is plain not
in theory only but in fact. Therefore the first heaven must be eternal.
There is therefore also something which moves it. And since that which
moves and is moved is intermediate, there is something which moves
without being moved, being eternal, substance, and actuality. And
the object of desire and the object of thought move in this way; they
move without being moved. The primary objects of desire and of thought
are the same. For the apparent good is the object of appetite, and
the real good is the primary object of rational wish. But desire is
consequent on opinion rather than opinion on desire; for the thinking
is the starting-point. And thought is moved by the object of thought,
and one of the two columns of opposites is in itself the object of
thought; and in this, substance is first, and in substance, that which
is simple and exists actually. (The one and the simple are not the
same; for 'one' means a measure, but 'simple' means that the thing
itself has a certain nature.) But the beautiful, also, and that which
is in itself desirable are in the same column; and the first in any
class is always best, or analogous to the best. 

"That a final cause may exist among unchangeable entities is shown
by the distinction of its meanings. For the final cause is (a) some
being for whose good an action is done, and (b) something at which
the action aims; and of these the latter exists among unchangeable
entities though the former does not. The final cause, then, produces
motion as being loved, but all other things move by being moved. Now
if something is moved it is capable of being otherwise than as it
is. Therefore if its actuality is the primary form of spatial motion,
then in so far as it is subject to change, in this respect it is capable
of being otherwise,-in place, even if not in substance. But since
there is something which moves while itself unmoved, existing actually,
this can in no way be otherwise than as it is. For motion in space
is the first of the kinds of change, and motion in a circle the first
kind of spatial motion; and this the first mover produces. The first
mover, then, exists of necessity; and in so far as it exists by necessity,
its mode of being is good, and it is in this sense a first principle.
For the necessary has all these senses-that which is necessary perforce
because it is contrary to the natural impulse, that without which
the good is impossible, and that which cannot be otherwise but can
exist only in a single way. 

"On such a principle, then, depend the heavens and the world of nature.
And it is a life such as the best which we enjoy, and enjoy for but
a short time (for it is ever in this state, which we cannot be), since
its actuality is also pleasure. (And for this reason are waking, perception,
and thinking most pleasant, and hopes and memories are so on account
of these.) And thinking in itself deals with that which is best in
itself, and that which is thinking in the fullest sense with that
which is best in the fullest sense. And thought thinks on itself because
it shares the nature of the object of thought; for it becomes an object
of thought in coming into contact with and thinking its objects, so
that thought and object of thought are the same. For that which is
capable of receiving the object of thought, i.e. the essence, is thought.
But it is active when it possesses this object. Therefore the possession
rather than the receptivity is the divine element which thought seems
to contain, and the act of contemplation is what is most pleasant
and best. If, then, God is always in that good state in which we sometimes
are, this compels our wonder; and if in a better this compels it yet
more. And God is in a better state. And life also belongs to God;
for the actuality of thought is life, and God is that actuality; and
God's self-dependent actuality is life most good and eternal. We say
therefore that God is a living being, eternal, most good, so that
life and duration continuous and eternal belong to God; for this is
God. 

"Those who suppose, as the Pythagoreans and Speusippus do, that supreme
beauty and goodness are not present in the beginning, because the
beginnings both of plants and of animals are causes, but beauty and
completeness are in the effects of these, are wrong in their opinion.
For the seed comes from other individuals which are prior and complete,
and the first thing is not seed but the complete being; e.g. we must
say that before the seed there is a man,-not the man produced from
the seed, but another from whom the seed comes. 

"It is clear then from what has been said that there is a substance
which is eternal and unmovable and separate from sensible things.
It has been shown also that this substance cannot have any magnitude,
but is without parts and indivisible (for it produces movement through
infinite time, but nothing finite has infinite power; and, while every
magnitude is either infinite or finite, it cannot, for the above reason,
have finite magnitude, and it cannot have infinite magnitude because
there is no infinite magnitude at all). But it has also been shown
that it is impassive and unalterable; for all the other changes are
posterior to change of place. 

Part 8 "

"It is clear, then, why these things are as they are. But we must
not ignore the question whether we have to suppose one such substance
or more than one, and if the latter, how many; we must also mention,
regarding the opinions expressed by others, that they have said nothing
about the number of the substances that can even be clearly stated.
For the theory of Ideas has no special discussion of the subject;
for those who speak of Ideas say the Ideas are numbers, and they speak
of numbers now as unlimited, now as limited by the number 10; but
as for the reason why there should be just so many numbers, nothing
is said with any demonstrative exactness. We however must discuss
the subject, starting from the presuppositions and distinctions we
have mentioned. The first principle or primary being is not movable
either in itself or accidentally, but produces the primary eternal
and single movement. But since that which is moved must be moved by
something, and the first mover must be in itself unmovable, and eternal
movement must be produced by something eternal and a single movement
by a single thing, and since we see that besides the simple spatial
movement of the universe, which we say the first and unmovable substance
produces, there are other spatial movements-those of the planets-which
are eternal (for a body which moves in a circle is eternal and unresting;
we have proved these points in the physical treatises), each of these
movements also must be caused by a substance both unmovable in itself
and eternal. For the nature of the stars is eternal just because it
is a certain kind of substance, and the mover is eternal and prior
to the moved, and that which is prior to a substance must be a substance.
Evidently, then, there must be substances which are of the same number
as the movements of the stars, and in their nature eternal, and in
themselves unmovable, and without magnitude, for the reason before
mentioned. That the movers are substances, then, and that one of these
is first and another second according to the same order as the movements
of the stars, is evident. But in the number of the movements we reach
a problem which must be treated from the standpoint of that one of
the mathematical sciences which is most akin to philosophy-viz. of
astronomy; for this science speculates about substance which is perceptible
but eternal, but the other mathematical sciences, i.e. arithmetic
and geometry, treat of no substance. That the movements are more numerous
than the bodies that are moved is evident to those who have given
even moderate attention to the matter; for each of the planets has
more than one movement. But as to the actual number of these movements,
we now-to give some notion of the subject-quote what some of the mathematicians
say, that our thought may have some definite number to grasp; but,
for the rest, we must partly investigate for ourselves, Partly learn
from other investigators, and if those who study this subject form
an opinion contrary to what we have now stated, we must esteem both
parties indeed, but follow the more accurate. 

"Eudoxus supposed that the motion of the sun or of the moon involves,
in either case, three spheres, of which the first is the sphere of
the fixed stars, and the second moves in the circle which runs along
the middle of the zodiac, and the third in the circle which is inclined
across the breadth of the zodiac; but the circle in which the moon
moves is inclined at a greater angle than that in which the sun moves.
And the motion of the planets involves, in each case, four spheres,
and of these also the first and second are the same as the first two
mentioned above (for the sphere of the fixed stars is that which moves
all the other spheres, and that which is placed beneath this and has
its movement in the circle which bisects the zodiac is common to all),
but the poles of the third sphere of each planet are in the circle
which bisects the zodiac, and the motion of the fourth sphere is in
the circle which is inclined at an angle to the equator of the third
sphere; and the poles of the third sphere are different for each of
the other planets, but those of Venus and Mercury are the same.

"Callippus made the position of the spheres the same as Eudoxus did,
but while he assigned the same number as Eudoxus did to Jupiter and
to Saturn, he thought two more spheres should be added to the sun
and two to the moon, if one is to explain the observed facts; and
one more to each of the other planets. 

"But it is necessary, if all the spheres combined are to explain the
observed facts, that for each of the planets there should be other
spheres (one fewer than those hitherto assigned) which counteract
those already mentioned and bring back to the same position the outermost
sphere of the star which in each case is situated below the star in
question; for only thus can all the forces at work produce the observed
motion of the planets. Since, then, the spheres involved in the movement
of the planets themselves are--eight for Saturn and Jupiter and twenty-five
for the others, and of these only those involved in the movement of
the lowest-situated planet need not be counteracted the spheres which
counteract those of the outermost two planets will be six in number,
and the spheres which counteract those of the next four planets will
be sixteen; therefore the number of all the spheres--both those which
move the planets and those which counteract these--will be fifty-five.
And if one were not to add to the moon and to the sun the movements
we mentioned, the whole set of spheres will be forty-seven in number.

"Let this, then, be taken as the number of the spheres, so that the
unmovable substances and principles also may probably be taken as
just so many; the assertion of necessity must be left to more powerful
thinkers. But if there can be no spatial movement which does not conduce
to the moving of a star, and if further every being and every substance
which is immune from change and in virtue of itself has attained to
the best must be considered an end, there can be no other being apart
from these we have named, but this must be the number of the substances.
For if there are others, they will cause change as being a final cause
of movement; but there cannot he other movements besides those mentioned.
And it is reasonable to infer this from a consideration of the bodies
that are moved; for if everything that moves is for the sake of that
which is moved, and every movement belongs to something that is moved,
no movement can be for the sake of itself or of another movement,
but all the movements must be for the sake of the stars. For if there
is to be a movement for the sake of a movement, this latter also will
have to be for the sake of something else; so that since there cannot
be an infinite regress, the end of every movement will be one of the
divine bodies which move through the heaven. 

"(Evidently there is but one heaven. For if there are many heavens
as there are many men, the moving principles, of which each heaven
will have one, will be one in form but in number many. But all things
that are many in number have matter; for one and the same definition,
e.g. that of man, applies to many things, while Socrates is one. But
the primary essence has not matter; for it is complete reality. So
the unmovable first mover is one both in definition and in number;
so too, therefore, is that which is moved always and continuously;
therefore there is one heaven alone.) Our forefathers in the most
remote ages have handed down to their posterity a tradition, in the
form of a myth, that these bodies are gods, and that the divine encloses
the whole of nature. The rest of the tradition has been added later
in mythical form with a view to the persuasion of the multitude and
to its legal and utilitarian expediency; they say these gods are in
the form of men or like some of the other animals, and they say other
things consequent on and similar to these which we have mentioned.
But if one were to separate the first point from these additions and
take it alone-that they thought the first substances to be gods, one
must regard this as an inspired utterance, and reflect that, while
probably each art and each science has often been developed as far
as possible and has again perished, these opinions, with others, have
been preserved until the present like relics of the ancient treasure.
Only thus far, then, is the opinion of our ancestors and of our earliest
predecessors clear to us. 

Part 9 "

"The nature of the divine thought involves certain problems; for while
thought is held to be the most divine of things observed by us, the
question how it must be situated in order to have that character involves
difficulties. For if it thinks of nothing, what is there here of dignity?
It is just like one who sleeps. And if it thinks, but this depends
on something else, then (since that which is its substance is not
the act of thinking, but a potency) it cannot be the best substance;
for it is through thinking that its value belongs to it. Further,
whether its substance is the faculty of thought or the act of thinking,
what does it think of? Either of itself or of something else; and
if of something else, either of the same thing always or of something
different. Does it matter, then, or not, whether it thinks of the
good or of any chance thing? Are there not some things about which
it is incredible that it should think? Evidently, then, it thinks
of that which is most divine and precious, and it does not change;
for change would be change for the worse, and this would be already
a movement. First, then, if 'thought' is not the act of thinking but
a potency, it would be reasonable to suppose that the continuity of
its thinking is wearisome to it. Secondly, there would evidently be
something else more precious than thought, viz. that which is thought
of. For both thinking and the act of thought will belong even to one
who thinks of the worst thing in the world, so that if this ought
to be avoided (and it ought, for there are even some things which
it is better not to see than to see), the act of thinking cannot be
the best of things. Therefore it must be of itself that the divine
thought thinks (since it is the most excellent of things), and its
thinking is a thinking on thinking. 

"But evidently knowledge and perception and opinion and understanding
have always something else as their object, and themselves only by
the way. Further, if thinking and being thought of are different,
in respect of which does goodness belong to thought? For to he an
act of thinking and to he an object of thought are not the same thing.
We answer that in some cases the knowledge is the object. In the productive
sciences it is the substance or essence of the object, matter omitted,
and in the theoretical sciences the definition or the act of thinking
is the object. Since, then, thought and the object of thought are
not different in the case of things that have not matter, the divine
thought and its object will be the same, i.e. the thinking will be
one with the object of its thought. 

"A further question is left-whether the object of the divine thought
is composite; for if it were, thought would change in passing from
part to part of the whole. We answer that everything which has not
matter is indivisible-as human thought, or rather the thought of composite
beings, is in a certain period of time (for it does not possess the
good at this moment or at that, but its best, being something different
from it, is attained only in a whole period of time), so throughout
eternity is the thought which has itself for its object.

Part 10 "

"We must consider also in which of two ways the nature of the universe
contains the good, and the highest good, whether as something separate
and by itself, or as the order of the parts. Probably in both ways,
as an army does; for its good is found both in its order and in its
leader, and more in the latter; for he does not depend on the order
but it depends on him. And all things are ordered together somehow,
but not all alike,-both fishes and fowls and plants; and the world
is not such that one thing has nothing to do with another, but they
are connected. For all are ordered together to one end, but it is
as in a house, where the freemen are least at liberty to act at random,
but all things or most things are already ordained for them, while
the slaves and the animals do little for the common good, and for
the most part live at random; for this is the sort of principle that
constitutes the nature of each. I mean, for instance, that all must
at least come to be dissolved into their elements, and there are other
functions similarly in which all share for the good of the whole.

"We must not fail to observe how many impossible or paradoxical results
confront those who hold different views from our own, and what are
the views of the subtler thinkers, and which views are attended by
fewest difficulties. All make all things out of contraries. But neither
'all things' nor 'out of contraries' is right; nor do these thinkers
tell us how all the things in which the contraries are present can
be made out of the contraries; for contraries are not affected by
one another. Now for us this difficulty is solved naturally by the
fact that there is a third element. These thinkers however make one
of the two contraries matter; this is done for instance by those who
make the unequal matter for the equal, or the many matter for the
one. But this also is refuted in the same way; for the one matter
which underlies any pair of contraries is contrary to nothing. Further,
all things, except the one, will, on the view we are criticizing,
partake of evil; for the bad itself is one of the two elements. But
the other school does not treat the good and the bad even as principles;
yet in all things the good is in the highest degree a principle. The
school we first mentioned is right in saying that it is a principle,
but how the good is a principle they do not say-whether as end or
as mover or as form. 

"Empedocles also has a paradoxical view; for he identifies the good
with love, but this is a principle both as mover (for it brings things
together) and as matter (for it is part of the mixture). Now even
if it happens that the same thing is a principle both as matter and
as mover, still the being, at least, of the two is not the same. In
which respect then is love a principle? It is paradoxical also that
strife should be imperishable; the nature of his 'evil' is just strife.

"Anaxagoras makes the good a motive principle; for his 'reason' moves
things. But it moves them for an end, which must be something other
than it, except according to our way of stating the case; for, on
our view, the medical art is in a sense health. It is paradoxical
also not to suppose a contrary to the good, i.e. to reason. But all
who speak of the contraries make no use of the contraries, unless
we bring their views into shape. And why some things are perishable
and others imperishable, no one tells us; for they make all existing
things out of the same principles. Further, some make existing things
out of the nonexistent; and others to avoid the necessity of this
make all things one. 

"Further, why should there always be becoming, and what is the cause
of becoming?-this no one tells us. And those who suppose two principles
must suppose another, a superior principle, and so must those who
believe in the Forms; for why did things come to participate, or why
do they participate, in the Forms? And all other thinkers are confronted
by the necessary consequence that there is something contrary to Wisdom,
i.e. to the highest knowledge; but we are not. For there is nothing
contrary to that which is primary; for all contraries have matter,
and things that have matter exist only potentially; and the ignorance
which is contrary to any knowledge leads to an object contrary to
the object of the knowledge; but what is primary has no contrary.

"Again, if besides sensible things no others exist, there will be
no first principle, no order, no becoming, no heavenly bodies, but
each principle will have a principle before it, as in the accounts
of the theologians and all the natural philosophers. But if the Forms
or the numbers are to exist, they will be causes of nothing; or if
not that, at least not of movement. Further, how is extension, i.e.
a continuum, to be produced out of unextended parts? For number will
not, either as mover or as form, produce a continuum. But again there
cannot be any contrary that is also essentially a productive or moving
principle; for it would be possible for it not to be. Or at least
its action would be posterior to its potency. The world, then, would
not be eternal. But it is; one of these premisses, then, must be denied.
And we have said how this must be done. Further, in virtue of what
the numbers, or the soul and the body, or in general the form and
the thing, are one-of this no one tells us anything; nor can any one
tell, unless he says, as we do, that the mover makes them one. And
those who say mathematical number is first and go on to generate one
kind of substance after another and give different principles for
each, make the substance of the universe a mere series of episodes
(for one substance has no influence on another by its existence or
nonexistence), and they give us many governing principles; but the
world refuses to be governed badly. "

"'The rule of many is not good; one ruler let there be.'

----------------------------------------------------------------------

BOOK XIII

Part 1 

"

"WE have stated what is the substance of sensible things, dealing
in the treatise on physics with matter, and later with the substance
which has actual existence. Now since our inquiry is whether there
is or is not besides the sensible substances any which is immovable
and eternal, and, if there is, what it is, we must first consider
what is said by others, so that, if there is anything which they say
wrongly, we may not be liable to the same objections, while, if there
is any opinion common to them and us, we shall have no private grievance
against ourselves on that account; for one must be content to state
some points better than one's predecessors, and others no worse.

"Two opinions are held on this subject; it is said that the objects
of mathematics-i.e. numbers and lines and the like-are substances,
and again that the Ideas are substances. And (1) since some recognize
these as two different classes-the Ideas and the mathematical numbers,
and (2) some recognize both as having one nature, while (3) some others
say that the mathematical substances are the only substances, we must
consider first the objects of mathematics, not qualifying them by
any other characteristic-not asking, for instance, whether they are
in fact Ideas or not, or whether they are the principles and substances
of existing things or not, but only whether as objects of mathematics
they exist or not, and if they exist, how they exist. Then after this
we must separately consider the Ideas themselves in a general way,
and only as far as the accepted mode of treatment demands; for most
of the points have been repeatedly made even by the discussions outside
our school, and, further, the greater part of our account must finish
by throwing light on that inquiry, viz. when we examine whether the
substances and the principles of existing things are numbers and Ideas;
for after the discussion of the Ideas this remans as a third inquiry.

"If the objects of mathematics exist, they must exist either in sensible
objects, as some say, or separate from sensible objects (and this
also is said by some); or if they exist in neither of these ways,
either they do not exist, or they exist only in some special sense.
So that the subject of our discussion will be not whether they exist
but how they exist. 

Part 2 "

"That it is impossible for mathematical objects to exist in sensible
things, and at the same time that the doctrine in question is an artificial
one, has been said already in our discussion of difficulties we have
pointed out that it is impossible for two solids to be in the same
place, and also that according to the same argument the other powers
and characteristics also should exist in sensible things and none
of them separately. This we have said already. But, further, it is
obvious that on this theory it is impossible for any body whatever
to be divided; for it would have to be divided at a plane, and the
plane at a line, and the line at a point, so that if the point cannot
be divided, neither can the line, and if the line cannot, neither
can the plane nor the solid. What difference, then, does it make whether
sensible things are such indivisible entities, or, without being so
themselves, have indivisible entities in them? The result will be
the same; if the sensible entities are divided the others will be
divided too, or else not even the sensible entities can be divided.

"But, again, it is not possible that such entities should exist separately.
For if besides the sensible solids there are to be other solids which
are separate from them and prior to the sensible solids, it is plain
that besides the planes also there must be other and separate planes
and points and lines; for consistency requires this. But if these
exist, again besides the planes and lines and points of the mathematical
solid there must be others which are separate. (For incomposites are
prior to compounds; and if there are, prior to the sensible bodies,
bodies which are not sensible, by the same argument the planes which
exist by themselves must be prior to those which are in the motionless
solids. Therefore these will be planes and lines other than those
that exist along with the mathematical solids to which these thinkers
assign separate existence; for the latter exist along with the mathematical
solids, while the others are prior to the mathematical solids.) Again,
therefore, there will be, belonging to these planes, lines, and prior
to them there will have to be, by the same argument, other lines and
points; and prior to these points in the prior lines there will have
to be other points, though there will be no others prior to these.
Now (1) the accumulation becomes absurd; for we find ourselves with
one set of solids apart from the sensible solids; three sets of planes
apart from the sensible planes-those which exist apart from the sensible
planes, and those in the mathematical solids, and those which exist
apart from those in the mathematical solids; four sets of lines, and
five sets of points. With which of these, then, will the mathematical
sciences deal? Certainly not with the planes and lines and points
in the motionless solid; for science always deals with what is prior.
And (the same account will apply also to numbers; for there will be
a different set of units apart from each set of points, and also apart
from each set of realities, from the objects of sense and again from
those of thought; so that there will be various classes of mathematical
numbers. 

"Again, how is it possible to solve the questions which we have already
enumerated in our discussion of difficulties? For the objects of astronomy
will exist apart from sensible things just as the objects of geometry
will; but how is it possible that a heaven and its parts-or anything
else which has movement-should exist apart? Similarly also the objects
of optics and of harmonics will exist apart; for there will be both
voice and sight besides the sensible or individual voices and sights.
Therefore it is plain that the other senses as well, and the other
objects of sense, will exist apart; for why should one set of them
do so and another not? And if this is so, there will also be animals
existing apart, since there will be senses. 

"Again, there are certain mathematical theorems that are universal,
extending beyond these substances. Here then we shall have another
intermediate substance separate both from the Ideas and from the intermediates,-a
substance which is neither number nor points nor spatial magnitude
nor time. And if this is impossible, plainly it is also impossible
that the former entities should exist separate from sensible things.

"And, in general, conclusion contrary alike to the truth and to the
usual views follow, if one is to suppose the objects of mathematics
to exist thus as separate entities. For because they exist thus they
must be prior to sensible spatial magnitudes, but in truth they must
be posterior; for the incomplete spatial magnitude is in the order
of generation prior, but in the order of substance posterior, as the
lifeless is to the living. 

"Again, by virtue of what, and when, will mathematical magnitudes
be one? For things in our perceptible world are one in virtue of soul,
or of a part of soul, or of something else that is reasonable enough;
when these are not present, the thing is a plurality, and splits up
into parts. But in the case of the subjects of mathematics, which
are divisible and are quantities, what is the cause of their being
one and holding together? 

"Again, the modes of generation of the objects of mathematics show
that we are right. For the dimension first generated is length, then
comes breadth, lastly depth, and the process is complete. If, then,
that which is posterior in the order of generation is prior in the
order of substantiality, the solid will be prior to the plane and
the line. And in this way also it is both more complete and more whole,
because it can become animate. How, on the other hand, could a line
or a plane be animate? The supposition passes the power of our senses.

"Again, the solid is a sort of substance; for it already has in a
sense completeness. But how can lines be substances? Neither as a
form or shape, as the soul perhaps is, nor as matter, like the solid;
for we have no experience of anything that can be put together out
of lines or planes or points, while if these had been a sort of material
substance, we should have observed things which could be put together
out of them. 

"Grant, then, that they are prior in definition. Still not all things
that are prior in definition are also prior in substantiality. For
those things are prior in substantiality which when separated from
other things surpass them in the power of independent existence, but
things are prior in definition to those whose definitions are compounded
out of their definitions; and these two properties are not coextensive.
For if attributes do not exist apart from the substances (e.g. a 'mobile'
or a pale'), pale is prior to the pale man in definition, but not
in substantiality. For it cannot exist separately, but is always along
with the concrete thing; and by the concrete thing I mean the pale
man. Therefore it is plain that neither is the result of abstraction
prior nor that which is produced by adding determinants posterior;
for it is by adding a determinant to pale that we speak of the pale
man. 

"It has, then, been sufficiently pointed out that the objects of mathematics
are not substances in a higher degree than bodies are, and that they
are not prior to sensibles in being, but only in definition, and that
they cannot exist somewhere apart. But since it was not possible for
them to exist in sensibles either, it is plain that they either do
not exist at all or exist in a special sense and therefore do not
'exist' without qualification. For 'exist' has many senses.

Part 3 "

"For just as the universal propositions of mathematics deal not with
objects which exist separately, apart from extended magnitudes and
from numbers, but with magnitudes and numbers, not however qua such
as to have magnitude or to be divisible, clearly it is possible that
there should also be both propositions and demonstrations about sensible
magnitudes, not however qua sensible but qua possessed of certain
definite qualities. For as there are many propositions about things
merely considered as in motion, apart from what each such thing is
and from their accidents, and as it is not therefore necessary that
there should be either a mobile separate from sensibles, or a distinct
mobile entity in the sensibles, so too in the case of mobiles there
will be propositions and sciences, which treat them however not qua
mobile but only qua bodies, or again only qua planes, or only qua
lines, or qua divisibles, or qua indivisibles having position, or
only qua indivisibles. Thus since it is true to say without qualification
that not only things which are separable but also things which are
inseparable exist (for instance, that mobiles exist), it is true also
to say without qualification that the objects of mathematics exist,
and with the character ascribed to them by mathematicians. And as
it is true to say of the other sciences too, without qualification,
that they deal with such and such a subject-not with what is accidental
to it (e.g. not with the pale, if the healthy thing is pale, and the
science has the healthy as its subject), but with that which is the
subject of each science-with the healthy if it treats its object qua
healthy, with man if qua man:-so too is it with geometry; if its subjects
happen to be sensible, though it does not treat them qua sensible,
the mathematical sciences will not for that reason be sciences of
sensibles-nor, on the other hand, of other things separate from sensibles.
Many properties attach to things in virtue of their own nature as
possessed of each such character; e.g. there are attributes peculiar
to the animal qua female or qua male (yet there is no 'female' nor
'male' separate from animals); so that there are also attributes which
belong to things merely as lengths or as planes. And in proportion
as we are dealing with things which are prior in definition and simpler,
our knowledge has more accuracy, i.e. simplicity. Therefore a science
which abstracts from spatial magnitude is more precise than one which
takes it into account; and a science is most precise if it abstracts
from movement, but if it takes account of movement, it is most precise
if it deals with the primary movement, for this is the simplest; and
of this again uniform movement is the simplest form. 

"The same account may be given of harmonics and optics; for neither
considers its objects qua sight or qua voice, but qua lines and numbers;
but the latter are attributes proper to the former. And mechanics
too proceeds in the same way. Therefore if we suppose attributes separated
from their fellow attributes and make any inquiry concerning them
as such, we shall not for this reason be in error, any more than when
one draws a line on the ground and calls it a foot long when it is
not; for the error is not included in the premisses. 

"Each question will be best investigated in this way-by setting up
by an act of separation what is not separate, as the arithmetician
and the geometer do. For a man qua man is one indivisible thing; and
the arithmetician supposed one indivisible thing, and then considered
whether any attribute belongs to a man qua indivisible. But the geometer
treats him neither qua man nor qua indivisible, but as a solid. For
evidently the properties which would have belonged to him even if
perchance he had not been indivisible, can belong to him even apart
from these attributes. Thus, then, geometers speak correctly; they
talk about existing things, and their subjects do exist; for being
has two forms-it exists not only in complete reality but also materially.

"Now since the good and the beautiful are different (for the former
always implies conduct as its subject, while the beautiful is found
also in motionless things), those who assert that the mathematical
sciences say nothing of the beautiful or the good are in error. For
these sciences say and prove a great deal about them; if they do not
expressly mention them, but prove attributes which are their results
or their definitions, it is not true to say that they tell us nothing
about them. The chief forms of beauty are order and symmetry and definiteness,
which the mathematical sciences demonstrate in a special degree. And
since these (e.g. order and definiteness) are obviously causes of
many things, evidently these sciences must treat this sort of causative
principle also (i.e. the beautiful) as in some sense a cause. But
we shall speak more plainly elsewhere about these matters.

Part 4 "

"So much then for the objects of mathematics; we have said that they
exist and in what sense they exist, and in what sense they are prior
and in what sense not prior. Now, regarding the Ideas, we must first
examine the ideal theory itself, not connecting it in any way with
the nature of numbers, but treating it in the form in which it was
originally understood by those who first maintained the existence
of the Ideas. The supporters of the ideal theory were led to it because
on the question about the truth of things they accepted the Heraclitean
sayings which describe all sensible things as ever passing away, so
that if knowledge or thought is to have an object, there must be some
other and permanent entities, apart from those which are sensible;
for there could be no knowledge of things which were in a state of
flux. But when Socrates was occupying himself with the excellences
of character, and in connexion with them became the first to raise
the problem of universal definition (for of the physicists Democritus
only touched on the subject to a small extent, and defined, after
a fashion, the hot and the cold; while the Pythagoreans had before
this treated of a few things, whose definitions-e.g. those of opportunity,
justice, or marriage-they connected with numbers; but it was natural
that Socrates should be seeking the essence, for he was seeking to
syllogize, and 'what a thing is' is the starting-point of syllogisms;
for there was as yet none of the dialectical power which enables people
even without knowledge of the essence to speculate about contraries
and inquire whether the same science deals with contraries; for two
things may be fairly ascribed to Socrates-inductive arguments and
universal definition, both of which are concerned with the starting-point
of science):-but Socrates did not make the universals or the definitions
exist apart: they, however, gave them separate existence, and this
was the kind of thing they called Ideas. Therefore it followed for
them, almost by the same argument, that there must be Ideas of all
things that are spoken of universally, and it was almost as if a man
wished to count certain things, and while they were few thought he
would not be able to count them, but made more of them and then counted
them; for the Forms are, one may say, more numerous than the particular
sensible things, yet it was in seeking the causes of these that they
proceeded from them to the Forms. For to each thing there answers
an entity which has the same name and exists apart from the substances,
and so also in the case of all other groups there is a one over many,
whether these be of this world or eternal. 

"Again, of the ways in which it is proved that the Forms exist, none
is convincing; for from some no inference necessarily follows, and
from some arise Forms even of things of which they think there are
no Forms. For according to the arguments from the sciences there will
be Forms of all things of which there are sciences, and according
to the argument of the 'one over many' there will be Forms even of
negations, and according to the argument that thought has an object
when the individual object has perished, there will be Forms of perishable
things; for we have an image of these. Again, of the most accurate
arguments, some lead to Ideas of relations, of which they say there
is no independent class, and others introduce the 'third man'.

"And in general the arguments for the Forms destroy things for whose
existence the believers in Forms are more zealous than for the existence
of the Ideas; for it follows that not the dyad but number is first,
and that prior to number is the relative, and that this is prior to
the absolute-besides all the other points on which certain people,
by following out the opinions held about the Forms, came into conflict
with the principles of the theory. 

"Again, according to the assumption on the belief in the Ideas rests,
there will be Forms not only of substances but also of many other
things; for the concept is single not only in the case of substances,
but also in that of non-substances, and there are sciences of other
things than substance; and a thousand other such difficulties confront
them. But according to the necessities of the case and the opinions
about the Forms, if they can be shared in there must be Ideas of substances
only. For they are not shared in incidentally, but each Form must
be shared in as something not predicated of a subject. (By 'being
shared in incidentally' I mean that if a thing shares in 'double itself',
it shares also in 'eternal', but incidentally; for 'the double' happens
to be eternal.) Therefore the Forms will be substance. But the same
names indicate substance in this and in the ideal world (or what will
be the meaning of saying that there is something apart from the particulars-the
one over many?). And if the Ideas and the things that share in them
have the same form, there will be something common: for why should
'2' be one and the same in the perishable 2's, or in the 2's which
are many but eternal, and not the same in the '2 itself' as in the
individual 2? But if they have not the same form, they will have only
the name in common, and it is as if one were to call both Callias
and a piece of wood a 'man', without observing any community between
them. 

"But if we are to suppose that in other respects the common definitions
apply to the Forms, e.g. that 'plane figure' and the other parts of
the definition apply to the circle itself, but 'what really is' has
to be added, we must inquire whether this is not absolutely meaningless.
For to what is this to be added? To 'centre' or to 'plane' or to all
the parts of the definition? For all the elements in the essence are
Ideas, e.g. 'animal' and 'two-footed'. Further, there must be some
Ideal answering to 'plane' above, some nature which will be present
in all the Forms as their genus. 

Part 5 "

"Above all one might discuss the question what in the world the Forms
contribute to sensible things, either to those that are eternal or
to those that come into being and cease to be; for they cause neither
movement nor any change in them. But again they help in no wise either
towards the knowledge of other things (for they are not even the substance
of these, else they would have been in them), or towards their being,
if they are not in the individuals which share in them; though if
they were, they might be thought to be causes, as white causes whiteness
in a white object by entering into its composition. But this argument,
which was used first by Anaxagoras, and later by Eudoxus in his discussion
of difficulties and by certain others, is very easily upset; for it
is easy to collect many and insuperable objections to such a view.

"But, further, all other things cannot come from the Forms in any
of the usual senses of 'from'. And to say that they are patterns and
the other things share in them is to use empty words and poetical
metaphors. For what is it that works, looking to the Ideas? And any
thing can both be and come into being without being copied from something
else, so that, whether Socrates exists or not, a man like Socrates
might come to be. And evidently this might be so even if Socrates
were eternal. And there will be several patterns of the same thing,
and therefore several Forms; e.g. 'animal' and 'two-footed', and also
'man-himself', will be Forms of man. Again, the Forms are patterns
not only of sensible things, but of Forms themselves also; i.e. the
genus is the pattern of the various forms-of-a-genus; therefore the
same thing will be pattern and copy. 

"Again, it would seem impossible that substance and that whose substance
it is should exist apart; how, therefore, could the Ideas, being the
substances of things, exist apart? 

"In the Phaedo the case is stated in this way-that the Forms are causes
both of being and of becoming. Yet though the Forms exist, still things
do not come into being, unless there is something to originate movement;
and many other things come into being (e.g. a house or a ring) of
which they say there are no Forms. Clearly therefore even the things
of which they say there are Ideas can both be and come into being
owing to such causes as produce the things just mentioned, and not
owing to the Forms. But regarding the Ideas it is possible, both in
this way and by more abstract and accurate arguments, to collect many
objections like those we have considered. 

Part 6 "

"Since we have discussed these points, it is well to consider again
the results regarding numbers which confront those who say that numbers
are separable substances and first causes of things. If number is
an entity and its substance is nothing other than just number, as
some say, it follows that either (1) there is a first in it and a
second, each being different in species,-and either (a) this is true
of the units without exception, and any unit is inassociable with
any unit, or (b) they are all without exception successive, and any
of them are associable with any, as they say is the case with mathematical
number; for in mathematical number no one unit is in any way different
from another. Or (c) some units must be associable and some not; e.g.
suppose that 2 is first after 1, and then comes 3 and then the rest
of the number series, and the units in each number are associable,
e.g. those in the first 2 are associable with one another, and those
in the first 3 with one another, and so with the other numbers; but
the units in the '2-itself' are inassociable with those in the '3-itself';
and similarly in the case of the other successive numbers. And so
while mathematical number is counted thus-after 1, 2 (which consists
of another 1 besides the former 1), and 3 which consists of another
1 besides these two), and the other numbers similarly, ideal number
is counted thus-after 1, a distinct 2 which does not include the first
1, and a 3 which does not include the 2 and the rest of the number
series similarly. Or (2) one kind of number must be like the first
that was named, one like that which the mathematicians speak of, and
that which we have named last must be a third kind. 

"Again, these kinds of numbers must either be separable from things,
or not separable but in objects of perception (not however in the
way which we first considered, in the sense that objects of perception
consists of numbers which are present in them)-either one kind and
not another, or all of them. 

"These are of necessity the only ways in which the numbers can exist.
And of those who say that the 1 is the beginning and substance and
element of all things, and that number is formed from the 1 and something
else, almost every one has described number in one of these ways;
only no one has said all the units are inassociable. And this has
happened reasonably enough; for there can be no way besides those
mentioned. Some say both kinds of number exist, that which has a before
and after being identical with the Ideas, and mathematical number
being different from the Ideas and from sensible things, and both
being separable from sensible things; and others say mathematical
number alone exists, as the first of realities, separate from sensible
things. And the Pythagoreans, also, believe in one kind of number-the
mathematical; only they say it is not separate but sensible substances
are formed out of it. For they construct the whole universe out of
numbers-only not numbers consisting of abstract units; they suppose
the units to have spatial magnitude. But how the first 1 was constructed
so as to have magnitude, they seem unable to say. 

"Another thinker says the first kind of number, that of the Forms,
alone exists, and some say mathematical number is identical with this.

"The case of lines, planes, and solids is similar. For some think
that those which are the objects of mathematics are different from
those which come after the Ideas; and of those who express themselves
otherwise some speak of the objects of mathematics and in a mathematical
way-viz. those who do not make the Ideas numbers nor say that Ideas
exist; and others speak of the objects of mathematics, but not mathematically;
for they say that neither is every spatial magnitude divisible into
magnitudes, nor do any two units taken at random make 2. All who say
the 1 is an element and principle of things suppose numbers to consist
of abstract units, except the Pythagoreans; but they suppose the numbers
to have magnitude, as has been said before. It is clear from this
statement, then, in how many ways numbers may be described, and that
all the ways have been mentioned; and all these views are impossible,
but some perhaps more than others. 

Part 7 "

"First, then, let us inquire if the units are associable or inassociable,
and if inassociable, in which of the two ways we distinguished. For
it is possible that any unity is inassociable with any, and it is
possible that those in the 'itself' are inassociable with those in
the 'itself', and, generally, that those in each ideal number are
inassociable with those in other ideal numbers. Now (1) all units
are associable and without difference, we get mathematical number-only
one kind of number, and the Ideas cannot be the numbers. For what
sort of number will man-himself or animal-itself or any other Form
be? There is one Idea of each thing e.g. one of man-himself and another
one of animal-itself; but the similar and undifferentiated numbers
are infinitely many, so that any particular 3 is no more man-himself
than any other 3. But if the Ideas are not numbers, neither can they
exist at all. For from what principles will the Ideas come? It is
number that comes from the 1 and the indefinite dyad, and the principles
or elements are said to be principles and elements of number, and
the Ideas cannot be ranked as either prior or posterior to the numbers.

"But (2) if the units are inassociable, and inassociable in the sense
that any is inassociable with any other, number of this sort cannot
be mathematical number; for mathematical number consists of undifferentiated
units, and the truths proved of it suit this character. Nor can it
be ideal number. For 2 will not proceed immediately from 1 and the
indefinite dyad, and be followed by the successive numbers, as they
say '2,3,4' for the units in the ideal are generated at the same time,
whether, as the first holder of the theory said, from unequals (coming
into being when these were equalized) or in some other way-since,
if one unit is to be prior to the other, it will be prior also to
2 the composed of these; for when there is one thing prior and another
posterior, the resultant of these will be prior to one and posterior
to the other. Again, since the 1-itself is first, and then there is
a particular 1 which is first among the others and next after the
1-itself, and again a third which is next after the second and next
but one after the first 1,-so the units must be prior to the numbers
after which they are named when we count them; e.g. there will be
a third unit in 2 before 3 exists, and a fourth and a fifth in 3 before
the numbers 4 and 5 exist.-Now none of these thinkers has said the
units are inassociable in this way, but according to their principles
it is reasonable that they should be so even in this way, though in
truth it is impossible. For it is reasonable both that the units should
have priority and posteriority if there is a first unit or first 1,
and also that the 2's should if there is a first 2; for after the
first it is reasonable and necessary that there should be a second,
and if a second, a third, and so with the others successively. (And
to say both things at the same time, that a unit is first and another
unit is second after the ideal 1, and that a 2 is first after it,
is impossible.) But they make a first unit or 1, but not also a second
and a third, and a first 2, but not also a second and a third. Clearly,
also, it is not possible, if all the units are inassociable, that
there should be a 2-itself and a 3-itself; and so with the other numbers.
For whether the units are undifferentiated or different each from
each, number must be counted by addition, e.g. 2 by adding another
1 to the one, 3 by adding another 1 to the two, and similarly. This
being so, numbers cannot be generated as they generate them, from
the 2 and the 1; for 2 becomes part of 3 and 3 of 4 and the same happens
in the case of the succeeding numbers, but they say 4 came from the
first 2 and the indefinite which makes it two 2's other than the 2-itself;
if not, the 2-itself will be a part of 4 and one other 2 will be added.
And similarly 2 will consist of the 1-itself and another 1; but if
this is so, the other element cannot be an indefinite 2; for it generates
one unit, not, as the indefinite 2 does, a definite 2. 

"Again, besides the 3-itself and the 2-itself how can there be other
3's and 2's? And how do they consist of prior and posterior units?
All this is absurd and fictitious, and there cannot be a first 2 and
then a 3-itself. Yet there must, if the 1 and the indefinite dyad
are to be the elements. But if the results are impossible, it is also
impossible that these are the generating principles. 

"If the units, then, are differentiated, each from each, these results
and others similar to these follow of necessity. But (3) if those
in different numbers are differentiated, but those in the same number
are alone undifferentiated from one another, even so the difficulties
that follow are no less. E.g. in the 10-itself their are ten units,
and the 10 is composed both of them and of two 5's. But since the
10-itself is not any chance number nor composed of any chance 5's--or,
for that matter, units--the units in this 10 must differ. For if they
do not differ, neither will the 5's of which the 10 consists differ;
but since these differ, the units also will differ. But if they differ,
will there be no other 5's in the 10 but only these two, or will there
be others? If there are not, this is paradoxical; and if there are,
what sort of 10 will consist of them? For there is no other in the
10 but the 10 itself. But it is actually necessary on their view that
the 4 should not consist of any chance 2's; for the indefinite as
they say, received the definite 2 and made two 2's; for its nature
was to double what it received. 

"Again, as to the 2 being an entity apart from its two units, and
the 3 an entity apart from its three units, how is this possible?
Either by one's sharing in the other, as 'pale man' is different from
'pale' and 'man' (for it shares in these), or when one is a differentia
of the other, as 'man' is different from 'animal' and 'two-footed'.

"Again, some things are one by contact, some by intermixture, some
by position; none of which can belong to the units of which the 2
or the 3 consists; but as two men are not a unity apart from both,
so must it be with the units. And their being indivisible will make
no difference to them; for points too are indivisible, but yet a pair
of them is nothing apart from the two. 

"But this consequence also we must not forget, that it follows that
there are prior and posterior 2 and similarly with the other numbers.
For let the 2's in the 4 be simultaneous; yet these are prior to those
in the 8 and as the 2 generated them, they generated the 4's in the
8-itself. Therefore if the first 2 is an Idea, these 2's also will
be Ideas of some kind. And the same account applies to the units;
for the units in the first 2 generate the four in 4, so that all the
units come to be Ideas and an Idea will be composed of Ideas. Clearly
therefore those things also of which these happen to be the Ideas
will be composite, e.g. one might say that animals are composed of
animals, if there are Ideas of them. 

"In general, to differentiate the units in any way is an absurdity
and a fiction; and by a fiction I mean a forced statement made to
suit a hypothesis. For neither in quantity nor in quality do we see
unit differing from unit, and number must be either equal or unequal-all
number but especially that which consists of abstract units-so that
if one number is neither greater nor less than another, it is equal
to it; but things that are equal and in no wise differentiated we
take to be the same when we are speaking of numbers. If not, not even
the 2 in the 10-itself will be undifferentiated, though they are equal;
for what reason will the man who alleges that they are not differentiated
be able to give? 

"Again, if every unit + another unit makes two, a unit from the 2-itself
and one from the 3-itself will make a 2. Now (a) this will consist
of differentiated units; and will it be prior to the 3 or posterior?
It rather seems that it must be prior; for one of the units is simultaneous
with the 3 and the other is simultaneous with the 2. And we, for our
part, suppose that in general 1 and 1, whether the things are equal
or unequal, is 2, e.g. the good and the bad, or a man and a horse;
but those who hold these views say that not even two units are 2.

"If the number of the 3-itself is not greater than that of the 2,
this is surprising; and if it is greater, clearly there is also a
number in it equal to the 2, so that this is not different from the
2-itself. But this is not possible, if there is a first and a second
number. 

"Nor will the Ideas be numbers. For in this particular point they
are right who claim that the units must be different, if there are
to be Ideas; as has been said before. For the Form is unique; but
if the units are not different, the 2's and the 3's also will not
be different. This is also the reason why they must say that when
we count thus-'1,2'-we do not proceed by adding to the given number;
for if we do, neither will the numbers be generated from the indefinite
dyad, nor can a number be an Idea; for then one Idea will be in another,
and all Forms will be parts of one Form. And so with a view to their
hypothesis their statements are right, but as a whole they are wrong;
for their view is very destructive, since they will admit that this
question itself affords some difficulty-whether, when we count and
say -1,2,3-we count by addition or by separate portions. But we do
both; and so it is absurd to reason back from this problem to so great
a difference of essence. 

Part 8 "

"First of all it is well to determine what is the differentia of a
number-and of a unit, if it has a differentia. Units must differ either
in quantity or in quality; and neither of these seems to be possible.
But number qua number differs in quantity. And if the units also did
differ in quantity, number would differ from number, though equal
in number of units. Again, are the first units greater or smaller,
and do the later ones increase or diminish? All these are irrational
suppositions. But neither can they differ in quality. For no attribute
can attach to them; for even to numbers quality is said to belong
after quantity. Again, quality could not come to them either from
the 1 or the dyad; for the former has no quality, and the latter gives
quantity; for this entity is what makes things to be many. If the
facts are really otherwise, they should state this quite at the beginning
and determine if possible, regarding the differentia of the unit,
why it must exist, and, failing this, what differentia they mean.

"Evidently then, if the Ideas are numbers, the units cannot all be
associable, nor can they be inassociable in either of the two ways.
But neither is the way in which some others speak about numbers correct.
These are those who do not think there are Ideas, either without qualification
or as identified with certain numbers, but think the objects of mathematics
exist and the numbers are the first of existing things, and the 1-itself
is the starting-point of them. It is paradoxical that there should
be a 1 which is first of 1's, as they say, but not a 2 which is first
of 2's, nor a 3 of 3's; for the same reasoning applies to all. If,
then, the facts with regard to number are so, and one supposes mathematical
number alone to exist, the 1 is not the starting-point (for this sort
of 1 must differ from the-other units; and if this is so, there must
also be a 2 which is first of 2's, and similarly with the other successive
numbers). But if the 1 is the starting-point, the truth about the
numbers must rather be what Plato used to say, and there must be a
first 2 and 3 and numbers must not be associable with one another.
But if on the other hand one supposes this, many impossible results,
as we have said, follow. But either this or the other must be the
case, so that if neither is, number cannot exist separately.

"It is evident, also, from this that the third version is the worst,-the
view ideal and mathematical number is the same. For two mistakes must
then meet in the one opinion. (1) Mathematical number cannot be of
this sort, but the holder of this view has to spin it out by making
suppositions peculiar to himself. And (2) he must also admit all the
consequences that confront those who speak of number in the sense
of 'Forms'. 

"The Pythagorean version in one way affords fewer difficulties than
those before named, but in another way has others peculiar to itself.
For not thinking of number as capable of existing separately removes
many of the impossible consequences; but that bodies should be composed
of numbers, and that this should be mathematical number, is impossible.
For it is not true to speak of indivisible spatial magnitudes; and
however much there might be magnitudes of this sort, units at least
have not magnitude; and how can a magnitude be composed of indivisibles?
But arithmetical number, at least, consists of units, while these
thinkers identify number with real things; at any rate they apply
their propositions to bodies as if they consisted of those numbers.

"If, then, it is necessary, if number is a self-subsistent real thing,
that it should exist in one of these ways which have been mentioned,
and if it cannot exist in any of these, evidently number has no such
nature as those who make it separable set up for it. 

"Again, does each unit come from the great and the small, equalized,
or one from the small, another from the great? (a) If the latter,
neither does each thing contain all the elements, nor are the units
without difference; for in one there is the great and in another the
small, which is contrary in its nature to the great. Again, how is
it with the units in the 3-itself? One of them is an odd unit. But
perhaps it is for this reason that they give 1-itself the middle place
in odd numbers. (b) But if each of the two units consists of both
the great and the small, equalized, how will the 2 which is a single
thing, consist of the great and the small? Or how will it differ from
the unit? Again, the unit is prior to the 2; for when it is destroyed
the 2 is destroyed. It must, then, be the Idea of an Idea since it
is prior to an Idea, and it must have come into being before it. From
what, then? Not from the indefinite dyad, for its function was to
double. 

"Again, number must be either infinite or finite; for these thinkers
think of number as capable of existing separately, so that it is not
possible that neither of those alternatives should be true. Clearly
it cannot be infinite; for infinite number is neither odd nor even,
but the generation of numbers is always the generation either of an
odd or of an even number; in one way, when 1 operates on an even number,
an odd number is produced; in another way, when 2 operates, the numbers
got from 1 by doubling are produced; in another way, when the odd
numbers operate, the other even numbers are produced. Again, if every
Idea is an Idea of something, and the numbers are Ideas, infinite
number itself will be an Idea of something, either of some sensible
thing or of something else. Yet this is not possible in view of their
thesis any more than it is reasonable in itself, at least if they
arrange the Ideas as they do. 

"But if number is finite, how far does it go? With regard to this
not only the fact but the reason should be stated. But if number goes
only up to 10 as some say, firstly the Forms will soon run short;
e.g. if 3 is man-himself, what number will be the horse-itself? The
series of the numbers which are the several things-themselves goes
up to 10. It must, then, be one of the numbers within these limits;
for it is these that are substances and Ideas. Yet they will run short;
for the various forms of animal will outnumber them. At the same time
it is clear that if in this way the 3 is man-himself, the other 3's
are so also (for those in identical numbers are similar), so that
there will be an infinite number of men; if each 3 is an Idea, each
of the numbers will be man-himself, and if not, they will at least
be men. And if the smaller number is part of the greater (being number
of such a sort that the units in the same number are associable),
then if the 4-itself is an Idea of something, e.g. of 'horse' or of
'white', man will be a part of horse, if man is It is paradoxical
also that there should be an Idea of 10 but not of 11, nor of the
succeeding numbers. Again, there both are and come to be certain things
of which there are no Forms; why, then, are there not Forms of them
also? We infer that the Forms are not causes. Again, it is paradoxical-if
the number series up to 10 is more of a real thing and a Form than
10 itself. There is no generation of the former as one thing, and
there is of the latter. But they try to work on the assumption that
the series of numbers up to 10 is a complete series. At least they
generate the derivatives-e.g. the void, proportion, the odd, and the
others of this kind-within the decade. For some things, e.g. movement
and rest, good and bad, they assign to the originative principles,
and the others to the numbers. This is why they identify the odd with
1; for if the odd implied 3 how would 5 be odd? Again, spatial magnitudes
and all such things are explained without going beyond a definite
number; e.g. the first, the indivisible, line, then the 2 &c.; these
entities also extend only up to 10. 

"Again, if number can exist separately, one might ask which is prior-
1, or 3 or 2? Inasmuch as the number is composite, 1 is prior, but
inasmuch as the universal and the form is prior, the number is prior;
for each of the units is part of the number as its matter, and the
number acts as form. And in a sense the right angle is prior to the
acute, because it is determinate and in virtue of its definition;
but in a sense the acute is prior, because it is a part and the right
angle is divided into acute angles. As matter, then, the acute angle
and the element and the unit are prior, but in respect of the form
and of the substance as expressed in the definition, the right angle,
and the whole consisting of the matter and the form, are prior; for
the concrete thing is nearer to the form and to what is expressed
in the definition, though in generation it is later. How then is 1
the starting-point? Because it is not divisiable, they say; but both
the universal, and the particular or the element, are indivisible.
But they are starting-points in different ways, one in definition
and the other in time. In which way, then, is 1 the starting-point?
As has been said, the right angle is thought to be prior to the acute,
and the acute to the right, and each is one. Accordingly they make
1 the starting-point in both ways. But this is impossible. For the
universal is one as form or substance, while the element is one as
a part or as matter. For each of the two is in a sense one-in truth
each of the two units exists potentially (at least if the number is
a unity and not like a heap, i.e. if different numbers consist of
differentiated units, as they say), but not in complete reality; and
the cause of the error they fell into is that they were conducting
their inquiry at the same time from the standpoint of mathematics
and from that of universal definitions, so that (1) from the former
standpoint they treated unity, their first principle, as a point;
for the unit is a point without position. They put things together
out of the smallest parts, as some others also have done. Therefore
the unit becomes the matter of numbers and at the same time prior
to 2; and again posterior, 2 being treated as a whole, a unity, and
a form. But (2) because they were seeking the universal they treated
the unity which can be predicated of a number, as in this sense also
a part of the number. But these characteristics cannot belong at the
same time to the same thing. 

"If the 1-itself must be unitary (for it differs in nothing from other
1's except that it is the starting-point), and the 2 is divisible
but the unit is not, the unit must be liker the 1-itself than the
2 is. But if the unit is liker it, it must be liker to the unit than
to the 2; therefore each of the units in 2 must be prior to the 2.
But they deny this; at least they generate the 2 first. Again, if
the 2-itself is a unity and the 3-itself is one also, both form a
2. From what, then, is this 2 produced? 

Part 9 

"Since there is not contact in numbers, but succession, viz. between
the units between which there is nothing, e.g. between those in 2
or in 3 one might ask whether these succeed the 1-itself or not, and
whether, of the terms that succeed it, 2 or either of the units in
2 is prior. 

"Similar difficulties occur with regard to the classes of things posterior
to number,-the line, the plane, and the solid. For some construct
these out of the species of the 'great and small'; e.g. lines from
the 'long and short', planes from the 'broad and narrow', masses from
the 'deep and shallow'; which are species of the 'great and small'.
And the originative principle of such things which answers to the
1 different thinkers describe in different ways, And in these also
the impossibilities, the fictions, and the contradictions of all probability
are seen to be innumerable. For (i) geometrical classes are severed
from one another, unless the principles of these are implied in one
another in such a way that the 'broad and narrow' is also 'long and
short' (but if this is so, the plane will be line and the solid a
plane; again, how will angles and figures and such things be explained?).
And (ii) the same happens as in regard to number; for 'long and short',
&c., are attributes of magnitude, but magnitude does not consist of
these, any more than the line consists of 'straight and curved', or
solids of 'smooth and rough'. 

"(All these views share a difficulty which occurs with regard to species-of-a-genus,
when one posits the universals, viz. whether it is animal-itself or
something other than animal-itself that is in the particular animal.
True, if the universal is not separable from sensible things, this
will present no difficulty; but if the 1 and the numbers are separable,
as those who express these views say, it is not easy to solve the
difficulty, if one may apply the words 'not easy' to the impossible.
For when we apprehend the unity in 2, or in general in a number, do
we apprehend a thing-itself or something else?). 

"Some, then, generate spatial magnitudes from matter of this sort,
others from the point -and the point is thought by them to be not
1 but something like 1-and from other matter like plurality, but not
identical with it; about which principles none the less the same difficulties
occur. For if the matter is one, line and plane-and soli will be the
same; for from the same elements will come one and the same thing.
But if the matters are more than one, and there is one for the line
and a second for the plane and another for the solid, they either
are implied in one another or not, so that the same results will follow
even so; for either the plane will not contain a line or it will he
a line. 

"Again, how number can consist of the one and plurality, they make
no attempt to explain; but however they express themselves, the same
objections arise as confront those who construct number out of the
one and the indefinite dyad. For the one view generates number from
the universally predicated plurality, and not from a particular plurality;
and the other generates it from a particular plurality, but the first;
for 2 is said to be a 'first plurality'. Therefore there is practically
no difference, but the same difficulties will follow,-is it intermixture
or position or blending or generation? and so on. Above all one might
press the question 'if each unit is one, what does it come from?'
Certainly each is not the one-itself. It must, then, come from the
one itself and plurality, or a part of plurality. To say that the
unit is a plurality is impossible, for it is indivisible; and to generate
it from a part of plurality involves many other objections; for (a)
each of the parts must be indivisible (or it will be a plurality and
the unit will be divisible) and the elements will not be the one and
plurality; for the single units do not come from plurality and the
one. Again, (,the holder of this view does nothing but presuppose
another number; for his plurality of indivisibles is a number. Again,
we must inquire, in view of this theory also, whether the number is
infinite or finite. For there was at first, as it seems, a plurality
that was itself finite, from which and from the one comes the finite
number of units. And there is another plurality that is plurality-itself
and infinite plurality; which sort of plurality, then, is the element
which co-operates with the one? One might inquire similarly about
the point, i.e. the element out of which they make spatial magnitudes.
For surely this is not the one and only point; at any rate, then,
let them say out of what each of the points is formed. Certainly not
of some distance + the point-itself. Nor again can there be indivisible
parts of a distance, as the elements out of which the units are said
to be made are indivisible parts of plurality; for number consists
of indivisibles, but spatial magnitudes do not. 

"All these objections, then, and others of the sort make it evident
that number and spatial magnitudes cannot exist apart from things.
Again, the discord about numbers between the various versions is a
sign that it is the incorrectness of the alleged facts themselves
that brings confusion into the theories. For those who make the objects
of mathematics alone exist apart from sensible things, seeing the
difficulty about the Forms and their fictitiousness, abandoned ideal
number and posited mathematical. But those who wished to make the
Forms at the same time also numbers, but did not see, if one assumed
these principles, how mathematical number was to exist apart from
ideal, made ideal and mathematical number the same-in words, since
in fact mathematical number has been destroyed; for they state hypotheses
peculiar to themselves and not those of mathematics. And he who first
supposed that the Forms exist and that the Forms are numbers and that
the objects of mathematics exist, naturally separated the two. Therefore
it turns out that all of them are right in some respect, but on the
whole not right. And they themselves confirm this, for their statements
do not agree but conflict. The cause is that their hypotheses and
their principles are false. And it is hard to make a good case out
of bad materials, according to Epicharmus: 'as soon as 'tis said,
'tis seen to be wrong.' 

"But regarding numbers the questions we have raised and the conclusions
we have reached are sufficient (for while he who is already convinced
might be further convinced by a longer discussion, one not yet convinced
would not come any nearer to conviction); regarding the first principles
and the first causes and elements, the views expressed by those who
discuss only sensible substance have been partly stated in our works
on nature, and partly do not belong to the present inquiry; but the
views of those who assert that there are other substances besides
the sensible must be considered next after those we have been mentioning.
Since, then, some say that the Ideas and the numbers are such substances,
and that the elements of these are elements and principles of real
things, we must inquire regarding these what they say and in what
sense they say it. 

"Those who posit numbers only, and these mathematical, must be considered
later; but as regards those who believe in the Ideas one might survey
at the same time their way of thinking and the difficulty into which
they fall. For they at the same time make the Ideas universal and
again treat them as separable and as individuals. That this is not
possible has been argued before. The reason why those who described
their substances as universal combined these two characteristics in
one thing, is that they did not make substances identical with sensible
things. They thought that the particulars in the sensible world were
a state of flux and none of them remained, but that the universal
was apart from these and something different. And Socrates gave the
impulse to this theory, as we said in our earlier discussion, by reason
of his definitions, but he did not separate universals from individuals;
and in this he thought rightly, in not separating them. This is plain
from the results; for without the universal it is not possible to
get knowledge, but the separation is the cause of the objections that
arise with regard to the Ideas. His successors, however, treating
it as necessary, if there are to be any substances besides the sensible
and transient substances, that they must be separable, had no others,
but gave separate existence to these universally predicated substances,
so that it followed that universals and individuals were almost the
same sort of thing. This in itself, then, would be one difficulty
in the view we have mentioned. 

Part 10 "

"Let us now mention a point which presents a certain difficulty both
to those who believe in the Ideas and to those who do not, and which
was stated before, at the beginning, among the problems. If we do
not suppose substances to be separate, and in the way in which individual
things are said to be separate, we shall destroy substance in the
sense in which we understand 'substance'; but if we conceive substances
to be separable, how are we to conceive their elements and their principles?

"If they are individual and not universal, (a) real things will be
just of the same number as the elements, and (b) the elements will
not be knowable. For (a) let the syllables in speech be substances,
and their elements elements of substances; then there must be only
one 'ba' and one of each of the syllables, since they are not universal
and the same in form but each is one in number and a 'this' and not
a kind possessed of a common name (and again they suppose that the
'just what a thing is' is in each case one). And if the syllables
are unique, so too are the parts of which they consist; there will
not, then, be more a's than one, nor more than one of any of the other
elements, on the same principle on which an identical syllable cannot
exist in the plural number. But if this is so, there will not be other
things existing besides the elements, but only the elements.

"(b) Again, the elements will not be even knowable; for they are not
universal, and knowledge is of universals. This is clear from demonstrations
and from definitions; for we do not conclude that this triangle has
its angles equal to two right angles, unless every triangle has its
angles equal to two right angles, nor that this man is an animal,
unless every man is an animal. 

"But if the principles are universal, either the substances composed
of them are also universal, or non-substance will be prior to substance;
for the universal is not a substance, but the element or principle
is universal, and the element or principle is prior to the things
of which it is the principle or element. 

"All these difficulties follow naturally, when they make the Ideas
out of elements and at the same time claim that apart from the substances
which have the same form there are Ideas, a single separate entity.
But if, e.g. in the case of the elements of speech, the a's and the
b's may quite well be many and there need be no a-itself and b-itself
besides the many, there may be, so far as this goes, an infinite number
of similar syllables. The statement that an knowledge is universal,
so that the principles of things must also be universal and not separate
substances, presents indeed, of all the points we have mentioned,
the greatest difficulty, but yet the statement is in a sense true,
although in a sense it is not. For knowledge, like the verb 'to know',
means two things, of which one is potential and one actual. The potency,
being, as matter, universal and indefinite, deals with the universal
and indefinite; but the actuality, being definite, deals with a definite
object, being a 'this', it deals with a 'this'. But per accidens sight
sees universal colour, because this individual colour which it sees
is colour; and this individual a which the grammarian investigates
is an a. For if the principles must be universal, what is derived
from them must also be universal, as in demonstrations; and if this
is so, there will be nothing capable of separate existence-i.e. no
substance. But evidently in a sense knowledge is universal, and in
a sense it is not. 

----------------------------------------------------------------------

BOOK XIV

Part 1 

"

"REGARDING this kind of substance, what we have said must be taken
as sufficient. All philosophers make the first principles contraries:
as in natural things, so also in the case of unchangeable substances.
But since there cannot be anything prior to the first principle of
all things, the principle cannot be the principle and yet be an attribute
of something else. To suggest this is like saying that the white is
a first principle, not qua anything else but qua white, but yet that
it is predicable of a subject, i.e. that its being white presupposes
its being something else; this is absurd, for then that subject will
be prior. But all things which are generated from their contraries
involve an underlying subject; a subject, then, must be present in
the case of contraries, if anywhere. All contraries, then, are always
predicable of a subject, and none can exist apart, but just as appearances
suggest that there is nothing contrary to substance, argument confirms
this. No contrary, then, is the first principle of all things in the
full sense; the first principle is something different. 

"But these thinkers make one of the contraries matter, some making
the unequal which they take to be the essence of plurality-matter
for the One, and others making plurality matter for the One. (The
former generate numbers out of the dyad of the unequal, i.e. of the
great and small, and the other thinker we have referred to generates
them out of plurality, while according to both it is generated by
the essence of the One.) For even the philosopher who says the unequal
and the One are the elements, and the unequal is a dyad composed of
the great and small, treats the unequal, or the great and the small,
as being one, and does not draw the distinction that they are one
in definition, but not in number. But they do not describe rightly
even the principles which they call elements, for some name the great
and the small with the One and treat these three as elements of numbers,
two being matter, one the form; while others name the many and few,
because the great and the small are more appropriate in their nature
to magnitude than to number; and others name rather the universal
character common to these-'that which exceeds and that which is exceeded'.
None of these varieties of opinion makes any difference to speak of,
in view of some of the consequences; they affect only the abstract
objections, which these thinkers take care to avoid because the demonstrations
they themselves offer are abstract,-with this exception, that if the
exceeding and the exceeded are the principles, and not the great and
the small, consistency requires that number should come from the elements
before does; for number is more universal than as the exceeding and
the exceeded are more universal than the great and the small. But
as it is, they say one of these things but do not say the other. Others
oppose the different and the other to the One, and others oppose plurality
to the One. But if, as they claim, things consist of contraries, and
to the One either there is nothing contrary, or if there is to be
anything it is plurality, and the unequal is contrary to the equal,
and the different to the same, and the other to the thing itself,
those who oppose the One to plurality have most claim to plausibility,
but even their view is inadequate, for the One would on their view
be a few; for plurality is opposed to fewness, and the many to the
few. 

"'The one' evidently means a measure. And in every case there is some
underlying thing with a distinct nature of its own, e.g. in the scale
a quarter-tone, in spatial magnitude a finger or a foot or something
of the sort, in rhythms a beat or a syllable; and similarly in gravity
it is a definite weight; and in the same way in all cases, in qualities
a quality, in quantities a quantity (and the measure is indivisible,
in the former case in kind, and in the latter to the sense); which
implies that the one is not in itself the substance of anything. And
this is reasonable; for 'the one' means the measure of some plurality,
and 'number' means a measured plurality and a plurality of measures.
(Thus it is natural that one is not a number; for the measure is not
measures, but both the measure and the one are starting-points.) The
measure must always be some identical thing predicable of all the
things it measures, e.g. if the things are horses, the measure is
'horse', and if they are men, 'man'. If they are a man, a horse, and
a god, the measure is perhaps 'living being', and the number of them
will be a number of living beings. If the things are 'man' and 'pale'
and 'walking', these will scarcely have a number, because all belong
to a subject which is one and the same in number, yet the number of
these will be a number of 'kinds' or of some such term. 

"Those who treat the unequal as one thing, and the dyad as an indefinite
compound of great and small, say what is very far from being probable
or possible. For (a) these are modifications and accidents, rather
than substrata, of numbers and magnitudes-the many and few of number,
and the great and small of magnitude-like even and odd, smooth and
rough, straight and curved. Again, (b) apart from this mistake, the
great and the small, and so on, must be relative to something; but
what is relative is least of all things a kind of entity or substance,
and is posterior to quality and quantity; and the relative is an accident
of quantity, as was said, not its matter, since something with a distinct
nature of its own must serve as matter both to the relative in general
and to its parts and kinds. For there is nothing either great or small,
many or few, or, in general, relative to something else, which without
having a nature of its own is many or few, great or small, or relative
to something else. A sign that the relative is least of all a substance
and a real thing is the fact that it alone has no proper generation
or destruction or movement, as in respect of quantity there is increase
and diminution, in respect of quality alteration, in respect of place
locomotion, in respect of substance simple generation and destruction.
In respect of relation there is no proper change; for, without changing,
a thing will be now greater and now less or equal, if that with which
it is compared has changed in quantity. And (c) the matter of each
thing, and therefore of substance, must be that which is potentially
of the nature in question; but the relative is neither potentially
nor actually substance. It is strange, then, or rather impossible,
to make not-substance an element in, and prior to, substance; for
all the categories are posterior to substance. Again, (d) elements
are not predicated of the things of which they are elements, but many
and few are predicated both apart and together of number, and long
and short of the line, and both broad and narrow apply to the plane.
If there is a plurality, then, of which the one term, viz. few, is
always predicated, e.g. 2 (which cannot be many, for if it were many,
1 would be few), there must be also one which is absolutely many,
e.g. 10 is many (if there is no number which is greater than 10),
or 10,000. How then, in view of this, can number consist of few and
many? Either both ought to be predicated of it, or neither; but in
fact only the one or the other is predicated. 

Part 2 "

"We must inquire generally, whether eternal things can consist of
elements. If they do, they will have matter; for everything that consists
of elements is composite. Since, then, even if a thing exists for
ever, out of that of which it consists it would necessarily also,
if it had come into being, have come into being, and since everything
comes to be what it comes to be out of that which is it potentially
(for it could not have come to be out of that which had not this capacity,
nor could it consist of such elements), and since the potential can
be either actual or not,-this being so, however everlasting number
or anything else that has matter is, it must be capable of not existing,
just as that which is any number of years old is as capable of not
existing as that which is a day old; if this is capable of not existing,
so is that which has lasted for a time so long that it has no limit.
They cannot, then, be eternal, since that which is capable of not
existing is not eternal, as we had occasion to show in another context.
If that which we are now saying is true universally-that no substance
is eternal unless it is actuality-and if the elements are matter that
underlies substance, no eternal substance can have elements present
in it, of which it consists. 

"There are some who describe the element which acts with the One as
an indefinite dyad, and object to 'the unequal', reasonably enough,
because of the ensuing difficulties; but they have got rid only of
those objections which inevitably arise from the treatment of the
unequal, i.e. the relative, as an element; those which arise apart
from this opinion must confront even these thinkers, whether it is
ideal number, or mathematical, that they construct out of those elements.

"There are many causes which led them off into these explanations,
and especially the fact that they framed the difficulty in an obsolete
form. For they thought that all things that are would be one (viz.
Being itself), if one did not join issue with and refute the saying
of Parmenides: "

"'For never will this he proved, that things that are not are.'
"

"They thought it necessary to prove that that which is not is; for
only thus-of that which is and something else-could the things that
are be composed, if they are many. 

"But, first, if 'being' has many senses (for it means sometimes substance,
sometimes that it is of a certain quality, sometimes that it is of
a certain quantity, and at other times the other categories), what
sort of 'one', then, are all the things that are, if non-being is
to be supposed not to be? Is it the substances that are one, or the
affections and similarly the other categories as well, or all together-so
that the 'this' and the 'such' and the 'so much' and the other categories
that indicate each some one class of being will all be one? But it
is strange, or rather impossible, that the coming into play of a single
thing should bring it about that part of that which is is a 'this',
part a 'such', part a 'so much', part a 'here'. 

"Secondly, of what sort of non-being and being do the things that
are consist? For 'nonbeing' also has many senses, since 'being' has;
and 'not being a man' means not being a certain substance, 'not being
straight' not being of a certain quality, 'not being three cubits
long' not being of a certain quantity. What sort of being and non-being,
then, by their union pluralize the things that are? This thinker means
by the non-being the union of which with being pluralizes the things
that are, the false and the character of falsity. This is also why
it used to be said that we must assume something that is false, as
geometers assume the line which is not a foot long to be a foot long.
But this cannot be so. For neither do geometers assume anything false
(for the enunciation is extraneous to the inference), nor is it non-being
in this sense that the things that are are generated from or resolved
into. But since 'non-being' taken in its various cases has as many
senses as there are categories, and besides this the false is said
not to be, and so is the potential, it is from this that generation
proceeds, man from that which is not man but potentially man, and
white from that which is not white but potentially white, and this
whether it is some one thing that is generated or many. 

"The question evidently is, how being, in the sense of 'the substances',
is many; for the things that are generated are numbers and lines and
bodies. Now it is strange to inquire how being in the sense of the
'what' is many, and not how either qualities or quantities are many.
For surely the indefinite dyad or 'the great and the small' is not
a reason why there should be two kinds of white or many colours or
flavours or shapes; for then these also would be numbers and units.
But if they had attacked these other categories, they would have seen
the cause of the plurality in substances also; for the same thing
or something analogous is the cause. This aberration is the reason
also why in seeking the opposite of being and the one, from which
with being and the one the things that are proceed, they posited the
relative term (i.e. the unequal), which is neither the contrary nor
the contradictory of these, and is one kind of being as 'what' and
quality also are. 

"They should have asked this question also, how relative terms are
many and not one. But as it is, they inquire how there are many units
besides the first 1, but do not go on to inquire how there are many
unequals besides the unequal. Yet they use them and speak of great
and small, many and few (from which proceed numbers), long and short
(from which proceeds the line), broad and narrow (from which proceeds
the plane), deep and shallow (from which proceed solids); and they
speak of yet more kinds of relative term. What is the reason, then,
why there is a plurality of these? 

"It is necessary, then, as we say, to presuppose for each thing that
which is it potentially; and the holder of these views further declared
what that is which is potentially a 'this' and a substance but is
not in itself being-viz. that it is the relative (as if he had said
'the qualitative'), which is neither potentially the one or being,
nor the negation of the one nor of being, but one among beings. And
it was much more necessary, as we said, if he was inquiring how beings
are many, not to inquire about those in the same category-how there
are many substances or many qualities-but how beings as a whole are
many; for some are substances, some modifications, some relations.
In the categories other than substance there is yet another problem
involved in the existence of plurality. Since they are not separable
from substances, qualities and quantities are many just because their
substratum becomes and is many; yet there ought to be a matter for
each category; only it cannot be separable from substances. But in
the case of 'thises', it is possible to explain how the 'this' is
many things, unless a thing is to be treated as both a 'this' and
a general character. The difficulty arising from the facts about substances
is rather this, how there are actually many substances and not one.

"But further, if the 'this' and the quantitative are not the same,
we are not told how and why the things that are are many, but how
quantities are many. For all 'number' means a quantity, and so does
the 'unit', unless it means a measure or the quantitatively indivisible.
If, then, the quantitative and the 'what' are different, we are not
told whence or how the 'what' is many; but if any one says they are
the same, he has to face many inconsistencies. 

"One might fix one's attention also on the question, regarding the
numbers, what justifies the belief that they exist. To the believer
in Ideas they provide some sort of cause for existing things, since
each number is an Idea, and the Idea is to other things somehow or
other the cause of their being; for let this supposition be granted
them. But as for him who does not hold this view because he sees the
inherent objections to the Ideas (so that it is not for this reason
that he posits numbers), but who posits mathematical number, why must
we believe his statement that such number exists, and of what use
is such number to other things? Neither does he who says it exists
maintain that it is the cause of anything (he rather says it is a
thing existing by itself), nor is it observed to be the cause of anything;
for the theorems of arithmeticians will all be found true even of
sensible things, as was said before. 

Part 3 "

"As for those, then, who suppose the Ideas to exist and to be numbers,
by their assumption in virtue of the method of setting out each term
apart from its instances-of the unity of each general term they try
at least to explain somehow why number must exist. Since their reasons,
however, are neither conclusive nor in themselves possible, one must
not, for these reasons at least, assert the existence of number. Again,
the Pythagoreans, because they saw many attributes of numbers belonging
te sensible bodies, supposed real things to be numbers-not separable
numbers, however, but numbers of which real things consist. But why?
Because the attributes of numbers are present in a musical scale and
in the heavens and in many other things. Those, however, who say that
mathematical number alone exists cannot according to their hypotheses
say anything of this sort, but it used to be urged that these sensible
things could not be the subject of the sciences. But we maintain that
they are, as we said before. And it is evident that the objects of
mathematics do not exist apart; for if they existed apart their attributes
would not have been present in bodies. Now the Pythagoreans in this
point are open to no objection; but in that they construct natural
bodies out of numbers, things that have lightness and weight out of
things that have not weight or lightness, they seem to speak of another
heaven and other bodies, not of the sensible. But those who make number
separable assume that it both exists and is separable because the
axioms would not be true of sensible things, while the statements
of mathematics are true and 'greet the soul'; and similarly with the
spatial magnitudes of mathematics. It is evident, then, both that
the rival theory will say the contrary of this, and that the difficulty
we raised just now, why if numbers are in no way present in sensible
things their attributes are present in sensible things, has to be
solved by those who hold these views. 

"There are some who, because the point is the limit and extreme of
the line, the line of the plane, and the plane of the solid, think
there must be real things of this sort. We must therefore examine
this argument too, and see whether it is not remarkably weak. For
(i) extremes are not substances, but rather all these things are limits.
For even walking, and movement in general, has a limit, so that on
their theory this will be a 'this' and a substance. But that is absurd.
Not but what (ii) even if they are substances, they will all be the
substances of the sensible things in this world; for it is to these
that the argument applied. Why then should they be capable of existing
apart? 

"Again, if we are not too easily satisfied, we may, regarding all
number and the objects of mathematics, press this difficulty, that
they contribute nothing to one another, the prior to the posterior;
for if number did not exist, none the less spatial magnitudes would
exist for those who maintain the existence of the objects of mathematics
only, and if spatial magnitudes did not exist, soul and sensible bodies
would exist. But the observed facts show that nature is not a series
of episodes, like a bad tragedy. As for the believers in the Ideas,
this difficulty misses them; for they construct spatial magnitudes
out of matter and number, lines out of the number planes doubtless
out of solids out of or they use other numbers, which makes no difference.
But will these magnitudes be Ideas, or what is their manner of existence,
and what do they contribute to things? These contribute nothing, as
the objects of mathematics contribute nothing. But not even is any
theorem true of them, unless we want to change the objects of mathematics
and invent doctrines of our own. But it is not hard to assume any
random hypotheses and spin out a long string of conclusions. These
thinkers, then, are wrong in this way, in wanting to unite the objects
of mathematics with the Ideas. And those who first posited two kinds
of number, that of the Forms and that which is mathematical, neither
have said nor can say how mathematical number is to exist and of what
it is to consist. For they place it between ideal and sensible number.
If (i) it consists of the great and small, it will be the same as
the other-ideal-number (he makes spatial magnitudes out of some other
small and great). And if (ii) he names some other element, he will
be making his elements rather many. And if the principle of each of
the two kinds of number is a 1, unity will be something common to
these, and we must inquire how the one is these many things, while
at the same time number, according to him, cannot be generated except
from one and an indefinite dyad. 

"All this is absurd, and conflicts both with itself and with the probabilities,
and we seem to see in it Simonides 'long rigmarole' for the long rigmarole
comes into play, like those of slaves, when men have nothing sound
to say. And the very elements-the great and the small-seem to cry
out against the violence that is done to them; for they cannot in
any way generate numbers other than those got from 1 by doubling.

"It is strange also to attribute generation to things that are eternal,
or rather this is one of the things that are impossible. There need
be no doubt whether the Pythagoreans attribute generation to them
or not; for they say plainly that when the one had been constructed,
whether out of planes or of surface or of seed or of elements which
they cannot express, immediately the nearest part of the unlimited
began to be constrained and limited by the limit. But since they are
constructing a world and wish to speak the language of natural science,
it is fair to make some examination of their physical theorics, but
to let them off from the present inquiry; for we are investigating
the principles at work in unchangeable things, so that it is numbers
of this kind whose genesis we must study. 

Part 4 

"These thinkers say there is no generation of the odd number, which
evidently implies that there is generation of the even; and some present
the even as produced first from unequals-the great and the small-when
these are equalized. The inequality, then, must belong to them before
they are equalized. If they had always been equalized, they would
not have been unequal before; for there is nothing before that which
is always. Therefore evidently they are not giving their account of
the generation of numbers merely to assist contemplation of their
nature. 

"A difficulty, and a reproach to any one who finds it no difficulty,
are contained in the question how the elements and the principles
are related to the good and the beautiful; the difficulty is this,
whether any of the elements is such a thing as we mean by the good
itself and the best, or this is not so, but these are later in origin
than the elements. The theologians seem to agree with some thinkers
of the present day, who answer the question in the negative, and say
that both the good and the beautiful appear in the nature of things
only when that nature has made some progress. (This they do to avoid
a real objection which confronts those who say, as some do, that the
one is a first principle. The objection arises not from their ascribing
goodness to the first principle as an attribute, but from their making
the one a principle-and a principle in the sense of an element-and
generating number from the one.) The old poets agree with this inasmuch
as they say that not those who are first in time, e.g. Night and Heaven
or Chaos or Ocean, reign and rule, but Zeus. These poets, however,
are led to speak thus only because they think of the rulers of the
world as changing; for those of them who combine the two characters
in that they do not use mythical language throughout, e.g. Pherecydes
and some others, make the original generating agent the Best, and
so do the Magi, and some of the later sages also, e.g. both Empedocles
and Anaxagoras, of whom one made love an element, and the other made
reason a principle. Of those who maintain the existence of the unchangeable
substances some say the One itself is the good itself; but they thought
its substance lay mainly in its unity. 

"This, then, is the problem,-which of the two ways of speaking is
right. It would be strange if to that which is primary and eternal
and most self-sufficient this very quality--self-sufficiency and self-maintenance--belongs
primarily in some other way than as a good. But indeed it can be for
no other reason indestructible or self-sufficient than because its
nature is good. Therefore to say that the first principle is good
is probably correct; but that this principle should be the One or,
if not that, at least an element, and an element of numbers, is impossible.
Powerful objections arise, to avoid which some have given up the theory
(viz. those who agree that the One is a first principle and element,
but only of mathematical number). For on this view all the units become
identical with species of good, and there is a great profusion of
goods. Again, if the Forms are numbers, all the Forms are identical
with species of good. But let a man assume Ideas of anything he pleases.
If these are Ideas only of goods, the Ideas will not be substances;
but if the Ideas are also Ideas of substances, all animals and plants
and all individuals that share in Ideas will be good. 

"These absurdities follow, and it also follows that the contrary element,
whether it is plurality or the unequal, i.e. the great and small,
is the bad-itself. (Hence one thinker avoided attaching the good to
the One, because it would necessarily follow, since generation is
from contraries, that badness is the fundamental nature of plurality;
while others say inequality is the nature of the bad.) It follows,
then, that all things partake of the bad except one--the One itself,
and that numbers partake of it in a more undiluted form than spatial
magnitudes, and that the bad is the space in which the good is realized,
and that it partakes in and desires that which tends to destroy it;
for contrary tends to destroy contrary. And if, as we were saying,
the matter is that which is potentially each thing, e.g. that of actual
fire is that which is potentially fire, the bad will be just the potentially
good. 

"All these objections, then, follow, partly because they make every
principle an element, partly because they make contraries principles,
partly because they make the One a principle, partly because they
treat the numbers as the first substances, and as capable of existing
apart, and as Forms. 

Part 5 "

"If, then, it is equally impossible not to put the good among the
first principles and to put it among them in this way, evidently the
principles are not being correctly described, nor are the first substances.
Nor does any one conceive the matter correctly if he compares the
principles of the universe to that of animals and plants, on the ground
that the more complete always comes from the indefinite and incomplete-which
is what leads this thinker to say that this is also true of the first
principles of reality, so that the One itself is not even an existing
thing. This is incorrect, for even in this world of animals and plants
the principles from which these come are complete; for it is a man
that produces a man, and the seed is not first. 

"It is out of place, also, to generate place simultaneously with the
mathematical solids (for place is peculiar to the individual things,
and hence they are separate in place; but mathematical objects are
nowhere), and to say that they must be somewhere, but not say what
kind of thing their place is. 

"Those who say that existing things come from elements and that the
first of existing things are the numbers, should have first distinguished
the senses in which one thing comes from another, and then said in
which sense number comes from its first principles. 

"By intermixture? But (1) not everything is capable of intermixture,
and (2) that which is produced by it is different from its elements,
and on this view the one will not remain separate or a distinct entity;
but they want it to be so. 

"By juxtaposition, like a syllable? But then (1) the elements must
have position; and (2) he who thinks of number will be able to think
of the unity and the plurality apart; number then will be this-a unit
and plurality, or the one and the unequal. 

"Again, coming from certain things means in one sense that these are
still to be found in the product, and in another that they are not;
which sense does number come from these elements? Only things that
are generated can come from elements which are present in them. Does
number come, then, from its elements as from seed? But nothing can
be excreted from that which is indivisible. Does it come from its
contrary, its contrary not persisting? But all things that come in
this way come also from something else which does persist. Since,
then, one thinker places the 1 as contrary to plurality, and another
places it as contrary to the unequal, treating the 1 as equal, number
must be being treated as coming from contraries. There is, then, something
else that persists, from which and from one contrary the compound
is or has come to be. Again, why in the world do the other things
that come from contraries, or that have contraries, perish (even when
all of the contrary is used to produce them), while number does not?
Nothing is said about this. Yet whether present or not present in
the compound the contrary destroys it, e.g. 'strife' destroys the
'mixture' (yet it should not; for it is not to that that is contrary).

"Once more, it has not been determined at all in which way numbers
are the causes of substances and of being-whether (1) as boundaries
(as points are of spatial magnitudes). This is how Eurytus decided
what was the number of what (e.g. one of man and another of horse),
viz. by imitating the figures of living things with pebbles, as some
people bring numbers into the forms of triangle and square. Or (2)
is it because harmony is a ratio of numbers, and so is man and everything
else? But how are the attributes-white and sweet and hot-numbers?
Evidently it is not the numbers that are the essence or the causes
of the form; for the ratio is the essence, while the number the causes
of the form; for the ratio is the essence, while the number is the
matter. E.g. the essence of flesh or bone is number only in this way,
'three parts of fire and two of earth'. And a number, whatever number
it is, is always a number of certain things, either of parts of fire
or earth or of units; but the essence is that there is so much of
one thing to so much of another in the mixture; and this is no longer
a number but a ratio of mixture of numbers, whether these are corporeal
or of any other kind. 

"Number, then, whether it be number in general or the number which
consists of abstract units, is neither the cause as agent, nor the
matter, nor the ratio and form of things. Nor, of course, is it the
final cause. 

Part 6 "

"One might also raise the question what the good is that things get
from numbers because their composition is expressible by a number,
either by one which is easily calculable or by an odd number. For
in fact honey-water is no more wholesome if it is mixed in the proportion
of three times three, but it would do more good if it were in no particular
ratio but well diluted than if it were numerically expressible but
strong. Again, the ratios of mixtures are expressed by the adding
of numbers, not by mere numbers; e.g. it is 'three parts to two',
not 'three times two'. For in any multiplication the genus of the
things multiplied must be the same; therefore the product 1X2X3 must
be measurable by 1, and 4X5X6 by 4 and therefore all products into
which the same factor enters must be measurable by that factor. The
number of fire, then, cannot be 2X5X3X6 and at the same time that
of water 2X3. 

"If all things must share in number, it must follow that many things
are the same, and the same number must belong to one thing and to
another. Is number the cause, then, and does the thing exist because
of its number, or is this not certain? E.g. the motions of the sun
have a number, and again those of the moon,-yes, and the life and
prime of each animal. Why, then, should not some of these numbers
be squares, some cubes, and some equal, others double? There is no
reason why they should not, and indeed they must move within these
limits, since all things were assumed to share in number. And it was
assumed that things that differed might fall under the same number.
Therefore if the same number had belonged to certain things, these
would have been the same as one another, since they would have had
the same form of number; e.g. sun and moon would have been the same.
But why need these numbers be causes? There are seven vowels, the
scale consists of seven strings, the Pleiades are seven, at seven
animals lose their teeth (at least some do, though some do not), and
the champions who fought against Thebes were seven. Is it then because
the number is the kind of number it is, that the champions were seven
or the Pleiad consists of seven stars? Surely the champions were seven
because there were seven gates or for some other reason, and the Pleiad
we count as seven, as we count the Bear as twelve, while other peoples
count more stars in both. Nay they even say that X, Ps and Z are concords
and that because there are three concords, the double consonants also
are three. They quite neglect the fact that there might be a thousand
such letters; for one symbol might be assigned to GP. But if they
say that each of these three is equal to two of the other letters,
and no other is so, and if the cause is that there are three parts
of the mouth and one letter is in each applied to sigma, it is for
this reason that there are only three, not because the concords are
three; since as a matter of fact the concords are more than three,
but of double consonants there cannot be more. 

"These people are like the old-fashioned Homeric scholars, who see
small resemblances but neglect great ones. Some say that there are
many such cases, e.g. that the middle strings are represented by nine
and eight, and that the epic verse has seventeen syllables, which
is equal in number to the two strings, and that the scansion is, in
the right half of the line nine syllables, and in the left eight.
And they say that the distance in the letters from alpha to omega
is equal to that from the lowest note of the flute to the highest,
and that the number of this note is equal to that of the whole choir
of heaven. It may be suspected that no one could find difficulty either
in stating such analogies or in finding them in eternal things, since
they can be found even in perishable things. 

"But the lauded characteristics of numbers, and the contraries of
these, and generally the mathematical relations, as some describe
them, making them causes of nature, seem, when we inspect them in
this way, to vanish; for none of them is a cause in any of the senses
that have been distinguished in reference to the first principles.
In a sense, however, they make it plain that goodness belongs to numbers,
and that the odd, the straight, the square, the potencies of certain
numbers, are in the column of the beautiful. For the seasons and a
particular kind of number go together; and the other agreements that
they collect from the theorems of mathematics all have this meaning.
Hence they are like coincidences. For they are accidents, but the
things that agree are all appropriate to one another, and one by analogy.
For in each category of being an analogous term is found-as the straight
is in length, so is the level in surface, perhaps the odd in number,
and the white in colour. 

"Again, it is not the ideal numbers that are the causes of musical
phenomena and the like (for equal ideal numbers differ from one another
in form; for even the units do); so that we need not assume Ideas
for this reason at least. 

"These, then, are the results of the theory, and yet more might be
brought together. The fact that our opponnts have much trouble with
the generation of numbers and can in no way make a system of them,
seems to indicate that the objects of mathematics are not separable
from sensible things, as some say, and that they are not the first
principles. "

THE END


% chapter metaphysics (end)
% \chapter{On the Soul} % (fold)
\label{cha:soul}


On the Soul
By Aristotle


Translated by J. A. Smith

----------------------------------------------------------------------

BOOK I

Part 1 

Holding as we do that, while knowledge of any kind is a thing to
be honoured and prized, one kind of it may, either by reason of its
greater exactness or of a higher dignity and greater wonderfulness
in its objects, be more honourable and precious than another, on both
accounts we should naturally be led to place in the front rank the
study of the soul. The knowledge of the soul admittedly contributes
greatly to the advance of truth in general, and, above all, to our
understanding of Nature, for the soul is in some sense the principle
of animal life. Our aim is to grasp and understand, first its essential
nature, and secondly its properties; of these some are taught to be
affections proper to the soul itself, while others are considered
to attach to the animal owing to the presence within it of soul.

To attain any assured knowledge about the soul is one of the most
difficult things in the world. As the form of question which here
presents itself, viz. the question 'What is it?', recurs in other
fields, it might be supposed that there was some single method of
inquiry applicable to all objects whose essential nature (as we are
endeavouring to ascertain there is for derived properties the single
method of demonstration); in that case what we should have to seek
for would be this unique method. But if there is no such single and
general method for solving the question of essence, our task becomes
still more difficult; in the case of each different subject we shall
have to determine the appropriate process of investigation. If to
this there be a clear answer, e.g. that the process is demonstration
or division, or some known method, difficulties and hesitations still
beset us-with what facts shall we begin the inquiry? For the facts
which form the starting-points in different subjects must be different,
as e.g. in the case of numbers and surfaces. 

First, no doubt, it is necessary to determine in which of the summa
genera soul lies, what it is; is it 'a this-somewhat, 'a substance,
or is it a quale or a quantum, or some other of the remaining kinds
of predicates which we have distinguished? Further, does soul belong
to the class of potential existents, or is it not rather an actuality?
Our answer to this question is of the greatest importance.

We must consider also whether soul is divisible or is without parts,
and whether it is everywhere homogeneous or not; and if not homogeneous,
whether its various forms are different specifically or generically:
up to the present time those who have discussed and investigated soul
seem to have confined themselves to the human soul. We must be careful
not to ignore the question whether soul can be defined in a single
unambiguous formula, as is the case with animal, or whether we must
not give a separate formula for each of it, as we do for horse, dog,
man, god (in the latter case the 'universal' animal-and so too every
other 'common predicate'-being treated either as nothing at all or
as a later product). Further, if what exists is not a plurality of
souls, but a plurality of parts of one soul, which ought we to investigate
first, the whole soul or its parts? (It is also a difficult problem
to decide which of these parts are in nature distinct from one another.)
Again, which ought we to investigate first, these parts or their functions,
mind or thinking, the faculty or the act of sensation, and so on?
If the investigation of the functions precedes that of the parts,
the further question suggests itself: ought we not before either to
consider the correlative objects, e.g. of sense or thought? It seems
not only useful for the discovery of the causes of the derived properties
of substances to be acquainted with the essential nature of those
substances (as in mathematics it is useful for the understanding of
the property of the equality of the interior angles of a triangle
to two right angles to know the essential nature of the straight and
the curved or of the line and the plane) but also conversely, for
the knowledge of the essential nature of a substance is largely promoted
by an acquaintance with its properties: for, when we are able to give
an account conformable to experience of all or most of the properties
of a substance, we shall be in the most favourable position to say
something worth saying about the essential nature of that subject;
in all demonstration a definition of the essence is required as a
starting-point, so that definitions which do not enable us to discover
the derived properties, or which fail to facilitate even a conjecture
about them, must obviously, one and all, be dialectical and futile.

A further problem presented by the affections of soul is this: are
they all affections of the complex of body and soul, or is there any
one among them peculiar to the soul by itself? To determine this is
indispensable but difficult. If we consider the majority of them,
there seems to be no case in which the soul can act or be acted upon
without involving the body; e.g. anger, courage, appetite, and sensation
generally. Thinking seems the most probable exception; but if this
too proves to be a form of imagination or to be impossible without
imagination, it too requires a body as a condition of its existence.
If there is any way of acting or being acted upon proper to soul,
soul will be capable of separate existence; if there is none, its
separate existence is impossible. In the latter case, it will be like
what is straight, which has many properties arising from the straightness
in it, e.g. that of touching a bronze sphere at a point, though straightness
divorced from the other constituents of the straight thing cannot
touch it in this way; it cannot be so divorced at all, since it is
always found in a body. It therefore seems that all the affections
of soul involve a body-passion, gentleness, fear, pity, courage, joy,
loving, and hating; in all these there is a concurrent affection of
the body. In support of this we may point to the fact that, while
sometimes on the occasion of violent and striking occurrences there
is no excitement or fear felt, on others faint and feeble stimulations
produce these emotions, viz. when the body is already in a state of
tension resembling its condition when we are angry. Here is a still
clearer case: in the absence of any external cause of terror we find
ourselves experiencing the feelings of a man in terror. From all this
it is obvious that the affections of soul are enmattered formulable
essences. 

Consequently their definitions ought to correspond, e.g. anger should
be defined as a certain mode of movement of such and such a body (or
part or faculty of a body) by this or that cause and for this or that
end. That is precisely why the study of the soul must fall within
the science of Nature, at least so far as in its affections it manifests
this double character. Hence a physicist would define an affection
of soul differently from a dialectician; the latter would define e.g.
anger as the appetite for returning pain for pain, or something like
that, while the former would define it as a boiling of the blood or
warm substance surround the heart. The latter assigns the material
conditions, the former the form or formulable essence; for what he
states is the formulable essence of the fact, though for its actual
existence there must be embodiment of it in a material such as is
described by the other. Thus the essence of a house is assigned in
such a formula as 'a shelter against destruction by wind, rain, and
heat'; the physicist would describe it as 'stones, bricks, and timbers';
but there is a third possible description which would say that it
was that form in that material with that purpose or end. Which, then,
among these is entitled to be regarded as the genuine physicist? The
one who confines himself to the material, or the one who restricts
himself to the formulable essence alone? Is it not rather the one
who combines both in a single formula? If this is so, how are we to
characterize the other two? Must we not say that there is no type
of thinker who concerns himself with those qualities or attributes
of the material which are in fact inseparable from the material, and
without attempting even in thought to separate them? The physicist
is he who concerns himself with all the properties active and passive
of bodies or materials thus or thus defined; attributes not considered
as being of this character he leaves to others, in certain cases it
may be to a specialist, e.g. a carpenter or a physician, in others
(a) where they are inseparable in fact, but are separable from any
particular kind of body by an effort of abstraction, to the mathematician,
(b) where they are separate both in fact and in thought from body
altogether, to the First Philosopher or metaphysician. But we must
return from this digression, and repeat that the affections of soul
are inseparable from the material substratum of animal life, to which
we have seen that such affections, e.g. passion and fear, attach,
and have not the same mode of being as a line or a plane.

Part 2

For our study of soul it is necessary, while formulating the problems
of which in our further advance we are to find the solutions, to call
into council the views of those of our predecessors who have declared
any opinion on this subject, in order that we may profit by whatever
is sound in their suggestions and avoid their errors. 

The starting-point of our inquiry is an exposition of those characteristics
which have chiefly been held to belong to soul in its very nature.
Two characteristic marks have above all others been recognized as
distinguishing that which has soul in it from that which has not-movement
and sensation. It may be said that these two are what our predecessors
have fixed upon as characteristic of soul. 

Some say that what originates movement is both pre-eminently and primarily
soul; believing that what is not itself moved cannot originate movement
in another, they arrived at the view that soul belongs to the class
of things in movement. This is what led Democritus to say that soul
is a sort of fire or hot substance; his 'forms' or atoms are infinite
in number; those which are spherical he calls fire and soul, and compares
them to the motes in the air which we see in shafts of light coming
through windows; the mixture of seeds of all sorts he calls the elements
of the whole of Nature (Leucippus gives a similar account); the spherical
atoms are identified with soul because atoms of that shape are most
adapted to permeate everywhere, and to set all the others moving by
being themselves in movement. This implies the view that soul is identical
with what produces movement in animals. That is why, further, they
regard respiration as the characteristic mark of life; as the environment
compresses the bodies of animals, and tends to extrude those atoms
which impart movement to them, because they themselves are never at
rest, there must be a reinforcement of these by similar atoms coming
in from without in the act of respiration; for they prevent the extrusion
of those which are already within by counteracting the compressing
and consolidating force of the environment; and animals continue to
live only so long as they are able to maintain this resistance.

The doctrine of the Pythagoreans seems to rest upon the same ideas;
some of them declared the motes in air, others what moved them, to
be soul. These motes were referred to because they are seen always
in movement, even in a complete calm. 

The same tendency is shown by those who define soul as that which
moves itself; all seem to hold the view that movement is what is closest
to the nature of soul, and that while all else is moved by soul, it
alone moves itself. This belief arises from their never seeing anything
originating movement which is not first itself moved. 

Similarly also Anaxagoras (and whoever agrees with him in saying that
mind set the whole in movement) declares the moving cause of things
to be soul. His position must, however, be distinguished from that
of Democritus. Democritus roundly identifies soul and mind, for he
identifies what appears with what is true-that is why he commends
Homer for the phrase 'Hector lay with thought distraught'; he does
not employ mind as a special faculty dealing with truth, but identifies
soul and mind. What Anaxagoras says about them is more obscure; in
many places he tells us that the cause of beauty and order is mind,
elsewhere that it is soul; it is found, he says, in all animals, great
and small, high and low, but mind (in the sense of intelligence) appears
not to belong alike to all animals, and indeed not even to all human
beings. 

All those, then, who had special regard to the fact that what has
soul in it is moved, adopted the view that soul is to be identified
with what is eminently originative of movement. All, on the other
hand, who looked to the fact that what has soul in it knows or perceives
what is, identify soul with the principle or principles of Nature,
according as they admit several such principles or one only. Thus
Empedocles declares that it is formed out of all his elements, each
of them also being soul; his words are: 

For 'tis by Earth we see Earth, by Water Water, 
By Ether Ether divine, by Fire destructive Fire, 
By Love Love, and Hate by cruel Hate. 

In the same way Plato in the Timaeus fashions soul out of his elements;
for like, he holds, is known by like, and things are formed out of
the principles or elements, so that soul must be so too. Similarly
also in his lectures 'On Philosophy' it was set forth that the Animal-itself
is compounded of the Idea itself of the One together with the primary
length, breadth, and depth, everything else, the objects of its perception,
being similarly constituted. Again he puts his view in yet other terms:
Mind is the monad, science or knowledge the dyad (because it goes
undeviatingly from one point to another), opinion the number of the
plane, sensation the number of the solid; the numbers are by him expressly
identified with the Forms themselves or principles, and are formed
out of the elements; now things are apprehended either by mind or
science or opinion or sensation, and these same numbers are the Forms
of things. 

Some thinkers, accepting both premisses, viz. that the soul is both
originative of movement and cognitive, have compounded it of both
and declared the soul to be a self-moving number. 

As to the nature and number of the first principles opinions differ.
The difference is greatest between those who regard them as corporeal
and those who regard them as incorporeal, and from both dissent those
who make a blend and draw their principles from both sources. The
number of principles is also in dispute; some admit one only, others
assert several. There is a consequent diversity in their several accounts
of soul; they assume, naturally enough, that what is in its own nature
originative of movement must be among what is primordial. That has
led some to regard it as fire, for fire is the subtlest of the elements
and nearest to incorporeality; further, in the most primary sense,
fire both is moved and originates movement in all the others.

Democritus has expressed himself more ingeniously than the rest on
the grounds for ascribing each of these two characters to soul; soul
and mind are, he says, one and the same thing, and this thing must
be one of the primary and indivisible bodies, and its power of originating
movement must be due to its fineness of grain and the shape of its
atoms; he says that of all the shapes the spherical is the most mobile,
and that this is the shape of the particles of fire and mind.

Anaxagoras, as we said above, seems to distinguish between soul and
mind, but in practice he treats them as a single substance, except
that it is mind that he specially posits as the principle of all things;
at any rate what he says is that mind alone of all that is simple,
unmixed, and pure. He assigns both characteristics, knowing and origination
of movement, to the same principle, when he says that it was mind
that set the whole in movement. 

Thales, too, to judge from what is recorded about him, seems to have
held soul to be a motive force, since he said that the magnet has
a soul in it because it moves the iron. 

Diogenes (and others) held the soul to be air because he believed
air to be finest in grain and a first principle; therein lay the grounds
of the soul's powers of knowing and originating movement. As the primordial
principle from which all other things are derived, it is cognitive;
as finest in grain, it has the power to originate movement.

Heraclitus too says that the first principle-the 'warm exhalation'
of which, according to him, everything else is composed-is soul; further,
that this exhalation is most incorporeal and in ceaseless flux; that
what is in movement requires that what knows it should be in movement;
and that all that is has its being essentially in movement (herein
agreeing with the majority). 

Alcmaeon also seems to have held a similar view about soul; he says
that it is immortal because it resembles 'the immortals,' and that
this immortality belongs to it in virtue of its ceaseless movement;
for all the 'things divine,' moon, sun, the planets, and the whole
heavens, are in perpetual movement. 

of More superficial writers, some, e.g. Hippo, have pronounced it
to be water; they seem to have argued from the fact that the seed
of all animals is fluid, for Hippo tries to refute those who say that
the soul is blood, on the ground that the seed, which is the primordial
soul, is not blood. 

Another group (Critias, for example) did hold it to be blood; they
take perception to be the most characteristic attribute of soul, and
hold that perceptiveness is due to the nature of blood. 

Each of the elements has thus found its partisan, except earth-earth
has found no supporter unless we count as such those who have declared
soul to be, or to be compounded of, all the elements. All, then, it
may be said, characterize the soul by three marks, Movement, Sensation,
Incorporeality, and each of these is traced back to the first principles.
That is why (with one exception) all those who define the soul by
its power of knowing make it either an element or constructed out
of the elements. The language they all use is similar; like, they
say, is known by like; as the soul knows everything, they construct
it out of all the principles. Hence all those who admit but one cause
or element, make the soul also one (e.g. fire or air), while those
who admit a multiplicity of principles make the soul also multiple.
The exception is Anaxagoras; he alone says that mind is impassible
and has nothing in common with anything else. But, if this is so,
how or in virtue of what cause can it know? That Anaxagoras has not
explained, nor can any answer be inferred from his words. All who
acknowledge pairs of opposites among their principles, construct the
soul also out of these contraries, while those who admit as principles
only one contrary of each pair, e.g. either hot or cold, likewise
make the soul some one of these. That is why, also, they allow themselves
to be guided by the names; those who identify soul with the hot argue
that sen (to live) is derived from sein (to boil), while those who
identify it with the cold say that soul (psuche) is so called from
the process of respiration and (katapsuxis). Such are the traditional
opinions concerning soul, together with the grounds on which they
are maintained. 

Part 3

We must begin our examination with movement; for doubtless, not only
is it false that the essence of soul is correctly described by those
who say that it is what moves (or is capable of moving) itself, but
it is an impossibility that movement should be even an attribute of
it. 

We have already pointed out that there is no necessity that what originates
movement should itself be moved. There are two senses in which anything
may be moved-either (a) indirectly, owing to something other than
itself, or (b) directly, owing to itself. Things are 'indirectly moved'
which are moved as being contained in something which is moved, e.g.
sailors in a ship, for they are moved in a different sense from that
in which the ship is moved; the ship is 'directly moved', they are
'indirectly moved', because they are in a moving vessel. This is clear
if we consider their limbs; the movement proper to the legs (and so
to man) is walking, and in this case the sailors tare not walking.
Recognizing the double sense of 'being moved', what we have to consider
now is whether the soul is 'directly moved' and participates in such
direct movement. 

There are four species of movement-locomotion, alteration, diminution,
growth; consequently if the soul is moved, it must be moved with one
or several or all of these species of movement. Now if its movement
is not incidental, there must be a movement natural to it, and, if
so, as all the species enumerated involve place, place must be natural
to it. But if the essence of soul be to move itself, its being moved
cannot be incidental to-as it is to what is white or three cubits
long; they too can be moved, but only incidentally-what is moved is
that of which 'white' and 'three cubits long' are the attributes,
the body in which they inhere; hence they have no place: but if the
soul naturally partakes in movement, it follows that it must have
a place. 

Further, if there be a movement natural to the soul, there must be
a counter-movement unnatural to it, and conversely. The same applies
to rest as well as to movement; for the terminus ad quem of a thing's
natural movement is the place of its natural rest, and similarly the
terminus ad quem of its enforced movement is the place of its enforced
rest. But what meaning can be attached to enforced movements or rests
of the soul, it is difficult even to imagine. 

Further, if the natural movement of the soul be upward, the soul must
be fire; if downward, it must be earth; for upward and downward movements
are the definitory characteristics of these bodies. The same reasoning
applies to the intermediate movements, termini, and bodies. Further,
since the soul is observed to originate movement in the body, it is
reasonable to suppose that it transmits to the body the movements
by which it itself is moved, and so, reversing the order, we may infer
from the movements of the body back to similar movements of the soul.
Now the body is moved from place to place with movements of locomotion.
Hence it would follow that the soul too must in accordance with the
body change either its place as a whole or the relative places of
its parts. This carries with it the possibility that the soul might
even quit its body and re-enter it, and with this would be involved
the possibility of a resurrection of animals from the dead. But, it
may be contended, the soul can be moved indirectly by something else;
for an animal can be pushed out of its course. Yes, but that to whose
essence belongs the power of being moved by itself, cannot be moved
by something else except incidentally, just as what is good by or
in itself cannot owe its goodness to something external to it or to
some end to which it is a means. 

If the soul is moved, the most probable view is that what moves it
is sensible things. 

We must note also that, if the soul moves itself, it must be the mover
itself that is moved, so that it follows that if movement is in every
case a displacement of that which is in movement, in that respect
in which it is said to be moved, the movement of the soul must be
a departure from its essential nature, at least if its self-movement
is essential to it, not incidental. 

Some go so far as to hold that the movements which the soul imparts
to the body in which it is are the same in kind as those with which
it itself is moved. An example of this is Democritus, who uses language
like that of the comic dramatist Philippus, who accounts for the movements
that Daedalus imparted to his wooden Aphrodite by saying that he poured
quicksilver into it; similarly Democritus says that the spherical
atoms which according to him constitute soul, owing to their own ceaseless
movements draw the whole body after them and so produce its movements.
We must urge the question whether it is these very same atoms which
produce rest also-how they could do so, it is difficult and even impossible
to say. And, in general, we may object that it is not in this way
that the soul appears to originate movement in animals-it is through
intention or process of thinking. 

It is in the same fashion that the Timaeus also tries to give a physical
account of how the soul moves its body; the soul, it is there said,
is in movement, and so owing to their mutual implication moves the
body also. After compounding the soul-substance out of the elements
and dividing it in accordance with the harmonic numbers, in order
that it may possess a connate sensibility for 'harmony' and that the
whole may move in movements well attuned, the Demiurge bent the straight
line into a circle; this single circle he divided into two circles
united at two common points; one of these he subdivided into seven
circles. All this implies that the movements of the soul are identified
with the local movements of the heavens. 

Now, in the first place, it is a mistake to say that the soul is a
spatial magnitude. It is evident that Plato means the soul of the
whole to be like the sort of soul which is called mind not like the
sensitive or the desiderative soul, for the movements of neither of
these are circular. Now mind is one and continuous in the sense in
which the process of thinking is so, and thinking is identical with
the thoughts which are its parts; these have a serial unity like that
of number, not a unity like that of a spatial magnitude. Hence mind
cannot have that kind of unity either; mind is either without parts
or is continuous in some other way than that which characterizes a
spatial magnitude. How, indeed, if it were a spatial magnitude, could
mind possibly think? Will it think with any one indifferently of its
parts? In this case, the 'part' must be understood either in the sense
of a spatial magnitude or in the sense of a point (if a point can
be called a part of a spatial magnitude). If we accept the latter
alternative, the points being infinite in number, obviously the mind
can never exhaustively traverse them; if the former, the mind must
think the same thing over and over again, indeed an infinite number
of times (whereas it is manifestly possible to think a thing once
only). If contact of any part whatsoever of itself with the object
is all that is required, why need mind move in a circle, or indeed
possess magnitude at all? On the other hand, if contact with the whole
circle is necessary, what meaning can be given to the contact of the
parts? Further, how could what has no parts think what has parts,
or what has parts think what has none? We must identify the circle
referred to with mind; for it is mind whose movement is thinking,
and it is the circle whose movement is revolution, so that if thinking
is a movement of revolution, the circle which has this characteristic
movement must be mind. 

If the circular movement is eternal, there must be something which
mind is always thinking-what can this be? For all practical processes
of thinking have limits-they all go on for the sake of something outside
the process, and all theoretical processes come to a close in the
same way as the phrases in speech which express processes and results
of thinking. Every such linguistic phrase is either definitory or
demonstrative. Demonstration has both a starting-point and may be
said to end in a conclusion or inferred result; even if the process
never reaches final completion, at any rate it never returns upon
itself again to its starting-point, it goes on assuming a fresh middle
term or a fresh extreme, and moves straight forward, but circular
movement returns to its starting-point. Definitions, too, are closed
groups of terms. 

Further, if the same revolution is repeated, mind must repeatedly
think the same object. 

Further, thinking has more resemblance to a coming to rest or arrest
than to a movement; the same may be said of inferring. 

It might also be urged that what is difficult and enforced is incompatible
with blessedness; if the movement of the soul is not of its essence,
movement of the soul must be contrary to its nature. It must also
be painful for the soul to be inextricably bound up with the body;
nay more, if, as is frequently said and widely accepted, it is better
for mind not to be embodied, the union must be for it undesirable.

Further, the cause of the revolution of the heavens is left obscure.
It is not the essence of soul which is the cause of this circular
movement-that movement is only incidental to soul-nor is, a fortiori,
the body its cause. Again, it is not even asserted that it is better
that soul should be so moved; and yet the reason for which God caused
the soul to move in a circle can only have been that movement was
better for it than rest, and movement of this kind better than any
other. But since this sort of consideration is more appropriate to
another field of speculation, let us dismiss it for the present.

The view we have just been examining, in company with most theories
about the soul, involves the following absurdity: they all join the
soul to a body, or place it in a body, without adding any specification
of the reason of their union, or of the bodily conditions required
for it. Yet such explanation can scarcely be omitted; for some community
of nature is presupposed by the fact that the one acts and the other
is acted upon, the one moves and the other is moved; interaction always
implies a special nature in the two interagents. All, however, that
these thinkers do is to describe the specific characteristics of the
soul; they do not try to determine anything about the body which is
to contain it, as if it were possible, as in the Pythagorean myths,
that any soul could be clothed upon with any body-an absurd view,
for each body seems to have a form and shape of its own. It is as
absurd as to say that the art of carpentry could embody itself in
flutes; each art must use its tools, each soul its body.

Part 4

There is yet another theory about soul, which has commended itself
to many as no less probable than any of those we have hitherto mentioned,
and has rendered public account of itself in the court of popular
discussion. Its supporters say that the soul is a kind of harmony,
for (a) harmony is a blend or composition of contraries, and (b) the
body is compounded out of contraries. Harmony, however, is a certain
proportion or composition of the constituents blended, and soul can
be neither the one nor the other of these. Further, the power of originating
movement cannot belong to a harmony, while almost all concur in regarding
this as a principal attribute of soul. It is more appropriate to call
health (or generally one of the good states of the body) a harmony
than to predicate it of the soul. The absurdity becomes most apparent
when we try to attribute the active and passive affections of the
soul to a harmony; the necessary readjustment of their conceptions
is difficult. Further, in using the word 'harmony' we have one or
other of two cases in our mind; the most proper sense is in relation
to spatial magnitudes which have motion and position, where harmony
means the disposition and cohesion of their parts in such a manner
as to prevent the introduction into the whole of anything homogeneous
with it, and the secondary sense, derived from the former, is that
in which it means the ratio between the constituents so blended; in
neither of these senses is it plausible to predicate it of soul. That
soul is a harmony in the sense of the mode of composition of the parts
of the body is a view easily refutable; for there are many composite
parts and those variously compounded; of what bodily part is mind
or the sensitive or the appetitive faculty the mode of composition?
And what is the mode of composition which constitutes each of them?
It is equally absurd to identify the soul with the ratio of the mixture;
for the mixture which makes flesh has a different ratio between the
elements from that which makes bone. The consequence of this view
will therefore be that distributed throughout the whole body there
will be many souls, since every one of the bodily parts is a different
mixture of the elements, and the ratio of mixture is in each case
a harmony, i.e. a soul. 

From Empedocles at any rate we might demand an answer to the following
question for he says that each of the parts of the body is what it
is in virtue of a ratio between the elements: is the soul identical
with this ratio, or is it not rather something over and above this
which is formed in the parts? Is love the cause of any and every mixture,
or only of those that are in the right ratio? Is love this ratio itself,
or is love something over and above this? Such are the problems raised
by this account. But, on the other hand, if the soul is different
from the mixture, why does it disappear at one and the same moment
with that relation between the elements which constitutes flesh or
the other parts of the animal body? Further, if the soul is not identical
with the ratio of mixture, and it is consequently not the case that
each of the parts has a soul, what is that which perishes when the
soul quits the body? 

That the soul cannot either be a harmony, or be moved in a circle,
is clear from what we have said. Yet that it can be moved incidentally
is, as we said above, possible, and even that in a sense it can move
itself, i.e. in the sense that the vehicle in which it is can be moved,
and moved by it; in no other sense can the soul be moved in space.

More legitimate doubts might remain as to its movement in view of
the following facts. We speak of the soul as being pained or pleased,
being bold or fearful, being angry, perceiving, thinking. All these
are regarded as modes of movement, and hence it might be inferred
that the soul is moved. This, however, does not necessarily follow.
We may admit to the full that being pained or pleased, or thinking,
are movements (each of them a 'being moved'), and that the movement
is originated by the soul. For example we may regard anger or fear
as such and such movements of the heart, and thinking as such and
such another movement of that organ, or of some other; these modifications
may arise either from changes of place in certain parts or from qualitative
alterations (the special nature of the parts and the special modes
of their changes being for our present purpose irrelevant). Yet to
say that it is the soul which is angry is as inexact as it would be
to say that it is the soul that weaves webs or builds houses. It is
doubtless better to avoid saying that the soul pities or learns or
thinks and rather to say that it is the man who does this with his
soul. What we mean is not that the movement is in the soul, but that
sometimes it terminates in the soul and sometimes starts from it,
sensation e.g. coming from without inwards, and reminiscence starting
from the soul and terminating with the movements, actual or residual,
in the sense organs. 

The case of mind is different; it seems to be an independent substance
implanted within the soul and to be incapable of being destroyed.
If it could be destroyed at all, it would be under the blunting influence
of old age. What really happens in respect of mind in old age is,
however, exactly parallel to what happens in the case of the sense
organs; if the old man could recover the proper kind of eye, he would
see just as well as the young man. The incapacity of old age is due
to an affection not of the soul but of its vehicle, as occurs in drunkenness
or disease. Thus it is that in old age the activity of mind or intellectual
apprehension declines only through the decay of some other inward
part; mind itself is impassible. Thinking, loving, and hating are
affections not of mind, but of that which has mind, so far as it has
it. That is why, when this vehicle decays, memory and love cease;
they were activities not of mind, but of the composite which has perished;
mind is, no doubt, something more divine and impassible. That the
soul cannot be moved is therefore clear from what we have said, and
if it cannot be moved at all, manifestly it cannot be moved by itself.

Of all the opinions we have enumerated, by far the most unreasonable
is that which declares the soul to be a self-moving number; it involves
in the first place all the impossibilities which follow from regarding
the soul as moved, and in the second special absurdities which follow
from calling it a number. How we to imagine a unit being moved? By
what agency? What sort of movement can be attributed to what is without
parts or internal differences? If the unit is both originative of
movement and itself capable of being moved, it must contain difference.

Further, since they say a moving line generates a surface and a moving
point a line, the movements of the psychic units must be lines (for
a point is a unit having position, and the number of the soul is,
of course, somewhere and has position). 

Again, if from a number a number or a unit is subtracted, the remainder
is another number; but plants and many animals when divided continue
to live, and each segment is thought to retain the same kind of soul.

It must be all the same whether we speak of units or corpuscles; for
if the spherical atoms of Democritus became points, nothing being
retained but their being a quantum, there must remain in each a moving
and a moved part, just as there is in what is continuous; what happens
has nothing to do with the size of the atoms, it depends solely upon
their being a quantum. That is why there must be something to originate
movement in the units. If in the animal what originates movement is
the soul, so also must it be in the case of the number, so that not
the mover and the moved together, but the mover only, will be the
soul. But how is it possible for one of the units to fulfil this function
of originating movement? There must be some difference between such
a unit and all the other units, and what difference can there be between
one placed unit and another except a difference of position? If then,
on the other hand, these psychic units within the body are different
from the points of the body, there will be two sets of units both
occupying the same place; for each unit will occupy a point. And yet,
if there can be two, why cannot there be an infinite number? For if
things can occupy an indivisible lace, they must themselves be indivisible.
If, on the other hand, the points of the body are identical with the
units whose number is the soul, or if the number of the points in
the body is the soul, why have not all bodies souls? For all bodies
contain points or an infinity of points. 

Further, how is it possible for these points to be isolated or separated
from their bodies, seeing that lines cannot be resolved into points?

Part 5

The result is, as we have said, that this view, while on the one side
identical with that of those who maintain that soul is a subtle kind
of body, is on the other entangled in the absurdity peculiar to Democritus'
way of describing the manner in which movement is originated by soul.
For if the soul is present throughout the whole percipient body, there
must, if the soul be a kind of body, be two bodies in the same place;
and for those who call it a number, there must be many points at one
point, or every body must have a soul, unless the soul be a different
sort of number-other, that is, than the sum of the points existing
in a body. Another consequence that follows is that the animal must
be moved by its number precisely in the way that Democritus explained
its being moved by his spherical psychic atoms. What difference does
it make whether we speak of small spheres or of large units, or, quite
simply, of units in movement? One way or another, the movements of
the animal must be due to their movements. Hence those who combine
movement and number in the same subject lay themselves open to these
and many other similar absurdities. It is impossible not only that
these characters should give the definition of soul-it is impossible
that they should even be attributes of it. The point is clear if the
attempt be made to start from this as the account of soul and explain
from it the affections and actions of the soul, e.g. reasoning, sensation,
pleasure, pain, &c. For, to repeat what we have said earlier, movement
and number do not facilitate even conjecture about the derivative
properties of soul. 

Such are the three ways in which soul has traditionally been defined;
one group of thinkers declared it to be that which is most originative
of movement because it moves itself, another group to be the subtlest
and most nearly incorporeal of all kinds of body. We have now sufficiently
set forth the difficulties and inconsistencies to which these theories
are exposed. It remains now to examine the doctrine that soul is composed
of the elements. 

The reason assigned for this doctrine is that thus the soul may perceive
or come to know everything that is, but the theory necessarily involves
itself in many impossibilities. Its upholders assume that like is
known only by like, and imagine that by declaring the soul to be composed
of the elements they succeed in identifying the soul with all the
things it is capable of apprehending. But the elements are not the
only things it knows; there are many others, or, more exactly, an
infinite number of others, formed out of the elements. Let us admit
that the soul knows or perceives the elements out of which each of
these composites is made up; but by what means will it know or perceive
the composite whole, e.g. what God, man, flesh, bone (or any other
compound) is? For each is, not merely the elements of which it is
composed, but those elements combined in a determinate mode or ratio,
as Empedocles himself says of bone, 

The kindly Earth in its broad-bosomed moulds 

Won of clear Water two parts out of eight, And four of Fire; and so
white bones were formed. 

Nothing, therefore, will be gained by the presence of the elements
in the soul, unless there be also present there the various formulae
of proportion and the various compositions in accordance with them.
Each element will indeed know its fellow outside, but there will be
no knowledge of bone or man, unless they too are present in the constitution
of the soul. The impossibility of this needs no pointing out; for
who would suggest that stone or man could enter into the constitution
of the soul? The same applies to 'the good' and 'the not-good', and
so on. 

Further, the word 'is' has many meanings: it may be used of a 'this'
or substance, or of a quantum, or of a quale, or of any other of the
kinds of predicates we have distinguished. Does the soul consist of
all of these or not? It does not appear that all have common elements.
Is the soul formed out of those elements alone which enter into substances?
so how will it be able to know each of the other kinds of thing? Will
it be said that each kind of thing has elements or principles of its
own, and that the soul is formed out of the whole of these? In that
case, the soul must be a quantum and a quale and a substance. But
all that can be made out of the elements of a quantum is a quantum,
not a substance. These (and others like them) are the consequences
of the view that the soul is composed of all the elements.

It is absurd, also, to say both (a) that like is not capable of being
affected by like, and (b) that like is perceived or known by like,
for perceiving, and also both thinking and knowing, are, on their
own assumption, ways of being affected or moved. 

There are many puzzles and difficulties raised by saying, as Empedocles
does, that each set of things is known by means of its corporeal elements
and by reference to something in soul which is like them, and additional
testimony is furnished by this new consideration; for all the parts
of the animal body which consist wholly of earth such as bones, sinews,
and hair seem to be wholly insensitive and consequently not perceptive
even of objects earthy like themselves, as they ought to have been.

Further, each of the principles will have far more ignorance than
knowledge, for though each of them will know one thing, there will
be many of which it will be ignorant. Empedocles at any rate must
conclude that his God is the least intelligent of all beings, for
of him alone is it true that there is one thing, Strife, which he
does not know, while there is nothing which mortal beings do not know,
for ere is nothing which does not enter into their composition.

In general, we may ask, Why has not everything a soul, since everything
either is an element, or is formed out of one or several or all of
the elements? Each must certainly know one or several or all.

The problem might also be raised, What is that which unifies the elements
into a soul? The elements correspond, it would appear, to the matter;
what unites them, whatever it is, is the supremely important factor.
But it is impossible that there should be something superior to, and
dominant over, the soul (and a fortiori over the mind); it is reasonable
to hold that mind is by nature most primordial and dominant, while
their statement that it is the elements which are first of all that
is. 

All, both those who assert that the soul, because of its knowledge
or perception of what is compounded out of the elements, and is those
who assert that it is of all things the most originative of movement,
fail to take into consideration all kinds of soul. In fact (1) not
all beings that perceive can originate movement; there appear to be
certain animals which stationary, and yet local movement is the only
one, so it seems, which the soul originates in animals. And (2) the
same object-on holds against all those who construct mind and the
perceptive faculty out of the elements; for it appears that plants
live, and yet are not endowed with locomotion or perception, while
a large number of animals are without discourse of reason. Even if
these points were waived and mind admitted to be a part of the soul
(and so too the perceptive faculty), still, even so, there would be
kinds and parts of soul of which they had failed to give any account.

The same objection lies against the view expressed in the 'Orphic'
poems: there it is said that the soul comes in from the whole when
breathing takes place, being borne in upon the winds. Now this cannot
take place in the case of plants, nor indeed in the case of certain
classes of animal, for not all classes of animal breathe. This fact
has escaped the notice of the holders of this view. 

If we must construct the soul out of the elements, there is no necessity
to suppose that all the elements enter into its construction; one
element in each pair of contraries will suffice to enable it to know
both that element itself and its contrary. By means of the straight
line we know both itself and the curved-the carpenter's rule enables
us to test both-but what is curved does not enable us to distinguish
either itself or the straight. Certain thinkers say that soul is intermingled
in the whole universe, and it is perhaps for that reason that Thales
came to the opinion that all things are full of gods. This presents
some difficulties: Why does the soul when it resides in air or fire
not form an animal, while it does so when it resides in mixtures of
the elements, and that although it is held to be of higher quality
when contained in the former? (One might add the question, why the
soul in air is maintained to be higher and more immortal than that
in animals.) Both possible ways of replying to the former question
lead to absurdity or paradox; for it is beyond paradox to say that
fire or air is an animal, and it is absurd to refuse the name of animal
to what has soul in it. The opinion that the elements have soul in
them seems to have arisen from the doctrine that a whole must be homogeneous
with its parts. If it is true that animals become animate by drawing
into themselves a portion of what surrounds them, the partisans of
this view are bound to say that the soul of the Whole too is homogeneous
with all its parts. If the air sucked in is homogeneous, but soul
heterogeneous, clearly while some part of soul will exist in the inbreathed
air, some other part will not. The soul must either be homogeneous,
or such that there are some parts of the Whole in which it is not
to be found. 

From what has been said it is now clear that knowing as an attribute
of soul cannot be explained by soul's being composed of the elements,
and that it is neither sound nor true to speak of soul as moved. But
since (a) knowing, perceiving, opining, and further (b) desiring,
wishing, and generally all other modes of appetition, belong to soul,
and (c) the local movements of animals, and (d) growth, maturity,
and decay are produced by the soul, we must ask whether each of these
is an attribute of the soul as a whole, i.e. whether it is with the
whole soul we think, perceive, move ourselves, act or are acted upon,
or whether each of them requires a different part of the soul? So
too with regard to life. Does it depend on one of the parts of soul?
Or is it dependent on more than one? Or on all? Or has it some quite
other cause? 

Some hold that the soul is divisible, and that one part thinks, another
desires. If, then, its nature admits of its being divided, what can
it be that holds the parts together? Surely not the body; on the contrary
it seems rather to be the soul that holds the body together; at any
rate when the soul departs the body disintegrates and decays. If,
then, there is something else which makes the soul one, this unifying
agency would have the best right to the name of soul, and we shall
have to repeat for it the question: Is it one or multipartite? If
it is one, why not at once admit that 'the soul' is one? If it has
parts, once more the question must be put: What holds its parts together,
and so ad infinitum? 

The question might also be raised about the parts of the soul: What
is the separate role of each in relation to the body? For, if the
whole soul holds together the whole body, we should expect each part
of the soul to hold together a part of the body. But this seems an
impossibility; it is difficult even to imagine what sort of bodily
part mind will hold together, or how it will do this. 

It is a fact of observation that plants and certain insects go on
living when divided into segments; this means that each of the segments
has a soul in it identical in species, though not numerically identical
in the different segments, for both of the segments for a time possess
the power of sensation and local movement. That this does not last
is not surprising, for they no longer possess the organs necessary
for self-maintenance. But, all the same, in each of the bodily parts
there are present all the parts of soul, and the souls so present
are homogeneous with one another and with the whole; this means that
the several parts of the soul are indisseverable from one another,
although the whole soul is divisible. It seems also that the principle
found in plants is also a kind of soul; for this is the only principle
which is common to both animals and plants; and this exists in isolation
from the principle of sensation, though there nothing which has the
latter without the former. 

----------------------------------------------------------------------

BOOK II

Part 1 

Let the foregoing suffice as our account of the views concerning
the soul which have been handed on by our predecessors; let us now
dismiss them and make as it were a completely fresh start, endeavouring
to give a precise answer to the question, What is soul? i.e. to formulate
the most general possible definition of it. 

We are in the habit of recognizing, as one determinate kind of what
is, substance, and that in several senses, (a) in the sense of matter
or that which in itself is not 'a this', and (b) in the sense of form
or essence, which is that precisely in virtue of which a thing is
called 'a this', and thirdly (c) in the sense of that which is compounded
of both (a) and (b). Now matter is potentiality, form actuality; of
the latter there are two grades related to one another as e.g. knowledge
to the exercise of knowledge. 

Among substances are by general consent reckoned bodies and especially
natural bodies; for they are the principles of all other bodies. Of
natural bodies some have life in them, others not; by life we mean
self-nutrition and growth (with its correlative decay). It follows
that every natural body which has life in it is a substance in the
sense of a composite. 

But since it is also a body of such and such a kind, viz. having life,
the body cannot be soul; the body is the subject or matter, not what
is attributed to it. Hence the soul must be a substance in the sense
of the form of a natural body having life potentially within it. But
substance is actuality, and thus soul is the actuality of a body as
above characterized. Now the word actuality has two senses corresponding
respectively to the possession of knowledge and the actual exercise
of knowledge. It is obvious that the soul is actuality in the first
sense, viz. that of knowledge as possessed, for both sleeping and
waking presuppose the existence of soul, and of these waking corresponds
to actual knowing, sleeping to knowledge possessed but not employed,
and, in the history of the individual, knowledge comes before its
employment or exercise. 

That is why the soul is the first grade of actuality of a natural
body having life potentially in it. The body so described is a body
which is organized. The parts of plants in spite of their extreme
simplicity are 'organs'; e.g. the leaf serves to shelter the pericarp,
the pericarp to shelter the fruit, while the roots of plants are analogous
to the mouth of animals, both serving for the absorption of food.
If, then, we have to give a general formula applicable to all kinds
of soul, we must describe it as the first grade of actuality of a
natural organized body. That is why we can wholly dismiss as unnecessary
the question whether the soul and the body are one: it is as meaningless
as to ask whether the wax and the shape given to it by the stamp are
one, or generally the matter of a thing and that of which it is the
matter. Unity has many senses (as many as 'is' has), but the most
proper and fundamental sense of both is the relation of an actuality
to that of which it is the actuality. We have now given an answer
to the question, What is soul?-an answer which applies to it in its
full extent. It is substance in the sense which corresponds to the
definitive formula of a thing's essence. That means that it is 'the
essential whatness' of a body of the character just assigned. Suppose
that what is literally an 'organ', like an axe, were a natural body,
its 'essential whatness', would have been its essence, and so its
soul; if this disappeared from it, it would have ceased to be an axe,
except in name. As it is, it is just an axe; it wants the character
which is required to make its whatness or formulable essence a soul;
for that, it would have had to be a natural body of a particular kind,
viz. one having in itself the power of setting itself in movement
and arresting itself. Next, apply this doctrine in the case of the
'parts' of the living body. Suppose that the eye were an animal-sight
would have been its soul, for sight is the substance or essence of
the eye which corresponds to the formula, the eye being merely the
matter of seeing; when seeing is removed the eye is no longer an eye,
except in name-it is no more a real eye than the eye of a statue or
of a painted figure. We must now extend our consideration from the
'parts' to the whole living body; for what the departmental sense
is to the bodily part which is its organ, that the whole faculty of
sense is to the whole sensitive body as such. 

We must not understand by that which is 'potentially capable of living'
what has lost the soul it had, but only what still retains it; but
seeds and fruits are bodies which possess the qualification. Consequently,
while waking is actuality in a sense corresponding to the cutting
and the seeing, the soul is actuality in the sense corresponding to
the power of sight and the power in the tool; the body corresponds
to what exists in potentiality; as the pupil plus the power of sight
constitutes the eye, so the soul plus the body constitutes the animal.

From this it indubitably follows that the soul is inseparable from
its body, or at any rate that certain parts of it are (if it has parts)
for the actuality of some of them is nothing but the actualities of
their bodily parts. Yet some may be separable because they are not
the actualities of any body at all. Further, we have no light on the
problem whether the soul may not be the actuality of its body in the
sense in which the sailor is the actuality of the ship. 

This must suffice as our sketch or outline determination of the nature
of soul. 

Part 2

Since what is clear or logically more evident emerges from what in
itself is confused but more observable by us, we must reconsider our
results from this point of view. For it is not enough for a definitive
formula to express as most now do the mere fact; it must include and
exhibit the ground also. At present definitions are given in a form
analogous to the conclusion of a syllogism; e.g. What is squaring?
The construction of an equilateral rectangle equal to a given oblong
rectangle. Such a definition is in form equivalent to a conclusion.
One that tells us that squaring is the discovery of a line which is
a mean proportional between the two unequal sides of the given rectangle
discloses the ground of what is defined. 

We resume our inquiry from a fresh starting-point by calling attention
to the fact that what has soul in it differs from what has not, in
that the former displays life. Now this word has more than one sense,
and provided any one alone of these is found in a thing we say that
thing is living. Living, that is, may mean thinking or perception
or local movement and rest, or movement in the sense of nutrition,
decay and growth. Hence we think of plants also as living, for they
are observed to possess in themselves an originative power through
which they increase or decrease in all spatial directions; they grow
up and down, and everything that grows increases its bulk alike in
both directions or indeed in all, and continues to live so long as
it can absorb nutriment. 

This power of self-nutrition can be isolated from the other powers
mentioned, but not they from it-in mortal beings at least. The fact
is obvious in plants; for it is the only psychic power they possess.

This is the originative power the possession of which leads us to
speak of things as living at all, but it is the possession of sensation
that leads us for the first time to speak of living things as animals;
for even those beings which possess no power of local movement but
do possess the power of sensation we call animals and not merely living
things. 

The primary form of sense is touch, which belongs to all animals.
just as the power of self-nutrition can be isolated from touch and
sensation generally, so touch can be isolated from all other forms
of sense. (By the power of self-nutrition we mean that departmental
power of the soul which is common to plants and animals: all animals
whatsoever are observed to have the sense of touch.) What the explanation
of these two facts is, we must discuss later. At present we must confine
ourselves to saying that soul is the source of these phenomena and
is characterized by them, viz. by the powers of self-nutrition, sensation,
thinking, and motivity. 

Is each of these a soul or a part of a soul? And if a part, a part
in what sense? A part merely distinguishable by definition or a part
distinct in local situation as well? In the case of certain of these
powers, the answers to these questions are easy, in the case of others
we are puzzled what to say. just as in the case of plants which when
divided are observed to continue to live though removed to a distance
from one another (thus showing that in their case the soul of each
individual plant before division was actually one, potentially many),
so we notice a similar result in other varieties of soul, i.e. in
insects which have been cut in two; each of the segments possesses
both sensation and local movement; and if sensation, necessarily also
imagination and appetition; for, where there is sensation, there is
also pleasure and pain, and, where these, necessarily also desire.

We have no evidence as yet about mind or the power to think; it seems
to be a widely different kind of soul, differing as what is eternal
from what is perishable; it alone is capable of existence in isolation
from all other psychic powers. All the other parts of soul, it is
evident from what we have said, are, in spite of certain statements
to the contrary, incapable of separate existence though, of course,
distinguishable by definition. If opining is distinct from perceiving,
to be capable of opining and to be capable of perceiving must be distinct,
and so with all the other forms of living above enumerated. Further,
some animals possess all these parts of soul, some certain of them
only, others one only (this is what enables us to classify animals);
the cause must be considered later.' A similar arrangement is found
also within the field of the senses; some classes of animals have
all the senses, some only certain of them, others only one, the most
indispensable, touch. 

Since the expression 'that whereby we live and perceive' has two meanings,
just like the expression 'that whereby we know'-that may mean either
(a) knowledge or (b) the soul, for we can speak of knowing by or with
either, and similarly that whereby we are in health may be either
(a) health or (b) the body or some part of the body; and since of
the two terms thus contrasted knowledge or health is the name of a
form, essence, or ratio, or if we so express it an actuality of a
recipient matter-knowledge of what is capable of knowing, health of
what is capable of being made healthy (for the operation of that which
is capable of originating change terminates and has its seat in what
is changed or altered); further, since it is the soul by or with which
primarily we live, perceive, and think:-it follows that the soul must
be a ratio or formulable essence, not a matter or subject. For, as
we said, word substance has three meanings form, matter, and the complex
of both and of these three what is called matter is potentiality,
what is called form actuality. Since then the complex here is the
living thing, the body cannot be the actuality of the soul; it is
the soul which is the actuality of a certain kind of body. Hence the
rightness of the view that the soul cannot be without a body, while
it csnnot he a body; it is not a body but something relative to a
body. That is why it is in a body, and a body of a definite kind.
It was a mistake, therefore, to do as former thinkers did, merely
to fit it into a body without adding a definite specification of the
kind or character of that body. Reflection confirms the observed fact;
the actuality of any given thing can only be realized in what is already
potentially that thing, i.e. in a matter of its own appropriate to
it. From all this it follows that soul is an actuality or formulable
essence of something that possesses a potentiality of being besouled.

Part 3

Of the psychic powers above enumerated some kinds of living things,
as we have said, possess all, some less than all, others one only.
Those we have mentioned are the nutritive, the appetitive, the sensory,
the locomotive, and the power of thinking. Plants have none but the
first, the nutritive, while another order of living things has this
plus the sensory. If any order of living things has the sensory, it
must also have the appetitive; for appetite is the genus of which
desire, passion, and wish are the species; now all animals have one
sense at least, viz. touch, and whatever has a sense has the capacity
for pleasure and pain and therefore has pleasant and painful objects
present to it, and wherever these are present, there is desire, for
desire is just appetition of what is pleasant. Further, all animals
have the sense for food (for touch is the sense for food); the food
of all living things consists of what is dry, moist, hot, cold, and
these are the qualities apprehended by touch; all other sensible qualities
are apprehended by touch only indirectly. Sounds, colours, and odours
contribute nothing to nutriment; flavours fall within the field of
tangible qualities. Hunger and thirst are forms of desire, hunger
a desire for what is dry and hot, thirst a desire for what is cold
and moist; flavour is a sort of seasoning added to both. We must later
clear up these points, but at present it may be enough to say that
all animals that possess the sense of touch have also appetition.
The case of imagination is obscure; we must examine it later. Certain
kinds of animals possess in addition the power of locomotion, and
still another order of animate beings, i.e. man and possibly another
order like man or superior to him, the power of thinking, i.e. mind.
It is now evident that a single definition can be given of soul only
in the same sense as one can be given of figure. For, as in that case
there is no figure distinguishable and apart from triangle, &c., so
here there is no soul apart from the forms of soul just enumerated.
It is true that a highly general definition can be given for figure
which will fit all figures without expressing the peculiar nature
of any figure. So here in the case of soul and its specific forms.
Hence it is absurd in this and similar cases to demand an absolutely
general definition which will fail to express the peculiar nature
of anything that is, or again, omitting this, to look for separate
definitions corresponding to each infima species. The cases of figure
and soul are exactly parallel; for the particulars subsumed under
the common name in both cases-figures and living beings-constitute
a series, each successive term of which potentially contains its predecessor,
e.g. the square the triangle, the sensory power the self-nutritive.
Hence we must ask in the case of each order of living things, What
is its soul, i.e. What is the soul of plant, animal, man? Why the
terms are related in this serial way must form the subject of later
examination. But the facts are that the power of perception is never
found apart from the power of self-nutrition, while-in plants-the
latter is found isolated from the former. Again, no sense is found
apart from that of touch, while touch is found by itself; many animals
have neither sight, hearing, nor smell. Again, among living things
that possess sense some have the power of locomotion, some not. Lastly,
certain living beings-a small minority-possess calculation and thought,
for (among mortal beings) those which possess calculation have all
the other powers above mentioned, while the converse does not hold-indeed
some live by imagination alone, while others have not even imagination.
The mind that knows with immediate intuition presents a different
problem. 

It is evident that the way to give the most adequate definition of
soul is to seek in the case of each of its forms for the most appropriate
definition. 

Part 4

It is necessary for the student of these forms of soul first to find
a definition of each, expressive of what it is, and then to investigate
its derivative properties, &c. But if we are to express what each
is, viz. what the thinking power is, or the perceptive, or the nutritive,
we must go farther back and first give an account of thinking or perceiving,
for in the order of investigation the question of what an agent does
precedes the question, what enables it to do what it does. If this
is correct, we must on the same ground go yet another step farther
back and have some clear view of the objects of each; thus we must
start with these objects, e.g. with food, with what is perceptible,
or with what is intelligible. 

It follows that first of all we must treat of nutrition and reproduction,
for the nutritive soul is found along with all the others and is the
most primitive and widely distributed power of soul, being indeed
that one in virtue of which all are said to have life. The acts in
which it manifests itself are reproduction and the use of food-reproduction,
I say, because for any living thing that has reached its normal development
and which is unmutilated, and whose mode of generation is not spontaneous,
the most natural act is the production of another like itself, an
animal producing an animal, a plant a plant, in order that, as far
as its nature allows, it may partake in the eternal and divine. That
is the goal towards which all things strive, that for the sake of
which they do whatsoever their nature renders possible. The phrase
'for the sake of which' is ambiguous; it may mean either (a) the end
to achieve which, or (b) the being in whose interest, the act is done.
Since then no living thing is able to partake in what is eternal and
divine by uninterrupted continuance (for nothing perishable can for
ever remain one and the same), it tries to achieve that end in the
only way possible to it, and success is possible in varying degrees;
so it remains not indeed as the self-same individual but continues
its existence in something like itself-not numerically but specifically
one. 

The soul is the cause or source of the living body. The terms cause
and source have many senses. But the soul is the cause of its body
alike in all three senses which we explicitly recognize. It is (a)
the source or origin of movement, it is (b) the end, it is (c) the
essence of the whole living body. 

That it is the last, is clear; for in everything the essence is identical
with the ground of its being, and here, in the case of living things,
their being is to live, and of their being and their living the soul
in them is the cause or source. Further, the actuality of whatever
is potential is identical with its formulable essence. 

It is manifest that the soul is also the final cause of its body.
For Nature, like mind, always does whatever it does for the sake of
something, which something is its end. To that something corresponds
in the case of animals the soul and in this it follows the order of
nature; all natural bodies are organs of the soul. This is true of
those that enter into the constitution of plants as well as of those
which enter into that of animals. This shows that that the sake of
which they are is soul. We must here recall the two senses of 'that
for the sake of which', viz. (a) the end to achieve which, and (b)
the being in whose interest, anything is or is done. 

We must maintain, further, that the soul is also the cause of the
living body as the original source of local movement. The power of
locomotion is not found, however, in all living things. But change
of quality and change of quantity are also due to the soul. Sensation
is held to be a qualitative alteration, and nothing except what has
soul in it is capable of sensation. The same holds of the quantitative
changes which constitute growth and decay; nothing grows or decays
naturally except what feeds itself, and nothing feeds itself except
what has a share of soul in it. 

Empedocles is wrong in adding that growth in plants is to be explained,
the downward rooting by the natural tendency of earth to travel downwards,
and the upward branching by the similar natural tendency of fire to
travel upwards. For he misinterprets up and down; up and down are
not for all things what they are for the whole Cosmos: if we are to
distinguish and identify organs according to their functions, the
roots of plants are analogous to the head in animals. Further, we
must ask what is the force that holds together the earth and the fire
which tend to travel in contrary directions; if there is no counteracting
force, they will be torn asunder; if there is, this must be the soul
and the cause of nutrition and growth. By some the element of fire
is held to be the cause of nutrition and growth, for it alone of the
primary bodies or elements is observed to feed and increase itself.
Hence the suggestion that in both plants and animals it is it which
is the operative force. A concurrent cause in a sense it certainly
is, but not the principal cause, that is rather the soul; for while
the growth of fire goes on without limit so long as there is a supply
of fuel, in the case of all complex wholes formed in the course of
nature there is a limit or ratio which determines their size and increase,
and limit and ratio are marks of soul but not of fire, and belong
to the side of formulable essence rather than that of matter.

Nutrition and reproduction are due to one and the same psychic power.
It is necessary first to give precision to our account of food, for
it is by this function of absorbing food that this psychic power is
distinguished from all the others. The current view is that what serves
as food to a living thing is what is contrary to it-not that in every
pair of contraries each is food to the other: to be food a contrary
must not only be transformable into the other and vice versa, it must
also in so doing increase the bulk of the other. Many a contrary is
transformed into its other and vice versa, where neither is even a
quantum and so cannot increase in bulk, e.g. an invalid into a healthy
subject. It is clear that not even those contraries which satisfy
both the conditions mentioned above are food to one another in precisely
the same sense; water may be said to feed fire, but not fire water.
Where the members of the pair are elementary bodies only one of the
contraries, it would appear, can be said to feed the other. But there
is a difficulty here. One set of thinkers assert that like fed, as
well as increased in amount, by like. Another set, as we have said,
maintain the very reverse, viz. that what feeds and what is fed are
contrary to one another; like, they argue, is incapable of being affected
by like; but food is changed in the process of digestion, and change
is always to what is opposite or to what is intermediate. Further,
food is acted upon by what is nourished by it, not the other way round,
as timber is worked by a carpenter and not conversely; there is a
change in the carpenter but it is merely a change from not-working
to working. In answering this problem it makes all the difference
whether we mean by 'the food' the 'finished' or the 'raw' product.
If we use the word food of both, viz. of the completely undigested
and the completely digested matter, we can justify both the rival
accounts of it; taking food in the sense of undigested matter, it
is the contrary of what is fed by it, taking it as digested it is
like what is fed by it. Consequently it is clear that in a certain
sense we may say that both parties are right, both wrong.

Since nothing except what is alive can be fed, what is fed is the
besouled body and just because it has soul in it. Hence food is essentially
related to what has soul in it. Food has a power which is other than
the power to increase the bulk of what is fed by it; so far forth
as what has soul in it is a quantum, food may increase its quantity,
but it is only so far as what has soul in it is a 'this-somewhat'
or substance that food acts as food; in that case it maintains the
being of what is fed, and that continues to be what it is so long
as the process of nutrition continues. Further, it is the agent in
generation, i.e. not the generation of the individual fed but the
reproduction of another like it; the substance of the individual fed
is already in existence; the existence of no substance is a self-generation
but only a self-maintenance. 

Hence the psychic power which we are now studying may be described
as that which tends to maintain whatever has this power in it of continuing
such as it was, and food helps it to do its work. That is why, if
deprived of food, it must cease to be. 

The process of nutrition involves three factors, (a) what is fed,
(b) that wherewith it is fed, (c) what does the feeding; of these
(c) is the first soul, (a) the body which has that soul in it, (b)
the food. But since it is right to call things after the ends they
realize, and the end of this soul is to generate another being like
that in which it is, the first soul ought to be named the reproductive
soul. The expression (b) 'wherewith it is fed' is ambiguous just as
is the expression 'wherewith the ship is steered'; that may mean either
(i) the hand or (ii) the rudder, i.e. either (i) what is moved and
sets in movement, or (ii) what is merely moved. We can apply this
analogy here if we recall that all food must be capable of being digested,
and that what produces digestion is warmth; that is why everything
that has soul in it possesses warmth. 

We have now given an outline account of the nature of food; further
details must be given in the appropriate place. 

Part 5

Having made these distinctions let us now speak of sensation in the
widest sense. Sensation depends, as we have said, on a process of
movement or affection from without, for it is held to be some sort
of change of quality. Now some thinkers assert that like is affected
only by like; in what sense this is possible and in what sense impossible,
we have explained in our general discussion of acting and being acted
upon. 

Here arises a problem: why do we not perceive the senses themselves
as well as the external objects of sense, or why without the stimulation
of external objects do they not produce sensation, seeing that they
contain in themselves fire, earth, and all the other elements, which
are the direct or indirect objects is so of sense? It is clear that
what is sensitive is only potentially, not actually. The power of
sense is parallel to what is combustible, for that never ignites itself
spontaneously, but requires an agent which has the power of starting
ignition; otherwise it could have set itself on fire, and would not
have needed actual fire to set it ablaze. 

In reply we must recall that we use the word 'perceive' in two ways,
for we say (a) that what has the power to hear or see, 'sees' or 'hears',
even though it is at the moment asleep, and also (b) that what is
actually seeing or hearing, 'sees' or 'hears'. Hence 'sense' too must
have two meanings, sense potential, and sense actual. Similarly 'to
be a sentient' means either (a) to have a certain power or (b) to
manifest a certain activity. To begin with, for a time, let us speak
as if there were no difference between (i) being moved or affected,
and (ii) being active, for movement is a kind of activity-an imperfect
kind, as has elsewhere been explained. Everything that is acted upon
or moved is acted upon by an agent which is actually at work. Hence
it is that in one sense, as has already been stated, what acts and
what is acted upon are like, in another unlike, i.e. prior to and
during the change the two factors are unlike, after it like.

But we must now distinguish not only between what is potential and
what is actual but also different senses in which things can be said
to be potential or actual; up to now we have been speaking as if each
of these phrases had only one sense. We can speak of something as
'a knower' either (a) as when we say that man is a knower, meaning
that man falls within the class of beings that know or have knowledge,
or (b) as when we are speaking of a man who possesses a knowledge
of grammar; each of these is so called as having in him a certain
potentiality, but there is a difference between their respective potentialities,
the one (a) being a potential knower, because his kind or matter is
such and such, the other (b), because he can in the absence of any
external counteracting cause realize his knowledge in actual knowing
at will. This implies a third meaning of 'a knower' (c), one who is
already realizing his knowledge-he is a knower in actuality and in
the most proper sense is knowing, e.g. this A. Both the former are
potential knowers, who realize their respective potentialities, the
one (a) by change of quality, i.e. repeated transitions from one state
to its opposite under instruction, the other (b) by the transition
from the inactive possession of sense or grammar to their active exercise.
The two kinds of transition are distinct. 

Also the expression 'to be acted upon' has more than one meaning;
it may mean either (a) the extinction of one of two contraries by
the other, or (b) the maintenance of what is potential by the agency
of what is actual and already like what is acted upon, with such likeness
as is compatible with one's being actual and the other potential.
For what possesses knowledge becomes an actual knower by a transition
which is either not an alteration of it at all (being in reality a
development into its true self or actuality) or at least an alteration
in a quite different sense from the usual meaning. 

Hence it is wrong to speak of a wise man as being 'altered' when he
uses his wisdom, just as it would be absurd to speak of a builder
as being altered when he is using his skill in building a house.

What in the case of knowing or understanding leads from potentiality
to actuality ought not to be called teaching but something else. That
which starting with the power to know learns or acquires knowledge
through the agency of one who actually knows and has the power of
teaching either (a) ought not to be said 'to be acted upon' at all
or (b) we must recognize two senses of alteration, viz. (i) the substitution
of one quality for another, the first being the contrary of the second,
or (ii) the development of an existent quality from potentiality in
the direction of fixity or nature. 

In the case of what is to possess sense, the first transition is due
to the action of the male parent and takes place before birth so that
at birth the living thing is, in respect of sensation, at the stage
which corresponds to the possession of knowledge. Actual sensation
corresponds to the stage of the exercise of knowledge. But between
the two cases compared there is a difference; the objects that excite
the sensory powers to activity, the seen, the heard, &c., are outside.
The ground of this difference is that what actual sensation apprehends
is individuals, while what knowledge apprehends is universals, and
these are in a sense within the soul. That is why a man can exercise
his knowledge when he wishes, but his sensation does not depend upon
himself a sensible object must be there. A similar statement must
be made about our knowledge of what is sensible-on the same ground,
viz. that the sensible objects are individual and external.

A later more appropriate occasion may be found thoroughly to clear
up all this. At present it must be enough to recognize the distinctions
already drawn; a thing may be said to be potential in either of two
senses, (a) in the sense in which we might say of a boy that he may
become a general or (b) in the sense in which we might say the same
of an adult, and there are two corresponding senses of the term 'a
potential sentient'. There are no separate names for the two stages
of potentiality; we have pointed out that they are different and how
they are different. We cannot help using the incorrect terms 'being
acted upon or altered' of the two transitions involved. As we have
said, has the power of sensation is potentially like what the perceived
object is actually; that is, while at the beginning of the process
of its being acted upon the two interacting factors are dissimilar,
at the end the one acted upon is assimilated to the other and is identical
in quality with it. 

Part 6

In dealing with each of the senses we shall have first to speak of
the objects which are perceptible by each. The term 'object of sense'
covers three kinds of objects, two kinds of which are, in our language,
directly perceptible, while the remaining one is only incidentally
perceptible. Of the first two kinds one (a) consists of what is perceptible
by a single sense, the other (b) of what is perceptible by any and
all of the senses. I call by the name of special object of this or
that sense that which cannot be perceived by any other sense than
that one and in respect of which no error is possible; in this sense
colour is the special object of sight, sound of hearing, flavour of
taste. Touch, indeed, discriminates more than one set of different
qualities. Each sense has one kind of object which it discerns, and
never errs in reporting that what is before it is colour or sound
(though it may err as to what it is that is coloured or where that
is, or what it is that is sounding or where that is.) Such objects
are what we propose to call the special objects of this or that sense.

'Common sensibles' are movement, rest, number, figure, magnitude;
these are not peculiar to any one sense, but are common to all. There
are at any rate certain kinds of movement which are perceptible both
by touch and by sight. 

We speak of an incidental object of sense where e.g. the white object
which we see is the son of Diares; here because 'being the son of
Diares' is incidental to the directly visible white patch we speak
of the son of Diares as being (incidentally) perceived or seen by
us. Because this is only incidentally an object of sense, it in no
way as such affects the senses. Of the two former kinds, both of which
are in their own nature perceptible by sense, the first kind-that
of special objects of the several senses-constitute the objects of
sense in the strictest sense of the term and it is to them that in
the nature of things the structure of each several sense is adapted.

Part 7

The object of sight is the visible, and what is visible is (a) colour
and (b) a certain kind of object which can be described in words but
which has no single name; what we mean by (b) will be abundantly clear
as we proceed. Whatever is visible is colour and colour is what lies
upon what is in its own nature visible; 'in its own nature' here means
not that visibility is involved in the definition of what thus underlies
colour, but that that substratum contains in itself the cause of visibility.
Every colour has in it the power to set in movement what is actually
transparent; that power constitutes its very nature. That is why it
is not visible except with the help of light; it is only in light
that the colour of a thing is seen. Hence our first task is to explain
what light is. 

Now there clearly is something which is transparent, and by 'transparent'
I mean what is visible, and yet not visible in itself, but rather
owing its visibility to the colour of something else; of this character
are air, water, and many solid bodies. Neither air nor water is transparent
because it is air or water; they are transparent because each of them
has contained in it a certain substance which is the same in both
and is also found in the eternal body which constitutes the uppermost
shell of the physical Cosmos. Of this substance light is the activity-the
activity of what is transparent so far forth as it has in it the determinate
power of becoming transparent; where this power is present, there
is also the potentiality of the contrary, viz. darkness. Light is
as it were the proper colour of what is transparent, and exists whenever
the potentially transparent is excited to actuality by the influence
of fire or something resembling 'the uppermost body'; for fire too
contains something which is one and the same with the substance in
question. 

We have now explained what the transparent is and what light is; light
is neither fire nor any kind whatsoever of body nor an efflux from
any kind of body (if it were, it would again itself be a kind of body)-it
is the presence of fire or something resembling fire in what is transparent.
It is certainly not a body, for two bodies cannot be present in the
same place. The opposite of light is darkness; darkness is the absence
from what is transparent of the corresponding positive state above
characterized; clearly therefore, light is just the presence of that.

Empedocles (and with him all others who used the same forms of expression)
was wrong in speaking of light as 'travelling' or being at a given
moment between the earth and its envelope, its movement being unobservable
by us; that view is contrary both to the clear evidence of argument
and to the observed facts; if the distance traversed were short, the
movement might have been unobservable, but where the distance is from
extreme East to extreme West, the draught upon our powers of belief
is too great. 

What is capable of taking on colour is what in itself is colourless,
as what can take on sound is what is soundless; what is colourless
includes (a) what is transparent and (b) what is invisible or scarcely
visible, i.e. what is 'dark'. The latter (b) is the same as what is
transparent, when it is potentially, not of course when it is actually
transparent; it is the same substance which is now darkness, now light.

Not everything that is visible depends upon light for its visibility.
This is only true of the 'proper' colour of things. Some objects of
sight which in light are invisible, in darkness stimulate the sense;
that is, things that appear fiery or shining. This class of objects
has no simple common name, but instances of it are fungi, flesh, heads,
scales, and eyes of fish. In none of these is what is seen their own
proper' colour. Why we see these at all is another question. At present
what is obvious is that what is seen in light is always colour. That
is why without the help of light colour remains invisible. Its being
colour at all means precisely its having in it the power to set in
movement what is already actually transparent, and, as we have seen,
the actuality of what is transparent is just light. 

The following experiment makes the necessity of a medium clear. If
what has colour is placed in immediate contact with the eye, it cannot
be seen. Colour sets in movement not the sense organ but what is transparent,
e.g. the air, and that, extending continuously from the object to
the organ, sets the latter in movement. Democritus misrepresents the
facts when he expresses the opinion that if the interspace were empty
one could distinctly see an ant on the vault of the sky; that is an
impossibility. Seeing is due to an affection or change of what has
the perceptive faculty, and it cannot be affected by the seen colour
itself; it remains that it must be affected by what comes between.
Hence it is indispensable that there be something in between-if there
were nothing, so far from seeing with greater distinctness, we should
see nothing at all. 

We have now explained the cause why colour cannot be seen otherwise
than in light. Fire on the other hand is seen both in darkness and
in light; this double possibility follows necessarily from our theory,
for it is just fire that makes what is potentially transparent actually
transparent. 

The same account holds also of sound and smell; if the object of either
of these senses is in immediate contact with the organ no sensation
is produced. In both cases the object sets in movement only what lies
between, and this in turn sets the organ in movement: if what sounds
or smells is brought into immediate contact with the organ, no sensation
will be produced. The same, in spite of all appearances, applies also
to touch and taste; why there is this apparent difference will be
clear later. What comes between in the case of sounds is air; the
corresponding medium in the case of smell has no name. But, corresponding
to what is transparent in the case of colour, there is a quality found
both in air and water, which serves as a medium for what has smell-I
say 'in water' because animals that live in water as well as those
that live on land seem to possess the sense of smell, and 'in air'
because man and all other land animals that breathe, perceive smells
only when they breathe air in. The explanation of this too will be
given later. 

Part 8

Now let us, to begin with, make certain distinctions about sound and
hearing. 

Sound may mean either of two things (a) actual, and (b) potential,
sound. There are certain things which, as we say, 'have no sound',
e.g. sponges or wool, others which have, e.g. bronze and in general
all things which are smooth and solid-the latter are said to have
a sound because they can make a sound, i.e. can generate actual sound
between themselves and the organ of hearing. 

Actual sound requires for its occurrence (i, ii) two such bodies and
(iii) a space between them; for it is generated by an impact. Hence
it is impossible for one body only to generate a sound-there must
be a body impinging and a body impinged upon; what sounds does so
by striking against something else, and this is impossible without
a movement from place to place. 

As we have said, not all bodies can by impact on one another produce
sound; impact on wool makes no sound, while the impact on bronze or
any body which is smooth and hollow does. Bronze gives out a sound
when struck because it is smooth; bodies which are hollow owing to
reflection repeat the original impact over and over again, the body
originally set in movement being unable to escape from the concavity.

Further, we must remark that sound is heard both in air and in water,
though less distinctly in the latter. Yet neither air nor water is
the principal cause of sound. What is required for the production
of sound is an impact of two solids against one another and against
the air. The latter condition is satisfied when the air impinged upon
does not retreat before the blow, i.e. is not dissipated by it.

That is why it must be struck with a sudden sharp blow, if it is to
sound-the movement of the whip must outrun the dispersion of the air,
just as one might get in a stroke at a heap or whirl of sand as it
was traveling rapidly past. 

An echo occurs, when, a mass of air having been unified, bounded,
and prevented from dissipation by the containing walls of a vessel,
the air originally struck by the impinging body and set in movement
by it rebounds from this mass of air like a ball from a wall. It is
probable that in all generation of sound echo takes place, though
it is frequently only indistinctly heard. What happens here must be
analogous to what happens in the case of light; light is always reflected-otherwise
it would not be diffused and outside what was directly illuminated
by the sun there would be blank darkness; but this reflected light
is not always strong enough, as it is when it is reflected from water,
bronze, and other smooth bodies, to cast a shadow, which is the distinguishing
mark by which we recognize light. 

It is rightly said that an empty space plays the chief part in the
production of hearing, for what people mean by 'the vacuum' is the
air, which is what causes hearing, when that air is set in movement
as one continuous mass; but owing to its friability it emits no sound,
being dissipated by impinging upon any surface which is not smooth.
When the surface on which it impinges is quite smooth, what is produced
by the original impact is a united mass, a result due to the smoothness
of the surface with which the air is in contact at the other end.

What has the power of producing sound is what has the power of setting
in movement a single mass of air which is continuous from the impinging
body up to the organ of hearing. The organ of hearing is physically
united with air, and because it is in air, the air inside is moved
concurrently with the air outside. Hence animals do not hear with
all parts of their bodies, nor do all parts admit of the entrance
of air; for even the part which can be moved and can sound has not
air everywhere in it. Air in itself is, owing to its friability, quite
soundless; only when its dissipation is prevented is its movement
sound. The air in the ear is built into a chamber just to prevent
this dissipating movement, in order that the animal may accurately
apprehend all varieties of the movements of the air outside. That
is why we hear also in water, viz. because the water cannot get into
the air chamber or even, owing to the spirals, into the outer ear.
If this does happen, hearing ceases, as it also does if the tympanic
membrane is damaged, just as sight ceases if the membrane covering
the pupil is damaged. It is also a test of deafness whether the ear
does or does not reverberate like a horn; the air inside the ear has
always a movement of its own, but the sound we hear is always the
sounding of something else, not of the organ itself. That is why we
say that we hear with what is empty and echoes, viz. because what
we hear with is a chamber which contains a bounded mass of air.

Which is it that 'sounds', the striking body or the struck? Is not
the answer 'it is both, but each in a different way'? Sound is a movement
of what can rebound from a smooth surface when struck against it.
As we have explained' not everything sounds when it strikes or is
struck, e.g. if one needle is struck against another, neither emits
any sound. In order, therefore, that sound may be generated, what
is struck must be smooth, to enable the air to rebound and be shaken
off from it in one piece. 

The distinctions between different sounding bodies show themselves
only in actual sound; as without the help of light colours remain
invisible, so without the help of actual sound the distinctions between
acute and grave sounds remain inaudible. Acute and grave are here
metaphors, transferred from their proper sphere, viz. that of touch,
where they mean respectively (a) what moves the sense much in a short
time, (b) what moves the sense little in a long time. Not that what
is sharp really moves fast, and what is grave, slowly, but that the
difference in the qualities of the one and the other movement is due
to their respective speeds. There seems to be a sort of parallelism
between what is acute or grave to hearing and what is sharp or blunt
to touch; what is sharp as it were stabs, while what is blunt pushes,
the one producing its effect in a short, the other in a long time,
so that the one is quick, the other slow. 

Let the foregoing suffice as an analysis of sound. Voice is a kind
of sound characteristic of what has soul in it; nothing that is without
soul utters voice, it being only by a metaphor that we speak of the
voice of the flute or the lyre or generally of what (being without
soul) possesses the power of producing a succession of notes which
differ in length and pitch and timbre. The metaphor is based on the
fact that all these differences are found also in voice. Many animals
are voiceless, e.g. all non-sanuineous animals and among sanguineous
animals fish. This is just what we should expect, since voice is a
certain movement of air. The fish, like those in the Achelous, which
are said to have voice, really make the sounds with their gills or
some similar organ. Voice is the sound made by an animal, and that
with a special organ. As we saw, everything that makes a sound does
so by the impact of something (a) against something else, (b) across
a space, (c) filled with air; hence it is only to be expected that
no animals utter voice except those which take in air. Once air is
inbreathed, Nature uses it for two different purposes, as the tongue
is used both for tasting and for articulating; in that case of the
two functions tasting is necessary for the animal's existence (hence
it is found more widely distributed), while articulate speech is a
luxury subserving its possessor's well-being; similarly in the former
case Nature employs the breath both as an indispensable means to the
regulation of the inner temperature of the living body and also as
the matter of articulate voice, in the interests of its possessor's
well-being. Why its former use is indispensable must be discussed
elsewhere. 

The organ of respiration is the windpipe, and the organ to which this
is related as means to end is the lungs. The latter is the part of
the body by which the temperature of land animals is raised above
that of all others. But what primarily requires the air drawn in by
respiration is not only this but the region surrounding the heart.
That is why when animals breathe the air must penetrate inwards.

Voice then is the impact of the inbreathed air against the 'windpipe',
and the agent that produces the impact is the soul resident in these
parts of the body. Not every sound, as we said, made by an animal
is voice (even with the tongue we may merely make a sound which is
not voice, or without the tongue as in coughing); what produces the
impact must have soul in it and must be accompanied by an act of imagination,
for voice is a sound with a meaning, and is not merely the result
of any impact of the breath as in coughing; in voice the breath in
the windpipe is used as an instrument to knock with against the walls
of the windpipe. This is confirmed by our inability to speak when
we are breathing either out or in-we can only do so by holding our
breath; we make the movements with the breath so checked. It is clear
also why fish are voiceless; they have no windpipe. And they have
no windpipe because they do not breathe or take in air. Why they do
not is a question belonging to another inquiry. 

Part 9

Smell and its object are much less easy to determine than what we
have hitherto discussed; the distinguishing characteristic of the
object of smell is less obvious than those of sound or colour. The
ground of this is that our power of smell is less discriminating and
in general inferior to that of many species of animals; men have a
poor sense of smell and our apprehension of its proper objects is
inseparably bound up with and so confused by pleasure and pain, which
shows that in us the organ is inaccurate. It is probable that there
is a parallel failure in the perception of colour by animals that
have hard eyes: probably they discriminate differences of colour only
by the presence or absence of what excites fear, and that it is thus
that human beings distinguish smells. It seems that there is an analogy
between smell and taste, and that the species of tastes run parallel
to those of smells-the only difference being that our sense of taste
is more discriminating than our sense of smell, because the former
is a modification of touch, which reaches in man the maximum of discriminative
accuracy. While in respect of all the other senses we fall below many
species of animals, in respect of touch we far excel all other species
in exactness of discrimination. That is why man is the most intelligent
of all animals. This is confirmed by the fact that it is to differences
in the organ of touch and to nothing else that the differences between
man and man in respect of natural endowment are due; men whose flesh
is hard are ill-endowed by nature, men whose flesh is soft, wellendowed.

As flavours may be divided into (a) sweet, (b) bitter, so with smells.
In some things the flavour and the smell have the same quality, i.e.
both are sweet or both bitter, in others they diverge. Similarly a
smell, like a flavour, may be pungent, astringent, acid, or succulent.
But, as we said, because smells are much less easy to discriminate
than flavours, the names of these varieties are applied to smells
only metaphorically; for example 'sweet' is extended from the taste
to the smell of saffron or honey, 'pungent' to that of thyme, and
so on. 

In the same sense in which hearing has for its object both the audible
and the inaudible, sight both the visible and the invisible, smell
has for its object both the odorous and the inodorous. 'Inodorous'
may be either (a) what has no smell at all, or (b) what has a small
or feeble smell. The same ambiguity lurks in the word 'tasteless'.

Smelling, like the operation of the senses previously examined, takes
place through a medium, i.e. through air or water-I add water, because
water-animals too (both sanguineous and non-sanguineous) seem to smell
just as much as land-animals; at any rate some of them make directly
for their food from a distance if it has any scent. That is why the
following facts constitute a problem for us. All animals smell in
the same way, but man smells only when he inhales; if he exhales or
holds his breath, he ceases to smell, no difference being made whether
the odorous object is distant or near, or even placed inside the nose
and actually on the wall of the nostril; it is a disability common
to all the senses not to perceive what is in immediate contact with
the organ of sense, but our failure to apprehend what is odorous without
the help of inhalation is peculiar (the fact is obvious on making
the experiment). Now since bloodless animals do not breathe, they
must, it might be argued, have some novel sense not reckoned among
the usual five. Our reply must be that this is impossible, since it
is scent that is perceived; a sense that apprehends what is odorous
and what has a good or bad odour cannot be anything but smell. Further,
they are observed to be deleteriously effected by the same strong
odours as man is, e.g. bitumen, sulphur, and the like. These animals
must be able to smell without being able to breathe. The probable
explanation is that in man the organ of smell has a certain superiority
over that in all other animals just as his eyes have over those of
hard-eyed animals. Man's eyes have in the eyelids a kind of shelter
or envelope, which must be shifted or drawn back in order that we
may see, while hardeyed animals have nothing of the kind, but at once
see whatever presents itself in the transparent medium. Similarly
in certain species of animals the organ of smell is like the eye of
hard-eyed animals, uncurtained, while in others which take in air
it probably has a curtain over it, which is drawn back in inhalation,
owing to the dilating of the veins or pores. That explains also why
such animals cannot smell under water; to smell they must first inhale,
and that they cannot do under water. 

Smells come from what is dry as flavours from what is moist. Consequently
the organ of smell is potentially dry. 

Part 10

What can be tasted is always something that can be touched, and just
for that reason it cannot be perceived through an interposed foreign
body, for touch means the absence of any intervening body. Further,
the flavoured and tasteable body is suspended in a liquid matter,
and this is tangible. Hence, if we lived in water, we should perceive
a sweet object introduced into the water, but the water would not
be the medium through which we perceived; our perception would be
due to the solution of the sweet substance in what we imbibed, just
as if it were mixed with some drink. There is no parallel here to
the perception of colour, which is due neither to any blending of
anything with anything, nor to any efflux of anything from anything.
In the case of taste, there is nothing corresponding to the medium
in the case of the senses previously discussed; but as the object
of sight is colour, so the object of taste is flavour. But nothing
excites a perception of flavour without the help of liquid; what acts
upon the sense of taste must be either actually or potentially liquid
like what is saline; it must be both (a) itself easily dissolved,
and (b) capable of dissolving along with itself the tongue. Taste
apprehends both (a) what has taste and (b) what has no taste, if we
mean by (b) what has only a slight or feeble flavour or what tends
to destroy the sense of taste. In this it is exactly parallel to sight,
which apprehends both what is visible and what is invisible (for darkness
is invisible and yet is discriminated by sight; so is, in a different
way, what is over brilliant), and to hearing, which apprehends both
sound and silence, of which the one is audible and the other inaudible,
and also over-loud sound. This corresponds in the case of hearing
to over-bright light in the case of sight. As a faint sound is 'inaudible',
so in a sense is a loud or violent sound. The word 'invisible' and
similar privative terms cover not only (a) what is simply without
some power, but also (b) what is adapted by nature to have it but
has not it or has it only in a very low degree, as when we say that
a species of swallow is 'footless' or that a variety of fruit is 'stoneless'.
So too taste has as its object both what can be tasted and the tasteless-the
latter in the sense of what has little flavour or a bad flavour or
one destructive of taste. The difference between what is tasteless
and what is not seems to rest ultimately on that between what is drinkable
and what is undrinkable both are tasteable, but the latter is bad
and tends to destroy taste, while the former is the normal stimulus
of taste. What is drinkable is the common object of both touch and
taste. 

Since what can be tasted is liquid, the organ for its perception cannot
be either (a) actually liquid or (b) incapable of becoming liquid.
Tasting means a being affected by what can be tasted as such; hence
the organ of taste must be liquefied, and so to start with must be
non-liquid but capable of liquefaction without loss of its distinctive
nature. This is confirmed by the fact that the tongue cannot taste
either when it is too dry or when it is too moist; in the latter case
what occurs is due to a contact with the pre-existent moisture in
the tongue itself, when after a foretaste of some strong flavour we
try to taste another flavour; it is in this way that sick persons
find everything they taste bitter, viz. because, when they taste,
their tongues are overflowing with bitter moisture. 

The species of flavour are, as in the case of colour, (a) simple,
i.e. the two contraries, the sweet and the bitter, (b) secondary,
viz. (i) on the side of the sweet, the succulent, (ii) on the side
of the bitter, the saline, (iii) between these come the pungent, the
harsh, the astringent, and the acid; these pretty well exhaust the
varieties of flavour. It follows that what has the power of tasting
is what is potentially of that kind, and that what is tasteable is
what has the power of making it actually what it itself already is.

Part 11

Whatever can be said of what is tangible, can be said of touch, and
vice versa; if touch is not a single sense but a group of senses,
there must be several kinds of what is tangible. It is a problem whether
touch is a single sense or a group of senses. It is also a problem,
what is the organ of touch; is it or is it not the flesh (including
what in certain animals is homologous with flesh)? On the second view,
flesh is 'the medium' of touch, the real organ being situated farther
inward. The problem arises because the field of each sense is according
to the accepted view determined as the range between a single pair
of contraries, white and black for sight, acute and grave for hearing,
bitter and sweet for taste; but in the field of what is tangible we
find several such pairs, hot cold, dry moist, hard soft, &c. This
problem finds a partial solution, when it is recalled that in the
case of the other senses more than one pair of contraries are to be
met with, e.g. in sound not only acute and grave but loud and soft,
smooth and rough, &c.; there are similar contrasts in the field of
colour. Nevertheless we are unable clearly to detect in the case of
touch what the single subject is which underlies the contrasted qualities
and corresponds to sound in the case of hearing. 

To the question whether the organ of touch lies inward or not (i.e.
whether we need look any farther than the flesh), no indication in
favour of the second answer can be drawn from the fact that if the
object comes into contact with the flesh it is at once perceived.
For even under present conditions if the experiment is made of making
a web and stretching it tight over the flesh, as soon as this web
is touched the sensation is reported in the same manner as before,
yet it is clear that the or is gan is not in this membrane. If the
membrane could be grown on to the flesh, the report would travel still
quicker. The flesh plays in touch very much the same part as would
be played in the other senses by an air-envelope growing round our
body; had we such an envelope attached to us we should have supposed
that it was by a single organ that we perceived sounds, colours, and
smells, and we should have taken sight, hearing, and smell to be a
single sense. But as it is, because that through which the different
movements are transmitted is not naturally attached to our bodies,
the difference of the various sense-organs is too plain to miss. But
in the case of touch the obscurity remains. 

There must be such a naturally attached 'medium' as flesh, for no
living body could be constructed of air or water; it must be something
solid. Consequently it must be composed of earth along with these,
which is just what flesh and its analogue in animals which have no
true flesh tend to be. Hence of necessity the medium through which
are transmitted the manifoldly contrasted tactual qualities must be
a body naturally attached to the organism. That they are manifold
is clear when we consider touching with the tongue; we apprehend at
the tongue all tangible qualities as well as flavour. Suppose all
the rest of our flesh was, like the tongue, sensitive to flavour,
we should have identified the sense of taste and the sense of touch;
what saves us from this identification is the fact that touch and
taste are not always found together in the same part of the body.
The following problem might be raised. Let us assume that every body
has depth, i.e. has three dimensions, and that if two bodies have
a third body between them they cannot be in contact with one another;
let us remember that what is liquid is a body and must be or contain
water, and that if two bodies touch one another under water, their
touching surfaces cannot be dry, but must have water between, viz.
the water which wets their bounding surfaces; from all this it follows
that in water two bodies cannot be in contact with one another. The
same holds of two bodies in air-air being to bodies in air precisely
what water is to bodies in water-but the facts are not so evident
to our observation, because we live in air, just as animals that live
in water would not notice that the things which touch one another
in water have wet surfaces. The problem, then, is: does the perception
of all objects of sense take place in the same way, or does it not,
e.g. taste and touch requiring contact (as they are commonly thought
to do), while all other senses perceive over a distance? The distinction
is unsound; we perceive what is hard or soft, as well as the objects
of hearing, sight, and smell, through a 'medium', only that the latter
are perceived over a greater distance than the former; that is why
the facts escape our notice. For we do perceive everything through
a medium; but in these cases the fact escapes us. Yet, to repeat what
we said before, if the medium for touch were a membrane separating
us from the object without our observing its existence, we should
be relatively to it in the same condition as we are now to air or
water in which we are immersed; in their case we fancy we can touch
objects, nothing coming in between us and them. But there remains
this difference between what can be touched and what can be seen or
can sound; in the latter two cases we perceive because the medium
produces a certain effect upon us, whereas in the perception of objects
of touch we are affected not by but along with the medium; it is as
if a man were struck through his shield, where the shock is not first
given to the shield and passed on to the man, but the concussion of
both is simultaneous. 

In general, flesh and the tongue are related to the real organs of
touch and taste, as air and water are to those of sight, hearing,
and smell. Hence in neither the one case nor the other can there be
any perception of an object if it is placed immediately upon the organ,
e.g. if a white object is placed on the surface of the eye. This again
shows that what has the power of perceiving the tangible is seated
inside. Only so would there be a complete analogy with all the other
senses. In their case if you place the object on the organ it is not
perceived, here if you place it on the flesh it is perceived; therefore
flesh is not the organ but the medium of touch. 

What can be touched are distinctive qualities of body as body; by
such differences I mean those which characterize the elements, viz,
hot cold, dry moist, of which we have spoken earlier in our treatise
on the elements. The organ for the perception of these is that of
touch-that part of the body in which primarily the sense of touch
resides. This is that part which is potentially such as its object
is actually: for all sense-perception is a process of being so affected;
so that that which makes something such as it itself actually is makes
the other such because the other is already potentially such. That
is why when an object of touch is equally hot and cold or hard and
soft we cannot perceive; what we perceive must have a degree of the
sensible quality lying beyond the neutral point. This implies that
the sense itself is a 'mean' between any two opposite qualities which
determine the field of that sense. It is to this that it owes its
power of discerning the objects in that field. What is 'in the middle'
is fitted to discern; relatively to either extreme it can put itself
in the place of the other. As what is to perceive both white and black
must, to begin with, be actually neither but potentially either (and
so with all the other sense-organs), so the organ of touch must be
neither hot nor cold. 

Further, as in a sense sight had for its object both what was visible
and what was invisible (and there was a parallel truth about all the
other senses discussed), so touch has for its object both what is
tangible and what is intangible. Here by 'intangible' is meant (a)
what like air possesses some quality of tangible things in a very
slight degree and (b) what possesses it in an excessive degree, as
destructive things do. 

We have now given an outline account of each of the several senses.

Part 12

The following results applying to any and every sense may now be formulated.

(A) By a 'sense' is meant what has the power of receiving into itself
the sensible forms of things without the matter. This must be conceived
of as taking place in the way in which a piece of wax takes on the
impress of a signet-ring without the iron or gold; we say that what
produces the impression is a signet of bronze or gold, but its particular
metallic constitution makes no difference: in a similar way the sense
is affected by what is coloured or flavoured or sounding, but it is
indifferent what in each case the substance is; what alone matters
is what quality it has, i.e. in what ratio its constituents are combined.

(B) By 'an organ of sense' is meant that in which ultimately such
a power is seated. 

The sense and its organ are the same in fact, but their essence is
not the same. What perceives is, of course, a spatial magnitude, but
we must not admit that either the having the power to perceive or
the sense itself is a magnitude; what they are is a certain ratio
or power in a magnitude. This enables us to explain why objects of
sense which possess one of two opposite sensible qualities in a degree
largely in excess of the other opposite destroy the organs of sense;
if the movement set up by an object is too strong for the organ, the
equipoise of contrary qualities in the organ, which just is its sensory
power, is disturbed; it is precisely as concord and tone are destroyed
by too violently twanging the strings of a lyre. This explains also
why plants cannot perceive. in spite of their having a portion of
soul in them and obviously being affected by tangible objects themselves;
for undoubtedly their temperature can be lowered or raised. The explanation
is that they have no mean of contrary qualities, and so no principle
in them capable of taking on the forms of sensible objects without
their matter; in the case of plants the affection is an affection
by form-and-matter together. The problem might be raised: Can what
cannot smell be said to be affected by smells or what cannot see by
colours, and so on? It might be said that a smell is just what can
be smelt, and if it produces any effect it can only be so as to make
something smell it, and it might be argued that what cannot smell
cannot be affected by smells and further that what can smell can be
affected by it only in so far as it has in it the power to smell (similarly
with the proper objects of all the other senses). Indeed that this
is so is made quite evident as follows. Light or darkness, sounds
and smells leave bodies quite unaffected; what does affect bodies
is not these but the bodies which are their vehicles, e.g. what splits
the trunk of a tree is not the sound of the thunder but the air which
accompanies thunder. Yes, but, it may be objected, bodies are affected
by what is tangible and by flavours. If not, by what are things that
are without soul affected, i.e. altered in quality? Must we not, then,
admit that the objects of the other senses also may affect them? Is
not the true account this, that all bodies are capable of being affected
by smells and sounds, but that some on being acted upon, having no
boundaries of their own, disintegrate, as in the instance of air,
which does become odorous, showing that some effect is produced on
it by what is odorous? But smelling is more than such an affection
by what is odorous-what more? Is not the answer that, while the air
owing to the momentary duration of the action upon it of what is odorous
does itself become perceptible to the sense of smell, smelling is
an observing of the result produced? 

----------------------------------------------------------------------

BOOK III

Part 1 

That there is no sixth sense in addition to the five enumerated-sight,
hearing, smell, taste, touch-may be established by the following considerations:

If we have actually sensation of everything of which touch can give
us sensation (for all the qualities of the tangible qua tangible are
perceived by us through touch); and if absence of a sense necessarily
involves absence of a sense-organ; and if (1) all objects that we
perceive by immediate contact with them are perceptible by touch,
which sense we actually possess, and (2) all objects that we perceive
through media, i.e. without immediate contact, are perceptible by
or through the simple elements, e.g. air and water (and this is so
arranged that (a) if more than one kind of sensible object is perceivable
through a single medium, the possessor of a sense-organ homogeneous
with that medium has the power of perceiving both kinds of objects;
for example, if the sense-organ is made of air, and air is a medium
both for sound and for colour; and that (b) if more than one medium
can transmit the same kind of sensible objects, as e.g. water as well
as air can transmit colour, both being transparent, then the possessor
of either alone will be able to perceive the kind of objects transmissible
through both); and if of the simple elements two only, air and water,
go to form sense-organs (for the pupil is made of water, the organ
of hearing is made of air, and the organ of smell of one or other
of these two, while fire is found either in none or in all-warmth
being an essential condition of all sensibility-and earth either in
none or, if anywhere, specially mingled with the components of the
organ of touch; wherefore it would remain that there can be no sense-organ
formed of anything except water and air); and if these sense-organs
are actually found in certain animals;-then all the possible senses
are possessed by those animals that are not imperfect or mutilated
(for even the mole is observed to have eyes beneath its skin); so
that, if there is no fifth element and no property other than those
which belong to the four elements of our world, no sense can be wanting
to such animals. 

Further, there cannot be a special sense-organ for the common sensibles
either, i.e. the objects which we perceive incidentally through this
or that special sense, e.g. movement, rest, figure, magnitude, number,
unity; for all these we perceive by movement, e.g. magnitude by movement,
and therefore also figure (for figure is a species of magnitude),
what is at rest by the absence of movement: number is perceived by
the negation of continuity, and by the special sensibles; for each
sense perceives one class of sensible objects. So that it is clearly
impossible that there should be a special sense for any one of the
common sensibles, e.g. movement; for, if that were so, our perception
of it would be exactly parallel to our present perception of what
is sweet by vision. That is so because we have a sense for each of
the two qualities, in virtue of which when they happen to meet in
one sensible object we are aware of both contemporaneously. If it
were not like this our perception of the common qualities would always
be incidental, i.e. as is the perception of Cleon's son, where we
perceive him not as Cleon's son but as white, and the white thing
which we really perceive happens to be Cleon's son. 

But in the case of the common sensibles there is already in us a general
sensibility which enables us to perceive them directly; there is therefore
no special sense required for their perception: if there were, our
perception of them would have been exactly like what has been above
described. 

The senses perceive each other's special objects incidentally; not
because the percipient sense is this or that special sense, but because
all form a unity: this incidental perception takes place whenever
sense is directed at one and the same moment to two disparate qualities
in one and the same object, e.g. to the bitterness and the yellowness
of bile, the assertion of the identity of both cannot be the act of
either of the senses; hence the illusion of sense, e.g. the belief
that if a thing is yellow it is bile. 

It might be asked why we have more senses than one. Is it to prevent
a failure to apprehend the common sensibles, e.g. movement, magnitude,
and number, which go along with the special sensibles? Had we no sense
but sight, and that sense no object but white, they would have tended
to escape our notice and everything would have merged for us into
an indistinguishable identity because of the concomitance of colour
and magnitude. As it is, the fact that the common sensibles are given
in the objects of more than one sense reveals their distinction from
each and all of the special sensibles. 

Part 2

Since it is through sense that we are aware that we are seeing or
hearing, it must be either by sight that we are aware of seeing, or
by some sense other than sight. But the sense that gives us this new
sensation must perceive both sight and its object, viz. colour: so
that either (1) there will be two senses both percipient of the same
sensible object, or (2) the sense must be percipient of itself. Further,
even if the sense which perceives sight were different from sight,
we must either fall into an infinite regress, or we must somewhere
assume a sense which is aware of itself. If so, we ought to do this
in the first case. 

This presents a difficulty: if to perceive by sight is just to see,
and what is seen is colour (or the coloured), then if we are to see
that which sees, that which sees originally must be coloured. It is
clear therefore that 'to perceive by sight' has more than one meaning;
for even when we are not seeing, it is by sight that we discriminate
darkness from light, though not in the same way as we distinguish
one colour from another. Further, in a sense even that which sees
is coloured; for in each case the sense-organ is capable of receiving
the sensible object without its matter. That is why even when the
sensible objects are gone the sensings and imaginings continue to
exist in the sense-organs. 

The activity of the sensible object and that of the percipient sense
is one and the same activity, and yet the distinction between their
being remains. Take as illustration actual sound and actual hearing:
a man may have hearing and yet not be hearing, and that which has
a sound is not always sounding. But when that which can hear is actively
hearing and which can sound is sounding, then the actual hearing and
the actual sound are merged in one (these one might call respectively
hearkening and sounding). 

If it is true that the movement, both the acting and the being acted
upon, is to be found in that which is acted upon, both the sound and
the hearing so far as it is actual must be found in that which has
the faculty of hearing; for it is in the passive factor that the actuality
of the active or motive factor is realized; that is why that which
causes movement may be at rest. Now the actuality of that which can
sound is just sound or sounding, and the actuality of that which can
hear is hearing or hearkening; 'sound' and 'hearing' are both ambiguous.
The same account applies to the other senses and their objects. For
as the-acting-and-being-acted-upon is to be found in the passive,
not in the active factor, so also the actuality of the sensible object
and that of the sensitive subject are both realized in the latter.
But while in some cases each aspect of the total actuality has a distinct
name, e.g. sounding and hearkening, in some one or other is nameless,
e.g. the actuality of sight is called seeing, but the actuality of
colour has no name: the actuality of the faculty of taste is called
tasting, but the actuality of flavour has no name. Since the actualities
of the sensible object and of the sensitive faculty are one actuality
in spite of the difference between their modes of being, actual hearing
and actual sounding appear and disappear from existence at one and
the same moment, and so actual savour and actual tasting, &c., while
as potentialities one of them may exist without the other. The earlier
students of nature were mistaken in their view that without sight
there was no white or black, without taste no savour. This statement
of theirs is partly true, partly false: 'sense' and 'the sensible
object' are ambiguous terms, i.e. may denote either potentialities
or actualities: the statement is true of the latter, false of the
former. This ambiguity they wholly failed to notice. 

If voice always implies a concord, and if the voice and the hearing
of it are in one sense one and the same, and if concord always implies
a ratio, hearing as well as what is heard must be a ratio. That is
why the excess of either the sharp or the flat destroys the hearing.
(So also in the case of savours excess destroys the sense of taste,
and in the case of colours excessive brightness or darkness destroys
the sight, and in the case of smell excess of strength whether in
the direction of sweetness or bitterness is destructive.) This shows
that the sense is a ratio. 

That is also why the objects of sense are (1) pleasant when the sensible
extremes such as acid or sweet or salt being pure and unmixed are
brought into the proper ratio; then they are pleasant: and in general
what is blended is more pleasant than the sharp or the flat alone;
or, to touch, that which is capable of being either warmed or chilled:
the sense and the ratio are identical: while (2) in excess the sensible
extremes are painful or destructive. 

Each sense then is relative to its particular group of sensible qualities:
it is found in a sense-organ as such and discriminates the differences
which exist within that group; e.g. sight discriminates white and
black, taste sweet and bitter, and so in all cases. Since we also
discriminate white from sweet, and indeed each sensible quality from
every other, with what do we perceive that they are different? It
must be by sense; for what is before us is sensible objects. (Hence
it is also obvious that the flesh cannot be the ultimate sense-organ:
if it were, the discriminating power could not do its work without
immediate contact with the object.) 

Therefore (1) discrimination between white and sweet cannot be effected
by two agencies which remain separate; both the qualities discriminated
must be present to something that is one and single. On any other
supposition even if I perceived sweet and you perceived white, the
difference between them would be apparent. What says that two things
are different must be one; for sweet is different from white. Therefore
what asserts this difference must be self-identical, and as what asserts,
so also what thinks or perceives. That it is not possible by means
of two agencies which remain separate to discriminate two objects
which are separate, is therefore obvious; and that (it is not possible
to do this in separate movements of time may be seen' if we look at
it as follows. For as what asserts the difference between the good
and the bad is one and the same, so also the time at which it asserts
the one to be different and the other to be different is not accidental
to the assertion (as it is for instance when I now assert a difference
but do not assert that there is now a difference); it asserts thus-both
now and that the objects are different now; the objects therefore
must be present at one and the same moment. Both the discriminating
power and the time of its exercise must be one and undivided.

But, it may be objected, it is impossible that what is self-identical
should be moved at me and the same time with contrary movements in
so far as it is undivided, and in an undivided moment of time. For
if what is sweet be the quality perceived, it moves the sense or thought
in this determinate way, while what is bitter moves it in a contrary
way, and what is white in a different way. Is it the case then that
what discriminates, though both numerically one and indivisible, is
at the same time divided in its being? In one sense, it is what is
divided that perceives two separate objects at once, but in another
sense it does so qua undivided; for it is divisible in its being but
spatially and numerically undivided. is not this impossible? For while
it is true that what is self-identical and undivided may be both contraries
at once potentially, it cannot be self-identical in its being-it must
lose its unity by being put into activity. It is not possible to be
at once white and black, and therefore it must also be impossible
for a thing to be affected at one and the same moment by the forms
of both, assuming it to be the case that sensation and thinking are
properly so described. 

The answer is that just as what is called a 'point' is, as being at
once one and two, properly said to be divisible, so here, that which
discriminates is qua undivided one, and active in a single moment
of time, while so far forth as it is divisible it twice over uses
the same dot at one and the same time. So far forth then as it takes
the limit as two' it discriminates two separate objects with what
in a sense is divided: while so far as it takes it as one, it does
so with what is one and occupies in its activity a single moment of
time. 

About the principle in virtue of which we say that animals are percipient,
let this discussion suffice. 

Part 3

There are two distinctive peculiarities by reference to which we characterize
the soul (1) local movement and (2) thinking, discriminating, and
perceiving. Thinking both speculative and practical is regarded as
akin to a form of perceiving; for in the one as well as the other
the soul discriminates and is cognizant of something which is. Indeed
the ancients go so far as to identify thinking and perceiving; e.g.
Empedocles says 'For 'tis in respect of what is present that man's
wit is increased', and again 'Whence it befalls them from time to
time to think diverse thoughts', and Homer's phrase 'For suchlike
is man's mind' means the same. They all look upon thinking as a bodily
process like perceiving, and hold that like is known as well as perceived
by like, as I explained at the beginning of our discussion. Yet they
ought at the same time to have accounted for error also; for it is
more intimately connected with animal existence and the soul continues
longer in the state of error than in that of truth. They cannot escape
the dilemma: either (1) whatever seems is true (and there are some
who accept this) or (2) error is contact with the unlike; for that
is the opposite of the knowing of like by like. 

But it is a received principle that error as well as knowledge in
respect to contraries is one and the same. 

That perceiving and practical thinking are not identical is therefore
obvious; for the former is universal in the animal world, the latter
is found in only a small division of it. Further, speculative thinking
is also distinct from perceiving-I mean that in which we find rightness
and wrongness-rightness in prudence, knowledge, true opinion, wrongness
in their opposites; for perception of the special objects of sense
is always free from error, and is found in all animals, while it is
possible to think falsely as well as truly, and thought is found only
where there is discourse of reason as well as sensibility. For imagination
is different from either perceiving or discursive thinking, though
it is not found without sensation, or judgement without it. That this
activity is not the same kind of thinking as judgement is obvious.
For imagining lies within our own power whenever we wish (e.g. we
can call up a picture, as in the practice of mnemonics by the use
of mental images), but in forming opinions we are not free: we cannot
escape the alternative of falsehood or truth. Further, when we think
something to be fearful or threatening, emotion is immediately produced,
and so too with what is encouraging; but when we merely imagine we
remain as unaffected as persons who are looking at a painting of some
dreadful or encouraging scene. Again within the field of judgement
itself we find varieties, knowledge, opinion, prudence, and their
opposites; of the differences between these I must speak elsewhere.

Thinking is different from perceiving and is held to be in part imagination,
in part judgement: we must therefore first mark off the sphere of
imagination and then speak of judgement. If then imagination is that
in virtue of which an image arises for us, excluding metaphorical
uses of the term, is it a single faculty or disposition relative to
images, in virtue of which we discriminate and are either in error
or not? The faculties in virtue of which we do this are sense, opinion,
science, intelligence. 

That imagination is not sense is clear from the following considerations:
Sense is either a faculty or an activity, e.g. sight or seeing: imagination
takes place in the absence of both, as e.g. in dreams. (Again, sense
is always present, imagination not. If actual imagination and actual
sensation were the same, imagination would be found in all the brutes:
this is held not to be the case; e.g. it is not found in ants or bees
or grubs. (Again, sensations are always true, imaginations are for
the most part false. (Once more, even in ordinary speech, we do not,
when sense functions precisely with regard to its object, say that
we imagine it to be a man, but rather when there is some failure of
accuracy in its exercise. And as we were saying before, visions appear
to us even when our eyes are shut. Neither is imagination any of the
things that are never in error: e.g. knowledge or intelligence; for
imagination may be false. 

It remains therefore to see if it is opinion, for opinion may be either
true or false. 

But opinion involves belief (for without belief in what we opine we
cannot have an opinion), and in the brutes though we often find imagination
we never find belief. Further, every opinion is accompanied by belief,
belief by conviction, and conviction by discourse of reason: while
there are some of the brutes in which we find imagination, without
discourse of reason. It is clear then that imagination cannot, again,
be (1) opinion plus sensation, or (2) opinion mediated by sensation,
or (3) a blend of opinion and sensation; this is impossible both for
these reasons and because the content of the supposed opinion cannot
be different from that of the sensation (I mean that imagination must
be the blending of the perception of white with the opinion that it
is white: it could scarcely be a blend of the opinion that it is good
with the perception that it is white): to imagine is therefore (on
this view) identical with the thinking of exactly the same as what
one in the strictest sense perceives. But what we imagine is sometimes
false though our contemporaneous judgement about it is true; e.g.
we imagine the sun to be a foot in diameter though we are convinced
that it is larger than the inhabited part of the earth, and the following
dilemma presents itself. Either (a while the fact has not changed
and the (observer has neither forgotten nor lost belief in the true
opinion which he had, that opinion has disappeared, or (b) if he retains
it then his opinion is at once true and false. A true opinion, however,
becomes false only when the fact alters without being noticed.

Imagination is therefore neither any one of the states enumerated,
nor compounded out of them. 

But since when one thing has been set in motion another thing may
be moved by it, and imagination is held to be a movement and to be
impossible without sensation, i.e. to occur in beings that are percipient
and to have for its content what can be perceived, and since movement
may be produced by actual sensation and that movement is necessarily
similar in character to the sensation itself, this movement must be
(1) necessarily (a) incapable of existing apart from sensation, (b)
incapable of existing except when we perceive, (such that in virtue
of its possession that in which it is found may present various phenomena
both active and passive, and (such that it may be either true or false.

The reason of the last characteristic is as follows. Perception (1)
of the special objects of sense is never in error or admits the least
possible amount of falsehood. (2) That of the concomitance of the
objects concomitant with the sensible qualities comes next: in this
case certainly we may be deceived; for while the perception that there
is white before us cannot be false, the perception that what is white
is this or that may be false. (3) Third comes the perception of the
universal attributes which accompany the concomitant objects to which
the special sensibles attach (I mean e.g. of movement and magnitude);
it is in respect of these that the greatest amount of sense-illusion
is possible. 

The motion which is due to the activity of sense in these three modes
of its exercise will differ from the activity of sense; (1) the first
kind of derived motion is free from error while the sensation is present;
(2) and (3) the others may be erroneous whether it is present or absent,
especially when the object of perception is far off. If then imagination
presents no other features than those enumerated and is what we have
described, then imagination must be a movement resulting from an actual
exercise of a power of sense. 

As sight is the most highly developed sense, the name Phantasia (imagination)
has been formed from Phaos (light) because it is not possible to see
without light. 

And because imaginations remain in the organs of sense and resemble
sensations, animals in their actions are largely guided by them, some
(i.e. the brutes) because of the non-existence in them of mind, others
(i.e. men) because of the temporary eclipse in them of mind by feeling
or disease or sleep. 

About imagination, what it is and why it exists, let so much suffice.

Part 4

Turning now to the part of the soul with which the soul knows and
thinks (whether this is separable from the others in definition only,
or spatially as well) we have to inquire (1) what differentiates this
part, and (2) how thinking can take place. 

If thinking is like perceiving, it must be either a process in which
the soul is acted upon by what is capable of being thought, or a process
different from but analogous to that. The thinking part of the soul
must therefore be, while impassible, capable of receiving the form
of an object; that is, must be potentially identical in character
with its object without being the object. Mind must be related to
what is thinkable, as sense is to what is sensible. 

Therefore, since everything is a possible object of thought, mind
in order, as Anaxagoras says, to dominate, that is, to know, must
be pure from all admixture; for the co-presence of what is alien to
its nature is a hindrance and a block: it follows that it too, like
the sensitive part, can have no nature of its own, other than that
of having a certain capacity. Thus that in the soul which is called
mind (by mind I mean that whereby the soul thinks and judges) is,
before it thinks, not actually any real thing. For this reason it
cannot reasonably be regarded as blended with the body: if so, it
would acquire some quality, e.g. warmth or cold, or even have an organ
like the sensitive faculty: as it is, it has none. It was a good idea
to call the soul 'the place of forms', though (1) this description
holds only of the intellective soul, and (2) even this is the forms
only potentially, not actually. 

Observation of the sense-organs and their employment reveals a distinction
between the impassibility of the sensitive and that of the intellective
faculty. After strong stimulation of a sense we are less able to exercise
it than before, as e.g. in the case of a loud sound we cannot hear
easily immediately after, or in the case of a bright colour or a powerful
odour we cannot see or smell, but in the case of mind thought about
an object that is highly intelligible renders it more and not less
able afterwards to think objects that are less intelligible: the reason
is that while the faculty of sensation is dependent upon the body,
mind is separable from it. 

Once the mind has become each set of its possible objects, as a man
of science has, when this phrase is used of one who is actually a
man of science (this happens when he is now able to exercise the power
on his own initiative), its condition is still one of potentiality,
but in a different sense from the potentiality which preceded the
acquisition of knowledge by learning or discovery: the mind too is
then able to think itself. 

Since we can distinguish between a spatial magnitude and what it is
to be such, and between water and what it is to be water, and so in
many other cases (though not in all; for in certain cases the thing
and its form are identical), flesh and what it is to be flesh are
discriminated either by different faculties, or by the same faculty
in two different states: for flesh necessarily involves matter and
is like what is snub-nosed, a this in a this. Now it is by means of
the sensitive faculty that we discriminate the hot and the cold, i.e.
the factors which combined in a certain ratio constitute flesh: the
essential character of flesh is apprehended by something different
either wholly separate from the sensitive faculty or related to it
as a bent line to the same line when it has been straightened out.

Again in the case of abstract objects what is straight is analogous
to what is snub-nosed; for it necessarily implies a continuum as its
matter: its constitutive essence is different, if we may distinguish
between straightness and what is straight: let us take it to be two-ness.
It must be apprehended, therefore, by a different power or by the
same power in a different state. To sum up, in so far as the realities
it knows are capable of being separated from their matter, so it is
also with the powers of mind. 

The problem might be suggested: if thinking is a passive affection,
then if mind is simple and impassible and has nothing in common with
anything else, as Anaxagoras says, how can it come to think at all?
For interaction between two factors is held to require a precedent
community of nature between the factors. Again it might be asked,
is mind a possible object of thought to itself? For if mind is thinkable
per se and what is thinkable is in kind one and the same, then either
(a) mind will belong to everything, or (b) mind will contain some
element common to it with all other realities which makes them all
thinkable. 

(1) Have not we already disposed of the difficulty about interaction
involving a common element, when we said that mind is in a sense potentially
whatever is thinkable, though actually it is nothing until it has
thought? What it thinks must be in it just as characters may be said
to be on a writingtablet on which as yet nothing actually stands written:
this is exactly what happens with mind. 

(Mind is itself thinkable in exactly the same way as its objects are.
For (a) in the case of objects which involve no matter, what thinks
and what is thought are identical; for speculative knowledge and its
object are identical. (Why mind is not always thinking we must consider
later., b) In the case of those which contain matter each of the
objects of thought is only potentially present. It follows that while
they will not have mind in them (for mind is a potentiality of them
only in so far as they are capable of being disengaged from matter)
mind may yet be thinkable. 

Part 5

Since in every class of things, as in nature as a whole, we find two
factors involved, (1) a matter which is potentially all the particulars
included in the class, (2) a cause which is productive in the sense
that it makes them all (the latter standing to the former, as e.g.
an art to its material), these distinct elements must likewise be
found within the soul. 

And in fact mind as we have described it is what it is what it is
by virtue of becoming all things, while there is another which is
what it is by virtue of making all things: this is a sort of positive
state like light; for in a sense light makes potential colours into
actual colours. 

Mind in this sense of it is separable, impassible, unmixed, since
it is in its essential nature activity (for always the active is superior
to the passive factor, the originating force to the matter which it
forms). 

Actual knowledge is identical with its object: in the individual,
potential knowledge is in time prior to actual knowledge, but in the
universe as a whole it is not prior even in time. Mind is not at one
time knowing and at another not. When mind is set free from its present
conditions it appears as just what it is and nothing more: this alone
is immortal and eternal (we do not, however, remember its former activity
because, while mind in this sense is impassible, mind as passive is
destructible), and without it nothing thinks. 

Part 6

The thinking then of the simple objects of thought is found in those
cases where falsehood is impossible: where the alternative of true
or false applies, there we always find a putting together of objects
of thought in a quasi-unity. As Empedocles said that 'where heads
of many a creature sprouted without necks' they afterwards by Love's
power were combined, so here too objects of thought which were given
separate are combined, e.g. 'incommensurate' and 'diagonal': if the
combination be of objects past or future the combination of thought
includes in its content the date. For falsehood always involves a
synthesis; for even if you assert that what is white is not white
you have included not white in a synthesis. It is possible also to
call all these cases division as well as combination. However that
may be, there is not only the true or false assertion that Cleon is
white but also the true or false assertion that he was or will he
white. In each and every case that which unifies is mind.

Since the word 'simple' has two senses, i.e. may mean either (a) 'not
capable of being divided' or (b) 'not actually divided', there is
nothing to prevent mind from knowing what is undivided, e.g. when
it apprehends a length (which is actually undivided) and that in an
undivided time; for the time is divided or undivided in the same manner
as the line. It is not possible, then, to tell what part of the line
it was apprehending in each half of the time: the object has no actual
parts until it has been divided: if in thought you think each half
separately, then by the same act you divide the time also, the half-lines
becoming as it were new wholes of length. But if you think it as a
whole consisting of these two possible parts, then also you think
it in a time which corresponds to both parts together. (But what is
not quantitatively but qualitatively simple is thought in a simple
time and by a simple act of the soul.) 

But that which mind thinks and the time in which it thinks are in
this case divisible only incidentally and not as such. For in them
too there is something indivisible (though, it may be, not isolable)
which gives unity to the time and the whole of length; and this is
found equally in every continuum whether temporal or spatial.

Points and similar instances of things that divide, themselves being
indivisible, are realized in consciousness in the same manner as privations.

A similar account may be given of all other cases, e.g. how evil or
black is cognized; they are cognized, in a sense, by means of their
contraries. That which cognizes must have an element of potentiality
in its being, and one of the contraries must be in it. But if there
is anything that has no contrary, then it knows itself and is actually
and possesses independent existence. 

Assertion is the saying of something concerning something, e.g. affirmation,
and is in every case either true or false: this is not always the
case with mind: the thinking of the definition in the sense of the
constitutive essence is never in error nor is it the assertion of
something concerning something, but, just as while the seeing of the
special object of sight can never be in error, the belief that the
white object seen is a man may be mistaken, so too in the case of
objects which are without matter. 

Part 7

Actual knowledge is identical with its object: potential knowledge
in the individual is in time prior to actual knowledge but in the
universe it has no priority even in time; for all things that come
into being arise from what actually is. In the case of sense clearly
the sensitive faculty already was potentially what the object makes
it to be actually; the faculty is not affected or altered. This must
therefore be a different kind from movement; for movement is, as we
saw, an activity of what is imperfect, activity in the unqualified
sense, i.e. that of what has been perfected, is different from movement.

To perceive then is like bare asserting or knowing; but when the object
is pleasant or painful, the soul makes a quasi-affirmation or negation,
and pursues or avoids the object. To feel pleasure or pain is to act
with the sensitive mean towards what is good or bad as such. Both
avoidance and appetite when actual are identical with this: the faculty
of appetite and avoidance are not different, either from one another
or from the faculty of sense-perception; but their being is different.

To the thinking soul images serve as if they were contents of perception
(and when it asserts or denies them to be good or bad it avoids or
pursues them). That is why the soul never thinks without an image.
The process is like that in which the air modifies the pupil in this
or that way and the pupil transmits the modification to some third
thing (and similarly in hearing), while the ultimate point of arrival
is one, a single mean, with different manners of being. 

With what part of itself the soul discriminates sweet from hot I have
explained before and must now describe again as follows: That with
which it does so is a sort of unity, but in the way just mentioned,
i.e. as a connecting term. And the two faculties it connects, being
one by analogy and numerically, are each to each as the qualities
discerned are to one another (for what difference does it make whether
we raise the problem of discrimination between disparates or between
contraries, e.g. white and black?). Let then C be to D as is to B:
it follows alternando that C: A:: D: B. If then C and D belong to
one subject, the case will be the same with them as with and B; and
B form a single identity with different modes of being; so too will
the former pair. The same reasoning holds if be sweet and B white.

The faculty of thinking then thinks the forms in the images, and as
in the former case what is to be pursued or avoided is marked out
for it, so where there is no sensation and it is engaged upon the
images it is moved to pursuit or avoidance. E.g.. perceiving by sense
that the beacon is fire, it recognizes in virtue of the general faculty
of sense that it signifies an enemy, because it sees it moving; but
sometimes by means of the images or thoughts which are within the
soul, just as if it were seeing, it calculates and deliberates what
is to come by reference to what is present; and when it makes a pronouncement,
as in the case of sensation it pronounces the object to be pleasant
or painful, in this case it avoids or persues and so generally in
cases of action. 

That too which involves no action, i.e. that which is true or false,
is in the same province with what is good or bad: yet they differ
in this, that the one set imply and the other do not a reference to
a particular person. 

The so-called abstract objects the mind thinks just as, if one had
thought of the snubnosed not as snub-nosed but as hollow, one would
have thought of an actuality without the flesh in which it is embodied:
it is thus that the mind when it is thinking the objects of Mathematics
thinks as separate elements which do not exist separate. In every
case the mind which is actively thinking is the objects which it thinks.
Whether it is possible for it while not existing separate from spatial
conditions to think anything that is separate, or not, we must consider
later. 

Part 8

Let us now summarize our results about soul, and repeat that the soul
is in a way all existing things; for existing things are either sensible
or thinkable, and knowledge is in a way what is knowable, and sensation
is in a way what is sensible: in what way we must inquire.

Knowledge and sensation are divided to correspond with the realities,
potential knowledge and sensation answering to potentialities, actual
knowledge and sensation to actualities. Within the soul the faculties
of knowledge and sensation are potentially these objects, the one
what is knowable, the other what is sensible. They must be either
the things themselves or their forms. The former alternative is of
course impossible: it is not the stone which is present in the soul
but its form. 

It follows that the soul is analogous to the hand; for as the hand
is a tool of tools, so the mind is the form of forms and sense the
form of sensible things. 

Since according to common agreement there is nothing outside and separate
in existence from sensible spatial magnitudes, the objects of thought
are in the sensible forms, viz. both the abstract objects and all
the states and affections of sensible things. Hence (1) no one can
learn or understand anything in the absence of sense, and (when the
mind is actively aware of anything it is necessarily aware of it along
with an image; for images are like sensuous contents except in that
they contain no matter. 

Imagination is different from assertion and denial; for what is true
or false involves a synthesis of concepts. In what will the primary
concepts differ from images? Must we not say that neither these nor
even our other concepts are images, though they necessarily involve
them? 

Part 9

The soul of animals is characterized by two faculties, (a) the faculty
of discrimination which is the work of thought and sense, and (b)
the faculty of originating local movement. Sense and mind we have
now sufficiently examined. Let us next consider what it is in the
soul which originates movement. Is it a single part of the soul separate
either spatially or in definition? Or is it the soul as a whole? If
it is a part, is that part different from those usually distinguished
or already mentioned by us, or is it one of them? The problem at once
presents itself, in what sense we are to speak of parts of the soul,
or how many we should distinguish. For in a sense there is an infinity
of parts: it is not enough to distinguish, with some thinkers, the
calculative, the passionate, and the desiderative, or with others
the rational and the irrational; for if we take the dividing lines
followed by these thinkers we shall find parts far more distinctly
separated from one another than these, namely those we have just mentioned:
(1) the nutritive, which belongs both to plants and to all animals,
and (2) the sensitive, which cannot easily be classed as either irrational
or rational; further (3) the imaginative, which is, in its being,
different from all, while it is very hard to say with which of the
others it is the same or not the same, supposing we determine to posit
separate parts in the soul; and lastly (4) the appetitive, which would
seem to be distinct both in definition and in power from all hitherto
enumerated. 

It is absurd to break up the last-mentioned faculty: as these thinkers
do, for wish is found in the calculative part and desire and passion
in the irrational; and if the soul is tripartite appetite will be
found in all three parts. Turning our attention to the present object
of discussion, let us ask what that is which originates local movement
of the animal. 

The movement of growth and decay, being found in all living things,
must be attributed to the faculty of reproduction and nutrition, which
is common to all: inspiration and expiration, sleep and waking, we
must consider later: these too present much difficulty: at present
we must consider local movement, asking what it is that originates
forward movement in the animal. 

That it is not the nutritive faculty is obvious; for this kind of
movement is always for an end and is accompanied either by imagination
or by appetite; for no animal moves except by compulsion unless it
has an impulse towards or away from an object. Further, if it were
the nutritive faculty, even plants would have been capable of originating
such movement and would have possessed the organs necessary to carry
it out. Similarly it cannot be the sensitive faculty either; for there
are many animals which have sensibility but remain fast and immovable
throughout their lives. 

If then Nature never makes anything without a purpose and never leaves
out what is necessary (except in the case of mutilated or imperfect
growths; and that here we have neither mutilation nor imperfection
may be argued from the facts that such animals (a) can reproduce their
species and (b) rise to completeness of nature and decay to an end),
it follows that, had they been capable of originating forward movement,
they would have possessed the organs necessary for that purpose. Further,
neither can the calculative faculty or what is called 'mind' be the
cause of such movement; for mind as speculative never thinks what
is practicable, it never says anything about an object to be avoided
or pursued, while this movement is always in something which is avoiding
or pursuing an object. No, not even when it is aware of such an object
does it at once enjoin pursuit or avoidance of it; e.g. the mind often
thinks of something terrifying or pleasant without enjoining the emotion
of fear. It is the heart that is moved (or in the case of a pleasant
object some other part). Further, even when the mind does command
and thought bids us pursue or avoid something, sometimes no movement
is produced; we act in accordance with desire, as in the case of moral
weakness. And, generally, we observe that the possessor of medical
knowledge is not necessarily healing, which shows that something else
is required to produce action in accordance with knowledge; the knowledge
alone is not the cause. Lastly, appetite too is incompetent to account
fully for movement; for those who successfully resist temptation have
appetite and desire and yet follow mind and refuse to enact that for
which they have appetite. 

Part 10

These two at all events appear to be sources of movement: appetite
and mind (if one may venture to regard imagination as a kind of thinking;
for many men follow their imaginations contrary to knowledge, and
in all animals other than man there is no thinking or calculation
but only imagination). 

Both of these then are capable of originating local movement, mind
and appetite: (1) mind, that is, which calculates means to an end,
i.e. mind practical (it differs from mind speculative in the character
of its end); while (2) appetite is in every form of it relative to
an end: for that which is the object of appetite is the stimulant
of mind practical; and that which is last in the process of thinking
is the beginning of the action. It follows that there is a justification
for regarding these two as the sources of movement, i.e. appetite
and practical thought; for the object of appetite starts a movement
and as a result of that thought gives rise to movement, the object
of appetite being it a source of stimulation. So too when imagination
originates movement, it necessarily involves appetite. 

That which moves therefore is a single faculty and the faculty of
appetite; for if there had been two sources of movement-mind and appetite-they
would have produced movement in virtue of some common character. As
it is, mind is never found producing movement without appetite (for
wish is a form of appetite; and when movement is produced according
to calculation it is also according to wish), but appetite can originate
movement contrary to calculation, for desire is a form of appetite.
Now mind is always right, but appetite and imagination may be either
right or wrong. That is why, though in any case it is the object of
appetite which originates movement, this object may be either the
real or the apparent good. To produce movement the object must be
more than this: it must be good that can be brought into being by
action; and only what can be otherwise than as it is can thus be brought
into being. That then such a power in the soul as has been described,
i.e. that called appetite, originates movement is clear. Those who
distinguish parts in the soul, if they distinguish and divide in accordance
with differences of power, find themselves with a very large number
of parts, a nutritive, a sensitive, an intellective, a deliberative,
and now an appetitive part; for these are more different from one
another than the faculties of desire and passion. 

Since appetites run counter to one another, which happens when a principle
of reason and a desire are contrary and is possible only in beings
with a sense of time (for while mind bids us hold back because of
what is future, desire is influenced by what is just at hand: a pleasant
object which is just at hand presents itself as both pleasant and
good, without condition in either case, because of want of foresight
into what is farther away in time), it follows that while that which
originates movement must be specifically one, viz. the faculty of
appetite as such (or rather farthest back of all the object of that
faculty; for it is it that itself remaining unmoved originates the
movement by being apprehended in thought or imagination), the things
that originate movement are numerically many. 

All movement involves three factors, (1) that which originates the
movement, (2) that by means of which it originates it, and (3) that
which is moved. The expression 'that which originates the movement'
is ambiguous: it may mean either (a) something which itself is unmoved
or (b) that which at once moves and is moved. Here that which moves
without itself being moved is the realizable good, that which at once
moves and is moved is the faculty of appetite (for that which is influenced
by appetite so far as it is actually so influenced is set in movement,
and appetite in the sense of actual appetite is a kind of movement),
while that which is in motion is the animal. The instrument which
appetite employs to produce movement is no longer psychical but bodily:
hence the examination of it falls within the province of the functions
common to body and soul. To state the matter summarily at present,
that which is the instrument in the production of movement is to be
found where a beginning and an end coincide as e.g. in a ball and
socket joint; for there the convex and the concave sides are respectively
an end and a beginning (that is why while the one remains at rest,
the other is moved): they are separate in definition but not separable
spatially. For everything is moved by pushing and pulling. Hence just
as in the case of a wheel, so here there must be a point which remains
at rest, and from that point the movement must originate.

To sum up, then, and repeat what I have said, inasmuch as an animal
is capable of appetite it is capable of self-movement; it is not capable
of appetite without possessing imagination; and all imagination is
either (1) calculative or (2) sensitive. In the latter an animals,
and not only man, partake. 

Part 11

We must consider also in the case of imperfect animals, sc. those
which have no sense but touch, what it is that in them originates
movement. Can they have imagination or not? or desire? Clearly they
have feelings of pleasure and pain, and if they have these they must
have desire. But how can they have imagination? Must not we say that,
as their movements are indefinite, they have imagination and desire,
but indefinitely? 

Sensitive imagination, as we have said, is found in all animals, deliberative
imagination only in those that are calculative: for whether this or
that shall be enacted is already a task requiring calculation; and
there must be a single standard to measure by, for that is pursued
which is greater. It follows that what acts in this way must be able
to make a unity out of several images. 

This is the reason why imagination is held not to involve opinion,
in that it does not involve opinion based on inference, though opinion
involves imagination. Hence appetite contains no deliberative element.
Sometimes it overpowers wish and sets it in movement: at times wish
acts thus upon appetite, like one sphere imparting its movement to
another, or appetite acts thus upon appetite, i.e. in the condition
of moral weakness (though by nature the higher faculty is always more
authoritative and gives rise to movement). Thus three modes of movement
are possible. 

The faculty of knowing is never moved but remains at rest. Since the
one premiss or judgement is universal and the other deals with the
particular (for the first tells us that such and such a kind of man
should do such and such a kind of act, and the second that this is
an act of the kind meant, and I a person of the type intended), it
is the latter opinion that really originates movement, not the universal;
or rather it is both, but the one does so while it remains in a state
more like rest, while the other partakes in movement. 

Part 12

The nutritive soul then must be possessed by everything that is alive,
and every such thing is endowed with soul from its birth to its death.
For what has been born must grow, reach maturity, and decay-all of
which are impossible without nutrition. Therefore the nutritive faculty
must be found in everything that grows and decays. 

But sensation need not be found in all things that live. For it is
impossible for touch to belong either (1) to those whose body is uncompounded
or (2) to those which are incapable of taking in the forms without
their matter. 

But animals must be endowed with sensation, since Nature does nothing
in vain. For all things that exist by Nature are means to an end,
or will be concomitants of means to an end. Every body capable of
forward movement would, if unendowed with sensation, perish and fail
to reach its end, which is the aim of Nature; for how could it obtain
nutriment? Stationary living things, it is true, have as their nutriment
that from which they have arisen; but it is not possible that a body
which is not stationary but produced by generation should have a soul
and a discerning mind without also having sensation. (Nor yet even
if it were not produced by generation. Why should it not have sensation?
Because it were better so either for the body or for the soul? But
clearly it would not be better for either: the absence of sensation
will not enable the one to think better or the other to exist better.)
Therefore no body which is not stationary has soul without sensation.

But if a body has sensation, it must be either simple or compound.
And simple it cannot be; for then it could not have touch, which is
indispensable. This is clear from what follows. An animal is a body
with soul in it: every body is tangible, i.e. perceptible by touch;
hence necessarily, if an animal is to survive, its body must have
tactual sensation. All the other senses, e.g. smell, sight, hearing,
apprehend through media; but where there is immediate contact the
animal, if it has no sensation, will be unable to avoid some things
and take others, and so will find it impossible to survive. That is
why taste also is a sort of touch; it is relative to nutriment, which
is just tangible body; whereas sound, colour, and odour are innutritious,
and further neither grow nor decay. Hence it is that taste also must
be a sort of touch, because it is the sense for what is tangible and
nutritious. 

Both these senses, then, are indispensable to the animal, and it is
clear that without touch it is impossible for an animal to be. All
the other senses subserve well-being and for that very reason belong
not to any and every kind of animal, but only to some, e.g. those
capable of forward movement must have them; for, if they are to survive,
they must perceive not only by immediate contact but also at a distance
from the object. This will be possible if they can perceive through
a medium, the medium being affected and moved by the perceptible object,
and the animal by the medium. just as that which produces local movement
causes a change extending to a certain point, and that which gave
an impulse causes another to produce a new impulse so that the movement
traverses a medium the first mover impelling without being impelled,
the last moved being impelled without impelling, while the medium
(or media, for there are many) is both-so is it also in the case of
alteration, except that the agent produces produces it without the
patient's changing its place. Thus if an object is dipped into wax,
the movement goes on until submersion has taken place, and in stone
it goes no distance at all, while in water the disturbance goes far
beyond the object dipped: in air the disturbance is propagated farthest
of all, the air acting and being acted upon, so long as it maintains
an unbroken unity. That is why in the case of reflection it is better,
instead of saying that the sight issues from the eye and is reflected,
to say that the air, so long as it remains one, is affected by the
shape and colour. On a smooth surface the air possesses unity; hence
it is that it in turn sets the sight in motion, just as if the impression
on the wax were transmitted as far as the wax extends. 

Part 13

It is clear that the body of an animal cannot be simple, i.e. consist
of one element such as fire or air. For without touch it is impossible
to have any other sense; for every body that has soul in it must,
as we have said, be capable of touch. All the other elements with
the exception of earth can constitute organs of sense, but all of
them bring about perception only through something else, viz. through
the media. Touch takes place by direct contact with its objects, whence
also its name. All the other organs of sense, no doubt, perceive by
contact, only the contact is mediate: touch alone perceives by immediate
contact. Consequently no animal body can consist of these other elements.

Nor can it consist solely of earth. For touch is as it were a mean
between all tangible qualities, and its organ is capable of receiving
not only all the specific qualities which characterize earth, but
also the hot and the cold and all other tangible qualities whatsoever.
That is why we have no sensation by means of bones, hair, &c., because
they consist of earth. So too plants, because they consist of earth,
have no sensation. Without touch there can be no other sense, and
the organ of touch cannot consist of earth or of any other single
element. 

It is evident, therefore, that the loss of this one sense alone must
bring about the death of an animal. For as on the one hand nothing
which is not an animal can have this sense, so on the other it is
the only one which is indispensably necessary to what is an animal.
This explains, further, the following difference between the other
senses and touch. In the case of all the others excess of intensity
in the qualities which they apprehend, i.e. excess of intensity in
colour, sound, and smell, destroys not the but only the organs of
the sense (except incidentally, as when the sound is accompanied by
an impact or shock, or where through the objects of sight or of smell
certain other things are set in motion, which destroy by contact);
flavour also destroys only in so far as it is at the same time tangible.
But excess of intensity in tangible qualities, e.g. heat, cold, or
hardness, destroys the animal itself. As in the case of every sensible
quality excess destroys the organ, so here what is tangible destroys
touch, which is the essential mark of life; for it has been shown
that without touch it is impossible for an animal to be. That is why
excess in intensity of tangible qualities destroys not merely the
organ, but the animal itself, because this is the only sense which
it must have. 

All the other senses are necessary to animals, as we have said, not
for their being, but for their well-being. Such, e.g. is sight, which,
since it lives in air or water, or generally in what is pellucid,
it must have in order to see, and taste because of what is pleasant
or painful to it, in order that it may perceive these qualities in
its nutriment and so may desire to be set in motion, and hearing that
it may have communication made to it, and a tongue that it may communicate
with its fellows. 

THE END

% chapter soul (end)
% \chapter{Ethics} % (fold)
\label{cha:ethics}

THE ETHICS OF ARISTOTLE


INTRODUCTION

The _Ethics_ of Aristotle is one half of a single treatise of which his
_Politics_ is the other half. Both deal with one and the same subject.
This subject is what Aristotle calls in one place the "philosophy of
human affairs;" but more frequently Political or Social Science. In the
two works taken together we have their author's whole theory of human
conduct or practical activity, that is, of all human activity which
is not directed merely to knowledge or truth. The two parts of this
treatise are mutually complementary, but in a literary sense each
is independent and self-contained. The proem to the _Ethics_ is an
introduction to the whole subject, not merely to the first part; the
last chapter of the _Ethics_ points forward to the _Politics_, and
sketches for that part of the treatise the order of enquiry to be
pursued (an order which in the actual treatise is not adhered to).

The principle of distribution of the subject-matter between the two
works is far from obvious, and has been much debated. Not much can be
gathered from their titles, which in any case were not given to them by
their author. Nor do these titles suggest any very compact unity in the
works to which they are applied: the plural forms, which survive so
oddly in English (Ethic_s_, Politic_s_), were intended to indicate the
treatment within a single work of a _group_ of connected questions. The
unity of the first group arises from their centring round the topic of
character, that of the second from their connection with the existence
and life of the city or state. We have thus to regard the _Ethics_ as
dealing with one group of problems and the _Politics_ with a second,
both falling within the wide compass of Political Science. Each of these
groups falls into sub-groups which roughly correspond to the several
books in each work. The tendency to take up one by one the various
problems which had suggested themselves in the wide field obscures both
the unity of the subject-matter and its proper articulation. But it is
to be remembered that what is offered us is avowedly rather an enquiry
than an exposition of hard and fast doctrine.

Nevertheless each work aims at a relative completeness, and it is
important to observe the relation of each to the other. The distinction
is not that the one treats of Moral and the other of Political
Philosophy, nor again that the one deals with the moral activity of the
individual and the other with that of the State, nor once more that the
one gives us the theory of human conduct, while the other discusses its
application in practice, though not all of these misinterpretations are
equally erroneous. The clue to the right interpretation is given by
Aristotle himself, where in the last chapter of the _Ethics_ he is
paving the way for the _Politics_. In the _Ethics_ he has not confined
himself to the abstract or isolated individual, but has always thought
of him, or we might say, in his social and political context, with a
given nature due to race and heredity and in certain surroundings.
So viewing him he has studied the nature and formation of his
character--all that he can make himself or be made by others to be.
Especially he has investigated the various admirable forms of human
character and the mode of their production. But all this, though it
brings more clearly before us what goodness or virtue is, and how it is
to be reached, remains mere theory or talk. By itself it does not
enable us to become, or to help others to become, good. For this it is
necessary to bring into play the great force of the Political Community
or State, of which the main instrument is Law. Hence arises the demand
for the necessary complement to the _Ethics, i.e._, a treatise devoted
to the questions which centre round the enquiry; by what organisation
of social or political forces, by what laws or institutions can we best
secure the greatest amount of good character?

We must, however, remember that the production of good character is not
the end of either individual or state action: that is the aim of the one
and the other because good character is the indispensable condition and
chief determinant of happiness, itself the goal of all human doing. The
end of all action, individual or collective, is the greatest happiness
of the greatest number. There is, Aristotle insists, no difference of
kind between the good of one and the good of many or all. The sole
difference is one of amount or scale. This does not mean simply that the
State exists to secure in larger measure the objects of degree which the
isolated individual attempts, but is too feeble, to secure without it.
On the contrary, it rather insists that whatever goods society alone
enables a man to secure have always had to the individual--whether he
realised it or not--the value which, when so secured, he recognises them
to possess. The best and happiest life for the individual is that which
the State renders possible, and this it does mainly by revealing to him
the value of new objects of desire and educating him to appreciate them.
To Aristotle or to Plato the State is, above all, a large and powerful
educative agency which gives the individual increased opportunities of
self-development and greater capacities for the enjoyment of life.

Looking forward, then, to the life of the State as that which aids
support, and combines the efforts of the individual to obtain happiness,
Aristotle draws no hard and fast distinction between the spheres of
action of Man as individual and Man as citizen. Nor does the division of
his discussion into the _Ethics_ and the _Politics_ rest upon any such
distinction. The distinction implied is rather between two stages in the
life of the civilised man--the stage of preparation for the full life of
the adult citizen, and the stage of the actual exercise or enjoyment of
citizenship. Hence the _Ethics_, where his attention is directed upon
the formation of character, is largely and centrally a treatise on Moral
Education. It discusses especially those admirable human qualities which
fit a man for life in an organised civic community, which makes him "a
good citizen," and considers how they can be fostered or created and
their opposites prevented.

This is the kernel of the _Ethics_, and all the rest is subordinate to
this main interest and purpose. Yet "the rest" is not irrelevant; the
whole situation in which character grows and operates is concretely
conceived. There is a basis of what we should call Psychology, sketched
in firm outlines, the deeper presuppositions and the wider issues of
human character and conduct are not ignored, and there is no little of
what we should call Metaphysics. But neither the Psychology nor the
Metaphysics is elaborated, and only so much is brought forward as
appears necessary to put the main facts in their proper perspective
and setting. It is this combination of width of outlook with close
observation of the concrete facts of conduct which gives its abiding
value to the work, and justifies the view of it as containing
Aristotle's Moral Philosophy. Nor is it important merely as summing up
the moral judgments and speculations of an age now long past. It seizes
and dwells upon those elements and features in human practice which are
most essential and permanent, and it is small wonder that so much in it
survives in our own ways of regarding conduct and speaking of it. Thus
it still remains one of the classics of Moral Philosophy, nor is its
value likely soon to be exhausted.

As was pointed out above, the proem (Book I., cc. i-iii.) is a prelude
to the treatment of the whole subject covered by the _Ethics_ and the
_Politics_ together. It sets forth the purpose of the enquiry, describes
the spirit in which it is to be undertaken and what ought to be the
expectation of the reader, and lastly states the necessary conditions
of studying it with profit. The aim of it is the acquisition and
propagation of a certain kind of knowledge (science), but this knowledge
and the thinking which brings it about are subsidiary to a practical
end. The knowledge aimed at is of what is best for man and of the
conditions of its realisation. Such knowledge is that which in its
consumate form we find in great statesmen, enabling them to organise and
administer their states and regulate by law the life of the citizens
to their advantage and happiness, but it is the same kind of knowledge
which on a smaller scale secures success in the management of the family
or of private life.

It is characteristic of such knowledge that it should be deficient
in "exactness," in precision of statement, and closeness of logical
concatenation. We must not look for a mathematics of conduct. The
subject-matter of Human Conduct is not governed by necessary and uniform
laws. But this does not mean that it is subject to no laws. There
are general principles at work in it, and these can be formulated in
"rules," which rules can be systematised or unified. It is all-important
to remember that practical or moral rules are only general and always
admit of exceptions, and that they arise not from the mere complexity
of the facts, but from the liability of the facts to a certain
unpredictable variation. At their very best, practical rules state
probabilities, not certainties; a relative constancy of connection is
all that exists, but it is enough to serve as a guide in life. Aristotle
here holds the balance between a misleading hope of reducing the
subject-matter of conduct to a few simple rigorous abstract principles,
with conclusions necessarily issuing from them, and the view that it is
the field of operation of inscrutable forces acting without predictable
regularity. He does not pretend to find in it absolute uniformities, or
to deduce the details from his principles. Hence, too, he insists on the
necessity of experience as the source or test of all that he has to
say. Moral experience--the actual possession and exercise of good
character--is necessary truly to understand moral principles and
profitably to apply them. The mere intellectual apprehension of them is
not possible, or if possible, profitless.

The _Ethics_ is addressed to students who are presumed both to have
enough general education to appreciate these points, and also to have a
solid foundation of good habits. More than that is not required for the
profitable study of it.

If the discussion of the nature and formation of character be regarded
as the central topic of the _Ethics_, the contents of Book I., cc.
iv.-xii. may be considered as still belonging to the introduction and
setting, but these chapters contain matter of profound importance and
have exercised an enormous influence upon subsequent thought. They lay
down a principle which governs all Greek thought about human life, viz.
that it is only intelligible when viewed as directed towards some end or
good. This is the Greek way of expressing that all human life involves
an ideal element--something which it is not yet and which under certain
conditions it is to be. In that sense Greek Moral Philosophy is
essentially idealistic. Further it is always assumed that all human
practical activity is directed or "oriented" to a _single_ end, and that
that end is knowable or definable in advance of its realisation. To know
it is not merely a matter of speculative interest, it is of the highest
practical moment for only in the light of it can life be duly guided,
and particularly only so can the state be properly organised and
administered. This explains the stress laid throughout by Greek Moral
Philosophy upon the necessity of knowledge as a condition of the best
life. This knowledge is not, though it includes knowledge of the nature
of man and his circumstances, it is knowledge of what is best--of man's
supreme end or good.

But this end is not conceived as presented to him by a superior power
nor even as something which _ought_ to be. The presentation of the Moral
Ideal as Duty is almost absent. From the outset it is identified with
the object of desire, of what we not merely judge desirable but actually
do desire, or that which would, if realised, satisfy human desire. In
fact it is what we all, wise and simple, agree in naming "Happiness"
(Welfare or Well-being)

In what then does happiness consist? Aristotle summarily sets aside the
more or less popular identifications of it with abundance of physical
pleasures, with political power and honour, with the mere possession of
such superior gifts or attainments as normally entitle men to these,
with wealth. None of these can constitute the end or good of man as
such. On the other hand, he rejects his master Plato's conception of a
good which is the end of the whole universe, or at least dismisses it
as irrelevant to his present enquiry. The good towards which all human
desires and practical activities are directed must be one conformable to
man's special nature and circumstances and attainable by his efforts.
There is in Aristotle's theory of human conduct no trace of Plato's
"other worldliness", he brings the moral ideal in Bacon's phrase down to
"right earth"--and so closer to the facts and problems of actual human
living. Turning from criticism of others he states his own positive view
of Happiness, and, though he avowedly states it merely in outline his
account is pregnant with significance. Human Happiness lies in activity
or energising, and that in a way peculiar to man with his given nature
and his given circumstances, it is not theoretical, but practical: it is
the activity not of reason but still of a being who possesses reason and
applies it, and it presupposes in that being the development, and
not merely the natural possession, of certain relevant powers and
capacities. The last is the prime condition of successful living
and therefore of satisfaction, but Aristotle does not ignore other
conditions, such as length of life, wealth and good luck, the absence or
diminution of which render happiness not impossible, but difficult of
attainment.

It is interesting to compare this account of Happiness with Mill's
in _Utilitarianism_. Mill's is much the less consistent: at times
he distinguishes and at times he identifies, happiness, pleasure,
contentment, and satisfaction. He wavers between belief in its general
attainability and an absence of hopefulness. He mixes up in an arbitrary
way such ingredients as "not expecting more from life than it is capable
of bestowing," "mental cultivation," "improved laws," etc., and in fact
leaves the whole conception vague, blurred, and uncertain. Aristotle
draws the outline with a firmer hand and presents a more definite ideal.
He allows for the influence on happiness of conditions only partly, if
at all, within the control of man, but he clearly makes the man positive
determinant of man's happiness he in himself, and more particularly
in what he makes directly of his own nature, and so indirectly of his
circumstances. "'Tis in ourselves that we are thus or thus" But once
more this does not involve an artificial or abstract isolation of the
individual moral agent from his relation to other persons or things from
his context in society and nature, nor ignore the relative dependence of
his life upon a favourable environment.

The main factor which determines success or failure in human life is the
acquisition of certain powers, for Happiness is just the exercise or
putting forth of these in actual living, everything else is secondary
and subordinate. These powers arise from the due development of certain
natural aptitudes which belong (in various degrees) to human nature as
such and therefore to all normal human beings. In their developed
form they are known as virtues (the Greek means simply "goodnesses,"
"perfections," "excellences," or "fitnesses"), some of them are
physical, but others are psychical, and among the latter some, and these
distinctively or peculiarly human, are "rational," _i e_, presuppose the
possession and exercise of mind or intelligence. These last fall into
two groups, which Aristotle distinguishes as Goodnesses of Intellect and
Goodnesses of Character. They have in common that they all excite in us
admiration and praise of their possessors, and that they are not natural
endowments, but acquired characteristics But they differ in important
ways. (1) the former are excellences or developed powers of the
reason as such--of that in us which sees and formulates laws, rules,
regularities systems, and is content in the vision of them, while the
latter involve a submission or obedience to such rules of something
in us which is in itself capricious and irregular, but capable of
regulation, viz our instincts and feelings, (2) the former are acquired
by study and instruction, the latter by discipline. The latter
constitute "character," each of them as a "moral virtue" (literally "a
goodness of character"), and upon them primarily depends the realisation
of happiness. This is the case at least for the great majority of men,
and for all men their possession is an indispensable basis of the
best, _i e_, the most desirable life. They form the chief or central
subject-matter of the _Ethics_.

Perhaps the truest way of conceiving Aristotle's meaning here is to
regard a moral virtue as a form of obedience to a maxim or rule of
conduct accepted by the agent as valid for a class of recurrent
situations in human life. Such obedience requires knowledge of the rule
and acceptance of it _as the rule_ of the agent's own actions, but not
necessarily knowledge of its ground or of its systematic connexion with
other similarly known and similarly accepted rules (It may be remarked
that the Greek word usually translated "reason," means in almost all
cases in the _Ethics_ such a rule, and not the faculty which apprehends,
formulates, considers them).

The "moral virtues and vices" make up what we call character, and the
important questions arise: (1) What is character? and (2) How is it
formed? (for character in this sense is not a natural endowment; it is
formed or produced). Aristotle deals with these questions in the reverse
order. His answers are peculiar and distinctive--not that they are
absolutely novel (for they are anticipated in Plato), but that by him
they are for the first time distinctly and clearly formulated.

(1.) Character, good or bad, is produced by what Aristotle calls
"habituation," that is, it is the result of the repeated doing of acts
which have a similar or common quality. Such repetition acting upon
natural aptitudes or propensities gradually fixes them in one or other
of two opposite directions, giving them a bias towards good or evil.
Hence the several acts which determine goodness or badness of character
must be done in a certain way, and thus the formation of good character
requires discipline and direction from without. Not that the agent
himself contributes nothing to the formation of his character, but that
at first he needs guidance. The point is not so much that the process
cannot be safely left to Nature, but that it cannot be entrusted to
merely intellectual instruction. The process is one of assimilation,
largely by imitation and under direction and control. The result is a
growing understanding of what is done, a choice of it for its own sake,
a fixity and steadiness of purpose. Right acts and feelings become,
through habit, easier and more pleasant, and the doing of them a "second
nature." The agent acquires the power of doing them freely, willingly,
more and more "of himself."

But what are "right" acts? In the first place, they are those that
conform to a rule--to the right rule, and ultimately to reason. The
Greeks never waver from the conviction that in the end moral conduct is
essentially reasonable conduct. But there is a more significant way of
describing their "rightness," and here for the first time Aristotle
introduces his famous "Doctrine of the Mean." Reasoning from the analogy
of "right" physical acts, he pronounces that rightness always means
adaptation or adjustment to the special requirements of a situation. To
this adjustment he gives a quantitative interpretation. To do (or to
feel) what is right in a given situation is to do or to feel just the
amount required--neither more nor less: to do wrong is to do or to
feel too much or too little--to fall short of or over-shoot, "a mean"
determined by the situation. The repetition of acts which lie in the
mean is the cause of the formation of each and every "goodness of
character," and for this "rules" can be given.

(2) What then is a "moral virtue," the result of such a process duly
directed? It is no mere mood of feeling, no mere liability to emotion,
no mere natural aptitude or endowment, it is a permanent _state_ of the
agent's self, or, as we might in modern phrase put it, of his will,
it consists in a steady self-imposed obedience to a rule of action
in certain situations which frequently recur in human life. The rule
prescribes the control and regulation within limits of the agent's
natural impulses to act and feel thus and thus. The situations fall into
groups which constitute the "fields" of the several "moral virtues",
for each there is a rule, conformity to which secures rightness in
the individual acts. Thus the moral ideal appears as a code of
rules, accepted by the agent, but as yet _to him_ without rational
justification and without system or unity. But the rules prescribe no
mechanical uniformity: each within its limits permits variety, and the
exactly right amount adopted to the requirements of the individual
situation (and every actual situation is individual) must be determined
by the intuition of the moment. There is no attempt to reduce the rich
possibilities of right action to a single monotonous type. On the
contrary, there are acknowledged to be many forms of moral virtue, and
there is a long list of them, with their correlative vices enumerated.

The Doctrine of the Mean here takes a form in which it has impressed
subsequent thinkers, but which has less importance than is usually
ascribed to it. In the "Table of the Virtues and Vices," each of the
virtues is flanked by two opposite vices, which are respectively the
excess and defect of that which in due measure constitutes the virtue.
Aristotle tries to show that this is the case in regard to every virtue
named and recognised as such, but his treatment is often forced and the
endeavour is not very successful. Except as a convenient principle
of arrangement of the various forms of praiseworthy or blameworthy
characters, generally acknowledged as such by Greek opinion, this form
of the doctrine is of no great significance.

Books III-V are occupied with a survey of the moral virtues and vices.
These seem to have been undertaken in order to verify in detail the
general account, but this aim is not kept steadily in view. Nor is there
any well-considered principle of classification. What we find is a sort
of portrait-gallery of the various types of moral excellence which
the Greeks of the author's age admired and strove to encourage. The
discussion is full of acute, interesting and sometimes profound
observations. Some of the types are those which are and will be admired
at all times, but others are connected with peculiar features of Greek
life which have now passed away. The most important is that of Justice
or the Just Man, to which we may later return. But the discussion is
preceded by an attempt to elucidate some difficult and obscure points in
the general account of moral virtue and action (Book III, cc i-v). This
section is concerned with the notion of Responsibility. The discussion
designedly excludes what we may call the metaphysical issues of the
problem, which here present themselves, it moves on the level of thought
of the practical man, the statesman, and the legislator. Coercion and
ignorance of relevant circumstances render acts involuntary and exempt
their doer from responsibility, otherwise the act is voluntary and the
agent responsible, choice or preference of what is done, and inner
consent to the deed, are to be presumed. Neither passion nor ignorance
of the right rule can extenuate responsibility. But there is a
difference between acts done voluntarily and acts done of _set_ choice
or purpose. The latter imply Deliberation. Deliberation involves
thinking, thinking out means to ends: in deliberate acts the whole
nature of the agent consents to and enters into the act, and in a
peculiar sense they are his, they _are_ him in action, and the most
significant evidence of what he is. Aristotle is unable wholly to avoid
allusion to the metaphysical difficulties and what he does here say upon
them is obscure and unsatisfactory. But he insists upon the importance
in moral action of the agent's inner consent, and on the reality of his
individual responsibility. For his present purpose the metaphysical
difficulties are irrelevant.

The treatment of Justice in Book V has always been a source of great
difficulty to students of the _Ethics_. Almost more than any other part
of the work it has exercised influence upon mediaeval and modern thought
upon the subject. The distinctions and divisions have become part of the
stock-in-trade of would be philosophic jurists. And yet, oddly enough,
most of these distinctions have been misunderstood and the whole purport
of the discussion misconceived. Aristotle is here dealing with justice
in a restricted sense viz as that special goodness of character which
is required of every adult citizen and which can be produced by early
discipline or habituation. It is the temper or habitual attitude
demanded of the citizen for the due exercise of his functions as taking
part in the administration of the civic community--as a member of the
judicature and executive. The Greek citizen was only exceptionally, and
at rare intervals if ever, a law-maker while at any moment he might
be called upon to act as a judge (juryman or arbitrator) or as an
administrator. For the work of a legislator far more than the moral
virtue of justice or fairmindedness was necessary, these were requisite
to the rarer and higher "intellectual virtue" of practical wisdom. Then
here, too, the discussion moves on a low level, and the raising of
fundamental problems is excluded. Hence "distributive justice" is
concerned not with the large question of the distribution of political
power and privileges among the constituent members or classes of the
state but with the smaller questions of the distribution among those of
casual gains and even with the division among private claimants of a
common fund or inheritance, while "corrective justice" is concerned
solely with the management of legal redress. The whole treatment is
confused by the unhappy attempt to give a precise mathematical form to
the principles of justice in the various fields distinguished. Still it
remains an interesting first endeavour to give greater exactness to some
of the leading conceptions of jurisprudence.

Book VI appears to have in view two aims: (1) to describe goodness of
intellect and discover its highest form or forms; (2) to show how this
is related to goodness of character, and so to conduct generally. As all
thinking is either theoretical or practical, goodness of intellect has
_two_ supreme forms--Theoretical and Practical Wisdom. The first, which
apprehends the eternal laws of the universe, has no direct relation to
human conduct: the second is identical with that master science of human
life of which the whole treatise, consisting of the _Ethics_ and the
_Politics_, is an exposition. It is this science which supplies the
right rules of conduct Taking them as they emerge in and from practical
experience, it formulates them more precisely and organises them into a
system where they are all seen to converge upon happiness. The mode in
which such knowledge manifests itself is in the power to show that such
and such rules of action follow from the very nature of the end or good
for man. It presupposes and starts from a clear conception of the end
and the wish for it as conceived, and it proceeds by a deduction which
is dehberation writ large. In the man of practical wisdom this process
has reached its perfect result, and the code of right rules is
apprehended as a system with a single principle and so as something
wholly rational or reasonable He has not on each occasion to seek and
find the right rule applicable to the situation, he produces it at
once from within himself, and can at need justify it by exhibiting its
rationale, _i.e._ , its connection with the end. This is the consummate
form of reason applied to conduct, but there are minor forms of it, less
independent or original, but nevertheless of great value, such as the
power to think out the proper cause of policy in novel circumstances or
the power to see the proper line of treatment to follow in a court of
law.

The form of the thinking which enters into conduct is that which
terminates in the production of a rule which declares some means to the
end of life. The process presupposes _(a)_ a clear and just apprehension
of the nature of that end--such as the _Ethics_ itself endeavours to
supply; _(b)_ a correct perception of the conditions of action, _(a)_ at
least is impossible except to a man whose character has been duly formed
by discipline; it arises only in a man who has acquired moral virtue.
For such action and feeling as forms bad character, blinds the eye of
the soul and corrupts the moral principle, and the place of practical
wisdom is taken by that parody of itself which Aristotle calls
"cleverness"--the "wisdom" of the unscrupulous man of the world. Thus
true practical wisdom and true goodness of character are interdependent;
neither is genuinely possible or "completely" present without the other.
This is Aristotle's contribution to the discussion of the question, so
central in Greek Moral Philosophy, of the relation of the intellectual
and the passionate factors in conduct.

Aristotle is not an intuitionist, but he recognises the implication in
conduct of a direct and immediate apprehension both of the end and of
the character of his circumstances under which it is from moment to
moment realised. The directness of such apprehension makes it analogous
to sensation or sense-perception; but it is on his view in the end due
to the existence or activity in man of that power in him which is the
highest thing in his nature, and akin to or identical with the divine
nature--mind, or intelligence. It is this which reveals to us what is
best for us--the ideal of a happiness which is the object of our real
wish and the goal of all our efforts. But beyond and above the practical
ideal of what is best _for man_ begins to show itself another and still
higher ideal--that of a life not distinctively human or in a narrow
sense practical, yet capable of being participated in by man even under
the actual circumstances of this world. For a time, however, this
further and higher ideal is ignored.

The next book (Book VII.), is concerned partly with moral conditions, in
which the agent seems to rise above the level of moral virtue or fall
below that of moral vice, but partly and more largely with conditions in
which the agent occupies a middle position between the two. Aristotle's
attention is here directed chiefly towards the phenomena of
"Incontinence," weakness of will or imperfect self-control. This
condition was to the Greeks a matter of only too frequent experience,
but it appeared to them peculiarly difficult to understand. How can a
man know what is good or best for him, and yet chronically fail to act
upon his knowledge? Socrates was driven to the paradox of denying the
possibility, but the facts are too strong for him. Knowledge of the
right rule may be present, nay the rightfulness of its authority may be
acknowledged, and yet time after time it may be disobeyed; the will may
be good and yet overmastered by the force of desire, so that the act
done is contrary to the agent's will. Nevertheless the act may be the
agent's, and the will therefore divided against itself. Aristotle is
aware of the seriousness and difficulty of the problem, but in spite of
the vividness with which he pictures, and the acuteness with which he
analyses, the situation in which such action occurs, it cannot be said
that he solves the problem. It is time that he rises above the abstract
view of it as a conflict between reason and passion, recognising that
passion is involved in the knowledge which in conduct prevails or is
overborne, and that the force which leads to the wrong act is not blind
or ignorant passion, but always has some reason in it. But he tends to
lapse back into the abstraction, and his final account is perplexed and
obscure. He finds the source of the phenomenon in the nature of the
desire for bodily pleasures, which is not irrational but has something
rational in it. Such pleasures are not necessarily or inherently bad, as
has sometimes been maintained; on the contrary, they are good, but only
in certain amounts or under certain conditions, so that the will is
often misled, hesitates, and is lost.

Books VIII. and IX. (on Friendship) are almost an interruption of the
argument. The subject-matter of them was a favourite topic of ancient
writers, and the treatment is smoother and more orderly than elsewhere
in the _Ethics_. The argument is clear, and may be left without
comment to the readers. These books contain a necessary and attractive
complement to the somewhat dry account of Greek morality in the
preceding books, and there are in them profound reflections on what may
be called the metaphysics of friendship or love.

At the beginning of Book X. we return to the topic of Pleasure, which
is now regarded from a different point of view. In Book VII. the
antagonists were those who over-emphasised the irrationality or badness
of Pleasure: here it is rather those who so exaggerate its value as to
confuse or identify it with the good or Happiness. But there is offered
us in this section much more than criticism of the errors of others.
Answers are given both to the psychological question, "What is
Pleasure?" and to the ethical question, "What is its value?" Pleasure,
we are told, is the natural concomitant and index of perfect activity,
distinguishable but inseparable from it--"the activity of a subject at
its best acting upon an object at its best." It is therefore always
and in itself a good, but its value rises and falls with that of the
activity with which it is conjoined, and which it intensifies and
perfects. Hence it follows that the highest and best pleasures are those
which accompany the highest and best activity.

Pleasure is, therefore, a necessary element in the best life, but it is
not the whole of it nor the principal ingredient. The value of a life
depends upon the nature and worth of the activity which it involves;
given the maximum of full free action, the maximum of pleasure necessary
follows. But on what sort of life is such activity possible? This leads
us back to the question, What is happiness? In what life can man find
the fullest satisfaction for his desires? To this question Aristotle
gives an answer which cannot but surprise us after what has preceded.
True Happiness, great satisfaction, cannot be found by man in any form
of "practical" life, no, not in the fullest and freest exercise possible
of the "moral virtues," not in the life of the citizen or of the
great soldier or statesman. To seek it there is to court failure and
disappointment. It is to be found in the life of the onlooker, the
disinterested spectator; or, to put it more distinctly, "in the life of
the philosopher, the life of scientific and philosophic contemplation."
The highest and most satisfying form of life possible to man is "the
contemplative life"; it is only in a secondary sense and for those
incapable of their life, that the practical or moral ideal is the best.
It is time that such a life is not distinctively human, but it is the
privilege of man to partake in it, and such participation, at however
rare intervals and for however short a period, is the highest Happiness
which human life can offer. All other activities have value only because
and in so far as they render _this_ life possible.

But it must not be forgotten that Aristotle conceives of this life as
one of intense activity or energising: it is just this which gives it
its supremacy. In spite of the almost religious fervour with which he
speaks of it ("the most orthodox of his disciples" paraphrases his
meaning by describing its content as "the service and vision of God"),
it is clear that he identified it with the life of the philosopher, as
he understood it, a life of ceaseless intellectual activity in which at
least at times all the distractions and disturbances inseparable from
practical life seemed to disappear and become as nothing. This ideal was
partly an inheritance from the more ardent idealism of his master Plato,
but partly it was the expression of personal experience.

The nobility of this ideal cannot be questioned; the conception of the
end of man or a life lived for truth--of a life blissfully absorbed in
the vision of truth--is a lofty and inspiring one. But we cannot resist
certain criticisms upon its presentation by Aristotle: (1) the relation
of it to the lower ideal of practice is left somewhat obscure; (2) it is
described in such a way as renders its realisation possible only to a
gifted few, and under exceptional circumstances; (3) it seems in various
ways, as regards its content, to be unnecessarily and unjustifiably
limited. But it must be borne in mind that this is a first endeavour to
determine its principle, and that similar failures have attended the
attempts to describe the "religious" or the "spiritual" ideals of
life, which have continually been suggested by the apparently inherent
limitations of the "practical" or "moral" life, which is the subject of
Moral Philosophy.

The Moral Ideal to those who have most deeply reflected on it leads
to the thought of an Ideal beyond and above it, which alone gives it
meaning, but which seems to escape from definite conception by man.
The richness and variety of this Ideal ceaselessly invite, but as
ceaselessly defy, our attempts to imprison it in a definite formula or
portray it in detailed imagination. Yet the thought of it is and remains
inexpungable from our minds.

This conception of the best life is not forgotten in the _Politics_ The
end of life in the state is itself well-living and well-doing--a life
which helps to produce the best life The great agency in the production
of such life is the State operating through Law, which is Reason backed
by Force. For its greatest efficiency there is required the development
of a science of legislation. The main drift of what he says here is that
the most desirable thing would be that the best reason of the community
should be embodied in its laws. But so far as that is not possible, it
still is true that anyone who would make himself and others better must
become a miniature legislator--must study the general principles of law,
morality, and education. The conception of [Grek: politikae] with which
he opened the _Ethics_ would serve as a guide to a father educating his
children as well as to the legislator legislating for the state. Finding
in his predecessors no developed doctrine on this subject, Aristotle
proposes himself to undertake the construction of it, and sketches in
advance the programme of the _Politics_ in the concluding sentence of
the _Ethics_ His ultimate object is to answer the questions, What is the
best form of Polity, how should each be constituted, and what laws and
customs should it adopt and employ? Not till this answer is given will
"the philosophy of human affairs" be complete.

On looking back it will be seen that the discussion of the central topic
of the nature and formation of character has expanded into a Philosophy
of Human Conduct, merging at its beginning and end into metaphysics
The result is a Moral Philosophy set against a background of Political
Theory and general Philosophy. The most characteristic features of this
Moral Philosophy are due to the fact of its essentially teleological
view of human life and action: (1) Every human activity, but especially
every human practical activity, is directed towards a simple End
discoverable by reflection, and this End is conceived of as the object
of universal human desire, as something to be enjoyed, not as something
which ought to be done or enacted. Anstotle's Moral Philosophy is not
hedonistic but it is eudæmomstic, the end is the enjoyment of Happiness,
not the fulfilment of Duty. (2) Every human practical activity derives
its value from its efficiency as a means to that end, it is good or bad,
right or wrong, as it conduces or fails to conduce to Happiness Thus his
Moral Philosophy is essentially utilitarian or prudential Right action
presupposes Thought or Thinking, partly on the development of a clearer
and distincter conception of the end of desire, partly as the deduction
from that of rules which state the normally effective conditions of
its realisation. The thinking involved in right conduct is
calculation--calculation of means to an end fixed by nature and
foreknowable Action itself is at its best just the realisation of a
scheme preconceived and thought out beforehand, commending itself by its
inherent attractiveness or promise of enjoyment.

This view has the great advantage of exhibiting morality as essentially
reasonable, but the accompanying disadvantage of lowering it into a
somewhat prosaic and unideal Prudentialism, nor is it saved from this
by the tacking on to it, by a sort of after-thought, of the second and
higher Ideal--an addition which ruins the coherence of the account
without really transmuting its substance The source of our
dissatisfaction with the whole theory lies deeper than in its tendency
to identify the end with the maximum of enjoyment or satisfaction, or to
regard the goodness or badness of acts and feelings as lying solely in
their efficacy to produce such a result It arises from the application
to morality of the distinction of means and end For this distinction,
for all its plausibility and usefulness in ordinary thought and speech,
cannot finally be maintained In morality--and this is vital to its
character--everything is both means and end, and so neither in
distinction or separation, and all thinking about it which presupposes
the finality of this distinction wanders into misconception and error.
The thinking which really matters in conduct is not a thinking which
imaginatively forecasts ideals which promise to fulfil desire, or
calculates means to their attainment--that is sometimes useful,
sometimes harmful, and always subordinate, but thinking which reveals
to the agent the situation in which he is to act, both, that is, the
universal situation on which as man he always and everywhere stands,
and the ever-varying and ever-novel situation in which he as this
individual, here and now, finds himself. In such knowledge of given
or historic fact lie the natural determinants of his conduct, in such
knowledge alone lies the condition of his freedom and his good.

But this does not mean that Moral Philosophy has not still much to
learn from Aristotle's _Ethics_. The work still remains one of the best
introductions to a study of its important subject-matter, it spreads
before us a view of the relevant facts, it reduces them to manageable
compass and order, it raises some of the central problems, and makes
acute and valuable suggestions towards their solution. Above all, it
perpetually incites to renewed and independent reflection upon them.

J. A. SMITH


  The following is a list of the works of Aristotle:--

  First edition of works (with omission of Rhetorica, Poetica, and
  second book of Economica), 5 vols by Aldus Manutius, Venice, 1495 8,
  re impression supervised by Erasmus and with certain corrections by
  Grynaeus (including Rhetorica and Poetica), 1531, 1539, revised 1550,
  later editions were followed by that of Immanuel Bekker and Brandis
  (Greek and Latin), 5 vols. The 5th vol contains the Index by Bomtz,
  1831-70, Didot edition (Greek and Latin), 5 vols 1848 74

  ENGLISH TRANSLATIONS Edited by T Taylor, with Porphyry's
  Introduction, 9 vols, 1812, under editorship of J A Smith and
  W D Ross, II vols, 1908-31, Loeb editions Ethica, Rhetorica,
  Poetica, Physica, Politica, Metaphysica, 1926-33

  Later editions of separate works
  _De Anima_ Torstrik, 1862, Trendelenburg, 2nd edition, 1877,
  with English translation, L Wallace, 1882, Biehl, 1884, 1896, with
  English, R D Hicks, 1907
_Ethica_ J S Brewer (Nicomachean), 1836, W E Jelf, 1856, J F T Rogers,
1865, A Grant, 1857 8, 1866, 1874, 1885, E Moore, 1871, 1878, 4th
edition, 1890, Ramsauer (Nicomachean), 1878, Susemihl, 1878, 1880,
revised by O Apelt, 1903, A Grant, 1885, I Bywater (Nicomachean), 1890,
J Burnet, 1900

_Historia Animalium_ Schneider, 1812, Aubert and Wimmer, 1860;
Dittmeyer, 1907

_Metaphysica_ Schwegler, 1848, W Christ, 1899

_Organon_ Waitz, 1844 6

_Poetica_ Vahlen, 1867, 1874, with Notes by E Moore, 1875, with English
translation by E R Wharton, 1883, 1885, Uberweg, 1870, 1875, with
German translation, Susemihl, 1874, Schmidt, 1875, Christ, 1878, I
Bywater, 1898, T G Tucker, 1899

_De Republica Athenientium_ Text and facsimile of Papyrus, F G Kenyon,
1891, 3rd edition, 1892, Kaibel and Wilamowitz-Moellendorf, 1891, 3rd
edition, 1898, Van Herwerden and Leeuwen (from Kenyon's text), 1891,
Blass, 1892, 1895, 1898, 1903, J E Sandys, 1893

_Politica_ Susemihl, 1872, with German, 1878, 3rd edition, 1882,
Susemihl and Hicks, 1894, etc, O Immisch, 1909

_Physica_ C Prantl, 1879

_Rhetorica_ Stahr, 1862, Sprengel (with Latin text), 1867, Cope and
Sandys, 1877, Roemer, 1885, 1898

ENGLISH TRANSLATIONS OF ONE OR MORE WORKS De Anima (with Parva
Naturalia), by W A Hammond, 1902 Ethica Of Morals to Nicomachus, by
E Pargiter, 1745, with Politica by J Gillies, 1797, 1804, 1813, with
Rhetorica and Poetica, by T Taylor, 1818, and later editions Nicomachean
Ethics, 1819, mainly from text of Bekker by D P Chase, 1847, revised
1861, and later editions, with an introductory essay by G H Lewes
(Camelot Classics) 1890, re-edited by J M Mitchell (New Universal
Library), 1906, 1910, by R W Browne (Bohn's Classical Library),
1848, etc, by R Williams, 1869, 1876, by W M Hatch and others (with
translation of paraphrase attributed to Andronicus of Rhodes), edited
by E Hatch, 1879 by F H Peters, 1881, J E C Welldon, 1892, J Gillies
(Lubbock's Hundred Books) 1893 Historia Animalium, by R Creswell (Bonn's
Classical Library) 1848, with Treatise on Physiognomy, by T Taylor,
1809 Metaphysica, by T Taylor, 1801, by J H M Mahon (Bohn's Classical
Library), 1848 Organon, with Porphyry's Introduction, by O F Owen
(Bohn's Classical Library), 1848 Posterior Analytics, E Poste, 1850, E S
Bourchier, 1901, On Fallacies, E Poste, 1866 Parva Naturaha (Greek and
English), by G R T Ross, 1906, with De Anima, by W A Hammond, 1902 Youth
and Old Age, Life and Death and Respiration, W Ogle 1897 Poetica, with
Notes from the French of D Acier, 1705, by H J Pye, 1788, 1792, T
Twining, 1789, 1812, with Preface and Notes by H Hamilton, 1851,
Treatise on Rhetorica and Poetica, by T Hobbes (Bohn's Classical
Library), 1850, by Wharton, 1883 (see Greek version), S H Butcher, 1895,
1898, 3rd edition, 1902, E S Bourchier, 1907, by Ingram Bywater, 1909 De
Partibus Animalium, W Ogle, 1882 De Republica Athenientium, by E Poste,
1891, F G Kenyon, 1891, T J Dymes, 1891 De Virtutibus et Vitus, by W
Bridgman, 1804 Politica, from the French of Regius, 1598, by W Ellis,
1776, 1778, 1888 (Morley's Universal Library), 1893 (Lubbock's Hundred
Books) by E Walford (with Æconomics, and Life by Dr Gillies), (Bohn's
Classical Library), 1848, J E. C. Welldon, 1883, B Jowett, 1885, with
Introduction and Index by H W C Davis, 1905, Books i iii iv (vii)
from Bekker's text by W E Bolland, with Introduction by A Lang, 1877.
Problemata (with writings of other philosophers), 1597, 1607, 1680,
1684, etc. Rhetorica, A summary by T Hobbes, 1655 (?), new edition,
1759, by the translators of the Art of Thinking, 1686, 1816, by D M
Crimmin, 1812, J Gillies, 1823, Anon 1847, J E C Welldon, 1886, R C
Jebb, with Introduction and Supplementary Notes by J E Sandys, 1909 (see
under Poetica and Ethica). Secreta Secretorum (supposititious work),
Anon 1702, from the Hebrew version by M Gaster, 1907, 1908. Version by
Lydgate and Burgh, edited by R Steele (E E T S), 1894, 1898.

LIFE, ETC J W Blakesley, 1839, A Crichton (Jardine's Naturalist's
Library), 1843, JS Blackie, Four Phases of Morals, Socrates, Aristotle,
etc, 1871, G Grote, Aristotle, edited by A Bain and G C Robertson, 1872,
1880, E Wallace, Outlines of the Philosophy of Aristotle, 1875, 1880,
A Grant (Ancient Classics for English readers), 1877, T Davidson,
Aristotle and Ancient Educational Ideals (Great Educators), 1892, F
Sewall, Swedenborg and Aristotle, 1895, W A Heidel, The Necessary
and the Contingent of the Aristotelian System (University of Chicago
Contributions to Philosophy), 1896, F W Bain, On the Realisation of the
Possible, and the Spirit of Aristotle, 1899, J H Hyslop, The Ethics of
the Greek Philosophers, etc (Evolution of Ethics), 1903, M V Williams,
Six Essays on the Platonic Theory of Knowledge as expounded in the later
dialogues and reviewed by Aristotle, 1908, J M Watson, Aristotle's
Criticism of Plato, 1909 A E Taylor, Aristotle, 1919, W D Ross,
Aristotle, 1923.




ARISTOTLE'S ETHICS




BOOK I


Every art, and every science reduced to a teachable form, and in like
manner every action and moral choice, aims, it is thought, at some good:
for which reason a common and by no means a bad description of the Chief
Good is, "that which all things aim at."

Now there plainly is a difference in the Ends proposed: for in some
cases they are acts of working, and in others certain works or tangible
results beyond and beside the acts of working: and where there are
certain Ends beyond and beside the actions, the works are in their
nature better than the acts of working. Again, since actions and arts
and sciences are many, the Ends likewise come to be many: of the healing
art, for instance, health; of the ship-building art, a vessel; of
the military art, victory; and of domestic management, wealth; are
respectively the Ends.

And whatever of such actions, arts, or sciences range under some one
faculty (as under that of horsemanship the art of making bridles, and
all that are connected with the manufacture of horse-furniture in
general; this itself again, and every action connected with war, under
the military art; and in the same way others under others), in all such,
the Ends of the master-arts are more choice-worthy than those ranging
under them, because it is with a view to the former that the latter are
pursued.

(And in this comparison it makes no difference whether the acts of
working are themselves the Ends of the actions, or something further
beside them, as is the case in the arts and sciences we have been just
speaking of.)

[Sidenote: II] Since then of all things which may be done there is some
one End which we desire for its own sake, and with a view to which we
desire everything else; and since we do not choose in all instances with
a further End in view (for then men would go on without limit, and so
the desire would be unsatisfied and fruitless), this plainly must be the
Chief Good, _i.e._ the best thing of all.

Surely then, even with reference to actual life and conduct, the
knowledge of it must have great weight; and like archers, with a mark in
view, we shall be more likely to hit upon what is right: and if so, we
ought to try to describe, in outline at least, what it is and of which
of the sciences and faculties it is the End.

[Sidenote: 1094b] Now one would naturally suppose it to be the End
of that which is most commanding and most inclusive: and to this
description, [Greek: _politikae_] plainly answers: for this it is that
determines which of the sciences should be in the communities, and which
kind individuals are to learn, and what degree of proficiency is to be
required. Again; we see also ranging under this the most highly esteemed
faculties, such as the art military, and that of domestic management,
and Rhetoric. Well then, since this uses all the other practical
sciences, and moreover lays down rules as to what men are to do, and
from what to abstain, the End of this must include the Ends of the rest,
and so must be _The Good_ of Man. And grant that this is the same to
the individual and to the community, yet surely that of the latter is
plainly greater and more perfect to discover and preserve: for to do
this even for a single individual were a matter for contentment; but to
do it for a whole nation, and for communities generally, were more noble
and godlike.


[Sidenote: III] Such then are the objects proposed by our treatise,
which is of the nature of [Greek: _politikae_]: and I conceive I shall
have spoken on them satisfactorily, if they be made as distinctly clear
as the nature of the subject-matter will admit: for exactness must not
be looked for in all discussions alike, any more than in all works
of handicraft. Now the notions of nobleness and justice, with the
examination of which _politikea_ is concerned, admit of variation
and error to such a degree, that they are supposed by some to exist
conventionally only, and not in the nature of things: but then, again,
the things which are allowed to be goods admit of a similar error,
because harm cornes to many from them: for before now some have perished
through wealth, and others through valour.

We must be content then, in speaking of such things and from such data,
to set forth the truth roughly and in outline; in other words, since
we are speaking of general matter and from general data, to draw also
conclusions merely general. And in the same spirit should each person
receive what we say: for the man of education will seek exactness so far
in each subject as the nature of the thing admits, it being plainly much
the same absurdity to put up with a mathematician who tries to persuade
instead of proving, and to demand strict demonstrative reasoning of a
Rhetorician.

[Sidenote: 1095a] Now each man judges well what he knows, and of these
things he is a good judge: on each particular matter then he is a good
judge who has been instructed in _it_, and in a general way the man of
general mental cultivation.

Hence the young man is not a fit student of Moral Philosophy, for he has
no experience in the actions of life, while all that is said presupposes
and is concerned with these: and in the next place, since he is apt to
follow the impulses of his passions, he will hear as though he heard
not, and to no profit, the end in view being practice and not mere
knowledge.

And I draw no distinction between young in years, and youthful in temper
and disposition: the defect to which I allude being no direct result of
the time, but of living at the beck and call of passion, and following
each object as it rises. For to them that are such the knowledge comes
to be unprofitable, as to those of imperfect self-control: but, to
those who form their desires and act in accordance with reason, to have
knowledge on these points must be very profitable.

Let thus much suffice by way of preface on these three points, the
student, the spirit in which our observations should be received, and
the object which we propose.

[Sidenote: IV] And now, resuming the statement with which we commenced,
since all knowledge and moral choice grasps at good of some kind or
another, what good is that which we say [Greek: _politikai_] aims at?
or, in other words, what is the highest of all the goods which are the
objects of action?

So far as name goes, there is a pretty general agreement: for HAPPINESS
both the multitude and the refined few call it, and "living well" and
"doing well" they conceive to be the same with "being happy;" but about
the Nature of this Happiness, men dispute, and the multitude do not in
their account of it agree with the wise. For some say it is some one of
those things which are palpable and apparent, as pleasure or wealth or
honour; in fact, some one thing, some another; nay, oftentimes the same
man gives a different account of it; for when ill, he calls it health;
when poor, wealth: and conscious of their own ignorance, men admire
those who talk grandly and above their comprehension. Some again held it
to be something by itself, other than and beside these many good things,
which is in fact to all these the cause of their being good.

Now to sift all the opinions would be perhaps rather a fruitless task;
so it shall suffice to sift those which are most generally current, or
are thought to have some reason in them.

[Sidenote: 1095b] And here we must not forget the difference between
reasoning from principles, and reasoning to principles: for with good
cause did Plato too doubt about this, and inquire whether the right road
is from principles or to principles, just as in the racecourse from the
judges to the further end, or _vice versâ_.

Of course, we must begin with what is known; but then this is of two
kinds, what we _do_ know, and what we _may_ know: perhaps then as
individuals we must begin with what we _do_ know. Hence the necessity
that he should have been well trained in habits, who is to study, with
any tolerable chance of profit, the principles of nobleness and justice
and moral philosophy generally. For a principle is a matter of fact,
and if the fact is sufficiently clear to a man there will be no need in
addition of the reason for the fact. And he that has been thus trained
either has principles already, or can receive them easily: as for him
who neither has nor can receive them, let him hear his sentence from
Hesiod:

  He is best of all who of himself conceiveth all things;
  Good again is he too who can adopt a good suggestion;
  But whoso neither of himself conceiveth nor hearing from
  another
  Layeth it to heart;--he is a useless man.

[Sidenote: V] But to return from this digression.

Now of the Chief Good (_i.e._ of Happiness) men seem to form their
notions from the different modes of life, as we might naturally expect:
the many and most low conceive it to be pleasure, and hence they are
content with the life of sensual enjoyment. For there are three lines of
life which stand out prominently to view: that just mentioned, and the
life in society, and, thirdly, the life of contemplation.

Now the many are plainly quite slavish, choosing a life like that of
brute animals: yet they obtain some consideration, because many of the
great share the tastes of Sardanapalus. The refined and active again
conceive it to be honour: for this may be said to be the end of the life
in society: yet it is plainly too superficial for the object of our
search, because it is thought to rest with those who pay rather than
with him who receives it, whereas the Chief Good we feel instinctively
must be something which is our own, and not easily to be taken from us.

And besides, men seem to pursue honour, that they may *[Sidenote: 1096a]
believe themselves to be good: for instance, they seek to be honoured
by the wise, and by those among whom they are known, and for virtue:
clearly then, in the opinion at least of these men, virtue is higher
than honour. In truth, one would be much more inclined to think this
to be the end of the life in society; yet this itself is plainly not
sufficiently final: for it is conceived possible, that a man possessed
of virtue might sleep or be inactive all through his life, or, as a
third case, suffer the greatest evils and misfortunes: and the man who
should live thus no one would call happy, except for mere disputation's
sake.

And for these let thus much suffice, for they have been treated of at
sufficient length in my Encyclia.

A third line of life is that of contemplation, concerning which we shall
make our examination in the sequel.

As for the life of money-making, it is one of constraint, and wealth
manifestly is not the good we are seeking, because it is for use, that
is, for the sake of something further: and hence one would rather
conceive the forementioned ends to be the right ones, for men rest
content with them for their own sakes. Yet, clearly, they are not the
objects of our search either, though many words have been wasted on
them. So much then for these.

[Sidenote: VI] Again, the notion of one Universal Good (the same, that
is, in all things), it is better perhaps we should examine, and discuss
the meaning of it, though such an inquiry is unpleasant, because they
are friends of ours who have introduced these [Greek: _eidae_]. Still
perhaps it may appear better, nay to be our duty where the safety of the
truth is concerned, to upset if need be even our own theories, specially
as we are lovers of wisdom: for since both are dear to us, we are bound
to prefer the truth. Now they who invented this doctrine of [Greek:
_eidae_], did not apply it to those things in which they spoke of
priority and posteriority, and so they never made any [Greek: _idea_] of
numbers; but good is predicated in the categories of Substance, Quality,
and Relation; now that which exists of itself, _i.e._ Substance, is
prior in the nature of things to that which is relative, because this
latter is an off-shoot, as it were, and result of that which is; on
their own principle then there cannot be a common [Greek: _idea_] in the
case of these.

In the next place, since good is predicated in as many ways as there are
modes of existence [for it is predicated in the category of Substance,
as God, Intellect--and in that of Quality, as The Virtues--and in that
of Quantity, as The Mean--and in that of Relation, as The Useful--and in
that of Time, as Opportunity--and in that of Place, as Abode; and
other such like things], it manifestly cannot be something common and
universal and one in all: else it would not have been predicated in all
the categories, but in one only.

[Sidenote: 1096b] Thirdly, since those things which range under one
[Greek: _idea_] are also under the cognisance of one science, there
would have been, on their theory, only one science taking cognisance of
all goods collectively: but in fact there are many even for those which
range under one category: for instance, of Opportunity or Seasonableness
(which I have before mentioned as being in the category of Time), the
science is, in war, generalship; in disease, medical science; and of the
Mean (which I quoted before as being in the category of Quantity), in
food, the medical science; and in labour or exercise, the gymnastic
science. A person might fairly doubt also what in the world they mean by
very-this that or the other, since, as they would themselves allow, the
account of the humanity is one and the same in the very-Man, and in any
individual Man: for so far as the individual and the very-Man are both
Man, they will not differ at all: and if so, then very-good and any
particular good will not differ, in so far as both are good. Nor will it
do to say, that the eternity of the very-good makes it to be more good;
for what has lasted white ever so long, is no whiter than what lasts but
for a day.

No. The Pythagoreans do seem to give a more credible account of the
matter, who place "One" among the goods in their double list of goods
and bads: which philosophers, in fact, Speusippus seems to have
followed.

But of these matters let us speak at some other time. Now there is
plainly a loophole to object to what has been advanced, on the plea that
the theory I have attacked is not by its advocates applied to all good:
but those goods only are spoken of as being under one [Greek: idea],
which are pursued, and with which men rest content simply for their own
sakes: whereas those things which have a tendency to produce or preserve
them in any way, or to hinder their contraries, are called good because
of these other goods, and after another fashion. It is manifest then
that the goods may be so called in two senses, the one class for their
own sakes, the other because of these.

Very well then, let us separate the independent goods from the
instrumental, and see whether they are spoken of as under one [Greek:
idea]. But the question next arises, what kind of goods are we to call
independent? All such as are pursued even when separated from other
goods, as, for instance, being wise, seeing, and certain pleasures and
honours (for these, though we do pursue them with some further end in
view, one would still place among the independent goods)? or does it
come in fact to this, that we can call nothing independent good except
the [Greek: idea], and so the concrete of it will be nought?

If, on the other hand, these are independent goods, then we shall
require that the account of the goodness be the same clearly in all,
just as that of the whiteness is in snow and white lead. But how stands
the fact? Why of honour and wisdom and pleasure the accounts are
distinct and different in so far as they are good. The Chief Good then
is not something common, and after one [Greek: idea].

But then, how does the name come to be common (for it is not seemingly a
case of fortuitous equivocation)? Are different individual things called
good by virtue of being from one source, or all conducing to one end, or
rather by way of analogy, for that intellect is to the soul as sight to
the body, and so on? However, perhaps we ought to leave these questions
now, for an accurate investigation of them is more properly the business
of a different philosophy. And likewise respecting the [Greek: idea]:
for even if there is some one good predicated in common of all things
that are good, or separable and capable of existing independently,
manifestly it cannot be the object of human action or attainable by Man;
but we are in search now of something that is so.

It may readily occur to any one, that it would be better to attain a
knowledge of it with a view to such concrete goods as are attainable and
practical, because, with this as a kind of model in our hands, we shall
the better know what things are good for us individually, and when we
know them, we shall attain them.

Some plausibility, it is true, this argument possesses, but it is
contradicted by the facts of the Arts and Sciences; for all these,
though aiming at some good, and seeking that which is deficient, yet
pretermit the knowledge of it: now it is not exactly probable that all
artisans without exception should be ignorant of so great a help as this
would be, and not even look after it; neither is it easy to see wherein
a weaver or a carpenter will be profited in respect of his craft by
knowing the very-good, or how a man will be the more apt to effect cures
or to command an army for having seen the [Greek: idea] itself. For
manifestly it is not health after this general and abstract fashion
which is the subject of the physician's investigation, but the health
of Man, or rather perhaps of this or that man; for he has to heal
individuals.--Thus much on these points.


VII

And now let us revert to the Good of which we are in search: what can it
be? for manifestly it is different in different actions and arts: for it
is different in the healing art and in the art military, and similarly
in the rest. What then is the Chief Good in each? Is it not "that for
the sake of which the other things are done?" and this in the healing
art is health, and in the art military victory, and in that of
house-building a house, and in any other thing something else; in short,
in every action and moral choice the End, because in all cases men do
everything else with a view to this. So that if there is some one End of
all things which are and may be done, this must be the Good proposed by
doing, or if more than one, then these.

Thus our discussion after some traversing about has come to the same
point which we reached before. And this we must try yet more to clear
up.

Now since the ends are plainly many, and of these we choose some with
a view to others (wealth, for instance, musical instruments, and, in
general, all instruments), it is clear that all are not final: but the
Chief Good is manifestly something final; and so, if there is some one
only which is final, this must be the object of our search: but if
several, then the most final of them will be it.

Now that which is an object of pursuit in itself we call more final than
that which is so with a view to something else; that again which is
never an object of choice with a view to something else than those which
are so both in themselves and with a view to this ulterior object: and
so by the term "absolutely final," we denote that which is an object of
choice always in itself, and never with a view to any other.

And of this nature Happiness is mostly thought to be, for this we choose
always for its own sake, and never with a view to anything further:
whereas honour, pleasure, intellect, in fact every excellence we choose
for their own sakes, it is true (because we would choose each of these
even if no result were to follow), but we choose them also with a view
to happiness, conceiving that through their instrumentality we shall be
happy: but no man chooses happiness with a view to them, nor in fact
with a view to any other thing whatsoever.

The same result is seen to follow also from the notion of
self-sufficiency, a quality thought to belong to the final good. Now
by sufficient for Self, we mean not for a single individual living a
solitary life, but for his parents also and children and wife, and,
in general, friends and countrymen; for man is by nature adapted to a
social existence. But of these, of course, some limit must be fixed: for
if one extends it to parents and descendants and friends' friends,
there is no end to it. This point, however, must be left for future
investigation: for the present we define that to be self-sufficient
"which taken alone makes life choice-worthy, and to be in want of
nothing;" now of such kind we think Happiness to be: and further, to
be most choice-worthy of all things; not being reckoned with any other
thing, for if it were so reckoned, it is plain we must then allow it,
with the addition of ever so small a good, to be more choice-worthy than
it was before: because what is put to it becomes an addition of so much
more good, and of goods the greater is ever the more choice-worthy.

So then Happiness is manifestly something final and self-sufficient,
being the end of all things which are and may be done.

But, it may be, to call Happiness the Chief Good is a mere truism, and
what is wanted is some clearer account of its real nature. Now this
object may be easily attained, when we have discovered what is the work
of man; for as in the case of flute-player, statuary, or artisan of any
kind, or, more generally, all who have any work or course of action,
their Chief Good and Excellence is thought to reside in their work, so
it would seem to be with man, if there is any work belonging to him.

Are we then to suppose, that while carpenter and cobbler have certain
works and courses of action, Man as Man has none, but is left by Nature
without a work? or would not one rather hold, that as eye, hand, and
foot, and generally each of his members, has manifestly some special
work; so too the whole Man, as distinct from all these, has some work of
his own?

What then can this be? not mere life, because that plainly is shared
with him even by vegetables, and we want what is peculiar to him. We
must separate off then the life of mere nourishment and growth, and next
will come the life of sensation: but this again manifestly is common to
horses, oxen, and every animal. There remains then a kind of life of
the Rational Nature apt to act: and of this Nature there are two parts
denominated Rational, the one as being obedient to Reason, the other as
having and exerting it. Again, as this life is also spoken of in two
ways, we must take that which is in the way of actual working, because
this is thought to be most properly entitled to the name. If then the
work of Man is a working of the soul in accordance with reason, or at
least not independently of reason, and we say that the work of any given
subject, and of that subject good of its kind, are the same in kind (as,
for instance, of a harp-player and a good harp-player, and so on in
every case, adding to the work eminence in the way of excellence; I
mean, the work of a harp-player is to play the harp, and of a good
harp-player to play it well); if, I say, this is so, and we assume the
work of Man to be life of a certain kind, that is to say a working of
the soul, and actions with reason, and of a good man to do these things
well and nobly, and in fact everything is finished off well in the way
of the excellence which peculiarly belongs to it: if all this is so,
then the Good of Man comes to be "a working of the Soul in the way of
Excellence," or, if Excellence admits of degrees, in the way of the best
and most perfect Excellence.

And we must add, in a complete life; for as it is not one swallow or one
fine day that makes a spring, so it is not one day or a short time that
makes a man blessed and happy.

Let this then be taken for a rough sketch of the Chief Good: since it
is probably the right way to give first the outline, and fill it in
afterwards. And it would seem that any man may improve and connect
what is good in the sketch, and that time is a good discoverer and
co-operator in such matters: it is thus in fact that all improvements
in the various arts have been brought about, for any man may fill up a
deficiency.

You must remember also what has been already stated, and not seek
for exactness in all matters alike, but in each according to the
subject-matter, and so far as properly belongs to the system. The
carpenter and geometrician, for instance, inquire into the right line in
different fashion: the former so far as he wants it for his work, the
latter inquires into its nature and properties, because he is concerned
with the truth.

So then should one do in other matters, that the incidental matters may
not exceed the direct ones.

And again, you must not demand the reason either in all things
alike, because in some it is sufficient that the fact has been well
demonstrated, which is the case with first principles; and the fact is
the first step, _i.e._ starting-point or principle.

And of these first principles some are obtained by induction, some by
perception, some by a course of habituation, others in other different
ways. And we must try to trace up each in their own nature, and take
pains to secure their being well defined, because they have
great influence on what follows: it is thought, I mean, that the
starting-point or principle is more than half the whole matter, and that
many of the points of inquiry come simultaneously into view thereby.


VIII

We must now inquire concerning Happiness, not only from our conclusion
and the data on which our reasoning proceeds, but likewise from what
is commonly said about it: because with what is true all things which
really are are in harmony, but with that which is false the true very
soon jars.

Now there is a common division of goods into three classes; one being
called external, the other two those of the soul and body respectively,
and those belonging to the soul we call most properly and specially
good. Well, in our definition we assume that the actions and workings of
the soul constitute Happiness, and these of course belong to the soul.
And so our account is a good one, at least according to this opinion,
which is of ancient date, and accepted by those who profess philosophy.
Rightly too are certain actions and workings said to be the end, for
thus it is brought into the number of the goods of the soul instead of
the external. Agreeing also with our definition is the common notion,
that the happy man lives well and does well, for it has been stated by
us to be pretty much a kind of living well and doing well.

But further, the points required in Happiness are found in combination
in our account of it.

For some think it is virtue, others practical wisdom, others a kind of
scientific philosophy; others that it is these, or else some one of
them, in combination with pleasure, or at least not independently of it;
while others again take in external prosperity.

Of these opinions, some rest on the authority of numbers or antiquity,
others on that of few, and those men of note: and it is not likely that
either of these classes should be wrong in all points, but be right at
least in some one, or even in most.

Now with those who assert it to be Virtue (Excellence), or some kind of
Virtue, our account agrees: for working in the way of Excellence surely
belongs to Excellence.

And there is perhaps no unimportant difference between conceiving of
the Chief Good as in possession or as in use, in other words, as a mere
state or as a working. For the state or habit may possibly exist in a
subject without effecting any good, as, for instance, in him who is
asleep, or in any other way inactive; but the working cannot so, for it
will of necessity act, and act well. And as at the Olympic games it is
not the finest and strongest men who are crowned, but they who enter the
lists, for out of these the prize-men are selected; so too in life, of
the honourable and the good, it is they who act who rightly win the
prizes.

Their life too is in itself pleasant: for the feeling of pleasure is a
mental sensation, and that is to each pleasant of which he is said to be
fond: a horse, for instance, to him who is fond of horses, and a sight
to him who is fond of sights: and so in like manner just acts to him who
is fond of justice, and more generally the things in accordance with
virtue to him who is fond of virtue. Now in the case of the multitude of
men the things which they individually esteem pleasant clash, because
they are not such by nature, whereas to the lovers of nobleness those
things are pleasant which are such by nature: but the actions in
accordance with virtue are of this kind, so that they are pleasant both
to the individuals and also in themselves.

So then their life has no need of pleasure as a kind of additional
appendage, but involves pleasure in itself. For, besides what I have
just mentioned, a man is not a good man at all who feels no pleasure in
noble actions, just as no one would call that man just who does not feel
pleasure in acting justly, or liberal who does not in liberal actions,
and similarly in the case of the other virtues which might be
enumerated: and if this be so, then the actions in accordance with
virtue must be in themselves pleasurable. Then again they are certainly
good and noble, and each of these in the highest degree; if we are to
take as right the judgment of the good man, for he judges as we have
said.

Thus then Happiness is most excellent, most noble, and most pleasant,
and these attributes are not separated as in the well-known Delian
inscription--

"Most noble is that which is most just, but best is health; And
naturally most pleasant is the obtaining one's desires."

For all these co-exist in the best acts of working: and we say that
Happiness is these, or one, that is, the best of them.

Still it is quite plain that it does require the addition of external
goods, as we have said: because without appliances it is impossible, or
at all events not easy, to do noble actions: for friends, money, and
political influence are in a manner instruments whereby many things
are done: some things there are again a deficiency in which mars
blessedness; good birth, for instance, or fine offspring, or even
personal beauty: for he is not at all capable of Happiness who is very
ugly, or is ill-born, or solitary and childless; and still less perhaps
supposing him to have very bad children or friends, or to have lost good
ones by death. As we have said already, the addition of prosperity of
this kind does seem necessary to complete the idea of Happiness; hence
some rank good fortune, and others virtue, with Happiness.

And hence too a question is raised, whether it is a thing that can be
learned, or acquired by habituation or discipline of some other kind, or
whether it comes in the way of divine dispensation, or even in the way
of chance.

Now to be sure, if anything else is a gift of the Gods to men, it is
probable that Happiness is a gift of theirs too, and specially because
of all human goods it is the highest. But this, it may be, is a question
belonging more properly to an investigation different from ours: and it
is quite clear, that on the supposition of its not being sent from the
Gods direct, but coming to us by reason of virtue and learning of a
certain kind, or discipline, it is yet one of the most Godlike things;
because the prize and End of virtue is manifestly somewhat most
excellent, nay divine and blessed.

It will also on this supposition be widely participated, for it may
through learning and diligence of a certain kind exist in all who have
not been maimed for virtue.

And if it is better we should be happy thus than as a result of chance,
this is in itself an argument that the case is so; because those things
which are in the way of nature, and in like manner of art, and of every
cause, and specially the best cause, are by nature in the best way
possible: to leave them to chance what is greatest and most noble would
be very much out of harmony with all these facts.

The question may be determined also by a reference to our definition of
Happiness, that it is a working of the soul in the way of excellence or
virtue of a certain kind: and of the other goods, some we must have to
begin with, and those which are co-operative and useful are given by
nature as instruments.

These considerations will harmonise also with what we said at the
commencement: for we assumed the End of [Greek Text: poletikae] to be
most excellent: now this bestows most care on making the members of the
community of a certain character; good that is and apt to do what is
honourable.

With good reason then neither ox nor horse nor any other brute animal
do we call happy, for none of them can partake in such working: and for
this same reason a child is not happy either, because by reason of his
tender age he cannot yet perform such actions: if the term is applied,
it is by way of anticipation.

For to constitute Happiness, there must be, as we have said, complete
virtue and a complete life: for many changes and chances of all kinds
arise during a life, and he who is most prosperous may become involved
in great misfortunes in his old age, as in the heroic poems the tale is
told of Priam: but the man who has experienced such fortune and died in
wretchedness, no man calls happy.

Are we then to call no man happy while he lives, and, as Solon would
have us, look to the end? And again, if we are to maintain this
position, is a man then happy when he is dead? or is not this a complete
absurdity, specially in us who say Happiness is a working of a certain
kind?

If on the other hand we do not assert that the dead man is happy, and
Solon does not mean this, but only that one would then be safe in
pronouncing a man happy, as being thenceforward out of the reach of
evils and misfortunes, this too admits of some dispute, since it is
thought that the dead has somewhat both of good and evil (if, as we must
allow, a man may have when alive but not aware of the circumstances),
as honour and dishonour, and good and bad fortune of children and
descendants generally.

Nor is this view again without its difficulties: for, after a man has
lived in blessedness to old age and died accordingly, many changes may
befall him in right of his descendants; some of them may be good and
obtain positions in life accordant to their merits, others again quite
the contrary: it is plain too that the descendants may at different
intervals or grades stand in all manner of relations to the ancestors.
Absurd indeed would be the position that even the dead man is to change
about with them and become at one time happy and at another miserable.
Absurd however it is on the other hand that the affairs of the
descendants should in no degree and during no time affect the ancestors.

But we must revert to the point first raised, since the present question
will be easily determined from that.

If then we are to look to the end and then pronounce the man blessed,
not as being so but as having been so at some previous time, surely it
is absurd that when he _is_ happy the truth is not to be asserted of
him, because we are unwilling to pronounce the living happy by reason of
their liability to changes, and because, whereas we have conceived of
happiness as something stable and no way easily changeable, the fact is
that good and bad fortune are constantly circling about the same people:
for it is quite plain, that if we are to depend upon the fortunes of
men, we shall often have to call the same man happy, and a little while
after miserable, thus representing our happy man

  "Chameleon-like, and based on rottenness."

Is not this the solution? that to make our sentence dependent on the
changes of fortune, is no way right: for not in them stands the well, or
the ill, but though human life needs these as accessories (which we have
allowed already), the workings in the way of virtue are what determine
Happiness, and the contrary the contrary.

And, by the way, the question which has been here discussed, testifies
incidentally to the truth of our account of Happiness. For to nothing
does a stability of human results attach so much as it does to the
workings in the way of virtue, since these are held to be more abiding
even than the sciences: and of these last again the most precious
are the most abiding, because the blessed live in them most and most
continuously, which seems to be the reason why they are not forgotten.
So then this stability which is sought will be in the happy man, and
he will be such through life, since always, or most of all, he will be
doing and contemplating the things which are in the way of virtue: and
the various chances of life he will bear most nobly, and at all times
and in all ways harmoniously, since he is the truly good man, or in the
terms of our proverb "a faultless cube."

And whereas the incidents of chance are many, and differ in greatness
and smallness, the small pieces of good or ill fortune evidently do not
affect the balance of life, but the great and numerous, if happening for
good, will make life more blessed (for it is their nature to contribute
to ornament, and the using of them comes to be noble and excellent), but
if for ill, they bruise as it were and maim the blessedness: for they
bring in positive pain, and hinder many acts of working. But still, even
in these, nobleness shines through when a man bears contentedly many and
great mischances not from insensibility to pain but because he is noble
and high-spirited.

And if, as we have said, the acts of working are what determine the
character of the life, no one of the blessed can ever become wretched,
because he will never do those things which are hateful and mean. For
the man who is truly good and sensible bears all fortunes, we presume,
becomingly, and always does what is noblest under the circumstances,
just as a good general employs to the best advantage the force he has
with him; or a good shoemaker makes the handsomest shoe he can out
of the leather which has been given him; and all other good artisans
likewise. And if this be so, wretched never can the happy man come to
be: I do not mean to say he will be blessed should he fall into fortunes
like those of Priam.

Nor, in truth, is he shifting and easily changeable, for on the one
hand from his happiness he will not be shaken easily nor by ordinary
mischances, but, if at all, by those which are great and numerous; and,
on the other, after such mischances he cannot regain his happiness in a
little time; but, if at all, in a long and complete period, during which
he has made himself master of great and noble things.

Why then should we not call happy the man who works in the way of
perfect virtue, and is furnished with external goods sufficient for
acting his part in the drama of life: and this during no ordinary period
but such as constitutes a complete life as we have been describing it.

Or we must add, that not only is he to live so, but his death must be in
keeping with such life, since the future is dark to us, and Happiness we
assume to be in every way an end and complete. And, if this be so, we
shall call them among the living blessed who have and will have the
things specified, but blessed _as Men_.

On these points then let it suffice to have denned thus much.


XI

Now that the fortunes of their descendants, and friends generally,
contribute nothing towards forming the condition of the dead, is plainly
a very heartless notion, and contrary to the current opinions.

But since things which befall are many, and differ in all kinds of ways,
and some touch more nearly, others less, to go into minute particular
distinctions would evidently be a long and endless task: and so it may
suffice to speak generally and in outline.

If then, as of the misfortunes which happen to one's self, some have a
certain weight and turn the balance of life, while others are, so to
speak, lighter; so it is likewise with those which befall all our
friends alike; if further, whether they whom each suffering befalls
be alive or dead makes much more difference than in a tragedy the
presupposing or actual perpetration of the various crimes and horrors,
we must take into our account this difference also, and still more
perhaps the doubt concerning the dead whether they really partake of any
good or evil; it seems to result from all these considerations, that if
anything does pierce the veil and reach them, be the same good or bad,
it must be something trivial and small, either in itself or to them; or
at least of such a magnitude or such a kind as neither to make happy
them that are not so otherwise, nor to deprive of their blessedness them
that are.

It is plain then that the good or ill fortunes of their friends do
affect the dead somewhat: but in such kind and degree as neither to make
the happy unhappy nor produce any other such effect.


XII

Having determined these points, let us examine with respect to
Happiness, whether it belongs to the class of things praiseworthy or
things precious; for to that of faculties it evidently does not.

Now it is plain that everything which is a subject of praise is praised
for being of a certain kind and bearing a certain relation to something
else: for instance, the just, and the valiant, and generally the good
man, and virtue itself, we praise because of the actions and the
results: and the strong man, and the quick runner, and so forth, we
praise for being of a certain nature and bearing a certain relation to
something good and excellent (and this is illustrated by attempts to
praise the gods; for they are presented in a ludicrous aspect by being
referred to our standard, and this results from the fact, that all
praise does, as we have said, imply reference to a standard). Now if
it is to such objects that praise belongs, it is evident that what is
applicable to the best objects is not praise, but something higher and
better: which is plain matter of fact, for not only do we call the gods
blessed and happy, but of men also we pronounce those blessed who most
nearly resemble the gods. And in like manner in respect of goods; no man
thinks of praising Happiness as he does the principle of justice, but
calls it blessed, as being somewhat more godlike and more excellent.

Eudoxus too is thought to have advanced a sound argument in support of
the claim of pleasure to the highest prize: for the fact that, though it
is one of the good things, it is not praised, he took for an indication
of its superiority to those which are subjects of praise: a superiority
he attributed also to a god and the Chief Good, on the ground that they
form the standard to which everything besides is referred. For praise
applies to virtue, because it makes men apt to do what is noble; but
encomia to definite works of body or mind.

However, it is perhaps more suitable to a regular treatise on encomia to
pursue this topic with exactness: it is enough for our purpose that from
what has been said it is evident that Happiness belongs to the class of
things precious and final. And it seems to be so also because of its
being a starting-point; which it is, in that with a view to it we all do
everything else that is done; now the starting-point and cause of good
things we assume to be something precious and divine.


XIII

Moreover, since Happiness is a kind of working of the soul in the way
of perfect Excellence, we must inquire concerning Excellence: for so
probably shall we have a clearer view concerning Happiness; and again,
he who is really a statesman is generally thought to have spent most
pains on this, for he wishes to make the citizens good and obedient
to the laws. (For examples of this class we have the lawgivers of the
Cretans and Lacedaemonians and whatever other such there have been.)
But if this investigation belongs properly to [Greek: politikae], then
clearly the inquiry will be in accordance with our original design.

Well, we are to inquire concerning Excellence, _i.e._ Human Excellence
of course, because it was the Chief Good of Man and the Happiness of Man
that we were inquiring of just now. By Human Excellence we mean not that
of man's body but that of his soul; for we call Happiness a working of
the Soul.

And if this is so, it is plain that some knowledge of the nature of the
Soul is necessary for the statesman, just as for the Oculist a knowledge
of the whole body, and the more so in proportion as [Greek: politikae]
is more precious and higher than the healing art: and in fact physicians
of the higher class do busy themselves much with the knowledge of the
body.

So then the statesman is to consider the nature of the Soul: but he must
do so with these objects in view, and so far only as may suffice for
the objects of his special inquiry: for to carry his speculations to a
greater exactness is perhaps a task more laborious than falls within his
province.

In fact, the few statements made on the subject in my popular treatises
are quite enough, and accordingly we will adopt them here: as, that
the Soul consists of two parts, the Irrational and the Rational (as to
whether these are actually divided, as are the parts of the body, and
everything that is capable of division; or are only metaphysically
speaking two, being by nature inseparable, as are convex and concave
circumferences, matters not in respect of our present purpose). And of
the Irrational, the one part seems common to other objects, and in fact
vegetative; I mean the cause of nourishment and growth (for such a
faculty of the Soul one would assume to exist in all things that receive
nourishment, even in embryos, and this the same as in the perfect
creatures; for this is more likely than that it should be a different
one).

Now the Excellence of this manifestly is not peculiar to the human
species but common to others: for this part and this faculty is thought
to work most in time of sleep, and the good and bad man are least
distinguishable while asleep; whence it is a common saying that during
one half of life there is no difference between the happy and the
wretched; and this accords with our anticipations, for sleep is an
inactivity of the soul, in so far as it is denominated good or bad,
except that in some wise some of its movements find their way through
the veil and so the good come to have better dreams than ordinary men.
But enough of this: we must forego any further mention of the nutritive
part, since it is not naturally capable of the Excellence which is
peculiarly human.

And there seems to be another Irrational Nature of the Soul, which yet
in a way partakes of Reason. For in the man who controls his appetites,
and in him who resolves to do so and fails, we praise the Reason or
Rational part of the Soul, because it exhorts aright and to the best
course: but clearly there is in them, beside the Reason, some other
natural principle which fights with and strains against the Reason. (For
in plain terms, just as paralysed limbs of the body when their owners
would move them to the right are borne aside in a contrary direction to
the left, so is it in the case of the Soul, for the impulses of men who
cannot control their appetites are to contrary points: the difference is
that in the case of the body we do see what is borne aside but in the
case of the soul we do not. But, it may be, not the less on that account
are we to suppose that there is in the Soul also somewhat besides the
Reason, which is opposed to this and goes against it; as to _how_ it is
different, that is irrelevant.)

But of Reason this too does evidently partake, as we have said: for
instance, in the man of self-control it obeys Reason: and perhaps in
the man of perfected self-mastery, or the brave man, it is yet more
obedient; in them it agrees entirely with the Reason.

So then the Irrational is plainly twofold: the one part, the merely
vegetative, has no share of Reason, but that of desire, or appetition
generally, does partake of it in a sense, in so far as it is obedient to
it and capable of submitting to its rule. (So too in common phrase we
say we have [Greek: _logos_] of our father or friends, and this in a
different sense from that in which we say we have [Greek: logos] of
mathematics.)

Now that the Irrational is in some way persuaded by the Reason,
admonition, and every act of rebuke and exhortation indicate. If then we
are to say that this also has Reason, then the Rational, as well as the
Irrational, will be twofold, the one supremely and in itself, the other
paying it a kind of filial regard.

The Excellence of Man then is divided in accordance with this
difference: we make two classes, calling the one Intellectual, and
the other Moral; pure science, intelligence, and practical
wisdom--Intellectual: liberality, and perfected self-mastery--Moral: in
speaking of a man's Moral character, we do not say he is a scientific
or intelligent but a meek man, or one of perfected self-mastery: and we
praise the man of science in right of his mental state; and of these
such as are praiseworthy we call Excellences.




BOOK II

Well: human Excellence is of two kinds, Intellectual and Moral: now the
Intellectual springs originally, and is increased subsequently, from
teaching (for the most part that is), and needs therefore experience
and time; whereas the Moral comes from custom, and so the Greek term
denoting it is but a slight deflection from the term denoting custom in
that language.

From this fact it is plain that not one of the Moral Virtues comes to be
in us merely by nature: because of such things as exist by nature, none
can be changed by custom: a stone, for instance, by nature gravitating
downwards, could never by custom be brought to ascend, not even if one
were to try and accustom it by throwing it up ten thousand times; nor
could file again be brought to descend, nor in fact could anything whose
nature is in one way be brought by custom to be in another. The Virtues
then come to be in us neither by nature, nor in despite of nature, but
we are furnished by nature with a capacity for receiving themu and are
perfected in them through custom.

Again, in whatever cases we get things by nature, we get the faculties
first and perform the acts of working afterwards; an illustration of
which is afforded by the case of our bodily senses, for it was not
from having often seen or heard that we got these senses, but just
the reverse: we had them and so exercised them, but did not have
them because we had exercised them. But the Virtues we get by first
performing single acts of working, which, again, is the case of other
things, as the arts for instance; for what we have to make when we
have learned how, these we learn how to make by making: men come to be
builders, for instance, by building; harp-players, by playing on the
harp: exactly so, by doing just actions we come to be just; by doing the
actions of self-mastery we come to be perfected in self-mastery; and by
doing brave actions brave.

And to the truth of this testimony is borne by what takes place in
communities: because the law-givers make the individual members good men
by habituation, and this is the intention certainly of every law-giver,
and all who do not effect it well fail of their intent; and herein
consists the difference between a good Constitution and a bad.

Again, every Virtue is either produced or destroyed from and by the very
same circumstances: art too in like manner; I mean it is by playing
the harp that both the good and the bad harp-players are formed: and
similarly builders and all the rest; by building well men will become
good builders; by doing it badly bad ones: in fact, if this had not been
so, there would have been no need of instructors, but all men would have
been at once good or bad in their several arts without them.

So too then is it with the Virtues: for by acting in the various
relations in which we are thrown with our fellow men, we come to be,
some just, some unjust: and by acting in dangerous positions and being
habituated to feel fear or confidence, we come to be, some brave, others
cowards.

Similarly is it also with respect to the occasions of lust and anger:
for some men come to be perfected in self-mastery and mild, others
destitute of all self-control and passionate; the one class by behaving
in one way under them, the other by behaving in another. Or, in one
word, the habits are produced from the acts of working like to them: and
so what we have to do is to give a certain character to these particular
acts, because the habits formed correspond to the differences of these.

So then, whether we are accustomed this way or that straight from
childhood, makes not a small but an important difference, or rather I
would say it makes all the difference.


II

Since then the object of the present treatise is not mere speculation,
as it is of some others (for we are inquiring not merely that we may
know what virtue is but that we may become virtuous, else it would have
been useless), we must consider as to the particular actions how we are
to do them, because, as we have just said, the quality of the habits
that shall be formed depends on these.

Now, that we are to act in accordance with Right Reason is a general
maxim, and may for the present be taken for granted: we will speak of it
hereafter, and say both what Right Reason is, and what are its relations
to the other virtues.

[Sidenote: 1104a]

But let this point be first thoroughly understood between us, that all
which can be said on moral action must be said in outline, as it were,
and not exactly: for as we remarked at the commencement, such reasoning
only must be required as the nature of the subject-matter admits of, and
matters of moral action and expediency have no fixedness any more than
matters of health. And if the subject in its general maxims is such,
still less in its application to particular cases is exactness
attainable: because these fall not under any art or system of rules, but
it must be left in each instance to the individual agents to look to the
exigencies of the particular case, as it is in the art of healing,
or that of navigating a ship. Still, though the present subject is
confessedly such, we must try and do what we can for it.

First then this must be noted, that it is the nature of such things to
be spoiled by defect and excess; as we see in the case of health and
strength (since for the illustration of things which cannot be seen we
must use those that can), for excessive training impairs the strength as
well as deficient: meat and drink, in like manner, in too great or too
small quantities, impair the health: while in due proportion they cause,
increase, and preserve it.

Thus it is therefore with the habits of perfected Self-Mastery and
Courage and the rest of the Virtues: for the man who flies from and
fears all things, and never stands up against anything, comes to be a
coward; and he who fears nothing, but goes at everything, comes to be
rash. In like manner too, he that tastes of every pleasure and abstains
from none comes to lose all self-control; while he who avoids all, as
do the dull and clownish, comes as it were to lose his faculties of
perception: that is to say, the habits of perfected Self-Mastery and
Courage are spoiled by the excess and defect, but by the mean state are
preserved.

Furthermore, not only do the origination, growth, and marring of the
habits come from and by the same circumstances, but also the acts of
working after the habits are formed will be exercised on the same: for
so it is also with those other things which are more directly matters of
sight, strength for instance: for this comes by taking plenty of food
and doing plenty of work, and the man who has attained strength is best
able to do these: and so it is with the Virtues, for not only do we by
abstaining from pleasures come to be perfected in Self-Mastery, but when
we have come to be so we can best abstain from them: similarly too with
Courage: for it is by accustoming ourselves to despise objects of fear
and stand up against them that we come to be brave; and [Sidenote(?):
1104_b_] after we have come to be so we shall be best able to stand up
against such objects.

And for a test of the formation of the habits we must [Sidenote(?): III]
take the pleasure or pain which succeeds the acts; for he is perfected
in Self-Mastery who not only abstains from the bodily pleasures but is
glad to do so; whereas he who abstains but is sorry to do it has not
Self-Mastery: he again is brave who stands up against danger, either
with positive pleasure or at least without any pain; whereas he who does
it with pain is not brave.

For Moral Virtue has for its object-matter pleasures and pains, because
by reason of pleasure we do what is bad, and by reason of pain decline
doing what is right (for which cause, as Plato observes, men should have
been trained straight from their childhood to receive pleasure and pain
from proper objects, for this is the right education). Again: since
Virtues have to do with actions and feelings, and on every feeling and
every action pleasure and pain follow, here again is another proof that
Virtue has for its object-matter pleasure and pain. The same is
shown also by the fact that punishments are effected through the
instrumentality of these; because they are of the nature of remedies,
and it is the nature of remedies to be the contraries of the ills they
cure. Again, to quote what we said before: every habit of the Soul by
its very nature has relation to, and exerts itself upon, things of the
same kind as those by which it is naturally deteriorated or improved:
now such habits do come to be vicious by reason of pleasures and pains,
that is, by men pursuing or avoiding respectively, either such as they
ought not, or at wrong times, or in wrong manner, and so forth (for
which reason, by the way, some people define the Virtues as certain
states of impassibility and utter quietude, but they are wrong because
they speak without modification, instead of adding "as they ought," "as
they ought not," and "when," and so on). Virtue then is assumed to be
that habit which is such, in relation to pleasures and pains, as to
effect the best results, and Vice the contrary.

The following considerations may also serve to set this in a clear
light. There are principally three things moving us to choice and three
to avoidance, the honourable, the expedient, the pleasant; and their
three contraries, the dishonourable, the hurtful, and the painful: now
the good man is apt to go right, and the bad man wrong, with respect
to all these of course, but most specially with respect to pleasure:
because not only is this common to him with all animals but also it is
a concomitant of all those things which move to choice, since both the
honourable and the expedient give an impression of pleasure.

[Sidenote: 1105a] Again, it grows up with us all from infancy, and so it
is a hard matter to remove from ourselves this feeling, engrained as it
is into our very life.

Again, we adopt pleasure and pain (some of us more, and some less) as
the measure even of actions: for this cause then our whole business must
be with them, since to receive right or wrong impressions of pleasure
and pain is a thing of no little importance in respect of the actions.
Once more; it is harder, as Heraclitus says, to fight against pleasure
than against anger: now it is about that which is more than commonly
difficult that art comes into being, and virtue too, because in that
which is difficult the good is of a higher order: and so for this
reason too both virtue and moral philosophy generally must wholly busy
themselves respecting pleasures and pains, because he that uses these
well will be good, he that does so ill will be bad.

Let us then be understood to have stated, that Virtue has for its
object-matter pleasures and pains, and that it is either increased or
marred by the same circumstances (differently used) by which it
is originally generated, and that it exerts itself on the same
circumstances out of which it was generated.

Now I can conceive a person perplexed as to the meaning of our
statement, that men must do just actions to become just, and those of
self-mastery to acquire the habit of self-mastery; "for," he would say,
"if men are doing the actions they have the respective virtues already,
just as men are grammarians or musicians when they do the actions of
either art." May we not reply by saying that it is not so even in the
case of the arts referred to: because a man may produce something
grammatical either by chance or the suggestion of another; but then only
will he be a grammarian when he not only produces something grammatical
but does so grammarian-wise, _i.e._ in virtue of the grammatical
knowledge he himself possesses.

Again, the cases of the arts and the virtues are not parallel: because
those things which are produced by the arts have their excellence in
themselves, and it is sufficient therefore [Sidenote: 1105b] that these
when produced should be in a certain state: but those which are produced
in the way of the virtues, are, strictly speaking, actions of a certain
kind (say of Justice or perfected Self-Mastery), not merely if in
themselves they are in a certain state but if also he who does them
does them being himself in a certain state, first if knowing what he is
doing, next if with deliberate preference, and with such preference for
the things' own sake; and thirdly if being himself stable and unapt to
change. Now to constitute possession of the arts these requisites are
not reckoned in, excepting the one point of knowledge: whereas for
possession of the virtues knowledge avails little or nothing, but the
other requisites avail not a little, but, in fact, are all in all, and
these requisites as a matter of fact do come from oftentimes doing the
actions of Justice and perfected Self-Mastery.

The facts, it is true, are called by the names of these habits when they
are such as the just or perfectly self-mastering man would do; but he is
not in possession of the virtues who merely does these facts, but he who
also so does them as the just and self-mastering do them.

We are right then in saying, that these virtues are formed in a man by
his doing the actions; but no one, if he should leave them undone, would
be even in the way to become a good man. Yet people in general do not
perform these actions, but taking refuge in talk they flatter themselves
they are philosophising, and that they will so be good men: acting in
truth very like those sick people who listen to the doctor with great
attention but do nothing that he tells them: just as these then cannot
be well bodily under such a course of treatment, so neither can those be
mentally by such philosophising.

[Sidenote: V] Next, we must examine what Virtue is. Well, since the
things which come to be in the mind are, in all, of three kinds,
Feelings, Capacities, States, Virtue of course must belong to one of the
three classes.

By Feelings, I mean such as lust, anger, fear, confidence, envy, joy,
friendship, hatred, longing, emulation, compassion, in short all such as
are followed by pleasure or pain: by Capacities, those in right of which
we are said to be capable of these feelings; as by virtue of which we
are able to have been made angry, or grieved, or to have compassionated;
by States, those in right of which we are in a certain relation good
or bad to the aforementioned feelings; to having been made angry, for
instance, we are in a wrong relation if in our anger we were too violent
or too slack, but if we were in the happy medium we are in a right
relation to the feeling. And so on of the rest.

Now Feelings neither the virtues nor vices are, because in right of the
Feelings we are not denominated either good or bad, but in right of the
virtues and vices we are.

[Sidenote: 1106_a_] Again, in right of the Feelings we are neither
praised nor blamed (for a man is not commended for being afraid or
being angry, nor blamed for being angry merely but for being so in a
particular way), but in right of the virtues and vices we are.

Again, both anger and fear we feel without moral choice, whereas the
virtues are acts of moral choice, or at least certainly not independent
of it.

Moreover, in right of the Feelings we are said to be moved, but in right
of the virtues and vices not to be moved, but disposed, in a certain
way.

And for these same reasons they are not Capacities, for we are not
called good or bad merely because we are able to feel, nor are we
praised or blamed.

And again, Capacities we have by nature, but we do not come to be good
or bad by nature, as we have said before.

Since then the virtues are neither Feelings nor Capacities, it remains
that they must be States.

[Sidenote: VI] Now what the genus of Virtue is has been said; but we
must not merely speak of it thus, that it is a state but say also what
kind of a state it is. We must observe then that all excellence makes
that whereof it is the excellence both to be itself in a good state and
to perform its work well. The excellence of the eye, for instance, makes
both the eye good and its work also: for by the excellence of the eye
we see well. So too the excellence of the horse makes a horse good, and
good in speed, and in carrying his rider, and standing up against the
enemy. If then this is universally the case, the excellence of Man, i.e.
Virtue, must be a state whereby Man comes to be good and whereby he will
perform well his proper work. Now how this shall be it is true we have
said already, but still perhaps it may throw light on the subject to see
what is its characteristic nature.

In all quantity then, whether continuous or discrete, one may take the
greater part, the less, or the exactly equal, and these either with
reference to the thing itself, or relatively to us: and the exactly
equal is a mean between excess and defect. Now by the mean of the thing,
_i.e._ absolute mean, I denote that which is equidistant from either
extreme (which of course is one and the same to all), and by the mean
relatively to ourselves, that which is neither too much nor too little
for the particular individual. This of course is not one nor the same to
all: for instance, suppose ten is too much and two too little, people
take six for the absolute mean; because it exceeds the smaller sum by
exactly as much as it is itself exceeded by the larger, and this mean is
according to arithmetical proportion.

[Sidenote: 1106_b_] But the mean relatively to ourselves must not be
so found ; for it does not follow, supposing ten minæ is too large a
quantity to eat and two too small, that the trainer will order his man
six; because for the person who is to take it this also may be too much
or too little: for Milo it would be too little, but for a man just
commencing his athletic exercises too much: similarly too of the
exercises themselves, as running or wrestling.

So then it seems every one possessed of skill avoids excess and defect,
but seeks for and chooses the mean, not the absolute but the relative.

Now if all skill thus accomplishes well its work by keeping an eye on
the mean, and bringing the works to this point (whence it is common
enough to say of such works as are in a good state, "one cannot add
to or take ought from them," under the notion of excess or defect
destroying goodness but the mean state preserving it), and good
artisans, as we say, work with their eye on this, and excellence, like
nature, is more exact and better than any art in the world, it must have
an aptitude to aim at the mean.

It is moral excellence, _i.e._ Virtue, of course which I mean, because
this it is which is concerned with feelings and actions, and in these
there can be excess and defect and the mean: it is possible, for
instance, to feel the emotions of fear, confidence, lust, anger,
compassion, and pleasure and pain generally, too much or too little,
and in either case wrongly; but to feel them when we ought, on what
occasions, towards whom, why, and as, we should do, is the mean, or in
other words the best state, and this is the property of Virtue.

In like manner too with respect to the actions, there may be excess and
defect and the mean. Now Virtue is concerned with feelings and actions,
in which the excess is wrong and the defect is blamed but the mean is
praised and goes right; and both these circumstances belong to Virtue.
Virtue then is in a sense a mean state, since it certainly has an
aptitude for aiming at the mean.

Again, one may go wrong in many different ways (because, as the
Pythagoreans expressed it, evil is of the class of the infinite, good
of the finite), but right only in one; and so the former is easy, the
latter difficult; easy to miss the mark, but hard to hit it: and for
these reasons, therefore, both the excess and defect belong to Vice, and
the mean state to Virtue; for, as the poet has it,

  "Men may be bad in many ways,
  But good in one alone."
Virtue then is "a state apt to exercise deliberate choice, being in the
relative mean, determined by reason, and as the man of practical wisdom
would determine."

It is a middle state between too faulty ones, in the way of excess on
one side and of defect on the other: and it is so moreover, because the
faulty states on one side fall short of, and those on the other exceed,
what is right, both in the case of the feelings and the actions; but
Virtue finds, and when found adopts, the mean.

And so, viewing it in respect of its essence and definition, Virtue is a
mean state; but in reference to the chief good and to excellence it is
the highest state possible.

But it must not be supposed that every action or every feeling is
capable of subsisting in this mean state, because some there are
which are so named as immediately to convey the notion of badness, as
malevolence, shamelessness, envy; or, to instance in actions, adultery,
theft, homicide; for all these and suchlike are blamed because they are
in themselves bad, not the having too much or too little of them.

In these then you never can go right, but must always be wrong: nor in
such does the right or wrong depend on the selection of a proper person,
time, or manner (take adultery for instance), but simply doing any one
soever of those things is being wrong.

You might as well require that there should be determined a mean state,
an excess and a defect in respect of acting unjustly, being cowardly, or
giving up all control of the passions: for at this rate there will be
of excess and defect a mean state; of excess, excess; and of defect,
defect.

But just as of perfected self-mastery and courage there is no excess and
defect, because the mean is in one point of view the highest possible
state, so neither of those faulty states can you have a mean state,
excess, or defect, but howsoever done they are wrong: you cannot, in
short, have of excess and defect a mean state, nor of a mean state
excess and defect.


VII

It is not enough, however, to state this in general terms, we must also
apply it to particular instances, because in treatises on moral conduct
general statements have an air of vagueness, but those which go into
detail one of greater reality: for the actions after all must be in
detail, and the general statements, to be worth anything, must hold good
here.

We must take these details then from the Table.

I. In respect of fears and confidence or boldness:

[Sidenote: 1107b]

The Mean state is Courage: men may exceed, of course, either in absence
of fear or in positive confidence: the former has no name (which is a
common case), the latter is called rash: again, the man who has too much
fear and too little confidence is called a coward.

II. In respect of pleasures and pains (but not all, and perhaps fewer
pains than pleasures):

The Mean state here is perfected Self-Mastery, the defect total absence
of Self-control. As for defect in respect of pleasure, there are really
no people who are chargeable with it, so, of course, there is really no
name for such characters, but, as they are conceivable, we will give
them one and call them insensible.

III. In respect of giving and taking wealth (a):

The mean state is Liberality, the excess Prodigality, the defect
Stinginess: here each of the extremes involves really an excess and
defect contrary to each other: I mean, the prodigal gives out too much
and takes in too little, while the stingy man takes in too much and
gives out too little. (It must be understood that we are now giving
merely an outline and summary, intentionally: and we will, in a later
part of the treatise, draw out the distinctions with greater exactness.)

IV. In respect of wealth (b):

There are other dispositions besides these just mentioned; a mean state
called Munificence (for the munificent man differs from the liberal, the
former having necessarily to do with great wealth, the latter with but
small); the excess called by the names either of Want of taste or
Vulgar Profusion, and the defect Paltriness (these also differ from the
extremes connected with liberality, and the manner of their difference
shall also be spoken of later).

V. In respect of honour and dishonour (a):

The mean state Greatness of Soul, the excess which may be called
braggadocio, and the defect Littleness of Soul.

VI. In respect of honour and dishonour (b):

[Sidenote: 1108a]

Now there is a state bearing the same relation to Greatness of Soul as
we said just now Liberality does to Munificence, with the difference
that is of being about a small amount of the same thing: this state
having reference to small honour, as Greatness of Soul to great honour;
a man may, of course, grasp at honour either more than he should or
less; now he that exceeds in his grasping at it is called ambitious, he
that falls short unambitious, he that is just as he should be has no
proper name: nor in fact have the states, except that the disposition of
the ambitious man is called ambition. For this reason those who are in
either extreme lay claim to the mean as a debateable land, and we call
the virtuous character sometimes by the name ambitious, sometimes by
that of unambitious, and we commend sometimes the one and sometimes
the other. Why we do it shall be said in the subsequent part of the
treatise; but now we will go on with the rest of the virtues after the
plan we have laid down.

VII. In respect of anger:

Here too there is excess, defect, and a mean state; but since they
may be said to have really no proper names, as we call the virtuous
character Meek, we will call the mean state Meekness, and of the
extremes, let the man who is excessive be denominated Passionate, and
the faulty state Passionateness, and him who is deficient Angerless, and
the defect Angerlessness.

There are also three other mean states, having some mutual resemblance,
but still with differences; they are alike in that they all have for
their object-matter intercourse of words and deeds, and they differ in
that one has respect to truth herein, the other two to what is pleasant;
and this in two ways, the one in relaxation and amusement, the other in
all things which occur in daily life. We must say a word or two about
these also, that we may the better see that in all matters the mean is
praiseworthy, while the extremes are neither right nor worthy of praise
but of blame.

Now of these, it is true, the majority have really no proper names, but
still we must try, as in the other cases, to coin some for them for the
sake of clearness and intelligibleness.

I. In respect of truth: The man who is in the mean state we will call
Truthful, and his state Truthfulness, and as to the disguise of truth,
if it be on the side of exaggeration, Braggadocia, and him that has it a
Braggadocio; if on that of diminution, Reserve and Reserved shall be the
terms.

II. In respect of what is pleasant in the way of relaxation or
amusement: The mean state shall be called Easy-pleasantry, and the
character accordingly a man of Easy-pleasantry; the excess Buffoonery,
and the man a Buffoon; the man deficient herein a Clown, and his state
Clownishness.

III. In respect of what is pleasant in daily life: He that is as he
should be may be called Friendly, and his mean state Friendliness: he
that exceeds, if it be without any interested motive, somewhat too
Complaisant, if with such motive, a Flatterer: he that is deficient and
in all instances unpleasant, Quarrelsome and Cross.

There are mean states likewise in feelings and matters concerning them.
Shamefacedness, for instance, is no virtue, still a man is praised for
being shamefaced: for in these too the one is denominated the man in the
mean state, the other in the excess; the Dumbfoundered, for instance,
who is overwhelmed with shame on all and any occasions: the man who is
in the defect, _i.e._ who has no shame at all in his composition, is
called Shameless: but the right character Shamefaced.

Indignation against successful vice, again, is a state in the mean
between Envy and Malevolence: they all three have respect to pleasure
and pain produced by what happens to one's neighbour: for the man who
has this right feeling is annoyed at undeserved success of others, while
the envious man goes beyond him and is annoyed at all success of others,
and the malevolent falls so far short of feeling annoyance that he even
rejoices [at misfortune of others].

But for the discussion of these also there will be another opportunity,
as of Justice too, because the term is used in more senses than one. So
after this we will go accurately into each and say how they are mean
states: and in like manner also with respect to the Intellectual
Excellences.

Now as there are three states in each case, two faulty either in the way
of excess or defect, and one right, which is the mean state, of course
all are in a way opposed to one another; the extremes, for instance, not
only to the mean but also to one another, and the mean to the extremes:
for just as the half is greater if compared with the less portion, and
less if compared with the greater, so the mean states, compared with the
defects, exceed, whether in feelings or actions, and _vice versa_. The
brave man, for instance, shows as rash when compared with the coward,
and cowardly when compared with the rash; similarly too the man of
perfected self-mastery, viewed in comparison with the man destitute of
all perception, shows like a man of no self-control, but in comparison
with the man who really has no self-control, he looks like one destitute
of all perception: and the liberal man compared with the stingy seems
prodigal, and by the side of the prodigal, stingy.

And so the extreme characters push away, so to speak, towards each other
the man in the mean state; the brave man is called a rash man by
the coward, and a coward by the rash man, and in the other cases
accordingly. And there being this mutual opposition, the contrariety
between the extremes is greater than between either and the mean,
because they are further from one another than from the mean, just as
the greater or less portion differ more from each other than either from
the exact half.

Again, in some cases an extreme will bear a resemblance to the mean;
rashness, for instance, to courage, and prodigality to liberality; but
between the extremes there is the greatest dissimilarity. Now things
which are furthest from one another are defined to be contrary, and so
the further off the more contrary will they be.

[Sidenote: 1109a] Further: of the extremes in some cases the excess,
and in others the defect, is most opposed to the mean: to courage, for
instance, not rashness which is the excess, but cowardice which is the
defect; whereas to perfected self-mastery not insensibility which is the
defect but absence of all self-control which is the excess.

And for this there are two reasons to be given; one from the nature of
the thing itself, because from the one extreme being nearer and more
like the mean, we do not put this against it, but the other; as, for
instance, since rashness is thought to be nearer to courage than
cowardice is, and to resemble it more, we put cowardice against courage
rather than rashness, because those things which are further from the
mean are thought to be more contrary to it. This then is one reason
arising from the thing itself; there is another arising from our own
constitution and make: for in each man's own case those things give the
impression of being more contrary to the mean to which we individually
have a natural bias. Thus we have a natural bias towards pleasures,
for which reason we are much more inclined to the rejection of all
self-control, than to self-discipline.

These things then to which the bias is, we call more contrary, and so
total want of self-control (the excess) is more contrary than the defect
is to perfected self-mastery.


IX

Now that Moral Virtue is a mean state, and how it is so, and that it
lies between two faulty states, one in the way of excess and another in
the way of defect, and that it is so because it has an aptitude to aim
at the mean both in feelings and actions, all this has been set forth
fully and sufficiently.

And so it is hard to be good: for surely hard it is in each instance to
find the mean, just as to find the mean point or centre of a circle is
not what any man can do, but only he who knows how: just so to be angry,
to give money, and be expensive, is what any man can do, and easy: but
to do these to the right person, in due proportion, at the right time,
with a right object, and in the right manner, this is not as before what
any man can do, nor is it easy; and for this cause goodness is rare, and
praiseworthy, and noble.

Therefore he who aims at the mean should make it his first care to keep
away from that extreme which is more contrary than the other to the
mean; just as Calypso in Homer advises Ulysses,

  "Clear of this smoke and surge thy barque direct;"

because of the two extremes the one is always more, and the other
less, erroneous; and, therefore, since to hit exactly on the mean is
difficult, one must take the least of the evils as the safest plan; and
this a man will be doing, if he follows this method.

[Sidenote: 1109b] We ought also to take into consideration our own
natural bias; which varies in each man's case, and will be ascertained
from the pleasure and pain arising in us. Furthermore, we should force
ourselves off in the contrary direction, because we shall find ourselves
in the mean after we have removed ourselves far from the wrong side,
exactly as men do in straightening bent timber.

But in all cases we must guard most carefully against what is pleasant,
and pleasure itself, because we are not impartial judges of it.

We ought to feel in fact towards pleasure as did the old counsellors
towards Helen, and in all cases pronounce a similar sentence; for so by
sending it away from us, we shall err the less.

Well, to speak very briefly, these are the precautions by adopting which
we shall be best able to attain the mean.

Still, perhaps, after all it is a matter of difficulty, and specially
in the particular instances: it is not easy, for instance, to determine
exactly in what manner, with what persons, for what causes, and for what
length of time, one ought to feel anger: for we ourselves sometimes
praise those who are defective in this feeling, and we call them meek;
at another, we term the hot-tempered manly and spirited.

Then, again, he who makes a small deflection from what is right, be it
on the side of too much or too little, is not blamed, only he who makes
a considerable one; for he cannot escape observation. But to what point
or degree a man must err in order to incur blame, it is not easy to
determine exactly in words: nor in fact any of those points which are
matter of perception by the Moral Sense: such questions are matters of
detail, and the decision of them rests with the Moral Sense.

At all events thus much is plain, that the mean state is in all things
praiseworthy, and that practically we must deflect sometimes towards
excess sometimes towards defect, because this will be the easiest method
of hitting on the mean, that is, on what is right.




BOOK III

I Now since Virtue is concerned with the regulation of feelings and
actions, and praise and blame arise upon such as are voluntary, while
for the involuntary allowance is made, and sometimes compassion is
excited, it is perhaps a necessary task for those who are investigating
the nature of Virtue to draw out the distinction between what is
voluntary and what involuntary; and it is certainly useful for
legislators, with respect to the assigning of honours and punishments.



III

Involuntary actions then are thought to be of two kinds, being
done either on compulsion, or by reason of ignorance. An action is,
properly speaking, compulsory, when the origination is external to the
agent, being such that in it the agent (perhaps we may more properly
say the patient) contributes nothing; as if a wind were to convey you
anywhere, or men having power over your person.

But when actions are done, either from fear of greater evils, or from
some honourable motive, as, for instance, if you were ordered to commit
some base act by a despot who had your parents or children in his power,
and they were to be saved upon your compliance or die upon your refusal,
in such cases there is room for a question whether the actions are
voluntary or involuntary.

A similar question arises with respect to cases of throwing goods
overboard in a storm: abstractedly no man throws away his property
willingly, but with a view to his own and his shipmates' safety any one
would who had any sense.

The truth is, such actions are of a mixed kind, but are most like
voluntary actions; for they are choiceworthy at the time when they are
being done, and the end or object of the action must be taken with
reference to the actual occasion. Further, we must denominate an action
voluntary or involuntary at the time of doing it: now in the given case
the man acts voluntarily, because the originating of the motion of his
limbs in such actions rests with himself; and where the origination is
in himself it rests with himself to do or not to do.

Such actions then are voluntary, though in the abstract perhaps
involuntary because no one would choose any of such things in and by
itself.

But for such actions men sometimes are even praised, as when they endure
any disgrace or pain to secure great and honourable equivalents; if
_vice versâ_, then they are blamed, because it shows a base mind to
endure things very disgraceful for no honourable object, or for a
trifling one.

For some again no praise is given, but allowance is made; as where a
man does what he should not by reason of such things as overstrain the
powers of human nature, or pass the limits of human endurance.

Some acts perhaps there are for which compulsion cannot be pleaded, but
a man should rather suffer the worst and die; how absurd, for instance,
are the pleas of compulsion with which Alcmaeon in Euripides' play
excuses his matricide!

But it is difficult sometimes to decide what kind of thing should be
chosen instead of what, or what endured in preference to what, and much
moreso to abide by one's decisions: for in general the alternatives are
painful, and the actions required are base, and so praise or blame is
awarded according as persons have been compelled or no.

1110b What kind of actions then are to be called compulsory? may we say,
simply and abstractedly whenever the cause is external and the agent
contributes nothing; and that where the acts are in themselves such
as one would not wish but choiceworthy at the present time and in
preference to such and such things, and where the origination rests with
the agent, the actions are in themselves involuntary but at the given
time and in preference to such and such things voluntary; and they are
more like voluntary than involuntary, because the actions consist of
little details, and these are voluntary.

But what kind of things one ought to choose instead of what, it is not
easy to settle, for there are many differences in particular instances.

But suppose a person should say, things pleasant and honourable exert
a compulsive force (for that they are external and do compel); at that
rate every action is on compulsion, because these are universal motives
of action.

Again, they who act on compulsion and against their will do so with
pain; but they who act by reason of what is pleasant or honourable act
with pleasure.

It is truly absurd for a man to attribute his actions to external things
instead of to his own capacity for being easily caught by them; or,
again, to ascribe the honourable to himself, and the base ones to
pleasure.

So then that seems to be compulsory "whose origination is from without,
the party compelled contributing nothing." Now every action of which
ignorance is the cause is not-voluntary, but that only is involuntary
which is attended with pain and remorse; for clearly the man who has
done anything by reason of ignorance, but is not annoyed at his own
action, cannot be said to have done it _with_ his will because he did
not know he was doing it, nor again _against_ his will because he is not
sorry for it.

So then of the class "acting by reason of ignorance," he who feels
regret afterwards is thought to be an involuntary agent, and him that
has no such feeling, since he certainly is different from the other, we
will call a not-voluntary agent; for as there is a real difference it is
better to have a proper name.

Again, there seems to be a difference between acting _because of_
ignorance and acting _with_ ignorance: for instance, we do not usually
assign ignorance as the cause of the actions of the drunken or angry
man, but either the drunkenness or the anger, yet they act not knowingly
but with ignorance.

Again, every bad man is ignorant what he ought to do and what to leave
undone, and by reason of such error men become unjust and wholly evil.

[Sidenote: 1111a] Again, we do not usually apply the term involuntary
when a man is ignorant of his own true interest; because ignorance which
affects moral choice constitutes depravity but not involuntariness: nor
does any ignorance of principle (because for this men are blamed)
but ignorance in particular details, wherein consists the action and
wherewith it is concerned, for in these there is both compassion and
allowance, because he who acts in ignorance of any of them acts in a
proper sense involuntarily.

It may be as well, therefore, to define these particular details; what
they are, and how many; viz. who acts, what he is doing, with respect to
what or in what, sometimes with what, as with what instrument, and with
what result (as that of preservation, for instance), and how, as whether
softly or violently.

All these particulars, in one and the same case, no man in his senses
could be ignorant of; plainly not of the agent, being himself. But
what he is doing a man may be ignorant, as men in speaking say a
thing escaped them unawares; or as Aeschylus did with respect to the
Mysteries, that he was not aware that it was unlawful to speak of them;
or as in the case of that catapult accident the other day the man said
he discharged it merely to display its operation. Or a person might
suppose a son to be an enemy, as Merope did; or that the spear really
pointed was rounded off; or that the stone was a pumice; or in striking
with a view to save might kill; or might strike when merely wishing to
show another, as people do in sham-fighting.

Now since ignorance is possible in respect to all these details in
which the action consists, he that acted in ignorance of any of them is
thought to have acted involuntarily, and he most so who was in ignorance
as regards the most important, which are thought to be those in which
the action consists, and the result.

Further, not only must the ignorance be of this kind, to constitute an
action involuntary, but it must be also understood that the action is
followed by pain and regret.

Now since all involuntary action is either upon compulsion or by reason
of ignorance, Voluntary Action would seem to be "that whose origination
is in the agent, he being aware of the particular details in which the
action consists."

For, it may be, men are not justified by calling those actions
involuntary, which are done by reason of Anger or Lust.

Because, in the first place, if this be so no other animal but man, and
not even children, can be said to act voluntarily. Next, is it meant
that we never act voluntarily when we act from Lust or Anger, or that we
act voluntarily in doing what is right and involuntarily in doing what
is discreditable? The latter supposition is absurd, since the cause
is one and the same. Then as to the former, it is a strange thing to
maintain actions to be involuntary which we are bound to grasp at: now
there are occasions on which anger is a duty, and there are things which
we are bound to lust after, health, for instance, and learning.

Again, whereas actions strictly involuntary are thought to be attended
with pain, those which are done to gratify lust are thought to be
pleasant.

Again: how does the involuntariness make any difference between wrong
actions done from deliberate calculation, and those done by reason of
anger? for both ought to be avoided, and the irrational feelings are
thought to be just as natural to man as reason, and so of course must be
such actions of the individual as are done from Anger and Lust. It is
absurd then to class these actions among the involuntary.

II

Having thus drawn out the distinction between voluntary and involuntary
action our next step is to examine into the nature of Moral Choice,
because this seems most intimately connected with Virtue and to be a
more decisive test of moral character than a man's acts are.

Now Moral Choice is plainly voluntary, but the two are not co-extensive,
voluntary being the more comprehensive term; for first, children and all
other animals share in voluntary action but not in Moral Choice; and
next, sudden actions we call voluntary but do not ascribe them to Moral
Choice.

Nor do they appear to be right who say it is lust or anger, or wish, or
opinion of a certain kind; because, in the first place, Moral Choice is
not shared by the irrational animals while Lust and Anger are. Next; the
man who fails of self-control acts from Lust but not from Moral Choice;
the man of self-control, on the contrary, from Moral Choice, not from
Lust. Again: whereas Lust is frequently opposed to Moral Choice, Lust is
not to Lust.

Lastly: the object-matter of Lust is the pleasant and the painful, but
of Moral Choice neither the one nor the other. Still less can it be
Anger, because actions done from Anger are thought generally to be least
of all consequent on Moral Choice.

Nor is it Wish either, though appearing closely connected with it;
because, in the first place, Moral Choice has not for its objects
impossibilities, and if a man were to say he chose them he would be
thought to be a fool; but Wish may have impossible things for its
objects, immortality for instance.

Wish again may be exercised on things in the accomplishment of which
one's self could have nothing to do, as the success of any particular
actor or athlete; but no man chooses things of this nature, only such as
he believes he may himself be instrumental in procuring.

Further: Wish has for its object the End rather, but Moral Choice the
means to the End; for instance, we wish to be healthy but we choose
the means which will make us so; or happiness again we wish for, and
commonly say so, but to say we choose is not an appropriate term,
because, in short, the province of Moral Choice seems to be those things
which are in our own power.

Neither can it be Opinion; for Opinion is thought to be unlimited in its
range of objects, and to be exercised as well upon things eternal and
impossible as on those which are in our own power: again, Opinion is
logically divided into true and false, not into good and bad as Moral
Choice is.

However, nobody perhaps maintains its identity with Opinion simply; but
it is not the same with opinion of any kind, because by choosing good
and bad things we are constituted of a certain character, but by having
opinions on them we are not.

Again, we choose to take or avoid, and so on, but we opine what a thing
is, or for what it is serviceable, or how; but we do not opine to take
or avoid.

Further, Moral Choice is commended rather for having a right object than
for being judicious, but Opinion for being formed in accordance with
truth.

Again, we choose such things as we pretty well know to be good, but we
form opinions respecting such as we do not know at all.

And it is not thought that choosing and opining best always go together,
but that some opine the better course and yet by reason of viciousness
choose not the things which they should.

It may be urged, that Opinion always precedes or accompanies Moral
Choice; be it so, this makes no difference, for this is not the point in
question, but whether Moral Choice is the same as Opinion of a certain
kind.

Since then it is none of the aforementioned things, what is it, or how
is it characterised? Voluntary it plainly is, but not all voluntary
action is an object of Moral Choice. May we not say then, it is "that
voluntary which has passed through a stage of previous deliberation?"
because Moral Choice is attended with reasoning and intellectual
process. The etymology of its Greek name seems to give a hint of it,
being when analysed "chosen in preference to somewhat else."


III

Well then; do men deliberate about everything, and is anything soever
the object of Deliberation, or are there some matters with respect to
which there is none? (It may be as well perhaps to say, that by "object
of Deliberation" is meant such matter as a sensible man would deliberate
upon, not what any fool or madman might.)

Well: about eternal things no one deliberates; as, for instance, the
universe, or the incommensurability of the diameter and side of a
square.

Nor again about things which are in motion but which always happen in
the same way either necessarily, or naturally, or from some other cause,
as the solstices or the sunrise.

Nor about those which are variable, as drought and rains; nor fortuitous
matters, as finding of treasure.

Nor in fact even about all human affairs; no Lacedæmonian, for instance,
deliberates as to the best course for the Scythian government to adopt;
because in such cases we have no power over the result.

But we do deliberate respecting such practical matters as are in our own
power (which are what are left after all our exclusions).

I have adopted this division because causes seem to be divisible into
nature, necessity, chance, and moreover intellect, and all human powers.

And as man in general deliberates about what man in general can effect,
so individuals do about such practical things as can be effected through
their own instrumentality.

[Sidenote: 1112b] Again, we do not deliberate respecting such arts or
sciences as are exact and independent: as, for instance, about written
characters, because we have no doubt how they should be formed; but we
do deliberate on all buch things as are usually done through our own
instrumentality, but not invariably in the same way; as, for instance,
about matters connected with the healing art, or with money-making; and,
again, more about piloting ships than gymnastic exercises, because the
former has been less exactly determined, and so forth; and more about
arts than sciences, because we more frequently doubt respecting the
former.

So then Deliberation takes place in such matters as are under general
laws, but still uncertain how in any given case they will issue,
_i.e._ in which there is some indefiniteness; and for great matters we
associate coadjutors in counsel, distrusting our ability to settle them
alone.

Further, we deliberate not about Ends, but Means to Ends. No physician,
for instance, deliberates whether he will cure, nor orator whether
he will persuade, nor statesman whether he will produce a good
constitution, nor in fact any man in any other function about his
particular End; but having set before them a certain End they look how
and through what means it may be accomplished: if there is a choice of
means, they examine further which are easiest and most creditable; or,
if there is but one means of accomplishing the object, then how it may
be through this, this again through what, till they come to the first
cause; and this will be the last found; for a man engaged in a process
of deliberation seems to seek and analyse, as a man, to solve a
problem, analyses the figure given him. And plainly not every search is
Deliberation, those in mathematics to wit, but every Deliberation is
a search, and the last step in the analysis is the first in the
constructive process. And if in the course of their search men come upon
an impossibility, they give it up; if money, for instance, be necessary,
but cannot be got: but if the thing appears possible they then attempt
to do it.

And by possible I mean what may be done through our own instrumentality
(of course what may be done through our friends is through our own
instrumentality in a certain sense, because the origination in such
cases rests with us). And the object of search is sometimes the
necessary instruments, sometimes the method of using them; and similarly
in the rest sometimes through what, and sometimes how or through what.

So it seems, as has been said, that Man is the originator of his
actions; and Deliberation has for its object whatever may be done
through one's own instrumentality, and the actions are with a view to
other things; and so it is, not the End, but the Means to Ends on which
Deliberation is employed.

[Sidenote: III3a]

Nor, again, is it employed on matters of detail, as whether the
substance before me is bread, or has been properly cooked; for these
come under the province of sense, and if a man is to be always
deliberating, he may go on _ad infinitum_.

Further, exactly the same matter is the object both of Deliberation
and Moral Choice; but that which is the object of Moral Choice is
thenceforward separated off and definite, because by object of Moral
Choice is denoted that which after Deliberation has been preferred to
something else: for each man leaves off searching how he shall do a
thing when he has brought the origination up to himself, _i.e_. to the
governing principle in himself, because it is this which makes the
choice. A good illustration of this is furnished by the old regal
constitutions which Homer drew from, in which the Kings would announce
to the commonalty what they had determined before.

Now since that which is the object of Moral Choice is something in our
own power, which is the object of deliberation and the grasping of the
Will, Moral Choice must be "a grasping after something in our own power
consequent upon Deliberation:" because after having deliberated we
decide, and then grasp by our Will in accordance with the result of our
deliberation.

Let this be accepted as a sketch of the nature and object of Moral
Choice, that object being "Means to Ends."

[Sidenote: IV] That Wish has for its object-matter the End, has been
already stated; but there are two opinions respecting it; some thinking
that its object is real good, others whatever impresses the mind with a
notion of good.

Now those who maintain that the object of Wish is real good are beset by
this difficulty, that what is wished for by him who chooses wrongly is
not really an object of Wish (because, on their theory, if it is an
object of wish, it must be good, but it is, in the case supposed, evil).
Those who maintain, on the contrary, that that which impresses the mind
with a notion of good is properly the object of Wish, have to meet this
difficulty, that there is nothing naturally an object of Wish but to
each individual whatever seems good to him; now different people have
different notions, and it may chance contrary ones.

But, if these opinions do not satisfy us, may we not say that,
abstractedly and as a matter of objective truth, the really good is the
object of Wish, but to each individual whatever impresses his mind with
the notion of good. And so to the good man that is an object of Wish
which is really and truly so, but to the bad man anything may be; just
as physically those things are wholesome to the healthy which are really
so, but other things to the sick. And so too of bitter and sweet, and
hot and heavy, and so on. For the good man judges in every instance
correctly, and in every instance the notion conveyed to his mind is the
true one.

For there are fair and pleasant things peculiar to, and so varying with,
each state; and perhaps the most distinguishing characteristic of the
good man is his seeing the truth in every instance, he being, in fact,
the rule and measure of these matters.

The multitude of men seem to be deceived by reason of pleasure, because
though it is not really a good it impresses their minds with the notion
of goodness, so they choose what is pleasant as good and avoid pain as
an evil.

Now since the End is the object of Wish, and the means to the End of
Deliberation and Moral Choice, the actions regarding these matters
must be in the way of Moral Choice, _i.e._ voluntary: but the acts of
working out the virtues are such actions, and therefore Virtue is in our
power.

And so too is Vice: because wherever it is in our power to do it is also
in our power to forbear doing, and _vice versâ_: therefore if the doing
(being in a given case creditable) is in our power, so too is the
forbearing (which is in the same case discreditable), and _vice versâ_.

But if it is in our power to do and to forbear doing what is creditable
or the contrary, and these respectively constitute the being good or
bad, then the being good or vicious characters is in our power.

As for the well-known saying, "No man voluntarily is wicked or
involuntarily happy," it is partly true, partly false; for no man is
happy against his will, of course, but wickedness is voluntary. Or must
we dispute the statements lately made, and not say that Man is the
originator or generator of his actions as much as of his children?

But if this is matter of plain manifest fact, and we cannot refer our
actions to any other originations beside those in our own power, those
things must be in our own power, and so voluntary, the originations of
which are in ourselves.

Moreover, testimony seems to be borne to these positions both privately
by individuals, and by law-givers too, in that they chastise and punish
those who do wrong (unless they do so on compulsion, or by reason of
ignorance which is not self-caused), while they honour those who act
rightly, under the notion of being likely to encourage the latter and
restrain the former. But such things as are not in our own power, _i.e._
not voluntary, no one thinks of encouraging us to do, knowing it to be
of no avail for one to have been persuaded not to be hot (for instance),
or feel pain, or be hungry, and so forth, because we shall have those
sensations all the same.

And what makes the case stronger is this: that they chastise for the
very fact of ignorance, when it is thought to be self-caused; to the
drunken, for instance, penalties are double, because the origination in
such case lies in a man's own self: for he might have helped getting
drunk, and this is the cause of his ignorance.

[Sidenote: III4_a_] Again, those also who are ignorant of legal
regulations which they are bound to know, and which are not hard to
know, they chastise; and similarly in all other cases where neglect is
thought to be the cause of the ignorance, under the notion that it was
in their power to prevent their ignorance, because they might have paid
attention.

But perhaps a man is of such a character that he cannot attend to such
things: still men are themselves the causes of having become such
characters by living carelessly, and also of being unjust or destitute
of self-control, the former by doing evil actions, the latter by
spending their time in drinking and such-like; because the particular
acts of working form corresponding characters, as is shown by those who
are practising for any contest or particular course of action, for such
men persevere in the acts of working.

As for the plea, that a man did not know that habits are produced
from separate acts of working, we reply, such ignorance is a mark of
excessive stupidity.

Furthermore, it is wholly irrelevant to say that the man who acts
unjustly or dissolutely does not _wish_ to attain the habits of these
vices: for if a man wittingly does those things whereby he must become
unjust he is to all intents and purposes unjust voluntarily; but he
cannot with a wish cease to be unjust and become just. For, to take the
analogous case, the sick man cannot with a wish be well again, yet in
a supposable case he is voluntarily ill because he has produced his
sickness by living intemperately and disregarding his physicians. There
was a time then when he might have helped being ill, but now he has let
himself go he cannot any longer; just as he who has let a stone out of
his hand cannot recall it, and yet it rested with him to aim and throw
it, because the origination was in his power. Just so the unjust man,
and he who has lost all self-control, might originally have helped being
what they are, and so they are voluntarily what they are; but now that
they are become so they no longer have the power of being otherwise.

And not only are mental diseases voluntary, but the bodily are so in
some men, whom we accordingly blame: for such as are naturally deformed
no one blames, only such as are so by reason of want of exercise, and
neglect: and so too of weakness and maiming: no one would think of
upbraiding, but would rather compassionate, a man who is blind by
nature, or from disease, or from an accident; but every one would blame
him who was so from excess of wine, or any other kind of intemperance.
It seems, then, that in respect of bodily diseases, those which depend
on ourselves are censured, those which do not are not censured; and if
so, then in the case of the mental disorders, those which are censured
must depend upon ourselves.

[Sidenote: III4_b_] But suppose a man to say, "that (by our own
admission) all men aim at that which conveys to their minds an
impression of good, and that men have no control over this impression,
but that the End impresses each with a notion correspondent to his own
individual character; that to be sure if each man is in a way the cause
of his own moral state, so he will be also of the kind of impression he
receives: whereas, if this is not so, no one is the cause to himself of
doing evil actions, but he does them by reason of ignorance of the true
End, supposing that through their means he will secure the chief good.
Further, that this aiming at the End is no matter of one's own choice,
but one must be born with a power of mental vision, so to speak, whereby
to judge fairly and choose that which is really good; and he is blessed
by nature who has this naturally well: because it is the most important
thing and the fairest, and what a man cannot get or learn from another
but will have such as nature has given it; and for this to be so given
well and fairly would be excellence of nature in the highest and truest
sense."

If all this be true, how will Virtue be a whit more voluntary than Vice?
Alike to the good man and the bad, the End gives its impression and is
fixed by nature or howsoever you like to say, and they act so and so,
referring everything else to this End.

Whether then we suppose that the End impresses each man's mind with
certain notions not merely by nature, but that there is somewhat also
dependent on himself; or that the End is given by nature, and yet Virtue
is voluntary because the good man does all the rest voluntarily, Vice
must be equally so; because his own agency equally attaches to the bad
man in the actions, even if not in the selection of the End.

If then, as is commonly said, the Virtues are voluntary (because we at
least co-operate in producing our moral states, and we assume the End
to be of a certain kind according as we are ourselves of certain
characters), the Vices must be voluntary also, because the cases are
exactly similar.

Well now, we have stated generally respecting the Moral Virtues, the
genus (in outline), that they are mean states, and that they are habits,
and how they are formed, and that they are of themselves calculated to
act upon the circumstances out of which they were formed, and that they
are in our own power and voluntary, and are to be done so as right
Reason may direct.

[Sidenote: III5_a_] But the particular actions and the habits are not
voluntary in the same sense; for of the actions we are masters from
beginning to end (supposing of course a knowledge of the particular
details), but only of the origination of the habits, the addition by
small particular accessions not being cognisiable (as is the case with
sicknesses): still they are voluntary because it rested with us to use
our circumstances this way or that.

Here we will resume the particular discussion of the Moral Virtues,
and say what they are, what is their object-matter, and how they stand
respectively related to it: of course their number will be thereby
shown. First, then, of Courage. Now that it is a mean state, in respect
of fear and boldness, has been already said: further, the objects of our
fears are obviously things fearful or, in a general way of statement,
evils; which accounts for the common definition of fear, viz.
"expectation of evil."

Of course we fear evils of all kinds: disgrace, for instance, poverty,
disease, desolateness, death; but not all these seem to be the
object-matter of the Brave man, because there are things which to fear
is right and noble, and not to fear is base; disgrace, for example,
since he who fears this is a good man and has a sense of honour, and he
who does not fear it is shameless (though there are those who call him
Brave by analogy, because he somewhat resembles the Brave man who agrees
with him in being free from fear); but poverty, perhaps, or disease, and
in fact whatever does not proceed from viciousness, nor is attributable
to his own fault, a man ought not to fear: still, being fearless in
respect of these would not constitute a man Brave in the proper sense of
the term.

Yet we do apply the term in right of the similarity of the cases; for
there are men who, though timid in the dangers of war, are liberal men
and are stout enough to face loss of wealth.

And, again, a man is not a coward for fearing insult to his wife or
children, or envy, or any such thing; nor is he a Brave man for being
bold when going to be scourged.

What kind of fearful things then do constitute the object-matter of the
Brave man? first of all, must they not be the greatest, since no man is
more apt to withstand what is dreadful. Now the object of the greatest
dread is death, because it is the end of all things, and the dead man is
thought to be capable neither of good nor evil. Still it would seem
that the Brave man has not for his object-matter even death in every
circumstance; on the sea, for example, or in sickness: in what
circumstances then? must it not be in the most honourable? now such is
death in war, because it is death in the greatest and most honourable
danger; and this is confirmed by the honours awarded in communities, and
by monarchs.

He then may be most properly denominated Brave who is fearless in
respect of honourable death and such sudden emergencies as threaten
death; now such specially are those which arise in the course of war.

[Sidenote: 1115b] It is not meant but that the Brave man will be
fearless also on the sea (and in sickness), but not in the same way as
sea-faring men; for these are light-hearted and hopeful by reason of
their experience, while landsmen though Brave are apt to give themselves
up for lost and shudder at the notion of such a death: to which it
should be added that Courage is exerted in circumstances which admit
of doing something to help one's self, or in which death would be
honourable; now neither of these requisites attach to destruction by
drowning or sickness.



VII


Again, fearful is a term of relation, the same thing not being so to
all, and there is according to common parlance somewhat so fearful as to
be beyond human endurance: this of course would be fearful to every
man of sense, but those objects which are level to the capacity of
man differ in magnitude and admit of degrees, so too the objects of
confidence or boldness.

Now the Brave man cannot be frighted from his propriety (but of course
only so far as he is man); fear such things indeed he will, but he will
stand up against them as he ought and as right reason may direct, with a
view to what is honourable, because this is the end of the virtue.

Now it is possible to fear these things too much, or too little, or
again to fear what is not really fearful as if it were such. So the
errors come to be either that a man fears when he ought not to fear at
all, or that he fears in an improper way, or at a wrong time, and so
forth; and so too in respect of things inspiring confidence. He is
Brave then who withstands, and fears, and is bold, in respect of right
objects, from a right motive, in right manner, and at right times:
since the Brave man suffers or acts as he ought and as right reason may
direct.

Now the end of every separate act of working is that which accords
with the habit, and so to the Brave man Courage; which is honourable;
therefore such is also the End, since the character of each is
determined by the End.

So honour is the motive from which the Brave man withstands things
fearful and performs the acts which accord with Courage.

Of the characters on the side of Excess, he who exceeds in utter absence
of fear has no appropriate name (I observed before that many states have
none), but he would be a madman or inaccessible to pain if he feared
nothing, neither earthquake, nor the billows, as they tell of the Celts.

He again who exceeds in confidence in respect of things fearful is rash.
He is thought moreover to be a braggart, and to advance unfounded claims
to the character of Brave: the relation which the Brave man really bears
to objects of fear this man wishes to appear to bear, and so imitates
him in whatever points he can; for this reason most of them exhibit a
curious mixture of rashness and cowardice; because, affecting rashness
in these circumstances, they do not withstand what is truly fearful.

[Sidenote: III6_a_] The man moreover who exceeds in feeling fear is a
coward, since there attach to him the circumstances of fearing wrong
objects, in wrong ways, and so forth. He is deficient also in feeling
confidence, but he is most clearly seen as exceeding in the case of
pains; he is a fainthearted kind of man, for he fears all things: the
Brave man is just the contrary, for boldness is the property of the
light-hearted and hopeful.

So the coward, the rash, and the Brave man have exactly the same
object-matter, but stand differently related to it: the two
first-mentioned respectively exceed and are deficient, the last is in a
mean state and as he ought to be. The rash again are precipitate, and,
being eager before danger, when actually in it fall away, while the
Brave are quick and sharp in action, but before are quiet and composed.

Well then, as has been said, Courage is a mean state in respect of
objects inspiring boldness or fear, in the circumstances which have been
stated, and the Brave man chooses his line and withstands danger either
because to do so is honourable, or because not to do so is base. But
dying to escape from poverty, or the pangs of love, or anything that is
simply painful, is the act not of a Brave man but of a coward; because
it is mere softness to fly from what is toilsome, and the suicide braves
the terrors of death not because it is honourable but to get out of the
reach of evil.


VIII

Courage proper is somewhat of the kind I have described, but there are
dispositions, differing in five ways, which also bear in common parlance
the name of Courage.

We will take first that which bears most resemblance to the true, the
Courage of Citizenship, so named because the motives which are thought
to actuate the members of a community in braving danger are the
penalties and disgrace held out by the laws to cowardice, and the
dignities conferred on the Brave; which is thought to be the reason
why those are the bravest people among whom cowards are visited with
disgrace and the Brave held in honour.

Such is the kind of Courage Homer exhibits in his characters; Diomed and
Hector for example. The latter says,

 "Polydamas will be the first to fix
  Disgrace upon me."

Diomed again,

 "For Hector surely will hereafter say,
  Speaking in Troy, Tydides by my hand"--

This I say most nearly resembles the Courage before spoken of, because
it arises from virtue, from a feeling of shame, and a desire of what is
noble (that is, of honour), and avoidance of disgrace which is base. In
the same rank one would be inclined to place those also who act under
compulsion from their commanders; yet are they really lower, because not
a sense of honour but fear is the motive from which they act, and what
they seek to avoid is not that which is base but that which is simply
painful: commanders do in fact compel their men sometimes, as Hector
says (to quote Homer again),

  "But whomsoever I shall find cowering afar from the fight,
  The teeth of dogs he shall by no means escape."

[Sidenote: III6_h_] Those commanders who station staunch troops by
doubtful ones, or who beat their men if they flinch, or who draw their
troops up in line with the trenches, or other similar obstacles,
in their rear, do in effect the same as Hector, for they all use
compulsion.

But a man is to be Brave, not on compulsion, but from a sense of honour.

In the next place, Experience and Skill in the various particulars is
thought to be a species of Courage: whence Socrates also thought that
Courage was knowledge.

This quality is exhibited of course by different men under different
circumstances, but in warlike matters, with which we are now concerned,
it is exhibited by the soldiers ("the regulars"): for there are, it
would seem, many things in war of no real importance which these have
been constantly used to see; so they have a show of Courage because
other people are not aware of the real nature of these things. Then
again by reason of their skill they are better able than any others to
inflict without suffering themselves, because they are able to use their
arms and have such as are most serviceable both with a view to offence
and defence: so that their case is parallel to that of armed men
fighting with unarmed or trained athletes with amateurs, since in
contests of this kind those are the best fighters, not who are the
bravest men, but who are the strongest and are in the best condition.

In fact, the regular troops come to be cowards whenever the danger is
greater than their means of meeting it; supposing, for example, that
they are inferior in numbers and resources: then they are the first to
fly, but the mere militia stand and fall on the ground (which as you
know really happened at the Hermæum), for in the eyes of these flight
was disgraceful and death preferable to safety bought at such a price:
while "the regulars" originally went into the danger under a notion
of their own superiority, but on discovering their error they took to
flight, having greater fear of death than of disgrace; but this is not
the feeling of the Brave man.

Thirdly, mere Animal Spirit is sometimes brought under the term Courage:
they are thought to be Brave who are carried on by mere Animal Spirit,
as are wild beasts against those who have wounded them, because in fact
the really Brave have much Spirit, there being nothing like it for going
at danger of any kind; whence those frequent expressions in Homer,
"infused strength into his spirit," "roused his strength and spirit," or
again, "and keen strength in his nostrils," "his blood boiled:" for all
these seem to denote the arousing and impetuosity of the Animal Spirit.

[Sidenote: III7_a_] Now they that are truly Brave act from a sense of
honour, and this Animal Spirit co-operates with them; but wild beasts
from pain, that is because they have been wounded, or are frightened;
since if they are quietly in their own haunts, forest or marsh, they do
not attack men. Surely they are not Brave because they rush into danger
when goaded on by pain and mere Spirit, without any view of the danger:
else would asses be Brave when they are hungry, for though beaten they
will not then leave their pasture: profligate men besides do many bold
actions by reason of their lust. We may conclude then that they are not
Brave who are goaded on to meet danger by pain and mere Spirit; but
still this temper which arises from Animal Spirit appears to be most
natural, and would be Courage of the true kind if it could have added
to it moral choice and the proper motive. So men also are pained by a
feeling of anger, and take pleasure in revenge; but they who fight from
these causes may be good fighters, but they are not truly Brave (in
that they do not act from a sense of honour, nor as reason directs, but
merely from the present feeling), still they bear some resemblance to
that character.

Nor, again, are the Sanguine and Hopeful therefore Brave: since their
boldness in dangers arises from their frequent victories over numerous
foes. The two characters are alike, however, in that both are confident;
but then the Brave are so from the afore-mentioned causes, whereas these
are so from a settled conviction of their being superior and not likely
to suffer anything in return (they who are intoxicated do much the
same, for they become hopeful when in that state); but when the event
disappoints their expectations they run away: now it was said to be the
character of a Brave man to withstand things which are fearful to man
or produce that impression, because it is honourable so to do and the
contrary is dishonourable.

For this reason it is thought to be a greater proof of Courage to be
fearless and undisturbed under the pressure of sudden fear than under
that which may be anticipated, because Courage then comes rather from a
fixed habit, or less from preparation: since as to foreseen dangers a
man might take his line even from calculation and reasoning, but in
those which are sudden he will do so according to his fixed habit of
mind.

Fifthly and lastly, those who are acting under Ignorance have a show
of Courage and are not very far from the Hopeful; but still they are
inferior inasmuch as they have no opinion of themselves; which the
others have, and therefore stay and contest a field for some little
time; but they who have been deceived fly the moment they know things to
be otherwise than they supposed, which the Argives experienced when they
fell on the Lacedæmonians, taking them for the men of Sicyon. We have
described then what kind of men the Brave are, and what they who are
thought to be, but are not really, Brave.

[Sidenote: IX]

It must be remarked, however, that though Courage has for its
object-matter boldness and fear it has not both equally so, but objects
of fear much more than the former; for he that under pressure of these
is undisturbed and stands related to them as he ought is better entitled
to the name of Brave than he who is properly affected towards objects
of confidence. So then men are termed Brave for withstanding painful
things.

It follows that Courage involves pain and is justly praised, since it
is a harder matter to withstand things that are painful than to abstain
from such as are pleasant.

[Sidenote: 1117_b_]

It must not be thought but that the End and object of Courage is
pleasant, but it is obscured by the surrounding circumstances: which
happens also in the gymnastic games; to the boxers the End is pleasant
with a view to which they act, I mean the crown and the honours; but the
receiving the blows they do is painful and annoying to flesh and blood,
and so is all the labour they have to undergo; and, as these drawbacks
are many, the object in view being small appears to have no pleasantness
in it.

If then we may say the same of Courage, of course death and wounds must
be painful to the Brave man and against his will: still he endures these
because it is honourable so to do or because it is dishonourable not to
do so. And the more complete his virtue and his happiness so much the
more will he be pained at the notion of death: since to such a man as
he is it is best worth while to live, and he with full consciousness is
deprived of the greatest goods by death, and this is a painful idea. But
he is not the less Brave for feeling it to be so, nay rather it may be
he is shown to be more so because he chooses the honour that may be
reaped in war in preference to retaining safe possession of these other
goods. The fact is that to act with pleasure does not belong to all the
virtues, except so far as a man realises the End of his actions.

But there is perhaps no reason why not such men should make the best
soldiers, but those who are less truly Brave but have no other good to
care for: these being ready to meet danger and bartering their lives
against small gain.

Let thus much be accepted as sufficient on the subject of Courage; the
true nature of which it is not difficult to gather, in outline at least,
from what has been said.

[Sidenote: X]

Next let us speak of Perfected Self-Mastery, which seems to claim the
next place to Courage, since these two are the Excellences of the
Irrational part of the Soul.

That it is a mean state, having for its object-matter Pleasures, we have
already said (Pains being in fact its object-matter in a less degree
and dissimilar manner), the state of utter absence of self-control has
plainly the same object-matter; the next thing then is to determine what
kind of Pleasures.

Let Pleasures then be understood to be divided into mental and bodily:
instances of the former being love of honour or of learning: it being
plain that each man takes pleasure in that of these two objects which he
has a tendency to like, his body being no way affected but rather his
intellect. Now men are not called perfectly self-mastering or wholly
destitute of self-control in respect of pleasures of this class: nor in
fact in respect of any which are not bodily; those for example who love
to tell long stories, and are prosy, and spend their days about
mere chance matters, we call gossips but not wholly destitute of
self-control, nor again those who are pained at the loss of money or
friends.

[Sidenote: 1118_a_]

It is bodily Pleasures then which are the object-matter of Perfected
Self-Mastery, but not even all these indifferently: I mean, that they
who take pleasure in objects perceived by the Sight, as colours, and
forms, and painting, are not denominated men of Perfected Self-Mastery,
or wholly destitute of self-control; and yet it would seem that one may
take pleasure even in such objects, as one ought to do, or excessively,
or too little.

So too of objects perceived by the sense of Hearing; no one applies the
terms before quoted respectively to those who are excessively pleased
with musical tunes or acting, or to those who take such pleasure as they
ought.

Nor again to those persons whose pleasure arises from the sense
of Smell, except incidentally: I mean, we do not say men have no
self-control because they take pleasure in the scent of fruit, or
flowers, or incense, but rather when they do so in the smells of
unguents and sauces: since men destitute of self-control take pleasure
herein, because hereby the objects of their lusts are recalled to their
imagination (you may also see other men take pleasure in the smell of
food when they are hungry): but to take pleasure in such is a mark of
the character before named since these are objects of desire to him.

Now not even brutes receive pleasure in right of these senses, except
incidentally. I mean, it is not the scent of hares' flesh but the eating
it which dogs take pleasure in, perception of which pleasure is caused
by the sense of Smell. Or again, it is not the lowing of the ox but
eating him which the lion likes; but of the fact of his nearness the
lion is made sensible by the lowing, and so he appears to take pleasure
in this. In like manner, he has no pleasure in merely seeing or finding
a stag or wild goat, but in the prospect of a meal.

The habits of Perfect Self-Mastery and entire absence of self-control
have then for their object-matter such pleasures as brutes also share
in, for which reason they are plainly servile and brutish: they are
Touch and Taste.

But even Taste men seem to make little or no use of; for to the sense of
Taste belongs the distinguishing of flavours; what men do, in fact, who
are testing the quality of wines or seasoning "made dishes."

But men scarcely take pleasure at all in these things, at least those
whom we call destitute of self-control do not, but only in the actual
enjoyment which arises entirely from the sense of Touch, whether in
eating or in drinking, or in grosser lusts. This accounts for the wish
said to have been expressed once by a great glutton, "that his throat
had been formed longer than a crane's neck," implying that his pleasure
was derived from the Touch.

[Sidenote: 1118b] The sense then with which is connected the habit of
absence of self-control is the most common of all the senses, and this
habit would seem to be justly a matter of reproach, since it attaches to
us not in so far as we are men but in so far as we are animals. Indeed
it is brutish to take pleasure in such things and to like them best of
all; for the most respectable of the pleasures arising from the touch
have been set aside; those, for instance, which occur in the course of
gymnastic training from the rubbing and the warm bath: because the touch
of the man destitute of self-control is not indifferently of _any_ part
of the body but only of particular parts.

XI

Now of lusts or desires some are thought to be universal, others
peculiar and acquired; thus desire for food is natural since every one
who really needs desires also food, whether solid or liquid, or both
(and, as Homer says, the man in the prime of youth needs and desires
intercourse with the other sex); but when we come to this or that
particular kind, then neither is the desire universal nor in all men is
it directed to the same objects. And therefore the conceiving of such
desires plainly attaches to us as individuals. It must be admitted,
however, that there is something natural in it: because different things
are pleasant to different men and a preference of some particular
objects to chance ones is universal. Well then, in the case of the
desires which are strictly and properly natural few men go wrong and all
in one direction, that is, on the side of too much: I mean, to eat and
drink of such food as happens to be on the table till one is overfilled
is exceeding in quantity the natural limit, since the natural desire
is simply a supply of a real deficiency. For this reason these men are
called belly-mad, as filling it beyond what they ought, and it is the
slavish who become of this character.

But in respect of the peculiar pleasures many men go wrong and in many
different ways; for whereas the term "fond of so and so" implies either
taking pleasure in wrong objects, or taking pleasure excessively, or as
the mass of men do, or in a wrong way, they who are destitute of all
self-control exceed in all these ways; that is to say, they take
pleasure in some things in which they ought not to do so (because they
are properly objects of detestation), and in such as it is right to take
pleasure in they do so more than they ought and as the mass of men do.

Well then, that excess with respect to pleasures is absence of
self-control, and blameworthy, is plain. But viewing these habits on the
side of pains, we find that a man is not said to have the virtue for
withstanding them (as in the case of Courage), nor the vice for not
withstanding them; but the man destitute of self-control is such,
because he is pained more than he ought to be at not obtaining things
which are pleasant (and thus his pleasure produces pain to him), and the
man of Perfected Self-Mastery is such in virtue of not being pained by
their absence, that is, by having to abstain from what is pleasant.

[Sidenote:III9a] Now the man destitute of self-control desires either
all pleasant things indiscriminately or those which are specially
pleasant, and he is impelled by his desire to choose these things in
preference to all others; and this involves pain, not only when he
misses the attainment of his objects but, in the very desiring them,
since all desire is accompanied by pain. Surely it is a strange case
this, being pained by reason of pleasure.

As for men who are defective on the side of pleasure, who take
less pleasure in things than they ought, they are almost imaginary
characters, because such absence of sensual perception is not natural to
man: for even the other animals distinguish between different kinds of
food, and like some kinds and dislike others. In fact, could a man be
found who takes no pleasure in anything and to whom all things are
alike, he would be far from being human at all: there is no name for
such a character because it is simply imaginary.

But the man of Perfected Self-Mastery is in the mean with respect to
these objects: that is to say, he neither takes pleasure in the things
which delight the vicious man, and in fact rather dislikes them, nor at
all in improper objects; nor to any great degree in any object of the
class; nor is he pained at their absence; nor does he desire them; or,
if he does, only in moderation, and neither more than he ought, nor at
improper times, and so forth; but such things as are conducive to health
and good condition of body, being also pleasant, these he will grasp at
in moderation and as he ought to do, and also such other pleasant things
as do not hinder these objects, and are not unseemly or disproportionate
to his means; because he that should grasp at such would be liking such
pleasures more than is proper; but the man of Perfected Self-Mastery
is not of this character, but regulates his desires by the dictates of
right reason.

XII

Now the vice of being destitute of all Self-Control seems to be more
truly voluntary than Cowardice, because pleasure is the cause of the
former and pain of the latter, and pleasure is an object of choice,
pain of avoidance. And again, pain deranges and spoils the natural
disposition of its victim, whereas pleasure has no such effect and is
more voluntary and therefore more justly open to reproach.

It is so also for the following reason; that it is easier to be inured
by habit to resist the objects of pleasure, there being many things of
this kind in life and the process of habituation being unaccompanied by
danger; whereas the case is the reverse as regards the objects of fear.

Again, Cowardice as a confirmed habit would seem to be voluntary in
a different way from the particular instances which form the habit;
because it is painless, but these derange the man by reason of pain so
that he throws away his arms and otherwise behaves himself unseemly,
for which reason they are even thought by some to exercise a power of
compulsion.

But to the man destitute of Self-Control the particular instances are on
the contrary quite voluntary, being done with desire and direct exertion
of the will, but the general result is less voluntary: since no man
desires to form the habit.

[Sidenote: 1119b]

The name of this vice (which signifies etymologically unchastened-ness)
we apply also to the faults of children, there being a certain
resemblance between the cases: to which the name is primarily applied,
and to which secondarily or derivatively, is not relevant to the present
subject, but it is evident that the later in point of time must get the
name from the earlier. And the metaphor seems to be a very good one;
for whatever grasps after base things, and is liable to great increase,
ought to be chastened; and to this description desire and the child
answer most truly, in that children also live under the direction of
desire and the grasping after what is pleasant is most prominently seen
in these.

Unless then the appetite be obedient and subjected to the governing
principle it will become very great: for in the fool the grasping after
what is pleasant is insatiable and undiscriminating; and every acting
out of the desire increases the kindred habit, and if the desires are
great and violent in degree they even expel Reason entirely; therefore
they ought to be moderate and few, and in no respect to be opposed
to Reason. Now when the appetite is in such a state we denominate it
obedient and chastened.

In short, as the child ought to live with constant regard to the orders
of its educator, so should the appetitive principle with regard to those
of Reason.

So then in the man of Perfected Self-Mastery, the appetitive principle
must be accordant with Reason: for what is right is the mark at which
both principles aim: that is to say, the man of perfected self-mastery
desires what he ought in right manner and at right times, which is
exactly what Reason directs. Let this be taken for our account of
Perfected Self-Mastery.




BOOK IV

I

We will next speak of Liberality. Now this is thought to be the mean
state, having for its object-matter Wealth: I mean, the Liberal man is
praised not in the circumstances of war, nor in those which constitute
the character of perfected self-mastery, nor again in judicial
decisions, but in respect of giving and receiving Wealth, chiefly the
former. By the term Wealth I mean "all those things whose worth is
measured by money."

Now the states of excess and defect in regard of Wealth are respectively
Prodigality and Stinginess: the latter of these terms we attach
invariably to those who are over careful about Wealth, but the former we
apply sometimes with a complex notion; that is to say, we give the name
to those who fail of self-control and spend money on the unrestrained
gratification of their passions; and this is why they are thought to be
most base, because they have many vices at once.

[Sidenote: 1120a]

It must be noted, however, that this is not a strict and proper use of
the term, since its natural etymological meaning is to denote him who
has one particular evil, viz. the wasting his substance: he is unsaved
(as the term literally denotes) who is wasting away by his own fault;
and this he really may be said to be; the destruction of his substance
is thought to be a kind of wasting of himself, since these things
are the means of living. Well, this is our acceptation of the term
Prodigality.

Again. Whatever things are for use may be used well or ill, and Wealth
belongs to this class. He uses each particular thing best who has the
virtue to whose province it belongs: so that he will use Wealth best
who has the virtue respecting Wealth, that is to say, the Liberal
man. Expenditure and giving are thought to be the using of money, but
receiving and keeping one would rather call the possessing of it. And so
the giving to proper persons is more characteristic of the Liberal man,
than the receiving from proper quarters and forbearing to receive
from the contrary. In fact generally, doing well by others is more
characteristic of virtue than being done well by, and doing things
positively honourable than forbearing to do things dishonourable;
and any one may see that the doing well by others and doing things
positively honourable attaches to the act of giving, but to that of
receiving only the being done well by or forbearing to do what is
dishonourable.

Besides, thanks are given to him who gives, not to him who merely
forbears to receive, and praise even more. Again, forbearing to receive
is easier than giving, the case of being too little freehanded with
one's own being commoner than taking that which is not one's own.

And again, it is they who give that are denominated Liberal, while they
who forbear to receive are commended, not on the score of Liberality but
of just dealing, while for receiving men are not, in fact, praised at
all.

And the Liberal are liked almost best of all virtuous characters,
because they are profitable to others, and this their profitableness
consists in their giving.

Furthermore: all the actions done in accordance with virtue are
honourable, and done from the motive of honour: and the Liberal man,
therefore, will give from a motive of honour, and will give rightly;
I mean, to proper persons, in right proportion, at right times, and
whatever is included in the term "right giving:" and this too with
positive pleasure, or at least without pain, since whatever is done in
accordance with virtue is pleasant or at least not unpleasant, most
certainly not attended with positive pain.

But the man who gives to improper people, or not from a motive of honour
but from some other cause, shall be called not Liberal but something
else. Neither shall he be so [Sidenote:1120b] denominated who does it
with pain: this being a sign that he would prefer his wealth to the
honourable action, and this is no part of the Liberal man's character;
neither will such an one receive from improper sources, because the so
receiving is not characteristic of one who values not wealth: nor again
will he be apt to ask, because one who does kindnesses to others does
not usually receive them willingly; but from proper sources (his own
property, for instance) he will receive, doing this not as honourable
but as necessary, that he may have somewhat to give: neither will he be
careless of his own, since it is his wish through these to help others
in need: nor will he give to chance people, that he may have wherewith
to give to those to whom he ought, at right times, and on occasions when
it is honourable so to do.

Again, it is a trait in the Liberal man's character even to exceed
very much in giving so as to leave too little for himself, it being
characteristic of such an one not to have a thought of self.

Now Liberality is a term of relation to a man's means, for the
Liberal-ness depends not on the amount of what is given but on the moral
state of the giver which gives in proportion to his means. There is then
no reason why he should not be the more Liberal man who gives the less
amount, if he has less to give out of.

Again, they are thought to be more Liberal who have inherited, not
acquired for themselves, their means; because, in the first place, they
have never experienced want, and next, all people love most their own
works, just as parents do and poets.

It is not easy for the Liberal man to be rich, since he is neither apt
to receive nor to keep but to lavish, and values not wealth for its own
sake but with a view to giving it away. Hence it is commonly charged
upon fortune that they who most deserve to be rich are least so. Yet
this happens reasonably enough; it is impossible he should have wealth
who does not take any care to have it, just as in any similar case.

Yet he will not give to improper people, nor at wrong times, and so on:
because he would not then be acting in accordance with Liberality, and
if he spent upon such objects, would have nothing to spend on those on
which he ought: for, as I have said before, he is Liberal who spends in
proportion to his means, and on proper objects, while he who does so
in excess is prodigal (this is the reason why we never call despots
prodigal, because it does not seem to be easy for them by their gifts
and expenditure to go beyond their immense possessions).

To sum up then. Since Liberality is a mean state in respect of the
giving and receiving of wealth, the Liberal man will give and spend on
proper objects, and in proper proportion, in great things and in small
alike, and all this with pleasure to himself; also he will receive from
right sources, and in right proportion: because, as the virtue is a mean
state in respect of both, he will do both as he ought, and, in fact,
upon proper giving follows the correspondent receiving, while that which
is not such is contrary to it. (Now those which follow one another come
to co-exist in the same person, those which are contraries plainly do
not.)

[Sidenote:1121a] Again, should it happen to him to spend money beyond
what is needful, or otherwise than is well, he will be vexed, but only
moderately and as he ought; for feeling pleasure and pain at right
objects, and in right manner, is a property of Virtue.

The Liberal man is also a good man to have for a partner in respect of
wealth: for he can easily be wronged, since he values not wealth, and
is more vexed at not spending where he ought to have done so than at
spending where he ought not, and he relishes not the maxim of Simonides.

But the Prodigal man goes wrong also in these points, for he is neither
pleased nor pained at proper objects or in proper manner, which will
become more plain as we proceed. We have said already that Prodigality
and Stinginess are respectively states of excess and defect, and this in
two things, giving and receiving (expenditure of course we class under
giving). Well now, Prodigality exceeds in giving and forbearing to
receive and is deficient in receiving, while Stinginess is deficient in
giving and exceeds in receiving, but it is in small things.

The two parts of Prodigality, to be sure, do not commonly go together;
it is not easy, I mean, to give to all if you receive from none, because
private individuals thus giving will soon find their means run short,
and such are in fact thought to be prodigal. He that should combine both
would seem to be no little superior to the Stingy man: for he may be
easily cured, both by advancing in years, and also by the want of means,
and he may come thus to the mean: he has, you see, already the _facts_
of the Liberal man, he gives and forbears to receive, only he does
neither in right manner or well. So if he could be wrought upon by
habituation in this respect, or change in any other way, he would be a
real Liberal man, for he will give to those to whom he should, and will
forbear to receive whence he ought not. This is the reason too why he is
thought not to be low in moral character, because to exceed in giving
and in forbearing to receive is no sign of badness or meanness, but only
of folly.

[Sidenote:1121b] Well then, he who is Prodigal in this fashion is
thought far superior to the Stingy man for the aforementioned reasons,
and also because he does good to many, but the Stingy man to no one,
not even to himself. But most Prodigals, as has been said, combine with
their other faults that of receiving from improper sources, and on this
point are Stingy: and they become grasping, because they wish to spend
and cannot do this easily, since their means soon run short and they are
necessitated to get from some other quarter; and then again, because
they care not for what is honourable, they receive recklessly, and from
all sources indifferently, because they desire to give but care not how
or whence. And for this reason their givings are not Liberal, inasmuch
as they are not honourable, nor purely disinterested, nor done in right
fashion; but they oftentimes make those rich who should be poor, and to
those who are quiet respectable kind of people they will give nothing,
but to flatterers, or those who subserve their pleasures in any way,
they will give much. And therefore most of them are utterly devoid
of self-restraint; for as they are open-handed they are liberal in
expenditure upon the unrestrained gratification of their passions, and
turn off to their pleasures because they do not live with reference to
what is honourable.

Thus then the Prodigal, if unguided, slides into these faults; but if he
could get care bestowed on him he might come to the mean and to what is
right.

Stinginess, on the contrary, is incurable: old age, for instance, and
incapacity of any kind, is thought to make people Stingy; and it is more
congenial to human nature than Prodigality, the mass of men being fond
of money rather than apt to give: moreover it extends far and has many
phases, the modes of stinginess being thought to be many. For as it
consists of two things, defect of giving and excess of receiving,
everybody does not have it entire, but it is sometimes divided, and one
class of persons exceed in receiving, the other are deficient in giving.
I mean those who are designated by such appellations as sparing,
close-fisted, niggards, are all deficient in giving; but other men's
property they neither desire nor are willing to receive, in some
instances from a real moderation and shrinking from what is base.

There are some people whose motive, either supposed or alleged, for
keeping their property is this, that they may never be driven to do
anything dishonourable: to this class belongs the skinflint, and every
one of similar character, so named from the excess of not-giving. Others
again decline to receive their neighbour's goods from a motive of fear;
their notion being that it is not easy to take other people's things
yourself without their taking yours: so they are content neither to
receive nor give.

[Sidenote:1122a] The other class again who are Stingy in respect of
receiving exceed in that they receive anything from any source; such as
they who work at illiberal employments, brothel keepers, and such-like,
and usurers who lend small sums at large interest: for all these receive
from improper sources, and improper amounts. Their common characteristic
is base-gaining, since they all submit to disgrace for the sake of gain
and that small; because those who receive great things neither whence
they ought, nor what they ought (as for instance despots who sack cities
and plunder temples), we denominate wicked, impious, and unjust, but not
Stingy.

Now the dicer and bath-plunderer and the robber belong to the class of
the Stingy, for they are given to base gain: both busy themselves and
submit to disgrace for the sake of gain, and the one class incur the
greatest dangers for the sake of their booty, while the others make gain
of their friends to whom they ought to be giving.

So both classes, as wishing to make gain from improper sources, are
given to base gain, and all such receivings are Stingy. And with good
reason is Stinginess called the contrary of Liberality: both because it
is a greater evil than Prodigality, and because men err rather in this
direction than in that of the Prodigality which we have spoken of as
properly and completely such.

Let this be considered as what we have to say respecting Liberality and
the contrary vices.

II

Next in order would seem to come a dissertation on Magnificence,
this being thought to be, like liberality, a virtue having for its
object-matter Wealth; but it does not, like that, extend to all
transactions in respect of Wealth, but only applies to such as are
expensive, and in these circumstances it exceeds liberality in respect
of magnitude, because it is (what the very name in Greek hints at)
fitting expense on a large scale: this term is of course relative: I
mean, the expenditure of equipping and commanding a trireme is not the
same as that of giving a public spectacle: "fitting" of course also is
relative to the individual, and the matter wherein and upon which he has
to spend. And a man is not denominated Magnificent for spending as he
should do in small or ordinary things, as, for instance,

  "Oft to the wandering beggar did I give,"

but for doing so in great matters: that is to say, the Magnificent man
is liberal, but the liberal is not thereby Magnificent. The falling
short of such a state is called Meanness, the exceeding it Vulgar
Profusion, Want of Taste, and so on; which are faulty, not because they
are on an excessive scale in respect of right objects but, because they
show off in improper objects, and in improper manner: of these we will
speak presently. The Magnificent man is like a man of skill, because he
can see what is fitting, and can spend largely in good taste; for, as
we said at the commencement, [Sidenote: 1122b] the confirmed habit is
determined by the separate acts of working, and by its object-matter.

Well, the expenses of the Magnificent man are great and fitting: such
also are his works (because this secures the expenditure being not great
merely, but befitting the work). So then the work is to be proportionate
to the expense, and this again to the work, or even above it: and the
Magnificent man will incur such expenses from the motive of honour, this
being common to all the virtues, and besides he will do it with pleasure
and lavishly; excessive accuracy in calculation being Mean. He will
consider also how a thing may be done most beautifully and fittingly,
rather, than for how much it may be done, and how at the least expense.

So the Magnificent man must be also a liberal man, because the liberal
man will also spend what he ought, and in right manner: but it is the
Great, that is to say tke large scale, which is distinctive of the
Magnificent man, the object-matter of liberality being the same, and
without spending more money than another man he will make the work more
magnificent. I mean, the excellence of a possession and of a work is not
the same: as a piece of property that thing is most valuable which is
worth most, gold for instance; but as a work that which is great and
beautiful, because the contemplation of such an object is admirable,
and so is that which is Magnificent. So the excellence of a work is
Magnificence on a large scale. There are cases of expenditure which we
call honourable, such as are dedicatory offerings to the gods, and the
furnishing their temples, and sacrifices, and in like manner everything
that has reference to the Deity, and all such public matters as are
objects of honourable ambition, as when men think in any case that it is
their duty to furnish a chorus for the stage splendidly, or fit out and
maintain a trireme, or give a general public feast.

Now in all these, as has been already stated, respect is had also to the
rank and the means of the man who is doing them: because they should be
proportionate to these, and befit not the work only but also the doer of
the work. For this reason a poor man cannot be a Magnificent man, since
he has not means wherewith to spend largely and yet becomingly; and if
he attempts it he is a fool, inasmuch as it is out of proportion and
contrary to propriety, whereas to be in accordance with virtue a thing
must be done rightly.

Such expenditure is fitting moreover for those to whom such things
previously belong, either through themselves or through their ancestors
or people with whom they are connected, and to the high-born or people
of high repute, and so on: because all these things imply greatness and
reputation.

So then the Magnificent man is pretty much as I have described him,
and Magnificence consists in such expenditures: because they are the
greatest and most honourable: [Sidenote:1123a] and of private ones such
as come but once for all, marriage to wit, and things of that kind; and
any occasion which engages the interest of the community in general, or
of those who are in power; and what concerns receiving and despatching
strangers; and gifts, and repaying gifts: because the Magnificent man
is not apt to spend upon himself but on the public good, and gifts are
pretty much in the same case as dedicatory offerings.

It is characteristic also of the Magnificent man to furnish his house
suitably to his wealth, for this also in a way reflects credit; and
again, to spend rather upon such works as are of long duration, these
being most honourable. And again, propriety in each case, because the
same things are not suitable to gods and men, nor in a temple and a
tomb. And again, in the case of expenditures, each must be great of its
kind, and great expense on a great object is most magnificent, that is
in any case what is great in these particular things.

There is a difference too between greatness of a work and greatness of
expenditure: for instance, a very beautiful ball or cup is magnificent
as a present to a child, while the price of it is small and almost
mean. Therefore it is characteristic of the Magnificent man to do
magnificently whatever he is about: for whatever is of this kind cannot
be easily surpassed, and bears a proper proportion to the expenditure.

Such then is the Magnificent man.

The man who is in the state of excess, called one of Vulgar Profusion,
is in excess because he spends improperly, as has been said. I mean in
cases requiring small expenditure he lavishes much and shows off out of
taste; giving his club a feast fit for a wedding-party, or if he has to
furnish a chorus for a comedy, giving the actors purple to wear in the
first scene, as did the Megarians. And all such things he will do, not
with a view to that which is really honourable, but to display his
wealth, and because he thinks he shall be admired for these things; and
he will spend little where he ought to spend much, and much where he
should spend little.

The Mean man will be deficient in every case, and even where he has
spent the most he will spoil the whole effect for want of some trifle;
he is procrastinating in all he does, and contrives how he may spend
the least, and does even that with lamentations about the expense, and
thinking that he does all things on a greater scale than he ought.

Of course, both these states are faulty, but they do not involve
disgrace because they are neither hurtful to others nor very unseemly.

III

The very name of Great-mindedness implies, that great things are its
object-matter; and we will first settle what kind of things. It makes no
difference, of course, whether we regard the moral state in the abstract
or as exemplified in an individual.

[Sidenote: 1123b] Well then, he is thought to be Great-minded who values
himself highly and at the same time justly, because he that does so
without grounds is foolish, and no virtuous character is foolish or
senseless. Well, the character I have described is Great-minded. The man
who estimates himself lowly, and at the same time justly, is modest; but
not Great-minded, since this latter quality implies greatness, just as
beauty implies a large bodily conformation while small people are neat
and well made but not beautiful.

Again, he who values himself highly without just grounds is a Vain
man: though the name must not be applied to every case of unduly
high self-estimation. He that values himself below his real worth is
Small-minded, and whether that worth is great, moderate, or small, his
own estimate falls below it. And he is the strongest case of this error
who is really a man of great worth, for what would he have done had his
worth been less?

The Great-minded man is then, as far as greatness is concerned, at
the summit, but in respect of propriety he is in the mean, because he
estimates himself at his real value (the other characters respectively
are in excess and defect). Since then he justly estimates himself at a
high, or rather at the highest possible rate, his character will have
respect specially to one thing: this term "rate" has reference of course
to external goods: and of these we should assume that to be the greatest
which we attribute to the gods, and which is the special object of
desire to those who are in power, and which is the prize proposed to the
most honourable actions: now honour answers to these descriptions, being
the greatest of external goods. So the Great-minded man bears himself as
he ought in respect of honour and dishonour. In fact, without need of
words, the Great-minded plainly have honour for their object-matter:
since honour is what the great consider themselves specially worthy of,
and according to a certain rate.

The Small-minded man is deficient, both as regards himself, and also
as regards the estimation of the Great-minded: while the Vain man is in
excess as regards himself, but does not get beyond the Great-minded
man. Now the Great-minded man, being by the hypothesis worthy of the
greatest things, must be of the highest excellence, since the better a
man is the more is he worth, and he who is best is worth the most: it
follows then, that to be truly Great-minded a man must be good,
and whatever is great in each virtue would seem to belong to the
Great-minded. It would no way correspond with the character of the
Great-minded to flee spreading his hands all abroad; nor to injure any
one; for with what object in view will he do what is base, in whose eyes
nothing is great? in short, if one were to go into particulars, the
Great-minded man would show quite ludicrously unless he were a good man:
he would not be in fact deserving of honour if he were a bad man, honour
being the prize of virtue and given to the good.

This virtue, then, of Great-mindedness seems to be a kind of ornament
of all the other virtues, in that it makes them better and cannot be
without them; and for this reason it is a hard matter to be really and
truly Great-minded; for it cannot be without thorough goodness and
nobleness of character.

[Sidenote:1124a] Honour then and dishonour are specially the
object-matter of the Great-minded man: and at such as is great, and
given by good men, he will be pleased moderately as getting his own, or
perhaps somewhat less for no honour can be quite adequate to perfect
virtue: but still he will accept this because they have nothing higher
to give him. But such as is given by ordinary people and on trifling
grounds he will entirely despise, because these do not come up to his
deserts: and dishonour likewise, because in his case there cannot be
just ground for it.

Now though, as I have said, honour is specially the object-matter of the
Great-minded man, I do not mean but that likewise in respect of wealth
and power, and good or bad fortune of every kind, he will bear himself
with moderation, fall out how they may, and neither in prosperity will
he be overjoyed nor in adversity will he be unduly pained. For not even
in respect of honour does he so bear himself; and yet it is the greatest
of all such objects, since it is the cause of power and wealth being
choiceworthy, for certainly they who have them desire to receive honour
through them. So to whom honour even is a small thing to him will all
other things also be so; and this is why such men are thought to be
supercilious.

It seems too that pieces of good fortune contribute to form this
character of Great-mindedness: I mean, the nobly born, or men of
influence, or the wealthy, are considered to be entitled to honour, for
they are in a position of eminence and whatever is eminent by good is
more entitled to honour: and this is why such circumstances dispose men
rather to Great-mindedness, because they receive honour at the hands of
some men.

Now really and truly the good man alone is entitled to honour; only if
a man unites in himself goodness with these external advantages he is
thought to be more entitled to honour: but they who have them without
also having virtue are not justified in their high estimate of
themselves, nor are they rightly denominated Great-minded; since perfect
virtue is one of the indispensable conditions to such & character.

[Sidenote:1124b] Further, such men become supercilious and insolent, it
not being easy to bear prosperity well without goodness; and not being
able to bear it, and possessed with an idea of their own superiority to
others, they despise them, and do just whatever their fancy prompts; for
they mimic the Great-minded man, though they are not like him, and they
do this in such points as they can, so without doing the actions which
can only flow from real goodness they despise others. Whereas the
Great-minded man despises on good grounds (for he forms his opinions
truly), but the mass of men do it at random.

Moreover, he is not a man to incur little risks, nor does he court
danger, because there are but few things he has a value for; but he will
incur great dangers, and when he does venture he is prodigal of his life
as knowing that there are terms on which it is not worth his while to
live. He is the sort of man to do kindnesses, but he is ashamed to
receive them; the former putting a man in the position of superiority,
the latter in that of inferiority; accordingly he will greatly overpay
any kindness done to him, because the original actor will thus be laid
under obligation and be in the position of the party benefited. Such men
seem likewise to remember those they have done kindnesses to, but not
those from whom they have received them: because he who has received is
inferior to him who has done the kindness and our friend wishes to be
superior; accordingly he is pleased to hear of his own kind acts but not
of those done to himself (and this is why, in Homer, Thetis does
not mention to Jupiter the kindnesses she had done him, nor did the
Lacedæmonians to the Athenians but only the benefits they had received).

Further, it is characteristic of the Great-minded man to ask favours not
at all, or very reluctantly, but to do a service very readily; and to
bear himself loftily towards the great or fortunate, but towards people
of middle station affably; because to be above the former is difficult
and so a grand thing, but to be above the latter is easy; and to be high
and mighty towards the former is not ignoble, but to do it towards those
of humble station would be low and vulgar; it would be like parading
strength against the weak.

And again, not to put himself in the way of honour, nor to go where
others are the chief men; and to be remiss and dilatory, except in the
case of some great honour or work; and to be concerned in few things,
and those great and famous. It is a property of him also to be open,
both in his dislikes and his likings, because concealment is a
consequent of fear. Likewise to be careful for reality rather than
appearance, and talk and act openly (for his contempt for others makes
him a bold man, for which same reason he is apt to speak the truth,
except where the principle of reserve comes in), but to be reserved
towards the generality of men.

[Sidenote: II25a] And to be unable to live with reference to any other
but a friend; because doing so is servile, as may be seen in that all
flatterers are low and men in low estate are flatterers. Neither is his
admiration easily excited, because nothing is great in his eyes; nor
does he bear malice, since remembering anything, and specially wrongs,
is no part of Great-mindedness, but rather overlooking them; nor does he
talk of other men; in fact, he will not speak either of himself or of
any other; he neither cares to be praised himself nor to have others
blamed; nor again does he praise freely, and for this reason he is
not apt to speak ill even of his enemies except to show contempt and
insolence.

And he is by no means apt to make laments about things which cannot be
helped, or requests about those which are trivial; because to be thus
disposed with respect to these things is consequent only upon real
anxiety about them. Again, he is the kind of man to acquire what
is beautiful and unproductive rather than what is productive and
profitable: this being rather the part of an independent man. Also slow
motion, deep-toned voice, and deliberate style of speech, are thought to
be characteristic of the Great-minded man: for he who is earnest about
few things is not likely to be in a hurry, nor he who esteems nothing
great to be very intent: and sharp tones and quickness are the result of
these.

This then is my idea of the Great-minded man; and he who is in the
defect is a Small-minded man, he who is in the excess a Vain man.
However, as we observed in respect of the last character we discussed,
these extremes are not thought to be vicious exactly, but only mistaken,
for they do no harm.

The Small-minded man, for instance, being really worthy of good deprives
himself of his deserts, and seems to have somewhat faulty from not
having a sufficiently high estimate of his own desert, in fact from
self-ignorance: because, but for this, he would have grasped after what
he really is entitled to, and that is good. Still such characters are
not thought to be foolish, but rather laggards. But the having such
an opinion of themselves seems to have a deteriorating effect on the
character: because in all cases men's aims are regulated by their
supposed desert, and thus these men, under a notion of their own want of
desert, stand aloof from honourable actions and courses, and similarly
from external goods.

But the Vain are foolish and self-ignorant, and that palpably: because
they attempt honourable things, as though they were worthy, and then
they are detected. They also set themselves off, by dress, and carriage,
and such-like things, and desire that their good circumstances may
be seen, and they talk of them under the notion of receiving
honour thereby. Small-mindedness rather than Vanity is opposed to
Great-mindedness, because it is more commonly met with and is worse.

[Sidenote:1125b] Well, the virtue of Great-mindedness has for its object
great Honour, as we have said: and there seems to be a virtue having
Honour also for its object (as we stated in the former book), which may
seem to bear to Great-mindedness the same relation that Liberality does
to Magnificence: that is, both these virtues stand aloof from what is
great but dispose us as we ought to be disposed towards moderate and
small matters. Further: as in giving and receiving of wealth there is
a mean state, an excess, and a defect, so likewise in grasping after
Honour there is the more or less than is right, and also the doing so
from right sources and in right manner.

For we blame the lover of Honour as aiming at Honour more than he ought,
and from wrong sources; and him who is destitute of a love of Honour as
not choosing to be honoured even for what is noble. Sometimes again we
praise the lover of Honour as manly and having a love for what is noble,
and him who has no love for it as being moderate and modest (as we
noticed also in the former discussion of these virtues).

It is clear then that since "Lover of so and so" is a term capable of
several meanings, we do not always denote the same quality by the term
"Lover of Honour;" but when we use it as a term of commendation we
denote more than the mass of men are; when for blame more than a man
should be.

And the mean state having no proper name the extremes seem to dispute
for it as unoccupied ground: but of course where there is excess and
defect there must be also the mean. And in point of fact, men do grasp
at Honour more than they should, and less, and sometimes just as they
ought; for instance, this state is praised, being a mean state in regard
of Honour, but without any appropriate name. Compared with what is
called Ambition it shows like a want of love for Honour, and compared
with this it shows like Ambition, or compared with both, like both
faults: nor is this a singular case among the virtues. Here the
extreme characters appear to be opposed, because the mean has no name
appropriated to it.


V

Meekness is a mean state, having for its object-matter Anger: and as the
character in the mean has no name, and we may almost say the same of the
extremes, we give the name of Meekness (leaning rather to the defect,
which has no name either) to the character in the mean.

The excess may be called an over-aptness to Anger: for the passion is
Anger, and the producing causes many and various. Now he who is angry at
what and with whom he ought, and further, in right manner and time, and
for proper length of time, is praised, so this Man will be Meek since
Meekness is praised. For the notion represented by the term Meek man is
the being imperturbable, and not being led away by passion, but being
angry in that manner, and at those things, and for that length of time,
which Reason may direct. This character however is thought to err rather
on [Sidenote:1126a] the side of defect, inasmuch as he is not apt to
take revenge but rather to make allowances and forgive. And the defect,
call it Angerlessness or what you will, is blamed: I mean, they who are
not angry at things at which they ought to be angry are thought to be
foolish, and they who are angry not in right manner, nor in right time,
nor with those with whom they ought; for a man who labours under this
defect is thought to have no perception, nor to be pained, and to have
no tendency to avenge himself, inasmuch as he feels no anger: now to
bear with scurrility in one's own person, and patiently see one's own
friends suffer it, is a slavish thing.

As for the excess, it occurs in all forms; men are angry with those with
whom, and at things with which, they ought not to be, and more than they
ought, and too hastily, and for too great a length of time. I do not
mean, however, that these are combined in any one person: that would
in fact be impossible, because the evil destroys itself, and if it is
developed in its full force it becomes unbearable.

Now those whom we term the Passionate are soon angry, and with people
with whom and at things at which they ought not, and in an excessive
degree, but they soon cool again, which is the best point about them.
And this results from their not repressing their anger, but repaying
their enemies (in that they show their feeings by reason of their
vehemence), and then they have done with it.

The Choleric again are excessively vehement, and are angry at
everything, and on every occasion; whence comes their Greek name
signifying that their choler lies high.

The Bitter-tempered are hard to reconcile and keep their anger for
a long while, because they repress the feeling: but when they have
revenged themselves then comes a lull; for the vengeance destroys their
anger by producing pleasure in lieu of pain. But if this does not happen
they keep the weight on their minds: because, as it does not show
itself, no one attempts to reason it away, and digesting anger within
one's self takes time. Such men are very great nuisances to themselves
and to their best friends.

Again, we call those Cross-grained who are angry at wrong objects, and
in excessive degree, and for too long a time, and who are not appeased
without vengeance or at least punishing the offender.

To Meekness we oppose the excess rather than the defect, because it is
of more common occurrence: for human nature is more disposed to take
than to forgo revenge. And the Cross-grained are worse to live with
[than they who are too phlegmatic].

Now, from what has been here said, that is also plain which was said
before. I mean, it is no easy matter to define how, and with what
persons, and at what kind of things, and how long one ought to be
angry, and up to what point a person is right or is wrong. For he that
transgresses the strict rule only a little, whether on the side of
too much or too little, is not blamed: sometimes we praise those who
[Sidenote:1126b] are deficient in the feeling and call them Meek,
sometimes we call the irritable Spirited as being well qualified for
government. So it is not easy to lay down, in so many words, for what
degree or kind of transgression a man is blameable: because the decision
is in particulars, and rests therefore with the Moral Sense. Thus much,
however, is plain, that the mean state is praiseworthy, in virtue of
which we are angry with those with whom, and at those things with which,
we ought to be angry, and in right manner, and so on; while the excesses
and defects are blameable, slightly so if only slight, more so if
greater, and when considerable very blameable.

It is clear, therefore, that the mean state is what we are to hold to.

This then is to be taken as our account of the various moral states
which have Anger for their object-matter.

VI

Next, as regards social intercourse and interchange of words and acts,
some men are thought to be Over-Complaisant who, with a view solely to
giving pleasure, agree to everything and never oppose, but think their
line is to give no pain to those they are thrown amongst: they, on
the other hand, are called Cross and Contentious who take exactly the
contrary line to these, and oppose in everything, and have no care at
all whether they give pain or not.

Now it is quite clear of course, that the states I have named are
blameable, and that the mean between them is praiseworthy, in virtue
of which a man will let pass what he ought as he ought, and also will
object in like manner. However, this state has no name appropriated, but
it is most like Friendship; since the man who exhibits it is just the
kind of man whom we would call the amiable friend, with the addition of
strong earnest affection; but then this is the very point in which it
differs from Friendship, that it is quite independent of any feeling or
strong affection for those among whom the man mixes: I mean, that he
takes everything as he ought, not from any feeling of love or hatred,
but simply because his natural disposition leads him to do so; he will
do it alike to those whom he does know and those whom he does not, and
those with whom he is intimate and those with whom he is not; only in
each case as propriety requires, because it is not fitting to care
alike for intimates and strangers, nor again to pain them alike.

It has been stated in a general way that his social intercourse will be
regulated by propriety, and his aim will be to avoid giving pain and to
contribute to pleasure, but with a constant reference to what is noble
and expedient.

His proper object-matter seems to be the pleasures and pains which arise
out of social intercourse, but whenever it is not honourable or even
hurtful to him to contribute to pleasure, in these instances he will run
counter and prefer to give pain.

Or if the things in question involve unseemliness to the doer, and this
not inconsiderable, or any harm, whereas his opposition will cause some
little pain, here he will not agree but will run counter.

[Sidenote:1127a] Again, he will regulate differently his intercourse
with great men and with ordinary men, and with all people according to
the knowledge he has of them; and in like manner, taking in any other
differences which may exist, giving to each his due, and in itself
preferring to give pleasure and cautious not to give pain, but still
guided by the results, I mean by what is noble and expedient according
as they preponderate.

Again, he will inflict trifling pain with a view to consequent pleasure.

Well, the man bearing the mean character is pretty well such as I have
described him, but he has no name appropriated to him: of those who try
to give pleasure, the man who simply and disinterestedly tries to be
agreeable is called Over-Complaisant, he who does it with a view to
secure some profit in the way of wealth, or those things which wealth
may procure, is a Flatterer: I have said before, that the man who is
"always non-content" is Cross and Contentious. Here the extremes have
the appearance of being opposed to one another, because the mean has no
appropriate name.



VII

The mean state which steers clear of Exaggeration has pretty much the
same object-matter as the last we described, and likewise has no name
appropriated to it. Still it may be as well to go over these states:
because, in the first place, by a particular discussion of each we shall
be better acquainted with the general subject of moral character, and
next we shall be the more convinced that the virtues are mean states by
seeing that this is universally the case.

In respect then of living in society, those who carry on this
intercourse with a view to pleasure and pain have been already spoken
of; we will now go on to speak of those who are True or False, alike in
their words and deeds and in the claims which they advance.

Now the Exaggerator is thought to have a tendency to lay claim to things
reflecting credit on him, both when they do not belong to him at all and
also in greater degree than that in which they really do: whereas the
Reserved man, on the contrary, denies those which really belong to
him or else depreciates them, while the mean character being a
Plain-matter-of-fact person is Truthful in life and word, admitting
the existence of what does really belong to him and making it neither
greater nor less than the truth.

It is possible of course to take any of these lines either with or
without some further view: but in general men speak, and act, and live,
each according to his particular character and disposition, unless
indeed a man is acting from any special motive.

Now since falsehood is in itself low and blameable, while truth is noble
and praiseworthy, it follows that the Truthful man (who is also in the
mean) is praiseworthy, and the two who depart from strict truth are both
blameable, but especially the Exaggerator.

We will now speak of each, and first of the Truthful man: I call him
Truthful, because we are not now meaning the man who is true in his
agreements nor in such matters as amount to justice or injustice (this
would come within the [Sidenote:1127b] province of a different virtue),
but, in such as do not involve any such serious difference as this, the
man we are describing is true in life and word simply because he is in a
certain moral state.

And he that is such must be judged to be a good man: for he that has a
love for Truth as such, and is guided by it in matters indifferent, will
be so likewise even more in such as are not indifferent; for surely he
will have a dread of falsehood as base, since he shunned it even in
itself: and he that is of such a character is praiseworthy, yet he leans
rather to that which is below the truth, this having an appearance of
being in better taste because exaggerations are so annoying.

As for the man who lays claim to things above what really belongs to him
_without_ any special motive, he is like a base man because he would
not otherwise have taken pleasure in falsehood, but he shows as a fool
rather than as a knave. But if a man does this _with_ a special motive,
suppose for honour or glory, as the Braggart does, then he is not
so very blameworthy, but if, directly or indirectly, for pecuniary
considerations, he is more unseemly.

Now the Braggart is such not by his power but by his purpose, that is to
say, in virtue of his moral state, and because he is a man of a certain
kind; just as there are liars who take pleasure in falsehood for its
own sake while others lie from a desire of glory or gain. They who
exaggerate with a view to glory pretend to such qualities as are
followed by praise or highest congratulation; they who do it with a view
to gain assume those which their neighbours can avail themselves of,
and the absence of which can be concealed, as a man's being a skilful
soothsayer or physician; and accordingly most men pretend to such things
and exaggerate in this direction, because the faults I have mentioned
are in them.

The Reserved, who depreciate their own qualities, have the appearance of
being more refined in their characters, because they are not thought to
speak with a view to gain but to avoid grandeur: one very common trait
in such characters is their denying common current opinions, as Socrates
used to do. There are people who lay claim falsely to small things and
things the falsity of their pretensions to which is obvious; these are
called Factotums and are very despicable.

This very Reserve sometimes shows like Exaggeration; take, for instance,
the excessive plainness of dress affected by the Lacedaemonians: in
fact, both excess and the extreme of deficiency partake of the nature of
Exaggeration. But they who practise Reserve in moderation, and in cases
in which the truth is not very obvious and plain, give an impression of
refinement. Here it is the Exaggerator (as being the worst character)
who appears to be opposed to the Truthful Man.

VIII

[Sidenote:II28a] Next, as life has its pauses and in them admits of
pastime combined with Jocularity, it is thought that in this respect
also there is a kind of fitting intercourse, and that rules may be
prescribed as to the kind of things one should say and the manner of
saying them; and in respect of hearing likewise (and there will be a
difference between the saying and hearing such and such things). It is
plain that in regard to these things also there will be an excess and
defect and a mean.

Now they who exceed in the ridiculous are judged to be Buffoons and
Vulgar, catching at it in any and every way and at any cost, and aiming
rather at raising laughter than at saying what is seemly and at avoiding
to pain the object of their wit. They, on the other hand, who would not
for the world make a joke themselves and are displeased with such as do
are thought to be Clownish and Stern. But they who are Jocular in good
taste are denominated by a Greek term expressing properly ease of
movement, because such are thought to be, as one may say, motions of the
moral character; and as bodies are judged of by their motions so too are
moral characters.

Now as the ridiculous lies on the surface, and the majority of men take
more pleasure than they ought in Jocularity and Jesting, the Buffoons
too get this name of Easy Pleasantry, as if refined and gentlemanlike;
but that they differ from these, and considerably too, is plain from
what has been said.

One quality which belongs to the mean state is Tact: it is
characteristic of a man of Tact to say and listen to such things as are
fit for a good man and a gentleman to say and listen to: for there are
things which are becoming for such a one to say and listen to in the way
of Jocularity, and there is a difference between the Jocularity of the
Gentleman and that of the Vulgarian; and again, between that of the
educated and uneducated man. This you may see from a comparison of the
Old and New Comedy: in the former obscene talk made the fun; in the
latter it is rather innuendo: and this is no slight difference _as
regards decency_.

Well then, are we to characterise him who jests well by his saying what
is becoming a gentleman, or by his avoiding to pain the object of his
wit, or even by his giving him pleasure? or will not such a definition
be vague, since different things are hateful and pleasant to different
men?

Be this as it may, whatever he says such things will he also listen to,
since it is commonly held that a man will do what he will bear to hear:
this must, however, be limited; a man will not do quite all that he will
hear: because jesting is a species of scurrility and there are some
points of scurrility forbidden by law; it may be certain points of
jesting should have been also so forbidden. So then the refined and
gentlemanlike man will bear himself thus as being a law to himself. Such
is the mean character, whether denominated the man of Tact or of Easy
Pleasantry.

But the Buffoon cannot resist the ridiculous, sparing neither himself
nor any one else so that he can but raise his laugh, saying things of
such kind as no man of refinement would say and some which he would not
even tolerate if said by others in his hearing. [Sidenote:1128b] The
Clownish man is for such intercourse wholly useless: inasmuch as
contributing nothing jocose of his own he is savage with all who do.

Yet some pause and amusement in life are generally judged to be
indispensable.

The three mean states which have been described do occur in life, and
the object-matter of all is interchange of words and deeds. They differ,
in that one of them is concerned with truth, and the other two with the
pleasurable: and of these two again, the one is conversant with
the jocosities of life, the other with all other points of social
intercourse.

IX

To speak of Shame as a Virtue is incorrect, because it is much more like
a feeling than a moral state. It is defined, we know, to be "a kind of
fear of disgrace," and its effects are similar to those of the fear of
danger, for they who feel Shame grow red and they who fear death turn
pale. So both are evidently in a way physical, which is thought to be a
mark of a feeling rather than a moral state.

Moreover, it is a feeling not suitable to every age, but only to youth:
we do think that the young should be Shamefaced, because since they live
at the beck and call of passion they do much that is wrong and Shame
acts on them as a check. In fact, we praise such young men as are
Shamefaced, but no one would ever praise an old man for being given
to it, inasmuch as we hold that he ought not to do things which cause
Shame; for Shame, since it arises at low bad actions, does not at all
belong to the good man, because such ought not to be done at all: nor
does it make any difference to allege that some things are disgraceful
really, others only because they are thought so; for neither should be
done, so that a man ought not to be in the position of feeling Shame. In
truth, to be such a man as to do anything disgraceful is the part of a
faulty character. And for a man to be such that he would feel Shame if
he should do anything disgraceful, and to think that this constitutes
him a good man, is absurd: because Shame is felt at voluntary actions
only, and a good man will never voluntarily do what is base.

True it is, that Shame may be good on a certain supposition, as "if a
man should do such things, he would feel Shame:" but then the Virtues
are good in themselves, and not merely in supposed cases. And, granted
that impudence and the not being ashamed to do what is disgraceful is
base, it does not the more follow that it is good for a man to do such
things and feel Shame.

Nor is Self-Control properly a Virtue, but a kind of mixed state:
however, all about this shall be set forth in a future Book.




BOOK V

[Sidenote:1129a] Now the points for our inquiry in respect of Justice
and Injustice are, what kind of actions are their object-matter, and
what kind of a mean state Justice is, and between what points the
abstract principle of it, i.e. the Just, is a mean. And our inquiry
shall be, if you please, conducted in the same method as we have
observed in the foregoing parts of this treatise.

We see then that all men mean by the term Justice a moral state such
that in consequence of it men have the capacity of doing what is
just, and actually do it, and wish it: similarly also with respect to
Injustice, a moral state such that in consequence of it men do unjustly
and wish what is unjust: let us also be content then with these as a
ground-work sketched out.

I mention the two, because the same does not hold with regard to States
whether of mind or body as with regard to Sciences or Faculties: I mean
that whereas it is thought that the same Faculty or Science embraces
contraries, a State will not: from health, for instance, not the
contrary acts are done but the healthy ones only; we say a man walks
healthily when he walks as the healthy man would.

However, of the two contrary states the one may be frequently known from
the other, and oftentimes the states from their subject-matter: if it be
seen clearly what a good state of body is, then is it also seen what a
bad state is, and from the things which belong to a good state of body
the good state itself is seen, and _vice versa_. If, for instance,
the good state is firmness of flesh it follows that the bad state is
flabbiness of flesh; and whatever causes firmness of flesh is connected
with the good state. It follows moreover in general, that if of two
contrary terms the one is used in many senses so also will the other be;
as, for instance, if "the Just," then also "the Unjust." Now Justice and
Injustice do seem to be used respectively in many senses, but, because
the line of demarcation between these is very fine and minute, it
commonly escapes notice that they are thus used, and it is not plain
and manifest as where the various significations of terms are widely
different for in these last the visible difference is great, for
instance, the word [Greek: klehis] is used equivocally to denote the
bone which is under the neck of animals and the instrument with which
people close doors.

Let it be ascertained then in how many senses the term "Unjust man" is
used. Well, he who violates the law, and he who is a grasping man, and
the unequal man, are all thought to be Unjust and so manifestly the Just
man will be, the man who acts according to law, and the equal man "The
Just" then will be the lawful and the equal, and "the Unjust" the
unlawful and the unequal.

[Sidenote:1129b] Well, since the Unjust man is also a grasping man, he
will be so, of course, with respect to good things, but not of every
kind, only those which are the subject-matter of good and bad fortune
and which are in themselves always good but not always to the
individual. Yet men pray for and pursue these things: this they should
not do but pray that things which are in the abstract good may be so
also to them, and choose what is good for themselves.

But the Unjust man does not always choose actually the greater part, but
even sometimes the less; as in the case of things which are simply evil:
still, since the less evil is thought to be in a manner a good and the
grasping is after good, therefore even in this case he is thought to be
a grasping man, i.e. one who strives for more good than fairly falls to
his share: of course he is also an unequal man, this being an inclusive
and common term.

We said that the violator of Law is Unjust, and the keeper of the Law
Just: further, it is plain that all Lawful things are in a manner
Just, because by Lawful we understand what have been defined by the
legislative power and each of these we say is Just. The Laws too give
directions on all points, aiming either at the common good of all, or
that of the best, or that of those in power (taking for the standard
real goodness or adopting some other estimate); in one way we mean by
Just, those things which are apt to produce and preserve happiness and
its ingredients for the social community.

Further, the Law commands the doing the deeds not only of the brave man
(as not leaving the ranks, nor flying, nor throwing away one's arms),
but those also of the perfectly self-mastering man, as abstinence from
adultery and wantonness; and those of the meek man, as refraining from
striking others or using abusive language: and in like manner in respect
of the other virtues and vices commanding some things and forbidding
others, rightly if it is a good law, in a way somewhat inferior if it is
one extemporised.

Now this Justice is in fact perfect Virtue, yet not simply so but as
exercised towards one's neighbour: and for this reason Justice is
thought oftentimes to be the best of the Virtues, and

  "neither Hesper nor the Morning-star
  So worthy of our admiration:"

and in a proverbial saying we express the same;

  "All virtue is in Justice comprehended."

And it is in a special sense perfect Virtue because it is the practice
of perfect Virtue. And perfect it is because he that has it is able to
practise his virtue towards his neighbour and not merely on himself; I
mean, there are many who can practise virtue in the regulation of their
own personal conduct who are wholly unable to do it in transactions with
[Sidenote:1130a] their neighbour. And for this reason that saying of
Bias is thought to be a good one,

  "Rule will show what a man is;"

for he who bears Rule is necessarily in contact with others, i.e. in a
community. And for this same reason Justice alone of all the Virtues is
thought to be a good to others, because it has immediate relation to
some other person, inasmuch as the Just man does what is advantageous to
another, either to his ruler or fellow-subject. Now he is the basest
of men who practises vice not only in his own person but towards his
friends also; but he the best who practises virtue not merely in his
own person but towards his neighbour, for this is a matter of some
difficulty.

However, Justice in this sense is not a part of Virtue but is
co-extensive with Virtue; nor is the Injustice which answers to it a
part of Vice but co-extensive with Vice. Now wherein Justice in this
sense differs from Virtue appears from what has been said: it is the
same really, but the point of view is not the same: in so far as it has
respect to one's neighbour it is Justice, in so far as it is such and
such a moral state it is simply Virtue.

II

But the object of our inquiry is Justice, in the sense in which it is
a part of Virtue (for there is such a thing, as we commonly say), and
likewise with respect to particular Injustice. And of the existence of
this last the following consideration is a proof: there are many vices
by practising which a man acts unjustly, of course, but does not grasp
at more than his share of good; if, for instance, by reason of cowardice
he throws away his shield, or by reason of ill-temper he uses abusive
language, or by reason of stinginess does not give a friend pecuniary
assistance; but whenever he does a grasping action, it is often in the
way of none of these vices, certainly not in all of them, still in
the way of some vice or other (for we blame him), and in the way of
Injustice. There is then some kind of Injustice distinct from that
co-extensive with Vice and related to it as a part to a whole, and some
"Unjust" related to that which is co-extensive with violation of the law
as a part to a whole.

Again, suppose one man seduces a man's wife with a view to gain and
actually gets some advantage by it, and another does the same from
impulse of lust, at an expense of money and damage; this latter will be
thought to be rather destitute of self-mastery than a grasping man, and
the former Unjust but not destitute of self-mastery: now why? plainly
because of his gaining.

Again, all other acts of Injustice we refer to some particular
depravity, as, if a man commits adultery, to abandonment to his
passions; if he deserts his comrade, to cowardice; if he strikes
another, to anger: but if he gains by the act to no other vice than to
Injustice.

[Sidenote:1131b] Thus it is clear that there is a kind of Injustice
different from and besides that which includes all Vice, having the same
name because the definition is in the same genus; for both have their
force in dealings with others, but the one acts upon honour, or wealth,
or safety, or by whatever one name we can include all these things, and
is actuated by pleasure attendant on gain, while the other acts upon all
things which constitute the sphere of the good man's action.

Now that there is more than one kind of Justice, and that there is one
which is distinct from and besides that which is co-extensive with,
Virtue, is plain: we must next ascertain what it is, and what are its
characteristics.

Well, the Unjust has been divided into the unlawful and the unequal, and
the Just accordingly into the lawful and the equal: the aforementioned
Injustice is in the way of the unlawful. And as the unequal and the more
are not the same, but differing as part to whole (because all more is
unequal, but not all unequal more), so the Unjust and the Injustice we
are now in search of are not the same with, but other than, those before
mentioned, the one being the parts, the other the wholes; for this
particular Injustice is a part of the Injustice co-extensive with Vice,
and likewise this Justice of the Justice co-extensive with Virtue.
So that what we have now to speak of is the particular Justice and
Injustice, and likewise the particular Just and Unjust.

Here then let us dismiss any further consideration of the Justice
ranking as co-extensive with Virtue (being the practice of Virtue in all
its bearings towards others), and of the co-relative Injustice (being
similarly the practice of Vice). It is clear too, that we must separate
off the Just and the Unjust involved in these: because one may pretty
well say that most lawful things are those which naturally result in
action from Virtue in its fullest sense, because the law enjoins the
living in accordance with each Virtue and forbids living in accordance
with each Vice. And the producing causes of Virtue in all its bearings
are those enactments which have been made respecting education for
society.

By the way, as to individual education, in respect of which a man is
simply good without reference to others, whether it is the province of
[Greek: politikhae] or some other science we must determine at a
future time: for it may be it is not the same thing to be a good man and
a good citizen in every case.

Now of the Particular Justice, and the Just involved in it, one species
is that which is concerned in the distributions of honour, or wealth, or
such other things as are to be shared among the members of the social
community (because in these one man as compared with another may have
either an equal or an unequal share), and the other is that which is
Corrective in the various transactions between man and man.

[Sidenote: 1131a] And of this latter there are two parts: because of
transactions some are voluntary and some involuntary; voluntary, such as
follow; selling, buying, use, bail, borrowing, deposit, hiring: and this
class is called voluntary because the origination of these transactions
is voluntary.

The involuntary again are either such as effect secrecy; as theft,
adultery, poisoning, pimping, kidnapping of slaves, assassination, false
witness; or accompanied with open violence; as insult, bonds, death,
plundering, maiming, foul language, slanderous abuse.

III

Well, the unjust man we have said is unequal, and the abstract "Unjust"
unequal: further, it is plain that there is some mean of the unequal,
that is to say, the equal or exact half (because in whatever action
there is the greater and the less there is also the equal, i.e. the
exact half). If then the Unjust is unequal the Just is equal, which all
must allow without further proof: and as the equal is a mean the Just
must be also a mean. Now the equal implies two terms at least: it
follows then that the Just is both a mean and equal, and these to
certain persons; and, in so far as it is a mean, between certain things
(that is, the greater and the less), and, so far as it is equal, between
two, and in so far as it is just it is so to certain persons. The Just
then must imply four terms at least, for those to which it is just are
two, and the terms representing the things are two.

And there will be the same equality between the terms representing the
persons, as between those representing the things: because as the latter
are to one another so are the former: for if the persons are not equal
they must not have equal shares; in fact this is the very source of all
the quarrelling and wrangling in the world, when either they who are
equal have and get awarded to them things not equal, or being not equal
those things which are equal. Again, the necessity of this equality of
ratios is shown by the common phrase "according to rate," for all agree
that the Just in distributions ought to be according to some rate:
but what that rate is to be, all do not agree; the democrats are for
freedom, oligarchs for wealth, others for nobleness of birth, and the
aristocratic party for virtue.

The Just, then, is a certain proportionable thing. For proportion does
not apply merely to number in the abstract, but to number generally,
since it is equality of ratios, and implies four terms at least (that
this is the case in what may be called discrete proportion is plain and
obvious, but it is true also in continual proportion, for this uses the
one [Sidenote: 1131b] term as two, and mentions it twice; thus A:B:C may
be expressed A:B::B:C. In the first, B is named twice; and so, if, as
in the second, B is actually written twice, the proportionals will be
four): and the Just likewise implies four terms at the least, and the
ratio between the two pair of terms is the same, because the persons and
the things are divided similarly. It will stand then thus, A:B::C:D, and
then permutando A:C::B:D, and then (supposing C and D to represent the
things) A+C:B+D::A:B. The distribution in fact consisting in putting
together these terms thus: and if they are put together so as to
preserve this same ratio, the distribution puts them together justly. So
then the joining together of the first and third and second and fourth
proportionals is the Just in the distribution, and this Just is the
mean relatively to that which violates the proportionate, for
the proportionate is a mean and the Just is proportionate. Now
mathematicians call this kind of proportion geometrical: for in
geometrical proportion the whole is to the whole as each part to each
part. Furthermore this proportion is not continual, because the person
and thing do not make up one term.

The Just then is this proportionate, and the Unjust that which violates
the proportionate; and so there comes to be the greater and the less:
which in fact is the case in actual transactions, because he who acts
unjustly has the greater share and he who is treated unjustly has the
less of what is good: but in the case of what is bad this is reversed:
for the less evil compared with the greater comes to be reckoned for
good, because the less evil is more choiceworthy than the greater, and
what is choiceworthy is good, and the more so the greater good.

This then is the one species of the Just.

IV

And the remaining one is the Corrective, which arises in voluntary as
well as involuntary transactions. Now this just has a different form
from the aforementioned; for that which is concerned in distribution of
common property is always according to the aforementioned proportion: I
mean that, if the division is made out of common property, the
shares will bear the same proportion to one another as the original
contributions did: and the Unjust which is opposite to this Just is that
which violates the proportionate.

But the Just which arises in transactions between men is an equal in a
certain sense, and the Unjust an unequal, only not in the way of that
proportion but of arithmetical. [Sidenote: 1132a ] Because it makes no
difference whether a robbery, for instance, is committed by a good man
on a bad or by a bad man on a good, nor whether a good or a bad man has
committed adultery: the law looks only to the difference created by the
injury and treats the men as previously equal, where the one does and
the other suffers injury, or the one has done and the other suffered
harm. And so this Unjust, being unequal, the judge endeavours to reduce
to equality again, because really when the one party has been wounded
and the other has struck him, or the one kills and the other dies, the
suffering and the doing are divided into unequal shares; well, the judge
tries to restore equality by penalty, thereby taking from the gain.

For these terms gain and loss are applied to these cases, though perhaps
the term in some particular instance may not be strictly proper, as
gain, for instance, to the man who has given a blow, and loss to him who
has received it: still, when the suffering has been estimated, the one
is called loss and the other gain.

And so the equal is a mean between the more and the less, which
represent gain and loss in contrary ways (I mean, that the more of good
and the less of evil is gain, the less of good and the more of evil is
loss): between which the equal was stated to be a mean, which equal we
say is Just: and so the Corrective Just must be the mean between loss
and gain. And this is the reason why, upon a dispute arising, men have
recourse to the judge: going to the judge is in fact going to the Just,
for the judge is meant to be the personification of the Just. And men
seek a judge as one in the mean, which is expressed in a name given by
some to judges ([Greek: mesidioi], or middle-men) under the notion that
if they can hit on the mean they shall hit on the Just. The Just is then
surely a mean since the judge is also.

So it is the office of a judge to make things equal, and the line, as it
were, having been unequally divided, he takes from the greater part that
by which it exceeds the half, and adds this on to the less. And when the
whole is divided into two exactly equal portions then men say they have
their own, when they have gotten the equal; and the equal is a mean
between the greater and the less according to arithmetical equality.

This, by the way, accounts for the etymology of the term by which we
in Greek express the ideas of Just and Judge; ([Greek: dikaion] quasi
[Greek: dichaion], that is in two parts, and [Greek: dikastaes] quasi
[Greek: dichastaes], he who divides into two parts). For when from one
of two equal magnitudes somewhat has been taken and added to the other,
this latter exceeds the former by twice that portion: if it had been
merely taken from the former and not added to the latter, then the
latter would [Sidenote:1132b] have exceeded the former only by that one
portion; but in the other case, the greater exceeds the mean by one, and
the mean exceeds also by one that magnitude from which the portion was
taken. By this illustration, then, we obtain a rule to determine what
one ought to take from him who has the greater, and what to add to him
who has the less. The excess of the mean over the less must be added to
the less, and the excess of the greater over the mean be taken from the
greater.

Thus let there be three straight lines equal to one another. From one of
them cut off a portion, and add as much to another of them. The whole
line thus made will exceed the remainder of the first-named line, by
twice the portion added, and will exceed the untouched line by that
portion. And these terms loss and gain are derived from voluntary
exchange: that is to say, the having more than what was one's own is
called gaining, and the having less than one's original stock is called
losing; for instance, in buying or selling, or any other transactions
which are guaranteed by law: but when the result is neither more nor
less, but exactly the same as there was originally, people say they have
their own, and neither lose nor gain.

So then the Just we have been speaking of is a mean between loss and
gain arising in involuntary transactions; that is, it is the having the
same after the transaction as one had before it took place.

[Sidenote: V] There are people who have a notion that Reciprocation is
simply just, as the Pythagoreans said: for they defined the Just simply
and without qualification as "That which reciprocates with another." But
this simple Reciprocation will not fit on either to the Distributive
Just, or the Corrective (and yet this is the interpretation they put
on the Rhadamanthian rule of Just, If a man should suffer what he hath
done, then there would be straightforward justice"), for in many
cases differences arise: as, for instance, suppose one in authority
has struck a man, he is not to be struck in turn; or if a man has
struck one in authority, he must not only be struck but punished also.
And again, the voluntariness or involuntariness of actions makes a
great difference.

[Sidenote: II33_a_] But in dealings of exchange such a principle of
Justice as this Reciprocation forms the bond of union, but then it must
be Reciprocation according to proportion and not exact equality, because
by proportionate reciprocity of action the social community is held
together, For either Reciprocation of evil is meant, and if this be
not allowed it is thought to be a servile condition of things: or else
Reciprocation of good, and if this be not effected then there is no
admission to participation which is the very bond of their union.

And this is the moral of placing the Temple of the Graces ([Greek:
charites]) in the public streets; to impress the notion that there may
be requital, this being peculiar to [Greek: charis] because a man ought
to requite with a good turn the man who has done him a favour and then
to become himself the originator of another [Greek: charis], by doing
him a favour.

Now the acts of mutual giving in due proportion may be represented
by the diameters of a parallelogram, at the four angles of which the
parties and their wares are so placed that the side connecting the
parties be opposite to that connecting the wares, and each party be
connected by one side with his own ware, as in the accompanying diagram.

[Illustration: Builder_Shoemaker House_Shoes.]

The builder is to receive from the shoemaker of his ware, and to give
him of his own: if then there be first proportionate equality, and
_then_ the Reciprocation takes place, there will be the just result
which we are speaking of: if not, there is not the equal, nor will the
connection stand: for there is no reason why the ware of the one may not
be better than that of the other, and therefore before the exchange is
made they must have been equalised. And this is so also in the other
arts: for they would have been destroyed entirely if there were not a
correspondence in point of quantity and quality between the producer and
the consumer. For, we must remember, no dealing arises between two of
the same kind, two physicians, for instance; but say between a physician
and agriculturist, or, to state it generally, between those who are
different and not equal, but these of course must have been equalised
before the exchange can take place.

It is therefore indispensable that all things which can be exchanged
should be capable of comparison, and for this purpose money has come
in, and comes to be a kind of medium, for it measures all things and so
likewise the excess and defect; for instance, how many shoes are equal
to a house or a given quantity of food. As then the builder to the
shoemaker, so many shoes must be to the house (or food, if instead of a
builder an agriculturist be the exchanging party); for unless there is
this proportion there cannot be exchange or dealing, and this proportion
cannot be unless the terms are in some way equal; hence the need, as was
stated above, of some one measure of all things. Now this is really
and truly the Demand for them, which is the common bond of all such
dealings. For if the parties were not in want at all or not similarly of
one another's wares, there would either not be any exchange, or at least
not the same.

And money has come to be, by general agreement, a representative of
Demand: and the account of its Greek name [Greek: nomisma] is this, that
it is what it is not naturally but by custom or law ([Greek: nomos]),
and it rests with us to change its value, or make it wholly useless.

[Sidenote: 1113b] Very well then, there will be Reciprocation when
the terms have been equalised so as to stand in this proportion;
Agriculturist : Shoemaker : : wares of Shoemaker : wares of
Agriculturist; but you must bring them to this form of proportion when
they exchange, otherwise the one extreme will combine both exceedings of
the mean: but when they have exactly their own then they are equal and
have dealings, because the same equality can come to be in their case.
Let A represent an agriculturist, C food, B a shoemaker, D his wares
equalised with A's. Then the proportion will be correct, A:B::C:D; _now_
Reciprocation will be practicable, if it were not, there would have been
no dealing.

Now that what connects men in such transactions is Demand, as being some
one thing, is shown by the fact that, when either one does not want the
other or neither want one another, they do not exchange at all: whereas
they do when one wants what the other man has, wine for instance, giving
in return corn for exportation.

And further, money is a kind of security to us in respect of exchange
at some future time (supposing that one wants nothing now that we shall
have it when we do): the theory of money being that whenever one brings
it one can receive commodities in exchange: of course this too is liable
to depreciation, for its purchasing power is not always the same,
but still it is of a more permanent nature than the commodities it
represents. And this is the reason why all things should have a price
set upon them, because thus there may be exchange at any time, and if
exchange then dealing. So money, like a measure, making all things
commensurable equalises them: for if there was not exchange there would
not have been dealing, nor exchange if there were not equality, nor
equality if there were not the capacity of being commensurate: it
is impossible that things so greatly different should be really
commensurate, but we can approximate sufficiently for all practical
purposes in reference to Demand. The common measure must be some one
thing, and also from agreement (for which reason it is called [Greek:
nomisma]), for this makes all things commensurable: in fact, all things
are measured by money. Let B represent ten minæ, A a house worth five
minæ, or in other words half B, C a bed worth 1/10th of B: it is clear
then how many beds are equal to one house, namely, five.

It is obvious also that exchange was thus conducted before the existence
of money: for it makes no difference whether you give for a house five
beds or the price of five beds. We have now said then what the abstract
Just and Unjust are, and these having been defined it is plain that
just acting is a mean between acting unjustly and being acted unjustly
towards: the former being equivalent to having more, and the latter to
having less.

But Justice, it must be observed, is a mean state not after the same
manner as the forementioned virtues, but because it aims at producing
the mean, while Injustice occupies _both_ the extremes.

[Sidenote: 1134_a_] And Justice is the moral state in virtue of which
the just man is said to have the aptitude for practising the Just in
the way of moral choice, and for making division between _, himself and
another, or between two other men, not so as to give to himself the
greater and to his neighbour the less share of what is choiceworthy and
contrariwise of what is hurtful, but what is proportionably equal, and
in like manner when adjudging the rights of two other men.

Injustice is all this with respect to the Unjust: and since the Unjust
is excess or defect of what is good or hurtful respectively, in
violation of the proportionate, therefore Injustice is both excess and
defect because it aims at producing excess and defect; excess, that is,
in a man's own case of what is simply advantageous, and defect of what
is hurtful: and in the case of other men in like manner generally
speaking, only that the proportionate is violated not always in one
direction as before but whichever way it happens in the given case. And
of the Unjust act the less is being acted unjustly towards, and the
greater the acting unjustly towards others.

Let this way of describing the nature of Justice and Injustice, and
likewise the Just and the Unjust generally, be accepted as sufficient.

[Sidenote: VI] Again, since a man may do unjust acts and not yet have
formed a character of injustice, the question arises whether a man is
unjust in each particular form of injustice, say a thief, or adulterer,
or robber, by doing acts of a given character.

We may say, I think, that this will not of itself make any difference; a
man may, for instance, have had connection with another's wife, knowing
well with whom he was sinning, but he may have done it not of deliberate
choice but from the impulse of passion: of course he acts unjustly, but
he has not necessarily formed an unjust character: that is, he may have
stolen yet not be a thief; or committed an act of adultery but still not
be an adulterer, and so on in other cases which might be enumerated.

Of the relation which Reciprocation bears to the Just we have already
spoken: and here it should be noticed that the Just which we are
investigating is both the Just in the abstract and also as exhibited in
Social Relations, which latter arises in the case of those who live in
communion with a view to independence and who are free and equal either
proportionately or numerically.

It follows then that those who are not in this position have not among
themselves the Social Just, but still Just of some kind and resembling
that other. For Just implies mutually acknowledged law, and law the
possibility of injustice, for adjudication is the act of distinguishing
between the Just and the Unjust.

And among whomsoever there is the possibility of injustice among these
there is that of acting unjustly; but it does not hold conversely that
injustice attaches to all among whom there is the possibility of acting
unjustly, since by the former we mean giving one's self the larger share
of what is abstractedly good and the less of what is abstractedly evil.

[Sidenote: 134_b_] This, by the way, is the reason why we do not allow
a man to govern, but Principle, because a man governs for himself and
comes to be a despot: but the office of a ruler is to be guardian of the
Just and therefore of the Equal. Well then, since he seems to have no
peculiar personal advantage, supposing him a Just man, for in this case
he does not allot to himself the larger share of what is abstractedly
good unless it falls to his share proportionately (for which reason he
really governs for others, and so Justice, men say, is a good not to
one's self so much as to others, as was mentioned before), therefore
some compensation must be given him, as there actually is in the shape
of honour and privilege; and wherever these are not adequate there
rulers turn into despots.

But the Just which arises in the relations of Master and Father, is not
identical with, but similar to, these; because there is no possibility
of injustice towards those things which are absolutely one's own; and
a slave or child (so long as this last is of a certain age and not
separated into an independent being), is, as it were, part of a man's
self, and no man chooses to hurt himself, for which reason there cannot
be injustice towards one's own self: therefore neither is there the
social Unjust or Just, which was stated to be in accordance with law and
to exist between those among whom law naturally exists, and these were
said to be they to whom belongs equality of ruling and being ruled.

Hence also there is Just rather between a man and his wife than between
a man and his children or slaves; this is in fact the Just arising in
domestic relations: and this too is different from the Social Just.

[Sidenote: VII] Further, this last-mentioned Just is of two kinds,
natural and conventional; the former being that which has everywhere the
same force and does not depend upon being received or not; the latter
being that which originally may be this way or that indifferently but
not after enactment: for instance, the price of ransom being fixed at
a mina, or the sacrificing a goat instead of two sheep; and again, all
cases of special enactment, as the sacrificing to Brasidas as a hero; in
short, all matters of special decree.

But there are some men who think that all the Justs are of this latter
kind, and on this ground: whatever exists by nature, they say, is
unchangeable and has everywhere the same force; fire, for instance,
burns not here only but in Persia as well, but the Justs they see
changed in various places.

Now this is not really so, and yet it is in a way (though among the gods
perhaps by no means): still even amongst ourselves there is somewhat
existing by nature: allowing that everything is subject to change, still
there is that which does exist by nature, and that which does not.

Nay, we may go further, and say that it is practically plain what among
things which can be otherwise does exist by nature, and what does not
but is dependent upon enactment and conventional, even granting
that both are alike subject to be changed: and the same distinctive
illustration will apply to this and other cases; the right hand is
naturally the stronger, still some men may become equally strong in
both.

[Sidenote: 1135_a_] A parallel may be drawn between the Justs which
depend upon convention and expedience, and measures; for wine and corn
measures are not equal in all places, but where men buy they are large,
and where these same sell again they are smaller: well, in like manner
the Justs which are not natural, but of human invention, are not
everywhere the same, for not even the forms of government are, and yet
there is one only which by nature would be best in all places.

Now of Justs and Lawfuls each bears to the acts which embody and
exemplify it the relation of an universal to a particular; the acts
being many, but each of the principles only singular because each is an
universal. And so there is a difference between an unjust act and the
abstract Unjust, and the just act and the abstract Just: I mean, a thing
is unjust in itself, by nature or by ordinance; well, when this has been
embodied in act, there is an unjust act, but not till then, only
some unjust thing. And similarly of a just act. (Perhaps [Greek:
dikaiopragaema] is more correctly the common or generic term for just
act, the word [Greek: dikaioma], which I have here used, meaning
generally and properly the act corrective of the unjust act.) Now as
to each of them, what kinds there are, and how many, and what is their
object-matter, we must examine afterwards.

[Sidenote: VIII] For the present we proceed to say that, the Justs
and the Unjusts being what have been mentioned, a man is said to act
unjustly or justly when he embodies these abstracts in voluntary
actions, but when in involuntary, then he neither acts unjustly or
justly except accidentally; I mean that the being just or unjust is
really only accidental to the agents in such cases.

So both unjust and just actions are limited by the being voluntary or
the contrary: for when an embodying of the Unjust is voluntary, then
it is blamed and is at the same time also an unjust action: but, if
voluntariness does not attach, there will be a thing which is in itself
unjust but not yet an unjust action.

By voluntary, I mean, as we stated before, whatsoever of things in his
own power a man does with knowledge, and the absence of ignorance as to
the person to whom, or the instrument with which, or the result with
which he does; as, for instance, whom he strikes, what he strikes him
with, and with what probable result; and each of these points again, not
accidentally nor by compulsion; as supposing another man were to seize
his hand and strike a third person with it, here, of course, the owner
of the hand acts not voluntarily, because it did not rest with him to do
or leave undone: or again, it is conceivable that the person struck may
be his father, and he may know that it is a man, or even one of the
present company, whom he is striking, but not know that it is his
father. And let these same distinctions be supposed to be carried into
the case of the result and in fact the whole of any given action. In
fine then, that is involuntary which is done through ignorance, or
which, not resulting from ignorance, is not in the agent's control or is
done on compulsion.

I mention these cases, because there are many natural *[Sidenote:
1135_b_] things which we do and suffer knowingly but still no one of
which is either voluntary or involuntary, growing old, or dying, for
instance.

Again, accidentality may attach to the unjust in like manner as to the
just acts. For instance, a man may have restored what was deposited
with him, but against his will and from fear of the consequences of
a refusal: we must not say that he either does what is just, or does
justly, except accidentally: and in like manner the man who through
compulsion and against his will fails to restore a deposit, must be said
to do unjustly, or to do what is unjust, accidentally only.

Again, voluntary actions we do either from deliberate choice or without
it; from it, when we act from previous deliberation; without it, when
without any previous deliberation. Since then hurts which may be done in
transactions between man and man are threefold, those mistakes which are
attended with ignorance are, when a man either does a thing not to the
man to whom he meant to do it, or not the thing he meant to do, or not
with the instrument, or not with the result which he intended: either he
did not think he should hit him at all, or not with this, or this is not
the man he thought he should hit, or he did not think this would be
the result of the blow but a result has followed which he did not
anticipate; as, for instance, he did it not to wound but merely to prick
him; or it is not the man whom, or the way in which, he meant.

Now when the hurt has come about contrary to all reasonable expectation,
it is a Misadventure; when though not contrary to expectation yet
without any viciousness, it is a Mistake; for a man makes a mistake when
the origination of the cause rests with himself, he has a misadventure
when it is external to himself. When again he acts with knowledge, but
not from previous deliberation, it is an unjust action; for instance,
whatever happens to men from anger or other passions which are necessary
or natural: for when doing these hurts or making these mistakes they act
unjustly of course and their actions are unjust, still they are not yet
confirmed unjust or wicked persons by reason of these, because the hurt
did not arise from depravity in the doer of it: but when it does arise
from deliberate choice, then the doer is a confirmed unjust and depraved
man.

And on this principle acts done from anger are fairly judged not to be
from malice prepense, because it is not the man who acts in wrath who
is the originator really but he who caused his wrath. And again,
the question at issue in such cases is not respecting the fact but
respecting the justice of the case, the occasion of anger being a notion
of injury. I mean, that the parties do not dispute about the fact, as in
questions of contract (where one of the two must be a rogue, unless real
forgetfulness can be pleaded), but, admitting the fact, they dispute on
which side the justice of the case lies (the one who plotted against the
other, _i.e._ the real aggressor, of course, cannot be ignorant), so
that the one thinks there is injustice committed while the other does
not.

[Sidenote: 11364] Well then, a man acts unjustly if he has hurt another
of deliberate purpose, and he who commits such acts of injustice is
_ipso facto_ an unjust character when they are in violation of the
proportionate or the equal; and in like manner also a man is a just
character when he acts justly of deliberate purpose, and he does act
justly if he acts voluntarily.

Then as for involuntary acts of harm, they are either such as are
excusable or such as are not: under the former head come all errors done
not merely in ignorance but from ignorance; under the latter all that
are done not from ignorance but in ignorance caused by some passion
which is neither natural nor fairly attributable to human infirmity.

[Sidenote: IX] Now a question may be raised whether we have spoken with
sufficient distinctness as to being unjustly dealt with, and dealing
unjustly towards others. First, whether the case is possible which
Euripides has put, saying somewhat strangely,

  "My mother he hath slain;  the tale is short,
  Either he willingly did slay her willing,
  Or else with her will but against his own."

I mean then, is it really possible for a person to be unjustly dealt
with with his own consent, or must every case of being unjustly dealt
with be against the will of the sufferer as every act of unjust dealing
is voluntary?

And next, are cases of being unjustly dealt with to be ruled all one way
as every act of unjust dealing is voluntary? or may we say that some
cases are voluntary and some involuntary?

Similarly also as regards being justly dealt with: all just acting is
voluntary, so that it is fair to suppose that the being dealt with
unjustly or justly must be similarly opposed, as to being either
voluntary or involuntary.

Now as for being justly dealt with, the position that every case of this
is voluntary is a strange one, for some are certainly justly dealt
with without their will. The fact is a man may also fairly raise this
question, whether in every case he who has suffered what is unjust is
therefore unjustly dealt with, or rather that the case is the same with
suffering as it is with acting; namely that in both it is possible to
participate in what is just, but only accidentally. Clearly the case of
what is unjust is similar: for doing things in themselves unjust is not
identical with acting unjustly, nor is suffering them the same as being
unjustly dealt with. So too of acting justly and being justly dealt
with, since it is impossible to be unjustly dealt with unless some one
else acts unjustly or to be justly dealt with unless some one else acts
justly.

Now if acting unjustly is simply "hurting another voluntarily" (by which
I mean, knowing whom you are hurting, and wherewith, and how you are
hurting him), and the man who fails of self-control voluntarily hurts
himself, then this will be a case of being voluntarily dealt unjustly
with, and it will be possible for a man to deal unjustly with himself.
(This by the way is one of the questions raised, whether it is possible
for a man to deal unjustly with himself.) Or again, a man may, by
reason of failing of self-control, receive hurt from another man acting
voluntarily, and so here will be another case of being unjustly dealt
with voluntarily. [Sidenote: 1136]

The solution, I take it, is this: the definition of being unjustly dealt
with is not correct, but we must add, to the hurting with the knowledge
of the person hurt and the instrument and the manner of hurting him, the
fact of its being against the wish of the man who is hurt.

So then a man may be hurt and suffer what is in itself unjust
voluntarily, but unjustly dealt with voluntarily no man can be: since no
man wishes to be hurt, not even he who fails of self-control, who really
acts contrary to his wish: for no man wishes for that which he does not
_think_ to be good, and the man who fails of self-control does not what
he thinks he ought to do.

And again, he that gives away his own property (as Homer says Glaucus
gave to Diomed, "armour of gold for brass, armour worth a hundred oxen
for that which was worth but nine") is not unjustly dealt with, because
the giving rests entirely with himself; but being unjustly dealt with
does not, there must be some other person who is dealing unjustly
towards him.

With respect to being unjustly dealt with then, it is clear that it is
not voluntary.

There remain yet two points on which we purposed to speak: first, is he
chargeable with an unjust act who in distribution has _given_ the larger
share to one party contrary to the proper rate, or he that _has_ the
larger share? next, can a man deal unjustly by himself?

In the first question, if the first-named alternative is possible and
it is the distributor who acts unjustly and not he who has the larger
share, then supposing that a person knowingly and willingly gives more
to another than to himself here is a case of a man dealing unjustly by
himself; which, in fact, moderate men are thought to do, for it is a
characteristic of the equitable man to take less than his due.

Is not this the answer? that the case is not quite fairly stated,
because of some other good, such as credit or the abstract honourable,
in the supposed case the man did get the larger share. And again, the
difficulty is solved by reference to the definition of unjust dealing:
for the man suffers nothing contrary to his own wish, so that, on this
score at least, he is not unjustly dealt with, but, if anything, he is
hurt only.

It is evident also that it is the distributor who acts unjustly and not
the man who has the greater share: because the mere fact of the abstract
Unjust attaching to what a man does, does not constitute unjust action,
but the doing this voluntarily: and voluntariness attaches to that
quarter whence is the origination of the action, which clearly is in the
distributor not in the receiver. And again the term doing is used in
several senses; in one sense inanimate objects kill, or the hand, or
the slave by his master's bidding; so the man in question does not act
unjustly but does things which are in themselves unjust.

[Sidenote: 1137a] Again, suppose that a man has made a wrongful award
in ignorance; in the eye of the law he does not act unjustly nor is
his awarding unjust, but yet he is in a certain sense: for the Just
according to law and primary or natural Just are not coincident: but, if
he knowingly decided unjustly, then he himself as well as the receiver
got the larger share, that is, either of favour from the receiver or
private revenge against the other party: and so the man who decided
unjustly from these motives gets a larger share, in exactly the same
sense as a man would who received part of the actual matter of the
unjust action: because in this case the man who wrongly adjudged, say a
field, did not actually get land but money by his unjust decision.

Now men suppose that acting Unjustly rests entirely with themselves,
and conclude that acting Justly is therefore also easy. But this is not
really so; to have connection with a neighbour's wife, or strike one's
neighbour, or give the money with one's hand, is of course easy and
rests with one's self: but the doing these acts with certain inward
dispositions neither is easy nor rests entirely with one's self. And in
like way, the knowing what is Just and what Unjust men think no great
instance of wisdom because it is not hard to comprehend those things
of which the laws speak. They forget that these are not Just actions,
except accidentally: to be Just they must be done and distributed in
a certain manner: and this is a more difficult task than knowing what
things are wholesome; for in this branch of knowledge it is an easy
matter to know honey, wine, hellebore, cautery, or the use of the knife,
but the knowing how one should administer these with a view to health,
and to whom and at what time, amounts in fact to being a physician.

From this very same mistake they suppose also, that acting Unjustly is
equally in the power of the Just man, for the Just man no less, nay even
more, than the Unjust, may be able to do the particular acts; he may be
able to have intercourse with a woman or strike a man; or the brave man
to throw away his shield and turn his back and run this way or that.
True: but then it is not the mere doing these things which constitutes
acts of cowardice or injustice (except accidentally), but the doing them
with certain inward dispositions: just as it is not the mere using or
not using the knife, administering or not administering certain drugs,
which constitutes medical treatment or curing, but doing these things in
a certain particular way.

Again the abstract principles of Justice have their province among those
who partake of what is abstractedly good, and can have too much or too
little of these. Now there are beings who cannot have too much of them,
as perhaps the gods; there are others, again, to whom no particle of
them is of use, those who are incurably wicked to whom all things are
hurtful; others to whom they are useful to a certain degree: for this
reason then the province of Justice is among Men.

[Sidenote: 1137b] We have next to speak of Equity and the Equitable,
that is to say, of the relations of Equity to Justice and the Equitable
to the Just; for when we look into the matter the two do not appear
identical nor yet different in kind; and we sometimes commend the
Equitable and the man who embodies it in his actions, so that by way of
praise we commonly transfer the term also to other acts instead of the
term good, thus showing that the more Equitable a thing is the better it
is: at other times following a certain train of reasoning we arrive at a
difficulty, in that the Equitable though distinct from the Just is yet
praiseworthy; it seems to follow either that the Just is not good or the
Equitable not Just, since they are by hypothesis different; or if both
are good then they are identical.

This is a tolerably fair statement of the difficulty which on these
grounds arises in respect of the Equitable; but, in fact, all these may
be reconciled and really involve no contradiction: for the Equitable is
Just, being also better than one form of Just, but is not better than
the Just as though it were different from it in kind: Just and Equitable
then are identical, and, both being good, the Equitable is the better of
the two.

What causes the difficulty is this; the Equitable is Just, but not the
Just which is in accordance with written law, being in fact a correction
of that kind of Just. And the account of this is, that every law is
necessarily universal while there are some things which it is not
possible to speak of rightly in any universal or general statement.
Where then there is a necessity for general statement, while a general
statement cannot apply rightly to all cases, the law takes the
generality of cases, being fully aware of the error thus involved; and
rightly too notwithstanding, because the fault is not in the law, or
in the framer of the law, but is inherent in the nature of the thing,
because the matter of all action is necessarily such.

When then the law has spoken in general terms, and there arises a
case of exception to the general rule, it is proper, in so far as the
lawgiver omits the case and by reason of his universality of statement
is wrong, to set right the omission by ruling it as the lawgiver himself
would rule were he there present, and would have provided by law had he
foreseen the case would arise. And so the Equitable is Just but better
than one form of Just; I do not mean the abstract Just but the error
which arises out of the universality of statement: and this is the
nature of the Equitable, "a correction of Law, where Law is defective by
reason of its universality."

This is the reason why not all things are according to law, because
there are things about which it is simply impossible to lay down a law,
and so we want special enactments for particular cases. For to speak
generally, the rule of the undefined must be itself undefined also, just
as the rule to measure Lesbian building is made of lead: for this rule
shifts according to the form of each stone and the special enactment
according to the facts of the case in question.

[Sidenote: 1138a] It is clear then what the Equitable is; namely that it
is Just but better than one form of Just: and hence it appears too who
the Equitable man is: he is one who has a tendency to choose and carry
out these principles, and who is not apt to press the letter of the law
on the worse side but content to waive his strict claims though backed
by the law: and this moral state is Equity, being a species of Justice,
not a different moral state from Justice.

XI

The answer to the second of the two questions indicated above, "whether
it is possible for a man to deal unjustly by himself," is obvious from
what has been already stated. In the first place, one class of Justs is
those which are enforced by law in accordance with Virtue in the most
extensive sense of the term: for instance, the law does not bid a man
kill himself; and whatever it does not bid it forbids: well, whenever a
man does hurt contrary to the law (unless by way of requital of hurt),
voluntarily, i.e. knowing to whom he does it and wherewith, he acts
Unjustly. Now he that from rage kills himself, voluntarily, does this
in contravention of Right Reason, which the law does not permit. He
therefore acts Unjustly: but towards whom? towards the Community, not
towards himself (because he suffers with his own consent, and no man can
be Unjustly dealt with with his own consent), and on this principle the
Community punishes him; that is a certain infamy is attached to the
suicide as to one who acts Unjustly towards the Community.

Next, a man cannot deal Unjustly by himself in the sense in which a man
is Unjust who only does Unjust acts without being entirely bad (for the
two things are different, because the Unjust man is in a way bad, as the
coward is, not as though he were chargeable with badness in the full
extent of the term, and so he does not act Unjustly in this sense),
because if it were so then it would be possible for the same thing to
have been taken away from and added to the same person: but this is
really not possible, the Just and the Unjust always implying a plurality
of persons.

Again, an Unjust action must be voluntary, done of deliberate purpose,
and aggressive (for the man who hurts because he has first suffered and
is merely requiting the same is not thought to act Unjustly), but here
the man does to himself and suffers the same things at the same time.

Again, it would imply the possibility of being Unjustly dealt with with
one's own consent.

And, besides all this, a man cannot act Unjustly without his act falling
under some particular crime; now a man cannot seduce his own wife,
commit a burglary on his own premises, or steal his own property. After
all, the general answer to the question is to allege what was settled
respecting being Unjustly dealt with with one's own consent.

It is obvious, moreover, that being Unjustly dealt by and dealing
Unjustly by others are both wrong; because the one is having less, the
other having more, than the mean, and the case is parallel to that of
the healthy in the healing art, and that of good condition in the art of
training: but still the dealing Unjustly by others is the worst of the
two, because this involves wickedness and is blameworthy; wickedness, I
mean, either wholly, or nearly so (for not all voluntary wrong implies
injustice), but the being Unjustly dealt by does not involve wickedness
or injustice.

[Sidenote: 1138b] In itself then, the being Unjustly dealt by is the
least bad, but accidentally it may be the greater evil of the two.
However, scientific statement cannot take in such considerations; a
pleurisy, for instance, is called a greater physical evil than a bruise:
and yet this last may be the greater accidentally; it may chance that a
bruise received in a fall may cause one to be captured by the enemy and
slain.

Further: Just, in the way of metaphor and similitude, there may be I do
not say between a man and himself exactly but between certain parts of
his nature; but not Just of every kind, only such as belongs to the
relation of master and slave, or to that of the head of a family. For
all through this treatise the rational part of the Soul has been viewed
as distinct from the irrational.

Now, taking these into consideration, there is thought to be a
possibility of injustice towards one's self, because herein it is
possible for men to suffer somewhat in contradiction of impulses really
their own; and so it is thought that there is Just of a certain kind
between these parts mutually, as between ruler and ruled.

Let this then be accepted as an account of the distinctions which we
recognise respecting Justice and the rest of the moral virtues.




BOOK VI


I having stated in a former part of this treatise that men should choose
the mean instead of either the excess or defect, and that the mean
is according to the dictates of Right Reason; we will now proceed to
explain this term.

For in all the habits which we have expressly mentioned, as likewise
in all the others, there is, so to speak, a mark with his eye fixed on
which the man who has Reason tightens or slacks his rope; and there is a
certain limit of those mean states which we say are in accordance with
Right Reason, and lie between excess on the one hand and defect on the
other.

Now to speak thus is true enough but conveys no very definite meaning:
as, in fact, in all other pursuits requiring attention and diligence on
which skill and science are brought to bear; it is quite true of course
to say that men are neither to labour nor relax too much or too little,
but in moderation, and as Right Reason directs; yet if this were all
a man had he would not be greatly the wiser; as, for instance, if in
answer to the question, what are proper applications to the body, he
were to be told, "Oh! of course, whatever the science of medicine, and
in such manner as the physician, directs."

And so in respect of the mental states it is requisite not merely that
this should be true which has been already stated, but further that it
should be expressly laid down what Right Reason is, and what is the
definition of it.

[Sidenote: 1139a] Now in our division of the Excellences of the Soul, we
said there were two classes, the Moral and the Intellectual: the former
we have already gone through; and we will now proceed to speak of the
others, premising a few words respecting the Soul itself. It was
stated before, you will remember, that the Soul consists of two parts,
the Rational, and Irrational: we must now make a similar division of the
Rational.

Let it be understood then that there are two parts of the Soul possessed
of Reason; one whereby we realise those existences whose causes cannot
be otherwise than they are, and one whereby we realise those which can
be otherwise than they are (for there must be, answering to things
generically different, generically different parts of the soul naturally
adapted to each, since these parts of the soul possess their knowledge
in virtue of a certain resemblance and appropriateness in themselves to
the objects of which they are percipients); and let us name the
former, "that which is apt to know," the latter, "that which is apt to
calculate" (because deliberating and calculating are the same, and no
one ever deliberates about things which cannot be otherwise than they
are: and so the Calculative will be one part of the Rational faculty of
the soul).

We must discover, then, which is the best state of each of these,
because that will be the Excellence of each; and this again is relative
to the work each has to do.

II

There are in the Soul three functions on which depend moral action and
truth; Sense, Intellect, Appetition, whether vague Desire or definite
Will. Now of these Sense is the originating cause of no moral action, as
is seen from the fact that brutes have Sense but are in no way partakers
of moral action.

[Intellect and Will are thus connected,] what in the Intellectual
operation is Affirmation and Negation that in the Will is Pursuit and
Avoidance, And so, since Moral Virtue is a State apt to exercise Moral
Choice and Moral Choice is Will consequent on deliberation, the Reason
must be true and the Will right, to constitute good Moral Choice, and
what the Reason affirms the Will must pursue. Now this Intellectual
operation and this Truth is what bears upon Moral Action; of course
truth and falsehood than the conclusion such knowledge as he has will be
merely accidental.

IV

[Sidenote:1140a] Let thus much be accepted as a definition of Knowledge.
Matter which may exist otherwise than it actually does in any given case
(commonly called Contingent) is of two kinds, that which is the object
of Making, and that which is the object of Doing; now Making and Doing
are two different things (as we show in the exoteric treatise), and
so that state of mind, conjoined with Reason, which is apt to Do, is
distinct from that also conjoined with Reason, which is apt to Make: and
for this reason they are not included one by the other, that is, Doing
is not Making, nor Making Doing. Now as Architecture is an Art, and is
the same as "a certain state of mind, conjoined with Reason, which is
apt to Make," and as there is no Art which is not such a state, nor any
such state which is not an Art, Art, in its strict and proper sense,
must be "a state of mind, conjoined with true Reason, apt to Make."

Now all Art has to do with production, and contrivance, and seeing how
any of those things may be produced which may either be or not be, and
the origination of which rests with the maker and not with the thing
made.

And, so neither things which exist or come into being necessarily, nor
things in the way of nature, come under the province of Art, because
these are self-originating. And since Making and Doing are distinct, Art
must be concerned with the former and not the latter. And in a certain
sense Art and Fortune are concerned with the same things, as, Agathon
says by the way,

  "Art Fortune loves, and is of her beloved."

So Art, as has been stated, is "a certain state of mind, apt to Make,
conjoined with true Reason;" its absence, on the contrary, is the same
state conjoined with false Reason, and both are employed upon Contingent
matter.

V

As for Practical Wisdom, we shall ascertain its nature by examining to
what kind of persons we in common language ascribe it.

[Sidenote: 1140b] It is thought then to be the property of the
Practically Wise man to be able to deliberate well respecting what is
good and expedient for himself, not in any definite line, as what is
conducive to health or strength, but what to living well. A proof of
this is that we call men Wise in this or that, when they calculate well
with a view to some good end in a case where there is no definite
rule. And so, in a general way of speaking, the man who is good at
deliberation will be Practically Wise. Now no man deliberates respecting
things which cannot be otherwise than they are, nor such as lie not
within the range of his own action: and so, since Knowledge requires
strict demonstrative reasoning, of which Contingent matter does not
admit (I say Contingent matter, because all matters of deliberation must
be Contingent and deliberation cannot take place with respect to things
which are Necessarily), Practical Wisdom cannot be Knowledge nor Art;
nor the former, because what falls under the province of Doing must be
Contingent; not the latter, because Doing and Making are different in
kind.

It remains then that it must be "a state of mind true, conjoined with
Reason, and apt to Do, having for its object those things which are good
or bad for Man:" because of Making something beyond itself is always the
object, but cannot be of Doing because the very well-doing is in itself
an End.

For this reason we think Pericles and men of that stamp to be
Practically Wise, because they can see what is good for themselves and
for men in general, and we also think those to be such who are skilled
in domestic management or civil government. In fact, this is the reason
why we call the habit of perfected self-mastery by the name which in
Greek it bears, etymologically signifying "that which preserves the
Practical Wisdom:" for what it does preserve is the Notion I have
mentioned, _i.e._ of one's own true interest, For it is not every kind
of Notion which the pleasant and the painful corrupt and pervert, as,
for instance, that "the three angles of every rectilineal triangle are
equal to two right angles," but only those bearing on moral action.

For the Principles of the matters of moral action are the final cause
of them: now to the man who has been corrupted by reason of pleasure or
pain the Principle immediately becomes obscured, nor does he see that it
is his duty to choose and act in each instance with a view to this final
cause and by reason of it: for viciousness has a tendency to destroy the
moral Principle: and so Practical Wisdom must be "a state conjoined with
reason, true, having human good for its object, and apt to do."

Then again Art admits of degrees of excellence, but Practical Wisdom
does not: and in Art he who goes wrong purposely is preferable to him
who does so unwittingly, but not so in respect of Practical Wisdom or
the other Virtues. It plainly is then an Excellence of a certain kind,
and not an Art.

Now as there are two parts of the Soul which have Reason, it must be the
Excellence of the Opinionative [which we called before calculative or
deliberative], because both Opinion and Practical Wisdom are exercised
upon Contingent matter. And further, it is not simply a state conjoined
with Reason, as is proved by the fact that such a state may be forgotten
and so lost while Practical Wisdom cannot.

VI

Now Knowledge is a conception concerning universals and Necessary
matter, and there are of course certain First Principles in all trains
of demonstrative reasoning (that is of all Knowledge because this is
connected with reasoning): that faculty, then, which takes in the first
principles of that which comes under the range of Knowledge, cannot be
either Knowledge, or Art, or Practical Wisdom: not Knowledge, because
what is the object of Knowledge must be derived from demonstrative
reasoning; not either of the other two, because they are exercised upon
Contingent matter only. [Sidenote: 1141a] Nor can it be Science which
takes in these, because the Scientific Man must in some cases depend on
demonstrative Reasoning.

It comes then to this: since the faculties whereby we always attain
truth and are never deceived when dealing with matter Necessary or even
Contingent are Knowledge, Practical Wisdom, Science, and Intuition, and
the faculty which takes in First Principles cannot be any of the three
first; the last, namely Intuition, must be it which performs this
function.

VII

Science is a term we use principally in two meanings: in the first
place, in the Arts we ascribe it to those who carry their arts to the
highest accuracy; Phidias, for instance, we call a Scientific or cunning
sculptor; Polycleitus a Scientific or cunning statuary; meaning, in this
instance, nothing else by Science than an excellence of art: in the
other sense, we think some to be Scientific in a general way, not in any
particular line or in any particular thing, just as Homer says of a man
in his Margites; "Him the Gods made neither a digger of the ground, nor
ploughman, nor in any other way Scientific."

So it is plain that Science must mean the most accurate of all
Knowledge; but if so, then the Scientific man must not merely know the
deductions from the First Principles but be in possession of truth
respecting the First Principles. So that Science must be equivalent
to Intuition and Knowledge; it is, so to speak, Knowledge of the most
precious objects, _with a head on_.

I say of the most precious things, because it is absurd to suppose
[Greek: politikae], or Practical Wisdom, to be the highest, unless it
can be shown that Man is the most excellent of all that exists in the
Universe. Now if "healthy" and "good" are relative terms, differing
when applied to men or to fish, but "white" and "straight" are the same
always, men must allow that the Scientific is the same always, but the
Practically Wise varies: for whatever provides all things well for
itself, to this they would apply the term Practically Wise, and commit
these matters to it; which is the reason, by the way, that they call
some brutes Practically Wise, such that is as plainly have a faculty of
forethought respecting their own subsistence.

And it is quite plain that Science and [Greek: politikae] cannot be
identical: because if men give the name of Science to that faculty which
is employed upon what is expedient for themselves, there will be many
instead of one, because there is not one and the same faculty employed
on the good of all animals collectively, unless in the same sense as you
may say there is one art of healing with respect to all living beings.

[Sidenote: 1141b] If it is urged that man is superior to all other
animals, that makes no difference: for there are many other things more
Godlike in their nature than Man, as, most obviously, the elements of
which the Universe is composed.

It is plain then that Science is the union of Knowledge and Intuition,
and has for its objects those things which are most precious in their
nature. Accordingly, Anexagoras, Thales, and men of that stamp, people
call Scientific, but not Practically Wise because they see them ignorant
of what concerns themselves; and they say that what they know is quite
out of the common run certainly, and wonderful, and hard, and very fine
no doubt, but still useless because they do not seek after what is good
for them as men.

But Practical Wisdom is employed upon human matters, and such as are
objects of deliberation (for we say, that to deliberate well is most
peculiarly the work of the man who possesses this Wisdom), and no man
deliberates about things which cannot be otherwise than they are, nor
about any save those that have some definite End and this End good
resulting from Moral Action; and the man to whom we should give the name
of Good in Counsel, simply and without modification, is he who in the
way of calculation has a capacity for attaining that of practical goods
which is the best for Man. Nor again does Practical Wisdom consist in
a knowledge of general principles only, but it is necessary that one
should know also the particular details, because it is apt to act, and
action is concerned with details: for which reason sometimes men who
have not much knowledge are more practical than others who have; among
others, they who derive all they know from actual experience: suppose a
man to know, for instance, that light meats are easy of digestion and
wholesome, but not what kinds of meat are light, he will not produce a
healthy state; that man will have a much better chance of doing so,
who knows that the flesh of birds is light and wholesome. Since then
Practical Wisdom is apt to act, one ought to have both kinds of
knowledge, or, if only one, the knowledge of details rather than
of Principles. So there will be in respect of Practical Wisdom the
distinction of supreme and subordinate.

VIII

Further: [Greek: politikhae] and Practical Wisdom are the same mental
state, but the point of view is not the same.

Of Practical Wisdom exerted upon a community that which I would call
the Supreme is the faculty of Legislation; the subordinate, which is
concerned with the details, generally has the common name [Greek:
politikhae], and its functions are Action and Deliberation (for the
particular enactment is a matter of action, being the ultimate issue of
this branch of Practical Wisdom, and therefore people commonly say, that
these men alone are really engaged in government, because they alone
act, filling the same place relatively to legislators, that workmen do
to a master).

Again, that is thought to be Practical Wisdom in the most proper sense
which has for its object the interest of the Individual: and this
usually appropriates the common name: the others are called respectively
Domestic Management, Legislation, Executive Government divided into two
branches, Deliberative and Judicial. Now of course, knowledge for
one's self is one kind of knowledge, but it admits of many shades of
difference: and it is a common notion that the man [Sidenote:1142a] who
knows and busies himself about his own concerns merely is the man of
Practical Wisdom, while they who extend their solicitude to society at
large are considered meddlesome.

Euripides has thus embodied this sentiment; "How," says one of his
Characters, "How foolish am I, who whereas I might have shared equally,
idly numbered among the multitude of the army ... for them that are busy
and meddlesome [Jove hates]," because the generality of mankind seek
their own good and hold that this is their proper business. It is then
from this opinion that the notion has arisen that such men are the
Practically-Wise. And yet it is just possible that the good of the
individual cannot be secured independently of connection with a family
or a community. And again, how a man should manage his own affairs is
sometimes not quite plain, and must be made a matter of inquiry.

A corroboration of what I have said is the fact, that the young come to
be geometricians, and mathematicians, and Scientific in such matters,
but it is not thought that a young man can come to be possessed of
Practical Wisdom: now the reason is, that this Wisdom has for its object
particular facts, which come to be known from experience, which a young
man has not because it is produced only by length of time.

By the way, a person might also inquire why a boy may be made a
mathematician but not Scientific or a natural philosopher. Is not this
the reason? that mathematics are taken in by the process of abstraction,
but the principles of Science and natural philosophy must be gained by
experiment; and the latter young men talk of but do not realise, while
the nature of the former is plain and clear.

Again, in matter of practice, error attaches either to the general rule,
in the process of deliberation, or to the particular fact: for instance,
this would be a general rule, "All water of a certain gravity is bad;"
the particular fact, "this water is of that gravity."

And that Practical Wisdom is not knowledge is plain, for it has to do
with the ultimate issue, as has been said, because every object of
action is of this nature.

To Intuition it is opposed, for this takes in those principles which
cannot be proved by reasoning, while Practical Wisdom is concerned with
the ultimate particular fact which cannot be realised by Knowledge but
by Sense; I do not mean one of the five senses, but the same by which
we take in the mathematical fact, that no rectilineal figure can be
contained by less than three lines, i.e. that a triangle is the ultimate
figure, because here also is a stopping point.

This however is Sense rather than Practical Wisdom, which is of another
kind.

IX

Now the acts of inquiring and deliberating differ, though deliberating
is a kind of inquiring. We ought to ascertain about Good Counsel
likewise what it is, whether a kind of Knowledge, or Opinion, or Happy
Conjecture, or some other kind of faculty. Knowledge it obviously is
not, because men do not inquire about what they know, and Good Counsel
is a kind of deliberation, and the man who is deliberating is inquiring
and calculating. [Sidenote:1142b]

Neither is it Happy Conjecture; because this is independent of
reasoning, and a rapid operation; but men deliberate a long time, and
it is a common saying that one should execute speedily what has been
resolved upon in deliberation, but deliberate slowly.

Quick perception of causes again is a different faculty from good
counsel, for it is a species of Happy Conjecture. Nor is Good Counsel
Opinion of any kind.

Well then, since he who deliberates ill goes wrong, and he who
deliberates well does so rightly, it is clear that Good Counsel is
rightness of some kind, but not of Knowledge nor of Opinion: for
Knowledge cannot be called right because it cannot be wrong, and
Rightness of Opinion is Truth: and again, all which is the object of
opinion is definitely marked out.

Still, however, Good Counsel is not independent of Reason, Does it
remain then that it is a rightness of Intellectual Operation simply,
because this does not amount to an assertion; and the objection to
Opinion was that it is not a process of inquiry but already a definite
assertion; whereas whosoever deliberates, whether well or ill, is
engaged in inquiry and calculation.

Well, Good Counsel is a Rightness of deliberation, and so the first
question must regard the nature and objects of deliberation. Now
remember Rightness is an equivocal term; we plainly do not mean
Rightness of any kind whatever; the [Greek: akrataes], for instance, or
the bad man, will obtain by his calculation what he sets before him as
an object, and so he may be said to have deliberated _rightly_ in one
sense, but will have attained a great evil. Whereas to have deliberated
well is thought to be a good, because Good Counsel is Rightness of
deliberation of such a nature as is apt to attain good.

But even this again you may get by false reasoning, and hit upon the
right effect though not through right means, your middle term being
fallacious: and so neither will this be yet Good Counsel in consequence
of which you get what you ought but not through proper means.

Again, one man may hit on a thing after long deliberation, another
quickly. And so that before described will not be yet Good Counsel, but
the Rightness must be with reference to what is expedient; and you must
have a proper end in view, pursue it in a right manner and right time.

Once more. One may deliberate well either generally or towards some
particular End. Good counsel in the general then is that which goes
right towards that which is the End in a general way of consideration;
in particular, that which does so towards some particular End.

Since then deliberating well is a quality of men possessed of Practical
Wisdom, Good Counsel must be "Rightness in respect of what conduces to a
given End, of which Practical Wisdom is the true conception." [Sidenote:
X 1143_a_] There is too the faculty of Judiciousness, and also its
absence, in virtue of which we call men Judicious or the contrary.

Now Judiciousness is neither entirely identical with Knowledge or
Opinion (for then all would have been Judicious), nor is it any one
specific science, as medical science whose object matter is things
wholesome; or geometry whose object matter is magnitude: for it has not
for its object things which always exist and are immutable, nor of those
things which come into being just any which may chance; but those in
respect of which a man might doubt and deliberate.

And so it has the same object matter as Practical Wisdom; yet the two
faculties are not identical, because Practical Wisdom has the capacity
for commanding and taking the initiative, for its End is "what one
should do or not do:" but Judiciousness is only apt to decide upon
suggestions (though we do in Greek put "well" on to the faculty and its
concrete noun, these really mean exactly the same as the plain words),
and Judiciousness is neither the having Practical Wisdom, nor attaining
it: but just as learning is termed [Greek: sunievai] when a man uses
his knowledge, so judiciousness consists in employing the Opinionative
faculty in judging concerning those things which come within the
province of Practical Wisdom, when another enunciates them; and not
judging merely, but judging well (for [Greek: eu] and [Greek: kalos]
mean exactly the same thing). And the Greek name of this faculty is
derived from the use of the term [Greek: suvievai] in learning: [Greek:
mavthaveiv] and [Greek: suvievai] being often used as synonymous.

[Sidenote: XI] The faculty called [Greek: gvomh], in right of which we
call men [Greek: euyvomoves], or say they have [Greek: gvomh], is "the
right judgment of the equitable man." A proof of which is that we most
commonly say that the equitable man has a tendency to make allowance,
and the making allowance in certain cases is equitable. And [Greek:
sungvomae] (the word denoting allowance) is right [Greek: gvomh] having
a capacity of making equitable decisions, By "right" I mean that which
attains the True. Now all these mental states tend to the same object,
as indeed common language leads us to expect: I mean, we speak of
[Greek: gnomae], Judiciousness, Practical Wisdom, and Practical
Intuition, attributing the possession of [Greek: gnomae] and Practical
Intuition to the same Individuals whom we denominate Practically-Wise
and Judicious: because all these faculties are employed upon the
extremes, i.e. on particular details; and in right of his aptitude
for deciding on the matters which come within the province of the
Practically-Wise, a man is Judicious and possessed of good [Greek:
gnomae]; i.e. he is disposed to make allowance, for considerations of
equity are entertained by all good men alike in transactions with their
fellows.

And all matters of Moral Action belong to the class of particulars,
otherwise called extremes: for the man of Practical Wisdom must know
them, and Judiciousness and [Greek: gnomae] are concerned with matters
of Moral Actions, which are extremes.

[Sidenote:1143b] Intuition, moreover, takes in the extremes at both
ends: I mean, the first and last terms must be taken in not by reasoning
but by Intuition [so that Intuition comes to be of two kinds], and that
which belongs to strict demonstrative reasonings takes in immutable,
i.e. Necessary, first terms; while that which is employed in practical
matters takes in the extreme, the Contingent, and the minor Premiss: for
the minor Premisses are the source of the Final Cause, Universals being
made up out of Particulars. To take in these, of course, we must have
Sense, i.e. in other words Practical Intuition. And for this reason
these are thought to be simply gifts of nature; and whereas no man is
thought to be Scientific by nature, men are thought to have [Greek:
gnomae], and Judiciousness, and Practical Intuition: a proof of which is
that we think these faculties are a consequence even of particular ages,
and this given age has Practical Intuition and [Greek: gnomae], we say,
as if under the notion that nature is the cause. And thus Intuition is
both the beginning and end, because the proofs are based upon the one
kind of extremes and concern the other.

And so one should attend to the undemonstrable dicta and opinions of the
skilful, the old and the Practically-Wise, no less than to those which
are based on strict reasoning, because they see aright, having gained
their power of moral vision from experience.

XII

Well, we have now stated the nature and objects of Practical Wisdom and
Science respectively, and that they belong each to a different part
of the Soul. But I can conceive a person questioning their utility.
"Science," he would say, "concerns itself with none of the causes of
human happiness (for it has nothing to do with producing anything):
Practical Wisdom has this recommendation, I grant, but where is the need
of it, since its province is those things which are just and honourable,
and good for man, and these are the things which the good man as such
does; but we are not a bit the more apt to do them because we know them,
since the Moral Virtues are Habits; just as we are not more apt to be
healthy or in good condition from mere knowledge of what relates to
these (I mean, of course, things so called not from their producing
health, etc., but from their evidencing it in a particular subject),
for we are not more apt to be healthy and in good condition merely from
knowing the art of medicine or training.

"If it be urged that _knowing what is_ good does not by itself make a
Practically-Wise man but _becoming_ good; still this Wisdom will be no
use either to those that are good, and so have it already, or to those
who have it not; because it will make no difference to them whether they
have it themselves or put themselves under the guidance of others who
have; and we might be contented to be in respect of this as in respect
of health: for though we wish to be healthy still we do not set about
learning the art of healing.

"Furthermore, it would seem to be strange that, though lower in the
scale than Science, it is to be its master; which it is, because
whatever produces results takes the rule and directs in each matter."

This then is what we are to talk about, for these are the only points
now raised.

[Sidenote:1144a] Now first we say that being respectively Excellences
of different parts of the Soul they must be choiceworthy, even on the
supposition that they neither of them produce results.

In the next place we say that they _do_ produce results; that Science
makes Happiness, not as the medical art but as healthiness makes health:
because, being a part of Virtue in its most extensive sense, it makes a
man happy by being possessed and by working.

Next, Man's work _as Man_ is accomplished by virtue of Practical Wisdom
and Moral Virtue, the latter giving the right aim and direction, the
former the right means to its attainment; but of the fourth part of the
Soul, the mere nutritive principle, there is no such Excellence, because
nothing is in its power to do or leave undone.

As to our not being more apt to do what is noble and just by reason of
possessing Practical Wisdom, we must begin a little higher up, taking
this for our starting-point. As we say that men may do things in
themselves just and yet not be just men; for instance, when men do what
the laws require of them, either against their will, or by reason of
ignorance or something else, at all events not for the sake of the
things themselves; and yet they do what they ought and all that the good
man should do; so it seems that to be a good man one must do each act in
a particular frame of mind, I mean from Moral Choice and for the sake of
the things themselves which are done. Now it is Virtue which makes the
Moral Choice right, but whatever is naturally required to carry out
that Choice comes under the province not of Virtue but of a different
faculty. We must halt, as it were, awhile, and speak more clearly on
these points.

There is then a certain faculty, commonly named Cleverness, of such a
nature as to be able to do and attain whatever conduces to _any_ given
purpose: now if that purpose be a good one the faculty is praiseworthy;
if otherwise, it goes by a name which, denoting strictly the ability,
implies the willingness to do _anything_; we accordingly call the
Practically-Wise Clever, and also those who can and will do anything.

Now Practical Wisdom is not identical with Cleverness, nor is it without
this power of adapting means to ends: but this Eye of the Soul (as we
may call it) does not attain its proper state without goodness, as we
have said before and as is quite plain, because the syllogisms into
which Moral Action may be analysed have for their Major Premiss, "since
----------is the End and the Chief Good" (fill up the blank with just
anything you please, for we merely want to exhibit the Form, so that
anything will do), but _how_ this blank should be filled is seen only by
the good man: because Vice distorts the moral vision and causes men to
be deceived in respect of practical principles.

It is clear, therefore, that a man cannot be a Practically-Wise,
without being a good, man.

XIII

[Sidenote:1144b] We must inquire again also about Virtue: for it may be
divided into Natural Virtue and Matured, which two bear to each other a
relation similar to that which Practical Wisdom bears to Cleverness, one
not of identity but resemblance. I speak of Natural Virtue, because men
hold that each of the moral dispositions attach to us all somehow by
nature: we have dispositions towards justice, self-mastery and courage,
for instance, immediately from our birth: but still we seek Goodness
in its highest sense as something distinct from these, and that these
dispositions should attach to us in a somewhat different fashion.
Children and brutes have these natural states, but then they are plainly
hurtful unless combined with an intellectual element: at least thus much
is matter of actual experience and observation, that as a strong body
destitute of sight must, if set in motion, fall violently because it has
not sight, so it is also in the case we are considering: but if it can
get the intellectual element it then excels in acting. Just so the
Natural State of Virtue, being like this strong body, will then
be Virtue in the highest sense when it too is combined with the
intellectual element.

So that, as in the case of the Opinionative faculty, there are two
forms, Cleverness and Practical Wisdom; so also in the case of the Moral
there are two, Natural Virtue and Matured; and of these the latter
cannot be formed without Practical Wisdom.

This leads some to say that all the Virtues are merely intellectual
Practical Wisdom, and Socrates was partly right in his inquiry and
partly wrong: wrong in that he thought all the Virtues were merely
intellectual Practical Wisdom, right in saying they were not independent
of that faculty.

A proof of which is that now all, in defining Virtue, add on the "state"
[mentioning also to what standard it has reference, namely that] "which
is accordant with Right Reason:" now "right" means in accordance with
Practical Wisdom. So then all seem to have an instinctive notion that
that state which is in accordance with Practical Wisdom is Virtue;
however, we must make a slight change in their statement, because that
state is Virtue, not merely which is in accordance with but which
implies the possession of Right Reason; which, upon such matters, is
Practical Wisdom. The difference between us and Socrates is this: he
thought the Virtues were reasoning processes (_i.e._ that they were all
instances of Knowledge in its strict sense), but we say they imply the
possession of Reason.

From what has been said then it is clear that one cannot be, strictly
speaking, good without Practical Wisdom nor Practically-Wise without
moral goodness.

And by the distinction between Natural and Matured Virtue one can
meet the reasoning by which it might be argued "that the Virtues are
separable because the same man is not by nature most inclined to all at
once so that he will have acquired this one before he has that other:"
we would reply that this is possible with respect to the Natural Virtues
but not with respect to those in right of which a man is denominated
simply good: because they will all belong to him together with the one
faculty of Practical Wisdom. [Sidenote:1145a]

It is plain too that even had it not been apt to act we should have
needed it, because it is the Excellence of a part of the Soul; and that
the moral choice cannot be right independently of Practical Wisdom and
Moral Goodness; because this gives the right End, that causes the doing
these things which conduce to the End.

Then again, it is not Master of Science (i.e. of the superior part of
the Soul), just as neither is the healing art Master of health; for it
does not make use of it, but looks how it may come to be: so it commands
for the sake of it but does not command it.

The objection is, in fact, about as valid as if a man should say
[Greek: politikae] governs the gods because it gives orders about all
things in the communty.


APPENDIX

On [Greek: epistaemae], from I. Post. Analyt. chap. i. and ii.

(Such parts only are translated as throw light on the Ethics.)

All teaching, and all intellectual learning, proceeds on the basis
of previous knowledge, as will appear on an examination of all. The
Mathematical Sciences, and every other system, draw their conclusions in
this method. So too of reasonings, whether by syllogism, or induction:
for both teach through what is previously known, the former assuming
the premisses as from wise men, the latter proving universals from
the evidentness of the particulars. In like manner too rhetoricians
persuade, either through examples (which amounts to induction), or
through enthymemes (which amounts to syllogism).

Well, we suppose that we _know_ things (in the strict and proper sense
of the word) when we suppose ourselves to know the cause by reason
of which the thing is to be the cause of it; and that this cannot be
otherwise. It is plain that the idea intended to be conveyed by the term
_knowing_ is something of this kind; because they who do not really know
suppose themselves thus related to the matter in hand and they who
do know really are so that of whatsoever there is properly speaking
Knowledge this cannot be otherwise than it is Whether or no there is
another way of knowing we will say afterwards, but we do say that we
know through demonstration, by which I mean a syllogism apt to produce
Knowledge, i.e. in right of which through having it, we know.

If Knowledge then is such as we have described it, the Knowledge
produced by demonstrative reasoning must be drawn from premisses _true_
and _first_, and _incapable of syllogistic proof_, and _better known_,
and _prior in order of time_, and _causes of the conclusion_, for so the
principles will be akin to the conclusion demonstrated.

(Syllogism, of course there may be without such premisses, but it will
not be demonstration because it will not produce knowledge).

_True_, they must be, because it is impossible to know that which is not.

_First_, that is indemonstrable, because, if demonstrable, he cannot be
said to _know_ them who has no demonstration of them for knowing such
things as are demonstrable is the same as having demonstration of them.

_Causes_ they must be, and _better known_, and _prior_ in time,
_causes_, because we then know when we are acquainted with the cause,
and _prior_, if causes, and _known beforehand_, not merely comprehended
in idea but known to exist (The terms prior, and better known, bear two
senses for _prior by nature_ and _prior relatively to ourselves_ are not
the same, nor _better known by nature_, and _better known to us_ I mean,
by _prior_ and _better known relatively to ourselves_, such things as
are nearer to sensation, but abstractedly so such as are further
Those are furthest which are most universal those nearest which are
particulars, and these are mutually opposed) And by _first_, I mean
_principles akin to the conclusion_, for principle means the same as
first And the principle or first step in demonstration is a proposition
incapable of syllogistic proof, i. e. one to which there is none prior.
Now of such syllogistic principles I call that a [Greek: thxsis] which
you cannot demonstrate, and which is unnecessary with a view to learning
something else. That which is necessary in order to learn something else
is an Axiom.

Further, since one is to believe and know the thing by having a
syllogism of the kind called demonstration, and what constitutes it to
be such is the nature of the premisses, it is necessary not merely to
_know before_, but to _know better than the conclusion_, either all or
at least some of, the principles, because that which is the cause of a
quality inhering in something else always inheres itself more as the
cause of our loving is itself more lovable. So, since the principles are
the cause of our knowing and behoving we know and believe them more,
because by reason of them we know also the conclusion following.

Further: the man who is to have the Knowledge which comes through
demonstration must not merely know and believe his principles better
than he does his conclusion, but he must believe nothing more firmly
than the contradictories of those principles out of which the contrary
fallacy may be constructed: since he who _knows_, is to be simply and
absolutely infallible.




BOOK VII



I

Next we must take a different point to start from, and observe that of
what is to be avoided in respect of moral character there are three
forms; Vice, Imperfect Self-Control, and Brutishness. Of the two former
it is plain what the contraries are, for we call the one Virtue, the
other Self-Control; and as answering to Brutishness it will be most
suitable to assign Superhuman, i.e. heroical and godlike Virtue, as, in
Homer, Priam says of Hector "that he was very excellent, nor was he like
the offspring of mortal man, but of a god." and so, if, as is commonly
said, men are raised to the position of gods by reason of very high
excellence in Virtue, the state opposed to the Brutish will plainly be
of this nature: because as brutes are not virtuous or vicious so neither
are gods; but the state of these is something more precious than Virtue,
of the former something different in kind from Vice.

And as, on the one hand, it is a rare thing for a man to be godlike (a
term the Lacedaemonians are accustomed to use when they admire a man
exceedingly; [Greek:seios anhæp] they call him), so the brutish man is
rare; the character is found most among barbarians, and some cases of it
are caused by disease or maiming; also such men as exceed in vice all
ordinary measures we therefore designate by this opprobrious term. Well,
we must in a subsequent place make some mention of this disposition,
and Vice has been spoken of before: for the present we must speak of
Imperfect Self-Control and its kindred faults of Softness and Luxury, on
the one hand, and of Self-Control and Endurance on the other; since we
are to conceive of them, not as being the same states exactly as Virtue
and Vice respectively, nor again as differing in kind. [Sidenote:1145b]
And we should adopt the same course as before, i.e. state the phenomena,
and, after raising and discussing difficulties which suggest themselves,
then exhibit, if possible, all the opinions afloat respecting these
affections of the moral character; or, if not all, the greater part and
the most important: for we may consider we have illustrated the matter
sufficiently when the difficulties have been solved, and such theories
as are most approved are left as a residuum.

The chief points may be thus enumerated. It is thought,

I. That Self-Control and Endurance belong to the class of things good
and praiseworthy, while Imperfect Self-Control and Softness belong to
that of things low and blameworthy.

II. That the man of Self-Control is identical with the man who is apt to
abide by his resolution, and the man of Imperfect Self-Control with him
who is apt to depart from his resolution.

III. That the man of Imperfect Self-Control does things at the
instigation of his passions, knowing them to be wrong, while the man of
Self-Control, knowing his lusts to be wrong, refuses, by the influence
of reason, to follow their suggestions.

IV. That the man of Perfected Self-Mastery unites the qualities of
Self-Control and Endurance, and some say that every one who unites these
is a man of Perfect Self-Mastery, others do not.

V. Some confound the two characters of the man who has _no_
Self-Control, and the man of _Imperfect Self-Control_, while others
distinguish between them.

VI. It is sometimes said that the man of Practical Wisdom cannot be a
man of Imperfect Self-Control, sometimes that men who are Practically
Wise and Clever are of Imperfect Self-Control.

VII. Again, men are said to be of Imperfect Self-Control, not simply
but with the addition of the thing wherein, as in respect of anger, of
honour, and gain.

These then are pretty well the common statements.

II

Now a man may raise a question as to the nature of the right conception
in violation of which a man fails of Self-Control.

That he can so fail when _knowing_ in the strict sense what is right
some say is impossible: for it is a strange thing, as Socrates thought,
that while Knowledge is present in his mind something else should
master him and drag him about like a slave. Socrates in fact contended
generally against the theory, maintaining there is no such state as that
of Imperfect Self-Control, for that no one acts contrary to what is best
conceiving it to be best but by reason of ignorance what is best.

With all due respect to Socrates, his account of the matter is at
variance with plain facts, and we must inquire with respect to the
affection, if it be caused by ignorance what is the nature of the
ignorance: for that the man so failing does not suppose his acts to be
right before he is under the influence of passion is quite plain.

There are people who partly agree with Socrates and partly not: that
nothing can be stronger than Knowledge they agree, but that no man acts
in contravention of his conviction of what is better they do not agree;
and so they say that it is not Knowledge, but only Opinion, which the
man in question has and yet yields to the instigation of his pleasures.

[Sidenote:1146a] But then, if it is Opinion and not Knowledge, that is
it the opposing conception be not strong but only mild (as in the case
of real doubt), the not abiding by it in the face of strong lusts would
be excusable: but wickedness is not excusable, nor is anything which
deserves blame.

Well then, is it Practical Wisdom which in this case offers opposition:
for that is the strongest principle? The supposition is absurd, for
we shall have the same man uniting Practical Wisdom and Imperfect
Self-Control, and surely no single person would maintain that it is
consistent with the character of Practical Wisdom to do voluntarily what
is very wrong; and besides we have shown before that the very mark of
a man of this character is aptitude to act, as distinguished from
mere knowledge of what is right; because he is a man conversant with
particular details, and possessed of all the other virtues.

Again, if the having strong and bad lusts is necessary to the idea of
the man of Self-Control, this character cannot be identical with the man
of Perfected Self-Mastery, because the having strong desires or bad ones
does not enter into the idea of this latter character: and yet the man
of Self-Control must have such: for suppose them good; then the moral
state which should hinder a man from following their suggestions must be
bad, and so Self-Control would not be in all cases good: suppose them on
the other hand to be weak and not wrong, it would be nothing grand; nor
anything great, supposing them to be wrong and weak.

Again, if Self-Control makes a man apt to abide by all opinions without
exception, it may be bad, as suppose the case of a false opinion: and
if Imperfect Self-Control makes a man apt to depart from all without
exception, we shall have cases where it will be good; take that of
Neoptolemus in the Philoctetes of Sophocles, for instance: he is to be
praised for not abiding by what he was persuaded to by Ulysses, because
he was pained at being guilty of falsehood.

Or again, false sophistical reasoning presents a difficulty: for because
men wish to prove paradoxes that they may be counted clever when they
succeed, the reasoning that has been used becomes a difficulty: for the
intellect is fettered; a man being unwilling to abide by the conclusion
because it does not please his judgment, but unable to advance because
he cannot disentangle the web of sophistical reasoning.

Or again, it is conceivable on this supposition that folly joined with
Imperfect Self-Control may turn out, in a given case, goodness: for by
reason of his imperfection of self-control a man acts in a way which
contradicts his notions; now his notion is that what is really good is
bad and ought not to be done; and so he will eventually do what is good
and not what is bad.

Again, on the same supposition, the man who acting on conviction pursues
and chooses things because they are pleasant must be thought a better
man than he who does so not by reason of a quasi-rational conviction but
of Imperfect Self-Control: because he is more open to cure by reason of
the possibility of his receiving a contrary conviction. But to the man
of Imperfect Self-Control would apply the proverb, "when water chokes,
what should a man drink then?" for had he never been convinced at all
in respect of [Sidenote: 1146b] what he does, then by a conviction in a
contrary direction he might have stopped in his course; but now though
he has had convictions he notwithstanding acts against them.

Again, if any and every thing is the object-matter of Imperfect and
Perfect Self-Control, who is the man of Imperfect Self-Control simply?
because no one unites all cases of it, and we commonly say that some men
are so simply, not adding any particular thing in which they are so.

Well, the difficulties raised are pretty near such as I have described
them, and of these theories we must remove some and leave others as
established; because the solving of a difficulty is a positive act of
establishing something as true.

III

Now we must examine first whether men of Imperfect Self-Control act with
a knowledge of what is right or not: next, if with such knowledge, in
what sense; and next what are we to assume is the object-matter of the
man of Imperfect Self-Control, and of the man of Self-Control; I mean,
whether pleasure and pain of all kinds or certain definite ones; and as
to Self-Control and Endurance, whether these are designations of the
same character or different. And in like manner we must go into all
questions which are connected with the present.

But the real starting point of the inquiry is, whether the two
characters of Self-Control and Imperfect Self-Control are distinguished
by their object-matter, or their respective relations to it. I mean,
whether the man of Imperfect Self-Control is such simply by virtue of
having such and such object-matter; or not, but by virtue of his being
related to it in such and such a way, or by virtue of both: next,
whether Self-Control and Imperfect Self-Control are unlimited in their
object-matter: because he who is designated without any addition a man
of Imperfect Self-Control is not unlimited in his object-matter, but has
exactly the same as the man who has lost all Self-Control: nor is he so
designated because of his relation to this object-matter merely (for
then his character would be identical with that just mentioned, loss
of all Self-Control), but because of his relation to it being such
and such. For the man who has lost all Self-Control is led on with
deliberate moral choice, holding that it is his line to pursue pleasure
as it rises: while the man of Imperfect Self-Control does not think that
he ought to pursue it, but does pursue it all the same.

Now as to the notion that it is True Opinion and not Knowledge in
contravention of which men fail in Self-Control, it makes no difference
to the point in question, because some of those who hold Opinions have
no doubt about them but suppose themselves to have accurate Knowledge;
if then it is urged that men holding Opinions will be more likely than
men who have Knowledge to act in contravention of their conceptions,
as having but a moderate belief in them; we reply, Knowledge will not
differ in this respect from Opinion: because some men believe their
own Opinions no less firmly than others do their positive Knowledge:
Heraclitus is a case in point.

Rather the following is the account of it: the term _knowing_ has two
senses; both the man who does not use his Knowledge, and he who does,
are said to _know_: there will be a difference between a man's acting
wrongly, who though possessed of Knowledge does not call it into
operation, and his doing so who has it and actually exercises it: the
latter is a strange case, but the mere having, if not exercising,
presents no anomaly.

[Sidenote:1147a] Again, as there are two kinds of propositions affecting
action, universal and particular, there is no reason why a man may not
act against his Knowledge, having both propositions in his mind, using
the universal but not the particular, for the particulars are the
objects of moral action.

There is a difference also in universal propositions; a universal
proposition may relate partly to a man's self and partly to the thing in
question: take the following for instance; "dry food is good for every
man," this may have the two minor premisses, "this is a man," and "so
and so is dry food;" but whether a given substance is so and so a man
either has not the Knowledge or does not exert it. According to these
different senses there will be an immense difference, so that for a
man to _know_ in the one sense, and yet act wrongly, would be nothing
strange, but in any of the other senses it would be a matter for wonder.

Again, men may have Knowledge in a way different from any of those which
have been now stated: for we constantly see a man's state so differing
by having and not using Knowledge, that he has it in a sense and also
has not; when a man is asleep, for instance, or mad, or drunk: well, men
under the actual operation of passion are in exactly similar conditions;
for anger, lust, and some other such-like things, manifestly make
changes even in the body, and in some they even cause madness; it is
plain then that we must say the men of Imperfect Self-Control are in a
state similar to these.

And their saying what embodies Knowledge is no proof of their actually
then exercising it, because they who are under the operation of these
passions repeat demonstrations; or verses of Empedocles, just as
children, when first learning, string words together, but as yet know
nothing of their meaning, because they must grow into it, and this is a
process requiring time: so that we must suppose these men who fail in
Self-Control to say these moral sayings just as actors do. Furthermore,
a man may look at the account of the phænomenon in the following way,
from an examination of the actual working of the mind: All action may
be analysed into a syllogism, in which the one premiss is an universal
maxim and the other concerns particulars of which Sense [moral or
physical, as the case may be] is cognisant: now when one results from
these two, it follows necessarily that, as far as theory goes the mind
must assert the conclusion, and in practical propositions the man must
act accordingly. For instance, let the universal be, "All that is
sweet should be tasted," the particular, "This is sweet;" it follows
necessarily that he who is able and is not hindered should not only
draw, but put in practice, the conclusion "This is to be tasted." When
then there is in the mind one universal proposition forbidding to taste,
and the other "All that is sweet is pleasant" with its minor "This is
sweet" (which is the one that really works), and desire happens to be in
the man, the first universal bids him avoid this but the desire leads
him on to taste; for it has the power of moving the various organs:
and so it results that he fails in Self-Control, [Sidenote:1147b] in a
certain sense under the influence of Reason and Opinion not contrary in
itself to Reason but only accidentally so; because it is the desire that
is contrary to Right Reason, but not the Opinion: and so for this reason
brutes are not accounted of Imperfect Self-Control, because they have
no power of conceiving universals but only of receiving and retaining
particular impressions.

As to the manner in which the ignorance is removed and the man of
Imperfect Self-Control recovers his Knowledge, the account is the same
as with respect to him who is drunk or asleep, and is not peculiar to
this affection, so physiologists are the right people to apply to. But
whereas the minor premiss of every practical syllogism is an opinion on
matter cognisable by Sense and determines the actions; he who is under
the influence of passion either has not this, or so has it that his
having does not amount to _knowing_ but merely saying, as a man when
drunk might repeat Empedocles' verses; and because the minor term
is neither universal, nor is thought to have the power of producing
Knowledge in like manner as the universal term: and so the result which
Socrates was seeking comes out, that is to say, the affection does not
take place in the presence of that which is thought to be specially
and properly Knowledge, nor is this dragged about by reason of the
affection, but in the presence of that Knowledge which is conveyed by
Sense.

Let this account then be accepted of the question respecting the failure
in Self-Control, whether it is with Knowledge or not; and, if with
knowledge, with what kind of knowledge such failure is possible.

IV

The next question to be discussed is whether there is a character to be
designated by the term "of Imperfect Self-Control" simply, or whether
all who are so are to be accounted such, in respect of some particular
thing; and, if there is such a character, what is his object-matter.

Now that pleasures and pains are the object-matter of men of
Self-Control and of Endurance, and also of men of Imperfect Self-Control
and Softness, is plain.

Further, things which produce pleasure are either necessary, or objects
of choice in themselves but yet admitting of excess. All bodily things
which produce pleasure are necessary; and I call such those which relate
to food and other grosser appetities, in short such bodily things as
we assumed were the Object-matter of absence of Self-Control and of
Perfected Self-Mastery.

The other class of objects are not necessary, but objects of choice in
themselves: I mean, for instance, victory, honour, wealth, and such-like
good or pleasant things. And those who are excessive in their liking for
such things contrary to the principle of Right Reason which is in their
own breasts we do not designate men of Imperfect Self-Control simply,
but with the addition of the thing wherein, as in respect of money, or
gain, or honour, or anger, and not simply; because we consider them as
different characters and only having that title in right of a kind of
resemblance (as when we add to a man's name "conqueror in the Olympic
games" the account of him as Man differs but little from the account
of him as the Man who conquered in the Olympic games, but still it is
different). And a proof of the real [Sidenote: 1148a] difference between
these so designated with an addition and those simply so called is this,
that Imperfect Self-Control is blamed, not as an error merely but also
as being a vice, either wholly or partially; but none of these other
cases is so blamed.

But of those who have for their object-matter the bodily enjoyments,
which we say are also the object-matter of the man of Perfected
Self-Mastery and the man who has lost all Self-Control, he that pursues
excessive pleasures and too much avoids things which are painful (as
hunger and thirst, heat and cold, and everything connected with touch
and taste), not from moral choice but in spite of his moral choice and
intellectual conviction, is termed "a man of Imperfect Self-Control,"
not with the addition of any particular object-matter as we do in
respect of want of control of anger but simply.

And a proof that the term is thus applied is that the kindred term
"Soft" is used in respect of these enjoyments but not in respect of any
of those others. And for this reason we put into the same rank the man
of Imperfect Self-Control, the man who has lost it entirely, the man
who has it, and the man of Perfected Self-Mastery; but not any of those
other characters, because the former have for their object-matter the
same pleasures and pains: but though they have the same object-matter,
they are not related to it in the same way, but two of them act upon
moral choice, two without it. And so we should say that man is more
entirely given up to his passions who pursues excessive pleasures, and
avoids moderate pains, being either not at all, or at least but little,
urged by desire, than the man who does so because his desire is very
strong: because we think what would the former be likely to do if he had
the additional stimulus of youthful lust and violent pain consequent on
the want of those pleasures which we have denominated necessary?

Well then, since of desires and pleasures there are some which are in
kind honourable and good (because things pleasant are divisible, as we
said before, into such as are naturally objects of choice, such as
are naturally objects of avoidance, and such as are in themselves
indifferent, money, gain, honour, victory, for instance); in respect of
all such and those that are indifferent, men are blamed not merely for
being affected by or desiring or liking them, but for exceeding in any
way in these feelings.

And so they are blamed, whosoever in spite of Reason are mastered by,
that is pursue, any object, though in its nature noble and good; they,
for instance, who are more earnest than they should be respecting
honour, or their children or parents; not but what these are good
objects and men are praised for being earnest about them: but still they
admit of excess; for instance, if any one, as Niobe did, should fight
even against the gods, or feel towards his father as Satyrus, who got
therefrom the nickname of [Greek: philophator], [Sidenote: 1148b]
because he was thought to be very foolish.

Now depravity there is none in regard of these things, for the reason
assigned above, that each of them in itself is a thing naturally
choiceworthy, yet the excesses in respect of them are wrong and matter
for blame: and similarly there is no Imperfect Self-Control in respect
of these things; that being not merely a thing that should be avoided
but blameworthy.

But because of the resemblance of the affection to the Imperfection of
Self-Control the term is used with the addition in each case of the
particular object-matter, just as men call a man a bad physician, or bad
actor, whom they would not think of calling simply bad. As then in these
cases we do not apply the term simply because each of the states is not
a vice, but only like a vice in the way of analogy, so it is plain that
in respect of Imperfect Self-Control and Self-Control we must limit the
names to those states which have the same object-matter as Perfected
Self-Mastery and utter loss of Self-Control, and that we do apply it to
the case of anger only in the way of resemblance: for which reason, with
an addition, we designate a man of Imperfect Self-Control in respect of
anger, as of honour or of gain.

V

As there are some things naturally pleasant, and of these two kinds;
those, namely, which are pleasant generally, and those which are so
relatively to particular kinds of animals and men; so there are others
which are not naturally pleasant but which come to be so in consequence
either of maimings, or custom, or depraved natural tastes: and one may
observe moral states similar to those we have been speaking of, having
respectively these classes of things for their object-matter.

I mean the Brutish, as in the case of the female who, they say, would
rip up women with child and eat the foetus; or the tastes which are
found among the savage tribes bordering on the Pontus, some liking raw
flesh, and some being cannibals, and some lending one another their
children to make feasts of; or what is said of Phalaris. These are
instances of Brutish states, caused in some by disease or madness; take,
for instance, the man who sacrificed and ate his mother, or him who
devoured the liver of his fellow-servant. Instances again of those
caused by disease or by custom, would be, plucking out of hair, or
eating one's nails, or eating coals and earth. ... Now wherever nature
is really the cause no one would think of calling men of Imperfect
Self-Control, ... nor, in like manner, such as are in a diseased state
through custom.

[Sidenote:1149a] Obviously the having any of these inclinations is
something foreign to what is denominated Vice, just as Brutishness is:
and when a man has them his mastering them is not properly Self-Control,
nor his being mastered by them Imperfection of Self-Control in the
proper sense, but only in the way of resemblance; just as we may say a
man of ungovernable wrath fails of Self-Control in respect of anger but
not simply fails of Self-Control. For all excessive folly, cowardice,
absence of Self-Control, or irritability, are either Brutish or morbid.
The man, for instance, who is naturally afraid of all things, even if
a mouse should stir, is cowardly after a Brutish sort; there was a man
again who, by reason of disease, was afraid of a cat: and of the fools,
they who are naturally destitute of Reason and live only by Sense are
Brutish, as are some tribes of the far-off barbarians, while others
who are so by reason of diseases, epileptic or frantic, are in morbid
states.

So then, of these inclinations, a man may sometimes merely have one
without yielding to it: I mean, suppose that Phalaris had restrained his
unnatural desire to eat a child: or he may both have and yield to it. As
then Vice when such as belongs to human nature is called Vice simply,
while the other is so called with the addition of "brutish" or "morbid,"
but not simply Vice, so manifestly there is Brutish and Morbid
Imperfection of Self-Control, but that alone is entitled to the name
without any qualification which is of the nature of utter absence of
Self-Control, as it is found in Man.

VI

It is plain then that the object-matter of Imperfect Self-Control and
Self-Control is restricted to the same as that of utter absence of
Self-Control and that of Perfected Self-Mastery, and that the rest is
the object-matter of a different species so named metaphorically and not
simply: we will now examine the position, "that Imperfect Self-Control
in respect of Anger is less disgraceful than that in respect of Lusts."

In the first place, it seems that Anger does in a way listen to Reason
but mishears it; as quick servants who run out before they have heard
the whole of what is said and then mistake the order; dogs, again, bark
at the slightest stir, before they have seen whether it be friend
or foe; just so Anger, by reason of its natural heat and quickness,
listening to Reason, but without having heard the command of Reason,
rushes to its revenge. That is to say, Reason or some impression on the
mind shows there is insolence or contempt in the offender, and then
Anger, reasoning as it were that one ought to fight against what is
such, fires up immediately: whereas Lust, if Reason or Sense, as the
case may be, merely says a thing is sweet, rushes to the enjoyment of
it: and so Anger follows Reason in a manner, but Lust does not and is
therefore more disgraceful: because he that cannot control his anger
yields in a manner to Reason, but the other to his Lust and not to
Reason at all. [Sidenote:1149b]

Again, a man is more excusable for following such desires as are
natural, just as he is for following such Lusts as are common to all and
to that degree in which they are common. Now Anger and irritability are
more natural than Lusts when in excess and for objects not necessary.
(This was the ground of the defence the man made who beat his father,
"My father," he said, "used to beat his, and his father his again, and
this little fellow here," pointing to his child, "will beat me when he
is grown a man: it runs in the family." And the father, as he was being
dragged along, bid his son leave off beating him at the door, because he
had himself been used to drag his father so far and no farther.)

Again, characters are less unjust in proportion as they involve less
insidiousness. Now the Angry man is not insidious, nor is Anger, but
quite open: but Lust is: as they say of Venus,

  "Cyprus-born Goddess, _weaver of deceits_"

Or Homer of the girdle called the Cestus,

  "Persuasiveness _cheating_ e'en the subtlest mind."

And so since this kind of Imperfect Self-Control is more unjust, it
is also more disgraceful than that in respect of Anger, and is simply
Imperfect Self-Control, and Vice in a certain sense. Again, no man feels
pain in being insolent, but every one who acts through Anger does act
with pain; and he who acts insolently does it with pleasure. If then
those things are most unjust with which we have most right to be angry,
then Imperfect Self-Control, arising from Lust, is more so than that
arising from Anger: because in Anger there is no insolence.

Well then, it is clear that Imperfect Self-Control in respect of
Lusts is more disgraceful than that in respect of Anger, and that the
object-matter of Self-Control, and the Imperfection of it, are bodily
Lusts and pleasures; but of these last we must take into account the
differences; for, as was said at the commencement, some are proper to
the human race and natural both in kind and degree, others Brutish, and
others caused by maimings and diseases.

Now the first of these only are the object-matter of Perfected
Self-Mastery and utter absence of Self-Control; and therefore we never
attribute either of these states to Brutes (except metaphorically,
and whenever any one kind of animal differs entirely from another in
insolence, mischievousness, or voracity), because they have not moral
choice or process of deliberation, but are quite different from that
kind of creature just as are madmen from other men.

[Sidenote: 1150a] Brutishness is not so low in the scale as Vice, yet
it is to be regarded with more fear: because it is not that the highest
principle has been corrupted, as in the human creature, but the subject
has it not at all.

It is much the same, therefore, as if one should compare an inanimate
with an animate being, which were the worse: for the badness of that
which has no principle of origination is always less harmful; now
Intellect is a principle of origination. A similar case would be the
comparing injustice and an unjust man together: for in different ways
each is the worst: a bad man would produce ten thousand times as much
harm as a bad brute.

VII

Now with respect to the pleasures and pains which come to a man through
Touch and Taste, and the desiring or avoiding such (which we determined
before to constitute the object-matter of the states of utter absence of
Self-Control and Perfected Self-Mastery), one may be so disposed as
to yield to temptations to which most men would be superior, or to
be superior to those to which most men would yield: in respect of
pleasures, these characters will be respectively the man of Imperfect
Self-Control, and the man of Self-Control; and, in respect of pains, the
man of Softness and the man of Endurance: but the moral state of most
men is something between the two, even though they lean somewhat to the
worse characters.

Again, since of the pleasures indicated some are necessary and some are
not, others are so to a certain degree but not the excess or defect of
them, and similarly also of Lusts and pains, the man who pursues the
excess of pleasant things, or such as are in themselves excess, or from
moral choice, for their own sake, and not for anything else which is to
result from them, is a man utterly void of Self-Control: for he must be
incapable of remorse, and so incurable, because he that has not remorse
is incurable. (He that has too little love of pleasure is the opposite
character, and the man of Perfected Self-Mastery the mean character.) He
is of a similar character who avoids the bodily pains, not because he
_cannot_, but because he _chooses not to_, withstand them.

But of the characters who go wrong without _choosing_ so to do, the one
is led on by reason of pleasure, the other because he avoids the pain it
would cost him to deny his lust; and so they are different the one from
the other. Now every one would pronounce a man worse for doing something
base without any impulse of desire, or with a very slight one, than for
doing the same from the impulse of a very strong desire; for striking
a man when not angry than if he did so in wrath: because one naturally
says, "What would he have done had he been under the influence of
passion?" (and on this ground, by the bye, the man utterly void of
Self-Control is worse than he who has it imperfectly). However, of the
two characters which have been mentioned [as included in that of utter
absence of Self-Control], the one is rather Softness, the other properly
the man of no Self-Control.

Furthermore, to the character of Imperfect Self-Control is opposed that
of Self-Control, and to that of Softness that of Endurance: because
Endurance consists in continued resistance but Self-Control in actual
mastery, and continued resistance and actual mastery are as different
as not being conquered is from conquering; and so Self-Control is more
choiceworthy than Endurance.

[Sidenote:1150b] Again, he who fails when exposed to those temptations
against which the common run of men hold out, and are well able to do
so, is Soft and Luxurious (Luxury being a kind of Softness): the kind of
man, I mean, to let his robe drag in the dirt to avoid the trouble
of lifting it, and who, aping the sick man, does not however suppose
himself wretched though he is like a wretched man. So it is too with
respect to Self-Control and the Imperfection of it: if a man yields to
pleasures or pains which are violent and excessive it is no matter for
wonder, but rather for allowance if he made what resistance he could
(instances are, Philoctetes in Theodectes' drama when wounded by the
viper; or Cercyon in the Alope of Carcinus, or men who in trying to
suppress laughter burst into a loud continuous fit of it, as happened,
you remember, to Xenophantus), but it is a matter for wonder when a man
yields to and cannot contend against those pleasures or pains which the
common herd are able to resist; always supposing his failure not to be
owing to natural constitution or disease, I mean, as the Scythian kings
are constitutionally Soft, or the natural difference between the sexes.

Again, the man who is a slave to amusement is commonly thought to be
destitute of Self-Control, but he really is Soft; because amusement
is an act of relaxing, being an act of resting, and the character in
question is one of those who exceed due bounds in respect of this.

Moreover of Imperfect Self-Control there are two forms, Precipitancy and
Weakness: those who have it in the latter form though they have made
resolutions do not abide by them by reason of passion; the others are
led by passion because they have never formed any resolutions at
all: while there are some who, like those who by tickling themselves
beforehand get rid of ticklishness, having felt and seen beforehand the
approach of temptation, and roused up themselves and their resolution,
yield not to passion; whether the temptation be somewhat pleasant or
somewhat painful. The Precipitate form of Imperfect Self-Control they
are most liable to who are constitutionally of a sharp or melancholy
temperament: because the one by reason of the swiftness, the other by
reason of the violence, of their passions, do not wait for Reason,
because they are disposed to follow whatever notion is impressed upon
their minds.

VIII

Again, the man utterly destitute of Self-Control, as was observed
before, is not given to remorse: for it is part of his character that
he abides by his moral choice: but the man of Imperfect Self-Control is
almost made up of remorse: and so the case is not as we determined it
before, but the former is incurable and the latter may be cured: for
depravity is like chronic diseases, dropsy and consumption for instance,
but Imperfect Self-Control is like acute disorders: the former being a
continuous evil, the latter not so. And, in fact, Imperfect Self-Control
and Confirmed Vice are different in kind: the latter being imperceptible
to its victim, the former not so.

[Sidenote: 1151a] But, of the different forms of Imperfect Self-Control,
those are better who are carried off their feet by a sudden access of
temptation than they who have Reason but do not abide by it; these
last being overcome by passion less in degree, and not wholly without
premeditation as are the others: for the man of Imperfect Self-Control
is like those who are soon intoxicated and by little wine and less than
the common run of men. Well then, that Imperfection of Self-Control is
not Confirmed Viciousness is plain: and yet perhaps it is such in a way,
because in one sense it is contrary to moral choice and in another the
result of it: at all events, in respect of the actions, the case is much
like what Demodocus said of the Miletians. "The people of Miletus are
not fools, but they do just the kind of things that fools do;" and so
they of Imperfect Self-Control are not unjust, but they do unjust acts.

But to resume. Since the man of Imperfect Self-Control is of such a
character as to follow bodily pleasures in excess and in defiance of
Right Reason, without acting on any deliberate conviction, whereas the
man utterly destitute of Self-Control does act upon a conviction which
rests on his natural inclination to follow after these pleasures; the
former may be easily persuaded to a different course, but the latter
not: for Virtue and Vice respectively preserve and corrupt the moral
principle; now the motive is the principle or starting point in moral
actions, just as axioms and postulates are in mathematics: and neither
in morals nor mathematics is it Reason which is apt to teach the
principle; but Excellence, either natural or acquired by custom, in
holding right notions with respect to the principle. He who does this in
morals is the man of Perfected Self-Mastery, and the contrary character
is the man utterly destitute of Self-Control.

Again, there is a character liable to be taken off his feet in defiance
of Right Reason because of passion; whom passion so far masters as to
prevent his acting in accordance with Right Reason, but not so far as to
make him be convinced that it is his proper line to follow after such
pleasures without limit: this character is the man of Imperfect Self-
Control, better than he who is utterly destitute of it, and not a bad
man simply and without qualification: because in him the highest and
best part, i.e. principle, is preserved: and there is another character
opposed to him who is apt to abide by his resolutions, and not to depart
from them; at all events, not at the instigation of passion. It is
evident then from all this, that Self-Control is a good state and the
Imperfection of it a bad one.

Next comes the question, whether a man is a man of Self-Control for
abiding by his conclusions and moral choice be they of what kind they
may, or only by the right one; or again, a man of Imperfect Self-Control
for not abiding by his conclusions and moral choice be they of whatever
kind; or, to put the case we did before, is he such for not abiding by
false conclusions and wrong moral choice?

Is not this the truth, that _incidentally_ it is by conclusions and
moral choice of any kind that the one character abides and the other
does not, but _per se_ true conclusions and right moral choice: to
explain what is meant by incidentally, and _per se_; suppose a man
chooses or pursues this thing for the sake of that, he is said to pursue
and choose that _per se_, but this only incidentally. For the term _per
se_ we use commonly the word "simply," and so, in a way, it is opinion
of any kind soever by which the two characters respectively abide or
not, but he is "simply" entitled to the designations who abides or not
by the true opinion.

There are also people, who have a trick of abiding by their, own
opinions, who are commonly called Positive, as they who are hard to
be persuaded, and whose convictions are not easily changed: now these
people bear some resemblance to the character of Self-Control, just as
the prodigal to the liberal or the rash man to the brave, but they are
different in many points. The man of Self-Control does not change by
reason of passion and lust, yet when occasion so requires he will be
easy of persuasion: but the Positive man changes not at the call of
Reason, though many of this class take up certain desires and are led by
their pleasures. Among the class of Positive are the Opinionated, the
Ignorant, and the Bearish: the first, from the motives of pleasure and
pain: I mean, they have the pleasurable feeling of a kind of victory in
not having their convictions changed, and they are pained when their
decrees, so to speak, are reversed: so that, in fact, they rather
resemble the man of Imperfect Self-Control than the man of Self-Control.

Again, there are some who depart from their resolutions not by reason of
any Imperfection of Self-Control; take, for instance, Neoptolemus in the
Philoctetes of Sophocles. Here certainly pleasure was the motive of his
departure from his resolution, but then it was one of a noble sort:
for to be truthful was noble in his eyes and he had been persuaded by
Ulysses to lie.

So it is not every one who acts from the motive of pleasure who is
utterly destitute of Self-Control or base or of Imperfect Self-Control,
only he who acts from the impulse of a base pleasure.

Moreover as there is a character who takes less pleasure than he ought
in bodily enjoyments, and he also fails to abide by the conclusion of
his Reason, the man of Self-Control is the mean between him and the man
of Imperfect Self-Control: that is to say, the latter fails to abide by
them because of somewhat too much, the former because of somewhat too
little; while the man of Self-Control abides by them, and never changes
by reason of anything else than such conclusions.

Now of course since Self-Control is good both the contrary States must
be bad, as indeed they plainly are: but because the one of them is seen
in few persons, and but rarely in them, Self-Control comes to be
viewed as if opposed only to the Imperfection of it, just as
Perfected Self-Mastery is thought to be opposed only to utter want of
Self-Control.

[Sidenote: 1152a] Again, as many terms are used in the way of
similitude, so people have come to talk of the Self-Control of the man
of Perfected Self-Mastery in the way of similitude: for the man of
Self-Control and the man of Perfected Self-Mastery have this in common,
that they do nothing against Right Reason on the impulse of bodily
pleasures, but then the former has bad desires, the latter not; and the
latter is so constituted as not even to feel pleasure contrary to his
Reason, the former feels but does not yield to it. Like again are the
man of Imperfect Self-Control and he who is utterly destitute of it,
though in reality distinct: both follow bodily pleasures, but the latter
under a notion that it is the proper line for him to take, his former
without any such notion.



X

And it is not possible for the same man to be at once a man of Practical
Wisdom and of Imperfect Self-Control: because the character of Practical
Wisdom includes, as we showed before, goodness of moral character.
And again, it is not knowledge merely, but aptitude for action, which
constitutes Practical Wisdom: and of this aptitude the man of Imperfect
Self-Control is destitute. But there is no reason why the Clever man
should not be of Imperfect Self-Control: and the reason why some men are
occasionally thought to be men of Practical Wisdom, and yet of Imperfect
Self-Control, is this, that Cleverness differs from Practical Wisdom in
the way I stated in a former book, and is very near it so far as the
intellectual element is concerned but differs in respect of the moral
choice.

Nor is the man of Imperfect Self-Control like the man who both has and
calls into exercise his knowledge, but like the man who, having it, is
overpowered by sleep or wine. Again, he acts voluntarily (because he
knows, in a certain sense, what he does and the result of it), but he is
not a confirmed bad man, for his moral choice is good, so he is at all
events only half bad. Nor is he unjust, because he does not act with
deliberate intent: for of the two chief forms of the character, the one
is not apt to abide by his deliberate resolutions, and the other, the
man of constitutional strength of passion, is not apt to deliberate at
all.

So in fact the man of Imperfect Self-Control is like a community which
makes all proper enactments, and has admirable laws, only does not act
on them, verifying the scoff of Anaxandrides,

  "That State did will it, which cares nought for laws;"
whereas the bad man is like one which acts upon its laws, but then
unfortunately they are bad ones. Imperfection of Self-Control and
Self-Control, after all, are above the average state of men; because he
of the latter character is more true to his Reason, and the former less
so, than is in the power of most men.

Again, of the two forms of Imperfect Self-Control that is more easily
cured which they have who are constitutionally of strong passions, than
that of those who form resolutions and break them; and they that are so
through habituation than they that are so naturally; since of course
custom is easier to change than nature, because the very resemblance of
custom to nature is what constitutes the difficulty of changing it; as
Evenus says,

  "Practice, I say, my friend, doth long endure,
  And at the last is even very nature."

We have now said then what Self-Control is, what Imperfection of
Self-Control, what Endurance, and what Softness, and how these states
are mutually related.

XI

[Sidenote: II52b]

To consider the subject of Pleasure and Pain falls within the province
of the Social-Science Philosopher, since he it is who has to fix the
Master-End which is to guide us in dominating any object absolutely evil
or good.

But we may say more: an inquiry into their nature is absolutely
necessary. First, because we maintained that Moral Virtue and Moral Vice
are both concerned with Pains and Pleasures: next, because the greater
part of mankind assert that Happiness must include Pleasure (which by
the way accounts for the word they use, makarioz; chaireiu being the
root of that word).

Now some hold that no one Pleasure is good, either in itself or as a
matter of result, because Good and Pleasure are not identical. Others
that some Pleasures are good but the greater number bad. There is yet a
third view; granting that every Pleasure is good, still the Chief Good
cannot possibly be Pleasure.

In support of the first opinion (that Pleasure is utterly not-good) it
is urged that:

I. Every Pleasure is a sensible process towards a complete state; but
no such process is akin to the end to be attained: _e.g._ no process of
building to the completed house.

2. The man of Perfected Self-Mastery avoids Pleasures.

3. The man of Practical Wisdom aims at avoiding Pain, not at attaining
Pleasure.

4. Pleasures are an impediment to thought, and the more so the more
keenly they are felt. An obvious instance will readily occur.

5. Pleasure cannot be referred to any Art: and yet every good is the
result of some Art.

6. Children and brutes pursue Pleasures.

In support of the second (that not all Pleasures are good), That there
are some base and matter of reproach, and some even hurtful: because
some things that are pleasant produce disease.

In support of the third (that Pleasure is not the Chief Good), That it
is not an End but a process towards creating an End.

This is, I think, a fair account of current views on the matter.

XII


But that the reasons alleged do not prove it either to be not-good or
the Chief Good is plain from the following considerations.

First. Good being either absolute or relative, of course the natures and
states embodying it will be so too; therefore also the movements and the
processes of creation. So, of those which are thought to be bad
some will be bad absolutely, but relatively not bad, perhaps even
choiceworthy; some not even choiceworthy relatively to any particular
person, only at certain times or for a short time but not in themselves
choiceworthy.

Others again are not even Pleasures at all though they produce that
impression on the mind: all such I mean as imply pain and whose purpose
is cure; those of sick people, for instance.

Next, since Good may be either an active working or a state, those
[Greek: _kinaeseis_ or _geneseis_] which tend to place us in our natural
state are pleasant incidentally because of that *[Sidenote: 1153a]
tendency: but the active working is really in the desires excited in the
remaining (sound) part of our state or nature: for there are Pleasures
which have no connection with pain or desire: the acts of contemplative
intellect, for instance, in which case there is no deficiency in the
nature or state of him who performs the acts.

A proof of this is that the same pleasant thing does not produce the
sensation of Pleasure when the natural state is being filled up or
completed as when it is already in its normal condition: in this latter
case what give the sensation are things pleasant _per se_, in the former
even those things which are contrary. I mean, you find people taking
pleasure in sharp or bitter things of which no one is naturally or in
itself pleasant; of course not therefore the Pleasures arising from
them, because it is obvious that as is the classification of pleasant
things such must be that of the Pleasures arising from them.

Next, it does not follow that there must be something else better than
any given pleasure because (as some say) the End must be better than
the process which creates it. For it is not true that all Pleasures
are processes or even attended by any process, but (some are) active
workings or even Ends: in fact they result not from our coming to be
something but from our using our powers. Again, it is not true that the
End is, in every case, distinct from the process: it is true only in
the case of such processes as conduce to the perfecting of the natural
state.

For which reason it is wrong to say that Pleasure is "a sensible process
of production." For "process etc." should be substituted "active working
of the natural state," for "sensible" "unimpeded." The reason of its
being thought to be a "process etc." is that it is good in the highest
sense: people confusing "active working" and "process," whereas they
really are distinct.

Next, as to the argument that there are bad Pleasures because some
things which are pleasant are also hurtful to health, it is the same as
saying that some healthful things are bad for "business." In this sense,
of course, both may be said to be bad, but then this does not make
them out to be bad _simpliciter_: the exercise of the pure Intellect
sometimes hurts a man's health: but what hinders Practical Wisdom or
any state whatever is, not the Pleasure peculiar to, but some Pleasure
foreign to it: the Pleasures arising from the exercise of the pure
Intellect or from learning only promote each.

Next. "No Pleasure is the work of any Art." What else would you expect?
No active working is the work of any Art, only the faculty of so
working. Still the perfumer's Art or the cook's are thought to belong to
Pleasure.

Next. "The man of Perfected Self-Mastery avoids Pleasures." "The man
of Practical Wisdom aims at escaping Pain rather than at attaining
Pleasure."

"Children and brutes pursue Pleasures."

One answer will do for all.

We have already said in what sense all Pleasures are good _per se_ and
in what sense not all are good: it is the latter class that brutes and
children pursue, such as are accompanied by desire and pain, that is the
bodily Pleasures (which answer to this description) and the excesses of
them: in short, those in respect of which the man utterly destitute of
Self-Control is thus utterly destitute. And it is the absence of the
pain arising from these Pleasures that the man of Practical Wisdom
aims at. It follows that these Pleasures are what the man of Perfected
Self-Mastery avoids: for obviously he has Pleasures peculiarly his own.

[Sidenote: XIII 1153_b_] Then again, it is allowed that Pain is an evil
and a thing to be avoided partly as bad _per se_, partly as being a
hindrance in some particular way. Now the contrary of that which is to
be avoided, _quâ_ it is to be avoided, _i.e._ evil, is good. Pleasure
then must be _a_ good.

The attempted answer of Speusippus, "that Pleasure may be opposed and
yet not contrary to Pain, just as the greater portion of any magnitude
is contrary to the less but only opposed to the exact half," will not
hold: for he cannot say that Pleasure is identical with evil of any
kind. Again. Granting that some Pleasures are low, there is no reason
why some particular Pleasure may not be very good, just as some
particular Science may be although there are some which are low.

Perhaps it even follows, since each state may have active working
unimpeded, whether the active workings of all be Happiness or that of
some one of them, that this active working, if it be unimpeded, must be
choiceworthy: now Pleasure is exactly this. So that the Chief Good may
be Pleasure of some kind, though most Pleasures be (let us assume) low
_per se_.

And for this reason all men think the happy life is pleasant, and
interweave Pleasure with Happiness. Reasonably enough: because Happiness
is perfect, but no impeded active working is perfect; and therefore
the happy man needs as an addition the goods of the body and the goods
external and fortune that in these points he may not be fettered. As for
those who say that he who is being tortured on the wheel, or falls into
great misfortunes is happy provided only he be good, they talk nonsense,
whether they mean to do so or not. On the other hand, because fortune
is needed as an addition, some hold good fortune to be identical with
Happiness: which it is not, for even this in excess is a hindrance, and
perhaps then has no right to be called good fortune since it is good
only in so far as it contributes to Happiness.

The fact that all animals, brute and human alike, pursue Pleasure, is
some presumption of its being in a sense the Chief Good;

("There must be something in what most folks say,") only as one and
the same nature or state neither is nor is thought to be the best, so
neither do all pursue the same Pleasure, Pleasure nevertheless all do.
Nay further, what they pursue is, perhaps, not what they think nor what
they would say they pursue, but really one and the same: for in all
there is some instinct above themselves. But the bodily Pleasures have
received the name exclusively, because theirs is the most frequent form
and that which is universally partaken of; and so, because to many these
alone are known they believe them to be the only ones which exist.

[Sidenote: II54a]

It is plain too that, unless Pleasure and its active working be good, it
will not be true that the happy man's life embodies Pleasure: for why
will he want it on the supposition that it is not good and that he can
live even with Pain? because, assuming that Pleasure is not good, then
Pain is neither evil nor good, and so why should he avoid it?

Besides, the life of the good man is not more pleasurable than any other
unless it be granted that his active workings are so too.

XIV

Some inquiry into the bodily Pleasures is also necessary for those who
say that some Pleasures, to be sure, are highly choiceworthy (the good
ones to wit), but not the bodily Pleasures; that is, those which are the
object-matter of the man utterly destitute of Self-Control.

If so, we ask, why are the contrary Pains bad? they cannot be (on their
assumption) because the contrary of bad is good.

May we not say that the necessary bodily Pleasures are good in the sense
in which that which is not-bad is good? or that they are good only up
to a certain point? because such states or movements as cannot have too
much of the better cannot have too much of Pleasure, but those which can
of the former can also of the latter. Now the bodily Pleasures do admit
of excess: in fact the low bad man is such because he pursues the excess
of them instead of those which are necessary (meat, drink, and the
objects of other animal appetites do give pleasure to all, but not in
right manner or degree to all). But his relation to Pain is exactly the
contrary: it is not excessive Pain, but Pain at all, that he avoids
[which makes him to be in this way too a bad low man], because only
in the case of him who pursues excessive Pleasure is Pain contrary to
excessive Pleasure.

It is not enough however merely to state the truth, we should also show
how the false view arises; because this strengthens conviction. I mean,
when we have given a probable reason why that impresses people as true
which really is not true, it gives them a stronger conviction of the
truth. And so we must now explain why the bodily Pleasures appear to
people to be more choiceworthy than any others.

The first obvious reason is, that bodily Pleasure drives out Pain; and
because Pain is felt in excess men pursue Pleasure in excess, _i.e._
generally bodily Pleasure, under the notion of its being a remedy for
that Pain. These remedies, moreover, come to be violent ones; which is
the very reason they are pursued, since the impression they produce
on the mind is owing to their being looked at side by side with their
contrary.

And, as has been said before, there are the two following reasons why
bodily Pleasure is thought to be not-good.

1. Some Pleasures of this class are actings of a low nature, whether
congenital as in brutes, or acquired by custom as in low bad men.

2. Others are in the nature of cures, cures that is of some deficiency;
now of course it is better to have [the healthy state] originally than
that it should accrue afterwards.

[Sidenote: 1154b] But some Pleasures result when natural states are
being perfected: these therefore are good as a matter of result.

Again, the very fact of their being violent causes them to be pursued by
such as can relish no others: such men in fact create violent thirsts
for themselves (if harmless ones then we find no fault, if harmful then
it is bad and low) because they have no other things to take
pleasure in, and the neutral state is distasteful to some people
constitutionally; for toil of some kind is inseparable from life, as
physiologists testify, telling us that the acts of seeing or hearing are
painful, only that we are used to the pain and do not find it out.

Similarly in youth the constant growth produces a state much like
that of vinous intoxication, and youth is pleasant. Again, men of the
melancholic temperament constantly need some remedial process (because
the body, from its temperament, is constantly being worried), and they
are in a chronic state of violent desire. But Pleasure drives out Pain;
not only such Pleasure as is directly contrary to Pain but even any
Pleasure provided it be strong: and this is how men come to be utterly
destitute of Self-Mastery, _i.e._ low and bad.

But those Pleasures which are unconnected with Pains do not admit of
excess: _i.e._ such as belong to objects which are naturally pleasant
and not merely as a matter of result: by the latter class I mean such
as are remedial, and the reason why these are thought to be pleasant is
that the cure results from the action in some way of that part of the
constitution which remains sound. By "pleasant naturally" I mean such as
put into action a nature which is pleasant.

The reason why no one and the same thing is invariably pleasant is that
our nature is, not simple, but complex, involving something different
from itself (so far as we are corruptible beings). Suppose then that one
part of this nature be doing something, this something is, to the other
part, unnatural: but, if there be an equilibrium of the two natures,
then whatever is being done is indifferent. It is obvious that if there
be any whose nature is simple and not complex, to such a being the same
course of acting will always be the most pleasurable.

For this reason it is that the Divinity feels Pleasure which is always
one, _i.e._ simple: not motion merely but also motionlessness acts, and
Pleasure resides rather in the absence than in the presence of motion.

The reason why the Poet's dictum "change is of all things most pleasant"
is true, is "a baseness in our blood;" for as the bad man is easily
changeable, bad must be also the nature that craves change, _i.e._ it is
neither simple nor good.

We have now said our say about Self-Control and its opposite; and about
Pleasure and Pain. What each is, and how the one set is good the other
bad. We have yet to speak of Friendship.




BOOK VIII

[Sidenote: I 1155_a_] Next would seem properly to follow a dissertation
on Friendship: because, in the first place, it is either itself a virtue
or connected with virtue; and next it is a thing most necessary for
life, since no one would choose to live without friends though he should
have all the other good things in the world: and, in fact, men who are
rich or possessed of authority and influence are thought to have special
need of friends: for where is the use of such prosperity if there be
taken away the doing of kindnesses of which friends are the most usual
and most commendable objects? Or how can it be kept or preserved without
friends? because the greater it is so much the more slippery and
hazardous: in poverty moreover and all other adversities men think
friends to be their only refuge.

Furthermore, Friendship helps the young to keep from error: the old, in
respect of attention and such deficiencies in action as their weakness
makes them liable to; and those who are in their prime, in respect of
noble deeds ("They _two_ together going," Homer says, you may remember),
because they are thus more able to devise plans and carry them out.

Again, it seems to be implanted in us by Nature: as, for instance, in
the parent towards the offspring and the offspring towards the parent
(not merely in the human species, but likewise in birds and most
animals), and in those of the same tribe towards one another, and
specially in men of the same nation; for which reason we commend those
men who love their fellows: and one may see in the course of travel how
close of kin and how friendly man is to man.

Furthermore, Friendship seems to be the bond of Social Communities, and
legislators seem to be more anxious to secure it than Justice even. I
mean, Unanimity is somewhat like to Friendship, and this they certainly
aim at and specially drive out faction as being inimical.

Again, where people are in Friendship Justice is not required; but, on
the other hand, though they are just they need Friendship in addition,
and that principle which is most truly just is thought to partake of the
nature of Friendship.

Lastly, not only is it a thing necessary but honourable likewise: since
we praise those who are fond of friends, and the having numerous friends
is thought a matter of credit to a man; some go so far as to hold, that
"good man" and "friend" are terms synonymous.

Yet the disputed points respecting it are not few: some men lay down
that it is a kind of resemblance, and that men who are like one another
are friends: whence come the common sayings, "Like will to like," "Birds
of a feather," and so on. Others, on the contrary, say, that all such
come under the maxim, "Two of a trade never agree."

[Sidenote: 1155b] Again, some men push their inquiries on these points
higher and reason physically: as Euripides, who says,

  "The earth by drought consumed doth love the rain,
  And the great heaven, overcharged with rain,
  Doth love to fall in showers upon the earth."

Heraclitus, again, maintains, that "contrariety is expedient, and that
the best agreement arises from things differing, and that all things
come into being in the way of the principle of antagonism."

Empedocles, among others, in direct opposition to these, affirms, that
"like aims at like."

These physical questions we will take leave to omit, inasmuch as they
are foreign to the present inquiry; and we will examine such as are
proper to man and concern moral characters and feelings: as, for
instance, "Does Friendship arise among all without distinction, or is it
impossible for bad men to be friends?" and, "Is there but one species of
Friendship, or several?" for they who ground the opinion that there is
but one on the fact that Friendship admits of degrees hold that upon
insufficient proof; because things which are different in species admit
likewise of degrees (on this point we have spoken before).


II

Our view will soon be cleared on these points when we have ascertained
what is properly the object-matter of Friendship: for it is thought that
not everything indiscriminately, but some peculiar matter alone, is the
object of this affection; that is to say, what is good, or pleasurable,
or useful. Now it would seem that that is useful through which accrues
any good or pleasure, and so the objects of Friendship, as absolute
Ends, are the good and the pleasurable.

A question here arises; whether it is good absolutely or that which is
good to the individuals, for which men feel Friendship (these two being
sometimes distinct): and similarly in respect of the pleasurable. It
seems then that each individual feels it towards that which is good to
himself, and that abstractedly it is the real good which is the object
of Friendship, and to each individual that which is good to each. It
comes then to this; that each individual feels Friendship not for what
_is_ but for that which _conveys to his mind the impression of being_
good to himself. But this will make no real difference, because that
which is truly the object of Friendship will also convey this impression
to the mind.

There are then three causes from which men feel Friendship: but the term
is not applied to the case of fondness for things inanimate because
there is no requital of the affection nor desire for the good of those
objects: it certainly savours of the ridiculous to say that a man fond
of wine wishes well to it: the only sense in which it is true being that
he wishes it to be kept safe and sound for his own use and benefit. But
to the friend they say one should wish all good for his sake. And when
men do thus wish good to another (he not *[Sidenote: 1156a]
reciprocating the feeling), people call them Kindly; because Friendship
they describe as being "Kindliness between persons who reciprocate it."
But must they not add that the feeling must be mutually known? for many
men are kindly disposed towards those whom they have never seen but whom
they conceive to be amiable or useful: and this notion amounts to the
same thing as a real feeling between them.

Well, these are plainly Kindly-disposed towards one another: but how can
one call them friends while their mutual feelings are unknown to one
another? to complete the idea of Friendship, then, it is requisite that
they have kindly feelings towards one another, and wish one another good
from one of the aforementioned causes, and that these kindly feelings
should be mutually known.

III


As the motives to Friendship differ in kind so do the respective
feelings and Friendships. The species then of Friendship are three, in
number equal to the objects of it, since in the line of each there may
be "mutual affection mutually known."

Now they who have Friendship for one another desire one another's good
according to the motive of their Friendship; accordingly they whose
motive is utility have no Friendship for one another really, but only in
so far as some good arises to them from one another.

And they whose motive is pleasure are in like case: I mean, they have
Friendship for men of easy pleasantry, not because they are of a given
character but because they are pleasant to themselves. So then they
whose motive to Friendship is utility love their friends for what is
good to themselves; they whose motive is pleasure do so for what is
pleasurable to themselves; that is to say, not in so far as the friend
beloved _is_ but in so far as he is useful or pleasurable. These
Friendships then are a matter of result: since the object is not beloved
in that he is the man he is but in that he furnishes advantage or
pleasure as the case may be. Such Friendships are of course very liable
to dissolution if the parties do not continue alike: I mean, that the
others cease to have any Friendship for them when they are no longer
pleasurable or useful. Now it is the nature of utility not to be
permanent but constantly varying: so, of course, when the motive which
made them friends is vanished, the Friendship likewise dissolves; since
it existed only relatively to those circumstances.

Friendship of this kind is thought to exist principally among the old
(because men at that time of life pursue not what is pleasurable but
what is profitable); and in such, of men in their prime and of the
young, as are given to the pursuit of profit. They that are such have no
intimate intercourse with one another; for sometimes they are not
even pleasurable to one another; nor, in fact, do they desire such
intercourse unless their friends are profitable to them, because they
are pleasurable only in so far as they have hopes of advantage. With
these Friendships is commonly ranked that of hospitality.

But the Friendship of the young is thought to be based on the motive
of pleasure: because they live at the beck and call of passion and
generally pursue what is pleasurable to themselves and the object of the
present moment: and as their age changes so likewise do their pleasures.

This is the reason why they form and dissolve Friendships rapidly: since
the Friendship changes with the pleasurable object and such pleasure
changes quickly.

[Sidenote: 1156b] The young are also much given up to Love; this passion
being, in great measure, a matter of impulse and based on pleasure: for
which cause they conceive Friendships and quickly drop them, changing
often in the same day: but these wish for society and intimate
intercourse with their friends, since they thus attain the object of
their Friendship.

That then is perfect Friendship which subsists between those who are
good and whose similarity consists in their goodness: for these men wish
one another's good in similar ways; in so far as they are good (and good
they are in themselves); and those are specially friends who wish good
to their friends for their sakes, because they feel thus towards them on
their own account and not as a mere matter of result; so the Friendship
between these men continues to subsist so long as they are good; and
goodness, we know, has in it a principle of permanence.

Moreover, each party is good abstractedly and also relatively to his
friend, for all good men are not only abstractedly good but also useful
to one another. Such friends are also mutually pleasurable because
all good men are so abstractedly, and also relatively to one another,
inasmuch as to each individual those actions are pleasurable which
correspond to his nature, and all such as are like them. Now when men
are good these will be always the same, or at least similar.

Friendship then under these circumstances is permanent, as we should
reasonably expect, since it combines in itself all the requisite
qualifications of friends. I mean, that Friendship of whatever kind is
based upon good or pleasure (either abstractedly or relatively to the
person entertaining the sentiment of Friendship), and results from a
similarity of some sort; and to this kind belong all the aforementioned
requisites in the parties themselves, because in this the parties are
similar, and so on: moreover, in it there is the abstractedly good and
the abstractedly pleasant, and as these are specially the object-matter
of Friendship so the feeling and the state of Friendship is found most
intense and most excellent in men thus qualified.

Rare it is probable Friendships of this kind will be, because men
of this kind are rare. Besides, all requisite qualifications being
presupposed, there is further required time and intimacy: for, as the
proverb says, men cannot know one another "till they have eaten the
requisite quantity of salt together;" nor can they in fact admit one
another to intimacy, much less be friends, till each has appeared to
the other and been proved to be a fit object of Friendship. They who
speedily commence an interchange of friendly actions may be said to wish
to be friends, but they are not so unless they are also proper objects
of Friendship and mutually known to be such: that is to say, a desire
for Friendship may arise quickly but not Friendship itself.


IV

Well, this Friendship is perfect both in respect of the time and in all
other points; and exactly the same and similar results accrue to each
party from the other; which ought to be the case between friends.

[Sidenote: II57a] The friendship based upon the pleasurable is, so to
say, a copy of this, since the good are sources of pleasure to one
another: and that based on utility likewise, the good being also
useful to one another. Between men thus connected Friendships are
most permanent when the same result accrues to both from one another,
pleasure, for instance; and not merely so but from the same source, as
in the case of two men of easy pleasantry; and not as it is in that of a
lover and the object of his affection, these not deriving their pleasure
from the same causes, but the former from seeing the latter and the
latter from receiving the attentions of the former: and when the bloom
of youth fades the Friendship sometimes ceases also, because then the
lover derives no pleasure from seeing and the object of his affection
ceases to receive the attentions which were paid before: in many cases,
however, people so connected continue friends, if being of similar
tempers they have come from custom to like one another's disposition.

Where people do not interchange pleasure but profit in matters of Love,
the Friendship is both less intense in degree and also less permanent:
in fact, they who are friends because of advantage commonly part when
the advantage ceases; for, in reality, they never were friends of one
another but of the advantage.

So then it appears that from motives of pleasure or profit bad men may
be friends to one another, or good men to bad men or men of neutral
character to one of any character whatever: but disinterestedly, for the
sake of one another, plainly the good alone can be friends; because
bad men have no pleasure even in themselves unless in so far as some
advantage arises.

And further, the Friendship of the good is alone superior to calumny;
it not being easy for men to believe a third person respecting one
whom they have long tried and proved: there is between good men mutual
confidence, and the feeling that one's friend would never have done one
wrong, and all other such things as are expected in Friendship really
worthy the name; but in the other kinds there is nothing to prevent all
such suspicions.

I call them Friendships, because since men commonly give the name of
friends to those who are connected from motives of profit (which is
justified by political language, for alliances between states are
thought to be contracted with a view to advantage), and to those who are
attached to one another by the motive of pleasure (as children are), we
may perhaps also be allowed to call such persons friends, and say there
are several species of Friendship; primarily and specially that of
the good, in that they are good, and the rest only in the way of
resemblance: I mean, people connected otherwise are friends in that way
in which there arises to them somewhat good and some mutual resemblance
(because, we must remember the pleasurable is good to those who are fond
of it).

These secondary Friendships, however, do not combine very well; that is
to say, the same persons do not become friends by reason of advantage
and by reason of the pleasurable, for these matters of result are not
often combined. And Friendship having been divided into these kinds, bad
[Sidenote: _1157b_] men will be friends by reason of pleasure or profit,
this being their point of resemblance; while the good are friends for
one another's sake, that is, in so far as they are good.

These last may be termed abstractedly and simply friends, the former as
a matter of result and termed friends from their resemblance to these
last.


V

Further; just as in respect of the different virtues some men are termed
good in respect of a certain inward state, others in respect of acts
of working, so is it in respect of Friendship: I mean, they who live
together take pleasure in, and impart good to, one another: but they who
are asleep or are locally separated do not perform acts, but only are in
such a state as to act in a friendly way if they acted at all: distance
has in itself no direct effect upon Friendship, but only prevents the
acting it out: yet, if the absence be protracted, it is thought to cause
a forgetfulness even of the Friendship: and hence it has been said,
"many and many a Friendship doth want of intercourse destroy."

Accordingly, neither the old nor the morose appear to be calculated for
Friendship, because the pleasurableness in them is small, and no one can
spend his days in company with that which is positively painful or even
not pleasurable; since to avoid the painful and aim at the pleasurable
is one of the most obvious tendencies of human nature. They who get on
with one another very fairly, but are not in habits of intimacy, are
rather like people having kindly feelings towards one another than
friends; nothing being so characteristic of friends as the living with
one another, because the necessitous desire assistance, and the happy
companionship, they being the last persons in the world for solitary
existence: but people cannot spend their time together unless they are
mutually pleasurable and take pleasure in the same objects, a quality
which is thought to appertain to the Friendship of companionship.

The connection then subsisting between the good is Friendship _par
excellence_, as has already been frequently said: since that which is
abstractedly good or pleasant is thought to be an object of Friendship
and choiceworthy, and to each individual whatever is such to him;
and the good man to the good man for both these reasons. (Now the
entertaining the sentiment is like a feeling, but Friendship itself
like a state: because the former may have for its object even things
inanimate, but requital of Friendship is attended with moral choice
which proceeds from a moral state: and again, men wish good to the
objects of their Friendship for their sakes, not in the way of a mere
feeling but of moral state.).

And the good, in loving their friend, love their own good (inasmuch as
the good man, when brought into that relation, becomes a good to him
with whom he is so connected), so that either party loves his own
good, and repays his friend equally both in wishing well and in the
pleasurable: for equality is said to be a tie of Friendship. Well, these
points belong most to the Friendship between good men.

But between morose or elderly men Friendship is less apt to arise,
because they are somewhat awkward-tempered, and take less pleasure in
intercourse and society; these being thought to be specially friendly
and productive of Friendship: and so young men become friends quickly,
old men not so (because people do not become friends with any, unless
they take pleasure in them); and in like manner neither do the morose.
Yet men of these classes entertain kindly feelings towards one another:
they wish good to one another and render mutual assistance in respect of
their needs, but they are not quite friends, because they neither
spend their time together nor take pleasure in one another, which
circumstances are thought specially to belong to Friendship.

To be a friend to many people, in the way of the perfect Friendship, is
not possible; just as you cannot be in love with many at once: it is,
so to speak, a state of excess which naturally has but one object; and
besides, it is not an easy thing for one man to be very much pleased
with many people at the same time, nor perhaps to find many really good.
Again, a man needs experience, and to be in habits of close intimacy,
which is very difficult.

But it _is_ possible to please many on the score of advantage and
pleasure: because there are many men of the kind, and the services may
be rendered in a very short time.

Of the two imperfect kinds that which most resembles the perfect is the
Friendship based upon pleasure, in which the same results accrue from
both and they take pleasure in one another or in the same objects; such
as are the Friendships of the young, because a generous spirit is most
found in these. The Friendship because of advantage is the connecting
link of shopkeepers.

Then again, the very happy have no need of persons who are profitable,
but of pleasant ones they have because they wish to have people to live
intimately with; and what is painful they bear for a short time indeed,
but continuously no one could support it, nay, not even the Chief Good
itself, if it were painful to him individually: and so they look out for
pleasant friends: perhaps they ought to require such to be good also;
and good moreover to themselves individually, because then they will
have all the proper requisites of Friendship.

Men in power are often seen to make use of several distinct friends:
for some are useful to them and others pleasurable, but the two are not
often united: because they do not, in fact, seek such as shall combine
pleasantness and goodness, nor such as shall be useful for honourable
purposes: but with a view to attain what is pleasant they look out for
men of easy-pleasantry; and again, for men who are clever at executing
any business put into their hands: and these qualifications are not
commonly found united in the same man.

It has been already stated that the good man unites the qualities of
pleasantness and usefulness: but then such a one will not be a friend to
a superior unless he be also his superior in goodness: for if this be
not the case, he cannot, being surpassed in one point, make things
equal by a proportionate degree of Friendship. And characters who unite
superiority of station and goodness are not common. Now all the kinds
of Friendship which have been already mentioned exist in a state of
equality, inasmuch as either the same results accrue to both and they
wish the same things to one another, or else they barter one thing
against another; pleasure, for instance, against profit: it has been
said already that Friendships of this latter kind are less intense in
degree and less permanent.

And it is their resemblance or dissimilarity to the same thing which
makes them to be thought to be and not to be Friendships: they show like
Friendships in right of their likeness to that which is based on virtue
(the one kind having the pleasurable, the other the profitable, both
of which belong also to the other); and again, they do not show like
Friendships by reason of their unlikeness to that true kind; which
unlikeness consists herein, that while that is above calumny and so
permanent these quickly change and differ in many other points.


VII

But there is another form of Friendship, that, namely, in which the one
party is superior to the other; as between father and son, elder and
younger, husband and wife, ruler and ruled. These also differ one from
another: I mean, the Friendship between parents and children is not the
same as between ruler and the ruled, nor has the father the same towards
the son as the son towards the father, nor the husband towards the wife
as she towards him; because the work, and therefore the excellence, of
each of these is different, and different therefore are the causes of
their feeling Friendship; distinct and different therefore are their
feelings and states of Friendship.

And the same results do not accrue to each from the other, nor in fact
ought they to be looked for: but, when children render to their parents
what they ought to the authors of their being, and parents to their sons
what they ought to their offspring, the Friendship between such parties
will be permanent and equitable.

Further; the feeling of Friendship should be in a due proportion in all
Friendships which are between superior and inferior; I mean, the better
man, or the more profitable, and so forth, should be the object of a
stronger feeling than he himself entertains, because when the feeling of
Friendship comes to be after a certain rate then equality in a certain
sense is produced, which is thought to be a requisite in Friendship.

(It must be remembered, however, that the equal is not in the same case
as regards Justice and Friendship: for in strict Justice the exactly
proportioned equal ranks first, and the actual numerically equal ranks
second, while in Friendship this is exactly reversed.)

[Sidenote: 1159a] And that equality is thus requisite is plainly shown
by the occurrence of a great difference of goodness or badness, or
prosperity, or something else: for in this case, people are not any
longer friends, nay they do not even feel that they ought to be. The
clearest illustration is perhaps the case of the gods, because they are
most superior in all good things. It is obvious too, in the case of
kings, for they who are greatly their inferiors do not feel entitled to
be friends to them; nor do people very insignificant to be friends to
those of very high excellence or wisdom. Of course, in such cases it
is out of the question to attempt to define up to what point they may
continue friends: for you may remove many points of agreement and the
Friendship last nevertheless; but when one of the parties is very far
separated (as a god from men), it cannot continue any longer.

This has given room for a doubt, whether friends do really wish to their
friends the very highest goods, as that they may be gods: because, in
case the wish were accomplished, they would no longer have them for
friends, nor in fact would they have the good things they had, because
friends are good things. If then it has been rightly said that a friend
wishes to his friend good things for that friend's sake, it must be
understood that he is to remain such as he now is: that is to say, he
will wish the greatest good to him of which as man he is capable: yet
perhaps not all, because each man desires good for himself most of all.

VIII

It is thought that desire for honour makes the mass of men wish rather
to be the objects of the feeling of Friendship than to entertain it
themselves (and for this reason they are fond of flatterers, a flatterer
being a friend inferior or at least pretending to be such and rather to
entertain towards another the feeling of Friendship than to be himself
the object of it), since the former is thought to be nearly the same as
being honoured, which the mass of men desire. And yet men seem to choose
honour, not for its own sake, but incidentally: I mean, the common run
of men delight to be honoured by those in power because of the hope it
raises; that is they think they shall get from them anything they may
happen to be in want of, so they delight in honour as an earnest of
future benefit. They again who grasp at honour at the hands of the good
and those who are really acquainted with their merits desire to confirm
their own opinion about themselves: so they take pleasure in the
conviction that they are good, which is based on the sentence of those
who assert it. But in being the objects of Friendship men delight for
its own sake, and so this may be judged to be higher than being honoured
and Friendship to be in itself choiceworthy. Friendship, moreover, is
thought to consist in feeling, rather than being the object of, the
sentiment of Friendship, which is proved by the delight mothers have in
the feeling: some there are who give their children to be adopted and
brought up by others, and knowing them bear this feeling towards them
never seeking to have it returned, if both are not possible; but seeming
to be content with seeing them well off and bearing this feeling
themselves towards them, even though they, by reason of ignorance, never
render to them any filial regard or love.

Since then Friendship stands rather in the entertaining, than in being
the object of, the sentiment, and they are praised who are fond of their
friends, it seems that entertaining--*[Sidenote: II59b]the sentiment is
the Excellence of friends; and so, in whomsoever this exists in due
proportion these are stable friends and their Friendship is permanent.
And in this way may they who are unequal best be friends, because they
may thus be made equal.

Equality, then, and similarity are a tie to Friendship, and specially
the similarity of goodness, because good men, being stable in
themselves, are also stable as regards others, and neither ask degrading
services nor render them, but, so to say, rather prevent them: for it is
the part of the good neither to do wrong themselves nor to allow their
friends in so doing.

The bad, on the contrary, have no principle of stability: in fact, they
do not even continue like themselves: only they come to be friends for
a short time from taking delight in one another's wickedness. Those
connected by motives of profit, or pleasure, hold together somewhat
longer: so long, that is to say, as they can give pleasure or profit
mutually.

The Friendship based on motives of profit is thought to be most of all
formed out of contrary elements: the poor man, for instance, is thus a
friend of the rich, and the ignorant of the man of information; that
is to say, a man desiring that of which he is, as it happens, in want,
gives something else in exchange for it. To this same class we may refer
the lover and beloved, the beautiful and the ill-favoured. For this
reason lovers sometimes show in a ridiculous light by claiming to be the
objects of as intense a feeling as they themselves entertain: of course
if they are equally fit objects of Friendship they are perhaps entitled
to claim this, but if they have nothing of the kind it is ridiculous.

Perhaps, moreover, the contrary does not aim at its contrary for its own
sake but incidentally: the mean is really what is grasped at; it being
good for the dry, for instance, not to become wet but to attain the
mean, and so of the hot, etc. However, let us drop these questions,
because they are in fact somewhat foreign to our purpose.

IX

It seems too, as was stated at the commencement, that Friendship and
Justice have the same object-matter, and subsist between the same
persons: I mean that in every Communion there is thought to be some
principle of Justice and also some Friendship: men address as friends,
for instance, those who are their comrades by sea, or in war, and in
like manner also those who are brought into Communion with them in other
ways: and the Friendship, because also the Justice, is co-extensive with
the Communion, This justifies the common proverb, "the goods of friends
are common," since Friendship rests upon Communion.

[1160a] Now brothers and intimate companions have all in common, but
other people have their property separate, and some have more in common
and others less, because the Friendships likewise differ in degree. So
too do the various principles of Justice involved, not being the same
between parents and children as between brothers, nor between companions
as between fellow-citizens merely, and so on of all the other
conceivable Friendships. Different also are the principles of Injustice
as regards these different grades, and the acts become intensified by
being done to friends; for instance, it is worse to rob your companion
than one who is merely a fellow-citizen; to refuse help to a brother
than to a stranger; and to strike your father than any one else. So then
the Justice naturally increases with the degree of Friendship, as being
between the same parties and of equal extent.

All cases of Communion are parts, so to say, of the great Social one,
since in them men associate with a view to some advantage and to procure
some of those things which are needful for life; and the great Social
Communion is thought originally to have been associated and to
continue for the sake of some advantage: this being the point at which
legislators aim, affirming that to be just which is generally expedient.
All the other cases of Communion aim at advantage in particular points;
the crew of a vessel at that which is to result from the voyage which is
undertaken with a view to making money, or some such object; comrades in
war at that which is to result from the war, grasping either at wealth
or victory, or it may be a political position; and those of the same
tribe, or Demus, in like manner.

Some of them are thought to be formed for pleasure's sake, those, for
instance, of bacchanals or club-fellows, which are with a view to
Sacrifice or merely company. But all these seem to be ranged under
the great Social one, inasmuch as the aim of this is, not merely the
expediency of the moment but, for life and at all times; with a view
to which the members of it institute sacrifices and their attendant
assemblies, to render honour to the gods and procure for themselves
respite from toil combined with pleasure. For it appears that
sacrifices and religious assemblies in old times were made as a kind of
first-fruits after the ingathering of the crops, because at such seasons
they had most leisure.

So then it appears that all the instances of Communion are parts of the
great Social one: and corresponding Friendships will follow upon such
Communions.


X

Of Political Constitutions there are three kinds; and equal in number
are the deflections from them, being, so to say, corruptions of them.

The former are Kingship, Aristocracy, and that which recognises the
principle of wealth, which it seems appropriate to call Timocracy (I
give to it the name of a political constitution because people commonly
do so). Of these the best is Monarchy, and Timocracy the worst.

[Sidenote: II6ob] From Monarchy the deflection is Despotism; both being
Monarchies but widely differing from each other; for the Despot looks to
his own advantage, but the King to that of his subjects: for he is in
fact no King who is not thoroughly independent and superior to the rest
in all good things, and he that is this has no further wants: he will
not then have to look to his own advantage but to that of his subjects,
for he that is not in such a position is a mere King elected by lot for
the nonce.

But Despotism is on a contrary footing to this Kingship, because the
Despot pursues his own good: and in the case of this its inferiority
is most evident, and what is worse is contrary to what is best. The
Transition to Despotism is made from Kingship, Despotism being a corrupt
form of Monarchy, that is to say, the bad King comes to be a Despot.

From Aristocracy to Oligarchy the transition is made by the fault of the
Rulers in distributing the public property contrary to right proportion;
and giving either all that is good, or the greatest share, to
themselves; and the offices to the same persons always, making wealth
their idol; thus a few bear rule and they bad men in the place of the
best.

From Timocracy the transition is to Democracy, they being contiguous:
for it is the nature of Timocracy to be in the hands of a multitude,
and all in the same grade of property are equal. Democracy is the least
vicious of all, since herein the form of the constitution undergoes
least change.

Well, these are generally the changes to which the various Constitutions
are liable, being the least in degree and the easiest to make.

Likenesses, and, as it were, models of them, one may find even in
Domestic life: for instance, the Communion between a Father and his Sons
presents the figure of Kingship, because the children are the Father's
care: and hence Homer names Jupiter Father because Kingship is intended
to be a paternal rule. Among the Persians, however, the Father's rule is
Despotic, for they treat their Sons as slaves. (The relation of Master
to Slaves is of the nature of Despotism because the point regarded
herein is the Master's interest): this now strikes me to be as it ought,
but the Persian custom to be mistaken; because for different persons
there should be different rules. [Sidenote: 1161a] Between Husband and
Wife the relation takes the form of Aristocracy, because he rules by
right and in such points only as the Husband should, and gives to
the Wife all that befits her to have. Where the Husband lords it in
everything he changes the relation into an Oligarchy; because he does
it contrary to right and not as being the better of the two. In some
instances the Wives take the reins of government, being heiresses: here
the rule is carried on not in right of goodness but by reason of wealth
and power, as it is in Oligarchies.

Timocracy finds its type in the relation of Brothers: they being equal
except as to such differences as age introduces: for which reason, if
they are very different in age, the Friendship comes to be no longer
a fraternal one: while Democracy is represented specially by families
which have no head (all being there equal), or in which the proper head
is weak and so every member does that which is right in his own eyes.


XI

Attendant then on each form of Political Constitution there plainly is
Friendship exactly co-extensive with the principle of Justice; that
between a King and his Subjects being in the relation of a superiority
of benefit, inasmuch as he benefits his subjects; it being assumed that
he is a good king and takes care of their welfare as a shepherd tends
his flock; whence Homer (to quote him again) calls Agamemnon, "shepherd
of the people." And of this same kind is the Paternal Friendship, only
that it exceeds the former in the greatness of the benefits done;
because the father is the author of being (which is esteemed the
greatest benefit) and of maintenance and education (these things are
also, by the way, ascribed to ancestors generally): and by the law of
nature the father has the right of rule over his sons, ancestors over
their descendants, and the king over his subjects.

These friendships are also between superiors and inferiors, for which
reason parents are not merely loved but also honoured. The principle of
Justice also between these parties is not exactly the same but according
to proportiton, because so also is the Friendship.

Now between Husband and Wife there is the same Friendship as in
Aristocracy: for the relation is determined by relative excellence, and
the better person has the greater good and each has what befits: so too
also is the principle of Justice between them.

The Fraternal Friendship is like that of Companions, because brothers
are equal and much of an age, and such persons have generally like
feelings and like dispositions. Like to this also is the Friendship of a
Timocracy, because the citizens are intended to be equal and equitable:
rule, therefore, passes from hand to hand, and is distributed on equal
terms: so too is the Friendship accordingly.

[Sidenote: 1161b] In the deflections from the constitutional forms, just
as the principle of Justice is but small so is the Friendship also: and
least of all in the most perverted form: in Despotism there is little
or no Friendship. For generally wherever the ruler and the ruled have
nothing in common there is no Friendship because there is no Justice;
but the case is as between an artisan and his tool, or between soul and
body, and master and slave; all these are benefited by those who use
them, but towards things inanimate there is neither Friendship nor
Justice: nor even towards a horse or an ox, or a slave _quâ_ slave,
because there is nothing in common: a slave as such is an animate tool,
a tool an inanimate slave. _Quâ_ slave, then, there is no Friendship
towards him, only _quâ_ man: for it is thought that there is some
principle of Justice between every man, and every other who can share in
law and be a party to an agreement; and so somewhat of Friendship, in so
far as he is man. So in Despotisms the Friendships and the principle of
Justice are inconsiderable in extent, but in Democracies they are most
considerable because they who are equal have much in common.

XII


Now of course all Friendship is based upon Communion, as has been
already stated: but one would be inclined to separate off from the rest
the Friendship of Kindred, and that of Companions: whereas those of men
of the same city, or tribe, or crew, and all such, are more peculiarly,
it would seem, based upon Communion, inasmuch as they plainly exist in
right of some agreement expressed or implied: among these one may rank
also the Friendship of Hospitality,

The Friendship of Kindred is likewise of many kinds, and appears in all
its varieties to depend on the Parental: parents, I mean, love their
children as being a part of themselves, children love their parents as
being themselves somewhat derived from them. But parents know their
offspring more than these know that they are from the parents, and the
source is more closely bound to that which is produced than that which
is produced is to that which formed it: of course, whatever is derived
from one's self is proper to that from which it is so derived (as, for
instance, a tooth or a hair, or any other thing whatever to him that
has it): but the source to it is in no degree proper, or in an inferior
degree at least.

Then again the greater length of time comes in: the parents love their
offspring from the first moment of their being, but their offspring
them only after a lapse of time when they have attained intelligence
or instinct. These considerations serve also to show why mothers have
greater strength of affection than fathers.

Now parents love their children as themselves (since what is derived
from themselves becomes a kind of other Self by the fact of separation),
but children their parents as being sprung from them. And brothers love
one another from being sprung from the same; that is, their sameness
with the common stock creates a sameness with one another; whence come
the phrases, "same blood," "root," and so on. In fact they are the same,
in a sense, even in the separate distinct individuals.

Then again the being brought up together, and the nearness of age, are
a great help towards Friendship, for a man likes one of his own age and
persons who are used to one another are companions, which accounts
for the resemblance between the Friendship of Brothers and that of
Companions.

[Sidenote:1162a] And cousins and all other relatives derive their bond
of union from these, that is to say, from their community of origin: and
the strength of this bond varies according to their respective distances
from the common ancestor.

Further: the Friendship felt by children towards parents, and by men
towards the gods, is as towards something good and above them; because
these have conferred the greatest possible benefits, in that they are
the causes of their being and being nourished, and of their having been
educated after they were brought into being.

And Friendship of this kind has also the pleasurable and the profitable
more than that between persons unconnected by blood, in proportion as
their life is also more shared in common. Then again in the Fraternal
Friendship there is all that there is in that of Companions, and more in
the good, and generally in those who are alike; in proportion as they
are more closely tied and from their very birth have a feeling of
affection for one another to begin with, and as they are more like in
disposition who spring from the same stock and have grown up together
and been educated alike: and besides this they have the greatest
opportunities in respect of time for proving one another, and can
therefore depend most securely upon the trial. The elements
of Friendship between other consanguinities will be of course
proportionably similar.

Between Husband and Wife there is thought to be Friendship by a law of
nature: man being by nature disposed to pair, more than to associate in
Communities: in proportion as the family is prior in order of time and
more absolutely necessary than the Community. And procreation is more
common to him with other animals; all the other animals have Communion
thus far, but human creatures cohabit not merely for the sake of
procreation but also with a view to life in general: because in this
connection the works are immediately divided, and some belong to the
man, others to the woman: thus they help one the other, putting what is
peculiar to each into the common stock.

And for these reasons this Friendship is thought to combine the
profitable and the pleasurable: it will be also based upon virtue if
they are good people; because each has goodness and they may take
delight in this quality in each other. Children too are thought to be a
tie: accordingly the childless sooner separate, for the children are a
good common to both and anything in common is a bond of union.

The question how a man is to live with his wife, or (more generally) one
friend with another, appears to be no other than this, how it is just
that they should: because plainly there is not the same principle
of Justice between a friend and friend, as between strangers, or
companions, or mere chance fellow-travellers.

XIII

[Sidenote:1162b] There are then, as was stated at the commencement of
this book, three kinds of Friendship, and in each there may be friends
on a footing of equality and friends in the relation of superior and
inferior; we find, I mean, that people who are alike in goodness, become
friends, and better with worse, and so also pleasant people; again,
because of advantage people are friends, either balancing exactly their
mutual profitableness or differing from one another herein. Well then,
those who are equal should in right of this equality be equalised also
by the degree of their Friendship and the other points, and those who
are on a footing of inequality by rendering Friendship in proportion to
the superiority of the other party.

Fault-finding and blame arises, either solely or most naturally, in
Friendship of which utility is the motive: for they who are friends by
reason of goodness, are eager to do kindnesses to one another because
this is a natural result of goodness and Friendship; and when men are
vying with each other for this End there can be no fault-finding nor
contention: since no one is annoyed at one who entertains for him the
sentiment of Friendship and does kindnesses to him, but if of a refined
mind he requites him with kind actions. And suppose that one of the two
exceeds the other, yet as he is attaining his object he will not find
fault with his friend, for good is the object of each party.

Neither can there well be quarrels between men who are friends for
pleasure's sake: because supposing them to delight in living together
then both attain their desire; or if not a man would be put in a
ridiculous light who should find fault with another for not pleasing
him, since it is in his power to forbear intercourse with him. But
the Friendship because of advantage is very liable to fault-finding;
because, as the parties use one another with a view to advantage, the
requirements are continually enlarging, and they think they have less
than of right belongs to them, and find fault because though justly
entitled they do not get as much as they want: while they who do the
kindnesses, can never come up to the requirements of those to whom they
are being done.

It seems also, that as the Just is of two kinds, the unwritten and the
legal, so Friendship because of advantage is of two kinds, what may
be called the Moral, and the Legal: and the most fruitful source of
complaints is that parties contract obligations and discharge them not
in the same line of Friendship. The Legal is upon specified conditions,
either purely tradesmanlike from hand to hand or somewhat more
gentlemanly as regards time but still by agreement a _quid pro quo_.

In this Legal kind the obligation is clear and admits of no dispute, the
friendly element is the delay in requiring its discharge: and for this
reason in some countries no actions can be maintained at Law for the
recovery of such debts, it being held that they who have dealt on the
footing of credit must be content to abide the issue.

That which may be termed the Moral kind is not upon specified
conditions, but a man gives as to his friend and so on: but still he
expects to receive an equivalent, or even more, as though he had not
given but lent: he also will find fault, because he does not get the
obligation discharged in the same way as it was contracted.

[Sidenote:1163a] Now this results from the fact, that all men, or the
generality at least, _wish_ what is honourable, but, when tested,
_choose_ what is profitable; and the doing kindnesses disinterestedly
is honourable while receiving benefits is profitable. In such cases one
should, if able, make a return proportionate to the good received, and
do so willingly, because one ought not to make a disinterested friend of
a man against his inclination: one should act, I say, as having made a
mistake originally in receiving kindness from one from whom one ought
not to have received it, he being not a friend nor doing the act
disinterestedly; one should therefore discharge one's self of the
obligation as having received a kindness on specified terms: and if able
a man would engage to repay the kindness, while if he were unable even
the doer of it would not expect it of him: so that if he is able he
ought to repay it. But one ought at the first to ascertain from whom
one is receiving kindness, and on what understanding, that on that same
understanding one may accept it or not.

A question admitting of dispute is whether one is to measure a kindness
by the good done to the receiver of it, and make this the standard by
which to requite, or by the kind intention of the doer?

For they who have received kindnesses frequently plead in depreciation
that they have received from their benefactors such things as were small
for them to give, or such as they themselves could have got from others:
while the doers of the kindnesses affirm that they gave the best they
had, and what could not have been got from others, and under danger, or
in such-like straits.

May we not say, that as utility is the motive of the Friendship the
advantage conferred on the receiver must be the standard? because he it
is who requests the kindness and the other serves him in his need on the
understanding that he is to get an equivalent: the assistance rendered
is then exactly proportionate to the advantage which the receiver has
obtained, and he should therefore repay as much as he gained by it, or
even more, this being more creditable.

In Friendships based on goodness, the question, of course, is never
raised, but herein the motive of the doer seems to be the proper
standard, since virtue and moral character depend principally on motive.


XIV

Quarrels arise also in those Friendships in which the parties are
unequal because each party thinks himself entitled to the greater share,
and of course, when this happens, the Friendship is broken up.

The man who is better than the other thinks that having the greater
share pertains to him of right, for that more is always awarded to the
good man: and similarly the man who is more profitable to another than
that other to him: "one who is useless," they say, "ought not to share
equally, for it comes to a tax, and not a Friendship, unless the fruits
of the Friendship are reaped in proportion to the works done:" their
notion being, that as in a money partnership they who contribute more
receive more so should it be in Friendship likewise.

On the other hand, the needy man and the less virtuous advance the
opposite claim: they urge that "it is the very business of a good friend
to help those who are in need, else what is the use of having a good or
powerful friend if one is not to reap the advantage at all?"

[Sidenote: 1163b] Now each seems to advance a right claim and to be
entitled to get more out of the connection than the other, only _not
more of the same thing_: but the superior man should receive more
respect, the needy man more profit: respect being the reward of goodness
and beneficence, profit being the aid of need.

This is plainly the principle acted upon in Political Communities:
he receives no honour who gives no good to the common stock: for the
property of the Public is given to him who does good to the Public, and
honour is the property of the Public; it is not possible both to make
money out of the Public and receive honour likewise; because no one will
put up with the less in every respect: so to him who suffers loss as
regards money they award honour, but money to him who can be paid by
gifts: since, as has been stated before, the observing due proportion
equalises and preserves Friendship.

Like rules then should be observed in the intercourse of friends who
are unequal; and to him who advantages another in respect of money, or
goodness, that other should repay honour, making requital according to
his power; because Friendship requires what is possible, not what is
strictly due, this being not possible in all cases, as in the honours
paid to the gods and to parents: no man could ever make the due return
in these cases, and so he is thought to be a good man who pays respect
according to his ability.

For this reason it may be judged never to be allowable for a son to
disown his father, whereas a father may his son: because he that owes
is bound to pay; now a son can never, by anything he has done, fully
requite the benefits first conferred on him by his father, and so is
always a debtor. But they to whom anything is owed may cast off their
debtors: therefore the father may his son. But at the same time it must
perhaps be admitted, that it seems no father ever _would_ sever himself
utterly from a son, except in a case of exceeding depravity: because,
independently of the natural Friendship, it is like human nature not to
put away from one's self the assistance which a son might render. But to
the son, if depraved, assisting his father is a thing to be avoided, or
at least one which he will not be very anxious to do; most men
being willing enough to receive kindness, but averse to doing it as
unprofitable.

Let thus much suffice on these points.




BOOK IX


I

[Sidenote: 1164a] Well, in all the Friendships the parties to which are
dissimilar it is the proportionate which equalises and preserves the
Friendship, as has been already stated: I mean, in the Social Friendship
the cobbler, for instance, gets an equivalent for his shoes after a
certain rate; and the weaver, and all others in like manner. Now in
this case a common measure has been provided in money, and to this
accordingly all things are referred and by this are measured: but in
the Friendship of Love the complaint is sometimes from the lover that,
though he loves exceedingly, his love is not requited; he having perhaps
all the time nothing that can be the object of Friendship: again,
oftentimes from the object of love that he who as a suitor promised any
and every thing now performs nothing. These cases occur because the
Friendship of the lover for the beloved object is based upon pleasure,
that of the other for him upon utility, and in one of the parties the
requisite quality is not found: for, as these are respectively the
grounds of the Friendship, the Friendship comes to be broken up because
the motives to it cease to exist: the parties loved not one another but
qualities in one another which are not permanent, and so neither are the
Friendships: whereas the Friendship based upon the moral character of
the parties, being independent and disinterested, is permanent, as we
have already stated.

Quarrels arise also when the parties realise different results and not
those which they desire; for the not attaining one's special object is
all one, in this case, with getting nothing at all: as in the well-known
case where a man made promises to a musician, rising in proportion to
the excellence of his music; but when, the next morning, the musician
claimed the performance of his promises, he said that he had given him
pleasure for pleasure: of course, if each party had intended this, it
would have been all right: but if the one desires amusement and the
other gain, and the one gets his object but the other not, the dealing
cannot be fair: because a man fixes his mind upon what he happens to
want, and will give so and so for that specific thing.

The question then arises, who is to fix the rate? the man who first
gives, or the man who first takes? because, _prima facie_, the man who
first gives seems to leave the rate to be fixed by the other party.
This, they say, was in fact the practice of Protagoras: when he taught
a man anything he would bid the learner estimate the worth of the
knowledge gained by his own private opinion; and then he used to take so
much from him. In such cases some people adopt the rule,

  "With specified reward a friend should be content."

They are certainly fairly found fault with who take the money in advance
and then do nothing of what they said they would do, their promises
having been so far beyond their ability; for such men do not perform
what they agreed, The Sophists, however, are perhaps obliged to take
this course, because no one would give a sixpence for their knowledge.
These then, I say, are fairly found fault with, because they do not what
they have already taken money for doing.

[Sidenote: 1164b] In cases where no stipulation as to the respective
services is made they who disinterestedly do the first service will not
raise the question (as we have said before), because it is the nature of
Friendship, based on mutual goodness to be reference to the intention of
the other, the intention being characteristic of the true friend and of
goodness.

And it would seem the same rule should be laid down for those who are
connected with one another as teachers and learners of philosophy; for
here the value of the commodity cannot be measured by money, and, in
fact, an exactly equivalent price cannot be set upon it, but perhaps it
is sufficient to do what one can, as in the case of the gods or one's
parents.

But where the original giving is not upon these terms but avowedly for
some return, the most proper course is perhaps for the requital to be
such as _both_ shall allow to be proportionate, and, where this cannot
be, then for the receiver to fix the value would seem to be not only
necessary but also fair: because when the first giver gets that which is
equivalent to the advantage received by the other, or to what he would
have given to secure the pleasure he has had, then he has the value from
him: for not only is this seen to be the course adopted in matters of
buying and selling but also in some places the law does not allow of
actions upon voluntary dealings; on the principle that when one man has
trusted another he must be content to have the obligation discharged in
the same spirit as he originally contracted it: that is to say, it is
thought fairer for the trusted, than for the trusting, party, to fix the
value. For, in general, those who have and those who wish to get things
do not set the same value on them: what is their own, and what they give
in each case, appears to them worth a great deal: but yet the return
is made according to the estimate of those who have received first, it
should perhaps be added that the receiver should estimate what he has
received, not by the value he sets upon it now that he has it, but by
that which he set upon it before he obtained it.


II

Questions also arise upon such points as the following: Whether one's
father has an unlimited claim on one's services and obedience, or
whether the sick man is to obey his physician? or, in an election of
a general, the warlike qualities of the candidates should be alone
regarded?

In like manner whether one should do a service rather to one's friend or
to a good man? whether one should rather requite a benefactor or give to
one's companion, supposing that both are not within one's power?

[Sidenote: 1165a] Is not the true answer that it is no easy task to
determine all such questions accurately, inasmuch as they involve
numerous differences of all kinds, in respect of amount and what is
honourable and what is necessary? It is obvious, of course, that no one
person can unite in himself all claims. Again, the requital of benefits
is, in general, a higher duty than doing unsolicited kindnesses to one's
companion; in other words, the discharging of a debt is more obligatory
upon one than the duty of giving to a companion. And yet this rule may
admit of exceptions; for instance, which is the higher duty? for one who
has been ransomed out of the hands of robbers to ransom in return his
ransomer, be he who he may, or to repay him on his demand though he has
not been taken by robbers, or to ransom his own father? for it would
seem that a man ought to ransom his father even in preference to
himself.

Well then, as has been said already, as a general rule the debt
should be discharged, but if in a particular case the giving greatly
preponderates as being either honourable or necessary, we must be swayed
by these considerations: I mean, in some cases the requital of the
obligation previously existing may not be equal; suppose, for instance,
that the original benefactor has conferred a kindness on a good man,
knowing him to be such, whereas this said good man has to repay it
believing him to be a scoundrel.

And again, in certain cases no obligation lies on a man to lend to one
who has lent to him; suppose, for instance, that a bad man lent to him,
as being a good man, under the notion that he should get repaid, whereas
the said good man has no hope of repayment from him being a bad man.
Either then the case is really as we have supposed it and then the claim
is not equal, or it is not so but supposed to be; and still in so acting
people are not to be thought to act wrongly. In short, as has been
oftentimes stated before, all statements regarding feelings and actions
can be definite only in proportion as their object-matter is so; it is
of course quite obvious that all people have not the same claim upon
one, nor are the claims of one's father unlimited; just as Jupiter does
not claim all kinds of sacrifice without distinction: and since the
claims of parents, brothers, companions, and benefactors, are all
different, we must give to each what belongs to and befits each.

And this is seen to be the course commonly pursued: to marriages men
commonly invite their relatives, because these are from a common stock
and therefore all the actions in any way pertaining thereto are common
also: and to funerals men think that relatives ought to assemble in
preference to other people, for the same reason.

And it would seem that in respect of maintenance it is our duty to
assist our parents in preference to all others, as being their debtors,
and because it is more honourable to succour in these respects the
authors of our existence than ourselves. Honour likewise we ought to pay
to our parents just as to the gods, but then, not all kinds of honour:
not the same, for instance, to a father as to a mother: nor again to a
father the honour due to a scientific man or to a general but that
which is a father's due, and in like manner to a mother that which is a
mother's.

To all our elders also the honour befitting their age, by rising up in
their presence, turning out of the way for them, and all similar marks
of respect: to our companions again, or brothers, frankness and free
participation in all we have. And to those of the same family, or tribe,
or city, with ourselves, and all similarly connected with us, we should
constantly try to render their due, and to discriminate what belongs to
each in respect of nearness of connection, or goodness, or intimacy:
of course in the case of those of the same class the discrimination is
easier; in that of those who are in different classes it is a matter of
more trouble. This, however, should not be a reason for giving up
the attempt, but we must observe the distinctions so far as it is
practicable to do so.

III

A question is also raised as to the propriety of dissolving or not
dissolving those Friendships the parties to which do not remain what
they were when the connection was formed.

[Sidenote: 1165b] Now surely in respect of those whose motive to
Friendship is utility or pleasure there can be nothing wrong in breaking
up the connection when they no longer have those qualities; because they
were friends [not of one another, but] of those qualities: and, these
having failed, it is only reasonable to expect that they should cease to
entertain the sentiment.

But a man has reason to find fault if the other party, being really
attached to him because of advantage or pleasure, pretended to be so
because of his moral character: in fact, as we said at the commencement,
the most common source of quarrels between friends is their not being
friends on the same grounds as they suppose themselves to be.

Now when a man has been deceived in having supposed himself to excite
the sentiment of Friendship by reason of his moral character, the other
party doing nothing to indicate he has but himself to blame: but when he
has been deceived by the pretence of the other he has a right to find
fault with the man who has so deceived him, aye even more than with
utterers of false coin, in proportion to the greater preciousness of
that which is the object-matter of the villany.

But suppose a man takes up another as being a good man, who turns out,
and is found by him, to be a scoundrel, is he bound still to entertain
Friendship for him? or may we not say at once it is impossible? since
it is not everything which is the object-matter of Friendship, but only
that which is good; and so there is no obligation to be a bad man's
friend, nor, in fact, ought one to be such: for one ought not to be a
lover of evil, nor to be assimilated to what is base; which would be
implied, because we have said before, like is friendly to like.

Are we then to break with him instantly? not in all cases; only where
our friends are incurably depraved; when there is a chance of amendment
we are bound to aid in repairing the moral character of our friends
even more than their substance, in proportion as it is better and
more closely related to Friendship. Still he who should break off the
connection is not to be judged to act wrongly, for he never was a friend
to such a character as the other now is, and therefore, since the man is
changed and he cannot reduce him to his original state, he backs out of
the connection.

To put another case: suppose that one party remains what he was when
the Friendship was formed, while the other becomes morally improved and
widely different from his friend in goodness; is the improved character
to treat the other as a friend?

May we not say it is impossible? The case of course is clearest where
there is a great difference, as in the Friendships of boys: for suppose
that of two boyish friends the one still continues a boy in mind and the
other becomes a man of the highest character, how can they be friends?
since they neither are pleased with the same objects nor like and
dislike the same things: for these points will not belong to them as
regards one another, and without them it was assumed they cannot be
friends because they cannot live in intimacy: and of the case of those
who cannot do so we have spoken before.

Well then, is the improved party to bear himself towards his former
friend in no way differently to what he would have done had the
connection never existed?

Surely he ought to bear in mind the intimacy of past times, and just as
we think ourselves bound to do favours for our friends in preference to
strangers, so to those who have been friends and are so no longer we
should allow somewhat on the score of previous Friendship, whenever the
cause of severance is not excessive depravity on their part.




IV

[Sidenote: II66a] Now the friendly feelings which are exhibited towards
our friends, and by which Friendships are characterised, seem to have
sprung out of those which we entertain toward ourselves. I mean, people
define a friend to be "one who intends and does what is good (or what
he believes to be good) to another for that other's sake," or "one who
wishes his friend to be and to live for that friend's own sake" (which
is the feeling of mothers towards their children, and of friends who
have come into collision). Others again, "one who lives with another and
chooses the same objects," or "one who sympathises with his friend in
his sorrows and in his joys" (this too is especially the case with
mothers).

Well, by some one of these marks people generally characterise
Friendship: and each of these the good man has towards himself, and all
others have them in so far as they suppose themselves to be good. (For,
as has been said before, goodness, that is the good man, seems to be a
measure to every one else.)

For he is at unity in himself, and with every part of his soul he
desires the same objects; and he wishes for himself both what is, and
what he believes to be, good; and he does it (it being characteristic
of the good man to work at what is good), and for the sake of himself,
inasmuch as he does it for the sake of his Intellectual Principle which
is generally thought to be a man's Self. Again, he wishes himself And
specially this Principle whereby he is an intelligent being, to live and
be preserved in life, because existence is a good to him that is a good
man.

But it is to himself that each individual wishes what is good, and no
man, conceiving the possibility of his becoming other than he now is,
chooses that that New Self should have all things indiscriminately: a
god, for instance, has at the present moment the Chief Good, but he has
it in right of being whatever he actually now is: and the Intelligent
Principle must be judged to be each man's Self, or at least eminently so
[though other Principles help, of course, to constitute him the man he
is]. Furthermore, the good man wishes to continue to live with himself;
for he can do it with pleasure, in that his memories of past actions are
full of delight and his anticipations of the future are good and such
are pleasurable. Then, again, he has good store of matter for his
Intellect to contemplate, and he most especially sympathises with his
Self in its griefs and joys, because the objects which give him pain and
pleasure are at all times the same, not one thing to-day and a different
one to-morrow: because he is not given to repentance, if one may so
speak. It is then because each of these feelings are entertained by the
good man towards his own Self and a friend feels towards a friend as
towards himself (a friend being in fact another Self), that Friendship
is thought to be some one of these things and they are accounted friends
in whom they are found. Whether or no there can really be Friendship
between a man and his Self is a question we will not at present
entertain: there may be thought to be Friendship, in so far as there are
two or more of the aforesaid requisites, and because the highest degree
of Friendship, in the usual acceptation of that term, resembles the
feeling entertained by a man towards himself.

[Sidenote: 1166b] But it may be urged that the aforesaid requisites are
to all appearance found in the common run of men, though they are men of
a low stamp.

May it not be answered, that they share in them only in so far as they
please themselves, and conceive themselves to be good? for certainly,
they are not either really, or even apparently, found in any one of
those who are very depraved and villainous; we may almost say not
even in those who are bad men at all: for they are at variance with
themselves and lust after different things from those which in cool
reason they wish for, just as men who fail of Self-Control: I mean, they
choose things which, though hurtful, are pleasurable, in preference to
those which in their own minds they believe to be good: others again,
from cowardice and indolence, decline to do what still they are
convinced is best for them: while they who from their depravity have
actually done many dreadful actions hate and avoid life, and accordingly
kill themselves: and the wicked seek others in whose company to spend
their time, but fly from themselves because they have many unpleasant
subjects of memory, and can only look forward to others like them when
in solitude but drown their remorse in the company of others: and as
they have nothing to raise the sentiment of Friendship so they never
feel it towards themselves.

Neither, in fact, can they who are of this character sympathise with
their Selves in their joys and sorrows, because their soul is, as it
were, rent by faction, and the one principle, by reason of the depravity
in them, is grieved at abstaining from certain things, while the other
and better principle is pleased thereat; and the one drags them this way
and the other that way, as though actually tearing them asunder. And
though it is impossible actually to have at the same time the sensations
of pain and pleasure; yet after a little time the man is sorry for
having been pleased, and he could wish that those objects had not given
him pleasure; for the wicked are full of remorse.

It is plain then that the wicked man cannot be in the position of a
friend even towards himself, because he has in himself nothing which can
excite the sentiment of Friendship. If then to be thus is exceedingly
wretched it is a man's duty to flee from wickedness with all his might
and to strive to be good, because thus may he be friends with himself
and may come to be a friend to another.

[Sidenote: V] Kindly Feeling, though resembling Friendship, is not
identical with it, because it may exist in reference to those whom we
do not know and without the object of it being aware of its existence,
which Friendship cannot. (This, by the way, has also been said before.)
And further, it is not even Affection because it does not imply
intensity nor yearning, which are both consequences of Affection. Again
Affection requires intimacy but Kindly Feeling may arise quite suddenly,
as happens sometimes in respect of men against whom people are matched
in any way, I mean they come to be kindly disposed to them and
sympathise in their wishes, but still they would not join them in any
action, because, as we said, they conceive this feeling of kindness
suddenly and so have but a superficial liking.

What it does seem to be is the starting point of a Friendship; just as
pleasure, received through the sight, is the commencement of Love: for
no one falls in love without being first pleased with the personal
appearance of the beloved object, and yet he who takes pleasure in it
does not therefore necessarily love, but when he wearies for the object
in its absence and desires its presence. Exactly in the same way men
cannot be friends without having passed through the stage of Kindly
Feeling, and yet they who are in that stage do not necessarily advance
to Friendship: they merely have an inert wish for the good of those
toward whom they entertain the feeling, but would not join them in
any action, nor put themselves out of the way for them. So that, in
a metaphorical way of speaking, one might say that it is dormant
Friendship, and when it has endured for a space and ripened into
intimacy comes to be real Friendship; but not that whose object is
advantage or pleasure, because such motives cannot produce even Kindly
Feeling.

I mean, he who has received a kindness requites it by Kindly Feeling
towards his benefactor, and is right in so doing: but he who wishes
another to be prosperous, because he has hope of advantage through his
instrumentality, does not seem to be kindly disposed to that person but
rather to himself; just as neither is he his friend if he pays court to
him for any interested purpose.

Kindly Feeling always arises by reason of goodness and a certain
amiability, when one man gives another the notion of being a fine
fellow, or brave man, etc., as we said was the case sometimes with those
matched against one another.

[Sidenote: VI] Unity of Sentiment is also plainly connected with
Friendship, and therefore is not the same as Unity of Opinion,
because this might exist even between people unacquainted with one
another.

Nor do men usually say people are united in sentiment merely because
they agree in opinion on _any_ point, as, for instance, on points
of astronomical science (Unity of Sentiment herein not having any
connection with Friendship), but they say that Communities have Unity of
Sentiment when they agree respecting points of expediency and take the
same line and carry out what has been determined in common consultation.

Thus we see that Unity of Sentiment has for its object matters of
action, and such of these as are of importance, and of mutual, or, in
the case of single States, common, interest: when, for instance, all
agree in the choice of magistrates, or forming alliance with the
Lacedæmonians, or appointing Pittacus ruler (that is to say, supposing
he himself was willing). [Sidenote: 1167_b_] But when each wishes
himself to be in power (as the brothers in the Phoenissæ), they quarrel
and form parties: for, plainly, Unity of Sentiment does not merely imply
that each entertains the same idea be it what it may, but that they do
so in respect of the same object, as when both the populace and the
sensible men of a State desire that the best men should be in office,
because then all attain their object.

Thus Unity of Sentiment is plainly a social Friendship, as it is also
said to be: since it has for its object-matter things expedient and
relating to life.

And this Unity exists among the good: for they have it towards
themselves and towards one another, being, if I may be allowed the
expression, in the same position: I mean, the wishes of such men are
steady and do not ebb and flow like the Euripus, and they wish what is
just and expedient and aim at these things in common.

The bad, on the contrary, can as little have Unity of Sentiment as they
can be real friends, except to a very slight extent, desiring as they
do unfair advantage in things profitable while they shirk labour and
service for the common good: and while each man wishes for these things
for himself he is jealous of and hinders his neighbour: and as they
do not watch over the common good it is lost. The result is that they
quarrel while they are for keeping one another to work but are not
willing to perform their just share.

[Sidenote: VII] Benefactors are commonly held to have more Friendship
for the objects of their kindness than these for them: and the fact
is made a subject of discussion and inquiry, as being contrary to
reasonable expectation.

The account of the matter which satisfies most persons is that the one
are debtors and the others creditors: and therefore that, as in the case
of actual loans the debtors wish their creditors out of the way while
the creditors are anxious for the preservation of their debtors, so
those who have done kindnesses desire the continued existence of the
people they have done them to, under the notion of getting a return
of their good offices, while these are not particularly anxious about
requital.

Epicharmus, I suspect, would very probably say that they who give this
solution judge from their own baseness; yet it certainly is like human
nature, for the generality of men have short memories on these points,
and aim rather at receiving than conferring benefits.

But the real cause, it would seem, rests upon nature, and the case is
not parallel to that of creditors; because in this there is no affection
to the persons, but merely a wish for their preservation with a view to
the return: whereas, in point of fact, they who have done kindnesses
feel friendship and love for those to whom they have done them, even
though they neither are, nor can by possibility hereafter be, in a
position to serve their benefactors.

[Sidenote: 1168_a_] And this is the case also with artisans; every one,
I mean, feels more affection for his own work than that work possibly
could for him if it were animate. It is perhaps specially the case with
poets: for these entertain very great affection for their poems, loving
them as their own children. It is to this kind of thing I should be
inclined to compare the case of benefactors: for the object of their
kindness is their own work, and so they love this more than this loves
its creator.

And the account of this is that existence is to all a thing choiceworthy
and an object of affection; now we exist by acts of working, that is, by
living and acting; he then that has created a given work exists, it may
be said, by his act of working: therefore he loves his work because he
loves existence. And this is natural, for the work produced displays in
act what existed before potentially.

Then again, the benefactor has a sense of honour in right of his action,
so that he may well take pleasure in him in whom this resides; but to
him who has received the benefit there is nothing honourable in respect
of his benefactor, only something advantageous which is both less
pleasant and less the object of Friendship.

Again, pleasure is derived from the actual working out of a present
action, from the anticipation of a future one, and from the recollection
of a past one: but the highest pleasure and special object of affection
is that which attends on the actual working. Now the benefactor's work
abides (for the honourable is enduring), but the advantage of him who
has received the kindness passes away.

Again, there is pleasure in recollecting honourable actions, but in
recollecting advantageous ones there is none at all or much less (by the
way though, the contrary is true of the expectation of advantage).

Further, the entertaining the feeling of Friendship is like acting on
another; but being the object of the feeling is like being acted upon.

So then, entertaining the sentiment of Friendship, and all feelings
connected with it, attend on those who, in the given case of a
benefaction, are the superior party.

Once more: all people value most what has cost them much labour in the
production; for instance, people who have themselves made their money
are fonder of it than those who have inherited it: and receiving
kindness is, it seems, unlaborious, but doing it is laborious. And this
is the reason why the female parents are most fond of their offspring;
for their part in producing them is attended with most labour, and they
know more certainly that they are theirs. This feeling would seem also
to belong to benefactors.

[Sidenote: VIII] A question is also raised as to whether it is right
to love one's Self best, or some one else: because men find fault with
those who love themselves best, and call them in a disparaging way
lovers of Self; and the bad man is thought to do everything he does
for his own sake merely, and the more so the more depraved he is;
accordingly men reproach him with never doing anything unselfish:
whereas the good man acts from a sense of honour (and the more so the
better man he is), and for his friend's sake, and is careless of his own
interest.

[Sidenote: 1168_b_] But with these theories facts are at variance, and
not unnaturally: for it is commonly said also that a man is to love most
him who is most his friend, and he is most a friend who wishes good to
him to whom he wishes it for that man's sake even though no one knows.
Now these conditions, and in fact all the rest by which a friend is
characterised, belong specially to each individual in respect of his
Self: for we have said before that all the friendly feelings are derived
to others from those which have Self primarily for their object. And all
the current proverbs support this view; for instance, "one soul," "the
goods of friends are common," "equality is a tie of Friendship," "the
knee is nearer than the shin." For all these things exist specially with
reference to a man's own Self: he is specially a friend to himself and
so he is bound to love himself the most.

It is with good reason questioned which of the two parties one should
follow, both having plausibility on their side. Perhaps then, in respect
of theories of this kind, the proper course is to distinguish and define
how far each is true, and in what way. If we could ascertain the sense
in which each uses the term "Self-loving," this point might be cleared
up.

Well now, they who use it disparagingly give the name to those who,
in respect of wealth, and honours, and pleasures of the body, give to
themselves the larger share: because the mass of mankind grasp after
these and are earnest about them as being the best things; which is the
reason why they are matters of contention. They who are covetous in
regard to these gratify their lusts and passions in general, that is to
say the irrational part of their soul: now the mass of mankind are so
disposed, for which reason the appellation has taken its rise from that
mass which is low and bad. Of course they are justly reproached who are
Self-loving in this sense.

And that the generality of men are accustomed to apply the term to
denominate those who do give such things to themselves is quite plain:
suppose, for instance, that a man were anxious to do, more than other
men, acts of justice, or self-mastery, or any other virtuous acts, and,
in general, were to secure to himself that which is abstractedly noble
and honourable, no one would call him Self-loving, nor blame him.

Yet might such an one be judged to be more truly Self-loving: certainly
he gives to himself the things which are most noble and most good,
and gratifies that Principle of his nature which is most rightfully
authoritative, and obeys it in everything: and just as that which
possesses the highest authority is thought to constitute a Community or
any other system, so also in the case of Man: and so he is most truly
Self-loving who loves and gratifies this Principle.

Again, men are said to have, or to fail of having, self-control,
according as the Intellect controls or not, it being plainly implied
thereby that this Principle constitutes each individual; and people are
thought to have done of themselves, and voluntarily, those things
specially which are done with Reason. [Sidenote: 1169_a_]

It is plain, therefore, that this Principle does, either entirely or
specially constitute the individual man, and that the good man specially
loves this. For this reason then he must be specially Self-loving, in a
kind other than that which is reproached, and as far superior to it as
living in accordance with Reason is to living at the beck and call of
passion, and aiming at the truly noble to aiming at apparent advantage.

Now all approve and commend those who are eminently earnest about
honourable actions, and if all would vie with one another in respect of
the [Greek: kalhon], and be intent upon doing what is most truly noble
and honourable, society at large would have all that is proper while
each individual in particular would have the greatest of goods, Virtue
being assumed to be such.

And so the good man ought to be Self-loving: because by doing what is
noble he will have advantage himself and will do good to others: but the
bad man ought not to be, because he will harm himself and his neighbours
by following low and evil passions. In the case of the bad man, what he
ought to do and what he does are at variance, but the good man does what
he ought to do, because all Intellect chooses what is best for itself
and the good man puts himself under the direction of Intellect.

Of the good man it is true likewise that he does many things for the
sake of his friends and his country, even to the extent of dying for
them, if need be: for money and honours, and, in short, all the good
things which others fight for, he will throw away while eager to secure
to himself the [Greek: kalhon]: he will prefer a brief and great joy
to a tame and enduring one, and to live nobly for one year rather than
ordinarily for many, and one great and noble action to many trifling
ones. And this is perhaps that which befals men who die for their
country and friends; they choose great glory for themselves: and they
will lavish their own money that their friends may receive more, for
hereby the friend gets the money but the man himself the [Greek:
kalhon]; so, in fact he gives to himself the greater good. It is the
same with honours and offices; all these things he will give up to his
friend, because this reflects honour and praise on himself: and so
with good reason is he esteemed a fine character since he chooses the
honourable before all things else. It is possible also to give up the
opportunities of action to a friend; and to have caused a friend's doing
a thing may be more noble than having done it one's self.

In short, in all praiseworthy things the good man does plainly give to
himself a larger share of the honourable. [Sidenote: 1169_b_] In this
sense it is right to be Self-loving, in the vulgar acceptation of the
term it is not.

[Sidenote: IX] A question is raised also respecting the Happy man,
whether he will want Friends, or no?

Some say that they who are blessed and independent have no need of
Friends, for they already have all that is good, and so, as being
independent, want nothing further: whereas the notion of a friend's
office is to be as it were a second Self and procure for a man what he
cannot get by himself: hence the saying,

  "When Fortune gives us good, what need we Friends?"

On the other hand, it looks absurd, while we are assigning to the Happy
man all other good things, not to give him Friends, which are, after
all, thought to be the greatest of external goods.

Again, if it is more characteristic of a friend to confer than to
receive kindnesses, and if to be beneficent belongs to the good man and
to the character of virtue, and if it is more noble to confer kindnesses
on friends than strangers, the good man will need objects for his
benefactions. And out of this last consideration springs a question
whether the need of Friends be greater in prosperity or adversity, since
the unfortunate man wants people to do him kindnesses and they who are
fortunate want objects for their kind acts.

Again, it is perhaps absurd to make our Happy man a solitary, because
no man would choose the possession of all goods in the world on the
condition of solitariness, man being a social animal and formed by
nature for living with others: of course the Happy man has this
qualification since he has all those things which are good by nature:
and it is obvious that the society of friends and good men must be
preferable to that of strangers and ordinary people, and we conclude,
therefore, that the Happy man does need Friends.

But then, what do they mean whom we quoted first, and how are they
right? Is it not that the mass of mankind mean by Friends those who are
useful? and of course the Happy man will not need such because he has
all good things already; neither will he need such as are Friends with
a view to the pleasurable, or at least only to a slight extent; because
his life, being already pleasurable, does not want pleasure imported
from without; and so, since the Happy man does not need Friends of these
kinds, he is thought not to need any at all.

But it may be, this is not true: for it was stated originally, that
Happiness is a kind of Working; now Working plainly is something
that must come into being, not be already there like a mere piece of
property.

[Sidenote: 1170_a_] If then the being happy consists in living and
working, and the good man's working is in itself excellent and
pleasurable (as we said at the commencement of the treatise), and if
what is our own reckons among things pleasurable, and if we can view our
neighbours better than ourselves and their actions better than we
can our own, then the actions of their Friends who are good men are
pleasurable to the good; inasmuch as they have both the requisites which
are naturally pleasant. So the man in the highest state of happiness
will need Friends of this kind, since he desires to contemplate good
actions, and actions of his own, which those of his friend, being a good
man, are. Again, common opinion requires that the Happy man live with
pleasure to himself: now life is burthensome to a man in solitude, for
it is not easy to work continuously by one's self, but in company with,
and in regard to others, it is easier, and therefore the working, being
pleasurable in itself will be more continuous (a thing which should be
in respect of the Happy man); for the good man, in that he is good takes
pleasure in the actions which accord with Virtue and is annoyed at those
which spring from Vice, just as a musical man is pleased with beautiful
music and annoyed by bad. And besides, as Theognis says, Virtue itself
may be improved by practice, from living with the good.

And, upon the following considerations more purely metaphysical, it will
probably appear that the good friend is naturally choiceworthy to the
good man. We have said before, that whatever is naturally good is also
in itself good and pleasant to the good man; now the fact of living, so
far as animals are concerned, is characterised generally by the power
of sentience, in man it is characterised by that of sentience, or
of rationality (the faculty of course being referred to the actual
operation of the faculty, certainly the main point is the actual
operation of it); so that living seems mainly to consist in the act of
sentience or exerting rationality: now the fact of living is in itself
one of the things that are good and pleasant (for it is a definite
totality, and whatever is such belongs to the nature of good), but what
is naturally good is good to the good man: for which reason it seems
to be pleasant to all. (Of course one must not suppose a life which is
depraved and corrupted, nor one spent in pain, for that which is such is
indefinite as are its inherent qualities: however, what is to be said of
pain will be clearer in what is to follow.)

If then the fact of living is in itself good and pleasant (and this
appears from the fact that all desire it, and specially those who are
good and in high happiness; their course of life being most choiceworthy
and their existence most choiceworthy likewise), then also he that sees
perceives that he sees; and he that hears perceives that he hears; and
he that walks perceives that he walks; and in all the other instances
in like manner there is a faculty which reflects upon and perceives the
fact that we are working, so that we can perceive that we perceive and
intellectually know that we intellectually know: but to perceive that we
perceive or that we intellectually know is to perceive that we exist,
since existence was defined to be perceiving or intellectually knowing.
[Sidenote: 1170_b_ Now to perceive that one lives is a thing pleasant
in itself, life being a thing naturally good, and the perceiving of the
presence in ourselves of things naturally good being pleasant.]

Therefore the fact of living is choiceworthy, and to the good specially
so since existence is good and pleasant to them: for they receive
pleasure from the internal consciousness of that which in itself is
good.

But the good man is to his friend as to himself, friend being but a name
for a second Self; therefore as his own existence is choiceworthy to
each so too, or similarly at least, is his friend's existence. But the
ground of one's own existence being choiceworthy is the perceiving of
one's self being good, any such perception being in itself pleasant.
Therefore one ought to be thoroughly conscious of one's friend's
existence, which will result from living with him, that is sharing in
his words and thoughts: for this is the meaning of the term as applied
to the human species, not mere feeding together as in the case of
brutes.

If then to the man in a high state of happiness existence is in itself
choiceworthy, being naturally good and pleasant, and so too a friend's
existence, then the friend also must be among things choiceworthy. But
whatever is choiceworthy to a man he should have or else he will be in
this point deficient. The man therefore who is to come up to our notion
"Happy" will need good Friends. Are we then to make our friends as
numerous as possible? or, as in respect of acquaintance it is thought
to have been well said "have not thou many acquaintances yet be not
without;" so too in respect of Friendship may we adopt the precept, and
say that a man should not be without friends, nor again have exceeding
many friends?

Now as for friends who are intended for use, the maxim I have quoted
will, it seems, fit in exceedingly well, because to requite the services
of many is a matter of labour, and a whole life would not be long enough
to do this for them. So that, if more numerous than what will suffice
for one's own life, they become officious, and are hindrances in respect
of living well: and so we do not want them. And again of those who are
to be for pleasure a few are quite enough, just like sweetening in our
food.




X


But of the good are we to make as many as ever we can, or is there
any measure of the number of friends, as there is of the number to
constitute a Political Community? I mean, you cannot make one out of ten
men, and if you increase the number to one hundred thousand it is not
any longer a Community. However, the number is not perhaps some one
definite number but any between certain extreme limits.

[Sidenote: 1171_a_] Well, of friends likewise there is a limited number,
which perhaps may be laid down to be the greatest number with whom it
would be possible to keep up intimacy; this being thought to be one of
the greatest marks of Friendship, and it being quite obvious that it is
not possible to be intimate with many, in other words, to part one's
self among many. And besides it must be remembered that they also are to
be friends to one another if they are all to live together: but it is a
matter of difficulty to find this in many men at once.

It comes likewise to be difficult to bring home to one's self the joys
and sorrows of many: because in all probability one would have to
sympathise at the same time with the joys of this one and the sorrows of
that other.

Perhaps then it is well not to endeavour to have very many friends but
so many as are enough for intimacy: because, in fact, it would seem not
to be possible to be very much a friend to many at the same time: and,
for the same reason, not to be in love with many objects at the same
time: love being a kind of excessive Friendship which implies but one
object: and all strong emotions must be limited in the number towards
whom they are felt.

And if we look to facts this seems to be so: for not many at a time
become friends in the way of companionship, all the famous Friendships
of the kind are between _two_ persons: whereas they who have many
friends, and meet everybody on the footing of intimacy, seem to be
friends really to no one except in the way of general society; I mean
the characters denominated as over-complaisant.

To be sure, in the way merely of society, a man may be a friend to many
without being necessarily over-complaisant, but being truly good: but
one cannot be a friend to many because of their virtue, and for the
persons' own sake; in fact, it is a matter for contentment to find even
a few such.


XI

Again: are friends most needed in prosperity or in adversity? they are
required, we know, in both states, because the unfortunate need help and
the prosperous want people to live with and to do kindnesses to: for
they have a desire to act kindly to some one.

To have friends is more necessary in adversity, and therefore in this
case useful ones are wanted; and to have them in prosperity is more
honourable, and this is why the prosperous want good men for friends, it
being preferable to confer benefits on, and to live with, these. For the
very presence of friends is pleasant even in adversity: since men when
grieved are comforted by the sympathy of their friends.

And from this, by the way, the question might be raised, whether it is
that they do in a manner take part of the weight of calamities, or only
that their presence, being pleasurable, and the consciousness of their
sympathy, make the pain of the sufferer less. However, we will not
further discuss whether these which have been suggested or some other
causes produce the relief, at least the effect we speak of is a matter
of plain fact.

[Sidenote: _1171b_] But their presence has probably a mixed effect: I
mean, not only is the very seeing friends pleasant, especially to one in
misfortune, and actual help towards lessening the grief is afforded
(the natural tendency of a friend, if he is gifted with tact, being
to comfort by look and word, because he is well acquainted with the
sufferer's temper and disposition and therefore knows what things give
him pleasure and pain), but also the perceiving a friend to be grieved
at his misfortunes causes the sufferer pain, because every one avoids
being cause of pain to his friends. And for this reason they who are
of a manly nature are cautious not to implicate their friends in their
pain; and unless a man is exceedingly callous to the pain of others he
cannot bear the pain which is thus caused to his friends: in short, he
does not admit men to wail with him, not being given to wail at all:
women, it is true, and men who resemble women, like to have others to
groan with them, and love such as friends and sympathisers. But it
is plain that it is our duty in all things to imitate the highest
character.

On the other hand, the advantages of friends in our prosperity are the
pleasurable intercourse and the consciousness that they are pleased at
our good fortune.

It would seem, therefore, that we ought to call in friends readily on
occasion of good fortune, because it is noble to be ready to do good to
others: but on occasion of bad fortune, we should do so with reluctance;
for we should as little as possible make others share in our ills; on
which principle goes the saying, "I am unfortunate, let that suffice."
The most proper occasion for calling them in is when with small trouble
or annoyance to themselves they can be of very great use to the person
who needs them.

But, on the contrary, it is fitting perhaps to go to one's friends in
their misfortunes unasked and with alacrity (because kindness is the
friend's office and specially towards those who are in need and who do
not demand it as a right, this being more creditable and more pleasant
to both); and on occasion of their good fortune to go readily, if we
can forward it in any way (because men need their friends for this
likewise), but to be backward in sharing it, any great eagerness to
receive advantage not being creditable.

One should perhaps be cautious not to present the appearance of
sullenness in declining the sympathy or help of friends, for this
happens occasionally.

It appears then that the presence of friends is, under all
circumstances, choiceworthy.

May we not say then that, as seeing the beloved object is most prized by
lovers and they choose this sense rather than any of the others because
Love

  "Is engendered in the eyes,
  With gazing fed,"

in like manner intimacy is to friends most choiceworthy, Friendship
being communion? Again, as a man is to himself so is he to his friend;
now with respect to himself the perception of his own existence is
choiceworthy, therefore is it also in respect of his friend.

And besides, their Friendship is acted out in intimacy, and so with good
reason they desire this. And whatever in each man's opinion constitutes
existence, or whatsoever it is for the sake of which they choose life,
herein they wish their friends to join with them; and so some men drink
together, others gamble, others join in gymnastic exercises or hunting,
others study philosophy together: in each case spending their days
together in that which they like best of all things in life, for since
they wish to be intimate with their friends they do and partake in those
things whereby they think to attain this object.

Therefore the Friendship of the wicked comes to be depraved; for, being
unstable, they share in what is bad and become depraved in being made
like to one another: but the Friendship of the good is good, growing
with their intercourse; they improve also, as it seems, by repeated
acts, and by mutual correction, for they receive impress from one
another in the points which give them pleasure; whence says the poet,

  "Thou from the good, good things shalt surely learn."

Here then we will terminate our discourse of Friendship. The next thing
is to go into the subject of Pleasure.




BOOK X


Next, it would seem, follows a discussion respecting Pleasure, for it is
thought to be most closely bound up with our kind: and so men train the
young, guiding them on their course by the rudders of Pleasure and Pain.
And to like and dislike what one ought is judged to be most important
for the formation of good moral character: because these feelings extend
all one's life through, giving a bias towards and exerting an influence
on the side of Virtue and Happiness, since men choose what is pleasant
and avoid what is painful.

Subjects such as these then, it would seem, we ought by no means to pass
by, and specially since they involve much difference of opinion. There
are those who call Pleasure the Chief Good; there are others who on the
contrary maintain that it is exceedingly bad; some perhaps from a real
conviction that such is the case, others from a notion that it is
better, in reference to our life and conduct, to show up Pleasure as
bad, even if it is not so really; arguing that, as the mass of men have
a bias towards it and are the slaves of their pleasures, it is right to
draw them to the contrary, for that so they may possibly arrive at the
mean.

I confess I suspect the soundness of this policy; in matters respecting
men's feelings and actions theories are less convincing than facts:
whenever, therefore, they are found conflicting with actual experience,
they not only are despised but involve the truth in their fall: he, for
instance, who deprecates Pleasure, if once seen to aim at it, gets the
credit of backsliding to it as being universally such as he said it was,
the mass of men being incapable of nice distinctions.

Real accounts, therefore, of such matters seem to be most expedient, not
with a view to knowledge merely but to life and conduct: for they are
believed as being in harm with facts, and so they prevail with the wise
to live in accordance with them.

But of such considerations enough: let us now proceed to the current
maxims respecting Pleasure.

II Now Eudoxus thought Pleasure to be the Chief Good because he saw all,
rational and irrational alike, aiming at it: and he argued that, since
in all what was the object of choice must be good and what most so the
best, the fact of all being drawn to the same thing proved this thing to
be the best for all: "For each," he said, "finds what is good for itself
just as it does its proper nourishment, and so that which is good for
all, and the object of the aim of all, is their Chief Good."

(And his theories were received, not so much for their own sake, as
because of his excellent moral character; for he was thought to be
eminently possessed of perfect self-mastery, and therefore it was not
thought that he said these things because he was a lover of Pleasure but
that he really was so convinced.)

And he thought his position was not less proved by the argument from the
contrary: that is, since Pain was in itself an object of avoidance to
all the contrary must be in like manner an object of choice.

Again he urged that that is most choiceworthy which we choose, not by
reason of, or with a view to, anything further; and that Pleasure is
confessedly of this kind because no one ever goes on to ask to what
purpose he is pleased, feeling that Pleasure is in itself choiceworthy.

Again, that when added to any other good it makes it more choiceworthy;
as, for instance, to actions of justice, or perfected self-mastery; and
good can only be increased by itself.

However, this argument at least seems to prove only that it belongs to
the class of goods, and not that it does so more than anything else: for
every good is more choicewortby in combination with some other than when
taken quite alone. In fact, it is by just such an argument that Plato
proves that Pleasure is not the Chief Good: "For," says he, "the life of
Pleasure is more choiceworthy in combination with Practical Wisdom than
apart from it; but, if the compound better then simple Pleasure cannot
be the Chief Good; because the very Chief Good cannot by any addition
become choiceworthy than it is already:" and it is obvious that nothing
else can be the Chief Good, which by combination with any of the things
in themselves good comes to be more choiceworthy.

What is there then of such a nature? (meaning, of course, whereof we can
partake; because that which we are in search of must be such).

As for those who object that "what all aim at is not necessarily good,"
I confess I cannot see much in what they say, because what all _think_
we say _is_. And he who would cut away this ground from under us will
not bring forward things more dependable: because if the argument had
rested on the desires of irrational creatures there might have been
something in what he says, but, since the rational also desire Pleasure,
how can his objection be allowed any weight? and it may be that, even in
the lower animals, there is some natural good principle above themselves
which aims at the good peculiar to them.

Nor does that seem to be sound which is urged respecting the argument
from the contrary: I mean, some people say "it does not follow that
Pleasure must be good because Pain is evil, since evil may be opposed to
evil, and both evil and good to what is indifferent:" now what they say
is right enough in itself but does not hold in the present instance.
If both Pleasure and Pain were bad both would have been objects of
avoidance; or if neither then neither would have been, at all events
they must have fared alike: but now men do plainly avoid the one as bad
and choose the other as good, and so there is a complete opposition. III
Nor again is Pleasure therefore excluded from being good because it
does not belong to the class of qualities: the acts of virtue are not
qualities, neither is Happiness [yet surely both are goods].

Again, they say the Chief Good is limited but Pleasure unlimited, in
that it admits of degrees.

Now if they judge this from the act of feeling Pleasure then the same
thing will apply to justice and all the other virtues, in respect of
which clearly it is said that men are more or less of such and such
characters (according to the different virtues), they are more just or
more brave, or one may practise justice and self-mastery more or less.

If, on the other hand, they judge in respect of the Pleasures themselves
then it may be they miss the true cause, namely that some are unmixed
and others mixed: for just as health being in itself limited, admits of
degrees, why should not Pleasure do so and yet be limited? in the former
case we account for it by the fact that there is not the same adjustment
of parts in all men, nor one and the same always in the same individual:
but health, though relaxed, remains up to a certain point, and differs
in degrees; and of course the same may be the case with Pleasure.

Again, assuming the Chief Good to be perfect and all Movements and
Generations imperfect, they try to shew that Pleasure is a Movement and
a Generation.

Yet they do not seem warranted in saying even that it is a Movement: for
to every Movement are thought to belong swiftness and slowness, and
if not in itself, as to that of the universe, yet relatively: but to
Pleasure neither of these belongs: for though one may have got quickly
into the state Pleasure, as into that of anger, one cannot be in the
state quickly, nor relatively to the state of any other person; but we
can walk or grow, and so on, quickly or slowly.

Of course it is possible to change into the state of Pleasure quickly or
slowly, but to act in the state (by which, I mean, have the perception
of Pleasure) quickly, is not possible. And how can it be a Generation?
because, according to notions generally held, not _any_thing is
generated from _any_thing, but a thing resolves itself into that out
of which it was generated: whereas of that of which Pleasure is a
Generation Pain is a Destruction.

Again, they say that Pain is a lack of something suitable to nature and
Pleasure a supply of it.

But these are affections of the body: now if Pleasure really is a
supplying of somewhat suitable to nature, that must feel the Pleasure in
which the supply takes place, therefore the body of course: yet this
is not thought to be so: neither then is Pleasure a supplying, only a
person of course will be pleased when a supply takes place just as he
will be pained when he is cut.

This notion would seem to have arisen out of the Pains and Pleasures
connected with natural nourishment; because, when people have felt a
lack and so have had Pain first, they, of course, are pleased with the
supply of their lack.

But this is not the case with all Pleasures: those attendant on
mathematical studies, for instance, are unconnected with any Pain; and
of such as attend on the senses those which arise through the sense of
Smell; and again, many sounds, and sights, and memories, and hopes: now
of what can these be Generations? because there has been here no lack of
anything to be afterwards supplied.

And to those who bring forward disgraceful Pleasures we may reply that
these are not really pleasant things; for it does not follow because
they are pleasant to the ill-disposed that we are to admit that they are
pleasant except to them; just as we should not say that those things
are really wholesome, or sweet, or bitter, which are so to the sick,
or those objects really white which give that impression to people
labouring under ophthalmia.

Or we might say thus, that the Pleasures are choiceworthy but not as
derived from these sources: just as wealth is, but not as the price of
treason; or health, but not on the terms of eating anything however
loathsome. Or again, may we not say that Pleasures differ in kind? those
derived from honourable objects, for instance are different from those
arising from disgraceful ones; and it is not possible to experience
the Pleasure of the just man without being just, or of the musical man
without being musical; and so on of others.

The distinction commonly drawn between the friend and the flatterer
would seem to show clearly either that Pleasure is not a good, or that
there are different kinds of Pleasure: for the former is thought to have
good as the object of his intercourse, the latter Pleasure only; and
this last is reproached, but the former men praise as having different
objects in his intercourse.

[Sidenote: 1174a]

Again, no one would choose to live with a child's intellect all his
life through, though receiving the highest possible Pleasure from such
objects as children receive it from; or to take Pleasure in doing any of
the most disgraceful things, though sure never to be pained.

There are many things also about which we should be diligent even though
they brought no Pleasure; as seeing, remembering, knowing, possessing
the various Excellences; and the fact that Pleasures do follow on these
naturally makes no difference, because we should certainly choose them
even though no Pleasure resulted from them.

It seems then to be plain that Pleasure is not the Chief Good, nor is
every kind of it choiceworthy: and that there are some choiceworthy in
themselves, differing in kind, _i.e._ in the sources from which they
are derived. Let this then suffice by way of an account of the current
maxims respecting Pleasure and Pain.

[Sidenote: IV]

Now what it is, and how characterised, will be more plain if we take up
the subject afresh.

An act of Sight is thought to be complete at any moment; that is to say,
it lacks nothing the accession of which subsequently will complete its
whole nature.

Well, Pleasure resembles this: because it is a whole, as one may say;
and one could not at any moment of time take a Pleasure whose whole
nature would be completed by its lasting for a longer time. And for this
reason it is not a Movement: for all Movement takes place in time of
certain duration and has a certain End to accomplish; for instance, the
Movement of house-building is then only complete when the builder has
produced what he intended, that is, either in the whole time [necessary
to complete the whole design], or in a given portion. But all the
subordinate Movements are incomplete in the parts of the time, and are
different in kind from the whole movement and from one another (I
mean, for instance, that the fitting the stones together is a Movement
different from that of fluting the column, and both again from the
construction of the Temple as a whole: but this last is complete as
lacking nothing to the result proposed; whereas that of the basement,
or of the triglyph, is incomplete, because each is a Movement of a part
merely).

As I said then, they differ in kind, and you cannot at any time you
choose find a Movement complete in its whole nature, but, if at all, in
the whole time requisite.

[Sidenote: 1174_b_]

And so it is with the Movement of walking and all others: for, if motion
be a Movement from one place to another place, then of it too there are
different kinds, flying, walking, leaping, and such-like. And not only
so, but there are different kinds even in walking: the where-from and
where-to are not the same in the whole Course as in a portion of it;
nor in one portion as in another; nor is crossing this line the same as
crossing that: because a man is not merely crossing a line but a line in
a given place, and this is in a different place from that.

Of Movement I have discoursed exactly in another treatise. I will now
therefore only say that it seems not to be complete at any given moment;
and that most movements are incomplete and specifically different, since
the whence and whither constitute different species.

But of Pleasure the whole nature is complete at any given moment: it
is plain then that Pleasure and Movement must be different from one
another, and that Pleasure belongs to the class of things whole and
complete. And this might appear also from the impossibility of moving
except in a definite time, whereas there is none with respect to the
sensation of Pleasure, for what exists at the very present moment is a
kind of "whole."

From these considerations then it is plain that people are not warranted
in saying that Pleasure is a Movement or a Generation: because these
terms are not applicable to all things, only to such as are divisible
and not "wholes:" I mean that of an act of Sight there is no Generation,
nor is there of a point, nor of a monad, nor is any one of these a
Movement or a Generation: neither then of Pleasure is there Movement or
Generation, because it is, as one may say, "a whole."

Now since every Percipient Faculty works upon the Object answering to
it, and perfectly the Faculty in a good state upon the most excellent of
the Objects within its range (for Perfect Working is thought to be much
what I have described; and we will not raise any question about saying
"the Faculty" works, instead of, "that subject wherein the Faculty
resides"), in each case the best Working is that of the Faculty in its
best state upon the best of the Objects answering to it. And this will
be, further, most perfect and most pleasant: for Pleasure is attendant
upon every Percipient Faculty, and in like manner on every intellectual
operation and speculation; and that is most pleasant which is most
perfect, and that most perfect which is the Working of the best Faculty
upon the most excellent of the Objects within its range.

And Pleasure perfects the Working. But Pleasure does not perfect it in
the same way as the Faculty and Object of Perception do, being good;
just as health and the physician are not in similar senses causes of a
healthy state.

And that Pleasure does arise upon the exercise of every Percipient
Faculty is evident, for we commonly say that sights and sounds are
pleasant; it is plain also that this is especially the case when the
Faculty is most excellent and works upon a similar Object: and when both
the Object and Faculty of Perception are such, Pleasure will always
exist, supposing of course an agent and a patient.

[Sidenote: 1175_a_]

Furthermore, Pleasure perfects the act of Working not in the way of an
inherent state but as a supervening finish, such as is bloom in people
at their prime. Therefore so long as the Object of intellectual or
sensitive Perception is such as it should be and also the Faculty which
discerns or realises the Object, there will be Pleasure in the Working:
because when that which has the capacity of being acted on and that
which is apt to act are alike and similarly related, the same result
follows naturally.

How is it then that no one feels Pleasure continuously? is it not that
he wearies, because all human faculties are incapable of unintermitting
exertion; and so, of course, Pleasure does not arise either, because
that follows upon the act of Working. But there are some things which
please when new, but afterwards not in the like way, for exactly the
same reason: that at first the mind is roused and works on these Objects
with its powers at full tension; just as they who are gazing stedfastly
at anything; but afterwards the act of Working is not of the kind it was
at first, but careless, and so the Pleasure too is dulled.

Again, a person may conclude that all men grasp at Pleasure, because all
aim likewise at Life and Life is an act of Working, and every man works
at and with those things which also he best likes; the musical man, for
instance, works with his hearing at music; the studious man with his
intellect at speculative questions, and so forth. And Pleasure perfects
the acts of Working, and so Life after which men grasp. No wonder then
that they aim also at Pleasure, because to each it perfects Life, which
is itself choiceworthy. (We will take leave to omit the question whether
we choose Life for Pleasure's sake of Pleasure for Life's sake; because
these two plainly are closely connected and admit not of separation;
since Pleasure comes not into being without Working, and again, every
Working Pleasure perfects.)

And this is one reason why Pleasures are thought to differ in kind,
because we suppose that things which differ in kind must be perfected by
things so differing: it plainly being the case with the productions of
Nature and Art; as animals, and trees, and pictures, and statues, and
houses, and furniture; and so we suppose that in like manner acts of
Working which are different in kind are perfected by things differing in
kind. Now Intellectual Workings differ specifically from those of the
Senses, and these last from one another; therefore so do the Pleasures
which perfect them.

This may be shown also from the intimate connection subsisting between
each Pleasure and the Working which it perfects: I mean, that the
Pleasure proper to any Working increases that Working; for they who
work with Pleasure sift all things more closely and carry them out to a
greater degree of nicety; for instance, those men become geometricians
who take Pleasure in geometry, and they apprehend particular points more
completely: in like manner men who are fond of music, or architecture,
or anything else, improve each on his own pursuit, because they feel
Pleasure in them. Thus the Pleasures aid in increasing the Workings, and
things which do so aid are proper and peculiar: but the things which are
proper and peculiar to others specifically different are themselves also
specifically different.

Yet even more clearly may this be shown from the fact that the Pleasures
arising from one kind of Workings hinder other Workings; for instance,
people who are fond of flute-music cannot keep their attention to
conversation or discourse when they catch the sound of a flute; because
they take more Pleasure in flute-playing than in the Working they are
at the time engaged on; in other words, the Pleasure attendant on
flute-playing destroys the Working of conversation or discourse. Much
the same kind of thing takes place in other cases, when a person is
engaged in two different Workings at the same time: that is, the
pleasanter of the two keeps pushing out the other, and, if the disparity
in pleasantness be great, then more and more till a man even ceases
altogether to work at the other.

This is the reason why, when we are very much pleased with anything
whatever, we do nothing else, and it is only when we are but moderately
pleased with one occupation that we vary it with another: people,
for instance, who eat sweetmeats in the theatre do so most when the
performance is indifferent.

Since then the proper and peculiar Pleasure gives accuracy to the
Workings and makes them more enduring and better of their kind, while
those Pleasures which are foreign to them mar them, it is plain there
is a wide difference between them: in fact, Pleasures foreign to any
Working have pretty much the same effect as the Pains proper to it,
which, in fact, destroy the Workings; I mean, if one man dislikes
writing, or another calculation, the one does not write, the other does
not calculate; because, in each case, the Working is attended with some
Pain: so then contrary effects are produced upon the Workings by the
Pleasures and Pains proper to them, by which I mean those which arise
upon the Working, in itself, independently of any other circumstances.
As for the Pleasures foreign to a Working, we have said already that
they produce a similar effect to the Pain proper to it; that is they
destroy the Working, only not in like way.

Well then, as Workings differ from one another in goodness and badness,
some being fit objects of choice, others of avoidance, and others in
their nature indifferent, Pleasures are similarly related; since its own
proper Pleasure attends or each Working: of course that proper to a good
Working is good, that proper to a bad, bad: for even the desires for
what is noble are praiseworthy, and for what is base blameworthy.

Furthermore, the Pleasures attendant on Workings are more closely
connected with them even than the desires after them: for these last
are separate both in time and nature, but the former are close to the
Workings, and so indivisible from them as to raise a question whether
the Working and the Pleasure are identical; but Pleasure does not seem
to be an Intellectual Operation nor a Faculty of Perception, because
that is absurd; but yet it gives some the impression of being the same
from not being separated from these.

As then the Workings are different so are their Pleasures; now Sight
differs from Touch in purity, and Hearing and Smelling from Taste;
therefore, in like manner, do their Pleasures; and again, Intellectual
Pleasures from these Sensual, and the different kinds both of
Intellectual and Sensual from one another.

It is thought, moreover, that each animal has a Pleasure proper to
itself, as it has a proper Work; that Pleasure of course which is
attendant on the Working. And the soundness of this will appear upon
particular inspection: for horse, dog, and man have different Pleasures;
as Heraclitus says, an ass would sooner have hay than gold; in other
words, provender is pleasanter to asses than gold. So then the Pleasures
of animals specifically different are also specifically different, but
those of the same, we may reasonably suppose, are without difference.

Yet in the case of human creatures they differ not a little: for the
very same things please some and pain others: and what are painful and
hateful to some are pleasant to and liked by others. The same is the
case with sweet things: the same will not seem so to the man in a fever
as to him who is in health: nor will the invalid and the person in
robust health have the same notion of warmth. The same is the case with
other things also.

Now in all such cases that is held to _be_ which impresses the good man
with the notion of being such and such; and if this is a second maxim
(as it is usually held to be), and Virtue, that is, the Good man, in
that he is such, is the measure of everything, then those must be real
Pleasures which gave him the impression of being so and those things
pleasant in which he takes Pleasure. Nor is it at all astonishing that
what are to him unpleasant should give another person the impression of
being pleasant, for men are liable to many corruptions and marrings; and
the things in question are not pleasant really, only to these particular
persons, and to them only as being thus disposed.

Well of course, you may say, it is obvious that we must assert those
which are confessedly disgraceful to be real Pleasures, except to
depraved tastes: but of those which are thought to be good what kind,
or which, must we say is _The Pleasure of Man?_ is not the answer plain
from considering the Workings, because the Pleasures follow upon these?

Whether then there be one or several Workings which belong to the
perfect and blessed man, the Pleasures which perfect these Workings must
be said to be specially and properly _The Pleasures of Man;_ and all
the rest in a secondary sense, and in various degrees according as the
Workings are related to those highest and best ones.


VI

Now that we have spoken about the Excellences of both kinds, and
Friendship in its varieties, and Pleasures, it remains to sketch out
Happiness, since we assume that to be the one End of all human things:
and we shall save time and trouble by recapitulating what was stated
before.

[Sidenote: 1176b] Well then, we said that it is not a State merely;
because, if it were, it might belong to one who slept all his life
through and merely vegetated, or to one who fell into very great
calamities: and so, if these possibilities displease us and we would
rather put it into the rank of some kind of Working (as was also said
before), and Workings are of different kinds (some being necessary
and choiceworthy with a view to other things, while others are so in
themselves), it is plain we must rank Happiness among those choiceworthy
for their own sakes and not among those which are so with a view to
something further: because Happiness has no lack of anything but is
self-sufficient.

By choiceworthy in themselves are meant those from which nothing is
sought beyond the act of Working: and of this kind are thought to be the
actions according to Virtue, because doing what is noble and excellent
is one of those things which are choiceworthy for their own sake alone.

And again, such amusements as are pleasant; because people do not choose
them with any further purpose: in fact they receive more harm than
profit from them, neglecting their persons and their property. Still the
common run of those who are judged happy take refuge in such pastimes,
which is the reason why they who have varied talent in such are highly
esteemed among despots; because they make themselves pleasant in those
things which these aim at, and these accordingly want such men.

Now these things are thought to be appurtenances of Happiness because
men in power spend their leisure herein: yet, it may be, we cannot
argue from the example of such men: because there is neither Virtue nor
Intellect necessarily involved in having power, and yet these are the
only sources of good Workings: nor does it follow that because these
men, never having tasted pure and generous Pleasure, take refuge in
bodily ones, we are therefore to believe them to be more choiceworthy:
for children too believe that those things are most excellent which are
precious in their eyes.

We may well believe that as children and men have different ideas as to
what is precious so too have the bad and the good: therefore, as we have
many times said, those things are really precious and pleasant which
seem so to the good man: and as to each individual that Working is most
choiceworthy which is in accordance with his own state to the good man
that is so which is in accordance with Virtue.

Happiness then stands not in amusement; in fact the very notion is
absurd of the End being amusement, and of one's toiling and enduring
hardness all one's life long with a view to amusement: for everything in
the world, so to speak, we choose with some further End in view, except
Happiness, for that is the End comprehending all others. Now to take
pains and to labour with a view to amusement is plainly foolish and
very childish: but to amuse one's self with a view to steady employment
afterwards, as Anacharsis says, is thought to be right: for amusement is
like rest, and men want rest because unable to labour continuously.

Rest, therefore, is not an End, because it is adopted with a view to
Working afterwards.

[Sidenote: 1177a] Again, it is held that the Happy Life must be one in
the way of Excellence, and this is accompanied by earnestness and stands
not in amusement. Moreover those things which are done in earnest, we
say, are better than things merely ludicrous and joined with amusement:
and we say that the Working of the better part, or the better man, is
more earnest; and the Working of the better is at once better and more
capable of Happiness.

Then, again, as for bodily Pleasures, any ordinary person, or even
a slave, might enjoy them, just as well as the best man living but
Happiness no one supposes a slave to share except so far as it is
implied in life: because Happiness stands not in such pastimes but in
the Workings in the way of Excellence, as has also been stated before.


VII

Now if Happiness is a Working in the way of Excellence of course that
Excellence must be the highest, that is to say, the Excellence of the
best Principle. Whether then this best Principle is Intellect or some
other which is thought naturally to rule and to lead and to conceive of
noble and divine things, whether being in its own nature divine or the
most divine of all our internal Principles, the Working of this in
accordance with its own proper Excellence must be the perfect Happiness.

That it is Contemplative has been already stated: and this would seem to
be consistent with what we said before and with truth: for, in the first
place, this Working is of the highest kind, since the Intellect is the
highest of our internal Principles and the subjects with which it
is conversant the highest of all which fall within the range of our
knowledge.

Next, it is also most Continuous: for we are better able to contemplate
than to do anything else whatever, continuously.

Again, we think Pleasure must be in some way an ingredient in Happiness,
and of all Workings in accordance with Excellence that in the way of
Science is confessedly most pleasant: at least the pursuit of Science is
thought to contain Pleasures admirable for purity and permanence; and it
is reasonable to suppose that the employment is more pleasant to those
who have mastered, than to those who are yet seeking for, it.

And the Self-Sufficiency which people speak of will attach chiefly to
the Contemplative Working: of course the actual necessaries of life are
needed alike by the man of science, and the just man, and all the other
characters; but, supposing all sufficiently supplied with these, the
just man needs people towards whom, and in concert with whom, to
practise his justice; and in like manner the man of perfected
self-mastery, and the brave man, and so on of the rest; whereas the man
of science can contemplate and speculate even when quite alone, and the
more entirely he deserves the appellation the more able is he to do so:
it may be he can do better for having fellow-workers but still he is
certainly most Self-Sufficient.

[Sidenote: 1177b] Again, this alone would seem to be rested in for
its own sake, since nothing results from it beyond the fact of having
contemplated; whereas from all things which are objects of moral action
we do mean to get something beside the doing them, be the same more or
less.

Also, Happiness is thought to stand in perfect rest; for we toil that we
may rest, and war that we may be at peace. Now all the Practical Virtues
require either society or war for their Working, and the actions
regarding these are thought to exclude rest; those of war entirely,
because no one chooses war, nor prepares for war, for war's sake: he
would indeed be thought a bloodthirsty villain who should make enemies
of his friends to secure the existence of fighting and bloodshed. The
Working also of the statesman excludes the idea of rest, and, beside the
actual work of government, seeks for power and dignities or at least
Happiness for the man himself and his fellow-citizens: a Happiness
distinct the national Happiness which we evidently seek as being
different and distinct.

If then of all the actions in accordance with the various virtues those
of policy and war are pre-eminent in honour and greatness, and these are
restless, and aim at some further End and are not choiceworthy for
their own sakes, but the Working of the Intellect, being apt for
contemplation, is thought to excel in earnestness, and to aim at no End
beyond itself and to have Pleasure of its own which helps to increase
the Working, and if the attributes of Self-Sufficiency, and capacity of
rest, and unweariedness (as far as is compatible with the infirmity
of human nature), and all other attributes of the highest Happiness,
plainly belong to this Working, this must be perfect Happiness, if
attaining a complete duration of life, which condition is added because
none of the points of Happiness is incomplete.

But such a life will be higher than mere human nature, because a man
will live thus, not in so far as he is man but in so far as there is in
him a divine Principle: and in proportion as this Principle excels
his composite nature so far does the Working thereof excel that in
accordance with any other kind of Excellence: and therefore, if pure
Intellect, as compared with human nature, is divine, so too will the
life in accordance with it be divine compared with man's ordinary life.
[Sidenote: 1178a] Yet must we not give ear to those who bid one as man
to mind only man's affairs, or as mortal only mortal things; but, so far
as we can, make ourselves like immortals and do all with a view to
living in accordance with the highest Principle in us, for small as it
may be in bulk yet in power and preciousness it far more excels all the
others.

In fact this Principle would seem to constitute each man's "Self," since
it is supreme and above all others in goodness it _would_ be absurd then
for a man not to choose his own life but that of some other.

And here will apply an observation made before, that whatever is proper
to each is naturally best and pleasantest to him: such then is to Man
the life in accordance with pure Intellect (since this Principle is most
truly Man), and if so, then it is also the happiest.


VIII

And second in degree of Happiness will be that Life which is in
accordance with the other kind of Excellence, for the Workings in
accordance with this are proper to Man: I mean, we do actions of
justice, courage, and the other virtues, towards one another, in
contracts, services of different kinds, and in all kinds of actions and
feelings too, by observing what is befitting for each: and all these
plainly are proper to man. Further, the Excellence of the Moral
character is thought to result in some points from physical
circumstances, and to be, in many, very closely connected with the
passions.

Again, Practical Wisdom and Excellence of the Moral character are
very closely united; since the Principles of Practical Wisdom are in
accordance with the Moral Virtues and these are right when they accord
with Practical Wisdom.

These moreover, as bound up with the passions, must belong to the
composite nature, and the Excellences or Virtues of the composite nature
are proper to man: therefore so too will be the life and Happiness which
is in accordance with them. But that of the Pure Intellect is separate
and distinct: and let this suffice upon the subject, since great
exactness is beyond our purpose,

It would seem, moreover, to require supply of external goods to a small
degree, or certainly less than the Moral Happiness: for, as far as
necessaries of life are concerned, we will suppose both characters to
need them equally (though, in point of fact, the man who lives in
society does take more pains about his person and all that kind of
thing; there will really be some little difference), but when we come to
consider their Workings there will be found a great difference.

I mean, the liberal man must have money to do his liberal actions with,
and the just man to meet his engagements (for mere intentions
are uncertain, and even those who are unjust make a pretence of
_wishing_ to do justly), and the brave man must have power, if
he is to perform any of the actions which appertain to his particular
Virtue, and the man of perfected self-mastery must have opportunity of
temptation, else how shall he or any of the others display his real
character?

[Sidenote: 1178b]

(By the way, a question is sometimes raised, whether the moral choice or
the actions have most to do with Virtue, since it consists in both: it
is plain that the perfection of virtuous action requires both: but for
the actions many things are required, and the greater and more numerous
they are the more.) But as for the man engaged in Contemplative
Speculation, not only are such things unnecessary for his Working, but,
so to speak, they are even hindrances: as regards the Contemplation at
least; because of course in so far as he is Man and lives in society he
chooses to do what Virtue requires, and so he will need such things
for maintaining his character as Man though not as a speculative
philosopher.

And that the perfect Happiness must be a kind of Contemplative Working
may appear also from the following consideration: our conception of the
gods is that they are above all blessed and happy: now what kind of
Moral actions are we to attribute to them? those of justice? nay,
will they not be set in a ridiculous light if represented as forming
contracts, and restoring deposits, and so on? well then, shall we
picture them performing brave actions, withstanding objects of fear and
meeting dangers, because it is noble to do so? or liberal ones? but to
whom shall they be giving? and further, it is absurd to think they have
money or anything of the kind. And as for actions of perfected
self-mastery, what can theirs be? would it not be a degrading praise
that they have no bad desires? In short, if one followed the subject
into all details all the circumstances connected with Moral actions
would appear trivial and unworthy of gods.

Still, every one believes that they live, and therefore that they
Work because it is not supposed that they sleep their time away like
Endymion: now if from a living being you take away Action, still more
if Creation, what remains but Contemplation? So then the Working of
the Gods, eminent in blessedness, will be one apt for Contemplative
Speculation; and of all human Workings that will have the greatest
capacity for Happiness which is nearest akin to this.

A corroboration of which position is the fact that the other animals
do not partake of Happiness, being completely shut out from any such
Working.

To the gods then all their life is blessed; and to men in so far as
there is in it some copy of such Working, but of the other animals none
is happy because it in no way shares in Contemplative Speculation.

Happiness then is co-extensive with this Contemplative Speculation, and
in proportion as people have the act of Contemplation so far have they
also the being happy, not incidentally, but in the way of Contemplative
Speculation because it is in itself precious.

So Happiness must be a kind of Contemplative Speculation; but since it
is Man we are speaking of he will need likewise External Prosperity,
because his Nature is not by itself sufficient for Speculation, but
there must be health of body, and nourishment, and tendance of all
kinds.

[Sidenote: 1179a] However, it must not be thought, because without
external goods a man cannot enjoy high Happiness, that therefore he
will require many and great goods in order to be happy: for neither
Self-sufficiency, nor Action, stand in Excess, and it is quite possible
to act nobly without being ruler of sea and land, since even with
moderate means a man may act in accordance with Virtue.

And this may be clearly seen in that men in private stations are thought
to act justly, not merely no less than men in power but even more: it
will be quite enough that just so much should belong to a man as is
necessary, for his life will be happy who works in accordance with
Virtue.

Solon perhaps drew a fair picture of the Happy, when he said that they
are men moderately supplied with external goods, and who have achieved
the most noble deeds, as he thought, and who have lived with perfect
self-mastery: for it is quite possible for men of moderate means to act
as they ought.

Anaxagoras also seems to have conceived of the Happy man not as either
rich or powerful, saying that he should not wonder if he were accounted
a strange man in the judgment of the multitude: for they judge by
outward circumstances of which alone they have any perception.

And thus the opinions of the Wise seem to be accordant with our account
of the matter: of course such things carry some weight, but truth, in
matters of moral action, is judged from facts and from actual life,
for herein rests the decision. So what we should do is to examine the
preceding statements by referring them to facts and to actual life, and
when they harmonise with facts we may accept them, when they are at
variance with them conceive of them as mere theories.

Now he that works in accordance with, and pays observance to, Pure
Intellect, and tends this, seems likely to be both in the best frame of
mind and dearest to the Gods: because if, as is thought, any care is
bestowed on human things by the Gods then it must be reasonable to think
that they take pleasure in what is best and most akin to themselves (and
this must be the Pure Intellect); and that they requite with kindness
those who love and honour this most, as paying observance to what is
dear to them, and as acting rightly and nobly. And it is quite obvious
that the man of Science chiefly combines all these: he is therefore
dearest to the Gods, and it is probable that he is at the same time most
Happy.

Thus then on this view also the man of Science will be most Happy.



IX

Now then that we have said enough in our sketchy kind of way
on these subjects; I mean, on the Virtues, and also on Friendship and
Pleasure; are we to suppose that our original purpose is completed? Must
we not rather acknowledge, what is commonly said, that in matters of
moral action mere Speculation and Knowledge is not the real End but
rather Practice: and if so, then neither in respect of Virtue is
Knowledge enough; we must further strive to have and exert it, and take
whatever other means there are of becoming good.

Now if talking and writing were of themselves sufficient to make men
good, they would justly, as Theognis observes have reaped numerous and
great rewards, and the thing to do would be to provide them: but in
point of fact, while they plainly have the power to guide and stimulate
the generous among the young and to base upon true virtuous principle
any noble and truly high-minded disposition, they as plainly are
powerless to guide the mass of men to Virtue and goodness; because it is
not their nature to be amenable to a sense of shame but only to fear;
nor to abstain from what is low and mean because it is disgraceful to do
it but because of the punishment attached to it: in fact, as they live
at the beck and call of passion, they pursue their own proper pleasures
and the means of securing them, and they avoid the contrary pains; but
as for what is noble and truly pleasurable they have not an idea of it,
inasmuch as they have never tasted of it.

Men such as these then what mere words can transform? No, indeed! it is
either actually impossible, or a task of no mean difficulty, to alter by
words what has been of old taken into men's very dispositions: and,
it may be, it is a ground for contentment if with all the means and
appliances for goodness in our hands we can attain to Virtue.

The formation of a virtuous character some ascribe to Nature, some to
Custom, and some to Teaching. Now Nature's part, be it what it may,
obviously does not rest with us, but belongs to those who in the truest
sense are fortunate, by reason of certain divine agency,

Then, as for Words and Precept, they, it is to be feared, will not avail
with all; but it may be necessary for the mind of the disciple to have
been previously prepared for liking and disliking as he ought; just as
the soil must, to nourish the seed sown. For he that lives in obedience
to passion cannot hear any advice that would dissuade him, nor, if he
heard, understand: now him that is thus how can one reform? in fact,
generally, passion is not thought to yield to Reason but to brute force.
So then there must be, to begin with, a kind of affinity to Virtue in
the disposition; which must cleave to what is honourable and loath
what is disgraceful. But to get right guidance towards Virtue from the
earliest youth is not easy unless one is brought up under laws of such
kind; because living with self-mastery and endurance is not pleasant to
the mass of men, and specially not to the young. For this reason the
food, and manner of living generally, ought to be the subject of
legal regulation, because things when become habitual will not be
disagreeable.

[Sidenote: 1180_a_] Yet perhaps it is not sufficient that men while
young should get right food and tendance, but, inasmuch as they will
have to practise and become accustomed to certain things even after they
have attained to man's estate, we shall want laws on these points as
well, and, in fine, respecting one's whole life, since the mass of men
are amenable to compulsion rather than Reason, and to punishment rather
than to a sense of honour.

And therefore some men hold that while lawgivers should employ the sense
of honour to exhort and guide men to Virtue, under the notion that they
will then obey who have been well trained in habits; they should
impose chastisement and penalties on those who disobey and are of less
promising nature; and the incurable expel entirely: because the good man
and he who lives under a sense of honour will be obedient to reason;
and the baser sort, who grasp at pleasure, will be kept in check, like
beasts of burthen by pain. Therefore also they say that the pains should
be such as are most contrary to the pleasures which are liked.

As has been said already, he who is to be good must have been brought up
and habituated well, and then live accordingly under good institutions,
and never do what is low and mean, either against or with his will. Now
these objects can be attained only by men living in accordance with some
guiding Intellect and right order, with power to back them.

As for the Paternal Rule, it possesses neither strength nor compulsory
power, nor in fact does the Rule of any one man, unless he is a king or
some one in like case: but the Law has power to compel, since it is a
declaration emanating from Practical Wisdom and Intellect. And people
feel enmity towards their fellow-men who oppose their impulses, however
rightly they may do so: the Law, on the contrary, is not the object of
hatred, though enforcing right rules.

The Lacedæmonian is nearly the only State in which the framer of the
Constitution has made any provision, it would seem, respecting the food
and manner of living of the people: in most States these points are
entirely neglected, and each man lives just as he likes, ruling his wife
and children Cyclops-Fashion.

Of course, the best thing would be that there should be a right Public
System and that we should be able to carry it out: but, since as a
public matter those points are neglected, the duty would seem to devolve
upon each individual to contribute to the cause of Virtue with his own
children and friends, or at least to make this his aim and purpose: and
this, it would seem, from what has been said, he will be best able to do
by making a Legislator of himself: since all public *[Sidenote: 1180_b_]
systems, it is plain, are formed by the instrumentality of laws and
those are good which are formed by that of good laws: whether they are
written or unwritten, whether they are applied to the training of one or
many, will not, it seems, make any difference, just as it does not in
music, gymnastics, or any other such accomplishments, which are gained
by practice.

For just as in Communities laws and customs prevail, so too in families
the express commands of the Head, and customs also: and even more in the
latter, because of blood-relationship and the benefits conferred:
for there you have, to begin with, people who have affection and are
naturally obedient to the authority which controls them.

Then, furthermore, Private training has advantages over Public, as in
the case of the healing art: for instance, as a general rule, a man who
is in a fever should keep quiet, and starve; but in a particular case,
perhaps, this may not hold good; or, to take a different illustration,
the boxer will not use the same way of fighting with all antagonists.

It would seem then that the individual will be most exactly attended to
under Private care, because so each will be more likely to obtain what
is expedient for him. Of course, whether in the art of healing, or
gymnastics, or any other, a man will treat individual cases the better
for being acquainted with general rules; as, "that so and so is good for
all, or for men in such and such cases:" because general maxims are not
only said to be but are the object-matter of sciences: still this is no
reason against the possibility of a man's taking excellent care of
some _one_ case, though he possesses no scientific knowledge but from
experience is exactly acquainted with what happens in each point; just
as some people are thought to doctor themselves best though they would
be wholly unable to administer relief to others. Yet it may seem to be
necessary nevertheless, for one who wishes to become a real artist and
well acquainted with the theory of his profession, to have recourse
to general principles and ascertain all their capacities: for we have
already stated that these are the object-matter of sciences.

If then it appears that we may become good through the instrumentality
of laws, of course whoso wishes to make men better by a system of care
and training must try to make a Legislator of himself; for to treat
skilfully just any one who may be put before you is not what any
ordinary person can do, but, if any one, he who has knowledge; as in the
healing art, and all others which involve careful practice and skill.

[Sidenote: 1181_a_] Will not then our next business be to inquire from
what sources, or how one may acquire this faculty of Legislation; or
shall we say, that, as in similar cases, Statesmen are the people to
learn from, since this faculty was thought to be a part of the Social
Science? Must we not admit that the Political Science plainly does not
stand on a similar footing to that of other sciences and faculties? I
mean, that while in all other cases those who impart the faculties
and themselves exert them are identical (physicians and painters for
instance) matters of Statesmanship the Sophists profess to teach, but
not one of them practises it, that being left to those actually engaged
in it: and these might really very well be thought to do it by some
singular knack and by mere practice rather than by any intellectual
process: for they neither write nor speak on these matters (though it
might be more to their credit than composing speeches for the courts or
the assembly), nor again have they made Statesmen of their own sons or
their friends.

One can hardly suppose but that they would have done so if they could,
seeing that they could have bequeathed no more precious legacy to their
communities, nor would they have preferred, for themselves or their
dearest friends, the possession of any faculty rather than this.

Practice, however, seems to contribute no little to its acquisition;
merely breathing the atmosphere of politics would never have made
Statesmen of them, and therefore we may conclude that they who would
acquire a knowledge of Statesmanship must have in addition practice.

But of the Sophists they who profess to teach it are plainly a long way
off from doing so: in fact, they have no knowledge at all of its nature
and objects; if they had, they would never have put it on the same
footing with Rhetoric or even on a lower: neither would they have
conceived it to be "an easy matter to legislate by simply collecting
such laws as are famous because of course one could select the best," as
though the selection were not a matter of skill, and the judging aright
a very great matter, as in Music: for they alone, who have practical
knowledge of a thing, can judge the performances rightly or understand
with what means and in what way they are accomplished, and what
harmonises with what: the unlearned must be content with being able to
discover whether the result is good or bad, as in painting.

[Sidenote: 1181_b_] Now laws may be called the performances or tangible
results of Political Science; how then can a man acquire from these
the faculty of Legislation, or choose the best? we do not see men made
physicians by compilations: and yet in these treatises men endeavour to
give not only the cases but also how they may be cured, and the proper
treatment in each case, dividing the various bodily habits. Well, these
are thought to be useful to professional men, but to the unprofessional
useless. In like manner it may be that collections of laws and
Constitutions would be exceedingly useful to such as are able to
speculate on them, and judge what is well, and what ill, and what
kind of things fit in with what others: but they who without this
qualification should go through such matters cannot have right judgment,
unless they have it by instinct, though they may become more intelligent
in such matters.

Since then those who have preceded us have left uninvestigated the
subject of Legislation, it will be better perhaps for us to investigate
it ourselves, and, in fact, the whole subject of Polity, that thus what
we may call Human Philosophy may be completed as far as in us lies.

First then, let us endeavour to get whatever fragments of good there may
be in the statements of our predecessors, next, from the Polities we
have collected, ascertain what kind of things preserve or destroy
Communities, and what, particular Constitutions; and the cause why some
are well and others ill managed, for after such inquiry, we shall be the
better able to take a concentrated view as to what kind of Constitution
is best, what kind of regulations are best for each, and what laws and
customs.

To this let us now proceed.

% chapter ethics (end)


%%
% The back matter contains appendices, bibliographies, indices, glossaries, etc.







\backmatter

\bibliography{introtext}
\bibliographystyle{plainnat}


\printindex

\end{document}

% 
