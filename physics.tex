\chapter{Physics} % (fold)
\label{cha:physics}

Physics
By Aristotle


Translated by R. P. Hardie and R. K. Gaye

----------------------------------------------------------------------

BOOK I

Part 1 

When the objects of an inquiry, in any department, have principles,
conditions, or elements, it is through acquaintance with these that
knowledge, that is to say scientific knowledge, is attained. For we
do not think that we know a thing until we are acquainted with its
primary conditions or first principles, and have carried our analysis
as far as its simplest elements. Plainly therefore in the science
of Nature, as in other branches of study, our first task will be to
try to determine what relates to its principles. 

The natural way of doing this is to start from the things which are
more knowable and obvious to us and proceed towards those which are
clearer and more knowable by nature; for the same things are not 'knowable
relatively to us' and 'knowable' without qualification. So in the
present inquiry we must follow this method and advance from what is
more obscure by nature, but clearer to us, towards what is more clear
and more knowable by nature. 

Now what is to us plain and obvious at first is rather confused masses,
the elements and principles of which become known to us later by analysis.
Thus we must advance from generalities to particulars; for it is a
whole that is best known to sense-perception, and a generality is
a kind of whole, comprehending many things within it, like parts.
Much the same thing happens in the relation of the name to the formula.
A name, e.g. 'round', means vaguely a sort of whole: its definition
analyses this into its particular senses. Similarly a child begins
by calling all men 'father', and all women 'mother', but later on
distinguishes each of them. 

Part 2

The principles in question must be either (a) one or (b) more than
one. If (a) one, it must be either (i) motionless, as Parmenides and
Melissus assert, or (ii) in motion, as the physicists hold, some declaring
air to be the first principle, others water. If (b) more than one,
then either (i) a finite or (ii) an infinite plurality. If (i) finite
(but more than one), then either two or three or four or some other
number. If (ii) infinite, then either as Democritus believed one in
kind, but differing in shape or form; or different in kind and even
contrary. 

A similar inquiry is made by those who inquire into the number of
existents: for they inquire whether the ultimate constituents of existing
things are one or many, and if many, whether a finite or an infinite
plurality. So they too are inquiring whether the principle or element
is one or many. 

Now to investigate whether Being is one and motionless is not a contribution
to the science of Nature. For just as the geometer has nothing more
to say to one who denies the principles of his science-this being
a question for a different science or for or common to all-so a man
investigating principles cannot argue with one who denies their existence.
For if Being is just one, and one in the way mentioned, there is a
principle no longer, since a principle must be the principle of some
thing or things. 

To inquire therefore whether Being is one in this sense would be like
arguing against any other position maintained for the sake of argument
(such as the Heraclitean thesis, or such a thesis as that Being is
one man) or like refuting a merely contentious argument-a description
which applies to the arguments both of Melissus and of Parmenides:
their premisses are false and their conclusions do not follow. Or
rather the argument of Melissus is gross and palpable and offers no
difficulty at all: accept one ridiculous proposition and the rest
follows-a simple enough proceeding. 

We physicists, on the other hand, must take for granted that the things
that exist by nature are, either all or some of them, in motion which
is indeed made plain by induction. Moreover, no man of science is
bound to solve every kind of difficulty that may be raised, but only
as many as are drawn falsely from the principles of the science: it
is not our business to refute those that do not arise in this way:
just as it is the duty of the geometer to refute the squaring of the
circle by means of segments, but it is not his duty to refute Antiphon's
proof. At the same time the holders of the theory of which we are
speaking do incidentally raise physical questions, though Nature is
not their subject: so it will perhaps be as well to spend a few words
on them, especially as the inquiry is not without scientific interest.

The most pertinent question with which to begin will be this: In what
sense is it asserted that all things are one? For 'is' is used in
many senses. Do they mean that all things 'are' substance or quantities
or qualities? And, further, are all things one substance-one man,
one horse, or one soul-or quality and that one and the same-white
or hot or something of the kind? These are all very different doctrines
and all impossible to maintain. 

For if both substance and quantity and quality are, then, whether
these exist independently of each other or not, Being will be many.

If on the other hand it is asserted that all things are quality or
quantity, then, whether substance exists or not, an absurdity results,
if the impossible can properly be called absurd. For none of the others
can exist independently: substance alone is independent: for everything
is predicated of substance as subject. Now Melissus says that Being
is infinite. It is then a quantity. For the infinite is in the category
of quantity, whereas substance or quality or affection cannot be infinite
except through a concomitant attribute, that is, if at the same time
they are also quantities. For to define the infinite you must use
quantity in your formula, but not substance or quality. If then Being
is both substance and quantity, it is two, not one: if only substance,
it is not infinite and has no magnitude; for to have that it will
have to be a quantity. 

Again, 'one' itself, no less than 'being', is used in many senses,
so we must consider in what sense the word is used when it is said
that the All is one. 

Now we say that (a) the continuous is one or that (b) the indivisible
is one, or (c) things are said to be 'one', when their essence is
one and the same, as 'liquor' and 'drink'. 

If (a) their One is one in the sense of continuous, it is many, for
the continuous is divisible ad infinitum. 

There is, indeed, a difficulty about part and whole, perhaps not relevant
to the present argument, yet deserving consideration on its own account-namely,
whether the part and the whole are one or more than one, and how they
can be one or many, and, if they are more than one, in what sense
they are more than one. (Similarly with the parts of wholes which
are not continuous.) Further, if each of the two parts is indivisibly
one with the whole, the difficulty arises that they will be indivisibly
one with each other also. 

But to proceed: If (b) their One is one as indivisible, nothing will
have quantity or quality, and so the one will not be infinite, as
Melissus says-nor, indeed, limited, as Parmenides says, for though
the limit is indivisible, the limited is not. 

But if (c) all things are one in the sense of having the same definition,
like 'raiment' and 'dress', then it turns out that they are maintaining
the Heraclitean doctrine, for it will be the same thing 'to be good'
and 'to be bad', and 'to be good' and 'to be not good', and so the
same thing will be 'good' and 'not good', and man and horse; in fact,
their view will be, not that all things are one, but that they are
nothing; and that 'to be of such-and-such a quality' is the same as
'to be of such-and-such a size'. 

Even the more recent of the ancient thinkers were in a pother lest
the same thing should turn out in their hands both one and many. So
some, like Lycophron, were led to omit 'is', others to change the
mode of expression and say 'the man has been whitened' instead of
'is white', and 'walks' instead of 'is walking', for fear that if
they added the word 'is' they should be making the one to be many-as
if 'one' and 'being' were always used in one and the same sense. What
'is' may be many either in definition (for example 'to be white' is
one thing, 'to be musical' another, yet the same thing be both, so
the one is many) or by division, as the whole and its parts. On this
point, indeed, they were already getting into difficulties and admitted
that the one was many-as if there was any difficulty about the same
thing being both one and many, provided that these are not opposites;
for 'one' may mean either 'potentially one' or 'actually one'.

Part 3

If, then, we approach the thesis in this way it seems impossible for
all things to be one. Further, the arguments they use to prove their
position are not difficult to expose. For both of them reason contentiously-I
mean both Melissus and Parmenides. [Their premisses are false and
their conclusions do not follow. Or rather the argument of Melissus
is gross and palpable and offers no difficulty at all: admit one ridiculous
proposition and the rest follows-a simple enough proceeding.] The
fallacy of Melissus is obvious. For he supposes that the assumption
'what has come into being always has a beginning' justifies the assumption
'what has not come into being has no beginning'. Then this also is
absurd, that in every case there should be a beginning of the thing-not
of the time and not only in the case of coming to be in the full sense
but also in the case of coming to have a quality-as if change never
took place suddenly. Again, does it follow that Being, if one, is
motionless? Why should it not move, the whole of it within itself,
as parts of it do which are unities, e.g. this water? Again, why is
qualitative change impossible? But, further, Being cannot be one in
form, though it may be in what it is made of. (Even some of the physicists
hold it to be one in the latter way, though not in the former.) Man
obviously differs from horse in form, and contraries from each other.

The same kind of argument holds good against Parmenides also, besides
any that may apply specially to his view: the answer to him being
that 'this is not true' and 'that does not follow'. His assumption
that one is used in a single sense only is false, because it is used
in several. His conclusion does not follow, because if we take only
white things, and if 'white' has a single meaning, none the less what
is white will be many and not one. For what is white will not be one
either in the sense that it is continuous or in the sense that it
must be defined in only one way. 'Whiteness' will be different from
'what has whiteness'. Nor does this mean that there is anything that
can exist separately, over and above what is white. For 'whiteness'
and 'that which is white' differ in definition, not in the sense that
they are things which can exist apart from each other. But Parmenides
had not come in sight of this distinction. 

It is necessary for him, then, to assume not only that 'being' has
the same meaning, of whatever it is predicated, but further that it
means (1) what just is and (2) what is just one. 

It must be so, for (1) an attribute is predicated of some subject,
so that the subject to which 'being' is attributed will not be, as
it is something different from 'being'. Something, therefore, which
is not will be. Hence 'substance' will not be a predicate of anything
else. For the subject cannot be a being, unless 'being' means several
things, in such a way that each is something. But ex hypothesi 'being'
means only one thing. 

If, then, 'substance' is not attributed to anything, but other things
are attributed to it, how does 'substance' mean what is rather than
what is not? For suppose that 'substance' is also 'white'. Since the
definition of the latter is different (for being cannot even be attributed
to white, as nothing is which is not 'substance'), it follows that
'white' is not-being--and that not in the sense of a particular not-being,
but in the sense that it is not at all. Hence 'substance' is not;
for it is true to say that it is white, which we found to mean not-being.
If to avoid this we say that even 'white' means substance, it follows
that 'being' has more than one meaning. 

In particular, then, Being will not have magnitude, if it is substance.
For each of the two parts must he in a different sense. 

(2) Substance is plainly divisible into other substances, if we consider
the mere nature of a definition. For instance, if 'man' is a substance,
'animal' and 'biped' must also be substances. For if not substances,
they must be attributes-and if attributes, attributes either of (a)
man or of (b) some other subject. But neither is possible.

(a) An attribute is either that which may or may not belong to the
subject or that in whose definition the subject of which it is an
attribute is involved. Thus 'sitting' is an example of a separable
attribute, while 'snubness' contains the definition of 'nose', to
which we attribute snubness. Further, the definition of the whole
is not contained in the definitions of the contents or elements of
the definitory formula; that of 'man' for instance in 'biped', or
that of 'white man' in 'white'. If then this is so, and if 'biped'
is supposed to be an attribute of 'man', it must be either separable,
so that 'man' might possibly not be 'biped', or the definition of
'man' must come into the definition of 'biped'-which is impossible,
as the converse is the case. 

(b) If, on the other hand, we suppose that 'biped' and 'animal' are
attributes not of man but of something else, and are not each of them
a substance, then 'man' too will be an attribute of something else.
But we must assume that substance is not the attribute of anything,
that the subject of which both 'biped' and 'animal' and each separately
are predicated is the subject also of the complex 'biped animal'.

Are we then to say that the All is composed of indivisible substances?
Some thinkers did, in point of fact, give way to both arguments. To
the argument that all things are one if being means one thing, they
conceded that not-being is; to that from bisection, they yielded by
positing atomic magnitudes. But obviously it is not true that if being
means one thing, and cannot at the same time mean the contradictory
of this, there will be nothing which is not, for even if what is not
cannot be without qualification, there is no reason why it should
not be a particular not-being. To say that all things will be one,
if there is nothing besides Being itself, is absurd. For who understands
'being itself' to be anything but a particular substance? But if this
is so, there is nothing to prevent there being many beings, as has
been said. 

It is, then, clearly impossible for Being to be one in this sense.

Part 4

The physicists on the other hand have two modes of explanation.

The first set make the underlying body one either one of the three
or something else which is denser than fire and rarer than air then
generate everything else from this, and obtain multiplicity by condensation
and rarefaction. Now these are contraries, which may be generalized
into 'excess and defect'. (Compare Plato's 'Great and Small'-except
that he make these his matter, the one his form, while the others
treat the one which underlies as matter and the contraries as differentiae,
i.e. forms). 

The second set assert that the contrarieties are contained in the
one and emerge from it by segregation, for example Anaximander and
also all those who assert that 'what is' is one and many, like Empedocles
and Anaxagoras; for they too produce other things from their mixture
by segregation. These differ, however, from each other in that the
former imagines a cycle of such changes, the latter a single series.
Anaxagoras again made both his 'homceomerous' substances and his contraries
infinite in multitude, whereas Empedocles posits only the so-called
elements. 

The theory of Anaxagoras that the principles are infinite in multitude
was probably due to his acceptance of the common opinion of the physicists
that nothing comes into being from not-being. For this is the reason
why they use the phrase 'all things were together' and the coming
into being of such and such a kind of thing is reduced to change of
quality, while some spoke of combination and separation. Moreover,
the fact that the contraries proceed from each other led them to the
conclusion. The one, they reasoned, must have already existed in the
other; for since everything that comes into being must arise either
from what is or from what is not, and it is impossible for it to arise
from what is not (on this point all the physicists agree), they thought
that the truth of the alternative necessarily followed, namely that
things come into being out of existent things, i.e. out of things
already present, but imperceptible to our senses because of the smallness
of their bulk. So they assert that everything has been mixed in every.
thing, because they saw everything arising out of everything. But
things, as they say, appear different from one another and receive
different names according to the nature of the particles which are
numerically predominant among the innumerable constituents of the
mixture. For nothing, they say, is purely and entirely white or black
or sweet, bone or flesh, but the nature of a thing is held to be that
of which it contains the most. 

Now (1) the infinite qua infinite is unknowable, so that what is infinite
in multitude or size is unknowable in quantity, and what is infinite
in variety of kind is unknowable in quality. But the principles in
question are infinite both in multitude and in kind. Therefore it
is impossible to know things which are composed of them; for it is
when we know the nature and quantity of its components that we suppose
we know a complex. 

Further (2) if the parts of a whole may be of any size in the direction
either of greatness or of smallness (by 'parts' I mean components
into which a whole can be divided and which are actually present in
it), it is necessary that the whole thing itself may be of any size.
Clearly, therefore, since it is impossible for an animal or plant
to be indefinitely big or small, neither can its parts be such, or
the whole will be the same. But flesh, bone, and the like are the
parts of animals, and the fruits are the parts of plants. Hence it
is obvious that neither flesh, bone, nor any such thing can be of
indefinite size in the direction either of the greater or of the less.

Again (3) according to the theory all such things are already present
in one another and do not come into being but are constituents which
are separated out, and a thing receives its designation from its chief
constituent. Further, anything may come out of anything-water by segregation
from flesh and flesh from water. Hence, since every finite body is
exhausted by the repeated abstraction of a finite body, it seems obviously
to follow that everything cannot subsist in everything else. For let
flesh be extracted from water and again more flesh be produced from
the remainder by repeating the process of separation: then, even though
the quantity separated out will continually decrease, still it will
not fall below a certain magnitude. If, therefore, the process comes
to an end, everything will not be in everything else (for there will
be no flesh in the remaining water); if on the other hand it does
not, and further extraction is always possible, there will be an infinite
multitude of finite equal particles in a finite quantity-which is
impossible. Another proof may be added: Since every body must diminish
in size when something is taken from it, and flesh is quantitatively
definite in respect both of greatness and smallness, it is clear that
from the minimum quantity of flesh no body can be separated out; for
the flesh left would be less than the minimum of flesh. 

Lastly (4) in each of his infinite bodies there would be already present
infinite flesh and blood and brain- having a distinct existence, however,
from one another, and no less real than the infinite bodies, and each
infinite: which is contrary to reason. 

The statement that complete separation never will take place is correct
enough, though Anaxagoras is not fully aware of what it means. For
affections are indeed inseparable. If then colours and states had
entered into the mixture, and if separation took place, there would
be a 'white' or a 'healthy' which was nothing but white or healthy,
i.e. was not the predicate of a subject. So his 'Mind' is an absurd
person aiming at the impossible, if he is supposed to wish to separate
them, and it is impossible to do so, both in respect of quantity and
of quality- of quantity, because there is no minimum magnitude, and
of quality, because affections are inseparable. 

Nor is Anaxagoras right about the coming to be of homogeneous bodies.
It is true there is a sense in which clay is divided into pieces of
clay, but there is another in which it is not. Water and air are,
and are generated 'from' each other, but not in the way in which bricks
come 'from' a house and again a house 'from' bricks; and it is better
to assume a smaller and finite number of principles, as Empedocles
does. 

Part 5

All thinkers then agree in making the contraries principles, both
those who describe the All as one and unmoved (for even Parmenides
treats hot and cold as principles under the names of fire and earth)
and those too who use the rare and the dense. The same is true of
Democritus also, with his plenum and void, both of which exist, be
says, the one as being, the other as not-being. Again he speaks of
differences in position, shape, and order, and these are genera of
which the species are contraries, namely, of position, above and below,
before and behind; of shape, angular and angle-less, straight and
round. 

It is plain then that they all in one way or another identify the
contraries with the principles. And with good reason. For first principles
must not be derived from one another nor from anything else, while
everything has to be derived from them. But these conditions are fulfilled
by the primary contraries, which are not derived from anything else
because they are primary, nor from each other because they are contraries.

But we must see how this can be arrived at as a reasoned result, as
well as in the way just indicated. 

Our first presupposition must be that in nature nothing acts on, or
is acted on by, any other thing at random, nor may anything come from
anything else, unless we mean that it does so in virtue of a concomitant
attribute. For how could 'white' come from 'musical', unless 'musical'
happened to be an attribute of the not-white or of the black? No,
'white' comes from 'not-white'-and not from any 'not-white', but from
black or some intermediate colour. Similarly, 'musical' comes to be
from 'not-musical', but not from any thing other than musical, but
from 'unmusical' or any intermediate state there may be.

Nor again do things pass into the first chance thing; 'white' does
not pass into 'musical' (except, it may be, in virtue of a concomitant
attribute), but into 'not-white'-and not into any chance thing which
is not white, but into black or an intermediate colour; 'musical'
passes into 'not-musical'-and not into any chance thing other than
musical, but into 'unmusical' or any intermediate state there may
be. 

The same holds of other things also: even things which are not simple
but complex follow the same principle, but the opposite state has
not received a name, so we fail to notice the fact. What is in tune
must come from what is not in tune, and vice versa; the tuned passes
into untunedness-and not into any untunedness, but into the corresponding
opposite. It does not matter whether we take attunement, order, or
composition for our illustration; the principle is obviously the same
in all, and in fact applies equally to the production of a house,
a statue, or any other complex. A house comes from certain things
in a certain state of separation instead of conjunction, a statue
(or any other thing that has been shaped) from shapelessness-each
of these objects being partly order and partly composition.

If then this is true, everything that comes to be or passes away from,
or passes into, its contrary or an intermediate state. But the intermediates
are derived from the contraries-colours, for instance, from black
and white. Everything, therefore, that comes to be by a natural process
is either a contrary or a product of contraries. 

Up to this point we have practically had most of the other writers
on the subject with us, as I have said already: for all of them identify
their elements, and what they call their principles, with the contraries,
giving no reason indeed for the theory, but contrained as it were
by the truth itself. They differ, however, from one another in that
some assume contraries which are more primary, others contraries which
are less so: some those more knowable in the order of explanation,
others those more familiar to sense. For some make hot and cold, or
again moist and dry, the conditions of becoming; while others make
odd and even, or again Love and Strife; and these differ from each
other in the way mentioned. 

Hence their principles are in one sense the same, in another different;
different certainly, as indeed most people think, but the same inasmuch
as they are analogous; for all are taken from the same table of columns,
some of the pairs being wider, others narrower in extent. In this
way then their theories are both the same and different, some better,
some worse; some, as I have said, take as their contraries what is
more knowable in the order of explanation, others what is more familiar
to sense. (The universal is more knowable in the order of explanation,
the particular in the order of sense: for explanation has to do with
the universal, sense with the particular.) 'The great and the small',
for example, belong to the former class, 'the dense and the rare'
to the latter. 

It is clear then that our principles must be contraries.

Part 6

The next question is whether the principles are two or three or more
in number. 

One they cannot be, for there cannot be one contrary. Nor can they
be innumerable, because, if so, Being will not be knowable: and in
any one genus there is only one contrariety, and substance is one
genus: also a finite number is sufficient, and a finite number, such
as the principles of Empedocles, is better than an infinite multitude;
for Empedocles professes to obtain from his principles all that Anaxagoras
obtains from his innumerable principles. Lastly, some contraries are
more primary than others, and some arise from others-for example sweet
and bitter, white and black-whereas the principles must always remain
principles. 

This will suffice to show that the principles are neither one nor
innumerable. 

Granted, then, that they are a limited number, it is plausible to
suppose them more than two. For it is difficult to see how either
density should be of such a nature as to act in any way on rarity
or rarity on density. The same is true of any other pair of contraries;
for Love does not gather Strife together and make things out of it,
nor does Strife make anything out of Love, but both act on a third
thing different from both. Some indeed assume more than one such thing
from which they construct the world of nature. 

Other objections to the view that it is not necessary to assume a
third principle as a substratum may be added. (1) We do not find that
the contraries constitute the substance of any thing. But what is
a first principle ought not to be the predicate of any subject. If
it were, there would be a principle of the supposed principle: for
the subject is a principle, and prior presumably to what is predicated
of it. Again (2) we hold that a substance is not contrary to another
substance. How then can substance be derived from what are not substances?
Or how can non-substances be prior to substance? 

If then we accept both the former argument and this one, we must,
to preserve both, assume a third somewhat as the substratum of the
contraries, such as is spoken of by those who describe the All as
one nature-water or fire or what is intermediate between them. What
is intermediate seems preferable; for fire, earth, air, and water
are already involved with pairs of contraries. There is, therefore,
much to be said for those who make the underlying substance different
from these four; of the rest, the next best choice is air, as presenting
sensible differences in a less degree than the others; and after air,
water. All, however, agree in this, that they differentiate their
One by means of the contraries, such as density and rarity and more
and less, which may of course be generalized, as has already been
said into excess and defect. Indeed this doctrine too (that the One
and excess and defect are the principles of things) would appear to
be of old standing, though in different forms; for the early thinkers
made the two the active and the one the passive principle, whereas
some of the more recent maintain the reverse. 

To suppose then that the elements are three in number would seem,
from these and similar considerations, a plausible view, as I said
before. On the other hand, the view that they are more than three
in number would seem to be untenable. 

For the one substratum is sufficient to be acted on; but if we have
four contraries, there will be two contrarieties, and we shall have
to suppose an intermediate nature for each pair separately. If, on
the other hand, the contrarieties, being two, can generate from each
other, the second contrariety will be superfluous. Moreover, it is
impossible that there should be more than one primary contrariety.
For substance is a single genus of being, so that the principles can
differ only as prior and posterior, not in genus; in a single genus
there is always a single contrariety, all the other contrarieties
in it being held to be reducible to one. 

It is clear then that the number of elements is neither one nor more
than two or three; but whether two or three is, as I said, a question
of considerable difficulty. 

Part 7

We will now give our own account, approaching the question first with
reference to becoming in its widest sense: for we shall be following
the natural order of inquiry if we speak first of common characteristics,
and then investigate the characteristics of special cases.

We say that one thing comes to be from another thing, and one sort
of thing from another sort of thing, both in the case of simple and
of complex things. I mean the following. We can say (1) 'man becomes
musical', (2) what is 'not-musical becomes musical', or (3), the 'not-musical
man becomes a musical man'. Now what becomes in (1) and (2)-'man'
and 'not musical'-I call simple, and what each becomes-'musical'-simple
also. But when (3) we say the 'not-musical man becomes a musical man',
both what becomes and what it becomes are complex. 

As regards one of these simple 'things that become' we say not only
'this becomes so-and-so', but also 'from being this, comes to be so-and-so',
as 'from being not-musical comes to be musical'; as regards the other
we do not say this in all cases, as we do not say (1) 'from being
a man he came to be musical' but only 'the man became musical'.

When a 'simple' thing is said to become something, in one case (1)
it survives through the process, in the other (2) it does not. For
man remains a man and is such even when he becomes musical, whereas
what is not musical or is unmusical does not continue to exist, either
simply or combined with the subject. 

These distinctions drawn, one can gather from surveying the various
cases of becoming in the way we are describing that, as we say, there
must always be an underlying something, namely that which becomes,
and that this, though always one numerically, in form at least is
not one. (By that I mean that it can be described in different ways.)
For 'to be man' is not the same as 'to be unmusical'. One part survives,
the other does not: what is not an opposite survives (for 'man' survives),
but 'not-musical' or 'unmusical' does not survive, nor does the compound
of the two, namely 'unmusical man'. 

We speak of 'becoming that from this' instead of 'this becoming that'
more in the case of what does not survive the change-'becoming musical
from unmusical', not 'from man'-but there are exceptions, as we sometimes
use the latter form of expression even of what survives; we speak
of 'a statue coming to be from bronze', not of the 'bronze becoming
a statue'. The change, however, from an opposite which does not survive
is described indifferently in both ways, 'becoming that from this'
or 'this becoming that'. We say both that 'the unmusical becomes musical',
and that 'from unmusical he becomes musical'. And so both forms are
used of the complex, 'becoming a musical man from an unmusical man',
and unmusical man becoming a musical man'. 

But there are different senses of 'coming to be'. In some cases we
do not use the expression 'come to be', but 'come to be so-and-so'.
Only substances are said to 'come to be' in the unqualified sense.

Now in all cases other than substance it is plain that there must
be some subject, namely, that which becomes. For we know that when
a thing comes to be of such a quantity or quality or in such a relation,
time, or place, a subject is always presupposed, since substance alone
is not predicated of another subject, but everything else of substance.

But that substances too, and anything else that can be said 'to be'
without qualification, come to be from some substratum, will appear
on examination. For we find in every case something that underlies
from which proceeds that which comes to be; for instance, animals
and plants from seed. 

Generally things which come to be, come to be in different ways: (1)
by change of shape, as a statue; (2) by addition, as things which
grow; (3) by taking away, as the Hermes from the stone; (4) by putting
together, as a house; (5) by alteration, as things which 'turn' in
respect of their material substance. 

It is plain that these are all cases of coming to be from a substratum.

Thus, clearly, from what has been said, whatever comes to be is always
complex. There is, on the one hand, (a) something which comes into
existence, and again (b) something which becomes that-the latter (b)
in two senses, either the subject or the opposite. By the 'opposite'
I mean the 'unmusical', by the 'subject' 'man', and similarly I call
the absence of shape or form or order the 'opposite', and the bronze
or stone or gold the 'subject'. 

Plainly then, if there are conditions and principles which constitute
natural objects and from which they primarily are or have come to
be-have come to be, I mean, what each is said to be in its essential
nature, not what each is in respect of a concomitant attribute-plainly,
I say, everything comes to be from both subject and form. For 'musical
man' is composed (in a way) of 'man' and 'musical': you can analyse
it into the definitions of its elements. It is clear then that what
comes to be will come to be from these elements. 

Now the subject is one numerically, though it is two in form. (For
it is the man, the gold-the 'matter' generally-that is counted, for
it is more of the nature of a 'this', and what comes to be does not
come from it in virtue of a concomitant attribute; the privation,
on the other hand, and the contrary are incidental in the process.)
And the positive form is one-the order, the acquired art of music,
or any similar predicate. 

There is a sense, therefore, in which we must declare the principles
to be two, and a sense in which they are three; a sense in which the
contraries are the principles-say for example the musical and the
unmusical, the hot and the cold, the tuned and the untuned-and a sense
in which they are not, since it is impossible for the contraries to
be acted on by each other. But this difficulty also is solved by the
fact that the substratum is different from the contraries, for it
is itself not a contrary. The principles therefore are, in a way,
not more in number than the contraries, but as it were two, nor yet
precisely two, since there is a difference of essential nature, but
three. For 'to be man' is different from 'to be unmusical', and 'to
be unformed' from 'to be bronze'. 

We have now stated the number of the principles of natural objects
which are subject to generation, and how the number is reached: and
it is clear that there must be a substratum for the contraries, and
that the contraries must be two. (Yet in another way of putting it
this is not necessary, as one of the contraries will serve to effect
the change by its successive absence and presence.) 

The underlying nature is an object of scientific knowledge, by an
analogy. For as the bronze is to the statue, the wood to the bed,
or the matter and the formless before receiving form to any thing
which has form, so is the underlying nature to substance, i.e. the
'this' or existent. 

This then is one principle (though not one or existent in the same
sense as the 'this'), and the definition was one as we agreed; then
further there is its contrary, the privation. In what sense these
are two, and in what sense more, has been stated above. Briefly, we
explained first that only the contraries were principles, and later
that a substratum was indispensable, and that the principles were
three; our last statement has elucidated the difference between the
contraries, the mutual relation of the principles, and the nature
of the substratum. Whether the form or the substratum is the essential
nature of a physical object is not yet clear. But that the principles
are three, and in what sense, and the way in which each is a principle,
is clear. 

So much then for the question of the number and the nature of the
principles. 

Part 8

We will now proceed to show that the difficulty of the early thinkers,
as well as our own, is solved in this way alone. 

The first of those who studied science were misled in their search
for truth and the nature of things by their inexperience, which as
it were thrust them into another path. So they say that none of the
things that are either comes to be or passes out of existence, because
what comes to be must do so either from what is or from what is not,
both of which are impossible. For what is cannot come to be (because
it is already), and from what is not nothing could have come to be
(because something must be present as a substratum). So too they exaggerated
the consequence of this, and went so far as to deny even the existence
of a plurality of things, maintaining that only Being itself is. Such
then was their opinion, and such the reason for its adoption.

Our explanation on the other hand is that the phrases 'something comes
to be from what is or from what is not', 'what is not or what is does
something or has something done to it or becomes some particular thing',
are to be taken (in the first way of putting our explanation) in the
same sense as 'a doctor does something or has something done to him',
'is or becomes something from being a doctor.' These expressions may
be taken in two senses, and so too, clearly, may 'from being', and
'being acts or is acted on'. A doctor builds a house, not qua doctor,
but qua housebuilder, and turns gray, not qua doctor, but qua dark-haired.
On the other hand he doctors or fails to doctor qua doctor. But we
are using words most appropriately when we say that a doctor does
something or undergoes something, or becomes something from being
a doctor, if he does, undergoes, or becomes qua doctor. Clearly then
also 'to come to be so-and-so from not-being' means 'qua not-being'.

It was through failure to make this distinction that those thinkers
gave the matter up, and through this error that they went so much
farther astray as to suppose that nothing else comes to be or exists
apart from Being itself, thus doing away with all becoming.

We ourselves are in agreement with them in holding that nothing can
be said without qualification to come from what is not. But nevertheless
we maintain that a thing may 'come to be from what is not'-that is,
in a qualified sense. For a thing comes to be from the privation,
which in its own nature is not-being,-this not surviving as a constituent
of the result. Yet this causes surprise, and it is thought impossible
that something should come to be in the way described from what is
not. 

In the same way we maintain that nothing comes to be from being, and
that being does not come to be except in a qualified sense. In that
way, however, it does, just as animal might come to be from animal,
and an animal of a certain kind from an animal of a certain kind.
Thus, suppose a dog to come to be from a horse. The dog would then,
it is true, come to be from animal (as well as from an animal of a
certain kind) but not as animal, for that is already there. But if
anything is to become an animal, not in a qualified sense, it will
not be from animal: and if being, not from being-nor from not-being
either, for it has been explained that by 'from not being' we mean
from not-being qua not-being. 

Note further that we do not subvert the principle that everything
either is or is not. 

This then is one way of solving the difficulty. Another consists in
pointing out that the same things can be explained in terms of potentiality
and actuality. But this has been done with greater precision elsewhere.
So, as we said, the difficulties which constrain people to deny the
existence of some of the things we mentioned are now solved. For it
was this reason which also caused some of the earlier thinkers to
turn so far aside from the road which leads to coming to be and passing
away and change generally. If they had come in sight of this nature,
all their ignorance would have been dispelled. 

Part 9

Others, indeed, have apprehended the nature in question, but not adequately.

In the first place they allow that a thing may come to be without
qualification from not being, accepting on this point the statement
of Parmenides. Secondly, they think that if the substratum is one
numerically, it must have also only a single potentiality-which is
a very different thing. 

Now we distinguish matter and privation, and hold that one of these,
namely the matter, is not-being only in virtue of an attribute which
it has, while the privation in its own nature is not-being; and that
the matter is nearly, in a sense is, substance, while the privation
in no sense is. They, on the other hand, identify their Great and
Small alike with not being, and that whether they are taken together
as one or separately. Their triad is therefore of quite a different
kind from ours. For they got so far as to see that there must be some
underlying nature, but they make it one-for even if one philosopher
makes a dyad of it, which he calls Great and Small, the effect is
the same, for he overlooked the other nature. For the one which persists
is a joint cause, with the form, of what comes to be-a mother, as
it were. But the negative part of the contrariety may often seem,
if you concentrate your attention on it as an evil agent, not to exist
at all. 

For admitting with them that there is something divine, good, and
desirable, we hold that there are two other principles, the one contrary
to it, the other such as of its own nature to desire and yearn for
it. But the consequence of their view is that the contrary desires
its wtextinction. Yet the form cannot desire itself, for it is not
defective; nor can the contrary desire it, for contraries are mutually
destructive. The truth is that what desires the form is matter, as
the female desires the male and the ugly the beautiful-only the ugly
or the female not per se but per accidens. 

The matter comes to be and ceases to be in one sense, while in another
it does not. As that which contains the privation, it ceases to be
in its own nature, for what ceases to be-the privation-is contained
within it. But as potentiality it does not cease to be in its own
nature, but is necessarily outside the sphere of becoming and ceasing
to be. For if it came to be, something must have existed as a primary
substratum from which it should come and which should persist in it;
but this is its own special nature, so that it will be before coming
to be. (For my definition of matter is just this-the primary substratum
of each thing, from which it comes to be without qualification, and
which persists in the result.) And if it ceases to be it will pass
into that at the last, so it will have ceased to be before ceasing
to be. 

The accurate determination of the first principle in respect of form,
whether it is one or many and what it is or what they are, is the
province of the primary type of science; so these questions may stand
over till then. But of the natural, i.e. perishable, forms we shall
speak in the expositions which follow. 

The above, then, may be taken as sufficient to establish that there
are principles and what they are and how many there are. Now let us
make a fresh start and proceed. 

----------------------------------------------------------------------

BOOK II

Part 1 

Of things that exist, some exist by nature, some from other causes.

'By nature' the animals and their parts exist, and the plants and
the simple bodies (earth, fire, air, water)-for we say that these
and the like exist 'by nature'. 

All the things mentioned present a feature in which they differ from
things which are not constituted by nature. Each of them has within
itself a principle of motion and of stationariness (in respect of
place, or of growth and decrease, or by way of alteration). On the
other hand, a bed and a coat and anything else of that sort, qua receiving
these designations i.e. in so far as they are products of art-have
no innate impulse to change. But in so far as they happen to be composed
of stone or of earth or of a mixture of the two, they do have such
an impulse, and just to that extent which seems to indicate that nature
is a source or cause of being moved and of being at rest in that to
which it belongs primarily, in virtue of itself and not in virtue
of a concomitant attribute. 

I say 'not in virtue of a concomitant attribute', because (for instance)
a man who is a doctor might cure himself. Nevertheless it is not in
so far as he is a patient that he possesses the art of medicine: it
merely has happened that the same man is doctor and patient-and that
is why these attributes are not always found together. So it is with
all other artificial products. None of them has in itself the source
of its own production. But while in some cases (for instance houses
and the other products of manual labour) that principle is in something
else external to the thing, in others those which may cause a change
in themselves in virtue of a concomitant attribute-it lies in the
things themselves (but not in virtue of what they are). 

'Nature' then is what has been stated. Things 'have a nature'which
have a principle of this kind. Each of them is a substance; for it
is a subject, and nature always implies a subject in which it inheres.

The term 'according to nature' is applied to all these things and
also to the attributes which belong to them in virtue of what they
are, for instance the property of fire to be carried upwards-which
is not a 'nature' nor 'has a nature' but is 'by nature' or 'according
to nature'. 

What nature is, then, and the meaning of the terms 'by nature' and
'according to nature', has been stated. That nature exists, it would
be absurd to try to prove; for it is obvious that there are many things
of this kind, and to prove what is obvious by what is not is the mark
of a man who is unable to distinguish what is self-evident from what
is not. (This state of mind is clearly possible. A man blind from
birth might reason about colours. Presumably therefore such persons
must be talking about words without any thought to correspond.)

Some identify the nature or substance of a natural object with that
immediate constituent of it which taken by itself is without arrangement,
e.g. the wood is the 'nature' of the bed, and the bronze the 'nature'
of the statue. 

As an indication of this Antiphon points out that if you planted a
bed and the rotting wood acquired the power of sending up a shoot,
it would not be a bed that would come up, but wood-which shows that
the arrangement in accordance with the rules of the art is merely
an incidental attribute, whereas the real nature is the other, which,
further, persists continuously through the process of making.

But if the material of each of these objects has itself the same relation
to something else, say bronze (or gold) to water, bones (or wood)
to earth and so on, that (they say) would be their nature and essence.
Consequently some assert earth, others fire or air or water or some
or all of these, to be the nature of the things that are. For whatever
any one of them supposed to have this character-whether one thing
or more than one thing-this or these he declared to be the whole of
substance, all else being its affections, states, or dispositions.
Every such thing they held to be eternal (for it could not pass into
anything else), but other things to come into being and cease to be
times without number. 

This then is one account of 'nature', namely that it is the immediate
material substratum of things which have in themselves a principle
of motion or change. 

Another account is that 'nature' is the shape or form which is specified
in the definition of the thing. 

For the word 'nature' is applied to what is according to nature and
the natural in the same way as 'art' is applied to what is artistic
or a work of art. We should not say in the latter case that there
is anything artistic about a thing, if it is a bed only potentially,
not yet having the form of a bed; nor should we call it a work of
art. The same is true of natural compounds. What is potentially flesh
or bone has not yet its own 'nature', and does not exist until it
receives the form specified in the definition, which we name in defining
what flesh or bone is. Thus in the second sense of 'nature' it would
be the shape or form (not separable except in statement) of things
which have in themselves a source of motion. (The combination of the
two, e.g. man, is not 'nature' but 'by nature' or 'natural'.)

The form indeed is 'nature' rather than the matter; for a thing is
more properly said to be what it is when it has attained to fulfilment
than when it exists potentially. Again man is born from man, but not
bed from bed. That is why people say that the figure is not the nature
of a bed, but the wood is-if the bed sprouted not a bed but wood would
come up. But even if the figure is art, then on the same principle
the shape of man is his nature. For man is born from man.

We also speak of a thing's nature as being exhibited in the process
of growth by which its nature is attained. The 'nature' in this sense
is not like 'doctoring', which leads not to the art of doctoring but
to health. Doctoring must start from the art, not lead to it. But
it is not in this way that nature (in the one sense) is related to
nature (in the other). What grows qua growing grows from something
into something. Into what then does it grow? Not into that from which
it arose but into that to which it tends. The shape then is nature.

'Shape' and 'nature', it should be added, are in two senses. For the
privation too is in a way form. But whether in unqualified coming
to be there is privation, i.e. a contrary to what comes to be, we
must consider later. 

Part 2

We have distinguished, then, the different ways in which the term
'nature' is used. 

The next point to consider is how the mathematician differs from the
physicist. Obviously physical bodies contain surfaces and volumes,
lines and points, and these are the subject-matter of mathematics.

Further, is astronomy different from physics or a department of it?
It seems absurd that the physicist should be supposed to know the
nature of sun or moon, but not to know any of their essential attributes,
particularly as the writers on physics obviously do discuss their
shape also and whether the earth and the world are spherical or not.

Now the mathematician, though he too treats of these things, nevertheless
does not treat of them as the limits of a physical body; nor does
he consider the attributes indicated as the attributes of such bodies.
That is why he separates them; for in thought they are separable from
motion, and it makes no difference, nor does any falsity result, if
they are separated. The holders of the theory of Forms do the same,
though they are not aware of it; for they separate the objects of
physics, which are less separable than those of mathematics. This
becomes plain if one tries to state in each of the two cases the definitions
of the things and of their attributes. 'Odd' and 'even', 'straight'
and 'curved', and likewise 'number', 'line', and 'figure', do not
involve motion; not so 'flesh' and 'bone' and 'man'-these are defined
like 'snub nose', not like 'curved'. 

Similar evidence is supplied by the more physical of the branches
of mathematics, such as optics, harmonics, and astronomy. These are
in a way the converse of geometry. While geometry investigates physical
lines but not qua physical, optics investigates mathematical lines,
but qua physical, not qua mathematical. 

Since 'nature' has two senses, the form and the matter, we must investigate
its objects as we would the essence of snubness. That is, such things
are neither independent of matter nor can be defined in terms of matter
only. Here too indeed one might raise a difficulty. Since there are
two natures, with which is the physicist concerned? Or should he investigate
the combination of the two? But if the combination of the two, then
also each severally. Does it belong then to the same or to different
sciences to know each severally? 

If we look at the ancients, physics would to be concerned with the
matter. (It was only very slightly that Empedocles and Democritus
touched on the forms and the essence.) 

But if on the other hand art imitates nature, and it is the part of
the same discipline to know the form and the matter up to a point
(e.g. the doctor has a knowledge of health and also of bile and phlegm,
in which health is realized, and the builder both of the form of the
house and of the matter, namely that it is bricks and beams, and so
forth): if this is so, it would be the part of physics also to know
nature in both its senses. 

Again, 'that for the sake of which', or the end, belongs to the same
department of knowledge as the means. But the nature is the end or
'that for the sake of which'. For if a thing undergoes a continuous
change and there is a stage which is last, this stage is the end or
'that for the sake of which'. (That is why the poet was carried away
into making an absurd statement when he said 'he has the end for the
sake of which he was born'. For not every stage that is last claims
to be an end, but only that which is best.) 

For the arts make their material (some simply 'make' it, others make
it serviceable), and we use everything as if it was there for our
sake. (We also are in a sense an end. 'That for the sake of which'
has two senses: the distinction is made in our work On Philosophy.)
The arts, therefore, which govern the matter and have knowledge are
two, namely the art which uses the product and the art which directs
the production of it. That is why the using art also is in a sense
directive; but it differs in that it knows the form, whereas the art
which is directive as being concerned with production knows the matter.
For the helmsman knows and prescribes what sort of form a helm should
have, the other from what wood it should be made and by means of what
operations. In the products of art, however, we make the material
with a view to the function, whereas in the products of nature the
matter is there all along. 

Again, matter is a relative term: to each form there corresponds a
special matter. How far then must the physicist know the form or essence?
Up to a point, perhaps, as the doctor must know sinew or the smith
bronze (i.e. until he understands the purpose of each): and the physicist
is concerned only with things whose forms are separable indeed, but
do not exist apart from matter. Man is begotten by man and by the
sun as well. The mode of existence and essence of the separable it
is the business of the primary type of philosophy to define.

Part 3

Now that we have established these distinctions, we must proceed to
consider causes, their character and number. Knowledge is the object
of our inquiry, and men do not think they know a thing till they have
grasped the 'why' of (which is to grasp its primary cause). So clearly
we too must do this as regards both coming to be and passing away
and every kind of physical change, in order that, knowing their principles,
we may try to refer to these principles each of our problems.

In one sense, then, (1) that out of which a thing comes to be and
which persists, is called 'cause', e.g. the bronze of the statue,
the silver of the bowl, and the genera of which the bronze and the
silver are species. 

In another sense (2) the form or the archetype, i.e. the statement
of the essence, and its genera, are called 'causes' (e.g. of the octave
the relation of 2:1, and generally number), and the parts in the definition.

Again (3) the primary source of the change or coming to rest; e.g.
the man who gave advice is a cause, the father is cause of the child,
and generally what makes of what is made and what causes change of
what is changed. 

Again (4) in the sense of end or 'that for the sake of which' a thing
is done, e.g. health is the cause of walking about. ('Why is he walking
about?' we say. 'To be healthy', and, having said that, we think we
have assigned the cause.) The same is true also of all the intermediate
steps which are brought about through the action of something else
as means towards the end, e.g. reduction of flesh, purging, drugs,
or surgical instruments are means towards health. All these things
are 'for the sake of' the end, though they differ from one another
in that some are activities, others instruments. 

This then perhaps exhausts the number of ways in which the term 'cause'
is used. 

As the word has several senses, it follows that there are several
causes of the same thing not merely in virtue of a concomitant attribute),
e.g. both the art of the sculptor and the bronze are causes of the
statue. These are causes of the statue qua statue, not in virtue of
anything else that it may be-only not in the same way, the one being
the material cause, the other the cause whence the motion comes. Some
things cause each other reciprocally, e.g. hard work causes fitness
and vice versa, but again not in the same way, but the one as end,
the other as the origin of change. Further the same thing is the cause
of contrary results. For that which by its presence brings about one
result is sometimes blamed for bringing about the contrary by its
absence. Thus we ascribe the wreck of a ship to the absence of the
pilot whose presence was the cause of its safety. 

All the causes now mentioned fall into four familiar divisions. The
letters are the causes of syllables, the material of artificial products,
fire, &c., of bodies, the parts of the whole, and the premisses of
the conclusion, in the sense of 'that from which'. Of these pairs
the one set are causes in the sense of substratum, e.g. the parts,
the other set in the sense of essence-the whole and the combination
and the form. But the seed and the doctor and the adviser, and generally
the maker, are all sources whence the change or stationariness originates,
while the others are causes in the sense of the end or the good of
the rest; for 'that for the sake of which' means what is best and
the end of the things that lead up to it. (Whether we say the 'good
itself or the 'apparent good' makes no difference.) 

Such then is the number and nature of the kinds of cause.

Now the modes of causation are many, though when brought under heads
they too can be reduced in number. For 'cause' is used in many senses
and even within the same kind one may be prior to another (e.g. the
doctor and the expert are causes of health, the relation 2:1 and number
of the octave), and always what is inclusive to what is particular.
Another mode of causation is the incidental and its genera, e.g. in
one way 'Polyclitus', in another 'sculptor' is the cause of a statue,
because 'being Polyclitus' and 'sculptor' are incidentally conjoined.
Also the classes in which the incidental attribute is included; thus
'a man' could be said to be the cause of a statue or, generally, 'a
living creature'. An incidental attribute too may be more or less
remote, e.g. suppose that 'a pale man' or 'a musical man' were said
to be the cause of the statue. 

All causes, both proper and incidental, may be spoken of either as
potential or as actual; e.g. the cause of a house being built is either
'house-builder' or 'house-builder building'. 

Similar distinctions can be made in the things of which the causes
are causes, e.g. of 'this statue' or of 'statue' or of 'image' generally,
of 'this bronze' or of 'bronze' or of 'material' generally. So too
with the incidental attributes. Again we may use a complex expression
for either and say, e.g. neither 'Polyclitus' nor 'sculptor' but 'Polyclitus,
sculptor'. 

All these various uses, however, come to six in number, under each
of which again the usage is twofold. Cause means either what is particular
or a genus, or an incidental attribute or a genus of that, and these
either as a complex or each by itself; and all six either as actual
or as potential. The difference is this much, that causes which are
actually at work and particular exist and cease to exist simultaneously
with their effect, e.g. this healing person with this being-healed
person and that house-building man with that being-built house; but
this is not always true of potential causes--the house and the housebuilder
do not pass away simultaneously. 

In investigating the cause of each thing it is always necessary to
seek what is most precise (as also in other things): thus man builds
because he is a builder, and a builder builds in virtue of his art
of building. This last cause then is prior: and so generally.

Further, generic effects should be assigned to generic causes, particular
effects to particular causes, e.g. statue to sculptor, this statue
to this sculptor; and powers are relative to possible effects, actually
operating causes to things which are actually being effected.

This must suffice for our account of the number of causes and the
modes of causation. 

Part 4

But chance also and spontaneity are reckoned among causes: many things
are said both to be and to come to be as a result of chance and spontaneity.
We must inquire therefore in what manner chance and spontaneity are
present among the causes enumerated, and whether they are the same
or different, and generally what chance and spontaneity are.

Some people even question whether they are real or not. They say that
nothing happens by chance, but that everything which we ascribe to
chance or spontaneity has some definite cause, e.g. coming 'by chance'
into the market and finding there a man whom one wanted but did not
expect to meet is due to one's wish to go and buy in the market. Similarly
in other cases of chance it is always possible, they maintain, to
find something which is the cause; but not chance, for if chance were
real, it would seem strange indeed, and the question might be raised,
why on earth none of the wise men of old in speaking of the causes
of generation and decay took account of chance; whence it would seem
that they too did not believe that anything is by chance. But there
is a further circumstance that is surprising. Many things both come
to be and are by chance and spontaneity, and although know that each
of them can be ascribed to some cause (as the old argument said which
denied chance), nevertheless they speak of some of these things as
happening by chance and others not. For this reason also they ought
to have at least referred to the matter in some way or other.

Certainly the early physicists found no place for chance among the
causes which they recognized-love, strife, mind, fire, or the like.
This is strange, whether they supposed that there is no such thing
as chance or whether they thought there is but omitted to mention
it-and that too when they sometimes used it, as Empedocles does when
he says that the air is not always separated into the highest region,
but 'as it may chance'. At any rate he says in his cosmogony that
'it happened to run that way at that time, but it often ran otherwise.'
He tells us also that most of the parts of animals came to be by chance.

There are some too who ascribe this heavenly sphere and all the worlds
to spontaneity. They say that the vortex arose spontaneously, i.e.
the motion that separated and arranged in its present order all that
exists. This statement might well cause surprise. For they are asserting
that chance is not responsible for the existence or generation of
animals and plants, nature or mind or something of the kind being
the cause of them (for it is not any chance thing that comes from
a given seed but an olive from one kind and a man from another); and
yet at the same time they assert that the heavenly sphere and the
divinest of visible things arose spontaneously, having no such cause
as is assigned to animals and plants. Yet if this is so, it is a fact
which deserves to be dwelt upon, and something might well have been
said about it. For besides the other absurdities of the statement,
it is the more absurd that people should make it when they see nothing
coming to be spontaneously in the heavens, but much happening by chance
among the things which as they say are not due to chance; whereas
we should have expected exactly the opposite. 

Others there are who, indeed, believe that chance is a cause, but
that it is inscrutable to human intelligence, as being a divine thing
and full of mystery. 

Thus we must inquire what chance and spontaneity are, whether they
are the same or different, and how they fit into our division of causes.

Part 5

First then we observe that some things always come to pass in the
same way, and others for the most part. It is clearly of neither of
these that chance is said to be the cause, nor can the 'effect of
chance' be identified with any of the things that come to pass by
necessity and always, or for the most part. But as there is a third
class of events besides these two-events which all say are 'by chance'-it
is plain that there is such a thing as chance and spontaneity; for
we know that things of this kind are due to chance and that things
due to chance are of this kind. 

But, secondly, some events are for the sake of something, others not.
Again, some of the former class are in accordance with deliberate
intention, others not, but both are in the class of things which are
for the sake of something. Hence it is clear that even among the things
which are outside the necessary and the normal, there are some in
connexion withwhich the phrase 'for the sake of something' is applicable.
(Events that are for the sake of something include whatever may be
done as a result of thought or of nature.) Things of this kind, then,
when they come to pass incidental are said to be 'by chance'. For
just as a thing is something either in virtue of itself or incidentally,
so may it be a cause. For instance, the housebuilding faculty is in
virtue of itself the cause of a house, whereas the pale or the musical
is the incidental cause. That which is per se cause of the effect
is determinate, but the incidental cause is indeterminable, for the
possible attributes of an individual are innumerable. To resume then;
when a thing of this kind comes to pass among events which are for
the sake of something, it is said to be spontaneous or by chance.
(The distinction between the two must be made later-for the present
it is sufficient if it is plain that both are in the sphere of things
done for the sake of something.) 

Example: A man is engaged in collecting subscriptions for a feast.
He would have gone to such and such a place for the purpose of getting
the money, if he had known. He actually went there for another purpose
and it was only incidentally that he got his money by going there;
and this was not due to the fact that he went there as a rule or necessarily,
nor is the end effected (getting the money) a cause present in himself-it
belongs to the class of things that are intentional and the result
of intelligent deliberation. It is when these conditions are satisfied
that the man is said to have gone 'by chance'. If he had gone of deliberate
purpose and for the sake of this-if he always or normally went there
when he was collecting payments-he would not be said to have gone
'by chance'. 

It is clear then that chance is an incidental cause in the sphere
of those actions for the sake of something which involve purpose.
Intelligent reflection, then, and chance are in the same sphere, for
purpose implies intelligent reflection. 

It is necessary, no doubt, that the causes of what comes to pass by
chance be indefinite; and that is why chance is supposed to belong
to the class of the indefinite and to be inscrutable to man, and why
it might be thought that, in a way, nothing occurs by chance. For
all these statements are correct, because they are well grounded.
Things do, in a way, occur by chance, for they occur incidentally
and chance is an incidental cause. But strictly it is not the cause-without
qualification-of anything; for instance, a housebuilder is the cause
of a house; incidentally, a fluteplayer may be so. 

And the causes of the man's coming and getting the money (when he
did not come for the sake of that) are innumerable. He may have wished
to see somebody or been following somebody or avoiding somebody, or
may have gone to see a spectacle. Thus to say that chance is a thing
contrary to rule is correct. For 'rule' applies to what is always
true or true for the most part, whereas chance belongs to a third
type of event. Hence, to conclude, since causes of this kind are indefinite,
chance too is indefinite. (Yet in some cases one might raise the question
whether any incidental fact might be the cause of the chance occurrence,
e.g. of health the fresh air or the sun's heat may be the cause, but
having had one's hair cut cannot; for some incidental causes are more
relevant to the effect than others.) 

Chance or fortune is called 'good' when the result is good, 'evil'
when it is evil. The terms 'good fortune' and 'ill fortune' are used
when either result is of considerable magnitude. Thus one who comes
within an ace of some great evil or great good is said to be fortunate
or unfortunate. The mind affirms the essence of the attribute, ignoring
the hair's breadth of difference. Further, it is with reason that
good fortune is regarded as unstable; for chance is unstable, as none
of the things which result from it can be invariable or normal.

Both are then, as I have said, incidental causes-both chance and spontaneity-in
the sphere of things which are capable of coming to pass not necessarily,
nor normally, and with reference to such of these as might come to
pass for the sake of something. 

Part 6

They differ in that 'spontaneity' is the wider term. Every result
of chance is from what is spontaneous, but not everything that is
from what is spontaneous is from chance. 

Chance and what results from chance are appropriate to agents that
are capable of good fortune and of moral action generally. Therefore
necessarily chance is in the sphere of moral actions. This is indicated
by the fact that good fortune is thought to be the same, or nearly
the same, as happiness, and happiness to be a kind of moral action,
since it is well-doing. Hence what is not capable of moral action
cannot do anything by chance. Thus an inanimate thing or a lower animal
or a child cannot do anything by chance, because it is incapable of
deliberate intention; nor can 'good fortune' or 'ill fortune' be ascribed
to them, except metaphorically, as Protarchus, for example, said that
the stones of which altars are made are fortunate because they are
held in honour, while their fellows are trodden under foot. Even these
things, however, can in a way be affected by chance, when one who
is dealing with them does something to them by chance, but not otherwise.

The spontaneous on the other hand is found both in the lower animals
and in many inanimate objects. We say, for example, that the horse
came 'spontaneously', because, though his coming saved him, he did
not come for the sake of safety. Again, the tripod fell 'of itself',
because, though when it fell it stood on its feet so as to serve for
a seat, it did not fall for the sake of that. 

Hence it is clear that events which (1) belong to the general class
of things that may come to pass for the sake of something, (2) do
not come to pass for the sake of what actually results, and (3) have
an external cause, may be described by the phrase 'from spontaneity'.
These 'spontaneous' events are said to be 'from chance' if they have
the further characteristics of being the objects of deliberate intention
and due to agents capable of that mode of action. This is indicated
by the phrase 'in vain', which is used when A which is for the sake
of B, does not result in B. For instance, taking a walk is for the
sake of evacuation of the bowels; if this does not follow after walking,
we say that we have walked 'in vain' and that the walking was 'vain'.
This implies that what is naturally the means to an end is 'in vain',
when it does not effect the end towards which it was the natural means-for
it would be absurd for a man to say that he had bathed in vain because
the sun was not eclipsed, since the one was not done with a view to
the other. Thus the spontaneous is even according to its derivation
the case in which the thing itself happens in vain. The stone that
struck the man did not fall for the purpose of striking him; therefore
it fell spontaneously, because it might have fallen by the action
of an agent and for the purpose of striking. The difference between
spontaneity and what results by chance is greatest in things that
come to be by nature; for when anything comes to be contrary to nature,
we do not say that it came to be by chance, but by spontaneity. Yet
strictly this too is different from the spontaneous proper; for the
cause of the latter is external, that of the former internal.

We have now explained what chance is and what spontaneity is, and
in what they differ from each other. Both belong to the mode of causation
'source of change', for either some natural or some intelligent agent
is always the cause; but in this sort of causation the number of possible
causes is infinite. 

Spontaneity and chance are causes of effects which though they might
result from intelligence or nature, have in fact been caused by something
incidentally. Now since nothing which is incidental is prior to what
is per se, it is clear that no incidental cause can be prior to a
cause per se. Spontaneity and chance, therefore, are posterior to
intelligence and nature. Hence, however true it may be that the heavens
are due to spontaneity, it will still be true that intelligence and
nature will be prior causes of this All and of many things in it besides.

Part 7

It is clear then that there are causes, and that the number of them
is what we have stated. The number is the same as that of the things
comprehended under the question 'why'. The 'why' is referred ultimately
either (1), in things which do not involve motion, e.g. in mathematics,
to the 'what' (to the definition of 'straight line' or 'commensurable',
&c.), or (2) to what initiated a motion, e.g. 'why did they go to
war?-because there had been a raid'; or (3) we are inquiring 'for
the sake of what?'-'that they may rule'; or (4), in the case of things
that come into being, we are looking for the matter. The causes, therefore,
are these and so many in number. 

Now, the causes being four, it is the business of the physicist to
know about them all, and if he refers his problems back to all of
them, he will assign the 'why' in the way proper to his science-the
matter, the form, the mover, 'that for the sake of which'. The last
three often coincide; for the 'what' and 'that for the sake of which'
are one, while the primary source of motion is the same in species
as these (for man generates man), and so too, in general, are all
things which cause movement by being themselves moved; and such as
are not of this kind are no longer inside the province of physics,
for they cause motion not by possessing motion or a source of motion
in themselves, but being themselves incapable of motion. Hence there
are three branches of study, one of things which are incapable of
motion, the second of things in motion, but indestructible, the third
of destructible things. 

The question 'why', then, is answered by reference to the matter,
to the form, and to the primary moving cause. For in respect of coming
to be it is mostly in this last way that causes are investigated-'what
comes to be after what? what was the primary agent or patient?' and
so at each step of the series. 

Now the principles which cause motion in a physical way are two, of
which one is not physical, as it has no principle of motion in itself.
Of this kind is whatever causes movement, not being itself moved,
such as (1) that which is completely unchangeable, the primary reality,
and (2) the essence of that which is coming to be, i.e. the form;
for this is the end or 'that for the sake of which'. Hence since nature
is for the sake of something, we must know this cause also. We must
explain the 'why' in all the senses of the term, namely, (1) that
from this that will necessarily result ('from this' either without
qualification or in most cases); (2) that 'this must be so if that
is to be so' (as the conclusion presupposes the premisses); (3) that
this was the essence of the thing; and (4) because it is better thus
(not without qualification, but with reference to the essential nature
in each case). 

Part 8

We must explain then (1) that Nature belongs to the class of causes
which act for the sake of something; (2) about the necessary and its
place in physical problems, for all writers ascribe things to this
cause, arguing that since the hot and the cold, &c., are of such and
such a kind, therefore certain things necessarily are and come to
be-and if they mention any other cause (one his 'friendship and strife',
another his 'mind'), it is only to touch on it, and then good-bye
to it. 

A difficulty presents itself: why should not nature work, not for
the sake of something, nor because it is better so, but just as the
sky rains, not in order to make the corn grow, but of necessity? What
is drawn up must cool, and what has been cooled must become water
and descend, the result of this being that the corn grows. Similarly
if a man's crop is spoiled on the threshing-floor, the rain did not
fall for the sake of this-in order that the crop might be spoiled-but
that result just followed. Why then should it not be the same with
the parts in nature, e.g. that our teeth should come up of necessity-the
front teeth sharp, fitted for tearing, the molars broad and useful
for grinding down the food-since they did not arise for this end,
but it was merely a coincident result; and so with all other parts
in which we suppose that there is purpose? Wherever then all the parts
came about just what they would have been if they had come be for
an end, such things survived, being organized spontaneously in a fitting
way; whereas those which grew otherwise perished and continue to perish,
as Empedocles says his 'man-faced ox-progeny' did. 

Such are the arguments (and others of the kind) which may cause difficulty
on this point. Yet it is impossible that this should be the true view.
For teeth and all other natural things either invariably or normally
come about in a given way; but of not one of the results of chance
or spontaneity is this true. We do not ascribe to chance or mere coincidence
the frequency of rain in winter, but frequent rain in summer we do;
nor heat in the dog-days, but only if we have it in winter. If then,
it is agreed that things are either the result of coincidence or for
an end, and these cannot be the result of coincidence or spontaneity,
it follows that they must be for an end; and that such things are
all due to nature even the champions of the theory which is before
us would agree. Therefore action for an end is present in things which
come to be and are by nature. 

Further, where a series has a completion, all the preceding steps
are for the sake of that. Now surely as in intelligent action, so
in nature; and as in nature, so it is in each action, if nothing interferes.
Now intelligent action is for the sake of an end; therefore the nature
of things also is so. Thus if a house, e.g. had been a thing made
by nature, it would have been made in the same way as it is now by
art; and if things made by nature were made also by art, they would
come to be in the same way as by nature. Each step then in the series
is for the sake of the next; and generally art partly completes what
nature cannot bring to a finish, and partly imitates her. If, therefore,
artificial products are for the sake of an end, so clearly also are
natural products. The relation of the later to the earlier terms of
the series is the same in both. This is most obvious in the animals
other than man: they make things neither by art nor after inquiry
or deliberation. Wherefore people discuss whether it is by intelligence
or by some other faculty that these creatures work,spiders, ants,
and the like. By gradual advance in this direction we come to see
clearly that in plants too that is produced which is conducive to
the end-leaves, e.g. grow to provide shade for the fruit. If then
it is both by nature and for an end that the swallow makes its nest
and the spider its web, and plants grow leaves for the sake of the
fruit and send their roots down (not up) for the sake of nourishment,
it is plain that this kind of cause is operative in things which come
to be and are by nature. And since 'nature' means two things, the
matter and the form, of which the latter is the end, and since all
the rest is for the sake of the end, the form must be the cause in
the sense of 'that for the sake of which'. 

Now mistakes come to pass even in the operations of art: the grammarian
makes a mistake in writing and the doctor pours out the wrong dose.
Hence clearly mistakes are possible in the operations of nature also.
If then in art there are cases in which what is rightly produced serves
a purpose, and if where mistakes occur there was a purpose in what
was attempted, only it was not attained, so must it be also in natural
products, and monstrosities will be failures in the purposive effort.
Thus in the original combinations the 'ox-progeny' if they failed
to reach a determinate end must have arisen through the corruption
of some principle corresponding to what is now the seed.

Further, seed must have come into being first, and not straightway
the animals: the words 'whole-natured first...' must have meant seed.

Again, in plants too we find the relation of means to end, though
the degree of organization is less. Were there then in plants also
'olive-headed vine-progeny', like the 'man-headed ox-progeny', or
not? An absurd suggestion; yet there must have been, if there were
such things among animals. 

Moreover, among the seeds anything must have come to be at random.
But the person who asserts this entirely does away with 'nature' and
what exists 'by nature'. For those things are natural which, by a
continuous movement originated from an internal principle, arrive
at some completion: the same completion is not reached from every
principle; nor any chance completion, but always the tendency in each
is towards the same end, if there is no impediment. 

The end and the means towards it may come about by chance. We say,
for instance, that a stranger has come by chance, paid the ransom,
and gone away, when he does so as if he had come for that purpose,
though it was not for that that he came. This is incidental, for chance
is an incidental cause, as I remarked before. But when an event takes
place always or for the most part, it is not incidental or by chance.
In natural products the sequence is invariable, if there is no impediment.

It is absurd to suppose that purpose is not present because we do
not observe the agent deliberating. Art does not deliberate. If the
ship-building art were in the wood, it would produce the same results
by nature. If, therefore, purpose is present in art, it is present
also in nature. The best illustration is a doctor doctoring himself:
nature is like that. 

It is plain then that nature is a cause, a cause that operates for
a purpose. 

Part 9

As regards what is 'of necessity', we must ask whether the necessity
is 'hypothetical', or 'simple' as well. The current view places what
is of necessity in the process of production, just as if one were
to suppose that the wall of a house necessarily comes to be because
what is heavy is naturally carried downwards and what is light to
the top, wherefore the stones and foundations take the lowest place,
with earth above because it is lighter, and wood at the top of all
as being the lightest. Whereas, though the wall does not come to be
without these, it is not due to these, except as its material cause:
it comes to be for the sake of sheltering and guarding certain things.
Similarly in all other things which involve production for an end;
the product cannot come to be without things which have a necessary
nature, but it is not due to these (except as its material); it comes
to be for an end. For instance, why is a saw such as it is? To effect
so-and-so and for the sake of so-and-so. This end, however, cannot
be realized unless the saw is made of iron. It is, therefore, necessary
for it to be of iron, it we are to have a saw and perform the operation
of sawing. What is necessary then, is necessary on a hypothesis; it
is not a result necessarily determined by antecedents. Necessity is
in the matter, while 'that for the sake of which' is in the definition.

Necessity in mathematics is in a way similar to necessity in things
which come to be through the operation of nature. Since a straight
line is what it is, it is necessary that the angles of a triangle
should equal two right angles. But not conversely; though if the angles
are not equal to two right angles, then the straight line is not what
it is either. But in things which come to be for an end, the reverse
is true. If the end is to exist or does exist, that also which precedes
it will exist or does exist; otherwise just as there, if-the conclusion
is not true, the premiss will not be true, so here the end or 'that
for the sake of which' will not exist. For this too is itself a starting-point,
but of the reasoning, not of the action; while in mathematics the
starting-point is the starting-point of the reasoning only, as there
is no action. If then there is to be a house, such-and-such things
must be made or be there already or exist, or generally the matter
relative to the end, bricks and stones if it is a house. But the end
is not due to these except as the matter, nor will it come to exist
because of them. Yet if they do not exist at all, neither will the
house, or the saw-the former in the absence of stones, the latter
in the absence of iron-just as in the other case the premisses will
not be true, if the angles of the triangle are not equal to two right
angles. 

The necessary in nature, then, is plainly what we call by the name
of matter, and the changes in it. Both causes must be stated by the
physicist, but especially the end; for that is the cause of the matter,
not vice versa; and the end is 'that for the sake of which', and the
beginning starts from the definition or essence; as in artificial
products, since a house is of such-and-such a kind, certain things
must necessarily come to be or be there already, or since health is
this, these things must necessarily come to be or be there already.
Similarly if man is this, then these; if these, then those. Perhaps
the necessary is present also in the definition. For if one defines
the operation of sawing as being a certain kind of dividing, then
this cannot come about unless the saw has teeth of a certain kind;
and these cannot be unless it is of iron. For in the definition too
there are some parts that are, as it were, its matter. 

----------------------------------------------------------------------

BOOK III

Part 1 

Nature has been defined as a 'principle of motion and change', and
it is the subject of our inquiry. We must therefore see that we understand
the meaning of 'motion'; for if it were unknown, the meaning of 'nature'
too would be unknown. 

When we have determined the nature of motion, our next task will be
to attack in the same way the terms which are involved in it. Now
motion is supposed to belong to the class of things which are continuous;
and the infinite presents itself first in the continuous-that is how
it comes about that 'infinite' is often used in definitions of the
continuous ('what is infinitely divisible is continuous'). Besides
these, place, void, and time are thought to be necessary conditions
of motion. 

Clearly, then, for these reasons and also because the attributes mentioned
are common to, and coextensive with, all the objects of our science,
we must first take each of them in hand and discuss it. For the investigation
of special attributes comes after that of the common attributes.

To begin then, as we said, with motion. 
We may start by distinguishing (1) what exists in a state of fulfilment
only, (2) what exists as potential, (3) what exists as potential and
also in fulfilment-one being a 'this', another 'so much', a third
'such', and similarly in each of the other modes of the predication
of being. 

Further, the word 'relative' is used with reference to (1) excess
and defect, (2) agent and patient and generally what can move and
what can be moved. For 'what can cause movement' is relative to 'what
can be moved', and vice versa. 

Again, there is no such thing as motion over and above the things.
It is always with respect to substance or to quantity or to quality
or to place that what changes changes. But it is impossible, as we
assert, to find anything common to these which is neither 'this' nor
quantum nor quale nor any of the other predicates. Hence neither will
motion and change have reference to something over and above the things
mentioned, for there is nothing over and above them. 

Now each of these belongs to all its subjects in either of two ways:
namely (1) substance-the one is positive form, the other privation;
(2) in quality, white and black; (3) in quantity, complete and incomplete;
(4) in respect of locomotion, upwards and downwards or light and heavy.
Hence there are as many types of motion or change as there are meanings
of the word 'is'. 

We have now before us the distinctions in the various classes of being
between what is full real and what is potential. 

Def. The fulfilment of what exists potentially, in so far as it exists
potentially, is motion-namely, of what is alterable qua alterable,
alteration: of what can be increased and its opposite what can be
decreased (there is no common name), increase and decrease: of what
can come to be and can pass away, coming to he and passing away: of
what can be carried along, locomotion. 

Examples will elucidate this definition of motion. When the buildable,
in so far as it is just that, is fully real, it is being built, and
this is building. Similarly, learning, doctoring, rolling, leaping,
ripening, ageing. 

The same thing, if it is of a certain kind, can be both potential
and fully real, not indeed at the same time or not in the same respect,
but e.g. potentially hot and actually cold. Hence at once such things
will act and be acted on by one another in many ways: each of them
will be capable at the same time of causing alteration and of being
altered. Hence, too, what effects motion as a physical agent can be
moved: when a thing of this kind causes motion, it is itself also
moved. This, indeed, has led some people to suppose that every mover
is moved. But this question depends on another set of arguments, and
the truth will be made clear later. is possible for a thing to cause
motion, though it is itself incapable of being moved. 

It is the fulfilment of what is potential when it is already fully
real and operates not as itself but as movable, that is motion. What
I mean by 'as' is this: Bronze is potentially a statue. But it is
not the fulfilment of bronze as bronze which is motion. For 'to be
bronze' and 'to be a certain potentiality' are not the same.

If they were identical without qualification, i.e. in definition,
the fulfilment of bronze as bronze would have been motion. But they
are not the same, as has been said. (This is obvious in contraries.
'To be capable of health' and 'to be capable of illness' are not the
same, for if they were there would be no difference between being
ill and being well. Yet the subject both of health and of sickness-whether
it is humour or blood-is one and the same.) 

We can distinguish, then, between the two-just as, to give another
example, 'colour' and visible' are different-and clearly it is the
fulfilment of what is potential as potential that is motion. So this,
precisely, is motion. 

Further it is evident that motion is an attribute of a thing just
when it is fully real in this way, and neither before nor after. For
each thing of this kind is capable of being at one time actual, at
another not. Take for instance the buildable as buildable. The actuality
of the buildable as buildable is the process of building. For the
actuality of the buildable must be either this or the house. But when
there is a house, the buildable is no longer buildable. On the other
hand, it is the buildable which is being built. The process then of
being built must be the kind of actuality required But building is
a kind of motion, and the same account will apply to the other kinds
also. 

Part 2

The soundness of this definition is evident both when we consider
the accounts of motion that the others have given, and also from the
difficulty of defining it otherwise. 

One could not easily put motion and change in another genus-this is
plain if we consider where some people put it; they identify motion
with or 'inequality' or 'not being'; but such things are not necessarily
moved, whether they are 'different' or 'unequal' or 'non-existent';
Nor is change either to or from these rather than to or from their
opposites. 

The reason why they put motion into these genera is that it is thought
to be something indefinite, and the principles in the second column
are indefinite because they are privative: none of them is either
'this' or 'such' or comes under any of the other modes of predication.
The reason in turn why motion is thought to be indefinite is that
it cannot be classed simply as a potentiality or as an actuality-a
thing that is merely capable of having a certain size is not undergoing
change, nor yet a thing that is actually of a certain size, and motion
is thought to be a sort of actuality, but incomplete, the reason for
this view being that the potential whose actuality it is is incomplete.
This is why it is hard to grasp what motion is. It is necessary to
class it with privation or with potentiality or with sheer actuality,
yet none of these seems possible. There remains then the suggested
mode of definition, namely that it is a sort of actuality, or actuality
of the kind described, hard to grasp, but not incapable of existing.

The mover too is moved, as has been said-every mover, that is, which
is capable of motion, and whose immobility is rest-when a thing is
subject to motion its immobility is rest. For to act on the movable
as such is just to move it. But this it does by contact, so that at
the same time it is also acted on. Hence we can define motion as the
fulfilment of the movable qua movable, the cause of the attribute
being contact with what can move so that the mover is also acted on.
The mover or agent will always be the vehicle of a form, either a
'this' or 'such', which, when it acts, will be the source and cause
of the change, e.g. the full-formed man begets man from what is potentially
man. 

Part 3

The solution of the difficulty that is raised about the motion-whether
it is in the movable-is plain. It is the fulfilment of this potentiality,
and by the action of that which has the power of causing motion; and
the actuality of that which has the power of causing motion is not
other than the actuality of the movable, for it must be the fulfilment
of both. A thing is capable of causing motion because it can do this,
it is a mover because it actually does it. But it is on the movable
that it is capable of acting. Hence there is a single actuality of
both alike, just as one to two and two to one are the same interval,
and the steep ascent and the steep descent are one-for these are one
and the same, although they can be described in different ways. So
it is with the mover and the moved. 

This view has a dialectical difficulty. Perhaps it is necessary that
the actuality of the agent and that of the patient should not be the
same. The one is 'agency' and the other 'patiency'; and the outcome
and completion of the one is an 'action', that of the other a 'passion'.
Since then they are both motions, we may ask: in what are they, if
they are different? Either (a) both are in what is acted on and moved,
or (b) the agency is in the agent and the patiency in the patient.
(If we ought to call the latter also 'agency', the word would be used
in two senses.) 

Now, in alternative (b), the motion will be in the mover, for the
same statement will hold of 'mover' and 'moved'. Hence either every
mover will be moved, or, though having motion, it will not be moved.

If on the other hand (a) both are in what is moved and acted on-both
the agency and the patiency (e.g. both teaching and learning, though
they are two, in the learner), then, first, the actuality of each
will not be present in each, and, a second absurdity, a thing will
have two motions at the same time. How will there be two alterations
of quality in one subject towards one definite quality? The thing
is impossible: the actualization will be one. 

But (some one will say) it is contrary to reason to suppose that there
should be one identical actualization of two things which are different
in kind. Yet there will be, if teaching and learning are the same,
and agency and patiency. To teach will be the same as to learn, and
to act the same as to be acted on-the teacher will necessarily be
learning everything that he teaches, and the agent will be acted on.
One may reply: 

(1) It is not absurd that the actualization of one thing should be
in another. Teaching is the activity of a person who can teach, yet
the operation is performed on some patient-it is not cut adrift from
a subject, but is of A on B. 

(2) There is nothing to prevent two things having one and the same
actualization, provided the actualizations are not described in the
same way, but are related as what can act to what is acting.

(3) Nor is it necessary that the teacher should learn, even if to
act and to be acted on are one and the same, provided they are not
the same in definition (as 'raiment' and 'dress'), but are the same
merely in the sense in which the road from Thebes to Athens and the
road from Athens to Thebes are the same, as has been explained above.
For it is not things which are in a way the same that have all their
attributes the same, but only such as have the same definition. But
indeed it by no means follows from the fact that teaching is the same
as learning, that to learn is the same as to teach, any more than
it follows from the fact that there is one distance between two things
which are at a distance from each other, that the two vectors AB and
Ba, are one and the same. To generalize, teaching is not the same
as learning, or agency as patiency, in the full sense, though they
belong to the same subject, the motion; for the 'actualization of
X in Y' and the 'actualization of Y through the action of X' differ
in definition. 

What then Motion is, has been stated both generally and particularly.
It is not difficult to see how each of its types will be defined-alteration
is the fulfillment of the alterable qua alterable (or, more scientifically,
the fulfilment of what can act and what can be acted on, as such)-generally
and again in each particular case, building, healing, &c. A similar
definition will apply to each of the other kinds of motion.

Part 4

The science of nature is concerned with spatial magnitudes and motion
and time, and each of these at least is necessarily infinite or finite,
even if some things dealt with by the science are not, e.g. a quality
or a point-it is not necessary perhaps that such things should be
put under either head. Hence it is incumbent on the person who specializes
in physics to discuss the infinite and to inquire whether there is
such a thing or not, and, if there is, what it is. 

The appropriateness to the science of this problem is clearly indicated.
All who have touched on this kind of science in a way worth considering
have formulated views about the infinite, and indeed, to a man, make
it a principle of things. 

(1) Some, as the Pythagoreans and Plato, make the infinite a principle
in the sense of a self-subsistent substance, and not as a mere attribute
of some other thing. Only the Pythagoreans place the infinite among
the objects of sense (they do not regard number as separable from
these), and assert that what is outside the heaven is infinite. Plato,
on the other hand, holds that there is no body outside (the Forms
are not outside because they are nowhere),yet that the infinite is
present not only in the objects of sense but in the Forms also.

Further, the Pythagoreans identify the infinite with the even. For
this, they say, when it is cut off and shut in by the odd, provides
things with the element of infinity. An indication of this is what
happens with numbers. If the gnomons are placed round the one, and
without the one, in the one construction the figure that results is
always different, in the other it is always the same. But Plato has
two infinites, the Great and the Small. 

The physicists, on the other hand, all of them, always regard the
infinite as an attribute of a substance which is different from it
and belongs to the class of the so-called elements-water or air or
what is intermediate between them. Those who make them limited in
number never make them infinite in amount. But those who make the
elements infinite in number, as Anaxagoras and Democritus do, say
that the infinite is continuous by contact-compounded of the homogeneous
parts according to the one, of the seed-mass of the atomic shapes
according to the other. 

Further, Anaxagoras held that any part is a mixture in the same way
as the All, on the ground of the observed fact that anything comes
out of anything. For it is probably for this reason that he maintains
that once upon a time all things were together. (This flesh and this
bone were together, and so of any thing: therefore all things: and
at the same time too.) For there is a beginning of separation, not
only for each thing, but for all. Each thing that comes to be comes
from a similar body, and there is a coming to be of all things, though
not, it is true, at the same time. Hence there must also be an origin
of coming to be. One such source there is which he calls Mind, and
Mind begins its work of thinking from some starting-point. So necessarily
all things must have been together at a certain time, and must have
begun to be moved at a certain time. 

Democritus, for his part, asserts the contrary, namely that no element
arises from another element. Nevertheless for him the common body
is a source of all things, differing from part to part in size and
in shape. 

It is clear then from these considerations that the inquiry concerns
the physicist. Nor is it without reason that they all make it a principle
or source. We cannot say that the infinite has no effect, and the
only effectiveness which we can ascribe to it is that of a principle.
Everything is either a source or derived from a source. But there
cannot be a source of the infinite or limitless, for that would be
a limit of it. Further, as it is a beginning, it is both uncreatable
and indestructible. For there must be a point at which what has come
to be reaches completion, and also a termination of all passing away.
That is why, as we say, there is no principle of this, but it is this
which is held to be the principle of other things, and to encompass
all and to steer all, as those assert who do not recognize, alongside
the infinite, other causes, such as Mind or Friendship. Further they
identify it with the Divine, for it is 'deathless and imperishable'
as Anaximander says, with the majority of the physicists.

Belief in the existence of the infinite comes mainly from five considerations:

(1) From the nature of time-for it is infinite. 
(2) From the division of magnitudes-for the mathematicians also use
the notion of the infinite. 

(3) If coming to be and passing away do not give out, it is only because
that from which things come to be is infinite. 

(4) Because the limited always finds its limit in something, so that
there must be no limit, if everything is always limited by something
different from itself. 

(5) Most of all, a reason which is peculiarly appropriate and presents
the difficulty that is felt by everybody-not only number but also
mathematical magnitudes and what is outside the heaven are supposed
to be infinite because they never give out in our thought.

The last fact (that what is outside is infinite) leads people to suppose
that body also is infinite, and that there is an infinite number of
worlds. Why should there be body in one part of the void rather than
in another? Grant only that mass is anywhere and it follows that it
must be everywhere. Also, if void and place are infinite, there must
be infinite body too, for in the case of eternal things what may be
must be. But the problem of the infinite is difficult: many contradictions
result whether we suppose it to exist or not to exist. If it exists,
we have still to ask how it exists; as a substance or as the essential
attribute of some entity? Or in neither way, yet none the less is
there something which is infinite or some things which are infinitely
many? 

The problem, however, which specially belongs to the physicist is
to investigate whether there is a sensible magnitude which is infinite.

We must begin by distinguishing the various senses in which the term
'infinite' is used. 

(1) What is incapable of being gone through, because it is not in
its nature to be gone through (the sense in which the voice is 'invisible').

(2) What admits of being gone through, the process however having
no termination, or what scarcely admits of being gone through.

(3) What naturally admits of being gone through, but is not actually
gone through or does not actually reach an end. 

Further, everything that is infinite may be so in respect of addition
or division or both. 

Part 5

Now it is impossible that the infinite should be a thing which is
itself infinite, separable from sensible objects. If the infinite
is neither a magnitude nor an aggregate, but is itself a substance
and not an attribute, it will be indivisible; for the divisible must
be either a magnitude or an aggregate. But if indivisible, then not
infinite, except in the sense (1) in which the voice is 'invisible'.
But this is not the sense in which it is used by those who say that
the infinite exists, nor that in which we are investigating it, namely
as (2) 'that which cannot be gone through'. But if the infinite exists
as an attribute, it would not be, qua infinite an element in substances,
any more than the invisible would be an element of speech, though
the voice is invisible. 

Further, how can the infinite be itself any thing, unless both number
and magnitude, of which it is an essential attribute, exist in that
way? If they are not substances, a fortiori the infinite is not.

It is plain, too, that the infinite cannot be an actual thing and
a substance and principle. For any part of it that is taken will be
infinite, if it has parts: for 'to be infinite' and 'the infinite'
are the same, if it is a substance and not predicated of a subject.
Hence it will be either indivisible or divisible into infinites. But
the same thing cannot be many infinites. (Yet just as part of air
is air, so a part of the infinite would be infinite, if it is supposed
to be a substance and principle.) Therefore the infinite must be without
parts and indivisible. But this cannot be true of what is infinite
in full completion: for it must be a definite quantity. 

Suppose then that infinity belongs to substance as an attribute. But,
if so, it cannot, as we have said, be described as a principle, but
rather that of which it is an attribute-the air or the even number.

Thus the view of those who speak after the manner of the Pythagoreans
is absurd. With the same breath they treat the infinite as substance,
and divide it into parts. 

This discussion, however, involves the more general question whether
the infinite can be present in mathematical objects and things which
are intelligible and do not have extension, as well as among sensible
objects. Our inquiry (as physicists) is limited to its special subject-matter,
the objects of sense, and we have to ask whether there is or is not
among them a body which is infinite in the direction of increase.

We may begin with a dialectical argument and show as follows that
there is no such thing. If 'bounded by a surface' is the definition
of body there cannot be an infinite body either intelligible or sensible.
Nor can number taken in abstraction be infinite, for number or that
which has number is numerable. If then the numerable can be numbered,
it would also be possible to go through the infinite. 

If, on the other hand, we investigate the question more in accordance
with principles appropriate to physics, we are led as follows to the
same result. 

The infinite body must be either (1) compound, or (2) simple; yet
neither alternative is possible. 

(1) Compound the infinite body will not be, if the elements are finite
in number. For they must be more than one, and the contraries must
always balance, and no one of them can be infinite. If one of the
bodies falls in any degree short of the other in potency-suppose fire
is finite in amount while air is infinite and a given quantity of
fire exceeds in power the same amount of air in any ratio provided
it is numerically definite-the infinite body will obviously prevail
over and annihilate the finite body. On the other hand, it is impossible
that each should be infinite. 'Body' is what has extension in all
directions and the infinite is what is boundlessly extended, so that
the infinite body would be extended in all directions ad infinitum.

Nor (2) can the infinite body be one and simple, whether it is, as
some hold, a thing over and above the elements (from which they generate
the elements) or is not thus qualified. 

(a) We must consider the former alternative; for there are some people
who make this the infinite, and not air or water, in order that the
other elements may not be annihilated by the element which is infinite.
They have contrariety with each other-air is cold, water moist, fire
hot; if one were infinite, the others by now would have ceased to
be. As it is, they say, the infinite is different from them and is
their source. 

It is impossible, however, that there should be such a body; not because
it is infinite on that point a general proof can be given which applies
equally to all, air, water, or anything else-but simply because there
is, as a matter of fact, no such sensible body, alongside the so-called
elements. Everything can be resolved into the elements of which it
is composed. Hence the body in question would have been present in
our world here, alongside air and fire and earth and water: but nothing
of the kind is observed. 

(b) Nor can fire or any other of the elements be infinite. For generally,
and apart from the question of how any of them could be infinite,
the All, even if it were limited, cannot either be or become one of
them, as Heraclitus says that at some time all things become fire.
(The same argument applies also to the one which the physicists suppose
to exist alongside the elements: for everything changes from contrary
to contrary, e.g. from hot to cold). 

The preceding consideration of the various cases serves to show us
whether it is or is not possible that there should be an infinite
sensible body. The following arguments give a general demonstration
that it is not possible. 

It is the nature of every kind of sensible body to be somewhere, and
there is a place appropriate to each, the same for the part and for
the whole, e.g. for the whole earth and for a single clod, and for
fire and for a spark. 

Suppose (a) that the infinite sensible body is homogeneous. Then each
part will be either immovable or always being carried along. Yet neither
is possible. For why downwards rather than upwards or in any other
direction? I mean, e.g, if you take a clod, where will it be moved
or where will it be at rest? For ex hypothesi the place of the body
akin to it is infinite. Will it occupy the whole place, then? And
how? What then will be the nature of its rest and of its movement,
or where will they be? It will either be at home everywhere-then it
will not be moved; or it will be moved everywhere-then it will not
come to rest. 

But if (b) the All has dissimilar parts, the proper places of the
parts will be dissimilar also, and the body of the All will have no
unity except that of contact. Then, further, the parts will be either
finite or infinite in variety of kind. (i) Finite they cannot be,
for if the All is to be infinite, some of them would have to be infinite,
while the others were not, e.g. fire or water will be infinite. But,
as we have seen before, such an element would destroy what is contrary
to it. (This indeed is the reason why none of the physicists made
fire or earth the one infinite body, but either water or air or what
is intermediate between them, because the abode of each of the two
was plainly determinate, while the others have an ambiguous place
between up and down.) 

But (ii) if the parts are infinite in number and simple, their proper
places too will be infinite in number, and the same will be true of
the elements themselves. If that is impossible, and the places are
finite, the whole too must be finite; for the place and the body cannot
but fit each other. Neither is the whole place larger than what can
be filled by the body (and then the body would no longer be infinite),
nor is the body larger than the place; for either there would be an
empty space or a body whose nature it is to be nowhere. 

Anaxagoras gives an absurd account of why the infinite is at rest.
He says that the infinite itself is the cause of its being fixed.
This because it is in itself, since nothing else contains it-on the
assumption that wherever anything is, it is there by its own nature.
But this is not true: a thing could be somewhere by compulsion, and
not where it is its nature to be. 

Even if it is true as true can be that the whole is not moved (for
what is fixed by itself and is in itself must be immovable), yet we
must explain why it is not its nature to be moved. It is not enough
just to make this statement and then decamp. Anything else might be
in a state of rest, but there is no reason why it should not be its
nature to be moved. The earth is not carried along, and would not
be carried along if it were infinite, provided it is held together
by the centre. But it would not be because there was no other region
in which it could be carried along that it would remain at the centre,
but because this is its nature. Yet in this case also we may say that
it fixes itself. If then in the case of the earth, supposed to be
infinite, it is at rest, not because it is infinite, but because it
has weight and what is heavy rests at the centre and the earth is
at the centre, similarly the infinite also would rest in itself, not
because it is infinite and fixes itself, but owing to some other cause.

Another difficulty emerges at the same time. Any part of the infinite
body ought to remain at rest. Just as the infinite remains at rest
in itself because it fixes itself, so too any part of it you may take
will remain in itself. The appropriate places of the whole and of
the part are alike, e.g. of the whole earth and of a clod the appropriate
place is the lower region; of fire as a whole and of a spark, the
upper region. If, therefore, to be in itself is the place of the infinite,
that also will be appropriate to the part. Therefore it will remain
in itself. 

In general, the view that there is an infinite body is plainly incompatible
with the doctrine that there is necessarily a proper place for each
kind of body, if every sensible body has either weight or lightness,
and if a body has a natural locomotion towards the centre if it is
heavy, and upwards if it is light. This would need to be true of the
infinite also. But neither character can belong to it: it cannot be
either as a whole, nor can it be half the one and half the other.
For how should you divide it? or how can the infinite have the one
part up and the other down, or an extremity and a centre?

Further, every sensible body is in place, and the kinds or differences
of place are up-down, before-behind, right-left; and these distinctions
hold not only in relation to us and by arbitrary agreement, but also
in the whole itself. But in the infinite body they cannot exist. In
general, if it is impossible that there should be an infinite place,
and if every body is in place, there cannot be an infinite body.

Surely what is in a special place is in place, and what is in place
is in a special place. Just, then, as the infinite cannot be quantity-that
would imply that it has a particular quantity, e,g, two or three cubits;
quantity just means these-so a thing's being in place means that it
is somewhere, and that is either up or down or in some other of the
six differences of position: but each of these is a limit.

It is plain from these arguments that there is no body which is actually
infinite. 

Part 6

But on the other hand to suppose that the infinite does not exist
in any way leads obviously to many impossible consequences: there
will be a beginning and an end of time, a magnitude will not be divisible
into magnitudes, number will not be infinite. If, then, in view of
the above considerations, neither alternative seems possible, an arbiter
must be called in; and clearly there is a sense in which the infinite
exists and another in which it does not. 

We must keep in mind that the word 'is' means either what potentially
is or what fully is. Further, a thing is infinite either by addition
or by division. 

Now, as we have seen, magnitude is not actually infinite. But by division
it is infinite. (There is no difficulty in refuting the theory of
indivisible lines.) The alternative then remains that the infinite
has a potential existence. 

But the phrase 'potential existence' is ambiguous. When we speak of
the potential existence of a statue we mean that there will be an
actual statue. It is not so with the infinite. There will not be an
actual infinite. The word 'is' has many senses, and we say that the
infinite 'is' in the sense in which we say 'it is day' or 'it is the
games', because one thing after another is always coming into existence.
For of these things too the distinction between potential and actual
existence holds. We say that there are Olympic games, both in the
sense that they may occur and that they are actually occurring.

The infinite exhibits itself in different ways-in time, in the generations
of man, and in the division of magnitudes. For generally the infinite
has this mode of existence: one thing is always being taken after
another, and each thing that is taken is always finite, but always
different. Again, 'being' has more than one sense, so that we must
not regard the infinite as a 'this', such as a man or a horse, but
must suppose it to exist in the sense in which we speak of the day
or the games as existing things whose being has not come to them like
that of a substance, but consists in a process of coming to be or
passing away; definite if you like at each stage, yet always different.

But when this takes place in spatial magnitudes, what is taken perists,
while in the succession of time and of men it takes place by the passing
away of these in such a way that the source of supply never gives
out. 

In a way the infinite by addition is the same thing as the infinite
by division. In a finite magnitude, the infinite by addition comes
about in a way inverse to that of the other. For in proportion as
we see division going on, in the same proportion we see addition being
made to what is already marked off. For if we take a determinate part
of a finite magnitude and add another part determined by the same
ratio (not taking in the same amount of the original whole), and so
on, we shall not traverse the given magnitude. But if we increase
the ratio of the part, so as always to take in the same amount, we
shall traverse the magnitude, for every finite magnitude is exhausted
by means of any determinate quantity however small. 

The infinite, then, exists in no other way, but in this way it does
exist, potentially and by reduction. It exists fully in the sense
in which we say 'it is day' or 'it is the games'; and potentially
as matter exists, not independently as what is finite does.

By addition then, also, there is potentially an infinite, namely,
what we have described as being in a sense the same as the infinite
in respect of division. For it will always be possible to take something
ah extra. Yet the sum of the parts taken will not exceed every determinate
magnitude, just as in the direction of division every determinate
magnitude is surpassed in smallness and there will be a smaller part.

But in respect of addition there cannot be an infinite which even
potentially exceeds every assignable magnitude, unless it has the
attribute of being actually infinite, as the physicists hold to be
true of the body which is outside the world, whose essential nature
is air or something of the kind. But if there cannot be in this way
a sensible body which is infinite in the full sense, evidently there
can no more be a body which is potentially infinite in respect of
addition, except as the inverse of the infinite by division, as we
have said. It is for this reason that Plato also made the infinites
two in number, because it is supposed to be possible to exceed all
limits and to proceed ad infinitum in the direction both of increase
and of reduction. Yet though he makes the infinites two, he does not
use them. For in the numbers the infinite in the direction of reduction
is not present, as the monad is the smallest; nor is the infinite
in the direction of increase, for the parts number only up to the
decad. 

The infinite turns out to be the contrary of what it is said to be.
It is not what has nothing outside it that is infinite, but what always
has something outside it. This is indicated by the fact that rings
also that have no bezel are described as 'endless', because it is
always possible to take a part which is outside a given part. The
description depends on a certain similarity, but it is not true in
the full sense of the word. This condition alone is not sufficient:
it is necessary also that the next part which is taken should never
be the same. In the circle, the latter condition is not satisfied:
it is only the adjacent part from which the new part is different.

Our definition then is as follows: 
A quantity is infinite if it is such that we can always take a part
outside what has been already taken. On the other hand, what has nothing
outside it is complete and whole. For thus we define the whole-that
from which nothing is wanting, as a whole man or a whole box. What
is true of each particular is true of the whole as such-the whole
is that of which nothing is outside. On the other hand that from which
something is absent and outside, however small that may be, is not
'all'. 'Whole' and 'complete' are either quite identical or closely
akin. Nothing is complete (teleion) which has no end (telos); and
the end is a limit. 

Hence Parmenides must be thought to have spoken better than Melissus.
The latter says that the whole is infinite, but the former describes
it as limited, 'equally balanced from the middle'. For to connect
the infinite with the all and the whole is not like joining two pieces
of string; for it is from this they get the dignity they ascribe to
the infinite-its containing all things and holding the all in itself-from
its having a certain similarity to the whole. It is in fact the matter
of the completeness which belongs to size, and what is potentially
a whole, though not in the full sense. It is divisible both in the
direction of reduction and of the inverse addition. It is a whole
and limited; not, however, in virtue of its own nature, but in virtue
of what is other than it. It does not contain, but, in so far as it
is infinite, is contained. Consequently, also, it is unknowable, qua
infinite; for the matter has no form. (Hence it is plain that the
infinite stands in the relation of part rather than of whole. For
the matter is part of the whole, as the bronze is of the bronze statue.)
If it contains in the case of sensible things, in the case of intelligible
things the great and the small ought to contain them. But it is absurd
and impossible to suppose that the unknowable and indeterminate should
contain and determine. 

Part 7

It is reasonable that there should not be held to be an infinite in
respect of addition such as to surpass every magnitude, but that there
should be thought to be such an infinite in the direction of division.
For the matter and the infinite are contained inside what contains
them, while it is the form which contains. It is natural too to suppose
that in number there is a limit in the direction of the minimum, and
that in the other direction every assigned number is surpassed. In
magnitude, on the contrary, every assigned magnitude is surpassed
in the direction of smallness, while in the other direction there
is no infinite magnitude. The reason is that what is one is indivisible
whatever it may be, e.g. a man is one man, not many. Number on the
other hand is a plurality of 'ones' and a certain quantity of them.
Hence number must stop at the indivisible: for 'two' and 'three' are
merely derivative terms, and so with each of the other numbers. But
in the direction of largeness it is always possible to think of a
larger number: for the number of times a magnitude can be bisected
is infinite. Hence this infinite is potential, never actual: the number
of parts that can be taken always surpasses any assigned number. But
this number is not separable from the process of bisection, and its
infinity is not a permanent actuality but consists in a process of
coming to be, like time and the number of time. 

With magnitudes the contrary holds. What is continuous is divided
ad infinitum, but there is no infinite in the direction of increase.
For the size which it can potentially be, it can also actually be.
Hence since no sensible magnitude is infinite, it is impossible to
exceed every assigned magnitude; for if it were possible there would
be something bigger than the heavens. 

The infinite is not the same in magnitude and movement and time, in
the sense of a single nature, but its secondary sense depends on its
primary sense, i.e. movement is called infinite in virtue of the magnitude
covered by the movement (or alteration or growth), and time because
of the movement. (I use these terms for the moment. Later I shall
explain what each of them means, and also why every magnitude is divisible
into magnitudes.) 

Our account does not rob the mathematicians of their science, by disproving
the actual existence of the infinite in the direction of increase,
in the sense of the untraversable. In point of fact they do not need
the infinite and do not use it. They postulate only that the finite
straight line may be produced as far as they wish. It is possible
to have divided in the same ratio as the largest quantity another
magnitude of any size you like. Hence, for the purposes of proof,
it will make no difference to them to have such an infinite instead,
while its existence will be in the sphere of real magnitudes.

In the fourfold scheme of causes, it is plain that the infinite is
a cause in the sense of matter, and that its essence is privation,
the subject as such being what is continuous and sensible. All the
other thinkers, too, evidently treat the infinite as matter-that is
why it is inconsistent in them to make it what contains, and not what
is contained. 

Part 8

It remains to dispose of the arguments which are supposed to support
the view that the infinite exists not only potentially but as a separate
thing. Some have no cogency; others can be met by fresh objections
that are valid. 

(1) In order that coming to be should not fail, it is not necessary
that there should be a sensible body which is actually infinite. The
passing away of one thing may be the coming to be of another, the
All being limited. 

(2) There is a difference between touching and being limited. The
former is relative to something and is the touching of something (for
everything that touches touches something), and further is an attribute
of some one of the things which are limited. On the other hand, what
is limited is not limited in relation to anything. Again, contact
is not necessarily possible between any two things taken at random.

(3) To rely on mere thinking is absurd, for then the excess or defect
is not in the thing but in the thought. One might think that one of
us is bigger than he is and magnify him ad infinitum. But it does
not follow that he is bigger than the size we are, just because some
one thinks he is, but only because he is the size he is. The thought
is an accident. 

(a) Time indeed and movement are infinite, and also thinking, in the
sense that each part that is taken passes in succession out of existence.

(b) Magnitude is not infinite either in the way of reduction or of
magnification in thought. 

This concludes my account of the way in which the infinite exists,
and of the way in which it does not exist, and of what it is.

----------------------------------------------------------------------

BOOK IV

Part 1 

The physicist must have a knowledge of Place, too, as well as of
the infinite-namely, whether there is such a thing or not, and the
manner of its existence and what it is-both because all suppose that
things which exist are somewhere (the non-existent is nowhere--where
is the goat-stag or the sphinx?), and because 'motion' in its most
general and primary sense is change of place, which we call 'locomotion'.

The question, what is place? presents many difficulties. An examination
of all the relevant facts seems to lead to divergent conclusions.
Moreover, we have inherited nothing from previous thinkers, whether
in the way of a statement of difficulties or of a solution.

The existence of place is held to be obvious from the fact of mutual
replacement. Where water now is, there in turn, when the water has
gone out as from a vessel, air is present. When therefore another
body occupies this same place, the place is thought to be different
from all the bodies which come to be in it and replace one another.
What now contains air formerly contained water, so that clearly the
place or space into which and out of which they passed was something
different from both. 

Further, the typical locomotions of the elementary natural bodies-namely,
fire, earth, and the like-show not only that place is something, but
also that it exerts a certain influence. Each is carried to its own
place, if it is not hindered, the one up, the other down. Now these
are regions or kinds of place-up and down and the rest of the six
directions. Nor do such distinctions (up and down and right and left,
&c.) hold only in relation to us. To us they are not always the same
but change with the direction in which we are turned: that is why
the same thing may be both right and left, up and down, before and
behind. But in nature each is distinct, taken apart by itself. It
is not every chance direction which is 'up', but where fire and what
is light are carried; similarly, too, 'down' is not any chance direction
but where what has weight and what is made of earth are carried-the
implication being that these places do not differ merely in relative
position, but also as possessing distinct potencies. This is made
plain also by the objects studied by mathematics. Though they have
no real place, they nevertheless, in respect of their position relatively
to us, have a right and left as attributes ascribed to them only in
consequence of their relative position, not having by nature these
various characteristics. Again, the theory that the void exists involves
the existence of place: for one would define void as place bereft
of body. 

These considerations then would lead us to suppose that place is something
distinct from bodies, and that every sensible body is in place. Hesiod
too might be held to have given a correct account of it when he made
chaos first. At least he says: 

'First of all things came chaos to being, then broad-breasted earth,'
implying that things need to have space first, because he thought,
with most people, that everything is somewhere and in place. If this
is its nature, the potency of place must be a marvellous thing, and
take precedence of all other things. For that without which nothing
else can exist, while it can exist without the others, must needs
be first; for place does not pass out of existence when the things
in it are annihilated. 

True, but even if we suppose its existence settled, the question of
its nature presents difficulty-whether it is some sort of 'bulk' of
body or some entity other than that, for we must first determine its
genus. 

(1) Now it has three dimensions, length, breadth, depth, the dimensions
by which all body also is bounded. But the place cannot be body; for
if it were there would be two bodies in the same place. 

(2) Further, if body has a place and space, clearly so too have surface
and the other limits of body; for the same statement will apply to
them: where the bounding planes of the water were, there in turn will
be those of the air. But when we come to a point we cannot make a
distinction between it and its place. Hence if the place of a point
is not different from the point, no more will that of any of the others
be different, and place will not be something different from each
of them. 

(3) What in the world then are we to suppose place to be? If it has
the sort of nature described, it cannot be an element or composed
of elements, whether these be corporeal or incorporeal: for while
it has size, it has not body. But the elements of sensible bodies
are bodies, while nothing that has size results from a combination
of intelligible elements. 

(4) Also we may ask: of what in things is space the cause? None of
the four modes of causation can be ascribed to it. It is neither in
the sense of the matter of existents (for nothing is composed of it),
nor as the form and definition of things, nor as end, nor does it
move existents. 

(5) Further, too, if it is itself an existent, where will it be? Zeno's
difficulty demands an explanation: for if everything that exists has
a place, place too will have a place, and so on ad infinitum.

(6) Again, just as every body is in place, so, too, every place has
a body in it. What then shall we say about growing things? It follows
from these premisses that their place must grow with them, if their
place is neither less nor greater than they are. 

By asking these questions, then, we must raise the whole problem about
place-not only as to what it is, but even whether there is such a
thing. 

Part 2

We may distinguish generally between predicating B of A because it
(A) is itself, and because it is something else; and particularly
between place which is common and in which all bodies are, and the
special place occupied primarily by each. I mean, for instance, that
you are now in the heavens because you are in the air and it is in
the heavens; and you are in the air because you are on the earth;
and similarly on the earth because you are in this place which contains
no more than you. 

Now if place is what primarily contains each body, it would be a limit,
so that the place would be the form or shape of each body by which
the magnitude or the matter of the magnitude is defined: for this
is the limit of each body. 

If, then, we look at the question in this way the place of a thing
is its form. But, if we regard the place as the extension of the magnitude,
it is the matter. For this is different from the magnitude: it is
what is contained and defined by the form, as by a bounding plane.
Matter or the indeterminate is of this nature; when the boundary and
attributes of a sphere are taken away, nothing but the matter is left.

This is why Plato in the Timaeus says that matter and space are the
same; for the 'participant' and space are identical. (It is true,
indeed, that the account he gives there of the 'participant' is different
from what he says in his so-called 'unwritten teaching'. Nevertheless,
he did identify place and space.) I mention Plato because, while all
hold place to be something, he alone tried to say what it is.

In view of these facts we should naturally expect to find difficulty
in determining what place is, if indeed it is one of these two things,
matter or form. They demand a very close scrutiny, especially as it
is not easy to recognize them apart. 

But it is at any rate not difficult to see that place cannot be either
of them. The form and the matter are not separate from the thing,
whereas the place can be separated. As we pointed out, where air was,
water in turn comes to be, the one replacing the other; and similarly
with other bodies. Hence the place of a thing is neither a part nor
a state of it, but is separable from it. For place is supposed to
be something like a vessel-the vessel being a transportable place.
But the vessel is no part of the thing. 

In so far then as it is separable from the thing, it is not the form:
qua containing, it is different from the matter. 

Also it is held that what is anywhere is both itself something and
that there is a different thing outside it. (Plato of course, if we
may digress, ought to tell us why the form and the numbers are not
in place, if 'what participates' is place-whether what participates
is the Great and the Small or the matter, as he called it in writing
in the Timaeus.) 

Further, how could a body be carried to its own place, if place was
the matter or the form? It is impossible that what has no reference
to motion or the distinction of up and down can be place. So place
must be looked for among things which have these characteristics.

If the place is in the thing (it must be if it is either shape or
matter) place will have a place: for both the form and the indeterminate
undergo change and motion along with the thing, and are not always
in the same place, but are where the thing is. Hence the place will
have a place. 

Further, when water is produced from air, the place has been destroyed,
for the resulting body is not in the same place. What sort of destruction
then is that? 

This concludes my statement of the reasons why space must be something,
and again of the difficulties that may be raised about its essential
nature. 

Part 3

The next step we must take is to see in how many senses one thing
is said to be 'in' another. 

(1) As the finger is 'in' the hand and generally the part 'in' the
whole. 

(2) As the whole is 'in' the parts: for there is no whole over and
above the parts. 

(3) As man is 'in' animal and generally species 'in' genus.

(4) As the genus is 'in' the species and generally the part of the
specific form 'in' the definition of the specific form. 

(5) As health is 'in' the hot and the cold and generally the form
'in' the matter. 

(6) As the affairs of Greece centre 'in' the king, and generally events
centre 'in' their primary motive agent. 

(7) As the existence of a thing centres 'in its good and generally
'in' its end, i.e. in 'that for the sake of which' it exists.

(8) In the strictest sense of all, as a thing is 'in' a vessel, and
generally 'in' place. 

One might raise the question whether a thing can be in itself, or
whether nothing can be in itself-everything being either nowhere or
in something else. 

The question is ambiguous; we may mean the thing qua itself or qua
something else. 

When there are parts of a whole-the one that in which a thing is,
the other the thing which is in it-the whole will be described as
being in itself. For a thing is described in terms of its parts, as
well as in terms of the thing as a whole, e.g. a man is said to be
white because the visible surface of him is white, or to be scientific
because his thinking faculty has been trained. The jar then will not
be in itself and the wine will not be in itself. But the jar of wine
will: for the contents and the container are both parts of the same
whole. 

In this sense then, but not primarily, a thing can be in itself, namely,
as 'white' is in body (for the visible surface is in body), and science
is in the mind. 

It is from these, which are 'parts' (in the sense at least of being
'in' the man), that the man is called white, &c. But the jar and the
wine in separation are not parts of a whole, though together they
are. So when there are parts, a thing will be in itself, as 'white'
is in man because it is in body, and in body because it resides in
the visible surface. We cannot go further and say that it is in surface
in virtue of something other than itself. (Yet it is not in itself:
though these are in a way the same thing,) they differ in essence,
each having a special nature and capacity, 'surface' and 'white'.

Thus if we look at the matter inductively we do not find anything
to be 'in' itself in any of the senses that have been distinguished;
and it can be seen by argument that it is impossible. For each of
two things will have to be both, e.g. the jar will have to be both
vessel and wine, and the wine both wine and jar, if it is possible
for a thing to be in itself; so that, however true it might be that
they were in each other, the jar will receive the wine in virtue not
of its being wine but of the wine's being wine, and the wine will
be in the jar in virtue not of its being a jar but of the jar's being
a jar. Now that they are different in respect of their essence is
evident; for 'that in which something is' and 'that which is in it'
would be differently defined. 

Nor is it possible for a thing to be in itself even incidentally:
for two things would at the same time in the same thing. The jar would
be in itself-if a thing whose nature it is to receive can be in itself;
and that which it receives, namely (if wine) wine, will be in it.

Obviously then a thing cannot be in itself primarily. 
Zeno's problem-that if Place is something it must be in something-is
not difficult to solve. There is nothing to prevent the first place
from being 'in' something else-not indeed in that as 'in' place, but
as health is 'in' the hot as a positive determination of it or as
the hot is 'in' body as an affection. So we escape the infinite regress.

Another thing is plain: since the vessel is no part of what is in
it (what contains in the strict sense is different from what is contained),
place could not be either the matter or the form of the thing contained,
but must different-for the latter, both the matter and the shape,
are parts of what is contained. 

This then may serve as a critical statement of the difficulties involved.

Part 4

What then after all is place? The answer to this question may be elucidated
as follows. 

Let us take for granted about it the various characteristics which
are supposed correctly to belong to it essentially. We assume then-

(1) Place is what contains that of which it is the place.

(2) Place is no part of the thing. 
(3) The immediate place of a thing is neither less nor greater than
the thing. 

(4) Place can be left behind by the thing and is separable. In addition:

(5) All place admits of the distinction of up and down, and each of
the bodies is naturally carried to its appropriate place and rests
there, and this makes the place either up or down. 

Having laid these foundations, we must complete the theory. We ought
to try to make our investigation such as will render an account of
place, and will not only solve the difficulties connected with it,
but will also show that the attributes supposed to belong to it do
really belong to it, and further will make clear the cause of the
trouble and of the difficulties about it. Such is the most satisfactory
kind of exposition. 

First then we must understand that place would not have been thought
of, if there had not been a special kind of motion, namely that with
respect to place. It is chiefly for this reason that we suppose the
heaven also to be in place, because it is in constant movement. Of
this kind of change there are two species-locomotion on the one hand
and, on the other, increase and diminution. For these too involve
variation of place: what was then in this place has now in turn changed
to what is larger or smaller. 

Again, when we say a thing is 'moved', the predicate either (1) belongs
to it actually, in virtue of its own nature, or (2) in virtue of something
conjoined with it. In the latter case it may be either (a) something
which by its own nature is capable of being moved, e.g. the parts
of the body or the nail in the ship, or (b) something which is not
in itself capable of being moved, but is always moved through its
conjunction with something else, as 'whiteness' or 'science'. These
have changed their place only because the subjects to which they belong
do so. 

We say that a thing is in the world, in the sense of in place, because
it is in the air, and the air is in the world; and when we say it
is in the air, we do not mean it is in every part of the air, but
that it is in the air because of the outer surface of the air which
surrounds it; for if all the air were its place, the place of a thing
would not be equal to the thing-which it is supposed to be, and which
the primary place in which a thing is actually is. 

When what surrounds, then, is not separate from the thing, but is
in continuity with it, the thing is said to be in what surrounds it,
not in the sense of in place, but as a part in a whole. But when the
thing is separate and in contact, it is immediately 'in' the inner
surface of the surrounding body, and this surface is neither a part
of what is in it nor yet greater than its extension, but equal to
it; for the extremities of things which touch are coincident.

Further, if one body is in continuity with another, it is not moved
in that but with that. On the other hand it is moved in that if it
is separate. It makes no difference whether what contains is moved
or not. 

Again, when it is not separate it is described as a part in a whole,
as the pupil in the eye or the hand in the body: when it is separate,
as the water in the cask or the wine in the jar. For the hand is moved
with the body and the water in the cask. 

It will now be plain from these considerations what place is. There
are just four things of which place must be one-the shape, or the
matter, or some sort of extension between the bounding surfaces of
the containing body, or this boundary itself if it contains no extension
over and above the bulk of the body which comes to be in it.

Three of these it obviously cannot be: 
(1) The shape is supposed to be place because it surrounds, for the
extremities of what contains and of what is contained are coincident.
Both the shape and the place, it is true, are boundaries. But not
of the same thing: the form is the boundary of the thing, the place
is the boundary of the body which contains it. 

(2) The extension between the extremities is thought to be something,
because what is contained and separate may often be changed while
the container remains the same (as water may be poured from a vessel)-the
assumption being that the extension is something over and above the
body displaced. But there is no such extension. One of the bodies
which change places and are naturally capable of being in contact
with the container falls in whichever it may chance to be.

If there were an extension which were such as to exist independently
and be permanent, there would be an infinity of places in the same
thing. For when the water and the air change places, all the portions
of the two together will play the same part in the whole which was
previously played by all the water in the vessel; at the same time
the place too will be undergoing change; so that there will be another
place which is the place of the place, and many places will be coincident.
There is not a different place of the part, in which it is moved,
when the whole vessel changes its place: it is always the same: for
it is in the (proximate) place where they are that the air and the
water (or the parts of the water) succeed each other, not in that
place in which they come to be, which is part of the place which is
the place of the whole world. 

(3) The matter, too, might seem to be place, at least if we consider
it in what is at rest and is thus separate but in continuity. For
just as in change of quality there is something which was formerly
black and is now white, or formerly soft and now hard-this is just
why we say that the matter exists-so place, because it presents a
similar phenomenon, is thought to exist-only in the one case we say
so because what was air is now water, in the other because where air
formerly was there a is now water. But the matter, as we said before,
is neither separable from the thing nor contains it, whereas place
has both characteristics. 

Well, then, if place is none of the three-neither the form nor the
matter nor an extension which is always there, different from, and
over and above, the extension of the thing which is displaced-place
necessarily is the one of the four which is left, namely, the boundary
of the containing body at which it is in contact with the contained
body. (By the contained body is meant what can be moved by way of
locomotion.) 

Place is thought to be something important and hard to grasp, both
because the matter and the shape present themselves along with it,
and because the displacement of the body that is moved takes place
in a stationary container, for it seems possible that there should
be an interval which is other than the bodies which are moved. The
air, too, which is thought to be incorporeal, contributes something
to the belief: it is not only the boundaries of the vessel which seem
to be place, but also what is between them, regarded as empty. Just,
in fact, as the vessel is transportable place, so place is a non-portable
vessel. So when what is within a thing which is moved, is moved and
changes its place, as a boat on a river, what contains plays the part
of a vessel rather than that of place. Place on the other hand is
rather what is motionless: so it is rather the whole river that is
place, because as a whole it is motionless. 

Hence we conclude that the innermost motionless boundary of what contains
is place. 

This explains why the middle of the heaven and the surface which faces
us of the rotating system are held to be 'up' and 'down' in the strict
and fullest sense for all men: for the one is always at rest, while
the inner side of the rotating body remains always coincident with
itself. Hence since the light is what is naturally carried up, and
the heavy what is carried down, the boundary which contains in the
direction of the middle of the universe, and the middle itself, are
down, and that which contains in the direction of the outermost part
of the universe, and the outermost part itself, are up. 

For this reason, too, place is thought to be a kind of surface, and
as it were a vessel, i.e. a container of the thing. 

Further, place is coincident with the thing, for boundaries are coincident
with the bounded. 

Part 5

If then a body has another body outside it and containing it, it is
in place, and if not, not. That is why, even if there were to be water
which had not a container, the parts of it, on the one hand, will
be moved (for one part is contained in another), while, on the other
hand, the whole will be moved in one sense, but not in another. For
as a whole it does not simultaneously change its place, though it
will be moved in a circle: for this place is the place of its parts.
(Some things are moved, not up and down, but in a circle; others up
and down, such things namely as admit of condensation and rarefaction.)

As was explained, some things are potentially in place, others actually.
So, when you have a homogeneous substance which is continuous, the
parts are potentially in place: when the parts are separated, but
in contact, like a heap, they are actually in place. 

Again, (1) some things are per se in place, namely every body which
is movable either by way of locomotion or by way of increase is per
se somewhere, but the heaven, as has been said, is not anywhere as
a whole, nor in any place, if at least, as we must suppose, no body
contains it. On the line on which it is moved, its parts have place:
for each is contiguous the next. 

But (2) other things are in place indirectly, through something conjoined
with them, as the soul and the heaven. The latter is, in a way, in
place, for all its parts are: for on the orb one part contains another.
That is why the upper part is moved in a circle, while the All is
not anywhere. For what is somewhere is itself something, and there
must be alongside it some other thing wherein it is and which contains
it. But alongside the All or the Whole there is nothing outside the
All, and for this reason all things are in the heaven; for the heaven,
we may say, is the All. Yet their place is not the same as the heaven.
It is part of it, the innermost part of it, which is in contact with
the movable body; and for this reason the earth is in water, and this
in the air, and the air in the aether, and the aether in heaven, but
we cannot go on and say that the heaven is in anything else.

It is clear, too, from these considerations that all the problems
which were raised about place will be solved when it is explained
in this way: 

(1) There is no necessity that the place should grow with the body
in it, 

(2) Nor that a point should have a place, 
(3) Nor that two bodies should be in the same place, 
(4) Nor that place should be a corporeal interval: for what is between
the boundaries of the place is any body which may chance to be there,
not an interval in body. 

Further, (5) place is also somewhere, not in the sense of being in
a place, but as the limit is in the limited; for not everything that
is is in place, but only movable body. 

Also (6) it is reasonable that each kind of body should be carried
to its own place. For a body which is next in the series and in contact
(not by compulsion) is akin, and bodies which are united do not affect
each other, while those which are in contact interact on each other.

Nor (7) is it without reason that each should remain naturally in
its proper place. For this part has the same relation to its place,
as a separable part to its whole, as when one moves a part of water
or air: so, too, air is related to water, for the one is like matter,
the other form-water is the matter of air, air as it were the actuality
of water, for water is potentially air, while air is potentially water,
though in another way. 

These distinctions will be drawn more carefully later. On the present
occasion it was necessary to refer to them: what has now been stated
obscurely will then be made more clear. If the matter and the fulfilment
are the same thing (for water is both, the one potentially, the other
completely), water will be related to air in a way as part to whole.
That is why these have contact: it is organic union when both become
actually one. 

This concludes my account of place-both of its existence and of its
nature. 

Part 6

The investigation of similar questions about the void, also, must
be held to belong to the physicist-namely whether it exists or not,
and how it exists or what it is-just as about place. The views taken
of it involve arguments both for and against, in much the same sort
of way. For those who hold that the void exists regard it as a sort
of place or vessel which is supposed to be 'full' when it holds the
bulk which it is capable of containing, 'void' when it is deprived
of that-as if 'void' and 'full' and 'place' denoted the same thing,
though the essence of the three is different. 

We must begin the inquiry by putting down the account given by those
who say that it exists, then the account of those who say that it
does not exist, and third the current view on these questions.

Those who try to show that the void does not exist do not disprove
what people really mean by it, but only their erroneous way of speaking;
this is true of Anaxagoras and of those who refute the existence of
the void in this way. They merely give an ingenious demonstration
that air is something--by straining wine-skins and showing the resistance
of the air, and by cutting it off in clepsydras. But people really
mean that there is an empty interval in which there is no sensible
body. They hold that everything which is in body is body and say that
what has nothing in it at all is void (so what is full of air is void).
It is not then the existence of air that needs to be proved, but the
non-existence of an interval, different from the bodies, either separable
or actual-an interval which divides the whole body so as to break
its continuity, as Democritus and Leucippus hold, and many other physicists-or
even perhaps as something which is outside the whole body, which remains
continuous. 

These people, then, have not reached even the threshold of the problem,
but rather those who say that the void exists. 

(1) They argue, for one thing, that change in place (i.e. locomotion
and increase) would not be. For it is maintained that motion would
seem not to exist, if there were no void, since what is full cannot
contain anything more. If it could, and there were two bodies in the
same place, it would also be true that any number of bodies could
be together; for it is impossible to draw a line of division beyond
which the statement would become untrue. If this were possible, it
would follow also that the smallest body would contain the greatest;
for 'many a little makes a mickle': thus if many equal bodies can
be together, so also can many unequal bodies. 

Melissus, indeed, infers from these considerations that the All is
immovable; for if it were moved there must, he says, be void, but
void is not among the things that exist. 

This argument, then, is one way in which they show that there is a
void. 

(2) They reason from the fact that some things are observed to contract
and be compressed, as people say that a cask will hold the wine which
formerly filled it, along with the skins into which the wine has been
decanted, which implies that the compressed body contracts into the
voids present in it. 

Again (3) increase, too, is thought to take always by means of void,
for nutriment is body, and it is impossible for two bodies to be together.
A proof of this they find also in what happens to ashes, which absorb
as much water as the empty vessel. 

The Pythagoreans, too, (4) held that void exists and that it enters
the heaven itself, which as it were inhales it, from the infinite
air. Further it is the void which distinguishes the natures of things,
as if it were like what separates and distinguishes the terms of a
series. This holds primarily in the numbers, for the void distinguishes
their nature. 

These, then, and so many, are the main grounds on which people have
argued for and against the existence of the void. 

Part 7

As a step towards settling which view is true, we must determine the
meaning of the name. 

The void is thought to be place with nothing in it. The reason for
this is that people take what exists to be body, and hold that while
every body is in place, void is place in which there is no body, so
that where there is no body, there must be void. 

Every body, again, they suppose to be tangible; and of this nature
is whatever has weight or lightness. 

Hence, by a syllogism, what has nothing heavy or light in it, is void.

This result, then, as I have said, is reached by syllogism. It would
be absurd to suppose that the point is void; for the void must be
place which has in it an interval in tangible body. 

But at all events we observe then that in one way the void is described
as what is not full of body perceptible to touch; and what has heaviness
and lightness is perceptible to touch. So we would raise the question:
what would they say of an interval that has colour or sound-is it
void or not? Clearly they would reply that if it could receive what
is tangible it was void, and if not, not. 

In another way void is that in which there is no 'this' or corporeal
substance. So some say that the void is the matter of the body (they
identify the place, too, with this), and in this they speak incorrectly;
for the matter is not separable from the things, but they are inquiring
about the void as about something separable. 

Since we have determined the nature of place, and void must, if it
exists, be place deprived of body, and we have stated both in what
sense place exists and in what sense it does not, it is plain that
on this showing void does not exist, either unseparated or separated;
the void is meant to be, not body but rather an interval in body.
This is why the void is thought to be something, viz. because place
is, and for the same reasons. For the fact of motion in respect of
place comes to the aid both of those who maintain that place is something
over and above the bodies that come to occupy it, and of those who
maintain that the void is something. They state that the void is the
condition of movement in the sense of that in which movement takes
place; and this would be the kind of thing that some say place is.

But there is no necessity for there being a void if there is movement.
It is not in the least needed as a condition of movement in general,
for a reason which, incidentally, escaped Melissus; viz. that the
full can suffer qualitative change. 

But not even movement in respect of place involves a void; for bodies
may simultaneously make room for one another, though there is no interval
separate and apart from the bodies that are in movement. And this
is plain even in the rotation of continuous things, as in that of
liquids. 

And things can also be compressed not into a void but because they
squeeze out what is contained in them (as, for instance, when water
is compressed the air within it is squeezed out); and things can increase
in size not only by the entrance of something but also by qualitative
change; e.g. if water were to be transformed into air. 

In general, both the argument about increase of size and that about
water poured on to the ashes get in their own way. For either not
any and every part of the body is increased, or bodies may be increased
otherwise than by the addition of body, or there may be two bodies
in the same place (in which case they are claiming to solve a quite
general difficulty, but are not proving the existence of void), or
the whole body must be void, if it is increased in every part and
is increased by means of void. The same argument applies to the ashes.

It is evident, then, that it is easy to refute the arguments by which
they prove the existence of the void. 

Part 8

Let us explain again that there is no void existing separately, as
some maintain. If each of the simple bodies has a natural locomotion,
e.g. fire upward and earth downward and towards the middle of the
universe, it is clear that it cannot be the void that is the condition
of locomotion. What, then, will the void be the condition of? It is
thought to be the condition of movement in respect of place, and it
is not the condition of this. 

Again, if void is a sort of place deprived of body, when there is
a void where will a body placed in it move to? It certainly cannot
move into the whole of the void. The same argument applies as against
those who think that place is something separate, into which things
are carried; viz. how will what is placed in it move, or rest? Much
the same argument will apply to the void as to the 'up' and 'down'
in place, as is natural enough since those who maintain the existence
of the void make it a place. 

And in what way will things be present either in place-or in the void?
For the expected result does not take place when a body is placed
as a whole in a place conceived of as separate and permanent; for
a part of it, unless it be placed apart, will not be in a place but
in the whole. Further, if separate place does not exist, neither will
void. 

If people say that the void must exist, as being necessary if there
is to be movement, what rather turns out to be the case, if one the
matter, is the opposite, that not a single thing can be moved if there
is a void; for as with those who for a like reason say the earth is
at rest, so, too, in the void things must be at rest; for there is
no place to which things can move more or less than to another; since
the void in so far as it is void admits no difference. 

The second reason is this: all movement is either compulsory or according
to nature, and if there is compulsory movement there must also be
natural (for compulsory movement is contrary to nature, and movement
contrary to nature is posterior to that according to nature, so that
if each of the natural bodies has not a natural movement, none of
the other movements can exist); but how can there be natural movement
if there is no difference throughout the void or the infinite? For
in so far as it is infinite, there will be no up or down or middle,
and in so far as it is a void, up differs no whit from down; for as
there is no difference in what is nothing, there is none in the void
(for the void seems to be a non-existent and a privation of being),
but natural locomotion seems to be differentiated, so that the things
that exist by nature must be differentiated. Either, then, nothing
has a natural locomotion, or else there is no void. 

Further, in point of fact things that are thrown move though that
which gave them their impulse is not touching them, either by reason
of mutual replacement, as some maintain, or because the air that has
been pushed pushes them with a movement quicker than the natural locomotion
of the projectile wherewith it moves to its proper place. But in a
void none of these things can take place, nor can anything be moved
save as that which is carried is moved. 

Further, no one could say why a thing once set in motion should stop
anywhere; for why should it stop here rather than here? So that a
thing will either be at rest or must be moved ad infinitum, unless
something more powerful get in its way. 

Further, things are now thought to move into the void because it yields;
but in a void this quality is present equally everywhere, so that
things should move in all directions. 

Further, the truth of what we assert is plain from the following considerations.
We see the same weight or body moving faster than another for two
reasons, either because there is a difference in what it moves through,
as between water, air, and earth, or because, other things being equal,
the moving body differs from the other owing to excess of weight or
of lightness. 

Now the medium causes a difference because it impedes the moving thing,
most of all if it is moving in the opposite direction, but in a secondary
degree even if it is at rest; and especially a medium that is not
easily divided, i.e. a medium that is somewhat dense. A, then, will
move through B in time G, and through D, which is thinner, in time
E (if the length of B is egual to D), in proportion to the density
of the hindering body. For let B be water and D air; then by so much
as air is thinner and more incorporeal than water, A will move through
D faster than through B. Let the speed have the same ratio to the
speed, then, that air has to water. Then if air is twice as thin,
the body will traverse B in twice the time that it does D, and the
time G will be twice the time E. And always, by so much as the medium
is more incorporeal and less resistant and more easily divided, the
faster will be the movement. 

Now there is no ratio in which the void is exceeded by body, as there
is no ratio of 0 to a number. For if 4 exceeds 3 by 1, and 2 by more
than 1, and 1 by still more than it exceeds 2, still there is no ratio
by which it exceeds 0; for that which exceeds must be divisible into
the excess + that which is exceeded, so that will be what it exceeds
0 by + 0. For this reason, too, a line does not exceed a point unless
it is composed of points! Similarly the void can bear no ratio to
the full, and therefore neither can movement through the one to movement
through the other, but if a thing moves through the thickest medium
such and such a distance in such and such a time, it moves through
the void with a speed beyond any ratio. For let Z be void, equal in
magnitude to B and to D. Then if A is to traverse and move through
it in a certain time, H, a time less than E, however, the void will
bear this ratio to the full. But in a time equal to H, A will traverse
the part O of A. And it will surely also traverse in that time any
substance Z which exceeds air in thickness in the ratio which the
time E bears to the time H. For if the body Z be as much thinner than
D as E exceeds H, A, if it moves through Z, will traverse it in a
time inverse to the speed of the movement, i.e. in a time equal to
H. If, then, there is no body in Z, A will traverse Z still more quickly.
But we supposed that its traverse of Z when Z was void occupied the
time H. So that it will traverse Z in an equal time whether Z be full
or void. But this is impossible. It is plain, then, that if there
is a time in which it will move through any part of the void, this
impossible result will follow: it will be found to traverse a certain
distance, whether this be full or void, in an equal time; for there
will be some body which is in the same ratio to the other body as
the time is to the time. 

To sum the matter up, the cause of this result is obvious, viz. that
between any two movements there is a ratio (for they occupy time,
and there is a ratio between any two times, so long as both are finite),
but there is no ratio of void to full. 

These are the consequences that result from a difference in the media;
the following depend upon an excess of one moving body over another.
We see that bodies which have a greater impulse either of weight or
of lightness, if they are alike in other respects, move faster over
an equal space, and in the ratio which their magnitudes bear to each
other. Therefore they will also move through the void with this ratio
of speed. But that is impossible; for why should one move faster?
(In moving through plena it must be so; for the greater divides them
faster by its force. For a moving thing cleaves the medium either
by its shape, or by the impulse which the body that is carried along
or is projected possesses.) Therefore all will possess equal velocity.
But this is impossible. 

It is evident from what has been said, then, that, if there is a void,
a result follows which is the very opposite of the reason for which
those who believe in a void set it up. They think that if movement
in respect of place is to exist, the void cannot exist, separated
all by itself; but this is the same as to say that place is a separate
cavity; and this has already been stated to be impossible.

But even if we consider it on its own merits the so-called vacuum
will be found to be really vacuous. For as, if one puts a cube in
water, an amount of water equal to the cube will be displaced; so
too in air; but the effect is imperceptible to sense. And indeed always
in the case of any body that can be displaced, must, if it is not
compressed, be displaced in the direction in which it is its nature
to be displaced-always either down, if its locomotion is downwards
as in the case of earth, or up, if it is fire, or in both directions-whatever
be the nature of the inserted body. Now in the void this is impossible;
for it is not body; the void must have penetrated the cube to a distance
equal to that which this portion of void formerly occupied in the
void, just as if the water or air had not been displaced by the wooden
cube, but had penetrated right through it. 

But the cube also has a magnitude equal to that occupied by the void;
a magnitude which, if it is also hot or cold, or heavy or light, is
none the less different in essence from all its attributes, even if
it is not separable from them; I mean the volume of the wooden cube.
So that even if it were separated from everything else and were neither
heavy nor light, it will occupy an equal amount of void, and fill
the same place, as the part of place or of the void equal to itself.
How then will the body of the cube differ from the void or place that
is equal to it? And if there can be two such things, why cannot there
be any number coinciding? 

This, then, is one absurd and impossible implication of the theory.
It is also evident that the cube will have this same volume even if
it is displaced, which is an attribute possessed by all other bodies
also. Therefore if this differs in no respect from its place, why
need we assume a place for bodies over and above the volume of each,
if their volume be conceived of as free from attributes? It contributes
nothing to the situation if there is an equal interval attached to
it as well. [Further it ought to be clear by the study of moving things
what sort of thing void is. But in fact it is found nowhere in the
world. For air is something, though it does not seem to be so-nor,
for that matter, would water, if fishes were made of iron; for the
discrimination of the tangible is by touch.] 

It is clear, then, from these considerations that there is no separate
void. 

Part 9

There are some who think that the existence of rarity and density
shows that there is a void. If rarity and density do not exist, they
say, neither can things contract and be compressed. But if this were
not to take place, either there would be no movement at all, or the
universe would bulge, as Xuthus said, or air and water must always
change into equal amounts (e.g. if air has been made out of a cupful
of water, at the same time out of an equal amount of air a cupful
of water must have been made), or void must necessarily exist; for
compression and expansion cannot take place otherwise. 

Now, if they mean by the rare that which has many voids existing separately,
it is plain that if void cannot exist separate any more than a place
can exist with an extension all to itself, neither can the rare exist
in this sense. But if they mean that there is void, not separately
existent, but still present in the rare, this is less impossible,
yet, first, the void turns out not to be a condition of all movement,
but only of movement upwards (for the rare is light, which is the
reason why they say fire is rare); second, the void turns out to be
a condition of movement not as that in which it takes place, but in
that the void carries things up as skins by being carried up themselves
carry up what is continuous with them. Yet how can void have a local
movement or a place? For thus that into which void moves is till then
void of a void. 

Again, how will they explain, in the case of what is heavy, its movement
downwards? And it is plain that if the rarer and more void a thing
is the quicker it will move upwards, if it were completely void it
would move with a maximum speed! But perhaps even this is impossible,
that it should move at all; the same reason which showed that in the
void all things are incapable of moving shows that the void cannot
move, viz. the fact that the speeds are incomparable. 

Since we deny that a void exists, but for the rest the problem has
been truly stated, that either there will be no movement, if there
is not to be condensation and rarefaction, or the universe will bulge,
or a transformation of water into air will always be balanced by an
equal transformation of air into water (for it is clear that the air
produced from water is bulkier than the water): it is necessary therefore,
if compression does not exist, either that the next portion will be
pushed outwards and make the outermost part bulge, or that somewhere
else there must be an equal amount of water produced out of air, so
that the entire bulk of the whole may be equal, or that nothing moves.
For when anything is displaced this will always happen, unless it
comes round in a circle; but locomotion is not always circular, but
sometimes in a straight line. 

These then are the reasons for which they might say that there is
a void; our statement is based on the assumption that there is a single
matter for contraries, hot and cold and the other natural contrarieties,
and that what exists actually is produced from a potential existent,
and that matter is not separable from the contraries but its being
is different, and that a single matter may serve for colour and heat
and cold. 

The same matter also serves for both a large and a small body. This
is evident; for when air is produced from water, the same matter has
become something different, not by acquiring an addition to it, but
has become actually what it was potentially, and, again, water is
produced from air in the same way, the change being sometimes from
smallness to greatness, and sometimes from greatness to smallness.
Similarly, therefore, if air which is large in extent comes to have
a smaller volume, or becomes greater from being smaller, it is the
matter which is potentially both that comes to be each of the two.

For as the same matter becomes hot from being cold, and cold from
being hot, because it was potentially both, so too from hot it can
become more hot, though nothing in the matter has become hot that
was not hot when the thing was less hot; just as, if the arc or curve
of a greater circle becomes that of a smaller, whether it remains
the same or becomes a different curve, convexity has not come to exist
in anything that was not convex but straight (for differences of degree
do not depend on an intermission of the quality); nor can we get any
portion of a flame, in which both heat and whiteness are not present.
So too, then, is the earlier heat related to the later. So that the
greatness and smallness, also, of the sensible volume are extended,
not by the matter's acquiring anything new, but because the matter
is potentially matter for both states; so that the same thing is dense
and rare, and the two qualities have one matter. 

The dense is heavy, and the rare is light. [Again, as the arc of a
circle when contracted into a smaller space does not acquire a new
part which is convex, but what was there has been contracted; and
as any part of fire that one takes will be hot; so, too, it is all
a question of contraction and expansion of the same matter.] There
are two types in each case, both in the dense and in the rare; for
both the heavy and the hard are thought to be dense, and contrariwise
both the light and the soft are rare; and weight and hardness fail
to coincide in the case of lead and iron. 

From what has been said it is evident, then, that void does not exist
either separate (either absolutely separate or as a separate element
in the rare) or potentially, unless one is willing to call the condition
of movement void, whatever it may be. At that rate the matter of the
heavy and the light, qua matter of them, would be the void; for the
dense and the rare are productive of locomotion in virtue of this
contrariety, and in virtue of their hardness and softness productive
of passivity and impassivity, i.e. not of locomotion but rather of
qualitative change. 

So much, then, for the discussion of the void, and of the sense in
which it exists and the sense in which it does not exist.

Part 10

Next for discussion after the subjects mentioned is Time. The best
plan will be to begin by working out the difficulties connected with
it, making use of the current arguments. First, does it belong to
the class of things that exist or to that of things that do not exist?
Then secondly, what is its nature? To start, then: the following considerations
would make one suspect that it either does not exist at all or barely,
and in an obscure way. One part of it has been and is not, while the
other is going to be and is not yet. Yet time-both infinite time and
any time you like to take-is made up of these. One would naturally
suppose that what is made up of things which do not exist could have
no share in reality. 

Further, if a divisible thing is to exist, it is necessary that, when
it exists, all or some of its parts must exist. But of time some parts
have been, while others have to be, and no part of it is though it
is divisible. For what is 'now' is not a part: a part is a measure
of the whole, which must be made up of parts. Time, on the other hand,
is not held to be made up of 'nows'. 

Again, the 'now' which seems to bound the past and the future-does
it always remain one and the same or is it always other and other?
It is hard to say. 

(1) If it is always different and different, and if none of the parts
in time which are other and other are simultaneous (unless the one
contains and the other is contained, as the shorter time is by the
longer), and if the 'now' which is not, but formerly was, must have
ceased-to-be at some time, the 'nows' too cannot be simultaneous with
one another, but the prior 'now' must always have ceased-to-be. But
the prior 'now' cannot have ceased-to-be in itself (since it then
existed); yet it cannot have ceased-to-be in another 'now'. For we
may lay it down that one 'now' cannot be next to another, any more
than point to point. If then it did not cease-to-be in the next 'now'
but in another, it would exist simultaneously with the innumerable
'nows' between the two-which is impossible. 

Yes, but (2) neither is it possible for the 'now' to remain always
the same. No determinate divisible thing has a single termination,
whether it is continuously extended in one or in more than one dimension:
but the 'now' is a termination, and it is possible to cut off a determinate
time. Further, if coincidence in time (i.e. being neither prior nor
posterior) means to be 'in one and the same "now"', then, if both
what is before and what is after are in this same 'now', things which
happened ten thousand years ago would be simultaneous with what has
happened to-day, and nothing would be before or after anything else.

This may serve as a statement of the difficulties about the attributes
of time. 

As to what time is or what is its nature, the traditional accounts
give us as little light as the preliminary problems which we have
worked through. 

Some assert that it is (1) the movement of the whole, others that
it is (2) the sphere itself. 

(1) Yet part, too, of the revolution is a time, but it certainly is
not a revolution: for what is taken is part of a revolution, not a
revolution. Besides, if there were more heavens than one, the movement
of any of them equally would be time, so that there would be many
times at the same time. 

(2) Those who said that time is the sphere of the whole thought so,
no doubt, on the ground that all things are in time and all things
are in the sphere of the whole. The view is too naive for it to be
worth while to consider the impossibilities implied in it.

But as time is most usually supposed to be (3) motion and a kind of
change, we must consider this view. 

Now (a) the change or movement of each thing is only in the thing
which changes or where the thing itself which moves or changes may
chance to be. But time is present equally everywhere and with all
things. 

Again, (b) change is always faster or slower, whereas time is not:
for 'fast' and 'slow' are defined by time-'fast' is what moves much
in a short time, 'slow' what moves little in a long time; but time
is not defined by time, by being either a certain amount or a certain
kind of it. 

Clearly then it is not movement. (We need not distinguish at present
between 'movement' and 'change'.) 

Part 11

But neither does time exist without change; for when the state of
our own minds does not change at all, or we have not noticed its changing,
we do not realize that time has elapsed, any more than those who are
fabled to sleep among the heroes in Sardinia do when they are awakened;
for they connect the earlier 'now' with the later and make them one,
cutting out the interval because of their failure to notice it. So,
just as, if the 'now' were not different but one and the same, there
would not have been time, so too when its difference escapes our notice
the interval does not seem to be time. If, then, the non-realization
of the existence of time happens to us when we do not distinguish
any change, but the soul seems to stay in one indivisible state, and
when we perceive and distinguish we say time has elapsed, evidently
time is not independent of movement and change. It is evident, then,
that time is neither movement nor independent of movement.

We must take this as our starting-point and try to discover-since
we wish to know what time is-what exactly it has to do with movement.

Now we perceive movement and time together: for even when it is dark
and we are not being affected through the body, if any movement takes
place in the mind we at once suppose that some time also has elapsed;
and not only that but also, when some time is thought to have passed,
some movement also along with it seems to have taken place. Hence
time is either movement or something that belongs to movement. Since
then it is not movement, it must be the other. 

But what is moved is moved from something to something, and all magnitude
is continuous. Therefore the movement goes with the magnitude. Because
the magnitude is continuous, the movement too must be continuous,
and if the movement, then the time; for the time that has passed is
always thought to be in proportion to the movement. 

The distinction of 'before' and 'after' holds primarily, then, in
place; and there in virtue of relative position. Since then 'before'
and 'after' hold in magnitude, they must hold also in movement, these
corresponding to those. But also in time the distinction of 'before'
and 'after' must hold, for time and movement always correspond with
each other. The 'before' and 'after' in motion is identical in substratum
with motion yet differs from it in definition, and is not identical
with motion. 

But we apprehend time only when we have marked motion, marking it
by 'before' and 'after'; and it is only when we have perceived 'before'
and 'after' in motion that we say that time has elapsed. Now we mark
them by judging that A and B are different, and that some third thing
is intermediate to them. When we think of the extremes as different
from the middle and the mind pronounces that the 'nows' are two, one
before and one after, it is then that we say that there is time, and
this that we say is time. For what is bounded by the 'now' is thought
to be time-we may assume this. 

When, therefore, we perceive the 'now' one, and neither as before
and after in a motion nor as an identity but in relation to a 'before'
and an 'after', no time is thought to have elapsed, because there
has been no motion either. On the other hand, when we do perceive
a 'before' and an 'after', then we say that there is time. For time
is just this-number of motion in respect of 'before' and 'after'.

Hence time is not movement, but only movement in so far as it admits
of enumeration. A proof of this: we discriminate the more or the less
by number, but more or less movement by time. Time then is a kind
of number. (Number, we must note, is used in two senses-both of what
is counted or the countable and also of that with which we count.
Time obviously is what is counted, not that with which we count: there
are different kinds of thing.) Just as motion is a perpetual succession,
so also is time. But every simultaneous time is self-identical; for
the 'now' as a subject is an identity, but it accepts different attributes.
The 'now' measures time, in so far as time involves the 'before and
after'. 

The 'now' in one sense is the same, in another it is not the same.
In so far as it is in succession, it is different (which is just what
its being was supposed to mean), but its substratum is an identity:
for motion, as was said, goes with magnitude, and time, as we maintain,
with motion. Similarly, then, there corresponds to the point the body
which is carried along, and by which we are aware of the motion and
of the 'before and after' involved in it. This is an identical substratum
(whether a point or a stone or something else of the kind), but it
has different attributes as the sophists assume that Coriscus' being
in the Lyceum is a different thing from Coriscus' being in the market-place.
And the body which is carried along is different, in so far as it
is at one time here and at another there. But the 'now' corresponds
to the body that is carried along, as time corresponds to the motion.
For it is by means of the body that is carried along that we become
aware of the 'before and after' the motion, and if we regard these
as countable we get the 'now'. Hence in these also the 'now' as substratum
remains the same (for it is what is before and after in movement),
but what is predicated of it is different; for it is in so far as
the 'before and after' is numerable that we get the 'now'. This is
what is most knowable: for, similarly, motion is known because of
that which is moved, locomotion because of that which is carried.
what is carried is a real thing, the movement is not. Thus what is
called 'now' in one sense is always the same; in another it is not
the same: for this is true also of what is carried. 

Clearly, too, if there were no time, there would be no 'now', and
vice versa. just as the moving body and its locomotion involve each
other mutually, so too do the number of the moving body and the number
of its locomotion. For the number of the locomotion is time, while
the 'now' corresponds to the moving body, and is like the unit of
number. 

Time, then, also is both made continuous by the 'now' and divided
at it. For here too there is a correspondence with the locomotion
and the moving body. For the motion or locomotion is made one by the
thing which is moved, because it is one-not because it is one in its
own nature (for there might be pauses in the movement of such a thing)-but
because it is one in definition: for this determines the movement
as 'before' and 'after'. Here, too there is a correspondence with
the point; for the point also both connects and terminates the length-it
is the beginning of one and the end of another. But when you take
it in this way, using the one point as two, a pause is necessary,
if the same point is to be the beginning and the end. The 'now' on
the other hand, since the body carried is moving, is always different.

Hence time is not number in the sense in which there is 'number' of
the same point because it is beginning and end, but rather as the
extremities of a line form a number, and not as the parts of the line
do so, both for the reason given (for we can use the middle point
as two, so that on that analogy time might stand still), and further
because obviously the 'now' is no part of time nor the section any
part of the movement, any more than the points are parts of the line-for
it is two lines that are parts of one line. 

In so far then as the 'now' is a boundary, it is not time, but an
attribute of it; in so far as it numbers, it is number; for boundaries
belong only to that which they bound, but number (e.g. ten) is the
number of these horses, and belongs also elsewhere. 

It is clear, then, that time is 'number of movement in respect of
the before and after', and is continuous since it is an attribute
of what is continuous. 

Part 12

The smallest number, in the strict sense of the word 'number', is
two. But of number as concrete, sometimes there is a minimum, sometimes
not: e.g. of a 'line', the smallest in respect of multiplicity is
two (or, if you like, one), but in respect of size there is no minimum;
for every line is divided ad infinitum. Hence it is so with time.
In respect of number the minimum is one (or two); in point of extent
there is no minimum. 

It is clear, too, that time is not described as fast or slow, but
as many or few and as long or short. For as continuous it is long
or short and as a number many or few, but it is not fast or slow-any
more than any number with which we number is fast or slow.

Further, there is the same time everywhere at once, but not the same
time before and after, for while the present change is one, the change
which has happened and that which will happen are different. Time
is not number with which we count, but the number of things which
are counted, and this according as it occurs before or after is always
different, for the 'nows' are different. And the number of a hundred
horses and a hundred men is the same, but the things numbered are
different-the horses from the men. Further, as a movement can be one
and the same again and again, so too can time, e.g. a year or a spring
or an autumn. 

Not only do we measure the movement by the time, but also the time
by the movement, because they define each other. The time marks the
movement, since it is its number, and the movement the time. We describe
the time as much or little, measuring it by the movement, just as
we know the number by what is numbered, e.g. the number of the horses
by one horse as the unit. For we know how many horses there are by
the use of the number; and again by using the one horse as unit we
know the number of the horses itself. So it is with the time and the
movement; for we measure the movement by the time and vice versa.
It is natural that this should happen; for the movement goes with
the distance and the time with the movement, because they are quanta
and continuous and divisible. The movement has these attributes because
the distance is of this nature, and the time has them because of the
movement. And we measure both the distance by the movement and the
movement by the distance; for we say that the road is long, if the
journey is long, and that this is long, if the road is long-the time,
too, if the movement, and the movement, if the time. 

Time is a measure of motion and of being moved, and it measures the
motion by determining a motion which will measure exactly the whole
motion, as the cubit does the length by determining an amount which
will measure out the whole. Further 'to be in time' means for movement,
that both it and its essence are measured by time (for simultaneously
it measures both the movement and its essence, and this is what being
in time means for it, that its essence should be measured).

Clearly then 'to be in time' has the same meaning for other things
also, namely, that their being should be measured by time. 'To be
in time' is one of two things: (1) to exist when time exists, (2)
as we say of some things that they are 'in number'. The latter means
either what is a part or mode of number-in general, something which
belongs to number-or that things have a number. 

Now, since time is number, the 'now' and the 'before' and the like
are in time, just as 'unit' and 'odd' and 'even' are in number, i.e.
in the sense that the one set belongs to number, the other to time.
But things are in time as they are in number. If this is so, they
are contained by time as things in place are contained by place.

Plainly, too, to be in time does not mean to co-exist with time, any
more than to be in motion or in place means to co-exist with motion
or place. For if 'to be in something' is to mean this, then all things
will be in anything, and the heaven will be in a grain; for when the
grain is, then also is the heaven. But this is a merely incidental
conjunction, whereas the other is necessarily involved: that which
is in time necessarily involves that there is time when it is, and
that which is in motion that there is motion when it is.

Since what is 'in time' is so in the same sense as what is in number
is so, a time greater than everything in time can be found. So it
is necessary that all the things in time should be contained by time,
just like other things also which are 'in anything', e.g. the things
'in place' by place. 

A thing, then, will be affected by time, just as we are accustomed
to say that time wastes things away, and that all things grow old
through time, and that there is oblivion owing to the lapse of time,
but we do not say the same of getting to know or of becoming young
or fair. For time is by its nature the cause rather of decay, since
it is the number of change, and change removes what is. 

Hence, plainly, things which are always are not, as such, in time,
for they are not contained time, nor is their being measured by time.
A proof of this is that none of them is affected by time, which indicates
that they are not in time. 

Since time is the measure of motion, it will be the measure of rest
too-indirectly. For all rest is in time. For it does not follow that
what is in time is moved, though what is in motion is necessarily
moved. For time is not motion, but 'number of motion': and what is
at rest, also, can be in the number of motion. Not everything that
is not in motion can be said to be 'at rest'-but only that which can
be moved, though it actually is not moved, as was said above.

'To be in number' means that there is a number of the thing, and that
its being is measured by the number in which it is. Hence if a thing
is 'in time' it will be measured by time. But time will measure what
is moved and what is at rest, the one qua moved, the other qua at
rest; for it will measure their motion and rest respectively.

Hence what is moved will not be measurable by the time simply in so
far as it has quantity, but in so far as its motion has quantity.
Thus none of the things which are neither moved nor at rest are in
time: for 'to be in time' is 'to be measured by time', while time
is the measure of motion and rest. 

Plainly, then, neither will everything that does not exist be in time,
i.e. those non-existent things that cannot exist, as the diagonal
cannot be commensurate with the side. 

Generally, if time is directly the measure of motion and indirectly
of other things, it is clear that a thing whose existence is measured
by it will have its existence in rest or motion. Those things therefore
which are subject to perishing and becoming-generally, those which
at one time exist, at another do not-are necessarily in time: for
there is a greater time which will extend both beyond their existence
and beyond the time which measures their existence. Of things which
do not exist but are contained by time some were, e.g. Homer once
was, some will be, e.g. a future event; this depends on the direction
in which time contains them; if on both, they have both modes of existence.
As to such things as it does not contain in any way, they neither
were nor are nor will be. These are those nonexistents whose opposites
always are, as the incommensurability of the diagonal always is-and
this will not be in time. Nor will the commensurability, therefore;
hence this eternally is not, because it is contrary to what eternally
is. A thing whose contrary is not eternal can be and not be, and it
is of such things that there is coming to be and passing away.

Part 13

The 'now' is the link of time, as has been said (for it connects past
and future time), and it is a limit of time (for it is the beginning
of the one and the end of the other). But this is not obvious as it
is with the point, which is fixed. It divides potentially, and in
so far as it is dividing the 'now' is always different, but in so
far as it connects it is always the same, as it is with mathematical
lines. For the intellect it is not always one and the same point,
since it is other and other when one divides the line; but in so far
as it is one, it is the same in every respect. 

So the 'now' also is in one way a potential dividing of time, in another
the termination of both parts, and their unity. And the dividing and
the uniting are the same thing and in the same reference, but in essence
they are not the same. 

So one kind of 'now' is described in this way: another is when the
time is near this kind of 'now'. 'He will come now' because he will
come to-day; 'he has come now' because he came to-day. But the things
in the Iliad have not happened 'now', nor is the flood 'now'-not that
the time from now to them is not continuous, but because they are
not near. 

'At some time' means a time determined in relation to the first of
the two types of 'now', e.g. 'at some time' Troy was taken, and 'at
some time' there will be a flood; for it must be determined with reference
to the 'now'. There will thus be a determinate time from this 'now'
to that, and there was such in reference to the past event. But if
there be no time which is not 'sometime', every time will be determined.

Will time then fail? Surely not, if motion always exists. Is time
then always different or does the same time recur? Clearly time is,
in the same way as motion is. For if one and the same motion sometimes
recurs, it will be one and the same time, and if not, not.

Since the 'now' is an end and a beginning of time, not of the same
time however, but the end of that which is past and the beginning
of that which is to come, it follows that, as the circle has its convexity
and its concavity, in a sense, in the same thing, so time is always
at a beginning and at an end. And for this reason it seems to be always
different; for the 'now' is not the beginning and the end of the same
thing; if it were, it would be at the same time and in the same respect
two opposites. And time will not fail; for it is always at a beginning.

'Presently' or 'just' refers to the part of future time which is near
the indivisible present 'now' ('When do you walk? 'Presently', because
the time in which he is going to do so is near), and to the part of
past time which is not far from the 'now' ('When do you walk?' 'I
have just been walking'). But to say that Troy has just been taken-we
do not say that, because it is too far from the 'now'. 'Lately', too,
refers to the part of past time which is near the present 'now'. 'When
did you go?' 'Lately', if the time is near the existing now. 'Long
ago' refers to the distant past. 

'Suddenly' refers to what has departed from its former condition in
a time imperceptible because of its smallness; but it is the nature
of all change to alter things from their former condition. In time
all things come into being and pass away; for which reason some called
it the wisest of all things, but the Pythagorean Paron called it the
most stupid, because in it we also forget; and his was the truer view.
It is clear then that it must be in itself, as we said before, the
condition of destruction rather than of coming into being (for change,
in itself, makes things depart from their former condition), and only
incidentally of coming into being, and of being. A sufficient evidence
of this is that nothing comes into being without itself moving somehow
and acting, but a thing can be destroyed even if it does not move
at all. And this is what, as a rule, we chiefly mean by a thing's
being destroyed by time. Still, time does not work even this change;
even this sort of change takes place incidentally in time.

We have stated, then, that time exists and what it is, and in how
many senses we speak of the 'now', and what 'at some time', 'lately',
'presently' or 'just', 'long ago', and 'suddenly' mean. 

Part 14

These distinctions having been drawn, it is evident that every change
and everything that moves is in time; for the distinction of faster
and slower exists in reference to all change, since it is found in
every instance. In the phrase 'moving faster' I refer to that which
changes before another into the condition in question, when it moves
over the same interval and with a regular movement; e.g. in the case
of locomotion, if both things move along the circumference of a circle,
or both along a straight line; and similarly in all other cases. But
what is before is in time; for we say 'before' and 'after' with reference
to the distance from the 'now', and the 'now' is the boundary of the
past and the future; so that since 'nows' are in time, the before
and the after will be in time too; for in that in which the 'now'
is, the distance from the 'now' will also be. But 'before' is used
contrariwise with reference to past and to future time; for in the
past we call 'before' what is farther from the 'now', and 'after'
what is nearer, but in the future we call the nearer 'before' and
the farther 'after'. So that since the 'before' is in time, and every
movement involves a 'before', evidently every change and every movement
is in time. 

It is also worth considering how time can be related to the soul;
and why time is thought to be in everything, both in earth and in
sea and in heaven. Is because it is an attribute, or state, or movement
(since it is the number of movement) and all these things are movable
(for they are all in place), and time and movement are together, both
in respect of potentiality and in respect of actuality? 

Whether if soul did not exist time would exist or not, is a question
that may fairly be asked; for if there cannot be some one to count
there cannot be anything that can be counted, so that evidently there
cannot be number; for number is either what has been, or what can
be, counted. But if nothing but soul, or in soul reason, is qualified
to count, there would not be time unless there were soul, but only
that of which time is an attribute, i.e. if movement can exist without
soul, and the before and after are attributes of movement, and time
is these qua numerable. 

One might also raise the question what sort of movement time is the
number of. Must we not say 'of any kind'? For things both come into
being in time and pass away, and grow, and are altered in time, and
are moved locally; thus it is of each movement qua movement that time
is the number. And so it is simply the number of continuous movement,
not of any particular kind of it. 

But other things as well may have been moved now, and there would
be a number of each of the two movements. Is there another time, then,
and will there be two equal times at once? Surely not. For a time
that is both equal and simultaneous is one and the same time, and
even those that are not simultaneous are one in kind; for if there
were dogs, and horses, and seven of each, it would be the same number.
So, too, movements that have simultaneous limits have the same time,
yet the one may in fact be fast and the other not, and one may be
locomotion and the other alteration; still the time of the two changes
is the same if their number also is equal and simultaneous; and for
this reason, while the movements are different and separate, the time
is everywhere the same, because the number of equal and simultaneous
movements is everywhere one and the same. 

Now there is such a thing as locomotion, and in locomotion there is
included circular movement, and everything is measured by some one
thing homogeneous with it, units by a unit, horses by a horse, and
similarly times by some definite time, and, as we said, time is measured
by motion as well as motion by time (this being so because by a motion
definite in time the quantity both of the motion and of the time is
measured): if, then, what is first is the measure of everything homogeneous
with it, regular circular motion is above all else the measure, because
the number of this is the best known. Now neither alteration nor increase
nor coming into being can be regular, but locomotion can be. This
also is why time is thought to be the movement of the sphere, viz.
because the other movements are measured by this, and time by this
movement. 

This also explains the common saying that human affairs form a circle,
and that there is a circle in all other things that have a natural
movement and coming into being and passing away. This is because all
other things are discriminated by time, and end and begin as though
conforming to a cycle; for even time itself is thought to be a circle.
And this opinion again is held because time is the measure of this
kind of locomotion and is itself measured by such. So that to say
that the things that come into being form a circle is to say that
there is a circle of time; and this is to say that it is measured
by the circular movement; for apart from the measure nothing else
to be measured is observed; the whole is just a plurality of measures.

It is said rightly, too, that the number of the sheep and of the dogs
is the same number if the two numbers are equal, but not the same
decad or the same ten; just as the equilateral and the scalene are
not the same triangle, yet they are the same figure, because they
are both triangles. For things are called the same so-and-so if they
do not differ by a differentia of that thing, but not if they do;
e.g. triangle differs from triangle by a differentia of triangle,
therefore they are different triangles; but they do not differ by
a differentia of figure, but are in one and the same division of it.
For a figure of the one kind is a circle and a figure of another kind
of triangle, and a triangle of one kind is equilateral and a triangle
of another kind scalene. They are the same figure, then, that, triangle,
but not the same triangle. Therefore the number of two groups also-is
the same number (for their number does not differ by a differentia
of number), but it is not the same decad; for the things of which
it is asserted differ; one group are dogs, and the other horses.

We have now discussed time-both time itself and the matters appropriate
to the consideration of it. 

----------------------------------------------------------------------

BOOK V

Part 1 

Everything which changes does so in one of three senses. It may change
(1) accidentally, as for instance when we say that something musical
walks, that which walks being something in which aptitude for music
is an accident. Again (2) a thing is said without qualification to
change because something belonging to it changes, i.e. in statements
which refer to part of the thing in question: thus the body is restored
to health because the eye or the chest, that is to say a part of the
whole body, is restored to health. And above all there is (3) the
case of a thing which is in motion neither accidentally nor in respect
of something else belonging to it, but in virtue of being itself directly
in motion. Here we have a thing which is essentially movable: and
that which is so is a different thing according to the particular
variety of motion: for instance it may be a thing capable of alteration:
and within the sphere of alteration it is again a different thing
according as it is capable of being restored to health or capable
of being heated. And there are the same distinctions in the case of
the mover: (1) one thing causes motion accidentally, (2) another partially
(because something belonging to it causes motion), (3) another of
itself directly, as, for instance, the physician heals, the hand strikes.
We have, then, the following factors: (a) on the one hand that which
directly causes motion, and (b) on the other hand that which is in
motion: further, we have (c) that in which motion takes place, namely
time, and (distinct from these three, d) that from which and (e)
that to which it proceeds: for every motion proceeds from something
and to something, that which is directly in motion being distinct
from that to which it is in motion and that from which it is in motion:
for instance, we may take the three things 'wood', 'hot', and 'cold',
of which the first is that which is in motion, the second is that
to which the motion proceeds, and the third is that from which it
proceeds. This being so, it is clear that the motion is in the wood,
not in its form: for the motion is neither caused nor experienced
by the form or the place or the quantity. So we are left with a mover,
a moved, and a goal of motion. I do not include the starting-point
of motion: for it is the goal rather than the starting-point of motion
that gives its name to a particular process of change. Thus 'perishing'
is change to not-being, though it is also true that that that which
perishes changes from being: and 'becoming' is change to being, though
it is also change from not-being. 

Now a definition of motion has been given above, from which it will
be seen that every goal of motion, whether it be a form, an affection,
or a place, is immovable, as, for instance, knowledge and heat. Here,
however, a difficulty may be raised. Affections, it may be said, are
motions, and whiteness is an affection: thus there may be change to
a motion. To this we may reply that it is not whiteness but whitening
that is a motion. Here also the same distinctions are to be observed:
a goal of motion may be so accidentally, or partially and with reference
to something other than itself, or directly and with no reference
to anything else: for instance, a thing which is becoming white changes
accidentally to an object of thought, the colour being only accidentally
the object of thought; it changes to colour, because white is a part
of colour, or to Europe, because Athens is a part of Europe; but it
changes essentially to white colour. It is now clear in what sense
a thing is in motion essentially, accidentally, or in respect of something
other than itself, and in what sense the phrase 'itself directly'
is used in the case both of the mover and of the moved: and it is
also clear that the motion is not in the form but in that which is
in motion, that is to say 'the movable in activity'. Now accidental
change we may leave out of account: for it is to be found in everything,
at any time, and in any respect. Change which is not accidental on
the other hand is not to be found in everything, but only in contraries,
in things intermediate contraries, and in contradictories, as may
be proved by induction. An intermediate may be a starting-point of
change, since for the purposes of the change it serves as contrary
to either of two contraries: for the intermediate is in a sense the
extremes. Hence we speak of the intermediate as in a sense a contrary
relatively to the extremes and of either extreme as a contrary relatively
to the intermediate: for instance, the central note is low relatively-to
the highest and high relatively to the lowest, and grey is light relatively
to black and dark relatively to white. 

And since every change is from something to something-as the word
itself (metabole) indicates, implying something 'after' (meta) something
else, that is to say something earlier and something later-that which
changes must change in one of four ways: from subject to subject,
from subject to nonsubject, from non-subject to subject, or from non-subject
to non-subject, where by 'subject' I mean what is affirmatively expressed.
So it follows necessarily from what has been said above that there
are only three kinds of change, that from subject to subject, that
from subject to non-subject, and that from non-subject to subject:
for the fourth conceivable kind, that from non-subject to nonsubject,
is not change, as in that case there is no opposition either of contraries
or of contradictories. 

Now change from non-subject to subject, the relation being that of
contradiction, is 'coming to be'-'unqualified coming to be' when the
change takes place in an unqualified way, 'particular coming to be'
when the change is change in a particular character: for instance,
a change from not-white to white is a coming to be of the particular
thing, white, while change from unqualified not-being to being is
coming to be in an unqualified way, in respect of which we say that
a thing 'comes to be' without qualification, not that it 'comes to
be' some particular thing. Change from subject to non-subject is 'perishing'-'unqualified
perishing' when the change is from being to not-being, 'particular
perishing' when the change is to the opposite negation, the distinction
being the same as that made in the case of coming to be.

Now the expression 'not-being' is used in several senses: and there
can be motion neither of that which 'is not' in respect of the affirmation
or negation of a predicate, nor of that which 'is not' in the sense
that it only potentially 'is', that is to say the opposite of that
which actually 'is' in an unqualified sense: for although that which
is 'not-white' or 'not-good' may nevertheless he in motion accidentally
(for example that which is 'not-white' might be a man), yet that which
is without qualification 'not-so-and-so' cannot in any sense be in
motion: therefore it is impossible for that which is not to be in
motion. This being so, it follows that 'becoming' cannot be a motion:
for it is that which 'is not' that 'becomes'. For however true it
may be that it accidentally 'becomes', it is nevertheless correct
to say that it is that which 'is not' that in an unqualified sense
'becomes'. And similarly it is impossible for that which 'is not'
to be at rest. 

There are these difficulties, then, in the way of the assumption that
that which 'is not' can be in motion: and it may be further objected
that, whereas everything which is in motion is in space, that which
'is not' is not in space: for then it would be somewhere.

So, too, 'perishing' is not a motion: for a motion has for its contrary
either another motion or rest, whereas 'perishing' is the contrary
of 'becoming'. 

Since, then, every motion is a kind of change, and there are only
the three kinds of change mentioned above, and since of these three
those which take the form of 'becoming' and 'perishing', that is to
say those which imply a relation of contradiction, are not motions:
it necessarily follows that only change from subject to subject is
motion. And every such subject is either a contrary or an intermediate
(for a privation may be allowed to rank as a contrary) and can be
affirmatively expressed, as naked, toothless, or black. If, then,
the categories are severally distinguished as Being, Quality, Place,
Time, Relation, Quantity, and Activity or Passivity, it necessarily
follows that there are three kinds of motion-qualitative, quantitative,
and local. 

Part 2

In respect of Substance there is no motion, because Substance has
no contrary among things that are. Nor is there motion in respect
of Relation: for it may happen that when one correlative changes,
the other, although this does not itself change, is no longer applicable,
so that in these cases the motion is accidental. Nor is there motion
in respect of Agent and Patient-in fact there can never be motion
of mover and moved, because there cannot be motion of motion or becoming
of becoming or in general change of change. 

For in the first place there are two senses in which motion of motion
is conceivable. (1) The motion of which there is motion might be conceived
as subject; e.g. a man is in motion because he changes from fair to
dark. Can it be that in this sense motion grows hot or cold, or changes
place, or increases or decreases? Impossible: for change is not a
subject. Or (2) can there be motion of motion in the sense that some
other subject changes from a change to another mode of being, as e.g.
a man changes from falling ill to getting well? Even this is possible
only in an accidental sense. For, whatever the subject may be, movement
is change from one form to another. (And the same holds good of becoming
and perishing, except that in these processes we have a change to
a particular kind of opposite, while the other, motion, is a change
to a different kind.) So, if there is to be motion of motion, that
which is changing from health to sickness must simultaneously be changing
from this very change to another. It is clear, then, that by the time
that it has become sick, it must also have changed to whatever may
be the other change concerned (for that it should be at rest, though
logically possible, is excluded by the theory). Moreover this other
can never be any casual change, but must be a change from something
definite to some other definite thing. So in this case it must be
the opposite change, viz. convalescence. It is only accidentally that
there can be change of change, e.g. there is a change from remembering
to forgetting only because the subject of this change changes at one
time to knowledge, at another to ignorance. 

In the second place, if there is to be change of change and becoming
of becoming, we shall have an infinite regress. Thus if one of a series
of changes is to be a change of change, the preceding change must
also be so: e.g. if simple becoming was ever in process of becoming,
then that which was becoming simple becoming was also in process of
becoming, so that we should not yet have arrived at what was in process
of simple becoming but only at what was already in process of becoming
in process of becoming. And this again was sometime in process of
becoming, so that even then we should not have arrived at what was
in process of simple becoming. And since in an infinite series there
is no first term, here there will be no first stage and therefore
no following stage either. On this hypothesis, then, nothing can become
or be moved or change. 

Thirdly, if a thing is capable of any particular motion, it is also
capable of the corresponding contrary motion or the corresponding
coming to rest, and a thing that is capable of becoming is also capable
of perishing: consequently, if there be becoming of becoming, that
which is in process of becoming is in process of perishing at the
very moment when it has reached the stage of becoming: since it cannot
be in process of perishing when it is just beginning to become or
after it has ceased to become: for that which is in process of perishing
must be in existence. 

Fourthly, there must be a substrate underlying all processes of becoming
and changing. What can this be in the present case? It is either the
body or the soul that undergoes alteration: what is it that correspondingly
becomes motion or becoming? And again what is the goal of their motion?
It must be the motion or becoming of something from something to something
else. But in what sense can this be so? For the becoming of learning
cannot be learning: so neither can the becoming of becoming be becoming,
nor can the becoming of any process be that process. 

Finally, since there are three kinds of motion, the substratum and
the goal of motion must be one or other of these, e.g. locomotion
will have to be altered or to be locally moved. 

To sum up, then, since everything that is moved is moved in one of
three ways, either accidentally, or partially, or essentially, change
can change only accidentally, as e.g. when a man who is being restored
to health runs or learns: and accidental change we have long ago decided
to leave out of account. 

Since, then, motion can belong neither to Being nor to Relation nor
to Agent and Patient, it remains that there can be motion only in
respect of Quality, Quantity, and Place: for with each of these we
have a pair of contraries. Motion in respect of Quality let us call
alteration, a general designation that is used to include both contraries:
and by Quality I do not here mean a property of substance (in that
sense that which constitutes a specific distinction is a quality)
but a passive quality in virtue of which a thing is said to be acted
on or to be incapable of being acted on. Motion in respect of Quantity
has no name that includes both contraries, but it is called increase
or decrease according as one or the other is designated: that is to
say motion in the direction of complete magnitude is increase, motion
in the contrary direction is decrease. Motion in respect of Place
has no name either general or particular: but we may designate it
by the general name of locomotion, though strictly the term 'locomotion'
is applicable to things that change their place only when they have
not the power to come to a stand, and to things that do not move themselves
locally. 

Change within the same kind from a lesser to a greater or from a greater
to a lesser degree is alteration: for it is motion either from a contrary
or to a contrary, whether in an unqualified or in a qualified sense:
for change to a lesser degree of a quality will be called change to
the contrary of that quality, and change to a greater degree of a
quality will be regarded as change from the contrary of that quality
to the quality itself. It makes no difference whether the change be
qualified or unqualified, except that in the former case the contraries
will have to be contrary to one another only in a qualified sense:
and a thing's possessing a quality in a greater or in a lesser degree
means the presence or absence in it of more or less of the opposite
quality. It is now clear, then, that there are only these three kinds
of motion. 

The term 'immovable' we apply in the first place to that which is
absolutely incapable of being moved (just as we correspondingly apply
the term invisible to sound); in the second place to that which is
moved with difficulty after a long time or whose movement is slow
at the start-in fact, what we describe as hard to move; and in the
third place to that which is naturally designed for and capable of
motion, but is not in motion when, where, and as it naturally would
be so. This last is the only kind of immovable thing of which I use
the term 'being at rest': for rest is contrary to motion, so that
rest will be negation of motion in that which is capable of admitting
motion. 

The foregoing remarks are sufficient to explain the essential nature
of motion and rest, the number of kinds of change, and the different
varieties of motion. 

Part 3

Let us now proceed to define the terms 'together' and 'apart', 'in
contact', 'between', 'in succession', 'contiguous', and 'continuous',
and to show in what circumstances each of these terms is naturally
applicable. 

Things are said to be together in place when they are in one place
(in the strictest sense of the word 'place') and to be apart when
they are in different places. 

Things are said to be in contact when their extremities are together.

That which a changing thing, if it changes continuously in a natural
manner, naturally reaches before it reaches that to which it changes
last, is between. Thus 'between' implies the presence of at least
three things: for in a process of change it is the contrary that is
'last': and a thing is moved continuously if it leaves no gap or only
the smallest possible gap in the material-not in the time (for a gap
in the time does not prevent things having a 'between', while, on
the other hand, there is nothing to prevent the highest note sounding
immediately after the lowest) but in the material in which the motion
takes place. This is manifestly true not only in local changes but
in every other kind as well. (Now every change implies a pair of opposites,
and opposites may be either contraries or contradictories; since then
contradiction admits of no mean term, it is obvious that 'between'
must imply a pair of contraries) That is locally contrary which is
most distant in a straight line: for the shortest line is definitely
limited, and that which is definitely limited constitutes a measure.

A thing is 'in succession' when it is after the beginning in position
or in form or in some other respect in which it is definitely so regarded,
and when further there is nothing of the same kind as itself between
it and that to which it is in succession, e.g. a line or lines if
it is a line, a unit or units if it is a unit, a house if it is a
house (there is nothing to prevent something of a different kind being
between). For that which is in succession is in succession to a particular
thing, and is something posterior: for one is not 'in succession'
to two, nor is the first day of the month to be second: in each case
the latter is 'in succession' to the former. 

A thing that is in succession and touches is 'contiguous'. The 'continuous'
is a subdivision of the contiguous: things are called continuous when
the touching limits of each become one and the same and are, as the
word implies, contained in each other: continuity is impossible if
these extremities are two. This definition makes it plain that continuity
belongs to things that naturally in virtue of their mutual contact
form a unity. And in whatever way that which holds them together is
one, so too will the whole be one, e.g. by a rivet or glue or contact
or organic union. 

It is obvious that of these terms 'in succession' is first in order
of analysis: for that which touches is necessarily in succession,
but not everything that is in succession touches: and so succession
is a property of things prior in definition, e.g. numbers, while contact
is not. And if there is continuity there is necessarily contact, but
if there is contact, that alone does not imply continuity: for the
extremities of things may be 'together' without necessarily being
one: but they cannot be one without being necessarily together. So
natural junction is last in coming to be: for the extremities must
necessarily come into contact if they are to be naturally joined:
but things that are in contact are not all naturally joined, while
there is no contact clearly there is no natural junction either. Hence,
if as some say 'point' and 'unit' have an independent existence of
their own, it is impossible for the two to be identical: for points
can touch while units can only be in succession. Moreover, there can
always be something between points (for all lines are intermediate
between points), whereas it is not necessary that there should possibly
be anything between units: for there can be nothing between the numbers
one and two. 

We have now defined what is meant by 'together' and 'apart', 'contact',
'between' and 'in succession', 'contiguous' and 'continuous': and
we have shown in what circumstances each of these terms is applicable.

Part 4

There are many senses in which motion is said to be 'one': for we
use the term 'one' in many senses. 

Motion is one generically according to the different categories to
which it may be assigned: thus any locomotion is one generically with
any other locomotion, whereas alteration is different generically
from locomotion. 

Motion is one specifically when besides being one generically it also
takes place in a species incapable of subdivision: e.g. colour has
specific differences: therefore blackening and whitening differ specifically;
but at all events every whitening will be specifically the same with
every other whitening and every blackening with every other blackening.
But white is not further subdivided by specific differences: hence
any whitening is specifically one with any other whitening. Where
it happens that the genus is at the same time a species, it is clear
that the motion will then in a sense be one specifically though not
in an unqualified sense: learning is an example of this, knowledge
being on the one hand a species of apprehension and on the other hand
a genus including the various knowledges. A difficulty, however, may
be raised as to whether a motion is specifically one when the same
thing changes from the same to the same, e.g. when one point changes
again and again from a particular place to a particular place: if
this motion is specifically one, circular motion will be the same
as rectilinear motion, and rolling the same as walking. But is not
this difficulty removed by the principle already laid down that if
that in which the motion takes place is specifically different (as
in the present instance the circular path is specifically different
from the straight) the motion itself is also different? We have explained,
then, what is meant by saying that motion is one generically or one
specifically. 

Motion is one in an unqualified sense when it is one essentially or
numerically: and the following distinctions will make clear what this
kind of motion is. There are three classes of things in connexion
with which we speak of motion, the 'that which', the 'that in which',
and the 'that during which'. I mean that there must he something that
is in motion, e.g. a man or gold, and it must be in motion in something,
e.g. a place or an affection, and during something, for all motion
takes place during a time. Of these three it is the thing in which
the motion takes place that makes it one generically or specifically,
it is the thing moved that makes the motion one in subject, and it
is the time that makes it consecutive: but it is the three together
that make it one without qualification: to effect this, that in which
the motion takes place (the species) must be one and incapable of
subdivision, that during which it takes place (the time) must be one
and unintermittent, and that which is in motion must be one-not in
an accidental sense (i.e. it must be one as the white that blackens
is one or Coriscus who walks is one, not in the accidental sense in
which Coriscus and white may be one), nor merely in virtue of community
of nature (for there might be a case of two men being restored to
health at the same time in the same way, e.g. from inflammation of
the eye, yet this motion is not really one, but only specifically
one). 

Suppose, however, that Socrates undergoes an alteration specifically
the same but at one time and again at another: in this case if it
is possible for that which ceased to be again to come into being and
remain numerically the same, then this motion too will be one: otherwise
it will be the same but not one. And akin to this difficulty there
is another; viz. is health one? and generally are the states and affections
in bodies severally one in essence although (as is clear) the things
that contain them are obviously in motion and in flux? Thus if a person's
health at daybreak and at the present moment is one and the same,
why should not this health be numerically one with that which he recovers
after an interval? The same argument applies in each case. There is,
however, we may answer, this difference: that if the states are two
then it follows simply from this fact that the activities must also
in point of number be two (for only that which is numerically one
can give rise to an activity that is numerically one), but if the
state is one, this is not in itself enough to make us regard the activity
also as one: for when a man ceases walking, the walking no longer
is, but it will again be if he begins to walk again. But, be this
as it may, if in the above instance the health is one and the same,
then it must be possible for that which is one and the same to come
to be and to cease to be many times. However, these difficulties lie
outside our present inquiry. 

Since every motion is continuous, a motion that is one in an unqualified
sense must (since every motion is divisible) be continuous, and a
continuous motion must be one. There will not be continuity between
any motion and any other indiscriminately any more than there is between
any two things chosen at random in any other sphere: there can be
continuity only when the extremities of the two things are one. Now
some things have no extremities at all: and the extremities of others
differ specifically although we give them the same name of 'end':
how should e.g. the 'end' of a line and the 'end' of walking touch
or come to be one? Motions that are not the same either specifically
or generically may, it is true, be consecutive (e.g. a man may run
and then at once fall ill of a fever), and again, in the torch-race
we have consecutive but not continuous locomotion: for according to
our definition there can be continuity only when the ends of the two
things are one. Hence motions may be consecutive or successive in
virtue of the time being continuous, but there can be continuity only
in virtue of the motions themselves being continuous, that is when
the end of each is one with the end of the other. Motion, therefore,
that is in an unqualified sense continuous and one must be specifically
the same, of one thing, and in one time. Unity is required in respect
of time in order that there may be no interval of immobility, for
where there is intermission of motion there must be rest, and a motion
that includes intervals of rest will be not one but many, so that
a motion that is interrupted by stationariness is not one or continuous,
and it is so interrupted if there is an interval of time. And though
of a motion that is not specifically one (even if the time is unintermittent)
the time is one, the motion is specifically different, and so cannot
really be one, for motion that is one must be specifically one, though
motion that is specifically one is not necessarily one in an unqualified
sense. We have now explained what we mean when we call a motion one
without qualification. 

Further, a motion is also said to be one generically, specifically,
or essentially when it is complete, just as in other cases completeness
and wholeness are characteristics of what is one: and sometimes a
motion even if incomplete is said to be one, provided only that it
is continuous. 

And besides the cases already mentioned there is another in which
a motion is said to be one, viz. when it is regular: for in a sense
a motion that is irregular is not regarded as one, that title belonging
rather to that which is regular, as a straight line is regular, the
irregular being as such divisible. But the difference would seem to
be one of degree. In every kind of motion we may have regularity or
irregularity: thus there may be regular alteration, and locomotion
in a regular path, e.g. in a circle or on a straight line, and it
is the same with regard to increase and decrease. The difference that
makes a motion irregular is sometimes to be found in its path: thus
a motion cannot be regular if its path is an irregular magnitude,
e.g. a broken line, a spiral, or any other magnitude that is not such
that any part of it taken at random fits on to any other that may
be chosen. Sometimes it is found neither in the place nor in the time
nor in the goal but in the manner of the motion: for in some cases
the motion is differentiated by quickness and slowness: thus if its
velocity is uniform a motion is regular, if not it is irregular. So
quickness and slowness are not species of motion nor do they constitute
specific differences of motion, because this distinction occurs in
connexion with all the distinct species of motion. The same is true
of heaviness and lightness when they refer to the same thing: e.g.
they do not specifically distinguish earth from itself or fire from
itself. Irregular motion, therefore, while in virtue of being continuous
it is one, is so in a lesser degree, as is the case with locomotion
in a broken line: and a lesser degree of something always means an
admixture of its contrary. And since every motion that is one can
be both regular and irregular, motions that are consecutive but not
specifically the same cannot be one and continuous: for how should
a motion composed of alteration and locomotion be regular? If a motion
is to be regular its parts ought to fit one another. 

Part 5

We have further to determine what motions are contrary to each other,
and to determine similarly how it is with rest. And we have first
to decide whether contrary motions are motions respectively from and
to the same thing, e.g. a motion from health and a motion to health
(where the opposition, it would seem, is of the same kind as that
between coming to be and ceasing to be); or motions respectively from
contraries, e.g. a motion from health and a motion from disease; or
motions respectively to contraries, e.g. a motion to health and a
motion to disease; or motions respectively from a contrary and to
the opposite contrary, e.g. a motion from health and a motion to disease;
or motions respectively from a contrary to the opposite contrary and
from the latter to the former, e.g. a motion from health to disease
and a motion from disease to health: for motions must be contrary
to one another in one or more of these ways, as there is no other
way in which they can be opposed. 

Now motions respectively from a contrary and to the opposite contrary,
e.g. a motion from health and a motion to disease, are not contrary
motions: for they are one and the same. (Yet their essence is not
the same, just as changing from health is different from changing
to disease.) Nor are motion respectively from a contrary and from
the opposite contrary contrary motions, for a motion from a contrary
is at the same time a motion to a contrary or to an intermediate (of
this, however, we shall speak later), but changing to a contrary rather
than changing from a contrary would seem to be the cause of the contrariety
of motions, the latter being the loss, the former the gain, of contrariness.
Moreover, each several motion takes its name rather from the goal
than from the starting-point of change, e.g. motion to health we call
convalescence, motion to disease sickening. Thus we are left with
motions respectively to contraries, and motions respectively to contraries
from the opposite contraries. Now it would seem that motions to contraries
are at the same time motions from contraries (though their essence
may not be the same; 'to health' is distinct, I mean, from 'from disease',
and 'from health' from 'to disease'). 

Since then change differs from motion (motion being change from a
particular subject to a particular subject), it follows that contrary
motions are motions respectively from a contrary to the opposite contrary
and from the latter to the former, e.g. a motion from health to disease
and a motion from disease to health. Moreover, the consideration of
particular examples will also show what kinds of processes are generally
recognized as contrary: thus falling ill is regarded as contrary to
recovering one's health, these processes having contrary goals, and
being taught as contrary to being led into error by another, it being
possible to acquire error, like knowledge, either by one's own agency
or by that of another. Similarly we have upward locomotion and downward
locomotion, which are contrary lengthwise, locomotion to the right
and locomotion to the left, which are contrary breadthwise, and forward
locomotion and backward locomotion, which too are contraries. On the
other hand, a process simply to a contrary, e.g. that denoted by the
expression 'becoming white', where no starting-point is specified,
is a change but not a motion. And in all cases of a thing that has
no contrary we have as contraries change from and change to the same
thing. Thus coming to be is contrary to ceasing to be, and losing
to gaining. But these are changes and not motions. And wherever a
pair of contraries admit of an intermediate, motions to that intermediate
must be held to be in a sense motions to one or other of the contraries:
for the intermediate serves as a contrary for the purposes of the
motion, in whichever direction the change may be, e.g. grey in a motion
from grey to white takes the place of black as starting-point, in
a motion from white to grey it takes the place of black as goal, and
in a motion from black to grey it takes the place of white as goal:
for the middle is opposed in a sense to either of the extremes, as
has been said above. Thus we see that two motions are contrary to
each other only when one is a motion from a contrary to the opposite
contrary and the other is a motion from the latter to the former.

Part 6

But since a motion appears to have contrary to it not only another
motion but also a state of rest, we must determine how this is so.
A motion has for its contrary in the strict sense of the term another
motion, but it also has for an opposite a state of rest (for rest
is the privation of motion and the privation of anything may be called
its contrary), and motion of one kind has for its opposite rest of
that kind, e.g. local motion has local rest. This statement, however,
needs further qualification: there remains the question, is the opposite
of remaining at a particular place motion from or motion to that place?
It is surely clear that since there are two subjects between which
motion takes place, motion from one of these (A) to its contrary (B)
has for its opposite remaining in A while the reverse motion has for
its opposite remaining in B. At the same time these two are also contrary
to each other: for it would be absurd to suppose that there are contrary
motions and not opposite states of rest. States of rest in contraries
are opposed. To take an example, a state of rest in health is (1)
contrary to a state of rest in disease, and (2) the motion to which
it is contrary is that from health to disease. For (2) it would be
absurd that its contrary motion should be that from disease to health,
since motion to that in which a thing is at rest is rather a coming
to rest, the coming to rest being found to come into being simultaneously
with the motion; and one of these two motions it must be. And (1)
rest in whiteness is of course not contrary to rest in health.

Of all things that have no contraries there are opposite changes (viz.
change from the thing and change to the thing, e.g. change from being
and change to being), but no motion. So, too, of such things there
is no remaining though there is absence of change. Should there be
a particular subject, absence of change in its being will be contrary
to absence of change in its not-being. And here a difficulty may be
raised: if not-being is not a particular something, what is it, it
may be asked, that is contrary to absence of change in a thing's being?
and is this absence of change a state of rest? If it is, then either
it is not true that every state of rest is contrary to a motion or
else coming to be and ceasing to be are motion. It is clear then that,
since we exclude these from among motions, we must not say that this
absence of change is a state of rest: we must say that it is similar
to a state of rest and call it absence of change. And it will have
for its contrary either nothing or absence of change in the thing's
not-being, or the ceasing to be of the thing: for such ceasing to
be is change from it and the thing's coming to be is change to it.

Again, a further difficulty may be raised. How is it, it may be asked,
that whereas in local change both remaining and moving may be natural
or unnatural, in the other changes this is not so? e.g. alteration
is not now natural and now unnatural, for convalescence is no more
natural or unnatural than falling ill, whitening no more natural or
unnatural than blackening; so, too, with increase and decrease: these
are not contrary to each other in the sense that either of them is
natural while the other is unnatural, nor is one increase contrary
to another in this sense; and the same account may be given of becoming
and perishing: it is not true that becoming is natural and perishing
unnatural (for growing old is natural), nor do we observe one becoming
to be natural and another unnatural. We answer that if what happens
under violence is unnatural, then violent perishing is unnatural and
as such contrary to natural perishing. Are there then also some becomings
that are violent and not the result of natural necessity, and are
therefore contrary to natural becomings, and violent increases and
decreases, e.g. the rapid growth to maturity of profligates and the
rapid ripening of seeds even when not packed close in the earth? And
how is it with alterations? Surely just the same: we may say that
some alterations are violent while others are natural, e.g. patients
alter naturally or unnaturally according as they throw off fevers
on the critical days or not. But, it may be objected, then we shall
have perishings contrary to one another, not to becoming. Certainly:
and why should not this in a sense be so? Thus it is so if one perishing
is pleasant and another painful: and so one perishing will be contrary
to another not in an unqualified sense, but in so far as one has this
quality and the other that. 

Now motions and states of rest universally exhibit contrariety in
the manner described above, e.g. upward motion and rest above are
respectively contrary to downward motion and rest below, these being
instances of local contrariety; and upward locomotion belongs naturally
to fire and downward to earth, i.e. the locomotions of the two are
contrary to each other. And again, fire moves up naturally and down
unnaturally: and its natural motion is certainly contrary to its unnatural
motion. Similarly with remaining: remaining above is contrary to motion
from above downwards, and to earth this remaining comes unnaturally,
this motion naturally. So the unnatural remaining of a thing is contrary
to its natural motion, just as we find a similar contrariety in the
motion of the same thing: one of its motions, the upward or the downward,
will be natural, the other unnatural. 

Here, however, the question arises, has every state of rest that is
not permanent a becoming, and is this becoming a coming to a standstill?
If so, there must be a becoming of that which is at rest unnaturally,
e.g. of earth at rest above: and therefore this earth during the time
that it was being carried violently upward was coming to a standstill.
But whereas the velocity of that which comes to a standstill seems
always to increase, the velocity of that which is carried violently
seems always to decrease: so it will he in a state of rest without
having become so. Moreover 'coming to a standstill' is generally recognized
to be identical or at least concomitant with the locomotion of a thing
to its proper place. 

There is also another difficulty involved in the view that remaining
in a particular place is contrary to motion from that place. For when
a thing is moving from or discarding something, it still appears to
have that which is being discarded, so that if a state of rest is
itself contrary to the motion from the state of rest to its contrary,
the contraries rest and motion will be simultaneously predicable of
the same thing. May we not say, however, that in so far as the thing
is still stationary it is in a state of rest in a qualified sense?
For, in fact, whenever a thing is in motion, part of it is at the
starting-point while part is at the goal to which it is changing:
and consequently a motion finds its true contrary rather in another
motion than in a state of rest. 

With regard to motion and rest, then, we have now explained in what
sense each of them is one and under what conditions they exhibit contrariety.

[With regard to coming to a standstill the question may be raised
whether there is an opposite state of rest to unnatural as well as
to natural motions. It would be absurd if this were not the case:
for a thing may remain still merely under violence: thus we shall
have a thing being in a non-permanent state of rest without having
become so. But it is clear that it must be the case: for just as there
is unnatural motion, so, too, a thing may be in an unnatural state
of rest. Further, some things have a natural and an unnatural motion,
e.g. fire has a natural upward motion and an unnatural downward motion:
is it, then, this unnatural downward motion or is it the natural downward
motion of earth that is contrary to the natural upward motion? Surely
it is clear that both are contrary to it though not in the same sense:
the natural motion of earth is contrary inasmuch as the motion of
fire is also natural, whereas the upward motion of fire as being natural
is contrary to the downward motion of fire as being unnatural. The
same is true of the corresponding cases of remaining. But there would
seem to be a sense in which a state of rest and a motion are opposites.]

----------------------------------------------------------------------

BOOK VI

Part 1 

Now if the terms 'continuous', 'in contact', and 'in succession'
are understood as defined above things being 'continuous' if their
extremities are one, 'in contact' if their extremities are together,
and 'in succession' if there is nothing of their own kind intermediate
between them-nothing that is continuous can be composed 'of indivisibles':
e.g. a line cannot be composed of points, the line being continuous
and the point indivisible. For the extremities of two points can neither
be one (since of an indivisible there can be no extremity as distinct
from some other part) nor together (since that which has no parts
can have no extremity, the extremity and the thing of which it is
the extremity being distinct). 

Moreover, if that which is continuous is composed of points, these
points must be either continuous or in contact with one another: and
the same reasoning applies in the case of all indivisibles. Now for
the reason given above they cannot be continuous: and one thing can
be in contact with another only if whole is in contact with whole
or part with part or part with whole. But since indivisibles have
no parts, they must be in contact with one another as whole with whole.
And if they are in contact with one another as whole with whole, they
will not be continuous: for that which is continuous has distinct
parts: and these parts into which it is divisible are different in
this way, i.e. spatially separate. 

Nor, again, can a point be in succession to a point or a moment to
a moment in such a way that length can be composed of points or time
of moments: for things are in succession if there is nothing of their
own kind intermediate between them, whereas that which is intermediate
between points is always a line and that which is intermediate between
moments is always a period of time. 

Again, if length and time could thus be composed of indivisibles,
they could be divided into indivisibles, since each is divisible into
the parts of which it is composed. But, as we saw, no continuous thing
is divisible into things without parts. Nor can there be anything
of any other kind intermediate between the parts or between the moments:
for if there could be any such thing it is clear that it must be either
indivisible or divisible, and if it is divisible, it must be divisible
either into indivisibles or into divisibles that are infinitely divisible,
in which case it is continuous. 

Moreover, it is plain that everything continuous is divisible into
divisibles that are infinitely divisible: for if it were divisible
into indivisibles, we should have an indivisible in contact with an
indivisible, since the extremities of things that are continuous with
one another are one and are in contact. 

The same reasoning applies equally to magnitude, to time, and to motion:
either all of these are composed of indivisibles and are divisible
into indivisibles, or none. This may be made clear as follows. If
a magnitude is composed of indivisibles, the motion over that magnitude
must be composed of corresponding indivisible motions: e.g. if the
magnitude ABG is composed of the indivisibles A, B, G, each corresponding
part of the motion DEZ of O over ABG is indivisible. Therefore, since
where there is motion there must be something that is in motion, and
where there is something in motion there must be motion, therefore
the being-moved will also be composed of indivisibles. So O traversed
A when its motion was D, B when its motion was E, and G similarly
when its motion was Z. Now a thing that is in motion from one place
to another cannot at the moment when it was in motion both be in motion
and at the same time have completed its motion at the place to which
it was in motion: e.g. if a man is walking to Thebes, he cannot be
walking to Thebes and at the same time have completed his walk to
Thebes: and, as we saw, O traverses a the partless section A in virtue
of the presence of the motion D. Consequently, if O actually passed
through A after being in process of passing through, the motion must
be divisible: for at the time when O was passing through, it neither
was at rest nor had completed its passage but was in an intermediate
state: while if it is passing through and has completed its passage
at the same moment, then that which is walking will at the moment
when it is walking have completed its walk and will be in the place
to which it is walking; that is to say, it will have completed its
motion at the place to which it is in motion. And if a thing is in
motion over the whole Kbg and its motion is the three D, E, and Z,
and if it is not in motion at all over the partless section A but
has completed its motion over it, then the motion will consist not
of motions but of starts, and will take place by a thing's having
completed a motion without being in motion: for on this assumption
it has completed its passage through A without passing through it.
So it will be possible for a thing to have completed a walk without
ever walking: for on this assumption it has completed a walk over
a particular distance without walking over that distance. Since, then,
everything must be either at rest or in motion, and O is therefore
at rest in each of the sections A, B, and G, it follows that a thing
can be continuously at rest and at the same time in motion: for, as
we saw, O is in motion over the whole ABG and at rest in any part
(and consequently in the whole) of it. Moreover, if the indivisibles
composing DEZ are motions, it would be possible for a thing in spite
of the presence in it of motion to be not in motion but at rest, while
if they are not motions, it would be possible for motion to be composed
of something other than motions. 

And if length and motion are thus indivisible, it is neither more
nor less necessary that time also be similarly indivisible, that is
to say be composed of indivisible moments: for if the whole distance
is divisible and an equal velocity will cause a thing to pass through
less of it in less time, the time must also be divisible, and conversely,
if the time in which a thing is carried over the section A is divisible,
this section A must also be divisible. 

Part 2

And since every magnitude is divisible into magnitudes-for we have
shown that it is impossible for anything continuous to be composed
of indivisible parts, and every magnitude is continuous-it necessarily
follows that the quicker of two things traverses a greater magnitude
in an equal time, an equal magnitude in less time, and a greater magnitude
in less time, in conformity with the definition sometimes given of
'the quicker'. Suppose that A is quicker than B. Now since of two
things that which changes sooner is quicker, in the time ZH, in which
A has changed from G to D, B will not yet have arrived at D but will
be short of it: so that in an equal time the quicker will pass over
a greater magnitude. More than this, it will pass over a greater magnitude
in less time: for in the time in which A has arrived at D, B being
the slower has arrived, let us say, at E. Then since A has occupied
the whole time ZH in arriving at D, will have arrived at O in less
time than this, say ZK. Now the magnitude GO that A has passed over
is greater than the magnitude GE, and the time ZK is less than the
whole time ZH: so that the quicker will pass over a greater magnitude
in less time. And from this it is also clear that the quicker will
pass over an equal magnitude in less time than the slower. For since
it passes over the greater magnitude in less time than the slower,
and (regarded by itself) passes over LM the greater in more time than
LX the lesser, the time PRh in which it passes over LM will be more
than the time PS, which it passes over LX: so that, the time PRh being
less than the time PCh in which the slower passes over LX, the time
PS will also be less than the time PX: for it is less than the time
PRh, and that which is less than something else that is less than
a thing is also itself less than that thing. Hence it follows that
the quicker will traverse an equal magnitude in less time than the
slower. Again, since the motion of anything must always occupy either
an equal time or less or more time in comparison with that of another
thing, and since, whereas a thing is slower if its motion occupies
more time and of equal velocity if its motion occupies an equal time,
the quicker is neither of equal velocity nor slower, it follows that
the motion of the quicker can occupy neither an equal time nor more
time. It can only be, then, that it occupies less time, and thus we
get the necessary consequence that the quicker will pass over an equal
magnitude (as well as a greater) in less time than the slower.

And since every motion is in time and a motion may occupy any time,
and the motion of everything that is in motion may be either quicker
or slower, both quicker motion and slower motion may occupy any time:
and this being so, it necessarily follows that time also is continuous.
By continuous I mean that which is divisible into divisibles that
are infinitely divisible: and if we take this as the definition of
continuous, it follows necessarily that time is continuous. For since
it has been shown that the quicker will pass over an equal magnitude
in less time than the slower, suppose that A is quicker and B slower,
and that the slower has traversed the magnitude GD in the time ZH.
Now it is clear that the quicker will traverse the same magnitude
in less time than this: let us say in the time ZO. Again, since the
quicker has passed over the whole D in the time ZO, the slower will
in the same time pass over GK, say, which is less than GD. And since
B, the slower, has passed over GK in the time ZO, the quicker will
pass over it in less time: so that the time ZO will again be divided.
And if this is divided the magnitude GK will also be divided just
as GD was: and again, if the magnitude is divided, the time will also
be divided. And we can carry on this process for ever, taking the
slower after the quicker and the quicker after the slower alternately,
and using what has been demonstrated at each stage as a new point
of departure: for the quicker will divide the time and the slower
will divide the length. If, then, this alternation always holds good,
and at every turn involves a division, it is evident that all time
must be continuous. And at the same time it is clear that all magnitude
is also continuous; for the divisions of which time and magnitude
respectively are susceptible are the same and equal. 

Moreover, the current popular arguments make it plain that, if time
is continuous, magnitude is continuous also, inasmuch as a thing asses
over half a given magnitude in half the time taken to cover the whole:
in fact without qualification it passes over a less magnitude in less
time; for the divisions of time and of magnitude will be the same.
And if either is infinite, so is the other, and the one is so in the
same way as the other; i.e. if time is infinite in respect of its
extremities, length is also infinite in respect of its extremities:
if time is infinite in respect of divisibility, length is also infinite
in respect of divisibility: and if time is infinite in both respects,
magnitude is also infinite in both respects. 

Hence Zeno's argument makes a false assumption in asserting that it
is impossible for a thing to pass over or severally to come in contact
with infinite things in a finite time. For there are two senses in
which length and time and generally anything continuous are called
'infinite': they are called so either in respect of divisibility or
in respect of their extremities. So while a thing in a finite time
cannot come in contact with things quantitatively infinite, it can
come in contact with things infinite in respect of divisibility: for
in this sense the time itself is also infinite: and so we find that
the time occupied by the passage over the infinite is not a finite
but an infinite time, and the contact with the infinites is made by
means of moments not finite but infinite in number. 

The passage over the infinite, then, cannot occupy a finite time,
and the passage over the finite cannot occupy an infinite time: if
the time is infinite the magnitude must be infinite also, and if the
magnitude is infinite, so also is the time. This may be shown as follows.
Let AB be a finite magnitude, and let us suppose that it is traversed
in infinite time G, and let a finite period GD of the time be taken.
Now in this period the thing in motion will pass over a certain segment
of the magnitude: let BE be the segment that it has thus passed over.
(This will be either an exact measure of AB or less or greater than
an exact measure: it makes no difference which it is.) Then, since
a magnitude equal to BE will always be passed over in an equal time,
and BE measures the whole magnitude, the whole time occupied in passing
over AB will be finite: for it will be divisible into periods equal
in number to the segments into which the magnitude is divisible. Moreover,
if it is the case that infinite time is not occupied in passing over
every magnitude, but it is possible to ass over some magnitude, say
BE, in a finite time, and if this BE measures the whole of which it
is a part, and if an equal magnitude is passed over in an equal time,
then it follows that the time like the magnitude is finite. That infinite
time will not be occupied in passing over BE is evident if the time
be taken as limited in one direction: for as the part will be passed
over in less time than the whole, the time occupied in traversing
this part must be finite, the limit in one direction being given.
The same reasoning will also show the falsity of the assumption that
infinite length can be traversed in a finite time. It is evident,
then, from what has been said that neither a line nor a surface nor
in fact anything continuous can be indivisible. 

This conclusion follows not only from the present argument but from
the consideration that the opposite assumption implies the divisibility
of the indivisible. For since the distinction of quicker and slower
may apply to motions occupying any period of time and in an equal
time the quicker passes over a greater length, it may happen that
it will pass over a length twice, or one and a half times, as great
as that passed over by the slower: for their respective velocities
may stand to one another in this proportion. Suppose, then, that the
quicker has in the same time been carried over a length one and a
half times as great as that traversed by the slower, and that the
respective magnitudes are divided, that of the quicker, the magnitude
ABGD, into three indivisibles, and that of the slower into the two
indivisibles EZ, ZH. Then the time may also be divided into three
indivisibles, for an equal magnitude will be passed over in an equal
time. Suppose then that it is thus divided into KL, Lm, MN. Again,
since in the same time the slower has been carried over Ez, ZH, the
time may also be similarly divided into two. Thus the indivisible
will be divisible, and that which has no parts will be passed over
not in an indivisible but in a greater time. It is evident, therefore,
that nothing continuous is without parts. 

Part 3

The present also is necessarily indivisible-the present, that is,
not in the sense in which the word is applied to one thing in virtue
of another, but in its proper and primary sense; in which sense it
is inherent in all time. For the present is something that is an extremity
of the past (no part of the future being on this side of it) and also
of the future (no part of the past being on the other side of it):
it is, as we have said, a limit of both. And if it is once shown that
it is essentially of this character and one and the same, it will
at once be evident also that it is indivisible. 

Now the present that is the extremity of both times must be one and
the same: for if each extremity were different, the one could not
be in succession to the other, because nothing continuous can be composed
of things having no parts: and if the one is apart from the other,
there will be time intermediate between them, because everything continuous
is such that there is something intermediate between its limits and
described by the same name as itself. But if the intermediate thing
is time, it will be divisible: for all time has been shown to be divisible.
Thus on this assumption the present is divisible. But if the present
is divisible, there will be part of the past in the future and part
of the future in the past: for past time will be marked off from future
time at the actual point of division. Also the present will be a present
not in the proper sense but in virtue of something else: for the division
which yields it will not be a division proper. Furthermore, there
will be a part of the present that is past and a part that is future,
and it will not always be the same part that is past or future: in
fact one and the same present will not be simultaneous: for the time
may be divided at many points. If, therefore, the present cannot possibly
have these characteristics, it follows that it must be the same present
that belongs to each of the two times. But if this is so it is evident
that the present is also indivisible: for if it is divisible it will
be involved in the same implications as before. It is clear, then,
from what has been said that time contains something indivisible,
and this is what we call a present. 

We will now show that nothing can be in motion in a present. For if
this is possible, there can be both quicker and slower motion in the
present. Suppose then that in the present N the quicker has traversed
the distance AB. That being so, the slower will in the same present
traverse a distance less than AB, say AG. But since the slower will
have occupied the whole present in traversing AG, the quicker will
occupy less than this in traversing it. Thus we shall have a division
of the present, whereas we found it to be indivisible. It is impossible,
therefore, for anything to be in motion in a present. 

Nor can anything be at rest in a present: for, as we were saying,
only can be at rest which is naturally designed to be in motion but
is not in motion when, where, or as it would naturally be so: since,
therefore, nothing is naturally designed to be in motion in a present,
it is clear that nothing can be at rest in a present either.

Moreover, inasmuch as it is the same present that belongs to both
the times, and it is possible for a thing to be in motion throughout
one time and to be at rest throughout the other, and that which is
in motion or at rest for the whole of a time will be in motion or
at rest as the case may be in any part of it in which it is naturally
designed to be in motion or at rest: this being so, the assumption
that there can be motion or rest in a present will carry with it the
implication that the same thing can at the same time be at rest and
in motion: for both the times have the same extremity, viz. the present.

Again, when we say that a thing is at rest, we imply that its condition
in whole and in part is at the time of speaking uniform with what
it was previously: but the present contains no 'previously': consequently,
there can be no rest in it. 

It follows then that the motion of that which is in motion and the
rest of that which is at rest must occupy time. 

Part 4

Further, everything that changes must be divisible. For since every
change is from something to something, and when a thing is at the
goal of its change it is no longer changing, and when both it itself
and all its parts are at the starting-point of its change it is not
changing (for that which is in whole and in part in an unvarying condition
is not in a state of change); it follows, therefore, that part of
that which is changing must be at the starting-point and part at the
goal: for as a whole it cannot be in both or in neither. (Here by
'goal of change' I mean that which comes first in the process of change:
e.g. in a process of change from white the goal in question will be
grey, not black: for it is not necessary that that that which is changing
should be at either of the extremes.) It is evident, therefore, that
everything that changes must be divisible. 

Now motion is divisible in two senses. In the first place it is divisible
in virtue of the time that it occupies. In the second place it is
divisible according to the motions of the several parts of that which
is in motion: e.g. if the whole AG is in motion, there will be a motion
of AB and a motion of BG. That being so, let DE be the motion of the
part AB and EZ the motion of the part BG. Then the whole Dz must be
the motion of AG: for DZ must constitute the motion of AG inasmuch
as DE and EZ severally constitute the motions of each of its parts.
But the motion of a thing can never be constituted by the motion of
something else: consequently the whole motion is the motion of the
whole magnitude. 

Again, since every motion is a motion of something, and the whole
motion DZ is not the motion of either of the parts (for each of the
parts DE, EZ is the motion of one of the parts AB, BG) or of anything
else (for, the whole motion being the motion of a whole, the parts
of the motion are the motions of the parts of that whole: and the
parts of DZ are the motions of AB, BG and of nothing else: for, as
we saw, a motion that is one cannot be the motion of more things than
one): since this is so, the whole motion will be the motion of the
magnitude ABG. 

Again, if there is a motion of the whole other than DZ, say the the
of each of the arts may be subtracted from it: and these motions will
be equal to DE, EZ respectively: for the motion of that which is one
must be one. So if the whole motion OI may be divided into the motions
of the parts, OI will be equal to DZ: if on the other hand there is
any remainder, say KI, this will be a motion of nothing: for it can
be the motion neither of the whole nor of the parts (as the motion
of that which is one must be one) nor of anything else: for a motion
that is continuous must be the motion of things that are continuous.
And the same result follows if the division of OI reveals a surplus
on the side of the motions of the parts. Consequently, if this is
impossible, the whole motion must be the same as and equal to DZ.

This then is what is meant by the division of motion according to
the motions of the parts: and it must be applicable to everything
that is divisible into parts. 

Motion is also susceptible of another kind of division, that according
to time. For since all motion is in time and all time is divisible,
and in less time the motion is less, it follows that every motion
must be divisible according to time. And since everything that is
in motion is in motion in a certain sphere and for a certain time
and has a motion belonging to it, it follows that the time, the motion,
the being-in-motion, the thing that is in motion, and the sphere of
the motion must all be susceptible of the same divisions (though spheres
of motion are not all divisible in a like manner: thus quantity is
essentially, quality accidentally divisible). For suppose that A is
the time occupied by the motion B. Then if all the time has been occupied
by the whole motion, it will take less of the motion to occupy half
the time, less again to occupy a further subdivision of the time,
and so on to infinity. Again, the time will be divisible similarly
to the motion: for if the whole motion occupies all the time half
the motion will occupy half the time, and less of the motion again
will occupy less of the time. 

In the same way the being-in-motion will also be divisible. For let
G be the whole being-in-motion. Then the being-in-motion that corresponds
to half the motion will be less than the whole being-in-motion, that
which corresponds to a quarter of the motion will be less again, and
so on to infinity. Moreover by setting out successively the being-in-motion
corresponding to each of the two motions DG (say) and GE, we may argue
that the whole being-in-motion will correspond to the whole motion
(for if it were some other being-in-motion that corresponded to the
whole motion, there would be more than one being-in motion corresponding
to the same motion), the argument being the same as that whereby we
showed that the motion of a thing is divisible into the motions of
the parts of the thing: for if we take separately the being-in motion
corresponding to each of the two motions, we shall see that the whole
being-in motion is continuous. 

The same reasoning will show the divisibility of the length, and in
fact of everything that forms a sphere of change (though some of these
are only accidentally divisible because that which changes is so):
for the division of one term will involve the division of all. So,
too, in the matter of their being finite or infinite, they will all
alike be either the one or the other. And we now see that in most
cases the fact that all the terms are divisible or infinite is a direct
consequence of the fact that the thing that changes is divisible or
infinite: for the attributes 'divisible' and 'infinite' belong in
the first instance to the thing that changes. That divisibility does
so we have already shown: that infinity does so will be made clear
in what follows? 

Part 5

Since everything that changes changes from something to something,
that which has changed must at the moment when it has first changed
be in that to which it has changed. For that which changes retires
from or leaves that from which it changes: and leaving, if not identical
with changing, is at any rate a consequence of it. And if leaving
is a consequence of changing, having left is a consequence of having
changed: for there is a like relation between the two in each case.

One kind of change, then, being change in a relation of contradiction,
where a thing has changed from not-being to being it has left not-being.
Therefore it will be in being: for everything must either be or not
be. It is evident, then, that in contradictory change that which has
changed must be in that to which it has changed. And if this is true
in this kind of change, it will be true in all other kinds as well:
for in this matter what holds good in the case of one will hold good
likewise in the case of the rest. 

Moreover, if we take each kind of change separately, the truth of
our conclusion will be equally evident, on the ground that that that
which has changed must be somewhere or in something. For, since it
has left that from which it has changed and must be somewhere, it
must be either in that to which it has changed or in something else.
If, then, that which has changed to B is in something other than B,
say G, it must again be changing from G to B: for it cannot be assumed
that there is no interval between G and B, since change is continuous.
Thus we have the result that the thing that has changed, at the moment
when it has changed, is changing to that to which it has changed,
which is impossible: that which has changed, therefore, must be in
that to which it has changed. So it is evident likewise that that
that which has come to be, at the moment when it has come to be, will
be, and that which has ceased to be will not-be: for what we have
said applies universally to every kind of change, and its truth is
most obvious in the case of contradictory change. It is clear, then,
that that which has changed, at the moment when it has first changed,
is in that to which it has changed. 

We will now show that the 'primary when' in which that which has changed
effected the completion of its change must be indivisible, where by
'primary' I mean possessing the characteristics in question of itself
and not in virtue of the possession of them by something else belonging
to it. For let AG be divisible, and let it be divided at B. If then
the completion of change has been effected in AB or again in BG, AG
cannot be the primary thing in which the completion of change has
been effected. If, on the other hand, it has been changing in both
AB and BG (for it must either have changed or be changing in each
of them), it must have been changing in the whole AG: but our assumption
was that AG contains only the completion of the change. It is equally
impossible to suppose that one part of AG contains the process and
the other the completion of the change: for then we shall have something
prior to what is primary. So that in which the completion of change
has been effected must be indivisible. It is also evident, therefore,
that that that in which that which has ceased to be has ceased to
be and that in which that which has come to be has come to be are
indivisible. 

But there are two senses of the expression 'the primary when in which
something has changed'. On the one hand it may mean the primary when
containing the completion of the process of change- the moment when
it is correct to say 'it has changed': on the other hand it may mean
the primary when containing the beginning of the process of change.
Now the primary when that has reference to the end of the change is
something really existent: for a change may really be completed, and
there is such a thing as an end of change, which we have in fact shown
to be indivisible because it is a limit. But that which has reference
to the beginning is not existent at all: for there is no such thing
as a beginning of a process of change, and the time occupied by the
change does not contain any primary when in which the change began.
For suppose that AD is such a primary when. Then it cannot be indivisible:
for, if it were, the moment immediately preceding the change and the
moment in which the change begins would be consecutive (and moments
cannot be consecutive). Again, if the changing thing is at rest in
the whole preceding time GA (for we may suppose that it is at rest),
it is at rest in A also: so if AD is without parts, it will simultaneously
be at rest and have changed: for it is at rest in A and has changed
in D. Since then AD is not without parts, it must be divisible, and
the changing thing must have changed in every part of it (for if it
has changed in neither of the two parts into which AD is divided,
it has not changed in the whole either: if, on the other hand, it
is in process of change in both parts, it is likewise in process of
change in the whole: and if, again, it has changed in one of the two
parts, the whole is not the primary when in which it has changed:
it must therefore have changed in every part). It is evident, then,
that with reference to the beginning of change there is no primary
when in which change has been effected: for the divisions are infinite.

So, too, of that which has changed there is no primary part that has
changed. For suppose that of AE the primary part that has changed
is Az (everything that changes having been shown to be divisible):
and let OI be the time in which DZ has changed. If, then, in the whole
time DZ has changed, in half the time there will be a part that has
changed, less than and therefore prior to DZ: and again there will
be another part prior to this, and yet another, and so on to infinity.
Thus of that which changes there cannot be any primary part that has
changed. It is evident, then, from what has been said, that neither
of that which changes nor of the time in which it changes is there
any primary part. 

With regard, however, to the actual subject of change-that is to say
that in respect of which a thing changes-there is a difference to
be observed. For in a process of change we may distinguish three terms-that
which changes, that in which it changes, and the actual subject of
change, e.g. the man, the time, and the fair complexion. Of these
the man and the time are divisible: but with the fair complexion it
is otherwise (though they are all divisible accidentally, for that
in which the fair complexion or any other quality is an accident is
divisible). For of actual subjects of change it will be seen that
those which are classed as essentially, not accidentally, divisible
have no primary part. Take the case of magnitudes: let AB be a magnitude,
and suppose that it has moved from B to a primary 'where' G. Then
if BG is taken to be indivisible, two things without parts will have
to be contiguous (which is impossible): if on the other hand it is
taken to be divisible, there will be something prior to G to which
the magnitude has changed, and something else again prior to that,
and so on to infinity, because the process of division may be continued
without end. Thus there can be no primary 'where' to which a thing
has changed. And if we take the case of quantitative change, we shall
get a like result, for here too the change is in something continuous.
It is evident, then, that only in qualitative motion can there be
anything essentially indivisible. 

Part 6

Now everything that changes changes time, and that in two senses:
for the time in which a thing is said to change may be the primary
time, or on the other hand it may have an extended reference, as e.g.
when we say that a thing changes in a particular year because it changes
in a particular day. That being so, that which changes must be changing
in any part of the primary time in which it changes. This is clear
from our definition of 'primary', in which the word is said to express
just this: it may also, however, be made evident by the following
argument. Let ChRh be the primary time in which that which is in motion
is in motion: and (as all time is divisible) let it be divided at
K. Now in the time ChK it either is in motion or is not in motion,
and the same is likewise true of the time KRh. Then if it is in motion
in neither of the two parts, it will be at rest in the whole: for
it is impossible that it should be in motion in a time in no part
of which it is in motion. If on the other hand it is in motion in
only one of the two parts of the time, ChRh cannot be the primary
time in which it is in motion: for its motion will have reference
to a time other than ChRh. It must, then, have been in motion in any
part of ChRh. 

And now that this has been proved, it is evident that everything that
is in motion must have been in motion before. For if that which is
in motion has traversed the distance KL in the primary time ChRh,
in half the time a thing that is in motion with equal velocity and
began its motion at the same time will have traversed half the distance.
But if this second thing whose velocity is equal has traversed a certain
distance in a certain time, the original thing that is in motion must
have traversed the same distance in the same time. Hence that which
is in motion must have been in motion before. 

Again, if by taking the extreme moment of the time-for it is the moment
that defines the time, and time is that which is intermediate between
moments-we are enabled to say that motion has taken place in the whole
time ChRh or in fact in any period of it, motion may likewise be said
to have taken place in every other such period. But half the time
finds an extreme in the point of division. Therefore motion will have
taken place in half the time and in fact in any part of it: for as
soon as any division is made there is always a time defined by moments.
If, then, all time is divisible, and that which is intermediate between
moments is time, everything that is changing must have completed an
infinite number of changes. 

Again, since a thing that changes continuously and has not perished
or ceased from its change must either be changing or have changed
in any part of the time of its change, and since it cannot be changing
in a moment, it follows that it must have changed at every moment
in the time: consequently, since the moments are infinite in number,
everything that is changing must have completed an infinite number
of changes. 

And not only must that which is changing have changed, but that which
has changed must also previously have been changing, since everything
that has changed from something to something has changed in a period
of time. For suppose that a thing has changed from A to B in a moment.
Now the moment in which it has changed cannot be the same as that
in which it is at A (since in that case it would be in A and B at
once): for we have shown above that that that which has changed, when
it has changed, is not in that from which it has changed. If, on the
other hand, it is a different moment, there will be a period of time
intermediate between the two: for, as we saw, moments are not consecutive.
Since, then, it has changed in a period of time, and all time is divisible,
in half the time it will have completed another change, in a quarter
another, and so on to infinity: consequently when it has changed,
it must have previously been changing. 

Moreover, the truth of what has been said is more evident in the case
of magnitude, because the magnitude over which what is changing changes
is continuous. For suppose that a thing has changed from G to D. Then
if GD is indivisible, two things without parts will be consecutive.
But since this is impossible, that which is intermediate between them
must be a magnitude and divisible into an infinite number of segments:
consequently, before the change is completed, the thing changes to
those segments. Everything that has changed, therefore, must previously
have been changing: for the same proof also holds good of change with
respect to what is not continuous, changes, that is to say, between
contraries and between contradictories. In such cases we have only
to take the time in which a thing has changed and again apply the
same reasoning. So that which has changed must have been changing
and that which is changing must have changed, and a process of change
is preceded by a completion of change and a completion by a process:
and we can never take any stage and say that it is absolutely the
first. The reason of this is that no two things without parts can
be contiguous, and therefore in change the process of division is
infinite, just as lines may be infinitely divided so that one part
is continually increasing and the other continually decreasing.

So it is evident also that that that which has become must previously
have been in process of becoming, and that which is in process of
becoming must previously have become, everything (that is) that is
divisible and continuous: though it is not always the actual thing
that is in process of becoming of which this is true: sometimes it
is something else, that is to say, some part of the thing in question,
e.g. the foundation-stone of a house. So, too, in the case of that
which is perishing and that which has perished: for that which becomes
and that which perishes must contain an element of infiniteness as
an immediate consequence of the fact that they are continuous things:
and so a thing cannot be in process of becoming without having become
or have become without having been in process of becoming. So, too,
in the case of perishing and having perished: perishing must be preceded
by having perished, and having perished must be preceded by perishing.
It is evident, then, that that which has become must previously have
been in process of becoming, and that which is in process of becoming
must previously have become: for all magnitudes and all periods of
time are infinitely divisible. 

Consequently no absolutely first stage of change can be represented
by any particular part of space or time which the changing thing may
occupy. 

Part 7

Now since the motion of everything that is in motion occupies a period
of time, and a greater magnitude is traversed in a longer time, it
is impossible that a thing should undergo a finite motion in an infinite
time, if this is understood to mean not that the same motion or a
part of it is continually repeated, but that the whole infinite time
is occupied by the whole finite motion. In all cases where a thing
is in motion with uniform velocity it is clear that the finite magnitude
is traversed in a finite time. For if we take a part of the motion
which shall be a measure of the whole, the whole motion is completed
in as many equal periods of the time as there are parts of the motion.
Consequently, since these parts are finite, both in size individually
and in number collectively, the whole time must also be finite: for
it will be a multiple of the portion, equal to the time occupied in
completing the aforesaid part multiplied by the number of the parts.

But it makes no difference even if the velocity is not uniform. For
let us suppose that the line AB represents a finite stretch over which
a thing has been moved in the given time, and let GD be the infinite
time. Now if one part of the stretch must have been traversed before
another part (this is clear, that in the earlier and in the later
part of the time a different part of the stretch has been traversed:
for as the time lengthens a different part of the motion will always
be completed in it, whether the thing in motion changes with uniform
velocity or not: and whether the rate of motion increases or diminishes
or remains stationary this is none the less so), let us then take
AE a part of the whole stretch of motion AB which shall be a measure
of AB. Now this part of the motion occupies a certain period of the
infinite time: it cannot itself occupy an infinite time, for we are
assuming that that is occupied by the whole AB. And if again I take
another part equal to AE, that also must occupy a finite time in consequence
of the same assumption. And if I go on taking parts in this way, on
the one hand there is no part which will be a measure of the infinite
time (for the infinite cannot be composed of finite parts whether
equal or unequal, because there must be some unity which will be a
measure of things finite in multitude or in magnitude, which, whether
they are equal or unequal, are none the less limited in magnitude);
while on the other hand the finite stretch of motion AB is a certain
multiple of AE: consequently the motion AB must be accomplished in
a finite time. Moreover it is the same with coming to rest as with
motion. And so it is impossible for one and the same thing to be infinitely
in process of becoming or of perishing. The reasoning he will prove
that in a finite time there cannot be an infinite extent of motion
or of coming to rest, whether the motion is regular or irregular.
For if we take a part which shall be a measure of the whole time,
in this part a certain fraction, not the whole, of the magnitude will
be traversed, because we assume that the traversing of the whole occupies
all the time. Again, in another equal part of the time another part
of the magnitude will be traversed: and similarly in each part of
the time that we take, whether equal or unequal to the part originally
taken. It makes no difference whether the parts are equal or not,
if only each is finite: for it is clear that while the time is exhausted
by the subtraction of its parts, the infinite magnitude will not be
thus exhausted, since the process of subtraction is finite both in
respect of the quantity subtracted and of the number of times a subtraction
is made. Consequently the infinite magnitude will not be traversed
in finite time: and it makes no difference whether the magnitude is
infinite in only one direction or in both: for the same reasoning
will hold good. 

This having been proved, it is evident that neither can a finite magnitude
traverse an infinite magnitude in a finite time, the reason being
the same as that given above: in part of the time it will traverse
a finite magnitude and in each several part likewise, so that in the
whole time it will traverse a finite magnitude. 

And since a finite magnitude will not traverse an infinite in a finite
time, it is clear that neither will an infinite traverse a finite
in a finite time. For if the infinite could traverse the finite, the
finite could traverse the infinite; for it makes no difference which
of the two is the thing in motion; either case involves the traversing
of the infinite by the finite. For when the infinite magnitude A is
in motion a part of it, say GD, will occupy the finite and then another,
and then another, and so on to infinity. Thus the two results will
coincide: the infinite will have completed a motion over the finite
and the finite will have traversed the infinite: for it would seem
to be impossible for the motion of the infinite over the finite to
occur in any way other than by the finite traversing the infinite
either by locomotion over it or by measuring it. Therefore, since
this is impossible, the infinite cannot traverse the finite.

Nor again will the infinite traverse the infinite in a finite time.
Otherwise it would also traverse the finite, for the infinite includes
the finite. We can further prove this in the same way by taking the
time as our starting-point. 

Since, then, it is established that in a finite time neither will
the finite traverse the infinite, nor the infinite the finite, nor
the infinite the infinite, it is evident also that in a finite time
there cannot be infinite motion: for what difference does it make
whether we take the motion or the magnitude to be infinite? If either
of the two is infinite, the other must be so likewise: for all locomotion
is in space. 

Part 8

Since everything to which motion or rest is natural is in motion or
at rest in the natural time, place, and manner, that which is coming
to a stand, when it is coming to a stand, must be in motion: for if
it is not in motion it must be at rest: but that which is at rest
cannot be coming to rest. From this it evidently follows that coming
to a stand must occupy a period of time: for the motion of that which
is in motion occupies a period of time, and that which is coming to
a stand has been shown to be in motion: consequently coming to a stand
must occupy a period of time. 

Again, since the terms 'quicker' and 'slower' are used only of that
which occupies a period of time, and the process of coming to a stand
may be quicker or slower, the same conclusion follows. 

And that which is coming to a stand must be coming to a stand in any
part of the primary time in which it is coming to a stand. For if
it is coming to a stand in neither of two parts into which the time
may be divided, it cannot be coming to a stand in the whole time,
with the result that that that which is coming to a stand will not
be coming to a stand. If on the other hand it is coming to a stand
in only one of the two parts of the time, the whole cannot be the
primary time in which it is coming to a stand: for it is coming to
a stand in the whole time not primarily but in virtue of something
distinct from itself, the argument being the same as that which we
used above about things in motion. 

And just as there is no primary time in which that which is in motion
is in motion, so too there is no primary time in which that which
is coming to a stand is coming to a stand, there being no primary
stage either of being in motion or of coming to a stand. For let AB
be the primary time in which a thing is coming to a stand. Now AB
cannot be without parts: for there cannot be motion in that which
is without parts, because the moving thing would necessarily have
been already moved for part of the time of its movement: and that
which is coming to a stand has been shown to be in motion. But since
Ab is therefore divisible, the thing is coming to a stand in every
one of the parts of AB: for we have shown above that it is coming
to a stand in every one of the parts in which it is primarily coming
to a stand. Since then, that in which primarily a thing is coming
to a stand must be a period of time and not something indivisible,
and since all time is infinitely divisible, there cannot be anything
in which primarily it is coming to a stand. 

Nor again can there be a primary time at which the being at rest of
that which is at rest occurred: for it cannot have occurred in that
which has no parts, because there cannot be motion in that which is
indivisible, and that in which rest takes place is the same as that
in which motion takes place: for we defined a state of rest to be
the state of a thing to which motion is natural but which is not in
motion when (that is to say in that in which) motion would be natural
to it. Again, our use of the phrase 'being at rest' also implies that
the previous state of a thing is still unaltered, not one point only
but two at least being thus needed to determine its presence: consequently
that in which a thing is at rest cannot be without parts. Since, then
it is divisible, it must be a period of time, and the thing must be
at rest in every one of its parts, as may be shown by the same method
as that used above in similar demonstrations. 

So there can be no primary part of the time: and the reason is that
rest and motion are always in a period of time, and a period of time
has no primary part any more than a magnitude or in fact anything
continuous: for everything continuous is divisible into an infinite
number of parts. 

And since everything that is in motion is in motion in a period of
time and changes from something to something, when its motion is comprised
within a particular period of time essentially-that is to say when
it fills the whole and not merely a part of the time in question-it
is impossible that in that time that which is in motion should be
over against some particular thing primarily. For if a thing-itself
and each of its parts-occupies the same space for a definite period
of time, it is at rest: for it is in just these circumstances that
we use the term 'being at rest'-when at one moment after another it
can be said with truth that a thing, itself and its parts, occupies
the same space. So if this is being at rest it is impossible for that
which is changing to be as a whole, at the time when it is primarily
changing, over against any particular thing (for the whole period
of time is divisible), so that in one part of it after another it
will be true to say that the thing, itself and its parts, occupies
the same space. If this is not so and the aforesaid proposition is
true only at a single moment, then the thing will be over against
a particular thing not for any period of time but only at a moment
that limits the time. It is true that at any moment it is always over
against something stationary: but it is not at rest: for at a moment
it is not possible for anything to be either in motion or at rest.
So while it is true to say that that which is in motion is at a moment
not in motion and is opposite some particular thing, it cannot in
a period of time be over against that which is at rest: for that would
involve the conclusion that that which is in locomotion is at rest.

Part 9

Zeno's reasoning, however, is fallacious, when he says that if everything
when it occupies an equal space is at rest, and if that which is in
locomotion is always occupying such a space at any moment, the flying
arrow is therefore motionless. This is false, for time is not composed
of indivisible moments any more than any other magnitude is composed
of indivisibles. 

Zeno's arguments about motion, which cause so much disquietude to
those who try to solve the problems that they present, are four in
number. The first asserts the non-existence of motion on the ground
that that which is in locomotion must arrive at the half-way stage
before it arrives at the goal. This we have discussed above.

The second is the so-called 'Achilles', and it amounts to this, that
in a race the quickest runner can never overtake the slowest, since
the pursuer must first reach the point whence the pursued started,
so that the slower must always hold a lead. This argument is the same
in principle as that which depends on bisection, though it differs
from it in that the spaces with which we successively have to deal
are not divided into halves. The result of the argument is that the
slower is not overtaken: but it proceeds along the same lines as the
bisection-argument (for in both a division of the space in a certain
way leads to the result that the goal is not reached, though the 'Achilles'
goes further in that it affirms that even the quickest runner in legendary
tradition must fail in his pursuit of the slowest), so that the solution
must be the same. And the axiom that that which holds a lead is never
overtaken is false: it is not overtaken, it is true, while it holds
a lead: but it is overtaken nevertheless if it is granted that it
traverses the finite distance prescribed. These then are two of his
arguments. 

The third is that already given above, to the effect that the flying
arrow is at rest, which result follows from the assumption that time
is composed of moments: if this assumption is not granted, the conclusion
will not follow. 

The fourth argument is that concerning the two rows of bodies, each
row being composed of an equal number of bodies of equal size, passing
each other on a race-course as they proceed with equal velocity in
opposite directions, the one row originally occupying the space between
the goal and the middle point of the course and the other that between
the middle point and the starting-post. This, he thinks, involves
the conclusion that half a given time is equal to double that time.
The fallacy of the reasoning lies in the assumption that a body occupies
an equal time in passing with equal velocity a body that is in motion
and a body of equal size that is at rest; which is false. For instance
(so runs the argument), let A, A...be the stationary bodies of equal
size, B, B...the bodies, equal in number and in size to A, A...,originally
occupying the half of the course from the starting-post to the middle
of the A's, and G, G...those originally occupying the other half from
the goal to the middle of the A's, equal in number, size, and velocity
to B, B....Then three consequences follow: 

First, as the B's and the G's pass one another, the first B reaches
the last G at the same moment as the first G reaches the last B. Secondly
at this moment the first G has passed all the A's, whereas the first
B has passed only half the A's, and has consequently occupied only
half the time occupied by the first G, since each of the two occupies
an equal time in passing each A. Thirdly, at the same moment all the
B's have passed all the G's: for the first G and the first B will
simultaneously reach the opposite ends of the course, since (so says
Zeno) the time occupied by the first G in passing each of the B's
is equal to that occupied by it in passing each of the A's, because
an equal time is occupied by both the first B and the first G in passing
all the A's. This is the argument, but it presupposed the aforesaid
fallacious assumption. 

Nor in reference to contradictory change shall we find anything unanswerable
in the argument that if a thing is changing from not-white, say, to
white, and is in neither condition, then it will be neither white
nor not-white: for the fact that it is not wholly in either condition
will not preclude us from calling it white or not-white. We call a
thing white or not-white not necessarily because it is be one or the
other, but cause most of its parts or the most essential parts of
it are so: not being in a certain condition is different from not
being wholly in that condition. So, too, in the case of being and
not-being and all other conditions which stand in a contradictory
relation: while the changing thing must of necessity be in one of
the two opposites, it is never wholly in either. 

Again, in the case of circles and spheres and everything whose motion
is confined within the space that it occupies, it is not true to say
the motion can be nothing but rest, on the ground that such things
in motion, themselves and their parts, will occupy the same position
for a period of time, and that therefore they will be at once at rest
and in motion. For in the first place the parts do not occupy the
same position for any period of time: and in the second place the
whole also is always changing to a different position: for if we take
the orbit as described from a point A on a circumference, it will
not be the same as the orbit as described from B or G or any other
point on the same circumference except in an accidental sense, the
sense that is to say in which a musical man is the same as a man.
Thus one orbit is always changing into another, and the thing will
never be at rest. And it is the same with the sphere and everything
else whose motion is confined within the space that it occupies.

Part 10

Our next point is that that which is without parts cannot be in motion
except accidentally: i.e. it can be in motion only in so far as the
body or the magnitude is in motion and the partless is in motion by
inclusion therein, just as that which is in a boat may be in motion
in consequence of the locomotion of the boat, or a part may be in
motion in virtue of the motion of the whole. (It must be remembered,
however, that by 'that which is without parts' I mean that which is
quantitatively indivisible (and that the case of the motion of a part
is not exactly parallel): for parts have motions belonging essentially
and severally to themselves distinct from the motion of the whole.
The distinction may be seen most clearly in the case of a revolving
sphere, in which the velocities of the parts near the centre and of
those on the surface are different from one another and from that
of the whole; this implies that there is not one motion but many).
As we have said, then, that which is without parts can be in motion
in the sense in which a man sitting in a boat is in motion when the
boat is travelling, but it cannot be in motion of itself. For suppose
that it is changing from AB to BG-either from one magnitude to another,
or from one form to another, or from some state to its contradictory-and
let D be the primary time in which it undergoes the change. Then in
the time in which it is changing it must be either in AB or in BG
or partly in one and partly in the other: for this, as we saw, is
true of everything that is changing. Now it cannot be partly in each
of the two: for then it would be divisible into parts. Nor again can
it be in BG: for then it will have completed the change, whereas the
assumption is that the change is in process. It remains, then, that
in the time in which it is changing, it is in Ab. That being so, it
will be at rest: for, as we saw, to be in the same condition for a
period of time is to be at rest. So it is not possible for that which
has no parts to be in motion or to change in any way: for only one
condition could have made it possible for it to have motion, viz.
that time should be composed of moments, in which case at any moment
it would have completed a motion or a change, so that it would never
be in motion, but would always have been in motion. But this we have
already shown above to be impossible: time is not composed of moments,
just as a line is not composed of points, and motion is not composed
of starts: for this theory simply makes motion consist of indivisibles
in exactly the same way as time is made to consist of moments or a
length of points. 

Again, it may be shown in the following way that there can be no motion
of a point or of any other indivisible. That which is in motion can
never traverse a space greater than itself without first traversing
a space equal to or less than itself. That being so, it is evident
that the point also must first traverse a space equal to or less than
itself. But since it is indivisible, there can be no space less than
itself for it to traverse first: so it will have to traverse a distance
equal to itself. Thus the line will be composed of points, for the
point, as it continually traverses a distance equal to itself, will
be a measure of the whole line. But since this is impossible, it is
likewise impossible for the indivisible to be in motion.

Again, since motion is always in a period of time and never in a moment,
and all time is divisible, for everything that is in motion there
must be a time less than that in which it traverses a distance as
great as itself. For that in which it is in motion will be a time,
because all motion is in a period of time; and all time has been shown
above to be divisible. Therefore, if a point is in motion, there must
be a time less than that in which it has itself traversed any distance.
But this is impossible, for in less time it must traverse less distance,
and thus the indivisible will be divisible into something less than
itself, just as the time is so divisible: the fact being that the
only condition under which that which is without parts and indivisible
could be in motion would have been the possibility of the infinitely
small being in motion in a moment: for in the two questions-that of
motion in a moment and that of motion of something indivisible-the
same principle is involved. 

Our next point is that no process of change is infinite: for every
change, whether between contradictories or between contraries, is
a change from something to something. Thus in contradictory changes
the positive or the negative, as the case may be, is the limit, e.g.
being is the limit of coming to be and not-being is the limit of ceasing
to be: and in contrary changes the particular contraries are the limits,
since these are the extreme points of any such process of change,
and consequently of every process of alteration: for alteration is
always dependent upon some contraries. Similarly contraries are the
extreme points of processes of increase and decrease: the limit of
increase is to be found in the complete magnitude proper to the peculiar
nature of the thing that is increasing, while the limit of decrease
is the complete loss of such magnitude. Locomotion, it is true, we
cannot show to be finite in this way, since it is not always between
contraries. But since that which cannot be cut (in the sense that
it is inconceivable that it should be cut, the term 'cannot' being
used in several senses)-since it is inconceivable that that which
in this sense cannot be cut should be in process of being cut, and
generally that that which cannot come to be should be in process of
coming to be, it follows that it is inconceivable that that which
cannot complete a change should be in process of changing to that
to which it cannot complete a change. If, then, it is to be assumed
that that which is in locomotion is in process of changing, it must
be capable of completing the change. Consequently its motion is not
infinite, and it will not be in locomotion over an infinite distance,
for it cannot traverse such a distance. 

It is evident, then, that a process of change cannot be infinite in
the sense that it is not defined by limits. But it remains to be considered
whether it is possible in the sense that one and the same process
of change may be infinite in respect of the time which it occupies.
If it is not one process, it would seem that there is nothing to prevent
its being infinite in this sense; e.g. if a process of locomotion
be succeeded by a process of alteration and that by a process of increase
and that again by a process of coming to be: in this way there may
be motion for ever so far as the time is concerned, but it will not
be one motion, because all these motions do not compose one. If it
is to be one process, no motion can be infinite in respect of the
time that it occupies, with the single exception of rotatory locomotion.

----------------------------------------------------------------------

BOOK VII

Part 1 

Everything that is in motion must be moved by something. For if it
has not the source of its motion in itself it is evident that it is
moved by something other than itself, for there must be something
else that moves it. If on the other hand it has the source of its
motion in itself, let AB be taken to represent that which is in motion
essentially of itself and not in virtue of the fact that something
belonging to it is in motion. Now in the first place to assume that
Ab, because it is in motion as a whole and is not moved by anything
external to itself, is therefore moved by itself-this is just as if,
supposing that KL is moving LM and is also itself in motion, we were
to deny that KM is moved by anything on the ground that it is not
evident which is the part that is moving it and which the part that
is moved. In the second place that which is in motion without being
moved by anything does not necessarily cease from its motion because
something else is at rest, but a thing must be moved by something
if the fact of something else having ceased from its motion causes
it to be at rest. Thus, if this is accepted, everything that is in
motion must be moved by something. For AB, which has been taken to
represent that which is in motion, must be divisible since everything
that is in motion is divisible. Let it be divided, then, at G. Now
if GB is not in motion, then AB will not be in motion: for if it is,
it is clear that AG would be in motion while BG is at rest, and thus
AB cannot be in motion essentially and primarily. But ex hypothesi
AB is in motion essentially and primarily. Therefore if GB is not
in motion AB will be at rest. But we have agreed that that which is
at rest if something else is not in motion must be moved by something.
Consequently, everything that is in motion must be moved by something:
for that which is in motion will always be divisible, and if a part
of it is not in motion the whole must be at rest. 

Since everything that is in motion must be moved by something, let
us take the case in which a thing is in locomotion and is moved by
something that is itself in motion, and that again is moved by something
else that is in motion, and that by something else, and so on continually:
then the series cannot go on to infinity, but there must be some first
movent. For let us suppose that this is not so and take the series
to be infinite. Let A then be moved by B, B by G, G by D, and so on,
each member of the series being moved by that which comes next to
it. Then since ex hypothesi the movent while causing motion is also
itself in motion, and the motion of the moved and the motion of the
movent must proceed simultaneously (for the movent is causing motion
and the moved is being moved simultaneously) it is evident that the
respective motions of A, B, G, and each of the other moved movents
are simultaneous. Let us take the motion of each separately and let
E be the motion of A, Z of B, and H and O respectively the motions
of G and D: for though they are all moved severally one by another,
yet we may still take the motion of each as numerically one, since
every motion is from something to something and is not infinite in
respect of its extreme points. By a motion that is numerically one
I mean a motion that proceeds from something numerically one and the
same to something numerically one and the same in a period of time
numerically one and the same: for a motion may be the same generically,
specifically, or numerically: it is generically the same if it belongs
to the same category, e.g. substance or quality: it is specifically
the same if it proceeds from something specifically the same to something
specifically the same, e.g. from white to black or from good to bad,
which is not of a kind specifically distinct: it is numerically the
same if it proceeds from something numerically one to something numerically
one in the same period of time, e.g. from a particular white to a
particular black, or from a particular place to a particular place,
in a particular period of time: for if the period of time were not
one and the same, the motion would no longer be numerically one though
it would still be specifically one. 

We have dealt with this question above. Now let us further take the
time in which A has completed its motion, and let it be represented
by K. Then since the motion of A is finite the time will also be finite.
But since the movents and the things moved are infinite, the motion
EZHO, i.e. the motion that is composed of all the individual motions,
must be infinite. For the motions of A, B, and the others may be equal,
or the motions of the others may be greater: but assuming what is
conceivable, we find that whether they are equal or some are greater,
in both cases the whole motion is infinite. And since the motion of
A and that of each of the others are simultaneous, the whole motion
must occupy the same time as the motion of A: but the time occupied
by the motion of A is finite: consequently the motion will be infinite
in a finite time, which is impossible. 

It might be thought that what we set out to prove has thus been shown,
but our argument so far does not prove it, because it does not yet
prove that anything impossible results from the contrary supposition:
for in a finite time there may be an infinite motion, though not of
one thing, but of many: and in the case that we are considering this
is so: for each thing accomplishes its own motion, and there is no
impossibility in many things being in motion simultaneously. But if
(as we see to be universally the case) that which primarily is moved
locally and corporeally must be either in contact with or continuous
with that which moves it, the things moved and the movents must be
continuous or in contact with one another, so that together they all
form a single unity: whether this unity is finite or infinite makes
no difference to our present argument; for in any case since the things
in motion are infinite in number the whole motion will be infinite,
if, as is theoretically possible, each motion is either equal to or
greater than that which follows it in the series: for we shall take
as actual that which is theoretically possible. If, then, A, B, G,
D form an infinite magnitude that passes through the motion EZHO in
the finite time K, this involves the conclusion that an infinite motion
is passed through in a finite time: and whether the magnitude in question
is finite or infinite this is in either case impossible. Therefore
the series must come to an end, and there must be a first movent and
a first moved: for the fact that this impossibility results only from
the assumption of a particular case is immaterial, since the case
assumed is theoretically possible, and the assumption of a theoretically
possible case ought not to give rise to any impossible result.

Part 2

That which is the first movement of a thing-in the sense that it supplies
not 'that for the sake of which' but the source of the motion-is always
together with that which is moved by it by 'together' I mean that
there is nothing intermediate between them). This is universally true
wherever one thing is moved by another. And since there are three
kinds of motion, local, qualitative, and quantitative, there must
also be three kinds of movent, that which causes locomotion, that
which causes alteration, and that which causes increase or decrease.

Let us begin with locomotion, for this is the primary motion. Everything
that is in locomotion is moved either by itself or by something else.
In the case of things that are moved by themselves it is evident that
the moved and the movent are together: for they contain within themselves
their first movent, so that there is nothing in between. The motion
of things that are moved by something else must proceed in one of
four ways: for there are four kinds of locomotion caused by something
other than that which is in motion, viz. pulling, pushing, carrying,
and twirling. All forms of locomotion are reducible to these. Thus
pushing on is a form of pushing in which that which is causing motion
away from itself follows up that which it pushes and continues to
push it: pushing off occurs when the movent does not follow up the
thing that it has moved: throwing when the movent causes a motion
away from itself more violent than the natural locomotion of the thing
moved, which continues its course so long as it is controlled by the
motion imparted to it. Again, pushing apart and pushing together are
forms respectively of pushing off and pulling: pushing apart is pushing
off, which may be a motion either away from the pusher or away from
something else, while pushing together is pulling, which may be a
motion towards something else as well as the puller. We may similarly
classify all the varieties of these last two, e.g. packing and combing:
the former is a form of pushing together, the latter a form of pushing
apart. The same is true of the other processes of combination and
separation (they will all be found to be forms of pushing apart or
of pushing together), except such as are involved in the processes
of becoming and perishing. (At same time it is evident that there
is no other kind of motion but combination and separation: for they
may all be apportioned to one or other of those already mentioned.)
Again, inhaling is a form of pulling, exhaling a form of pushing:
and the same is true of spitting and of all other motions that proceed
through the body, whether secretive or assimilative, the assimilative
being forms of pulling, the secretive of pushing off. All other kinds
of locomotion must be similarly reduced, for they all fall under one
or other of our four heads. And again, of these four, carrying and
twirling are to pulling and pushing. For carrying always follows one
of the other three methods, for that which is carried is in motion
accidentally, because it is in or upon something that is in motion,
and that which carries it is in doing so being either pulled or pushed
or twirled; thus carrying belongs to all the other three kinds of
motion in common. And twirling is a compound of pulling and pushing,
for that which is twirling a thing must be pulling one part of the
thing and pushing another part, since it impels one part away from
itself and another part towards itself. If, therefore, it can be shown
that that which is pushing and that which is pushing and pulling are
adjacent respectively to that which is being pushed and that which
is being pulled, it will be evident that in all locomotion there is
nothing intermediate between moved and movent. But the former fact
is clear even from the definitions of pushing and pulling, for pushing
is motion to something else from oneself or from something else, and
pulling is motion from something else to oneself or to something else,
when the motion of that which is pulling is quicker than the motion
that would separate from one another the two things that are continuous:
for it is this that causes one thing to be pulled on along with the
other. (It might indeed be thought that there is a form of pulling
that arises in another way: that wood, e.g. pulls fire in a manner
different from that described above. But it makes no difference whether
that which pulls is in motion or is stationary when it is pulling:
in the latter case it pulls to the place where it is, while in the
former it pulls to the place where it was.) Now it is impossible to
move anything either from oneself to something else or something else
to oneself without being in contact with it: it is evident, therefore,
that in all locomotion there is nothing intermediate between moved
and movent. 

Nor again is there anything intermediate between that which undergoes
and that which causes alteration: this can be proved by induction:
for in every case we find that the respective extremities of that
which causes and that which undergoes alteration are adjacent. For
our assumption is that things that are undergoing alteration are altered
in virtue of their being affected in respect of their so-called affective
qualities, since that which is of a certain quality is altered in
so far as it is sensible, and the characteristics in which bodies
differ from one another are sensible characteristics: for every body
differs from another in possessing a greater or lesser number of sensible
characteristics or in possessing the same sensible characteristics
in a greater or lesser degree. But the alteration of that which undergoes
alteration is also caused by the above-mentioned characteristics,
which are affections of some particular underlying quality. Thus we
say that a thing is altered by becoming hot or sweet or thick or dry
or white: and we make these assertions alike of what is inanimate
and of what is animate, and further, where animate things are in question,
we make them both of the parts that have no power of sense-perception
and of the senses themselves. For in a way even the senses undergo
alteration, since the active sense is a motion through the body in
the course of which the sense is affected in a certain way. We see,
then, that the animate is capable of every kind of alteration of which
the inanimate is capable: but the inanimate is not capable of every
kind of alteration of which the animate is capable, since it is not
capable of alteration in respect of the senses: moreover the inanimate
is unconscious of being affected by alteration, whereas the animate
is conscious of it, though there is nothing to prevent the animate
also being unconscious of it when the process of the alteration does
not concern the senses. Since, then, the alteration of that which
undergoes alteration is caused by sensible things, in every case of
such alteration it is evident that the respective extremities of that
which causes and that which undergoes alteration are adjacent. Thus
the air is continuous with that which causes the alteration, and the
body that undergoes alteration is continuous with the air. Again,
the colour is continuous with the light and the light with the sight.
And the same is true of hearing and smelling: for the primary movent
in respect to the moved is the air. Similarly, in the case of tasting,
the flavour is adjacent to the sense of taste. And it is just the
same in the case of things that are inanimate and incapable of sense-perception.
Thus there can be nothing intermediate between that which undergoes
and that which causes alteration. 

Nor, again, can there be anything intermediate between that which
suffers and that which causes increase: for the part of the latter
that starts the increase does so by becoming attached in such a way
to the former that the whole becomes one. Again, the decrease of that
which suffers decrease is caused by a part of the thing becoming detached.
So that which causes increase and that which causes decrease must
be continuous with that which suffers increase and that which suffers
decrease respectively: and if two things are continuous with one another
there can be nothing intermediate between them. 

It is evident, therefore, that between the extremities of the moved
and the movent that are respectively first and last in reference to
the moved there is nothing intermediate. 

Part 3

Everything, we say, that undergoes alteration is altered by sensible
causes, and there is alteration only in things that are said to be
essentially affected by sensible things. The truth of this is to be
seen from the following considerations. Of all other things it would
be most natural to suppose that there is alteration in figures and
shapes, and in acquired states and in the processes of acquiring and
losing these: but as a matter of fact in neither of these two classes
of things is there alteration. 

In the first place, when a particular formation of a thing is completed,
we do not call it by the name of its material: e.g. we do not call
the statue 'bronze' or the pyramid 'wax' or the bed 'wood', but we
use a derived expression and call them 'of bronze', 'waxen', and 'wooden'
respectively. But when a thing has been affected and altered in any
way we still call it by the original name: thus we speak of the bronze
or the wax being dry or fluid or hard or hot. 

And not only so: we also speak of the particular fluid or hot substance
as being bronze, giving the material the same name as that which we
use to describe the affection. 

Since, therefore, having regard to the figure or shape of a thing
we no longer call that which has become of a certain figure by the
name of the material that exhibits the figure, whereas having regard
to a thing's affections or alterations we still call it by the name
of its material, it is evident that becomings of the former kind cannot
be alterations. 

Moreover it would seem absurd even to speak in this way, to speak,
that is to say, of a man or house or anything else that has come into
existence as having been altered. Though it may be true that every
such becoming is necessarily the result of something's being altered,
the result, e.g. of the material's being condensed or rarefied or
heated or cooled, nevertheless it is not the things that are coming
into existence that are altered, and their becoming is not an alteration.

Again, acquired states, whether of the body or of the soul, are not
alterations. For some are excellences and others are defects, and
neither excellence nor defect is an alteration: excellence is a perfection
(for when anything acquires its proper excellence we call it perfect,
since it is then if ever that we have a thing in its natural state:
e.g. we have a perfect circle when we have one as good as possible),
while defect is a perishing of or departure from this condition. So
as when speaking of a house we do not call its arrival at perfection
an alteration (for it would be absurd to suppose that the coping or
the tiling is an alteration or that in receiving its coping or its
tiling a house is altered and not perfected), the same also holds
good in the case of excellences and defects and of the persons or
things that possess or acquire them: for excellences are perfections
of a thing's nature and defects are departures from it: consequently
they are not alterations. 

Further, we say that all excellences depend upon particular relations.
Thus bodily excellences such as health and a good state of body we
regard as consisting in a blending of hot and cold elements within
the body in due proportion, in relation either to one another or to
the surrounding atmosphere: and in like manner we regard beauty, strength,
and all the other bodily excellences and defects. Each of them exists
in virtue of a particular relation and puts that which possesses it
in a good or bad condition with regard to its proper affections, where
by 'proper' affections I mean those influences that from the natural
constitution of a thing tend to promote or destroy its existence.
Since then, relatives are neither themselves alterations nor the subjects
of alteration or of becoming or in fact of any change whatever, it
is evident that neither states nor the processes of losing and acquiring
states are alterations, though it may be true that their becoming
or perishing is necessarily, like the becoming or perishing of a specific
character or form, the result of the alteration of certain other things,
e.g. hot and cold or dry and wet elements or the elements, whatever
they may be, on which the states primarily depend. For each several
bodily defect or excellence involves a relation with those things
from which the possessor of the defect or excellence is naturally
subject to alteration: thus excellence disposes its possessor to be
unaffected by these influences or to be affected by those of them
that ought to be admitted, while defect disposes its possessor to
be affected by them or to be unaffected by those of them that ought
to be admitted. 

And the case is similar in regard to the states of the soul, all of
which (like those of body) exist in virtue of particular relations,
the excellences being perfections of nature and the defects departures
from it: moreover, excellence puts its possessor in good condition,
while defect puts its possessor in a bad condition, to meet his proper
affections. Consequently these cannot any more than the bodily states
be alterations, nor can the processes of losing and acquiring them
be so, though their becoming is necessarily the result of an alteration
of the sensitive part of the soul, and this is altered by sensible
objects: for all moral excellence is concerned with bodily pleasures
and pains, which again depend either upon acting or upon remembering
or upon anticipating. Now those that depend upon action are determined
by sense-perception, i.e. they are stimulated by something sensible:
and those that depend upon memory or anticipation are likewise to
be traced to sense-perception, for in these cases pleasure is felt
either in remembering what one has experienced or in anticipating
what one is going to experience. Thus all pleasure of this kind must
be produced by sensible things: and since the presence in any one
of moral defect or excellence involves the presence in him of pleasure
or pain (with which moral excellence and defect are always concerned),
and these pleasures and pains are alterations of the sensitive part,
it is evident that the loss and acquisition of these states no less
than the loss and acquisition of the states of the body must be the
result of the alteration of something else. Consequently, though their
becoming is accompanied by an alteration, they are not themselves
alterations. 

Again, the states of the intellectual part of the soul are not alterations,
nor is there any becoming of them. In the first place it is much more
true of the possession of knowledge that it depends upon a particular
relation. And further, it is evident that there is no becoming of
these states. For that which is potentially possessed of knowledge
becomes actually possessed of it not by being set in motion at all
itself but by reason of the presence of something else: i.e. it is
when it meets with the particular object that it knows in a manner
the particular through its knowledge of the universal. (Again, there
is no becoming of the actual use and activity of these states, unless
it is thought that there is a becoming of vision and touching and
that the activity in question is similar to these.) And the original
acquisition of knowledge is not a becoming or an alteration: for the
terms 'knowing' and 'understanding' imply that the intellect has reached
a state of rest and come to a standstill, and there is no becoming
that leads to a state of rest, since, as we have said above, change
at all can have a becoming. Moreover, just as to say, when any one
has passed from a state of intoxication or sleep or disease to the
contrary state, that he has become possessed of knowledge again is
incorrect in spite of the fact that he was previously incapable of
using his knowledge, so, too, when any one originally acquires the
state, it is incorrect to say that he becomes possessed of knowledge:
for the possession of understanding and knowledge is produced by the
soul's settling down out of the restlessness natural to it. Hence,
too, in learning and in forming judgements on matters relating to
their sense-perceptions children are inferior to adults owing to the
great amount of restlessness and motion in their souls. Nature itself
causes the soul to settle down and come to a state of rest for the
performance of some of its functions, while for the performance of
others other things do so: but in either case the result is brought
about through the alteration of something in the body, as we see in
the case of the use and activity of the intellect arising from a man's
becoming sober or being awakened. It is evident, then, from the preceding
argument that alteration and being altered occur in sensible things
and in the sensitive part of the soul, and, except accidentally, in
nothing else. 

Part 4

A difficulty may be raised as to whether every motion is commensurable
with every other or not. Now if they are all commensurable and if
two things to have the same velocity must accomplish an equal motion
in an equal time, then we may have a circumference equal to a straight
line, or, of course, the one may be greater or less than the other.
Further, if one thing alters and another accomplishes a locomotion
in an equal time, we may have an alteration and a locomotion equal
to one another: thus an affection will be equal to a length, which
is impossible. But is it not only when an equal motion is accomplished
by two things in an equal time that the velocities of the two are
equal? Now an affection cannot be equal to a length. Therefore there
cannot be an alteration equal to or less than a locomotion: and consequently
it is not the case that every motion is commensurable with every other.

But how will our conclusion work out in the case of the circle and
the straight line? It would be absurd to suppose that the motion of
one in a circle and of another in a straight line cannot be similar,
but that the one must inevitably move more quickly or more slowly
than the other, just as if the course of one were downhill and of
the other uphill. Moreover it does not as a matter of fact make any
difference to the argument to say that the one motion must inevitably
be quicker or slower than the other: for then the circumference can
be greater or less than the straight line; and if so it is possible
for the two to be equal. For if in the time A the quicker (B) passes
over the distance B' and the slower (G) passes over the distance G',
B' will be greater than G': for this is what we took 'quicker' to
mean: and so quicker motion also implies that one thing traverses
an equal distance in less time than another: consequently there will
be a part of A in which B will pass over a part of the circle equal
to G', while G will occupy the whole of A in passing over G'. None
the less, if the two motions are commensurable, we are confronted
with the consequence stated above, viz. that there may be a straight
line equal to a circle. But these are not commensurable: and so the
corresponding motions are not commensurable either. 

But may we say that things are always commensurable if the same terms
are applied to them without equivocation? e.g. a pen, a wine, and
the highest note in a scale are not commensurable: we cannot say whether
any one of them is sharper than any other: and why is this? they are
incommensurable because it is only equivocally that the same term
'sharp' is applied to them: whereas the highest note in a scale is
commensurable with the leading-note, because the term 'sharp' has
the same meaning as applied to both. Can it be, then, that the term
'quick' has not the same meaning as applied to straight motion and
to circular motion respectively? If so, far less will it have the
same meaning as applied to alteration and to locomotion.

Or shall we in the first place deny that things are always commensurable
if the same terms are applied to them without equivocation? For the
term 'much' has the same meaning whether applied to water or to air,
yet water and air are not commensurable in respect of it: or, if this
illustration is not considered satisfactory, 'double' at any rate
would seem to have the same meaning as applied to each (denoting in
each case the proportion of two to one), yet water and air are not
commensurable in respect of it. But here again may we not take up
the same position and say that the term 'much' is equivocal? In fact
there are some terms of which even the definitions are equivocal;
e.g. if 'much' were defined as 'so much and more','so much' would
mean something different in different cases: 'equal' is similarly
equivocal; and 'one' again is perhaps inevitably an equivocal term;
and if 'one' is equivocal, so is 'two'. Otherwise why is it that some
things are commensurable while others are not, if the nature of the
attribute in the two cases is really one and the same? 

Can it be that the incommensurability of two things in respect of
any attribute is due to a difference in that which is primarily capable
of carrying the attribute? Thus horse and dog are so commensurable
that we may say which is the whiter, since that which primarily contains
the whiteness is the same in both, viz. the surface: and similarly
they are commensurable in respect of size. But water and speech are
not commensurable in respect of clearness, since that which primarily
contains the attribute is different in the two cases. It would seem,
however that we must reject this solution, since clearly we could
thus make all equivocal attributes univocal and say merely that that
contains each of them is different in different cases: thus 'equality',
'sweetness', and 'whiteness' will severally always be the same, though
that which contains them is different in different cases. Moreover,
it is not any casual thing that is capable of carrying any attribute:
each single attribute can be carried primarily only by one single
thing. 

Must we then say that, if two things are to be commensurable in respect
of any attribute, not only must the attribute in question be applicable
to both without equivocation, but there must also be no specific differences
either in the attribute itself or in that which contains the attribute-that
these, I mean, must not be divisible in the way in which colour is
divided into kinds? Thus in this respect one thing will not be commensurable
with another, i.e. we cannot say that one is more coloured than the
other where only colour in general and not any particular colour is
meant; but they are commensurable in respect of whiteness.

Similarly in the case of motion: two things are of the same velocity
if they occupy an equal time in accomplishing a certain equal amount
of motion. Suppose, then, that in a certain time an alteration is
undergone by one half of a body's length and a locomotion is accomplished
the other half: can be say that in this case the alteration is equal
to the locomotion and of the same velocity? That would be absurd,
and the reason is that there are different species of motion. And
if in consequence of this we must say that two things are of equal
velocity if they accomplish locomotion over an equal distance in an
equal time, we have to admit the equality of a straight line and a
circumference. What, then, is the reason of this? Is it that locomotion
is a genus or that line is a genus? (We may leave the time out of
account, since that is one and the same.) If the lines are specifically
different, the locomotions also differ specifically from one another:
for locomotion is specifically differentiated according to the specific
differentiation of that over which it takes place. (It is also similarly
differentiated, it would seem, accordingly as the instrument of the
locomotion is different: thus if feet are the instrument, it is walking,
if wings it is flying; but perhaps we should rather say that this
is not so, and that in this case the differences in the locomotion
are merely differences of posture in that which is in motion.) We
may say, therefore, that things are of equal velocity in an equal
time they traverse the same magnitude: and when I call it 'the same'
I mean that it contains no specific difference and therefore no difference
in the motion that takes place over it. So we have now to consider
how motion is differentiated: and this discussion serves to show that
the genus is not a unity but contains a plurality latent in it and
distinct from it, and that in the case of equivocal terms sometimes
the different senses in which they are used are far removed from one
another, while sometimes there is a certain likeness between them,
and sometimes again they are nearly related either generically or
analogically, with the result that they seem not to be equivocal though
they really are. 

When, then, is there a difference of species? Is an attribute specifically
different if the subject is different while the attribute is the same,
or must the attribute itself be different as well? And how are we
to define the limits of a species? What will enable us to decide that
particular instances of whiteness or sweetness are the same or different?
Is it enough that it appears different in one subject from what appears
in another? Or must there be no sameness at all? And further, where
alteration is in question, how is one alteration to be of equal velocity
with another? One person may be cured quickly and another slowly,
and cures may also be simultaneous: so that, recovery of health being
an alteration, we have here alterations of equal velocity, since each
alteration occupies an equal time. But what alteration? We cannot
here speak of an 'equal' alteration: what corresponds in the category
of quality to equality in the category of quantity is 'likeness'.
However, let us say that there is equal velocity where the same change
is accomplished in an equal time. Are we, then, to find the commensurability
in the subject of the affection or in the affection itself? In the
case that we have just been considering it is the fact that health
is one and the same that enables us to arrive at the conclusion that
the one alteration is neither more nor less than the other, but that
both are alike. If on the other hand the affection is different in
the two cases, e.g. when the alterations take the form of becoming
white and becoming healthy respectively, here there is no sameness
or equality or likeness inasmuch as the difference in the affections
at once makes the alterations specifically different, and there is
no unity of alteration any more than there would be unity of locomotion
under like conditions. So we must find out how many species there
are of alteration and of locomotion respectively. Now if the things
that are in motion-that is to say, the things to which the motions
belong essentially and not accidentally-differ specifically, then
their respective motions will also differ specifically: if on the
other hand they differ generically or numerically, the motions also
will differ generically or numerically as the case may be. But there
still remains the question whether, supposing that two alterations
are of equal velocity, we ought to look for this equality in the sameness
(or likeness) of the affections, or in the things altered, to see
e.g. whether a certain quantity of each has become white. Or ought
we not rather to look for it in both? That is to say, the alterations
are the same or different according as the affections are the same
or different, while they are equal or unequal according as the things
altered are equal or unequal. 

And now we must consider the same question in the case of becoming
and perishing: how is one becoming of equal velocity with another?
They are of equal velocity if in an equal time there are produced
two things that are the same and specifically inseparable, e.g. two
men (not merely generically inseparable as e.g. two animals). Similarly
one is quicker than the other if in an equal time the product is different
in the two cases. I state it thus because we have no pair of terms
that will convey this 'difference' in the way in which unlikeness
is conveyed. If we adopt the theory that it is number that constitutes
being, we may indeed speak of a 'greater number' and a 'lesser number'
within the same species, but there is no common term that will include
both relations, nor are there terms to express each of them separately
in the same way as we indicate a higher degree or preponderance of
an affection by 'more', of a quantity by 'greater.' 

Part 5

Now since wherever there is a movent, its motion always acts upon
something, is always in something, and always extends to something
(by 'is always in something' I mean that it occupies a time: and by
'extends to something' I mean that it involves the traversing of a
certain amount of distance: for at any moment when a thing is causing
motion, it also has caused motion, so that there must always be a
certain amount of distance that has been traversed and a certain amount
of time that has been occupied). then, A the movement have moved B
a distance G in a time D, then in the same time the same force A will
move 1/2B twice the distance G, and in 1/2D it will move 1/2B the
whole distance for G: thus the rules of proportion will be observed.
Again if a given force move a given weight a certain distance in a
certain time and half the distance in half the time, half the motive
power will move half the weight the same distance in the same time.
Let E represent half the motive power A and Z half the weight B: then
the ratio between the motive power and the weight in the one case
is similar and proportionate to the ratio in the other, so that each
force will cause the same distance to be traversed in the same time.
But if E move Z a distance G in a time D, it does not necessarily
follow that E can move twice Z half the distance G in the same time.
If, then, A move B a distance G in a time D, it does not follow that
E, being half of A, will in the time D or in any fraction of it cause
B to traverse a part of G the ratio between which and the whole of
G is proportionate to that between A and E (whatever fraction of AE
may be): in fact it might well be that it will cause no motion at
all; for it does not follow that, if a given motive power causes a
certain amount of motion, half that power will cause motion either
of any particular amount or in any length of time: otherwise one man
might move a ship, since both the motive power of the ship-haulers
and the distance that they all cause the ship to traverse are divisible
into as many parts as there are men. Hence Zeno's reasoning is false
when he argues that there is no part of the millet that does not make
a sound: for there is no reason why any such part should not in any
length of time fail to move the air that the whole bushel moves in
falling. In fact it does not of itself move even such a quantity of
the air as it would move if this part were by itself: for no part
even exists otherwise than potentially. 

If on the other hand we have two forces each of which separately moves
one of two weights a given distance in a given time, then the forces
in combination will move the combined weights an equal distance in
an equal time: for in this case the rules of proportion apply.

Then does this hold good of alteration and of increase also? Surely
it does, for in any given case we have a definite thing that cause
increase and a definite thing that suffers increase, and the one causes
and the other suffers a certain amount of increase in a certain amount
of time. Similarly we have a definite thing that causes alteration
and a definite thing that undergoes alteration, and a certain amount,
or rather degree, of alteration is completed in a certain amount of
time: thus in twice as much time twice as much alteration will be
completed and conversely twice as much alteration will occupy twice
as much time: and the alteration of half of its object will occupy
half as much time and in half as much time half of the object will
be altered: or again, in the same amount of time it will be altered
twice as much. 

On the other hand if that which causes alteration or increase causes
a certain amount of increase or alteration respectively in a certain
amount of time, it does not necessarily follow that half the force
will occupy twice the time in altering or increasing the object, or
that in twice the time the alteration or increase will be completed
by it: it may happen that there will be no alteration or increase
at all, the case being the same as with the weight. 

----------------------------------------------------------------------

BOOK VIII

Part 1 

It remains to consider the following question. Was there ever a becoming
of motion before which it had no being, and is it perishing again
so as to leave nothing in motion? Or are we to say that it never had
any becoming and is not perishing, but always was and always will
be? Is it in fact an immortal never-failing property of things that
are, a sort of life as it were to all naturally constituted things?

Now the existence of motion is asserted by all who have anything to
say about nature, because they all concern themselves with the construction
of the world and study the question of becoming and perishing, which
processes could not come about without the existence of motion. But
those who say that there is an infinite number of worlds, some of
which are in process of becoming while others are in process of perishing,
assert that there is always motion (for these processes of becoming
and perishing of the worlds necessarily involve motion), whereas those
who hold that there is only one world, whether everlasting or not,
make corresponding assumptions in regard to motion. If then it is
possible that at any time nothing should be in motion, this must come
about in one of two ways: either in the manner described by Anaxagoras,
who says that all things were together and at rest for an infinite
period of time, and that then Mind introduced motion and separated
them; or in the manner described by Empedocles, according to whom
the universe is alternately in motion and at rest-in motion, when
Love is making the one out of many, or Strife is making many out of
one, and at rest in the intermediate periods of time-his account being
as follows: 

'Since One hath learned to spring from Manifold, And One disjoined
makes manifold arise, Thus they Become, nor stable is their life:
But since their motion must alternate be, Thus have they ever Rest
upon their round': for we must suppose that he means by this that
they alternate from the one motion to the other. We must consider,
then, how this matter stands, for the discovery of the truth about
it is of importance, not only for the study of nature, but also for
the investigation of the First Principle. 

Let us take our start from what we have already laid down in our course
on Physics. Motion, we say, is the fulfilment of the movable in so
far as it is movable. Each kind of motion, therefore, necessarily
involves the presence of the things that are capable of that motion.
In fact, even apart from the definition of motion, every one would
admit that in each kind of motion it is that which is capable of that
motion that is in motion: thus it is that which is capable of alteration
that is altered, and that which is capable of local change that is
in locomotion: and so there must be something capable of being burned
before there can be a process of being burned, and something capable
of burning before there can be a process of burning. Moreover, these
things also must either have a beginning before which they had no
being, or they must be eternal. Now if there was a becoming of every
movable thing, it follows that before the motion in question another
change or motion must have taken place in which that which was capable
of being moved or of causing motion had its becoming. To suppose,
on the other hand, that these things were in being throughout all
previous time without there being any motion appears unreasonable
on a moment's thought, and still more unreasonable, we shall find,
on further consideration. For if we are to say that, while there are
on the one hand things that are movable, and on the other hand things
that are motive, there is a time when there is a first movent and
a first moved, and another time when there is no such thing but only
something that is at rest, then this thing that is at rest must previously
have been in process of change: for there must have been some cause
of its rest, rest being the privation of motion. Therefore, before
this first change there will be a previous change. For some things
cause motion in only one way, while others can produce either of two
contrary motions: thus fire causes heating but not cooling, whereas
it would seem that knowledge may be directed to two contrary ends
while remaining one and the same. Even in the former class, however,
there seems to be something similar, for a cold thing in a sense causes
heating by turning away and retiring, just as one possessed of knowledge
voluntarily makes an error when he uses his knowledge in the reverse
way. But at any rate all things that are capable respectively of affecting
and being affected, or of causing motion and being moved, are capable
of it not under all conditions, but only when they are in a particular
condition and approach one another: so it is on the approach of one
thing to another that the one causes motion and the other is moved,
and when they are present under such conditions as rendered the one
motive and the other movable. So if the motion was not always in process,
it is clear that they must have been in a condition not such as to
render them capable respectively of being moved and of causing motion,
and one or other of them must have been in process of change: for
in what is relative this is a necessary consequence: e.g. if one thing
is double another when before it was not so, one or other of them,
if not both, must have been in process of change. It follows then,
that there will be a process of change previous to the first.

(Further, how can there be any 'before' and 'after' without the existence
of time? Or how can there be any time without the existence of motion?
If, then, time is the number of motion or itself a kind of motion,
it follows that, if there is always time, motion must also be eternal.
But so far as time is concerned we see that all with one exception
are in agreement in saying that it is uncreated: in fact, it is just
this that enables Democritus to show that all things cannot have had
a becoming: for time, he says, is uncreated. Plato alone asserts the
creation of time, saying that it had a becoming together with the
universe, the universe according to him having had a becoming. Now
since time cannot exist and is unthinkable apart from the moment,
and the moment a kind of middle-point, uniting as it does in itself
both a beginning and an end, a beginning of future time and an end
of past time, it follows that there must always be time: for the extremity
of the last period of time that we take must be found in some moment,
since time contains no point of contact for us except the moment.
Therefore, since the moment is both a beginning and an end, there
must always be time on both sides of it. But if this is true of time,
it is evident that it must also be true of motion, time being a kind
of affection of motion.) 

The same reasoning will also serve to show the imperishability of
motion: just as a becoming of motion would involve, as we saw, the
existence of a process of change previous to the first, in the same
way a perishing of motion would involve the existence of a process
of change subsequent to the last: for when a thing ceases to be moved,
it does not therefore at the same time cease to be movable-e.g. the
cessation of the process of being burned does not involve the cessation
of the capacity of being burned, since a thing may be capable of being
burned without being in process of being burned-nor, when a thing
ceases to be movent, does it therefore at the same time cease to a
be motive. Again, the destructive agent will have to be destroyed,
after what it destroys has been destroyed, and then that which has
the capacity of destroying it will have to be destroyed afterwards,
(so that there will be a process of change subsequent to the last,)
for being destroyed also is a kind of change. If, then, view which
we are criticizing involves these impossible consequences, it is clear
that motion is eternal and cannot have existed at one time and not
at another: in fact such a view can hardly be described as anythling
else than fantastic. 

And much the same may be said of the view that such is the ordinance
of nature and that this must be regarded as a principle, as would
seem to be the view of Empedocles when he says that the constitution
of the world is of necessity such that Love and Strife alternately
predominate and cause motion, while in the intermediate period of
time there is a state of rest. Probably also those who like like Anaxagoras,
assert a single principle (of motion) would hold this view. But that
which is produced or directed by nature can never be anything disorderly:
for nature is everywhere the cause of order. Moreover, there is no
ratio in the relation of the infinite to the infinite, whereas order
always means ratio. But if we say that there is first a state of rest
for an infinite time, and then motion is started at some moment, and
that the fact that it is this rather than a previous moment is of
no importance, and involves no order, then we can no longer say that
it is nature's work: for if anything is of a certain character naturally,
it either is so invariably and is not sometimes of this and sometimes
of another character (e.g. fire, which travels upwards naturally,
does not sometimes do so and sometimes not) or there is a ratio in
the variation. It would be better, therefore, to say with Empedocles
and any one else who may have maintained such a theory as his that
the universe is alternately at rest and in motion: for in a system
of this kind we have at once a certain order. But even here the holder
of the theory ought not only to assert the fact: he ought to explain
the cause of it: i.e. he should not make any mere assumption or lay
down any gratuitous axiom, but should employ either inductive or demonstrative
reasoning. The Love and Strife postulated by Empedocles are not in
themselves causes of the fact in question, nor is it of the essence
of either that it should be so, the essential function of the former
being to unite, of the latter to separate. If he is to go on to explain
this alternate predominance, he should adduce cases where such a state
of things exists, as he points to the fact that among mankind we have
something that unites men, namely Love, while on the other hand enemies
avoid one another: thus from the observed fact that this occurs in
certain cases comes the assumption that it occurs also in the universe.
Then, again, some argument is needed to explain why the predominance
of each of the two forces lasts for an equal period of time. But it
is a wrong assumption to suppose universally that we have an adequate
first principle in virtue of the fact that something always is so
or always happens so. Thus Democritus reduces the causes that explain
nature to the fact that things happened in the past in the same way
as they happen now: but he does not think fit to seek for a first
principle to explain this 'always': so, while his theory is right
in so far as it is applied to certain individual cases, he is wrong
in making it of universal application. Thus, a triangle always has
its angles equal to two right angles, but there is nevertheless an
ulterior cause of the eternity of this truth, whereas first principles
are eternal and have no ulterior cause. Let this conclude what we
have to say in support of our contention that there never was a time
when there was not motion, and never will be a time when there will
not be motion. 

Part 2

The arguments that may be advanced against this position are not difficult
to dispose of. The chief considerations that might be thought to indicate
that motion may exist though at one time it had not existed at all
are the following: 

First, it may be said that no process of change is eternal: for the
nature of all change is such that it proceeds from something to something,
so that every process of change must be bounded by the contraries
that mark its course, and no motion can go on to infinity.

Secondly, we see that a thing that neither is in motion nor contains
any motion within itself can be set in motion; e.g. inanimate things
that are (whether the whole or some part is in question) not in motion
but at rest, are at some moment set in motion: whereas, if motion
cannot have a becoming before which it had no being, these things
ought to be either always or never in motion. 

Thirdly, the fact is evident above all in the case of animate beings:
for it sometimes happens that there is no motion in us and we are
quite still, and that nevertheless we are then at some moment set
in motion, that is to say it sometimes happens that we produce a beginning
of motion in ourselves spontaneously without anything having set us
in motion from without. We see nothing like this in the case of inanimate
things, which are always set in motion by something else from without:
the animal, on the other hand, we say, moves itself: therefore, if
an animal is ever in a state of absolute rest, we have a motionless
thing in which motion can be produced from the thing itself, and not
from without. Now if this can occur in an animal, why should not the
same be true also of the universe as a whole? If it can occur in a
small world it could also occur in a great one: and if it can occur
in the world, it could also occur in the infinite; that is, if the
infinite could as a whole possibly be in motion or at rest.

Of these objections, then, the first-mentioned motion to opposites
is not always the same and numerically one a correct statement; in
fact, this may be said to be a necessary conclusion, provided that
it is possible for the motion of that which is one and the same to
be not always one and the same. (I mean that e.g. we may question
whether the note given by a single string is one and the same, or
is different each time the string is struck, although the string is
in the same condition and is moved in the same way.) But still, however
this may be, there is nothing to prevent there being a motion that
is the same in virtue of being continuous and eternal: we shall have
something to say later that will make this point clearer.

As regards the second objection, no absurdity is involved in the fact
that something not in motion may be set in motion, that which caused
the motion from without being at one time present, and at another
absent. Nevertheless, how this can be so remains matter for inquiry;
how it comes about, I mean, that the same motive force at one time
causes a thing to be in motion, and at another does not do so: for
the difficulty raised by our objector really amounts to this-why is
it that some things are not always at rest, and the rest always in
motion? 

The third objection may be thought to present more difficulty than
the others, namely, that which alleges that motion arises in things
in which it did not exist before, and adduces in proof the case of
animate things: thus an animal is first at rest and afterwards walks,
not having been set in motion apparently by anything from without.
This, however, is false: for we observe that there is always some
part of the animal's organism in motion, and the cause of the motion
of this part is not the animal itself, but, it may be, its environment.
Moreover, we say that the animal itself originates not all of its
motions but its locomotion. So it may well be the case-or rather we
may perhaps say that it must necessarily be the case-that many motions
are produced in the body by its environment, and some of these set
in motion the intellect or the appetite, and this again then sets
the whole animal in motion: this is what happens when animals are
asleep: though there is then no perceptive motion in them, there is
some motion that causes them to wake up again. But we will leave this
point also to be elucidated at a later stage in our discussion.

Part 3

Our enquiry will resolve itself at the outset into a consideration
of the above-mentioned problem-what can be the reason why some things
in the world at one time are in motion and at another are at rest
again? Now one of three things must be true: either all things are
always at rest, or all things are always in motion, or some things
are in motion and others at rest: and in this last case again either
the things that are in motion are always in motion and the things
that are at rest are always at rest, or they are all constituted so
as to be capable alike of motion and of rest; or there is yet a third
possibility remaining-it may be that some things in the world are
always motionless, others always in motion, while others again admit
of both conditions. This last is the account of the matter that we
must give: for herein lies the solution of all the difficulties raised
and the conclusion of the investigation upon which we are engaged.

To maintain that all things are at rest, and to disregard sense-perception
in an attempt to show the theory to be reasonable, would be an instance
of intellectual weakness: it would call in question a whole system,
not a particular detail: moreover, it would be an attack not only
on the physicist but on almost all sciences and all received opinions,
since motion plays a part in all of them. Further, just as in arguments
about mathematics objections that involve first principles do not
affect the mathematician-and the other sciences are in similar case-so,
too, objections involving the point that we have just raised do not
affect the physicist: for it is a fundamental assumption with him
that motion is ultimately referable to nature herself. 

The assertion that all things are in motion we may fairly regard as
equally false, though it is less subversive of physical science: for
though in our course on physics it was laid down that rest no less
than motion is ultimately referable to nature herself, nevertheless
motion is the characteristic fact of nature: moreover, the view is
actually held by some that not merely some things but all things in
the world are in motion and always in motion, though we cannot apprehend
the fact by sense-perception. Although the supporters of this theory
do not state clearly what kind of motion they mean, or whether they
mean all kinds, it is no hard matter to reply to them: thus we may
point out that there cannot be a continuous process either of increase
or of decrease: that which comes between the two has to be included.
The theory resembles that about the stone being worn away by the drop
of water or split by plants growing out of it: if so much has been
extruded or removed by the drop, it does not follow that half the
amount has previously been extruded or removed in half the time: the
case of the hauled ship is exactly comparable: here we have so many
drops setting so much in motion, but a part of them will not set as
much in motion in any period of time. The amount removed is, it is
true, divisible into a number of parts, but no one of these was set
in motion separately: they were all set in motion together. It is
evident, then, that from the fact that the decrease is divisible into
an infinite number of parts it does not follow that some part must
always be passing away: it all passes away at a particular moment.
Similarly, too, in the case of any alteration whatever if that which
suffers alteration is infinitely divisible it does not follow from
this that the same is true of the alteration itself, which often occurs
all at once, as in freezing. Again, when any one has fallen ill, there
must follow a period of time in which his restoration to health is
in the future: the process of change cannot take place in an instant:
yet the change cannot be a change to anything else but health. The
assertion. therefore, that alteration is continuous is an extravagant
calling into question of the obvious: for alteration is a change from
one contrary to another. Moreover, we notice that a stone becomes
neither harder nor softer. Again, in the matter of locomotion, it
would be a strange thing if a stone could be falling or resting on
the ground without our being able to perceive the fact. Further, it
is a law of nature that earth and all other bodies should remain in
their proper places and be moved from them only by violence: from
the fact then that some of them are in their proper places it follows
that in respect of place also all things cannot be in motion. These
and other similar arguments, then, should convince us that it is impossible
either that all things are always in motion or that all things are
always at rest. 

Nor again can it be that some things are always at rest, others always
in motion, and nothing sometimes at rest and sometimes in motion.
This theory must be pronounced impossible on the same grounds as those
previously mentioned: viz. that we see the above-mentioned changes
occurring in the case of the same things. We may further point out
that the defender of this position is fighting against the obvious,
for on this theory there can be no such thing as increase: nor can
there be any such thing as compulsory motion, if it is impossible
that a thing can be at rest before being set in motion unnaturally.
This theory, then, does away with becoming and perishing. Moreover,
motion, it would seem, is generally thought to be a sort of becoming
and perishing, for that to which a thing changes comes to be, or occupancy
of it comes to be, and that from which a thing changes ceases to be,
or there ceases to be occupancy of it. It is clear, therefore, that
there are cases of occasional motion and occasional rest.

We have now to take the assertion that all things are sometimes at
rest and sometimes in motion and to confront it with the arguments
previously advanced. We must take our start as before from the possibilities
that we distinguished just above. Either all things are at rest, or
all things are in motion, or some things are at rest and others in
motion. And if some things are at rest and others in motion, then
it must be that either all things are sometimes at rest and sometimes
in motion, or some things are always at rest and the remainder always
in motion, or some of the things are always at rest and others always
in motion while others again are sometimes at rest and sometimes in
motion. Now we have said before that it is impossible that all things
should be at rest: nevertheless we may now repeat that assertion.
We may point out that, even if it is really the case, as certain persons
assert, that the existent is infinite and motionless, it certainly
does not appear to be so if we follow sense-perception: many things
that exist appear to be in motion. Now if there is such a thing as
false opinion or opinion at all, there is also motion; and similarly
if there is such a thing as imagination, or if it is the case that
anything seems to be different at different times: for imagination
and opinion are thought to be motions of a kind. But to investigate
this question at all-to seek a reasoned justification of a belief
with regard to which we are too well off to require reasoned justification-implies
bad judgement of what is better and what is worse, what commends itself
to belief and what does not, what is ultimate and what is not. It
is likewise impossible that all things should be in motion or that
some things should be always in motion and the remainder always at
rest. We have sufficient ground for rejecting all these theories in
the single fact that we see some things that are sometimes in motion
and sometimes at rest. It is evident, therefore, that it is no less
impossible that some things should be always in motion and the remainder
always at rest than that all things should be at rest or that all
things should be in motion continuously. It remains, then, to consider
whether all things are so constituted as to be capable both of being
in motion and of being at rest, or whether, while some things are
so constituted, some are always at rest and some are always in motion:
for it is this last view that we have to show to be true.

Part 4

Now of things that cause motion or suffer motion, to some the motion
is accidental, to others essential: thus it is accidental to what
merely belongs to or contains as a part a thing that causes motion
or suffers motion, essential to a thing that causes motion or suffers
motion not merely by belonging to such a thing or containing it as
a part. 

Of things to which the motion is essential some derive their motion
from themselves, others from something else: and in some cases their
motion is natural, in others violent and unnatural. Thus in things
that derive their motion from themselves, e.g. all animals, the motion
is natural (for when an animal is in motion its motion is derived
from itself): and whenever the source of the motion of a thing is
in the thing itself we say that the motion of that thing is natural.
Therefore the animal as a whole moves itself naturally: but the body
of the animal may be in motion unnaturally as well as naturally: it
depends upon the kind of motion that it may chance to be suffering
and the kind of element of which it is composed. And the motion of
things that derive their motion from something else is in some cases
natural, in other unnatural: e.g. upward motion of earthy things and
downward motion of fire are unnatural. Moreover the parts of animals
are often in motion in an unnatural way, their positions and the character
of the motion being abnormal. The fact that a thing that is in motion
derives its motion from something is most evident in things that are
in motion unnaturally, because in such cases it is clear that the
motion is derived from something other than the thing itself. Next
to things that are in motion unnaturally those whose motion while
natural is derived from themselves-e.g. animals-make this fact clear:
for here the uncertainty is not as to whether the motion is derived
from something but as to how we ought to distinguish in the thing
between the movent and the moved. It would seem that in animals, just
as in ships and things not naturally organized, that which causes
motion is separate from that which suffers motion, and that it is
only in this sense that the animal as a whole causes its own motion.

The greatest difficulty, however, is presented by the remaining case
of those that we last distinguished. Where things derive their motion
from something else we distinguished the cases in which the motion
is unnatural: we are left with those that are to be contrasted with
the others by reason of the fact that the motion is natural. It is
in these cases that difficulty would be experienced in deciding whence
the motion is derived, e.g. in the case of light and heavy things.
When these things are in motion to positions the reverse of those
they would properly occupy, their motion is violent: when they are
in motion to their proper positions-the light thing up and the heavy
thing down-their motion is natural; but in this latter case it is
no longer evident, as it is when the motion is unnatural, whence their
motion is derived. It is impossible to say that their motion is derived
from themselves: this is a characteristic of life and peculiar to
living things. Further, if it were, it would have been in their power
to stop themselves (I mean that if e.g. a thing can cause itself to
walk it can also cause itself not to walk), and so, since on this
supposition fire itself possesses the power of upward locomotion,
it is clear that it should also possess the power of downward locomotion.
Moreover if things move themselves, it would be unreasonable to suppose
that in only one kind of motion is their motion derived from themselves.
Again, how can anything of continuous and naturally connected substance
move itself? In so far as a thing is one and continuous not merely
in virtue of contact, it is impassive: it is only in so far as a thing
is divided that one part of it is by nature active and another passive.
Therefore none of the things that we are now considering move themselves
(for they are of naturally connected substance), nor does anything
else that is continuous: in each case the movent must be separate
from the moved, as we see to be the case with inanimate things when
an animate thing moves them. It is the fact that these things also
always derive their motion from something: what it is would become
evident if we were to distinguish the different kinds of cause.

The above-mentioned distinctions can also be made in the case of things
that cause motion: some of them are capable of causing motion unnaturally
(e.g. the lever is not naturally capable of moving the weight), others
naturally (e.g. what is actually hot is naturally capable of moving
what is potentially hot): and similarly in the case of all other things
of this kind. 

In the same way, too, what is potentially of a certain quality or
of a certain quantity in a certain place is naturally movable when
it contains the corresponding principle in itself and not accidentally
(for the same thing may be both of a certain quality and of a certain
quantity, but the one is an accidental, not an essential property
of the other). So when fire or earth is moved by something the motion
is violent when it is unnatural, and natural when it brings to actuality
the proper activities that they potentially possess. But the fact
that the term 'potentially' is used in more than one sense is the
reason why it is not evident whence such motions as the upward motion
of fire and the downward motion of earth are derived. One who is learning
a science potentially knows it in a different sense from one who while
already possessing the knowledge is not actually exercising it. Wherever
we have something capable of acting and something capable of being
correspondingly acted on, in the event of any such pair being in contact
what is potential becomes at times actual: e.g. the learner becomes
from one potential something another potential something: for one
who possesses knowledge of a science but is not actually exercising
it knows the science potentially in a sense, though not in the same
sense as he knew it potentially before he learnt it. And when he is
in this condition, if something does not prevent him, he actively
exercises his knowledge: otherwise he would be in the contradictory
state of not knowing. In regard to natural bodies also the case is
similar. Thus what is cold is potentially hot: then a change takes
place and it is fire, and it burns, unless something prevents and
hinders it. So, too, with heavy and light: light is generated from
heavy, e.g. air from water (for water is the first thing that is potentially
light), and air is actually light, and will at once realize its proper
activity as such unless something prevents it. The activity of lightness
consists in the light thing being in a certain situation, namely high
up: when it is in the contrary situation, it is being prevented from
rising. The case is similar also in regard to quantity and quality.
But, be it noted, this is the question we are trying to answer-how
can we account for the motion of light things and heavy things to
their proper situations? The reason for it is that they have a natural
tendency respectively towards a certain position: and this constitutes
the essence of lightness and heaviness, the former being determined
by an upward, the latter by a downward, tendency. As we have said,
a thing may be potentially light or heavy in more senses than one.
Thus not only when a thing is water is it in a sense potentially light,
but when it has become air it may be still potentially light: for
it may be that through some hindrance it does not occupy an upper
position, whereas, if what hinders it is removed, it realizes its
activity and continues to rise higher. The process whereby what is
of a certain quality changes to a condition of active existence is
similar: thus the exercise of knowledge follows at once upon the possession
of it unless something prevents it. So, too, what is of a certain
quantity extends itself over a certain space unless something prevents
it. The thing in a sense is and in a sense is not moved by one who
moves what is obstructing and preventing its motion (e.g. one who
pulls away a pillar from under a roof or one who removes a stone from
a wineskin in the water is the accidental cause of motion): and in
the same way the real cause of the motion of a ball rebounding from
a wall is not the wall but the thrower. So it is clear that in all
these cases the thing does not move itself, but it contains within
itself the source of motion-not of moving something or of causing
motion, but of suffering it. 

If then the motion of all things that are in motion is either natural
or unnatural and violent, and all things whose motion is violent and
unnatural are moved by something, and something other than themselves,
and again all things whose motion is natural are moved by something-both
those that are moved by themselves and those that are not moved by
themselves (e.g. light things and heavy things, which are moved either
by that which brought the thing into existence as such and made it
light and heavy, or by that which released what was hindering and
preventing it); then all things that are in motion must be moved by
something. 

Part 5

Now this may come about in either of two ways. Either the movent is
not itself responsible for the motion, which is to be referred to
something else which moves the movent, or the movent is itself responsible
for the motion. Further, in the latter case, either the movent immediately
precedes the last thing in the series, or there may be one or more
intermediate links: e.g. the stick moves the stone and is moved by
the hand, which again is moved by the man: in the man, however, we
have reached a movent that is not so in virtue of being moved by something
else. Now we say that the thing is moved both by the last and by the
first movent in the series, but more strictly by the first, since
the first movent moves the last, whereas the last does not move the
first, and the first will move the thing without the last, but the
last will not move it without the first: e.g. the stick will not move
anything unless it is itself moved by the man. If then everything
that is in motion must be moved by something, and the movent must
either itself be moved by something else or not, and in the former
case there must be some first movent that is not itself moved by anything
else, while in the case of the immediate movent being of this kind
there is no need of an intermediate movent that is also moved (for
it is impossible that there should be an infinite series of movents,
each of which is itself moved by something else, since in an infinite
series there is no first term)-if then everything that is in motion
is moved by something, and the first movent is moved but not by anything
else, it much be moved by itself. 

This same argument may also be stated in another way as follows. Every
movent moves something and moves it with something, either with itself
or with something else: e.g. a man moves a thing either himself or
with a stick, and a thing is knocked down either by the wind itself
or by a stone propelled by the wind. But it is impossible for that
with which a thing is moved to move it without being moved by that
which imparts motion by its own agency: on the other hand, if a thing
imparts motion by its own agency, it is not necessary that there should
be anything else with which it imparts motion, whereas if there is
a different thing with which it imparts motion, there must be something
that imparts motion not with something else but with itself, or else
there will be an infinite series. If, then, anything is a movent while
being itself moved, the series must stop somewhere and not be infinite.
Thus, if the stick moves something in virtue of being moved by the
hand, the hand moves the stick: and if something else moves with the
hand, the hand also is moved by something different from itself. So
when motion by means of an instrument is at each stage caused by something
different from the instrument, this must always be preceded by something
else which imparts motion with itself. Therefore, if this last movent
is in motion and there is nothing else that moves it, it must move
itself. So this reasoning also shows that when a thing is moved, if
it is not moved immediately by something that moves itself, the series
brings us at some time or other to a movent of this kind.

And if we consider the matter in yet a third wa Ly we shall get this
same result as follows. If everything that is in motion is moved by
something that is in motion, ether this being in motion is an accidental
attribute of the movents in question, so that each of them moves something
while being itself in motion, but not always because it is itself
in motion, or it is not accidental but an essential attribute. Let
us consider the former alternative. If then it is an accidental attribute,
it is not necessary that that is in motion should be in motion: and
if this is so it is clear that there may be a time when nothing that
exists is in motion, since the accidental is not necessary but contingent.
Now if we assume the existence of a possibility, any conclusion that
we thereby reach will not be an impossibility though it may be contrary
to fact. But the nonexistence of motion is an impossibility: for we
have shown above that there must always be motion. 

Moreover, the conclusion to which we have been led is a reasonable
one. For there must be three things-the moved, the movent, and the
instrument of motion. Now the moved must be in motion, but it need
not move anything else: the instrument of motion must both move something
else and be itself in motion (for it changes together with the moved,
with which it is in contact and continuous, as is clear in the case
of things that move other things locally, in which case the two things
must up to a certain point be in contact): and the movent-that is
to say, that which causes motion in such a manner that it is not merely
the instrument of motion-must be unmoved. Now we have visual experience
of the last term in this series, namely that which has the capacity
of being in motion, but does not contain a motive principle, and also
of that which is in motion but is moved by itself and not by anything
else: it is reasonable, therefore, not to say necessary, to suppose
the existence of the third term also, that which causes motion but
is itself unmoved. So, too, Anaxagoras is right when he says that
Mind is impassive and unmixed, since he makes it the principle of
motion: for it could cause motion in this sense only by being itself
unmoved, and have supreme control only by being unmixed.

We will now take the second alternative. If the movement is not accidentally
but necessarily in motion-so that, if it were not in motion, it would
not move anything-then the movent, in so far as it is in motion, must
be in motion in one of two ways: it is moved either as that is which
is moved with the same kind of motion, or with a different kind-either
that which is heating, I mean, is itself in process of becoming hot,
that which is making healthy in process of becoming healthy, and that
which is causing locomotion in process of locomotion, or else that
which is making healthy is, let us say, in process of locomotion,
and that which is causing locomotion in process of, say, increase.
But it is evident that this is impossible. For if we adopt the first
assumption we have to make it apply within each of the very lowest
species into which motion can be divided: e.g. we must say that if
some one is teaching some lesson in geometry, he is also in process
of being taught that same lesson in geometry, and that if he is throwing
he is in process of being thrown in just the same manner. Or if we
reject this assumption we must say that one kind of motion is derived
from another; e.g. that that which is causing locomotion is in process
of increase, that which is causing this increase is in process of
being altered by something else, and that which is causing this alteration
is in process of suffering some different kind of motion. But the
series must stop somewhere, since the kinds of motion are limited;
and if we say that the process is reversible, and that that which
is causing alteration is in process of locomotion, we do no more than
if we had said at the outset that that which is causing locomotion
is in process of locomotion, and that one who is teaching is in process
of being taught: for it is clear that everything that is moved is
moved by the movent that is further back in the series as well as
by that which immediately moves it: in fact the earlier movent is
that which more strictly moves it. But this is of course impossible:
for it involves the consequence that one who is teaching is in process
of learning what he is teaching, whereas teaching necessarily implies
possessing knowledge, and learning not possessing it. Still more unreasonable
is the consequence involved that, since everything that is moved is
moved by something that is itself moved by something else, everything
that has a capacity for causing motion has as such a corresponding
capacity for being moved: i.e. it will have a capacity for being moved
in the sense in which one might say that everything that has a capacity
for making healthy, and exercises that capacity, has as such a capacity
for being made healthy, and that which has a capacity for building
has as such a capacity for being built. It will have the capacity
for being thus moved either immediately or through one or more links
(as it will if, while everything that has a capacity for causing motion
has as such a capacity for being moved by something else, the motion
that it has the capacity for suffering is not that with which it affects
what is next to it, but a motion of a different kind; e.g. that which
has a capacity for making healthy might as such have a capacity for
learn. the series, however, could be traced back, as we said before,
until at some time or other we arrived at the same kind of motion).
Now the first alternative is impossible, and the second is fantastic:
it is absurd that that which has a capacity for causing alteration
should as such necessarily have a capacity, let us say, for increase.
It is not necessary, therefore, that that which is moved should always
be moved by something else that is itself moved by something else:
so there will be an end to the series. Consequently the first thing
that is in motion will derive its motion either from something that
is at rest or from itself. But if there were any need to consider
which of the two, that which moves itself or that which is moved by
something else, is the cause and principle of motion, every one would
decide the former: for that which is itself independently a cause
is always prior as a cause to that which is so only in virtue of being
itself dependent upon something else that makes it so. 

We must therefore make a fresh start and consider the question; if
a thing moves itself, in what sense and in what manner does it do
so? Now everything that is in motion must be infinitely divisible,
for it has been shown already in our general course on Physics, that
everything that is essentially in motion is continuous. Now it is
impossible that that which moves itself should in its entirety move
itself: for then, while being specifically one and indivisible, it
would as a Whole both undergo and cause the same locomotion or alteration:
thus it would at the same time be both teaching and being taught (the
same thing), or both restoring to and being restored to the same health.
Moreover, we have established the fact that it is the movable that
is moved; and this is potentially, not actually, in motion, but the
potential is in process to actuality, and motion is an incomplete
actuality of the movable. The movent on the other hand is already
in activity: e.g. it is that which is hot that produces heat: in fact,
that which produces the form is always something that possesses it.
Consequently (if a thing can move itself as a whole), the same thing
in respect of the same thing may be at the same time both hot and
not hot. So, too, in every other case where the movent must be described
by the same name in the same sense as the moved. Therefore when a
thing moves itself it is one part of it that is the movent and another
part that is moved. But it is not self-moving in the sense that each
of the two parts is moved by the other part: the following considerations
make this evident. In the first place, if each of the two parts is
to move the other, there will be no first movent. If a thing is moved
by a series of movents, that which is earlier in the series is more
the cause of its being moved than that which comes next, and will
be more truly the movent: for we found that there are two kinds of
movent, that which is itself moved by something else and that which
derives its motion from itself: and that which is further from the
thing that is moved is nearer to the principle of motion than that
which is intermediate. In the second place, there is no necessity
for the movent part to be moved by anything but itself: so it can
only be accidentally that the other part moves it in return. I take
then the possible case of its not moving it: then there will be a
part that is moved and a part that is an unmoved movent. In the third
place, there is no necessity for the movent to be moved in return:
on the contrary the necessity that there should always be motion makes
it necessary that there should be some movent that is either unmoved
or moved by itself. In the fourth place we should then have a thing
undergoing the same motion that it is causing-that which is producing
heat, therefore, being heated. But as a matter of fact that which
primarily moves itself cannot contain either a single part that moves
itself or a number of parts each of which moves itself. For, if the
whole is moved by itself, it must be moved either by some part of
itself or as a whole by itself as a whole. If, then, it is moved in
virtue of some part of it being moved by that part itself, it is this
part that will be the primary self-movent, since, if this part is
separated from the whole, the part will still move itself, but the
whole will do so no longer. If on the other hand the whole is moved
by itself as a whole, it must be accidentally that the parts move
themselves: and therefore, their self-motion not being necessary,
we may take the case of their not being moved by themselves. Therefore
in the whole of the thing we may distinguish that which imparts motion
without itself being moved and that which is moved: for only in this
way is it possible for a thing to be self-moved. Further, if the whole
moves itself we may distinguish in it that which imparts the motion
and that which is moved: so while we say that AB is moved by itself,
we may also say that it is moved by A. And since that which imparts
motion may be either a thing that is moved by something else or a
thing that is unmoved, and that which is moved may be either a thing
that imparts motion to something else or a thing that does not, that
which moves itself must be composed of something that is unmoved but
imparts motion and also of something that is moved but does not necessarily
impart motion but may or may not do so. Thus let A be something that
imparts motion but is unmoved, B something that is moved by A and
moves G, G something that is moved by B but moves nothing (granted
that we eventually arrive at G we may take it that there is only one
intermediate term, though there may be more). Then the whole ABG moves
itself. But if I take away G, AB will move itself, A imparting motion
and B being moved, whereas G will not move itself or in fact be moved
at all. Nor again will BG move itself apart from A: for B imparts
motion only through being moved by something else, not through being
moved by any part of itself. So only AB moves itself. That which moves
itself, therefore, must comprise something that imparts motion but
is unmoved and something that is moved but does not necessarily move
anything else: and each of these two things, or at any rate one of
them, must be in contact with the other. If, then, that which imparts
motion is a continuous substance-that which is moved must of course
be so-it is clear that it is not through some part of the whole being
of such a nature as to be capable of moving itself that the whole
moves itself: it moves itself as a whole, both being moved and imparting
motion through containing a part that imparts motion and a part that
is moved. It does not impart motion as a whole nor is it moved as
a whole: it is A alone that imparts motion and B alone that is moved.
It is not true, further, that G is moved by A, which is impossible.

Here a difficulty arises: if something is taken away from A (supposing
that that which imparts motion but is unmoved is a continuous substance),
or from B the part that is moved, will the remainder of A continue
to impart motion or the remainder of B continue to be moved? If so,
it will not be AB primarily that is moved by itself, since, when something
is taken away from AB, the remainder of AB will still continue to
move itself. Perhaps we may state the case thus: there is nothing
to prevent each of the two parts, or at any rate one of them, that
which is moved, being divisible though actually undivided, so that
if it is divided it will not continue in the possession of the same
capacity: and so there is nothing to prevent self-motion residing
primarily in things that are potentially divisible. 

From what has been said, then, it is evident that that which primarily
imparts motion is unmoved: for, whether the series is closed at once
by that which is in motion but moved by something else deriving its
motion directly from the first unmoved, or whether the motion is derived
from what is in motion but moves itself and stops its own motion,
on both suppositions we have the result that in all cases of things
being in motion that which primarily imparts motion is unmoved.

Part 6

Since there must always be motion without intermission, there must
necessarily be something, one thing or it may be a plurality, that
first imparts motion, and this first movent must be unmoved. Now the
question whether each of the things that are unmoved but impart motion
is eternal is irrelevant to our present argument: but the following
considerations will make it clear that there must necessarily be some
such thing, which, while it has the capacity of moving something else,
is itself unmoved and exempt from all change, which can affect it
neither in an unqualified nor in an accidental sense. Let us suppose,
if any one likes, that in the case of certain things it is possible
for them at different times to be and not to be, without any process
of becoming and perishing (in fact it would seem to be necessary,
if a thing that has not parts at one time is and at another time is
not, that any such thing should without undergoing any process of
change at one time be and at another time not be). And let us further
suppose it possible that some principles that are unmoved but capable
of imparting motion at one time are and at another time are not. Even
so, this cannot be true of all such principles, since there must clearly
be something that causes things that move themselves at one time to
be and at another not to be. For, since nothing that has not parts
can be in motion, that which moves itself must as a whole have magnitude,
though nothing that we have said makes this necessarily true of every
movent. So the fact that some things become and others perish, and
that this is so continuously, cannot be caused by any one of those
things that, though they are unmoved, do not always exist: nor again
can it be caused by any of those which move certain particular things,
while others move other things. The eternity and continuity of the
process cannot be caused either by any one of them singly or by the
sum of them, because this causal relation must be eternal and necessary,
whereas the sum of these movents is infinite and they do not all exist
together. It is clear, then, that though there may be countless instances
of the perishing of some principles that are unmoved but impart motion,
and though many things that move themselves perish and are succeeded
by others that come into being, and though one thing that is unmoved
moves one thing while another moves another, nevertheless there is
something that comprehends them all, and that as something apart from
each one of them, and this it is that is the cause of the fact that
some things are and others are not and of the continuous process of
change: and this causes the motion of the other movents, while they
are the causes of the motion of other things. Motion, then, being
eternal, the first movent, if there is but one, will be eternal also:
if there are more than one, there will be a plurality of such eternal
movents. We ought, however, to suppose that there is one rather than
many, and a finite rather than an infinite number. When the consequences
of either assumption are the same, we should always assume that things
are finite rather than infinite in number, since in things constituted
by nature that which is finite and that which is better ought, if
possible, to be present rather than the reverse: and here it is sufficient
to assume only one movent, the first of unmoved things, which being
eternal will be the principle of motion to everything else.

The following argument also makes it evident that the first movent
must be something that is one and eternal. We have shown that there
must always be motion. That being so, motion must also be continuous,
because what is always is continuous, whereas what is merely in succession
is not continuous. But further, if motion is continuous, it is one:
and it is one only if the movent and the moved that constitute it
are each of them one, since in the event of a thing's being moved
now by one thing and now by another the whole motion will not be continuous
but successive. 

Moreover a conviction that there is a first unmoved something may
be reached not only from the foregoing arguments, but also by considering
again the principles operative in movents. Now it is evident that
among existing things there are some that are sometimes in motion
and sometimes at rest. This fact has served above to make it clear
that it is not true either that all things are in motion or that all
things are at rest or that some things are always at rest and the
remainder always in motion: on this matter proof is supplied by things
that fluctuate between the two and have the capacity of being sometimes
in motion and sometimes at rest. The existence of things of this kind
is clear to all: but we wish to explain also the nature of each of
the other two kinds and show that there are some things that are always
unmoved and some things that are always in motion. In the course of
our argument directed to this end we established the fact that everything
that is in motion is moved by something, and that the movent is either
unmoved or in motion, and that, if it is in motion, it is moved either
by itself or by something else and so on throughout the series: and
so we proceeded to the position that the first principle that directly
causes things that are in motion to be moved is that which moves itself,
and the first principle of the whole series is the unmoved. Further
it is evident from actual observation that there are things that have
the characteristic of moving themselves, e.g. the animal kingdom and
the whole class of living things. This being so, then, the view was
suggested that perhaps it may be possible for motion to come to be
in a thing without having been in existence at all before, because
we see this actually occurring in animals: they are unmoved at one
time and then again they are in motion, as it seems. We must grasp
the fact, therefore, that animals move themselves only with one kind
of motion, and that this is not strictly originated by them. The cause
of it is not derived from the animal itself: it is connected with
other natural motions in animals, which they do not experience through
their own instrumentality, e.g. increase, decrease, and respiration:
these are experienced by every animal while it is at rest and not
in motion in respect of the motion set up by its own agency: here
the motion is caused by the atmosphere and by many things that enter
into the animal: thus in some cases the cause is nourishment: when
it is being digested animals sleep, and when it is being distributed
through the system they awake and move themselves, the first principle
of this motion being thus originally derived from outside. Therefore
animals are not always in continuous motion by their own agency: it
is something else that moves them, itself being in motion and changing
as it comes into relation with each several thing that moves itself.
(Moreover in all these self-moving things the first movent and cause
of their self-motion is itself moved by itself, though in an accidental
sense: that is to say, the body changes its place, so that that which
is in the body changes its place also and is a self-movent through
its exercise of leverage.) Hence we may confidently conclude that
if a thing belongs to the class of unmoved movents that are also themselves
moved accidentally, it is impossible that it should cause continuous
motion. So the necessity that there should be motion continuously
requires that there should be a first movent that is unmoved even
accidentally, if, as we have said, there is to be in the world of
things an unceasing and undying motion, and the world is to remain
permanently self-contained and within the same limits: for if the
first principle is permanent, the universe must also be permanent,
since it is continuous with the first principle. (We must distinguish,
however, between accidental motion of a thing by itself and such motion
by something else, the former being confined to perishable things,
whereas the latter belongs also to certain first principles of heavenly
bodies, of all those, that is to say, that experience more than one
locomotion.) 

And further, if there is always something of this nature, a movent
that is itself unmoved and eternal, then that which is first moved
by it must be eternal. Indeed this is clear also from the consideration
that there would otherwise be no becoming and perishing and no change
of any kind in other things, which require something that is in motion
to move them: for the motion imparted by the unmoved will always be
imparted in the same way and be one and the same, since the unmoved
does not itself change in relation to that which is moved by it. But
that which is moved by something that, though it is in motion, is
moved directly by the unmoved stands in varying relations to the things
that it moves, so that the motion that it causes will not be always
the same: by reason of the fact that it occupies contrary positions
or assumes contrary forms at different times it will produce contrary
motions in each several thing that it moves and will cause it to be
at one time at rest and at another time in motion. 

The foregoing argument, then, has served to clear up the point about
which we raised a difficulty at the outset-why is it that instead
of all things being either in motion or at rest, or some things being
always in motion and the remainder always at rest, there are things
that are sometimes in motion and sometimes not? The cause of this
is now plain: it is because, while some things are moved by an eternal
unmoved movent and are therefore always in motion, other things are
moved by a movent that is in motion and changing, so that they too
must change. But the unmoved movent, as has been said, since it remains
permanently simple and unvarying and in the same state, will cause
motion that is one and simple. 

Part 7

This matter will be made clearer, however, if we start afresh from
another point. We must consider whether it is or is not possible that
there should be a continuous motion, and, if it is possible, which
this motion is, and which is the primary motion: for it is plain that
if there must always be motion, and a particular motion is primary
and continuous, then it is this motion that is imparted by the first
movent, and so it is necessarily one and the same and continuous and
primary. 

Now of the three kinds of motion that there are-motion in respect
of magnitude, motion in respect of affection, and motion in respect
of place-it is this last, which we call locomotion, that must be primary.
This may be shown as follows. It is impossible that there should be
increase without the previous occurrence of alteration: for that which
is increased, although in a sense it is increased by what is like
itself, is in a sense increased by what is unlike itself: thus it
is said that contrary is nourishment to contrary: but growth is effected
only by things becoming like to like. There must be alteration, then,
in that there is this change from contrary to contrary. But the fact
that a thing is altered requires that there should be something that
alters it, something e.g. that makes the potentially hot into the
actually hot: so it is plain that the movent does not maintain a uniform
relation to it but is at one time nearer to and at another farther
from that which is altered: and we cannot have this without locomotion.
If, therefore, there must always be motion, there must also always
be locomotion as the primary motion, and, if there is a primary as
distinguished from a secondary form of locomotion, it must be the
primary form. Again, all affections have their origin in condensation
and rarefaction: thus heavy and light, soft and hard, hot and cold,
are considered to be forms of density and rarity. But condensation
and rarefaction are nothing more than combination and separation,
processes in accordance with which substances are said to become and
perish: and in being combined and separated things must change in
respect of place. And further, when a thing is increased or decreased
its magnitude changes in respect of place. 

Again, there is another point of view from which it will be clearly
seen that locomotion is primary. As in the case of other things so
too in the case of motion the word 'primary' may be used in several
senses. A thing is said to be prior to other things when, if it does
not exist, the others will not exist, whereas it can exist without
the others: and there is also priority in time and priority in perfection
of existence. Let us begin, then, with the first sense. Now there
must be motion continuously, and there may be continuously either
continuous motion or successive motion, the former, however, in a
higher degree than the latter: moreover it is better that it should
be continuous rather than successive motion, and we always assume
the presence in nature of the better, if it be possible: since, then,
continuous motion is possible (this will be proved later: for the
present let us take it for granted), and no other motion can be continuous
except locomotion, locomotion must be primary. For there is no necessity
for the subject of locomotion to be the subject either of increase
or of alteration, nor need it become or perish: on the other hand
there cannot be any one of these processes without the existence of
the continuous motion imparted by the first movent. 

Secondly, locomotion must be primary in time: for this is the only
motion possible for things. It is true indeed that, in the case of
any individual thing that has a becoming, locomotion must be the last
of its motions: for after its becoming it first experiences alteration
and increase, and locomotion is a motion that belongs to such things
only when they are perfected. But there must previously be something
else that is in process of locomotion to be the cause even of the
becoming of things that become, without itself being in process of
becoming, as e.g. the begotten is preceded by what begot it: otherwise
becoming might be thought to be the primary motion on the ground that
the thing must first become. But though this is so in the case of
any individual thing that becomes, nevertheless before anything becomes,
something else must be in motion, not itself becoming but being, and
before this there must again be something else. And since becoming
cannot be primary-for, if it were, everything that is in motion would
be perishable-it is plain that no one of the motions next in order
can be prior to locomotion. By the motions next in order I mean increase
and then alteration, decrease, and perishing. All these are posterior
to becoming: consequently, if not even becoming is prior to locomotion,
then no one of the other processes of change is so either.

Thirdly, that which is in process of becoming appears universally
as something imperfect and proceeding to a first principle: and so
what is posterior in the order of becoming is prior in the order of
nature. Now all things that go through the process of becoming acquire
locomotion last. It is this that accounts for the fact that some living
things, e.g. plants and many kinds of animals, owing to lack of the
requisite organ, are entirely without motion, whereas others acquire
it in the course of their being perfected. Therefore, if the degree
in which things possess locomotion corresponds to the degree in which
they have realized their natural development, then this motion must
be prior to all others in respect of perfection of existence: and
not only for this reason but also because a thing that is in motion
loses its essential character less in the process of locomotion than
in any other kind of motion: it is the only motion that does not involve
a change of being in the sense in which there is a change in quality
when a thing is altered and a change in quantity when a thing is increased
or decreased. Above all it is plain that this motion, motion in respect
of place, is what is in the strictest sense produced by that which
moves itself; but it is the self-movent that we declare to be the
first principle of things that are moved and impart motion and the
primary source to which things that are in motion are to be referred.

It is clear, then, from the foregoing arguments that locomotion is
the primary motion. We have now to show which kind of locomotion is
primary. The same process of reasoning will also make clear at the
same time the truth of the assumption we have made both now and at
a previous stage that it is possible that there should be a motion
that is continuous and eternal. Now it is clear from the following
considerations that no other than locomotion can be continuous. Every
other motion and change is from an opposite to an opposite: thus for
the processes of becoming and perishing the limits are the existent
and the non-existent, for alteration the various pairs of contrary
affections, and for increase and decrease either greatness and smallness
or perfection and imperfection of magnitude: and changes to the respective
contraries are contrary changes. Now a thing that is undergoing any
particular kind of motion, but though previously existent has not
always undergone it, must previously have been at rest so far as that
motion is concerned. It is clear, then, that for the changing thing
the contraries will be states of rest. And we have a similar result
in the case of changes that are not motions: for becoming and perishing,
whether regarded simply as such without qualification or as affecting
something in particular, are opposites: therefore provided it is impossible
for a thing to undergo opposite changes at the same time, the change
will not be continuous, but a period of time will intervene between
the opposite processes. The question whether these contradictory changes
are contraries or not makes no difference, provided only it is impossible
for them both to be present to the same thing at the same time: the
point is of no importance to the argument. Nor does it matter if the
thing need not rest in the contradictory state, or if there is no
state of rest as a contrary to the process of change: it may be true
that the non-existent is not at rest, and that perishing is a process
to the non-existent. All that matters is the intervention of a time:
it is this that prevents the change from being continuous: so, too,
in our previous instances the important thing was not the relation
of contrariety but the impossibility of the two processes being present
to a thing at the same time. And there is no need to be disturbed
by the fact that on this showing there may be more than one contrary
to the same thing, that a particular motion will be contrary both
to rest and to motion in the contrary direction. We have only to grasp
the fact that a particular motion is in a sense the opposite both
of a state of rest and of the contrary motion, in the same way as
that which is of equal or standard measure is the opposite both of
that which surpasses it and of that which it surpasses, and that it
is impossible for the opposite motions or changes to be present to
a thing at the same time. Furthermore, in the case of becoming and
perishing it would seem to be an utterly absurd thing if as soon as
anything has become it must necessarily perish and cannot continue
to exist for any time: and, if this is true of becoming and perishing,
we have fair grounds for inferring the same to be true of the other
kinds of change, since it would be in the natural order of things
that they should be uniform in this respect. 

Part 8

Let us now proceed to maintain that it is possible that there should
be an infinite motion that is single and continuous, and that this
motion is rotatory motion. The motion of everything that is in process
of locomotion is either rotatory or rectilinear or a compound of the
two: consequently, if one of the former two is not continuous, that
which is composed of them both cannot be continuous either. Now it
is plain that if the locomotion of a thing is rectilinear and finite
it is not continuous locomotion: for the thing must turn back, and
that which turns back in a straight line undergoes two contrary locomotions,
since, so far as motion in respect of place is concerned, upward motion
is the contrary of downward motion, forward motion of backward motion,
and motion to the left of motion to the right, these being the pairs
of contraries in the sphere of place. But we have already defined
single and continuous motion to be motion of a single thing in a single
period of time and operating within a sphere admitting of no further
specific differentiation (for we have three things to consider, first
that which is in motion, e.g. a man or a god, secondly the 'when'
of the motion, that is to say, the time, and thirdly the sphere within
which it operates, which may be either place or affection or essential
form or magnitude): and contraries are specifically not one and the
same but distinct: and within the sphere of place we have the above-mentioned
distinctions. Moreover we have an indication that motion from A to
B is the contrary of motion from B to A in the fact that, if they
occur at the same time, they arrest and stop each other. And the same
is true in the case of a circle: the motion from A towards B is the
contrary of the motion from A towards G: for even if they are continuous
and there is no turning back they arrest each other, because contraries
annihilate or obstruct one another. On the other hand lateral motion
is not the contrary of upward motion. But what shows most clearly
that rectilinear motion cannot be continuous is the fact that turning
back necessarily implies coming to a stand, not only when it is a
straight line that is traversed, but also in the case of locomotion
in a circle (which is not the same thing as rotatory locomotion: for,
when a thing merely traverses a circle, it may either proceed on its
course without a break or turn back again when it has reached the
same point from which it started). We may assure ourselves of the
necessity of this coming to a stand not only on the strength of observation,
but also on theoretical grounds. We may start as follows: we have
three points, starting-point, middle-point, and finishing-point, of
which the middle-point in virtue of the relations in which it stands
severally to the other two is both a starting-point and a finishing-point,
and though numerically one is theoretically two. We have further the
distinction between the potential and the actual. So in the straight
line in question any one of the points lying between the two extremes
is potentially a middle-point: but it is not actually so unless that
which is in motion divides the line by coming to a stand at that point
and beginning its motion again: thus the middle-point becomes both
a starting-point and a goal, the starting-point of the latter part
and the finishing-point of the first part of the motion. This is the
case e.g. when A in the course of its locomotion comes to a stand
at B and starts again towards G: but when its motion is continuous
A cannot either have come to be or have ceased to be at the point
B: it can only have been there at the moment of passing, its passage
not being contained within any period of time except the whole of
which the particular moment is a dividing-point. To maintain that
it has come to be and ceased to be there will involve the consequence
that A in the course of its locomotion will always be coming to a
stand: for it is impossible that A should simultaneously have come
to be at B and ceased to be there, so that the two things must have
happened at different points of time, and therefore there will be
the intervening period of time: consequently A will be in a state
of rest at B, and similarly at all other points, since the same reasoning
holds good in every case. When to A, that which is in process of locomotion,
B, the middle-point, serves both as a finishing-point and as a starting-point
for its motion, A must come to a stand at B, because it makes it two
just as one might do in thought. However, the point A is the real
starting-point at which the moving body has ceased to be, and it is
at G that it has really come to be when its course is finished and
it comes to a stand. So this is how we must meet the difficulty that
then arises, which is as follows. Suppose the line E is equal to the
line Z, that A proceeds in continuous locomotion from the extreme
point of E to G, and that, at the moment when A is at the point B,
D is proceeding in uniform locomotion and with the same velocity as
A from the extremity of Z to H: then, says the argument, D will have
reached H before A has reached G for that which makes an earlier start
and departure must make an earlier arrival: the reason, then, for
the late arrival of A is that it has not simultaneously come to be
and ceased to be at B: otherwise it will not arrive later: for this
to happen it will be necessary that it should come to a stand there.
Therefore we must not hold that there was a moment when A came to
be at B and that at the same moment D was in motion from the extremity
of Z: for the fact of A's having come to be at B will involve the
fact of its also ceasing to be there, and the two events will not
be simultaneous, whereas the truth is that A is at B at a sectional
point of time and does not occupy time there. In this case, therefore,
where the motion of a thing is continuous, it is impossible to use
this form of expression. On the other hand in the case of a thing
that turns back in its course we must do so. For suppose H in the
course of its locomotion proceeds to D and then turns back and proceeds
downwards again: then the extreme point D has served as finishing-point
and as starting-point for it, one point thus serving as two: therefore
H must have come to a stand there: it cannot have come to be at D
and departed from D simultaneously, for in that case it would simultaneously
be there and not be there at the same moment. And here we cannot apply
the argument used to solve the difficulty stated above: we cannot
argue that H is at D at a sectional point of time and has not come
to be or ceased to be there. For here the goal that is reached is
necessarily one that is actually, not potentially, existent. Now the
point in the middle is potential: but this one is actual, and regarded
from below it is a finishing-point, while regarded from above it is
a starting-point, so that it stands in these same two respective relations
to the two motions. Therefore that which turns back in traversing
a rectilinear course must in so doing come to a stand. Consequently
there cannot be a continuous rectilinear motion that is eternal.

The same method should also be adopted in replying to those who ask,
in the terms of Zeno's argument, whether we admit that before any
distance can be traversed half the distance must be traversed, that
these half-distances are infinite in number, and that it is impossible
to traverse distances infinite in number-or some on the lines of this
same argument put the questions in another form, and would have us
grant that in the time during which a motion is in progress it should
be possible to reckon a half-motion before the whole for every half-distance
that we get, so that we have the result that when the whole distance
is traversed we have reckoned an infinite number, which is admittedly
impossible. Now when we first discussed the question of motion we
put forward a solution of this difficulty turning on the fact that
the period of time occupied in traversing the distance contains within
itself an infinite number of units: there is no absurdity, we said,
in supposing the traversing of infinite distances in infinite time,
and the element of infinity is present in the time no less than in
the distance. But, although this solution is adequate as a reply to
the questioner (the question asked being whether it is possible in
a finite time to traverse or reckon an infinite number of units),
nevertheless as an account of the fact and explanation of its true
nature it is inadequate. For suppose the distance to be left out of
account and the question asked to be no longer whether it is possible
in a finite time to traverse an infinite number of distances, and
suppose that the inquiry is made to refer to the time taken by itself
(for the time contains an infinite number of divisions): then this
solution will no longer be adequate, and we must apply the truth that
we enunciated in our recent discussion, stating it in the following
way. In the act of dividing the continuous distance into two halves
one point is treated as two, since we make it a starting-point and
a finishing-point: and this same result is also produced by the act
of reckoning halves as well as by the act of dividing into halves.
But if divisions are made in this way, neither the distance nor the
motion will be continuous: for motion if it is to be continuous must
relate to what is continuous: and though what is continuous contains
an infinite number of halves, they are not actual but potential halves.
If the halves are made actual, we shall get not a continuous but an
intermittent motion. In the case of reckoning the halves, it is clear
that this result follows: for then one point must be reckoned as two:
it will be the finishing-point of the one half and the starting-point
of the other, if we reckon not the one continuous whole but the two
halves. Therefore to the question whether it is possible to pass through
an infinite number of units either of time or of distance we must
reply that in a sense it is and in a sense it is not. If the units
are actual, it is not possible: if they are potential, it is possible.
For in the course of a continuous motion the traveller has traversed
an infinite number of units in an accidental sense but not in an unqualified
sense: for though it is an accidental characteristic of the distance
to be an infinite number of half-distances, this is not its real and
essential character. It is also plain that unless we hold that the
point of time that divides earlier from later always belongs only
to the later so far as the thing is concerned, we shall be involved
in the consequence that the same thing is at the same moment existent
and not existent, and that a thing is not existent at the moment when
it has become. It is true that the point is common to both times,
the earlier as well as the later, and that, while numerically one
and the same, it is theoretically not so, being the finishing-point
of the one and the starting-point of the other: but so far as the
thing is concerned it belongs to the later stage of what happens to
it. Let us suppose a time ABG and a thing D, D being white in the
time A and not-white in the time B. Then D is at the moment G white
and not-white: for if we were right in saying that it is white during
the whole time A, it is true to call it white at any moment of A,
and not-white in B, and G is in both A and B. We must not allow, therefore,
that it is white in the whole of A, but must say that it is so in
all of it except the last moment G. G belongs already to the later
period, and if in the whole of A not-white was in process of becoming
and white of perishing, at G the process is complete. And so G is
the first moment at which it is true to call the thing white or not
white respectively. Otherwise a thing may be non-existent at the moment
when it has become and existent at the moment when it has perished:
or else it must be possible for a thing at the same time to be white
and not white and in fact to be existent and non-existent. Further,
if anything that exists after having been previously non-existent
must become existent and does not exist when it is becoming, time
cannot be divisible into time-atoms. For suppose that D was becoming
white in the time A and that at another time B, a time-atom consecutive
with the last atom of A, D has already become white and so is white
at that moment: then, inasmuch as in the time A it was becoming white
and so was not white and at the moment B it is white, there must have
been a becoming between A and B and therefore also a time in which
the becoming took place. On the other hand, those who deny atoms of
time (as we do) are not affected by this argument: according to them
D has become and so is white at the last point of the actual time
in which it was becoming white: and this point has no other point
consecutive with or in succession to it, whereas time-atoms are conceived
as successive. Moreover it is clear that if D was becoming white in
the whole time A, the time occupied by it in having become white in
addition to having been in process of becoming white is no more than
all that it occupied in the mere process of becoming white.

These and such-like, then, are the arguments for our conclusion that
derive cogency from the fact that they have a special bearing on the
point at issue. If we look at the question from the point of view
of general theory, the same result would also appear to be indicated
by the following arguments. Everything whose motion is continuous
must, on arriving at any point in the course of its locomotion, have
been previously also in process of locomotion to that point, if it
is not forced out of its path by anything: e.g. on arriving at B a
thing must also have been in process of locomotion to B, and that
not merely when it was near to B, but from the moment of its starting
on its course, since there can be, no reason for its being so at any
particular stage rather than at an earlier one. So, too, in the case
of the other kinds of motion. Now we are to suppose that a thing proceeds
in locomotion from A to G and that at the moment of its arrival at
G the continuity of its motion is unbroken and will remain so until
it has arrived back at A. Then when it is undergoing locomotion from
A to G it is at the same time undergoing also its locomotion to A
from G: consequently it is simultaneously undergoing two contrary
motions, since the two motions that follow the same straight line
are contrary to each other. With this consequence there also follows
another: we have a thing that is in process of change from a position
in which it has not yet been: so, inasmuch as this is impossible,
the thing must come to a stand at G. Therefore the motion is not a
single motion, since motion that is interrupted by stationariness
is not single. 

Further, the following argument will serve better to make this point
clear universally in respect of every kind of motion. If the motion
undergone by that which is in motion is always one of those already
enumerated, and the state of rest that it undergoes is one of those
that are the opposites of the motions (for we found that there could
be no other besides these), and moreover that which is undergoing
but does not always undergo a particular motion (by this I mean one
of the various specifically distinct motions, not some particular
part of the whole motion) must have been previously undergoing the
state of rest that is the opposite of the motion, the state of rest
being privation of motion; then, inasmuch as the two motions that
follow the same straight line are contrary motions, and it is impossible
for a thing to undergo simultaneously two contrary motions, that which
is undergoing locomotion from A to G cannot also simultaneously be
undergoing locomotion from G to A: and since the latter locomotion
is not simultaneous with the former but is still to be undergone,
before it is undergone there must occur a state of rest at G: for
this, as we found, is the state of rest that is the opposite of the
motion from G. The foregoing argument, then, makes it plain that the
motion in question is not continuous. 

Our next argument has a more special bearing than the foregoing on
the point at issue. We will suppose that there has occurred in something
simultaneously a perishing of not-white and a becoming of white. Then
if the alteration to white and from white is a continuous process
and the white does not remain any time, there must have occurred simultaneously
a perishing of not-white, a becoming of white, and a becoming of not-white:
for the time of the three will be the same. 

Again, from the continuity of the time in which the motion takes place
we cannot infer continuity in the motion, but only successiveness:
in fact, how could contraries, e.g. whiteness and blackness, meet
in the same extreme point? 

On the other hand, in motion on a circular line we shall find singleness
and continuity: for here we are met by no impossible consequence:
that which is in motion from A will in virtue of the same direction
of energy be simultaneously in motion to A (since it is in motion
to the point at which it will finally arrive), and yet will not be
undergoing two contrary or opposite motions: for a motion to a point
and a motion from that point are not always contraries or opposites:
they are contraries only if they are on the same straight line (for
then they are contrary to one another in respect of place, as e.g.
the two motions along the diameter of the circle, since the ends of
this are at the greatest possible distance from one another), and
they are opposites only if they are along the same line. Therefore
in the case we are now considering there is nothing to prevent the
motion being continuous and free from all intermission: for rotatory
motion is motion of a thing from its place to its place, whereas rectilinear
motion is motion from its place to another place. 

Moreover the progress of rotatory motion is never localized within
certain fixed limits, whereas that of rectilinear motion repeatedly
is so. Now a motion that is always shifting its ground from moment
to moment can be continuous: but a motion that is repeatedly localized
within certain fixed limits cannot be so, since then the same thing
would have to undergo simultaneously two opposite motions. So, too,
there cannot be continuous motion in a semicircle or in any other
arc of a circle, since here also the same ground must be traversed
repeatedly and two contrary processes of change must occur. The reason
is that in these motions the starting-point and the termination do
not coincide, whereas in motion over a circle they do coincide, and
so this is the only perfect motion. 

This differentiation also provides another means of showing that the
other kinds of motion cannot be continuous either: for in all of them
we find that there is the same ground to be traversed repeatedly;
thus in alteration there are the intermediate stages of the process,
and in quantitative change there are the intervening degrees of magnitude:
and in becoming and perishing the same thing is true. It makes no
difference whether we take the intermediate stages of the process
to be few or many, or whether we add or subtract one: for in either
case we find that there is still the same ground to be traversed repeatedly.
Moreover it is plain from what has been said that those physicists
who assert that all sensible things are always in motion are wrong:
for their motion must be one or other of the motions just mentioned:
in fact they mostly conceive it as alteration (things are always in
flux and decay, they say), and they go so far as to speak even of
becoming and perishing as a process of alteration. On the other hand,
our argument has enabled us to assert the fact, applying universally
to all motions, that no motion admits of continuity except rotatory
motion: consequently neither alteration nor increase admits of continuity.
We need now say no more in support of the position that there is no
process of change that admits of infinity or continuity except rotatory
locomotion. 

Part 9

It can now be shown plainly that rotation is the primary locomotion.
Every locomotion, as we said before, is either rotatory or rectilinear
or a compound of the two: and the two former must be prior to the
last, since they are the elements of which the latter consists. Moreover
rotatory locomotion is prior to rectilinear locomotion, because it
is more simple and complete, which may be shown as follows. The straight
line traversed in rectilinear motion cannot be infinite: for there
is no such thing as an infinite straight line; and even if there were,
it would not be traversed by anything in motion: for the impossible
does not happen and it is impossible to traverse an infinite distance.
On the other hand rectilinear motion on a finite straight line is
if it turns back a composite motion, in fact two motions, while if
it does not turn back it is incomplete and perishable: and in the
order of nature, of definition, and of time alike the complete is
prior to the incomplete and the imperishable to the perishable. Again,
a motion that admits of being eternal is prior to one that does not.
Now rotatory motion can be eternal: but no other motion, whether locomotion
or motion of any other kind, can be so, since in all of them rest
must occur and with the occurrence of rest the motion has perished.
Moreover the result at which we have arrived, that rotatory motion
is single and continuous, and rectilinear motion is not, is a reasonable
one. In rectilinear motion we have a definite starting-point, finishing-point,
middle-point, which all have their place in it in such a way that
there is a point from which that which is in motion can be said to
start and a point at which it can be said to finish its course (for
when anything is at the limits of its course, whether at the starting-point
or at the finishing-point, it must be in a state of rest). On the
other hand in circular motion there are no such definite points: for
why should any one point on the line be a limit rather than any other?
Any one point as much as any other is alike starting-point, middle-point,
and finishing-point, so that we can say of certain things both that
they are always and that they never are at a starting-point and at
a finishing-point (so that a revolving sphere, while it is in motion,
is also in a sense at rest, for it continues to occupy the same place).
The reason of this is that in this case all these characteristics
belong to the centre: that is to say, the centre is alike starting-point,
middle-point, and finishing-point of the space traversed; consequently
since this point is not a point on the circular line, there is no
point at which that which is in process of locomotion can be in a
state of rest as having traversed its course, because in its locomotion
it is proceeding always about a central point and not to an extreme
point: therefore it remains still, and the whole is in a sense always
at rest as well as continuously in motion. Our next point gives a
convertible result: on the one hand, because rotation is the measure
of motions it must be the primary motion (for all things are measured
by what is primary): on the other hand, because rotation is the primary
motion it is the measure of all other motions. Again, rotatory motion
is also the only motion that admits of being regular. In rectilinear
locomotion the motion of things in leaving the starting-point is not
uniform with their motion in approaching the finishing-point, since
the velocity of a thing always increases proportionately as it removes
itself farther from its position of rest: on the other hand rotatory
motion is the only motion whose course is naturally such that it has
no starting-point or finishing-point in itself but is determined from
elsewhere. 

As to locomotion being the primary motion, this is a truth that is
attested by all who have ever made mention of motion in their theories:
they all assign their first principles of motion to things that impart
motion of this kind. Thus 'separation' and 'combination' are motions
in respect of place, and the motion imparted by 'Love' and 'Strife'
takes these forms, the latter 'separating' and the former 'combining'.
Anaxagoras, too, says that 'Mind', his first movent, 'separates'.
Similarly those who assert no cause of this kind but say that 'void'
accounts for motion-they also hold that the motion of natural substance
is motion in respect of place: for their motion that is accounted
for by 'void' is locomotion, and its sphere of operation may be said
to be place. Moreover they are of opinion that the primary substances
are not subject to any of the other motions, though the things that
are compounds of these substances are so subject: the processes of
increase and decrease and alteration, they say, are effects of the
'combination' and 'separation' of atoms. It is the same, too, with
those who make out that the becoming or perishing of a thing is accounted
for by 'density' or 'rarity': for it is by 'combination' and 'separation'
that the place of these things in their systems is determined. Moreover
to these we may add those who make Soul the cause of motion: for they
say that things that undergo motion have as their first principle
'that which moves itself': and when animals and all living things
move themselves, the motion is motion in respect of place. Finally
it is to be noted that we say that a thing 'is in motion' in the strict
sense of the term only when its motion is motion in respect of place:
if a thing is in process of increase or decrease or is undergoing
some alteration while remaining at rest in the same place, we say
that it is in motion in some particular respect: we do not say that
it 'is in motion' without qualification. 

Our present position, then, is this: We have argued that there always
was motion and always will be motion throughout all time, and we have
explained what is the first principle of this eternal motion: we have
explained further which is the primary motion and which is the only
motion that can be eternal: and we have pronounced the first movent
to be unmoved. 

Part 10

We have now to assert that the first movent must be without parts
and without magnitude, beginning with the establishment of the premisses
on which this conclusion depends. 

One of these premisses is that nothing finite can cause motion during
an infinite time. We have three things, the movent, the moved, and
thirdly that in which the motion takes place, namely the time: and
these are either all infinite or all finite or partly-that is to say
two of them or one of them-finite and partly infinite. Let A be the
movement, B the moved, and G the infinite time. Now let us suppose
that D moves E, a part of B. Then the time occupied by this motion
cannot be equal to G: for the greater the amount moved, the longer
the time occupied. It follows that the time Z is not infinite. Now
we see that by continuing to add to D, I shall use up A and by continuing
to add to E, I shall use up B: but I shall not use up the time by
continually subtracting a corresponding amount from it, because it
is infinite. Consequently the duration of the part of G which is occupied
by all A in moving the whole of B, will be finite. Therefore a finite
thing cannot impart to anything an infinite motion. It is clear, then,
that it is impossible for the finite to cause motion during an infinite
time. 

It has now to be shown that in no case is it possible for an infinite
force to reside in a finite magnitude. This can be shown as follows:
we take it for granted that the greater force is always that which
in less time than another does an equal amount of work when engaged
in any activity-in heating, for example, or sweetening or throwing;
in fact, in causing any kind of motion. Then that on which the forces
act must be affected to some extent by our supposed finite magnitude
possessing an infinite force as well as by anything else, in fact
to a greater extent than by anything else, since the infinite force
is greater than any other. But then there cannot be any time in which
its action could take place. Suppose that A is the time occupied by
the infinite power in the performance of an act of heating or pushing,
and that AB is the time occupied by a finite power in the performance
of the same act: then by adding to the latter another finite power
and continually increasing the magnitude of the power so added I shall
at some time or other reach a point at which the finite power has
completed the motive act in the time A: for by continual addition
to a finite magnitude I must arrive at a magnitude that exceeds any
assigned limit, and in the same way by continual subtraction I must
arrive at one that falls short of any assigned limit. So we get the
result that the finite force will occupy the same amount of time in
performing the motive act as the infinite force. But this is impossible.
Therefore nothing finite can possess an infinite force. So it is also
impossible for a finite force to reside in an infinite magnitude.
It is true that a greater force can reside in a lesser magnitude:
but the superiority of any such greater force can be still greater
if the magnitude in which it resides is greater. Now let AB be an
infinite magnitude. Then BG possesses a certain force that occupies
a certain time, let us say the time Z in moving D. Now if I take a
magnitude twice as great at BG, the time occupied by this magnitude
in moving D will be half of EZ (assuming this to be the proportion):
so we may call this time ZH. That being so, by continually taking
a greater magnitude in this way I shall never arrive at the full AB,
whereas I shall always be getting a lesser fraction of the time given.
Therefore the force must be infinite, since it exceeds any finite
force. Moreover the time occupied by the action of any finite force
must also be finite: for if a given force moves something in a certain
time, a greater force will do so in a lesser time, but still a definite
time, in inverse proportion. But a force must always be infinite-just
as a number or a magnitude is-if it exceeds all definite limits. This
point may also be proved in another way-by taking a finite magnitude
in which there resides a force the same in kind as that which resides
in the infinite magnitude, so that this force will be a measure of
the finite force residing in the infinite magnitude. 

It is plain, then, from the foregoing arguments that it is impossible
for an infinite force to reside in a finite magnitude or for a finite
force to reside in an infinite magnitude. But before proceeding to
our conclusion it will be well to discuss a difficulty that arises
in connexion with locomotion. If everything that is in motion with
the exception of things that move themselves is moved by something
else, how is it that some things, e.g. things thrown, continue to
be in motion when their movent is no longer in contact with them?
If we say that the movent in such cases moves something else at the
same time, that the thrower e.g. also moves the air, and that this
in being moved is also a movent, then it would be no more possible
for this second thing than for the original thing to be in motion
when the original movent is not in contact with it or moving it: all
the things moved would have to be in motion simultaneously and also
to have ceased simultaneously to be in motion when the original movent
ceases to move them, even if, like the magnet, it makes that which
it has moved capable of being a movent. Therefore, while we must accept
this explanation to the extent of saying that the original movent
gives the power of being a movent either to air or to water or to
something else of the kind, naturally adapted for imparting and undergoing
motion, we must say further that this thing does not cease simultaneously
to impart motion and to undergo motion: it ceases to be in motion
at the moment when its movent ceases to move it, but it still remains
a movent, and so it causes something else consecutive with it to be
in motion, and of this again the same may be said. The motion begins
to cease when the motive force produced in one member of the consecutive
series is at each stage less than that possessed by the preceding
member, and it finally ceases when one member no longer causes the
next member to be a movent but only causes it to be in motion. The
motion of these last two-of the one as movent and of the other as
moved-must cease simultaneously, and with this the whole motion ceases.
Now the things in which this motion is produced are things that admit
of being sometimes in motion and sometimes at rest, and the motion
is not continuous but only appears so: for it is motion of things
that are either successive or in contact, there being not one movent
but a number of movents consecutive with one another: and so motion
of this kind takes place in air and water. Some say that it is 'mutual
replacement': but we must recognize that the difficulty raised cannot
be solved otherwise than in the way we have described. So far as they
are affected by 'mutual replacement', all the members of the series
are moved and impart motion simultaneously, so that their motions
also cease simultaneously: but our present problem concerns the appearance
of continuous motion in a single thing, and therefore, since it cannot
be moved throughout its motion by the same movent, the question is,
what moves it? 

Resuming our main argument, we proceed from the positions that there
must be continuous motion in the world of things, that this is a single
motion, that a single motion must be a motion of a magnitude (for
that which is without magnitude cannot be in motion), and that the
magnitude must be a single magnitude moved by a single movent (for
otherwise there will not be continuous motion but a consecutive series
of separate motions), and that if the movement is a single thing,
it is either itself in motion or itself unmoved: if, then, it is in
motion, it will have to be subject to the same conditions as that
which it moves, that is to say it will itself be in process of change
and in being so will also have to be moved by something: so we have
a series that must come to an end, and a point will be reached at
which motion is imparted by something that is unmoved. Thus we have
a movent that has no need to change along with that which it moves
but will be able to cause motion always (for the causing of motion
under these conditions involves no effort): and this motion alone
is regular, or at least it is so in a higher degree than any other,
since the movent is never subject to any change. So, too, in order
that the motion may continue to be of the same character, the moved
must not be subject to change in respect of its relation to the movent.
Moreover the movent must occupy either the centre or the circumference,
since these are the first principles from which a sphere is derived.
But the things nearest the movent are those whose motion is quickest,
and in this case it is the motion of the circumference that is the
quickest: therefore the movent occupies the circumference.

There is a further difficulty in supposing it to be possible for anything
that is in motion to cause motion continuously and not merely in the
way in which it is caused by something repeatedly pushing (in which
case the continuity amounts to no more than successiveness). Such
a movent must either itself continue to push or pull or perform both
these actions, or else the action must be taken up by something else
and be passed on from one movent to another (the process that we described
before as occurring in the case of things thrown, since the air or
the water, being divisible, is a movent only in virtue of the fact
that different parts of the air are moved one after another): and
in either case the motion cannot be a single motion, but only a consecutive
series of motions. The only continuous motion, then, is that which
is caused by the unmoved movent: and this motion is continuous because
the movent remains always invariable, so that its relation to that
which it moves remains also invariable and continuous. 

Now that these points are settled, it is clear that the first unmoved
movent cannot have any magnitude. For if it has magnitude, this must
be either a finite or an infinite magnitude. Now we have already'proved
in our course on Physics that there cannot be an infinite magnitude:
and we have now proved that it is impossible for a finite magnitude
to have an infinite force, and also that it is impossible for a thing
to be moved by a finite magnitude during an infinite time. But the
first movent causes a motion that is eternal and does cause it during
an infinite time. It is clear, therefore, that the first movent is
indivisible and is without parts and without magnitude. 

THE END

% chapter physics (end)