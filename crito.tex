\chapter{Crito} % (fold)
\label{cha:crito}



Translated by Benjamin Jowett

Persons of the Dialogue
SOCRATES
CRITO

Scene
The Prison of Socrates.
----------------------------------------------------------------------

Socrates. WHY have you come at this hour, Crito? it must be quite
early. 

Crito. Yes, certainly. 

Soc. What is the exact time? 

Cr. The dawn is breaking. 

Soc. I wonder the keeper of the prison would let you in.

Cr. He knows me because I often come, Socrates; moreover. I have done
him a kindness. 

Soc. And are you only just come? 

Cr. No, I came some time ago. 

Soc. Then why did you sit and say nothing, instead of awakening me
at once? 

Cr. Why, indeed, Socrates, I myself would rather not have all this
sleeplessness and sorrow. But I have been wondering at your peaceful
slumbers, and that was the reason why I did not awaken you, because
I wanted you to be out of pain. I have always thought you happy in
the calmness of your temperament; but never did I see the like of
the easy, cheerful way in which you bear this calamity. 

Soc. Why, Crito, when a man has reached my age he ought not to be
repining at the prospect of death. 

Cr. And yet other old men find themselves in similar misfortunes,
and age does not prevent them from repining. 

Soc. That may be. But you have not told me why you come at this early
hour. 

Cr. I come to bring you a message which is sad and painful; not, as
I believe, to yourself but to all of us who are your friends, and
saddest of all to me. 

Soc. What! I suppose that the ship has come from Delos, on the arrival
of which I am to die? 

Cr. No, the ship has not actually arrived, but she will probably be
here to-day, as persons who have come from Sunium tell me that they
have left her there; and therefore to-morrow, Socrates, will be the
last day of your life. 

Soc. Very well, Crito; if such is the will of God, I am willing; but
my belief is that there will be a delay of a day. 

Cr. Why do you say this? 

Soc. I will tell you. I am to die on the day after the arrival of
the ship? 

Cr. Yes; that is what the authorities say. 

Soc. But I do not think that the ship will be here until to-morrow;
this I gather from a vision which I had last night, or rather only
just now, when you fortunately allowed me to sleep. 

Cr. And what was the nature of the vision? 

Soc. There came to me the likeness of a woman, fair and comely, clothed
in white raiment, who called to me and said: O Socrates-

``The third day hence, to Phthia shalt thou go." 

Cr. What a singular dream, Socrates! 

Soc. There can be no doubt about the meaning Crito, I think.

Cr. Yes: the meaning is only too clear. But, O! my beloved Socrates,
let me entreat you once more to take my advice and escape. For if
you die I shall not only lose a friend who can never be replaced,
but there is another evil: people who do not know you and me will
believe that I might have saved you if I had been willing to give
money, but that I did not care. Now, can there be a worse disgrace
than this- that I should be thought to value money more than the life
of a friend? For the many will not be persuaded that I wanted you
to escape, and that you refused. 

Soc. But why, my dear Crito, should we care about the opinion of the
many? Good men, and they are the only persons who are worth considering,
will think of these things truly as they happened. 

Cr. But do you see. Socrates, that the opinion of the many must be
regarded, as is evident in your own case, because they can do the
very greatest evil to anyone who has lost their good opinion?

Soc. I only wish, Crito, that they could; for then they could also
do the greatest good, and that would be well. But the truth is, that
they can do neither good nor evil: they cannot make a man wise or
make him foolish; and whatever they do is the result of chance.

Cr. Well, I will not dispute about that; but please to tell me, Socrates,
whether you are not acting out of regard to me and your other friends:
are you not afraid that if you escape hence we may get into trouble
with the informers for having stolen you away, and lose either the
whole or a great part of our property; or that even a worse evil may
happen to us? Now, if this is your fear, be at ease; for in order
to save you, we ought surely to run this or even a greater risk; be
persuaded, then, and do as I say. 

Soc. Yes, Crito, that is one fear which you mention, but by no means
the only one. 

Cr. Fear not. There are persons who at no great cost are willing to
save you and bring you out of prison; and as for the informers, you
may observe that they are far from being exorbitant in their demands;
a little money will satisfy them. My means, which, as I am sure, are
ample, are at your service, and if you have a scruple about spending
all mine, here are strangers who will give you the use of theirs;
and one of them, Simmias the Theban, has brought a sum of money for
this very purpose; and Cebes and many others are willing to spend
their money too. I say, therefore, do not on that account hesitate
about making your escape, and do not say, as you did in the court,
that you will have a difficulty in knowing what to do with yourself
if you escape. For men will love you in other places to which you
may go, and not in Athens only; there are friends of mine in Thessaly,
if you like to go to them, who will value and protect you, and no
Thessalian will give you any trouble. Nor can I think that you are
justified, Socrates, in betraying your own life when you might be
saved; this is playing into the hands of your enemies and destroyers;
and moreover I should say that you were betraying your children; for
you might bring them up and educate them; instead of which you go
away and leave them, and they will have to take their chance; and
if they do not meet with the usual fate of orphans, there will be
small thanks to you. No man should bring children into the world who
is unwilling to persevere to the end in their nurture and education.
But you are choosing the easier part, as I think, not the better and
manlier, which would rather have become one who professes virtue in
all his actions, like yourself. And, indeed, I am ashamed not only
of you, but of us who are your friends, when I reflect that this entire
business of yours will be attributed to our want of courage. The trial
need never have come on, or might have been brought to another issue;
and the end of all, which is the crowning absurdity, will seem to
have been permitted by us, through cowardice and baseness, who might
have saved you, as you might have saved yourself, if we had been good
for anything (for there was no difficulty in escaping); and we did
not see how disgraceful, Socrates, and also miserable all this will
be to us as well as to you. Make your mind up then, or rather have
your mind already made up, for the time of deliberation is over, and
there is only one thing to be done, which must be done, if at all,
this very night, and which any delay will render all but impossible;
I beseech you therefore, Socrates, to be persuaded by me, and to do
as I say. 

Soc. Dear Crito, your zeal is invaluable, if a right one; but if wrong,
the greater the zeal the greater the evil; and therefore we ought
to consider whether these things shall be done or not. For I am and
always have been one of those natures who must be guided by reason,
whatever the reason may be which upon reflection appears to me to
be the best; and now that this fortune has come upon me, I cannot
put away the reasons which I have before given: the principles which
I have hitherto honored and revered I still honor, and unless we can
find other and better principles on the instant, I am certain not
to agree with you; no, not even if the power of the multitude could
inflict many more imprisonments, confiscations, deaths, frightening
us like children with hobgoblin terrors. But what will be the fairest
way of considering the question? Shall I return to your old argument
about the opinions of men, some of which are to be regarded, and others,
as we were saying, are not to be regarded? Now were we right in maintaining
this before I was condemned? And has the argument which was once good
now proved to be talk for the sake of talking; in fact an amusement
only, and altogether vanity? That is what I want to consider with
your help, Crito: whether, under my present circumstances, the argument
appears to be in any way different or not; and is to be allowed by
me or disallowed. That argument, which, as I believe, is maintained
by many who assume to be authorities, was to the effect, as I was
saying, that the opinions of some men are to be regarded, and of other
men not to be regarded. Now you, Crito, are a disinterested person
who are not going to die to-morrow- at least, there is no human probability
of this, and you are therefore not liable to be deceived by the circumstances
in which you are placed. Tell me, then, whether I am right in saying
that some opinions, and the opinions of some men only, are to be valued,
and other opinions, and the opinions of other men, are not to be valued.
I ask you whether I was right in maintaining this? 

Cr. Certainly. 

Soc. The good are to be regarded, and not the bad? 

Cr. Yes. 

Soc. And the opinions of the wise are good, and the opinions of the
unwise are evil? 

Cr. Certainly. 

Soc. And what was said about another matter? Was the disciple in gymnastics
supposed to attend to the praise and blame and opinion of every man,
or of one man only- his physician or trainer, whoever that was?

Cr. Of one man only. 

Soc. And he ought to fear the censure and welcome the praise of that
one only, and not of the many? 

Cr. That is clear. 

Soc. And he ought to live and train, and eat and drink in the way
which seems good to his single master who has understanding, rather
than according to the opinion of all other men put together?

Cr. True. 

Soc. And if he disobeys and disregards the opinion and approval of
the one, and regards the opinion of the many who have no understanding,
will he not suffer evil? 

Cr. Certainly he will. 

Soc. And what will the evil be, whither tending and what affcting,
in the disobedient person? 

Cr. Clearly, affecting the body; that is what is destroyed by the
evil. 

Soc. Very good; and is not this true, Crito, of other things which
we need not separately enumerate? In the matter of just and unjust,
fair and foul, good and evil, which are the subjects of our present
consultation, ought we to follow the opinion of the many and to fear
them; or the opinion of the one man who has understanding, and whom
we ought to fear and reverence more than all the rest of the world:
and whom deserting we shall destroy and injure that principle in us
which may be assumed to be improved by justice and deteriorated by
injustice; is there not such a principle? 

Cr. Certainly there is, Socrates. 

Soc. Take a parallel instance; if, acting under the advice of men
who have no understanding, we destroy that which is improvable by
health and deteriorated by disease- when that has been destroyed,
I say, would life be worth having? And that is- the body?

Cr. Yes. 

Soc. Could we live, having an evil and corrupted body? 

Cr. Certainly not. 

Soc. And will life be worth having, if that higher part of man be
depraved, which is improved by justice and deteriorated by injustice?
Do we suppose that principle, whatever it may be in man, which has
to do with justice and injustice, to be inferior to the body?

Cr. Certainly not. 

Soc. More honored, then? 

Cr. Far more honored. 

Soc. Then, my friend, we must not regard what the many say of us:
but what he, the one man who has understanding of just and unjust,
will say, and what the truth will say. And therefore you begin in
error when you suggest that we should regard the opinion of the many
about just and unjust, good and evil, honorable and dishonorable.
Well, someone will say, ``But the many can kill us." 

Cr. Yes, Socrates; that will clearly be the answer. 

Soc. That is true; but still I find with surprise that the old argument
is, as I conceive, unshaken as ever. And I should like to know Whether
I may say the same of another proposition- that not life, but a good
life, is to be chiefly valued? 

Cr. Yes, that also remains. 

Soc. And a good life is equivalent to a just and honorable one- that
holds also? 

Cr. Yes, that holds. 

Soc. From these premises I proceed to argue the question whether I
ought or ought not to try to escape without the consent of the Athenians:
and if I am clearly right in escaping, then I will make the attempt;
but if not, I will abstain. The other considerations which you mention,
of money and loss of character, and the duty of educating children,
are, I fear, only the doctrines of the multitude, who would be as
ready to call people to life, if they were able, as they are to put
them to death- and with as little reason. But now, since the argument
has thus far prevailed, the only question which remains to be considered
is, whether we shall do rightly either in escaping or in suffering
others to aid in our escape and paying them in money and thanks, or
whether we shan not do rightly; and if the latter, then death or any
other calamity which may ensue on my remaining here must not be allowed
to enter into the calculation. 

Cr. I think that you are right, Socrates; how then shall we proceed?

Soc. Let us consider the matter together, and do you either refute
me if you can, and I will be convinced; or else cease, my dear friend,
from repeating to me that I ought to escape against the wishes of
the Athenians: for I am extremely desirous to be persuaded by you,
but not against my own better judgment. And now please to consider
my first position, and do your best to answer me. 

Cr. I will do my best. 

Soc. Are we to say that we are never intentionally to do wrong, or
that in one way we ought and in another way we ought not to do wrong,
or is doing wrong always evil and dishonorable, as I was just now
saying, and as has been already acknowledged by us? Are all our former
admissions which were made within a few days to be thrown away? And
have we, at our age, been earnestly discoursing with one another all
our life long only to discover that we are no better than children?
Or are we to rest assured, in spite of the opinion of the many, and
in spite of consequences whether better or worse, of the truth of
what was then said, that injustice is always an evil and dishonor
to him who acts unjustly? Shall we affirm that? 

Cr. Yes. 

Soc. Then we must do no wrong? 

Cr. Certainly not. 

Soc. Nor when injured injure in return, as the many imagine; for we
must injure no one at all? 

Cr. Clearly not. 

Soc. Again, Crito, may we do evil? 

Cr. Surely not, Socrates. 

Soc. And what of doing evil in return for evil, which is the morality
of the many-is that just or not? 

Cr. Not just. 

Soc. For doing evil to another is the same as injuring him?

Cr. Very true. 

Soc. Then we ought not to retaliate or render evil for evil to anyone,
whatever evil we may have suffered from him. But I would have you
consider, Crito, whether you really mean what you are saying. For
this opinion has never been held, and never will be held, by any considerable
number of persons; and those who are agreed and those who are not
agreed upon this point have no common ground, and can only despise
one another, when they see how widely they differ. Tell me, then,
whether you agree with and assent to my first principle, that neither
injury nor retaliation nor warding off evil by evil is ever right.
And shall that be the premise of our agreement? Or do you decline
and dissent from this? For this has been of old and is still my opinion;
but, if you are of another opinion, let me hear what you have to say.
If, however, you remain of the same mind as formerly, I will proceed
to the next step. 

Cr. You may proceed, for I have not changed my mind. 

Soc. Then I will proceed to the next step, which may be put in the
form of a question: Ought a man to do what he admits to be right,
or ought he to betray the right? 

Cr. He ought to do what he thinks right. 

Soc. But if this is true, what is the application? In leaving the
prison against the will of the Athenians, do I wrong any? or rather
do I not wrong those whom I ought least to wrong? Do I not desert
the principles which were acknowledged by us to be just? What do you
say? 

Cr. I cannot tell, Socrates, for I do not know. 

Soc. Then consider the matter in this way: Imagine that I am about
to play truant (you may call the proceeding by any name which you
like), and the laws and the government come and interrogate me: ``Tell
us, Socrates," they say; ``what are you about? are you going by an
act of yours to overturn us- the laws and the whole State, as far
as in you lies? Do you imagine that a State can subsist and not be
overthrown, in which the decisions of law have no power, but are set
aside and overthrown by individuals?" What will be our answer, Crito,
to these and the like words? Anyone, and especially a clever rhetorician,
will have a good deal to urge about the evil of setting aside the
law which requires a sentence to be carried out; and we might reply,
"Yes; but the State has injured us and given an unjust sentence."
Suppose I say that? 

Cr. Very good, Socrates. 
Soc. ``And was that our agreement with you?" the law would sar, ``or
were you to abide by the sentence of the State?" And if I were to
express astonishment at their saying this, the law would probably
add: ``Answer, Socrates, instead of opening your eyes: you are in the
habit of asking and answering questions. Tell us what complaint you
have to make against us which justifies you in attempting to destroy
us and the State? In the first place did we not bring you into existence?
Your father married your mother by our aid and begat you. Say whether
you have any objection to urge against those of us who regulate marriage?"
None, I should reply. ``Or against those of us who regulate the system
of nurture and education of children in which you were trained? Were
not the laws, who have the charge of this, right in commanding your
father to train you in music and gymnastic?" Right, I should reply.
``Well, then, since you were brought into the world and nurtured and
educated by us, can you deny in the first place that you are our child
and slave, as your fathers were before you? And if this is true you
are not on equal terms with us; nor can you think that you have a
right to do to us what we are doing to you. Would you have any right
to strike or revile or do any other evil to a father or to your master,
if you had one, when you have been struck or reviled by him, or received
some other evil at his hands?- you would not say this? And because
we think right to destroy you, do you think that you have any right
to destroy us in return, and your country as far as in you lies? And
will you, O professor of true virtue, say that you are justified in
this? Has a philosopher like you failed to discover that our country
is more to be valued and higher and holier far than mother or father
or any ancestor, and more to be regarded in the eyes of the gods and
of men of understanding? also to be soothed, and gently and reverently
entreated when angry, even more than a father, and if not persuaded,
obeyed? And when we are punished by her, whether with imprisonment
or stripes, the punishment is to be endured in silence; and if she
leads us to wounds or death in battle, thither we follow as is right;
neither may anyone yield or retreat or leave his rank, but whether
in battle or in a court of law, or in any other place, he must do
what his city and his country order him; or he must change their view
of what is just: and if he may do no violence to his father or mother,
much less may he do violence to his country." What answer shall we
make to this, Crito? Do the laws speak truly, or do they not?

Cr. I think that they do. 

Soc. Then the laws will say: ``Consider, Socrates, if this is true,
that in your present attempt you are going to do us wrong. For, after
having brought you into the world, and nurtured and educated you,
and given you and every other citizen a share in every good that we
had to give, we further proclaim and give the right to every Athenian,
that if he does not like us when he has come of age and has seen the
ways of the city, and made our acquaintance, he may go where he pleases
and take his goods with him; and none of us laws will forbid him or
interfere with him. Any of you who does not like us and the city,
and who wants to go to a colony or to any other city, may go where
he likes, and take his goods with him. But he who has experience of
the manner in which we order justice and administer the State, and
still remains, has entered into an implied contract that he will do
as we command him. And he who disobeys us is, as we maintain, thrice
wrong: first, because in disobeying us he is disobeying his parents;
secondly, because we are the authors of his education; thirdly, because
he has made an agreement with us that he will duly obey our commands;
and he neither obeys them nor convinces us that our commands are wrong;
and we do not rudely impose them, but give him the alternative of
obeying or convincing us; that is what we offer and he does neither.
These are the sort of accusations to which, as we were saying, you,
Socrates, will be exposed if you accomplish your intentions; you,
above all other Athenians." Suppose I ask, why is this? they will
justly retort upon me that I above all other men have acknowledged
the agreement. ``There is clear proof," they will say, ``Socrates, that
we and the city were not displeasing to you. Of all Athenians you
have been the most constant resident in the city, which, as you never
leave, you may be supposed to love. For you never went out of the
city either to see the games, except once when you went to the Isthmus,
or to any other place unless when you were on military service; nor
did you travel as other men do. Nor had you any curiosity to know
other States or their laws: your affections did not go beyond us and
our State; we were your especial favorites, and you acquiesced in
our government of you; and this is the State in which you begat your
children, which is a proof of your satisfaction. Moreover, you might,
if you had liked, have fixed the penalty at banishment in the course
of the trial-the State which refuses to let you go now would have
let you go then. But you pretended that you preferred death to exile,
and that you were not grieved at death. And now you have forgotten
these fine sentiments, and pay no respect to us, the laws, of whom
you are the destroyer; and are doing what only a miserable slave would
do, running away and turning your back upon the compacts and agreements
which you made as a citizen. And first of all answer this very question:
Are we right in saying that you agreed to be governed according to
us in deed, and not in word only? Is that true or not?" How shall
we answer that, Crito? Must we not agree? 

Cr. There is no help, Socrates. 

Soc. Then will they not say: ``You, Socrates, are breaking the covenants
and agreements which you made with us at your leisure, not in any
haste or under any compulsion or deception, but having had seventy
years to think of them, during which time you were at liberty to leave
the city, if we were not to your mind, or if our covenants appeared
to you to be unfair. You had your choice, and might have gone either
to Lacedaemon or Crete, which you often praise for their good government,
or to some other Hellenic or foreign State. Whereas you, above all
other Athenians, seemed to be so fond of the State, or, in other words,
of us her laws (for who would like a State that has no laws?), that
you never stirred out of her: the halt, the blind, the maimed, were
not more stationary in her than you were. And now you run away and
forsake your agreements. Not so, Socrates, if you will take our advice;
do not make yourself ridiculous by escaping out of the city.

"For just consider, if you transgress and err in this sort of way,
what good will you do, either to yourself or to your friends? That
your friends will be driven into exile and deprived of citizenship,
or will lose their property, is tolerably certain; and you yourself,
if you fly to one of the neighboring cities, as, for example, Thebes
or Megara, both of which are well-governed cities, will come to them
as an enemy, Socrates, and their government will be against you, and
all patriotic citizens will cast an evil eye upon you as a subverter
of the laws, and you will confirm in the minds of the judges the justice
of their own condemnation of you. For he who is a corrupter of the
laws is more than likely to be corrupter of the young and foolish
portion of mankind. Will you then flee from well-ordered cities and
virtuous men? and is existence worth having on these terms? Or will
you go to them without shame, and talk to them, Socrates? And what
will you say to them? What you say here about virtue and justice and
institutions and laws being the best things among men? Would that
be decent of you? Surely not. But if you go away from well-governed
States to Crito's friends in Thessaly, where there is great disorder
and license, they will be charmed to have the tale of your escape
from prison, set off with ludicrous particulars of the manner in which
you were wrapped in a goatskin or some other disguise, and metamorphosed
as the fashion of runaways is- that is very likely; but will there
be no one to remind you that in your old age you violated the most
sacred laws from a miserable desire of a little more life? Perhaps
not, if you keep them in a good temper; but if they are out of temper
you will hear many degrading things; you will live, but how?- as the
flatterer of all men, and the servant of all men; and doing what?-
eating and drinking in Thessaly, having gone abroad in order that
you may get a dinner. And where will be your fine sentiments about
justice and virtue then? Say that you wish to live for the sake of
your children, that you may bring them up and educate them- will you
take them into Thessaly and deprive them of Athenian citizenship?
Is that the benefit which you would confer upon them? Or are you under
the impression that they will be better cared for and educated here
if you are still alive, although absent from them; for that your friends
will take care of them? Do you fancy that if you are an inhabitant
of Thessaly they will take care of them, and if you are an inhabitant
of the other world they will not take care of them? Nay; but if they
who call themselves friends are truly friends, they surely will.

``Listen, then, Socrates, to us who have brought you up. Think not
of life and children first, and of justice afterwards, but of justice
first, that you may be justified before the princes of the world below.
For neither will you nor any that belong to you be happier or holier
or juster in this life, or happier in another, if you do as Crito
bids. Now you depart in innocence, a sufferer and not a doer of evil;
a victim, not of the laws, but of men. But if you go forth, returning
evil for evil, and injury for injury, breaking the covenants and agreements
which you have made with us, and wronging those whom you ought least
to wrong, that is to say, yourself, your friends, your country, and
us, we shall be angry with you while you live, and our brethren, the
laws in the world below, will receive you as an enemy; for they will
know that you have done your best to destroy us. Listen, then, to
us and not to Crito." 

This is the voice which I seem to hear murmuring in my ears, like
the sound of the flute in the ears of the mystic; that voice, I say,
is humming in my ears, and prevents me from hearing any other. And
I know that anything more which you will say will be in vain. Yet
speak, if you have anything to say. 

Cr. I have nothing to say, Socrates. 

Soc. Then let me follow the intimations of the will of God.

THE END

% chapter crito (end)