\chapter{Categories} % (fold)
\label{cha:categories}





The Categories


By

Aristotle


Translated by E. M. Edghill



Section 1

Part 1

Things are said to be named 'equivocally' when, though they have a
common name, the definition corresponding with the name differs for
each. Thus, a real man and a figure in a picture can both lay claim to
the name 'animal'; yet these are equivocally so named, for, though they
have a common name, the definition corresponding with the name differs
for each. For should any one define in what sense each is an animal,
his definition in the one case will be appropriate to that case only.

On the other hand, things are said to be named 'univocally' which have
both the name and the definition answering to the name in common. A man
and an ox are both 'animal', and these are univocally so named,
inasmuch as not only the name, but also the definition, is the same in
both cases: for if a man should state in what sense each is an animal,
the statement in the one case would be identical with that in the other.

Things are said to be named 'derivatively', which derive their name
from some other name, but differ from it in termination. Thus the
grammarian derives his name from the word 'grammar', and the courageous
man from the word 'courage'.



Part 2

Forms of speech are either simple or composite. Examples of the latter
are such expressions as 'the man runs', 'the man wins'; of the former
'man', 'ox', 'runs', 'wins'.

Of things themselves some are predicable of a subject, and are never
present in a subject. Thus 'man' is predicable of the individual man,
and is never present in a subject.

By being 'present in a subject' I do not mean present as parts are
present in a whole, but being incapable of existence apart from the
said subject.

Some things, again, are present in a subject, but are never predicable
of a subject. For instance, a certain point of grammatical knowledge is
present in the mind, but is not predicable of any subject; or again, a
certain whiteness may be present in the body (for colour requires a
material basis), yet it is never predicable of anything.

Other things, again, are both predicable of a subject and present in a
subject. Thus while knowledge is present in the human mind, it is
predicable of grammar.

There is, lastly, a class of things which are neither present in a
subject nor predicable of a subject, such as the individual man or the
individual horse. But, to speak more generally, that which is
individual and has the character of a unit is never predicable of a
subject. Yet in some cases there is nothing to prevent such being
present in a subject. Thus a certain point of grammatical knowledge is
present in a subject.



Part 3

When one thing is predicated of another, all that which is predicable
of the predicate will be predicable also of the subject. Thus, 'man' is
predicated of the individual man; but 'animal' is predicated of 'man';
it will, therefore, be predicable of the individual man also: for the
individual man is both 'man' and 'animal'.

If genera are different and co-ordinate, their differentiae are
themselves different in kind. Take as an instance the genus 'animal'
and the genus 'knowledge'. 'With feet', 'two-footed', 'winged',
'aquatic', are differentiae of 'animal'; the species of knowledge are
not distinguished by the same differentiae. One species of knowledge
does not differ from another in being 'two-footed'.

But where one genus is subordinate to another, there is nothing to
prevent their having the same differentiae: for the greater class is
predicated of the lesser, so that all the differentiae of the predicate
will be differentiae also of the subject.



Part 4

Expressions which are in no way composite signify substance, quantity,
quality, relation, place, time, position, state, action, or affection.
To sketch my meaning roughly, examples of substance are 'man' or 'the
horse', of quantity, such terms as 'two cubits long' or 'three cubits
long', of quality, such attributes as 'white', 'grammatical'. 'Double',
'half', 'greater', fall under the category of relation; 'in a the
market place', 'in the Lyceum', under that of place; 'yesterday', 'last
year', under that of time. 'Lying', 'sitting', are terms indicating
position, 'shod', 'armed', state; 'to lance', 'to cauterize', action;
'to be lanced', 'to be cauterized', affection.

No one of these terms, in and by itself, involves an affirmation; it is
by the combination of such terms that positive or negative statements
arise. For every assertion must, as is admitted, be either true or
false, whereas expressions which are not in any way composite such as
'man', 'white', 'runs', 'wins', cannot be either true or false.



Part 5

Substance, in the truest and primary and most definite sense of the
word, is that which is neither predicable of a subject nor present in a
subject; for instance, the individual man or horse. But in a secondary
sense those things are called substances within which, as species, the
primary substances are included; also those which, as genera, include
the species. For instance, the individual man is included in the
species 'man', and the genus to which the species belongs is 'animal';
these, therefore-that is to say, the species 'man' and the genus
'animal,-are termed secondary substances.

It is plain from what has been said that both the name and the
definition of the predicate must be predicable of the subject. For
instance, 'man' is predicted of the individual man. Now in this case
the name of the species man' is applied to the individual, for we use
the term 'man' in describing the individual; and the definition of
'man' will also be predicated of the individual man, for the individual
man is both man and animal. Thus, both the name and the definition of
the species are predicable of the individual.

With regard, on the other hand, to those things which are present in a
subject, it is generally the case that neither their name nor their
definition is predicable of that in which they are present. Though,
however, the definition is never predicable, there is nothing in
certain cases to prevent the name being used. For instance, 'white'
being present in a body is predicated of that in which it is present,
for a body is called white: the definition, however, of the colour
white' is never predicable of the body.

Everything except primary substances is either predicable of a primary
substance or present in a primary substance. This becomes evident by
reference to particular instances which occur. 'Animal' is predicated
of the species 'man', therefore of the individual man, for if there
were no individual man of whom it could be predicated, it could not be
predicated of the species 'man' at all. Again, colour is present in
body, therefore in individual bodies, for if there were no individual
body in which it was present, it could not be present in body at all.
Thus everything except primary substances is either predicated of
primary substances, or is present in them, and if these last did not
exist, it would be impossible for anything else to exist.

Of secondary substances, the species is more truly substance than the
genus, being more nearly related to primary substance. For if any one
should render an account of what a primary substance is, he would
render a more instructive account, and one more proper to the subject,
by stating the species than by stating the genus. Thus, he would give a
more instructive account of an individual man by stating that he was
man than by stating that he was animal, for the former description is
peculiar to the individual in a greater degree, while the latter is too
general. Again, the man who gives an account of the nature of an
individual tree will give a more instructive account by mentioning the
species 'tree' than by mentioning the genus 'plant'.

Moreover, primary substances are most properly called substances in
virtue of the fact that they are the entities which underlie everything
else, and that everything else is either predicated of them or present
in them. Now the same relation which subsists between primary substance
and everything else subsists also between the species and the genus:
for the species is to the genus as subject is to predicate, since the
genus is predicated of the species, whereas the species cannot be
predicated of the genus. Thus we have a second ground for asserting
that the species is more truly substance than the genus.

Of species themselves, except in the case of such as are genera, no one
is more truly substance than another. We should not give a more
appropriate account of the individual man by stating the species to
which he belonged, than we should of an individual horse by adopting
the same method of definition. In the same way, of primary substances,
no one is more truly substance than another; an individual man is not
more truly substance than an individual ox.

It is, then, with good reason that of all that remains, when we exclude
primary substances, we concede to species and genera alone the name
'secondary substance', for these alone of all the predicates convey a
knowledge of primary substance. For it is by stating the species or the
genus that we appropriately define any individual man; and we shall
make our definition more exact by stating the former than by stating
the latter. All other things that we state, such as that he is white,
that he runs, and so on, are irrelevant to the definition. Thus it is
just that these alone, apart from primary substances, should be called
substances.

Further, primary substances are most properly so called, because they
underlie and are the subjects of everything else. Now the same relation
that subsists between primary substance and everything else subsists
also between the species and the genus to which the primary substance
belongs, on the one hand, and every attribute which is not included
within these, on the other. For these are the subjects of all such. If
we call an individual man 'skilled in grammar', the predicate is
applicable also to the species and to the genus to which he belongs.
This law holds good in all cases.

It is a common characteristic of all substance that it is never present
in a subject. For primary substance is neither present in a subject nor
predicated of a subject; while, with regard to secondary substances, it
is clear from the following arguments (apart from others) that they are
not present in a subject. For 'man' is predicated of the individual
man, but is not present in any subject: for manhood is not present in
the individual man. In the same way, 'animal' is also predicated of the
individual man, but is not present in him. Again, when a thing is
present in a subject, though the name may quite well be applied to that
in which it is present, the definition cannot be applied. Yet of
secondary substances, not only the name, but also the definition,
applies to the subject: we should use both the definition of the
species and that of the genus with reference to the individual man.
Thus substance cannot be present in a subject.

Yet this is not peculiar to substance, for it is also the case that
differentiae cannot be present in subjects. The characteristics
'terrestrial' and 'two-footed' are predicated of the species 'man', but
not present in it. For they are not in man. Moreover, the definition of
the differentia may be predicated of that of which the differentia
itself is predicated. For instance, if the characteristic 'terrestrial'
is predicated of the species 'man', the definition also of that
characteristic may be used to form the predicate of the species 'man':
for 'man' is terrestrial.

The fact that the parts of substances appear to be present in the
whole, as in a subject, should not make us apprehensive lest we should
have to admit that such parts are not substances: for in explaining the
phrase 'being present in a subject', we stated' that we meant
'otherwise than as parts in a whole'.

It is the mark of substances and of differentiae that, in all
propositions of which they form the predicate, they are predicated
univocally. For all such propositions have for their subject either the
individual or the species. It is true that, inasmuch as primary
substance is not predicable of anything, it can never form the
predicate of any proposition. But of secondary substances, the species
is predicated of the individual, the genus both of the species and of
the individual. Similarly the differentiae are predicated of the
species and of the individuals. Moreover, the definition of the species
and that of the genus are applicable to the primary substance, and that
of the genus to the species. For all that is predicated of the
predicate will be predicated also of the subject. Similarly, the
definition of the differentiae will be applicable to the species and to
the individuals. But it was stated above that the word 'univocal' was
applied to those things which had both name and definition in common.
It is, therefore, established that in every proposition, of which
either substance or a differentia forms the predicate, these are
predicated univocally.

All substance appears to signify that which is individual. In the case
of primary substance this is indisputably true, for the thing is a
unit. In the case of secondary substances, when we speak, for instance,
of 'man' or 'animal', our form of speech gives the impression that we
are here also indicating that which is individual, but the impression
is not strictly true; for a secondary substance is not an individual,
but a class with a certain qualification; for it is not one and single
as a primary substance is; the words 'man', 'animal', are predicable of
more than one subject.

Yet species and genus do not merely indicate quality, like the term
'white'; 'white' indicates quality and nothing further, but species and
genus determine the quality with reference to a substance: they signify
substance qualitatively differentiated. The determinate qualification
covers a larger field in the case of the genus that in that of the
species: he who uses the word 'animal' is herein using a word of wider
extension than he who uses the word 'man'.

Another mark of substance is that it has no contrary. What could be the
contrary of any primary substance, such as the individual man or
animal? It has none. Nor can the species or the genus have a contrary.
Yet this characteristic is not peculiar to substance, but is true of
many other things, such as quantity. There is nothing that forms the
contrary of 'two cubits long' or of 'three cubits long', or of 'ten',
or of any such term. A man may contend that 'much' is the contrary of
'little', or 'great' of 'small', but of definite quantitative terms no
contrary exists.

Substance, again, does not appear to admit of variation of degree. I do
not mean by this that one substance cannot be more or less truly
substance than another, for it has already been stated' that this is
the case; but that no single substance admits of varying degrees within
itself. For instance, one particular substance, 'man', cannot be more
or less man either than himself at some other time or than some other
man. One man cannot be more man than another, as that which is white
may be more or less white than some other white object, or as that
which is beautiful may be more or less beautiful than some other
beautiful object. The same quality, moreover, is said to subsist in a
thing in varying degrees at different times. A body, being white, is
said to be whiter at one time than it was before, or, being warm, is
said to be warmer or less warm than at some other time. But substance
is not said to be more or less that which it is: a man is not more
truly a man at one time than he was before, nor is anything, if it is
substance, more or less what it is. Substance, then, does not admit of
variation of degree.

The most distinctive mark of substance appears to be that, while
remaining numerically one and the same, it is capable of admitting
contrary qualities. From among things other than substance, we should
find ourselves unable to bring forward any which possessed this mark.
Thus, one and the same colour cannot be white and black. Nor can the
same one action be good and bad: this law holds good with everything
that is not substance. But one and the selfsame substance, while
retaining its identity, is yet capable of admitting contrary qualities.
The same individual person is at one time white, at another black, at
one time warm, at another cold, at one time good, at another bad. This
capacity is found nowhere else, though it might be maintained that a
statement or opinion was an exception to the rule. The same statement,
it is agreed, can be both true and false. For if the statement 'he is
sitting' is true, yet, when the person in question has risen, the same
statement will be false. The same applies to opinions. For if any one
thinks truly that a person is sitting, yet, when that person has risen,
this same opinion, if still held, will be false. Yet although this
exception may be allowed, there is, nevertheless, a difference in the
manner in which the thing takes place. It is by themselves changing
that substances admit contrary qualities. It is thus that that which
was hot becomes cold, for it has entered into a different state.
Similarly that which was white becomes black, and that which was bad
good, by a process of change; and in the same way in all other cases it
is by changing that substances are capable of admitting contrary
qualities. But statements and opinions themselves remain unaltered in
all respects: it is by the alteration in the facts of the case that the
contrary quality comes to be theirs. The statement 'he is sitting'
remains unaltered, but it is at one time true, at another false,
according to circumstances. What has been said of statements applies
also to opinions. Thus, in respect of the manner in which the thing
takes place, it is the peculiar mark of substance that it should be
capable of admitting contrary qualities; for it is by itself changing
that it does so.

If, then, a man should make this exception and contend that statements
and opinions are capable of admitting contrary qualities, his
contention is unsound. For statements and opinions are said to have
this capacity, not because they themselves undergo modification, but
because this modification occurs in the case of something else. The
truth or falsity of a statement depends on facts, and not on any power
on the part of the statement itself of admitting contrary qualities. In
short, there is nothing which can alter the nature of statements and
opinions. As, then, no change takes place in themselves, these cannot
be said to be capable of admitting contrary qualities.

But it is by reason of the modification which takes place within the
substance itself that a substance is said to be capable of admitting
contrary qualities; for a substance admits within itself either disease
or health, whiteness or blackness. It is in this sense that it is said
to be capable of admitting contrary qualities.

To sum up, it is a distinctive mark of substance, that, while remaining
numerically one and the same, it is capable of admitting contrary
qualities, the modification taking place through a change in the
substance itself.

Let these remarks suffice on the subject of substance.



Part 6

Quantity is either discrete or continuous. Moreover, some quantities
are such that each part of the whole has a relative position to the
other parts: others have within them no such relation of part to part.

Instances of discrete quantities are number and speech; of continuous,
lines, surfaces, solids, and, besides these, time and place.

In the case of the parts of a number, there is no common boundary at
which they join. For example: two fives make ten, but the two fives
have no common boundary, but are separate; the parts three and seven
also do not join at any boundary. Nor, to generalize, would it ever be
possible in the case of number that there should be a common boundary
among the parts; they are always separate. Number, therefore, is a
discrete quantity.

The same is true of speech. That speech is a quantity is evident: for
it is measured in long and short syllables. I mean here that speech
which is vocal. Moreover, it is a discrete quantity for its parts have
no common boundary. There is no common boundary at which the syllables
join, but each is separate and distinct from the rest.

A line, on the other hand, is a continuous quantity, for it is possible
to find a common boundary at which its parts join. In the case of the
line, this common boundary is the point; in the case of the plane, it
is the line: for the parts of the plane have also a common boundary.
Similarly you can find a common boundary in the case of the parts of a
solid, namely either a line or a plane.

Space and time also belong to this class of quantities. Time, past,
present, and future, forms a continuous whole. Space, likewise, is a
continuous quantity; for the parts of a solid occupy a certain space,
and these have a common boundary; it follows that the parts of space
also, which are occupied by the parts of the solid, have the same
common boundary as the parts of the solid. Thus, not only time, but
space also, is a continuous quantity, for its parts have a common
boundary.

Quantities consist either of parts which bear a relative position each
to each, or of parts which do not. The parts of a line bear a relative
position to each other, for each lies somewhere, and it would be
possible to distinguish each, and to state the position of each on the
plane and to explain to what sort of part among the rest each was
contiguous. Similarly the parts of a plane have position, for it could
similarly be stated what was the position of each and what sort of
parts were contiguous. The same is true with regard to the solid and to
space. But it would be impossible to show that the arts of a number had
a relative position each to each, or a particular position, or to state
what parts were contiguous. Nor could this be done in the case of time,
for none of the parts of time has an abiding existence, and that which
does not abide can hardly have position. It would be better to say that
such parts had a relative order, in virtue of one being prior to
another. Similarly with number: in counting, 'one' is prior to 'two',
and 'two' to 'three', and thus the parts of number may be said to
possess a relative order, though it would be impossible to discover any
distinct position for each. This holds good also in the case of speech.
None of its parts has an abiding existence: when once a syllable is
pronounced, it is not possible to retain it, so that, naturally, as the
parts do not abide, they cannot have position. Thus, some quantities
consist of parts which have position, and some of those which have not.

Strictly speaking, only the things which I have mentioned belong to the
category of quantity: everything else that is called quantitative is a
quantity in a secondary sense. It is because we have in mind some one
of these quantities, properly so called, that we apply quantitative
terms to other things. We speak of what is white as large, because the
surface over which the white extends is large; we speak of an action or
a process as lengthy, because the time covered is long; these things
cannot in their own right claim the quantitative epithet. For instance,
should any one explain how long an action was, his statement would be
made in terms of the time taken, to the effect that it lasted a year,
or something of that sort. In the same way, he would explain the size
of a white object in terms of surface, for he would state the area
which it covered. Thus the things already mentioned, and these alone,
are in their intrinsic nature quantities; nothing else can claim the
name in its own right, but, if at all, only in a secondary sense.

Quantities have no contraries. In the case of definite quantities this
is obvious; thus, there is nothing that is the contrary of 'two cubits
long' or of 'three cubits long', or of a surface, or of any such
quantities. A man might, indeed, argue that 'much' was the contrary of
'little', and 'great' of 'small'. But these are not quantitative, but
relative; things are not great or small absolutely, they are so called
rather as the result of an act of comparison. For instance, a mountain
is called small, a grain large, in virtue of the fact that the latter
is greater than others of its kind, the former less. Thus there is a
reference here to an external standard, for if the terms 'great' and
'small' were used absolutely, a mountain would never be called small or
a grain large. Again, we say that there are many people in a village,
and few in Athens, although those in the city are many times as
numerous as those in the village: or we say that a house has many in
it, and a theatre few, though those in the theatre far outnumber those
in the house. The terms 'two cubits long, 'three cubits long,' and so
on indicate quantity, the terms 'great' and 'small' indicate relation,
for they have reference to an external standard. It is, therefore,
plain that these are to be classed as relative.

Again, whether we define them as quantitative or not, they have no
contraries: for how can there be a contrary of an attribute which is
not to be apprehended in or by itself, but only by reference to
something external? Again, if 'great' and 'small' are contraries, it
will come about that the same subject can admit contrary qualities at
one and the same time, and that things will themselves be contrary to
themselves. For it happens at times that the same thing is both small
and great. For the same thing may be small in comparison with one
thing, and great in comparison with another, so that the same thing
comes to be both small and great at one and the same time, and is of
such a nature as to admit contrary qualities at one and the same
moment. Yet it was agreed, when substance was being discussed, that
nothing admits contrary qualities at one and the same moment. For
though substance is capable of admitting contrary qualities, yet no one
is at the same time both sick and healthy, nothing is at the same time
both white and black. Nor is there anything which is qualified in
contrary ways at one and the same time.

Moreover, if these were contraries, they would themselves be contrary
to themselves. For if 'great' is the contrary of 'small', and the same
thing is both great and small at the same time, then 'small' or 'great'
is the contrary of itself. But this is impossible. The term 'great',
therefore, is not the contrary of the term 'small', nor 'much' of
'little'. And even though a man should call these terms not relative
but quantitative, they would not have contraries.

It is in the case of space that quantity most plausibly appears to
admit of a contrary. For men define the term 'above' as the contrary of
'below', when it is the region at the centre they mean by 'below'; and
this is so, because nothing is farther from the extremities of the
universe than the region at the centre. Indeed, it seems that in
defining contraries of every kind men have recourse to a spatial
metaphor, for they say that those things are contraries which, within
the same class, are separated by the greatest possible distance.

Quantity does not, it appears, admit of variation of degree. One thing
cannot be two cubits long in a greater degree than another. Similarly
with regard to number: what is 'three' is not more truly three than
what is 'five' is five; nor is one set of three more truly three than
another set. Again, one period of time is not said to be more truly
time than another. Nor is there any other kind of quantity, of all that
have been mentioned, with regard to which variation of degree can be
predicated. The category of quantity, therefore, does not admit of
variation of degree.

The most distinctive mark of quantity is that equality and inequality
are predicated of it. Each of the aforesaid quantities is said to be
equal or unequal. For instance, one solid is said to be equal or
unequal to another; number, too, and time can have these terms applied
to them, indeed can all those kinds of quantity that have been
mentioned.

That which is not a quantity can by no means, it would seem, be termed
equal or unequal to anything else. One particular disposition or one
particular quality, such as whiteness, is by no means compared with
another in terms of equality and inequality but rather in terms of
similarity. Thus it is the distinctive mark of quantity that it can be
called equal and unequal.



Section 2


Part 7

Those things are called relative, which, being either said to be of
something else or related to something else, are explained by reference
to that other thing. For instance, the word 'superior' is explained by
reference to something else, for it is superiority over something else
that is meant. Similarly, the expression 'double' has this external
reference, for it is the double of something else that is meant. So it
is with everything else of this kind. There are, moreover, other
relatives, e.g. habit, disposition, perception, knowledge, and
attitude. The significance of all these is explained by a reference to
something else and in no other way. Thus, a habit is a habit of
something, knowledge is knowledge of something, attitude is the
attitude of something. So it is with all other relatives that have been
mentioned. Those terms, then, are called relative, the nature of which
is explained by reference to something else, the preposition 'of' or
some other preposition being used to indicate the relation. Thus, one
mountain is called great in comparison with son with another; for the
mountain claims this attribute by comparison with something. Again,
that which is called similar must be similar to something else, and all
other such attributes have this external reference. It is to be noted
that lying and standing and sitting are particular attitudes, but
attitude is itself a relative term. To lie, to stand, to be seated, are
not themselves attitudes, but take their name from the aforesaid
attitudes.

It is possible for relatives to have contraries. Thus virtue has a
contrary, vice, these both being relatives; knowledge, too, has a
contrary, ignorance. But this is not the mark of all relatives;
'double' and 'triple' have no contrary, nor indeed has any such term.

It also appears that relatives can admit of variation of degree. For
'like' and 'unlike', 'equal' and 'unequal', have the modifications
'more' and 'less' applied to them, and each of these is relative in
character: for the terms 'like' and 'unequal' bear 'unequal' bear a
reference to something external. Yet, again, it is not every relative
term that admits of variation of degree. No term such as 'double'
admits of this modification. All relatives have correlatives: by the
term 'slave' we mean the slave of a master, by the term 'master', the
master of a slave; by 'double', the double of its hall; by 'half', the
half of its double; by 'greater', greater than that which is less; by
'less,' less than that which is greater.

So it is with every other relative term; but the case we use to express
the correlation differs in some instances. Thus, by knowledge we mean
knowledge the knowable; by the knowable, that which is to be
apprehended by knowledge; by perception, perception of the perceptible;
by the perceptible, that which is apprehended by perception.

Sometimes, however, reciprocity of correlation does not appear to
exist. This comes about when a blunder is made, and that to which the
relative is related is not accurately stated. If a man states that a
wing is necessarily relative to a bird, the connexion between these two
will not be reciprocal, for it will not be possible to say that a bird
is a bird by reason of its wings. The reason is that the original
statement was inaccurate, for the wing is not said to be relative to
the bird qua bird, since many creatures besides birds have wings, but
qua winged creature. If, then, the statement is made accurate, the
connexion will be reciprocal, for we can speak of a wing, having
reference necessarily to a winged creature, and of a winged creature as
being such because of its wings.

Occasionally, perhaps, it is necessary to coin words, if no word exists
by which a correlation can adequately be explained. If we define a
rudder as necessarily having reference to a boat, our definition will
not be appropriate, for the rudder does not have this reference to a
boat qua boat, as there are boats which have no rudders. Thus we cannot
use the terms reciprocally, for the word 'boat' cannot be said to find
its explanation in the word 'rudder'. As there is no existing word, our
definition would perhaps be more accurate if we coined some word like
'ruddered' as the correlative of 'rudder'. If we express ourselves thus
accurately, at any rate the terms are reciprocally connected, for the
'ruddered' thing is 'ruddered' in virtue of its rudder. So it is in all
other cases. A head will be more accurately defined as the correlative
of that which is 'headed', than as that of an animal, for the animal
does not have a head qua animal, since many animals have no head.

Thus we may perhaps most easily comprehend that to which a thing is
related, when a name does not exist, if, from that which has a name, we
derive a new name, and apply it to that with which the first is
reciprocally connected, as in the aforesaid instances, when we derived
the word 'winged' from 'wing' and from 'rudder'.

All relatives, then, if properly defined, have a correlative. I add
this condition because, if that to which they are related is stated as
haphazard and not accurately, the two are not found to be
interdependent. Let me state what I mean more clearly. Even in the case
of acknowledged correlatives, and where names exist for each, there
will be no interdependence if one of the two is denoted, not by that
name which expresses the correlative notion, but by one of irrelevant
significance. The term 'slave,' if defined as related, not to a master,
but to a man, or a biped, or anything of that sort, is not reciprocally
connected with that in relation to which it is defined, for the
statement is not exact. Further, if one thing is said to be correlative
with another, and the terminology used is correct, then, though all
irrelevant attributes should be removed, and only that one attribute
left in virtue of which it was correctly stated to be correlative with
that other, the stated correlation will still exist. If the correlative
of 'the slave' is said to be 'the master', then, though all irrelevant
attributes of the said 'master', such as 'biped', 'receptive of
knowledge', 'human', should be removed, and the attribute 'master'
alone left, the stated correlation existing between him and the slave
will remain the same, for it is of a master that a slave is said to be
the slave. On the other hand, if, of two correlatives, one is not
correctly termed, then, when all other attributes are removed and that
alone is left in virtue of which it was stated to be correlative, the
stated correlation will be found to have disappeared.

For suppose the correlative of 'the slave' should be said to be 'the
man', or the correlative of 'the wing is the bird'; if the attribute
'master' be withdrawn from' the man', the correlation between 'the man'
and 'the slave' will cease to exist, for if the man is not a master,
the slave is not a slave. Similarly, if the attribute 'winged' be
withdrawn from 'the bird', 'the wing' will no longer be relative; for
if the so-called correlative is not winged, it follows that 'the wing'
has no correlative.

Thus it is essential that the correlated terms should be exactly
designated; if there is a name existing, the statement will be easy; if
not, it is doubtless our duty to construct names. When the terminology
is thus correct, it is evident that all correlatives are interdependent.

Correlatives are thought to come into existence simultaneously. This is
for the most part true, as in the case of the double and the half. The
existence of the half necessitates the existence of that of which it is
a half. Similarly the existence of a master necessitates the existence
of a slave, and that of a slave implies that of a master; these are
merely instances of a general rule. Moreover, they cancel one another;
for if there is no double it follows that there is no half, and vice
versa; this rule also applies to all such correlatives. Yet it does not
appear to be true in all cases that correlatives come into existence
simultaneously. The object of knowledge would appear to exist before
knowledge itself, for it is usually the case that we acquire knowledge
of objects already existing; it would be difficult, if not impossible,
to find a branch of knowledge the beginning of the existence of which
was contemporaneous with that of its object.

Again, while the object of knowledge, if it ceases to exist, cancels at
the same time the knowledge which was its correlative, the converse of
this is not true. It is true that if the object of knowledge does not
exist there can be no knowledge: for there will no longer be anything
to know. Yet it is equally true that, if knowledge of a certain object
does not exist, the object may nevertheless quite well exist. Thus, in
the case of the squaring of the circle, if indeed that process is an
object of knowledge, though it itself exists as an object of knowledge,
yet the knowledge of it has not yet come into existence. Again, if all
animals ceased to exist, there would be no knowledge, but there might
yet be many objects of knowledge.

This is likewise the case with regard to perception: for the object of
perception is, it appears, prior to the act of perception. If the
perceptible is annihilated, perception also will cease to exist; but
the annihilation of perception does not cancel the existence of the
perceptible. For perception implies a body perceived and a body in
which perception takes place. Now if that which is perceptible is
annihilated, it follows that the body is annihilated, for the body is a
perceptible thing; and if the body does not exist, it follows that
perception also ceases to exist. Thus the annihilation of the
perceptible involves that of perception.

But the annihilation of perception does not involve that of the
perceptible. For if the animal is annihilated, it follows that
perception also is annihilated, but perceptibles such as body, heat,
sweetness, bitterness, and so on, will remain.

Again, perception is generated at the same time as the perceiving
subject, for it comes into existence at the same time as the animal.
But the perceptible surely exists before perception; for fire and water
and such elements, out of which the animal is itself composed, exist
before the animal is an animal at all, and before perception. Thus it
would seem that the perceptible exists before perception.

It may be questioned whether it is true that no substance is relative,
as seems to be the case, or whether exception is to be made in the case
of certain secondary substances. With regard to primary substances, it
is quite true that there is no such possibility, for neither wholes nor
parts of primary substances are relative. The individual man or ox is
not defined with reference to something external. Similarly with the
parts: a particular hand or head is not defined as a particular hand or
head of a particular person, but as the hand or head of a particular
person. It is true also, for the most part at least, in the case of
secondary substances; the species 'man' and the species 'ox' are not
defined with reference to anything outside themselves. Wood, again, is
only relative in so far as it is some one's property, not in so far as
it is wood. It is plain, then, that in the cases mentioned substance is
not relative. But with regard to some secondary substances there is a
difference of opinion; thus, such terms as 'head' and 'hand' are
defined with reference to that of which the things indicated are a
part, and so it comes about that these appear to have a relative
character. Indeed, if our definition of that which is relative was
complete, it is very difficult, if not impossible, to prove that no
substance is relative. If, however, our definition was not complete, if
those things only are properly called relative in the case of which
relation to an external object is a necessary condition of existence,
perhaps some explanation of the dilemma may be found.

The former definition does indeed apply to all relatives, but the fact
that a thing is explained with reference to something else does not
make it essentially relative.

From this it is plain that, if a man definitely apprehends a relative
thing, he will also definitely apprehend that to which it is relative.
Indeed this is self-evident: for if a man knows that some particular
thing is relative, assuming that we call that a relative in the case of
which relation to something is a necessary condition of existence, he
knows that also to which it is related. For if he does not know at all
that to which it is related, he will not know whether or not it is
relative. This is clear, moreover, in particular instances. If a man
knows definitely that such and such a thing is 'double', he will also
forthwith know definitely that of which it is the double. For if there
is nothing definite of which he knows it to be the double, he does not
know at all that it is double. Again, if he knows that a thing is more
beautiful, it follows necessarily that he will forthwith definitely
know that also than which it is more beautiful. He will not merely know
indefinitely that it is more beautiful than something which is less
beautiful, for this would be supposition, not knowledge. For if he does
not know definitely that than which it is more beautiful, he can no
longer claim to know definitely that it is more beautiful than
something else which is less beautiful: for it might be that nothing
was less beautiful. It is, therefore, evident that if a man apprehends
some relative thing definitely, he necessarily knows that also
definitely to which it is related.

Now the head, the hand, and such things are substances, and it is
possible to know their essential character definitely, but it does not
necessarily follow that we should know that to which they are related.
It is not possible to know forthwith whose head or hand is meant. Thus
these are not relatives, and, this being the case, it would be true to
say that no substance is relative in character. It is perhaps a
difficult matter, in such cases, to make a positive statement without
more exhaustive examination, but to have raised questions with regard
to details is not without advantage.



Part 8

By 'quality' I mean that in virtue of which people are said to be such
and such.

Quality is a term that is used in many senses. One sort of quality let
us call 'habit' or 'disposition'. Habit differs from disposition in
being more lasting and more firmly established. The various kinds of
knowledge and of virtue are habits, for knowledge, even when acquired
only in a moderate degree, is, it is agreed, abiding in its character
and difficult to displace, unless some great mental upheaval takes
place, through disease or any such cause. The virtues, also, such as
justice, self-restraint, and so on, are not easily dislodged or
dismissed, so as to give place to vice.

By a disposition, on the other hand, we mean a condition that is easily
changed and quickly gives place to its opposite. Thus, heat, cold,
disease, health, and so on are dispositions. For a man is disposed in
one way or another with reference to these, but quickly changes,
becoming cold instead of warm, ill instead of well. So it is with all
other dispositions also, unless through lapse of time a disposition has
itself become inveterate and almost impossible to dislodge: in which
case we should perhaps go so far as to call it a habit.

It is evident that men incline to call those conditions habits which
are of a more or less permanent type and difficult to displace; for
those who are not retentive of knowledge, but volatile, are not said to
have such and such a 'habit' as regards knowledge, yet they are
disposed, we may say, either better or worse, towards knowledge. Thus
habit differs from disposition in this, that while the latter in
ephemeral, the former is permanent and difficult to alter.

Habits are at the same time dispositions, but dispositions are not
necessarily habits. For those who have some specific habit may be said
also, in virtue of that habit, to be thus or thus disposed; but those
who are disposed in some specific way have not in all cases the
corresponding habit.

Another sort of quality is that in virtue of which, for example, we
call men good boxers or runners, or healthy or sickly: in fact it
includes all those terms which refer to inborn capacity or incapacity.
Such things are not predicated of a person in virtue of his
disposition, but in virtue of his inborn capacity or incapacity to do
something with ease or to avoid defeat of any kind. Persons are called
good boxers or good runners, not in virtue of such and such a
disposition, but in virtue of an inborn capacity to accomplish
something with ease. Men are called healthy in virtue of the inborn
capacity of easy resistance to those unhealthy influences that may
ordinarily arise; unhealthy, in virtue of the lack of this capacity.
Similarly with regard to softness and hardness. Hardness is predicated
of a thing because it has that capacity of resistance which enables it
to withstand disintegration; softness, again, is predicated of a thing
by reason of the lack of that capacity.

A third class within this category is that of affective qualities and
affections. Sweetness, bitterness, sourness, are examples of this sort
of quality, together with all that is akin to these; heat, moreover,
and cold, whiteness, and blackness are affective qualities. It is
evident that these are qualities, for those things that possess them
are themselves said to be such and such by reason of their presence.
Honey is called sweet because it contains sweetness; the body is called
white because it contains whiteness; and so in all other cases.

The term 'affective quality' is not used as indicating that those
things which admit these qualities are affected in any way. Honey is
not called sweet because it is affected in a specific way, nor is this
what is meant in any other instance. Similarly heat and cold are called
affective qualities, not because those things which admit them are
affected. What is meant is that these said qualities are capable of
producing an 'affection' in the way of perception. For sweetness has
the power of affecting the sense of taste; heat, that of touch; and so
it is with the rest of these qualities.

Whiteness and blackness, however, and the other colours, are not said
to be affective qualities in this sense, but -because they themselves
are the results of an affection. It is plain that many changes of
colour take place because of affections. When a man is ashamed, he
blushes; when he is afraid, he becomes pale, and so on. So true is
this, that when a man is by nature liable to such affections, arising
from some concomitance of elements in his constitution, it is a
probable inference that he has the corresponding complexion of skin.
For the same disposition of bodily elements, which in the former
instance was momentarily present in the case of an access of shame,
might be a result of a man's natural temperament, so as to produce the
corresponding colouring also as a natural characteristic. All
conditions, therefore, of this kind, if caused by certain permanent and
lasting affections, are called affective qualities. For pallor and
duskiness of complexion are called qualities, inasmuch as we are said
to be such and such in virtue of them, not only if they originate in
natural constitution, but also if they come about through long disease
or sunburn, and are difficult to remove, or indeed remain throughout
life. For in the same way we are said to be such and such because of
these.

Those conditions, however, which arise from causes which may easily be
rendered ineffective or speedily removed, are called, not qualities,
but affections: for we are not said to be such virtue of them. The man
who blushes through shame is not said to be a constitutional blusher,
nor is the man who becomes pale through fear said to be
constitutionally pale. He is said rather to have been affected.

Thus such conditions are called affections, not qualities. In like
manner there are affective qualities and affections of the soul. That
temper with which a man is born and which has its origin in certain
deep-seated affections is called a quality. I mean such conditions as
insanity, irascibility, and so on: for people are said to be mad or
irascible in virtue of these. Similarly those abnormal psychic states
which are not inborn, but arise from the concomitance of certain other
elements, and are difficult to remove, or altogether permanent, are
called qualities, for in virtue of them men are said to be such and
such.

Those, however, which arise from causes easily rendered ineffective are
called affections, not qualities. Suppose that a man is irritable when
vexed: he is not even spoken of as a bad-tempered man, when in such
circumstances he loses his temper somewhat, but rather is said to be
affected. Such conditions are therefore termed, not qualities, but
affections.

The fourth sort of quality is figure and the shape that belongs to a
thing; and besides this, straightness and curvedness and any other
qualities of this type; each of these defines a thing as being such and
such. Because it is triangular or quadrangular a thing is said to have
a specific character, or again because it is straight or curved; in
fact a thing's shape in every case gives rise to a qualification of it.

Rarity and density, roughness and smoothness, seem to be terms
indicating quality: yet these, it would appear, really belong to a
class different from that of quality. For it is rather a certain
relative position of the parts composing the thing thus qualified
which, it appears, is indicated by each of these terms. A thing is
dense, owing to the fact that its parts are closely combined with one
another; rare, because there are interstices between the parts; smooth,
because its parts lie, so to speak, evenly; rough, because some parts
project beyond others.

There may be other sorts of quality, but those that are most properly
so called have, we may safely say, been enumerated.

These, then, are qualities, and the things that take their name from
them as derivatives, or are in some other way dependent on them, are
said to be qualified in some specific way. In most, indeed in almost
all cases, the name of that which is qualified is derived from that of
the quality. Thus the terms 'whiteness', 'grammar', 'justice', give us
the adjectives 'white', 'grammatical', 'just', and so on.

There are some cases, however, in which, as the quality under
consideration has no name, it is impossible that those possessed of it
should have a name that is derivative. For instance, the name given to
the runner or boxer, who is so called in virtue of an inborn capacity,
is not derived from that of any quality; for lob those capacities have
no name assigned to them. In this, the inborn capacity is distinct from
the science, with reference to which men are called, e.g. boxers or
wrestlers. Such a science is classed as a disposition; it has a name,
and is called 'boxing' or 'wrestling' as the case may be, and the name
given to those disposed in this way is derived from that of the
science. Sometimes, even though a name exists for the quality, that
which takes its character from the quality has a name that is not a
derivative. For instance, the upright man takes his character from the
possession of the quality of integrity, but the name given him is not
derived from the word 'integrity'. Yet this does not occur often.

We may therefore state that those things are said to be possessed of
some specific quality which have a name derived from that of the
aforesaid quality, or which are in some other way dependent on it.

One quality may be the contrary of another; thus justice is the
contrary of injustice, whiteness of blackness, and so on. The things,
also, which are said to be such and such in virtue of these qualities,
may be contrary the one to the other; for that which is unjust is
contrary to that which is just, that which is white to that which is
black. This, however, is not always the case. Red, yellow, and such
colours, though qualities, have no contraries.

If one of two contraries is a quality, the other will also be a
quality. This will be evident from particular instances, if we apply
the names used to denote the other categories; for instance, granted
that justice is the contrary of injustice and justice is a quality,
injustice will also be a quality: neither quantity, nor relation, nor
place, nor indeed any other category but that of quality, will be
applicable properly to injustice. So it is with all other contraries
falling under the category of quality.

Qualities admit of variation of degree. Whiteness is predicated of one
thing in a greater or less degree than of another. This is also the
case with reference to justice. Moreover, one and the same thing may
exhibit a quality in a greater degree than it did before: if a thing is
white, it may become whiter.

Though this is generally the case, there are exceptions. For if we
should say that justice admitted of variation of degree, difficulties
might ensue, and this is true with regard to all those qualities which
are dispositions. There are some, indeed, who dispute the possibility
of variation here. They maintain that justice and health cannot very
well admit of variation of degree themselves, but that people vary in
the degree in which they possess these qualities, and that this is the
case with grammatical learning and all those qualities which are
classed as dispositions. However that may be, it is an incontrovertible
fact that the things which in virtue of these qualities are said to be
what they are vary in the degree in which they possess them; for one
man is said to be better versed in grammar, or more healthy or just,
than another, and so on.

The qualities expressed by the terms 'triangular' and 'quadrangular' do
not appear to admit of variation of degree, nor indeed do any that have
to do with figure. For those things to which the definition of the
triangle or circle is applicable are all equally triangular or
circular. Those, on the other hand, to which the same definition is not
applicable, cannot be said to differ from one another in degree; the
square is no more a circle than the rectangle, for to neither is the
definition of the circle appropriate. In short, if the definition of
the term proposed is not applicable to both objects, they cannot be
compared. Thus it is not all qualities which admit of variation of
degree.

Whereas none of the characteristics I have mentioned are peculiar to
quality, the fact that likeness and unlikeness can be predicated with
reference to quality only, gives to that category its distinctive
feature. One thing is like another only with reference to that in
virtue of which it is such and such; thus this forms the peculiar mark
of quality.

We must not be disturbed because it may be argued that, though
proposing to discuss the category of quality, we have included in it
many relative terms. We did say that habits and dispositions were
relative. In practically all such cases the genus is relative, the
individual not. Thus knowledge, as a genus, is explained by reference
to something else, for we mean a knowledge of something. But particular
branches of knowledge are not thus explained. The knowledge of grammar
is not relative to anything external, nor is the knowledge of music,
but these, if relative at all, are relative only in virtue of their
genera; thus grammar is said be the knowledge of something, not the
grammar of something; similarly music is the knowledge of something,
not the music of something.

Thus individual branches of knowledge are not relative. And it is
because we possess these individual branches of knowledge that we are
said to be such and such. It is these that we actually possess: we are
called experts because we possess knowledge in some particular branch.
Those particular branches, therefore, of knowledge, in virtue of which
we are sometimes said to be such and such, are themselves qualities,
and are not relative. Further, if anything should happen to fall within
both the category of quality and that of relation, there would be
nothing extraordinary in classing it under both these heads.



Section 3


Part 9

Action and affection both admit of contraries and also of variation of
degree. Heating is the contrary of cooling, being heated of being
cooled, being glad of being vexed. Thus they admit of contraries. They
also admit of variation of degree: for it is possible to heat in a
greater or less degree; also to be heated in a greater or less degree.
Thus action and affection also admit of variation of degree. So much,
then, is stated with regard to these categories.

We spoke, moreover, of the category of position when we were dealing
with that of relation, and stated that such terms derived their names
from those of the corresponding attitudes.

As for the rest, time, place, state, since they are easily
intelligible, I say no more about them than was said at the beginning,
that in the category of state are included such states as 'shod',
'armed', in that of place 'in the Lyceum' and so on, as was explained
before.



Part 10

The proposed categories have, then, been adequately dealt with. We must
next explain the various senses in which the term 'opposite' is used.
Things are said to be opposed in four senses: (i) as correlatives to
one another, (ii) as contraries to one another, (iii) as privatives to
positives, (iv) as affirmatives to negatives.

Let me sketch my meaning in outline. An instance of the use of the word
'opposite' with reference to correlatives is afforded by the
expressions 'double' and 'half'; with reference to contraries by 'bad'
and 'good'. Opposites in the sense of 'privatives' and 'positives' are'
blindness' and 'sight'; in the sense of affirmatives and negatives, the
propositions 'he sits', 'he does not sit'.

(i) Pairs of opposites which fall under the category of relation are
explained by a reference of the one to the other, the reference being
indicated by the preposition 'of' or by some other preposition. Thus,
double is a relative term, for that which is double is explained as the
double of something. Knowledge, again, is the opposite of the thing
known, in the same sense; and the thing known also is explained by its
relation to its opposite, knowledge. For the thing known is explained
as that which is known by something, that is, by knowledge. Such
things, then, as are opposite the one to the other in the sense of
being correlatives are explained by a reference of the one to the other.

(ii) Pairs of opposites which are contraries are not in any way
interdependent, but are contrary the one to the other. The good is not
spoken of as the good of the bad, but as the contrary of the bad, nor
is white spoken of as the white of the black, but as the contrary of
the black. These two types of opposition are therefore distinct. Those
contraries which are such that the subjects in which they are naturally
present, or of which they are predicated, must necessarily contain
either the one or the other of them, have no intermediate, but those in
the case of which no such necessity obtains, always have an
intermediate. Thus disease and health are naturally present in the body
of an animal, and it is necessary that either the one or the other
should be present in the body of an animal. Odd and even, again, are
predicated of number, and it is necessary that the one or the other
should be present in numbers. Now there is no intermediate between the
terms of either of these two pairs. On the other hand, in those
contraries with regard to which no such necessity obtains, we find an
intermediate. Blackness and whiteness are naturally present in the
body, but it is not necessary that either the one or the other should
be present in the body, inasmuch as it is not true to say that
everybody must be white or black. Badness and goodness, again, are
predicated of man, and of many other things, but it is not necessary
that either the one quality or the other should be present in that of
which they are predicated: it is not true to say that everything that
may be good or bad must be either good or bad. These pairs of
contraries have intermediates: the intermediates between white and
black are grey, sallow, and all the other colours that come between;
the intermediate between good and bad is that which is neither the one
nor the other.

Some intermediate qualities have names, such as grey and sallow and all
the other colours that come between white and black; in other cases,
however, it is not easy to name the intermediate, but we must define it
as that which is not either extreme, as in the case of that which is
neither good nor bad, neither just nor unjust.

(iii) 'privatives' and 'Positives' have reference to the same subject.
Thus, sight and blindness have reference to the eye. It is a universal
rule that each of a pair of opposites of this type has reference to
that to which the particular 'positive' is natural. We say that that is
capable of some particular faculty or possession has suffered privation
when the faculty or possession in question is in no way present in that
in which, and at the time at which, it should naturally be present. We
do not call that toothless which has not teeth, or that blind which has
not sight, but rather that which has not teeth or sight at the time
when by nature it should. For there are some creatures which from birth
are without sight, or without teeth, but these are not called toothless
or blind.

To be without some faculty or to possess it is not the same as the
corresponding 'privative' or 'positive'. 'Sight' is a 'positive',
'blindness' a 'privative', but 'to possess sight' is not equivalent to
'sight', 'to be blind' is not equivalent to 'blindness'. Blindness is a
'privative', to be blind is to be in a state of privation, but is not a
'privative'. Moreover, if 'blindness' were equivalent to 'being blind',
both would be predicated of the same subject; but though a man is said
to be blind, he is by no means said to be blindness.

To be in a state of 'possession' is, it appears, the opposite of being
in a state of 'privation', just as 'positives' and 'privatives'
themselves are opposite. There is the same type of antithesis in both
cases; for just as blindness is opposed to sight, so is being blind
opposed to having sight.

That which is affirmed or denied is not itself affirmation or denial.
By 'affirmation' we mean an affirmative proposition, by 'denial' a
negative. Now, those facts which form the matter of the affirmation or
denial are not propositions; yet these two are said to be opposed in
the same sense as the affirmation and denial, for in this case also the
type of antithesis is the same. For as the affirmation is opposed to
the denial, as in the two propositions 'he sits', 'he does not sit', so
also the fact which constitutes the matter of the proposition in one
case is opposed to that in the other, his sitting, that is to say, to
his not sitting.

It is evident that 'positives' and 'privatives' are not opposed each to
each in the same sense as relatives. The one is not explained by
reference to the other; sight is not sight of blindness, nor is any
other preposition used to indicate the relation. Similarly blindness is
not said to be blindness of sight, but rather, privation of sight.
Relatives, moreover, reciprocate; if blindness, therefore, were a
relative, there would be a reciprocity of relation between it and that
with which it was correlative. But this is not the case. Sight is not
called the sight of blindness.

That those terms which fall under the heads of 'positives' and
'privatives' are not opposed each to each as contraries, either, is
plain from the following facts: Of a pair of contraries such that they
have no intermediate, one or the other must needs be present in the
subject in which they naturally subsist, or of which they are
predicated; for it is those, as we proved,' in the case of which this
necessity obtains, that have no intermediate. Moreover, we cited health
and disease, odd and even, as instances. But those contraries which
have an intermediate are not subject to any such necessity. It is not
necessary that every substance, receptive of such qualities, should be
either black or white, cold or hot, for something intermediate between
these contraries may very well be present in the subject. We proved,
moreover, that those contraries have an intermediate in the case of
which the said necessity does not obtain. Yet when one of the two
contraries is a constitutive property of the subject, as it is a
constitutive property of fire to be hot, of snow to be white, it is
necessary determinately that one of the two contraries, not one or the
other, should be present in the subject; for fire cannot be cold, or
snow black. Thus, it is not the case here that one of the two must
needs be present in every subject receptive of these qualities, but
only in that subject of which the one forms a constitutive property.
Moreover, in such cases it is one member of the pair determinately, and
not either the one or the other, which must be present.

In the case of 'positives' and 'privatives', on the other hand, neither
of the aforesaid statements holds good. For it is not necessary that a
subject receptive of the qualities should always have either the one or
the other; that which has not yet advanced to the state when sight is
natural is not said either to be blind or to see. Thus 'positives' and
'privatives' do not belong to that class of contraries which consists
of those which have no intermediate. On the other hand, they do not
belong either to that class which consists of contraries which have an
intermediate. For under certain conditions it is necessary that either
the one or the other should form part of the constitution of every
appropriate subject. For when a thing has reached the stage when it is
by nature capable of sight, it will be said either to see or to be
blind, and that in an indeterminate sense, signifying that the capacity
may be either present or absent; for it is not necessary either that it
should see or that it should be blind, but that it should be either in
the one state or in the other. Yet in the case of those contraries
which have an intermediate we found that it was never necessary that
either the one or the other should be present in every appropriate
subject, but only that in certain subjects one of the pair should be
present, and that in a determinate sense. It is, therefore, plain that
'positives' and 'privatives' are not opposed each to each in either of
the senses in which contraries are opposed.

Again, in the case of contraries, it is possible that there should be
changes from either into the other, while the subject retains its
identity, unless indeed one of the contraries is a constitutive
property of that subject, as heat is of fire. For it is possible that
that that which is healthy should become diseased, that which is white,
black, that which is cold, hot, that which is good, bad, that which is
bad, good. The bad man, if he is being brought into a better way of
life and thought, may make some advance, however slight, and if he
should once improve, even ever so little, it is plain that he might
change completely, or at any rate make very great progress; for a man
becomes more and more easily moved to virtue, however small the
improvement was at first. It is, therefore, natural to suppose that he
will make yet greater progress than he has made in the past; and as
this process goes on, it will change him completely and establish him
in the contrary state, provided he is not hindered by lack of time. In
the case of 'positives' and 'privatives', however, change in both
directions is impossible. There may be a change from possession to
privation, but not from privation to possession. The man who has become
blind does not regain his sight; the man who has become bald does not
regain his hair; the man who has lost his teeth does not grow a new
set. (iv) Statements opposed as affirmation and negation belong
manifestly to a class which is distinct, for in this case, and in this
case only, it is necessary for the one opposite to be true and the
other false.

Neither in the case of contraries, nor in the case of correlatives, nor
in the case of 'positives' and 'privatives', is it necessary for one to
be true and the other false. Health and disease are contraries: neither
of them is true or false. 'Double' and 'half' are opposed to each other
as correlatives: neither of them is true or false. The case is the
same, of course, with regard to 'positives' and 'privatives' such as
'sight' and 'blindness'. In short, where there is no sort of
combination of words, truth and falsity have no place, and all the
opposites we have mentioned so far consist of simple words.

At the same time, when the words which enter into opposed statements
are contraries, these, more than any other set of opposites, would seem
to claim this characteristic. 'Socrates is ill' is the contrary of
'Socrates is well', but not even of such composite expressions is it
true to say that one of the pair must always be true and the other
false. For if Socrates exists, one will be true and the other false,
but if he does not exist, both will be false; for neither 'Socrates is
ill' nor 'Socrates is well' is true, if Socrates does not exist at all.

In the case of 'positives' and 'privatives', if the subject does not
exist at all, neither proposition is true, but even if the subject
exists, it is not always the fact that one is true and the other false.
For 'Socrates has sight' is the opposite of 'Socrates is blind' in the
sense of the word 'opposite' which applies to possession and privation.
Now if Socrates exists, it is not necessary that one should be true and
the other false, for when he is not yet able to acquire the power of
vision, both are false, as also if Socrates is altogether non-existent.

But in the case of affirmation and negation, whether the subject exists
or not, one is always false and the other true. For manifestly, if
Socrates exists, one of the two propositions 'Socrates is ill',
'Socrates is not ill', is true, and the other false. This is likewise
the case if he does not exist; for if he does not exist, to say that he
is ill is false, to say that he is not ill is true. Thus it is in the
case of those opposites only, which are opposite in the sense in which
the term is used with reference to affirmation and negation, that the
rule holds good, that one of the pair must be true and the other false.



Part 11

That the contrary of a good is an evil is shown by induction: the
contrary of health is disease, of courage, cowardice, and so on. But
the contrary of an evil is sometimes a good, sometimes an evil. For
defect, which is an evil, has excess for its contrary, this also being
an evil, and the mean, which is a good, is equally the contrary of the
one and of the other. It is only in a few cases, however, that we see
instances of this: in most, the contrary of an evil is a good.

In the case of contraries, it is not always necessary that if one
exists the other should also exist: for if all become healthy there
will be health and no disease, and again, if everything turns white,
there will be white, but no black. Again, since the fact that Socrates
is ill is the contrary of the fact that Socrates is well, and two
contrary conditions cannot both obtain in one and the same individual
at the same time, both these contraries could not exist at once: for if
that Socrates was well was a fact, then that Socrates was ill could not
possibly be one.

It is plain that contrary attributes must needs be present in subjects
which belong to the same species or genus. Disease and health require
as their subject the body of an animal; white and black require a body,
without further qualification; justice and injustice require as their
subject the human soul.

Moreover, it is necessary that pairs of contraries should in all cases
either belong to the same genus or belong to contrary genera or be
themselves genera. White and black belong to the same genus, colour;
justice and injustice, to contrary genera, virtue and vice; while good
and evil do not belong to genera, but are themselves actual genera,
with terms under them.



Part 12

There are four senses in which one thing can be said to be 'prior' to
another. Primarily and most properly the term has reference to time: in
this sense the word is used to indicate that one thing is older or more
ancient than another, for the expressions 'older' and 'more ancient'
imply greater length of time.

Secondly, one thing is said to be 'prior' to another when the sequence
of their being cannot be reversed. In this sense 'one' is 'prior' to
'two'. For if 'two' exists, it follows directly that 'one' must exist,
but if 'one' exists, it does not follow necessarily that 'two' exists:
thus the sequence subsisting cannot be reversed. It is agreed, then,
that when the sequence of two things cannot be reversed, then that one
on which the other depends is called 'prior' to that other.

In the third place, the term 'prior' is used with reference to any
order, as in the case of science and of oratory. For in sciences which
use demonstration there is that which is prior and that which is
posterior in order; in geometry, the elements are prior to the
propositions; in reading and writing, the letters of the alphabet are
prior to the syllables. Similarly, in the case of speeches, the
exordium is prior in order to the narrative.

Besides these senses of the word, there is a fourth. That which is
better and more honourable is said to have a natural priority. In
common parlance men speak of those whom they honour and love as 'coming
first' with them. This sense of the word is perhaps the most
far-fetched.

Such, then, are the different senses in which the term 'prior' is used.

Yet it would seem that besides those mentioned there is yet another.
For in those things, the being of each of which implies that of the
other, that which is in any way the cause may reasonably be said to be
by nature 'prior' to the effect. It is plain that there are instances
of this. The fact of the being of a man carries with it the truth of
the proposition that he is, and the implication is reciprocal: for if a
man is, the proposition wherein we allege that he is true, and
conversely, if the proposition wherein we allege that he is true, then
he is. The true proposition, however, is in no way the cause of the
being of the man, but the fact of the man's being does seem somehow to
be the cause of the truth of the proposition, for the truth or falsity
of the proposition depends on the fact of the man's being or not being.

Thus the word 'prior' may be used in five senses.



Part 13

The term 'simultaneous' is primarily and most appropriately applied to
those things the genesis of the one of which is simultaneous with that
of the other; for in such cases neither is prior or posterior to the
other. Such things are said to be simultaneous in point of time. Those
things, again, are 'simultaneous' in point of nature, the being of each
of which involves that of the other, while at the same time neither is
the cause of the other's being. This is the case with regard to the
double and the half, for these are reciprocally dependent, since, if
there is a double, there is also a half, and if there is a half, there
is also a double, while at the same time neither is the cause of the
being of the other.

Again, those species which are distinguished one from another and
opposed one to another within the same genus are said to be
'simultaneous' in nature. I mean those species which are distinguished
each from each by one and the same method of division. Thus the
'winged' species is simultaneous with the 'terrestrial' and the 'water'
species. These are distinguished within the same genus, and are opposed
each to each, for the genus 'animal' has the 'winged', the
'terrestrial', and the 'water' species, and no one of these is prior or
posterior to another; on the contrary, all such things appear to be
'simultaneous' in nature. Each of these also, the terrestrial, the
winged, and the water species, can be divided again into subspecies.
Those species, then, also will be 'simultaneous' point of nature,
which, belonging to the same genus, are distinguished each from each by
one and the same method of differentiation.

But genera are prior to species, for the sequence of their being cannot
be reversed. If there is the species 'water-animal', there will be the
genus 'animal', but granted the being of the genus 'animal', it does
not follow necessarily that there will be the species 'water-animal'.

Those things, therefore, are said to be 'simultaneous' in nature, the
being of each of which involves that of the other, while at the same
time neither is in any way the cause of the other's being; those
species, also, which are distinguished each from each and opposed
within the same genus. Those things, moreover, are 'simultaneous' in
the unqualified sense of the word which come into being at the same
time.



Part 14

There are six sorts of movement: generation, destruction, increase,
diminution, alteration, and change of place.

It is evident in all but one case that all these sorts of movement are
distinct each from each. Generation is distinct from destruction,
increase and change of place from diminution, and so on. But in the
case of alteration it may be argued that the process necessarily
implies one or other of the other five sorts of motion. This is not
true, for we may say that all affections, or nearly all, produce in us
an alteration which is distinct from all other sorts of motion, for
that which is affected need not suffer either increase or diminution or
any of the other sorts of motion. Thus alteration is a distinct sort of
motion; for, if it were not, the thing altered would not only be
altered, but would forthwith necessarily suffer increase or diminution
or some one of the other sorts of motion in addition; which as a matter
of fact is not the case. Similarly that which was undergoing the
process of increase or was subject to some other sort of motion would,
if alteration were not a distinct form of motion, necessarily be
subject to alteration also. But there are some things which undergo
increase but yet not alteration. The square, for instance, if a gnomon
is applied to it, undergoes increase but not alteration, and so it is
with all other figures of this sort. Alteration and increase,
therefore, are distinct.

Speaking generally, rest is the contrary of motion. But the different
forms of motion have their own contraries in other forms; thus
destruction is the contrary of generation, diminution of increase, rest
in a place, of change of place. As for this last, change in the reverse
direction would seem to be most truly its contrary; thus motion upwards
is the contrary of motion downwards and vice versa.

In the case of that sort of motion which yet remains, of those that
have been enumerated, it is not easy to state what is its contrary. It
appears to have no contrary, unless one should define the contrary here
also either as 'rest in its quality' or as 'change in the direction of
the contrary quality', just as we defined the contrary of change of
place either as rest in a place or as change in the reverse direction.
For a thing is altered when change of quality takes place; therefore
either rest in its quality or change in the direction of the contrary
may be called the contrary of this qualitative form of motion. In this
way becoming white is the contrary of becoming black; there is
alteration in the contrary direction, since a change of a qualitative
nature takes place.



Part 15

The term 'to have' is used in various senses. In the first place it is
used with reference to habit or disposition or any other quality, for
we are said to 'have' a piece of knowledge or a virtue. Then, again, it
has reference to quantity, as, for instance, in the case of a man's
height; for he is said to 'have' a height of three or four cubits. It
is used, moreover, with regard to apparel, a man being said to 'have' a
coat or tunic; or in respect of something which we have on a part of
ourselves, as a ring on the hand: or in respect of something which is a
part of us, as hand or foot. The term refers also to content, as in the
case of a vessel and wheat, or of a jar and wine; a jar is said to
'have' wine, and a corn-measure wheat. The expression in such cases has
reference to content. Or it refers to that which has been acquired; we
are said to 'have' a house or a field. A man is also said to 'have' a
wife, and a wife a husband, and this appears to be the most remote
meaning of the term, for by the use of it we mean simply that the
husband lives with the wife.

Other senses of the word might perhaps be found, but the most ordinary
ones have all been enumerated.






% chapter categories (end)