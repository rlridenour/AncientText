\chapter{On the Soul} % (fold)
\label{cha:soul}


On the Soul
By Aristotle


Translated by J. A. Smith

----------------------------------------------------------------------

BOOK I

Part 1 

Holding as we do that, while knowledge of any kind is a thing to
be honoured and prized, one kind of it may, either by reason of its
greater exactness or of a higher dignity and greater wonderfulness
in its objects, be more honourable and precious than another, on both
accounts we should naturally be led to place in the front rank the
study of the soul. The knowledge of the soul admittedly contributes
greatly to the advance of truth in general, and, above all, to our
understanding of Nature, for the soul is in some sense the principle
of animal life. Our aim is to grasp and understand, first its essential
nature, and secondly its properties; of these some are taught to be
affections proper to the soul itself, while others are considered
to attach to the animal owing to the presence within it of soul.

To attain any assured knowledge about the soul is one of the most
difficult things in the world. As the form of question which here
presents itself, viz. the question 'What is it?', recurs in other
fields, it might be supposed that there was some single method of
inquiry applicable to all objects whose essential nature (as we are
endeavouring to ascertain there is for derived properties the single
method of demonstration); in that case what we should have to seek
for would be this unique method. But if there is no such single and
general method for solving the question of essence, our task becomes
still more difficult; in the case of each different subject we shall
have to determine the appropriate process of investigation. If to
this there be a clear answer, e.g. that the process is demonstration
or division, or some known method, difficulties and hesitations still
beset us-with what facts shall we begin the inquiry? For the facts
which form the starting-points in different subjects must be different,
as e.g. in the case of numbers and surfaces. 

First, no doubt, it is necessary to determine in which of the summa
genera soul lies, what it is; is it 'a this-somewhat, 'a substance,
or is it a quale or a quantum, or some other of the remaining kinds
of predicates which we have distinguished? Further, does soul belong
to the class of potential existents, or is it not rather an actuality?
Our answer to this question is of the greatest importance.

We must consider also whether soul is divisible or is without parts,
and whether it is everywhere homogeneous or not; and if not homogeneous,
whether its various forms are different specifically or generically:
up to the present time those who have discussed and investigated soul
seem to have confined themselves to the human soul. We must be careful
not to ignore the question whether soul can be defined in a single
unambiguous formula, as is the case with animal, or whether we must
not give a separate formula for each of it, as we do for horse, dog,
man, god (in the latter case the 'universal' animal-and so too every
other 'common predicate'-being treated either as nothing at all or
as a later product). Further, if what exists is not a plurality of
souls, but a plurality of parts of one soul, which ought we to investigate
first, the whole soul or its parts? (It is also a difficult problem
to decide which of these parts are in nature distinct from one another.)
Again, which ought we to investigate first, these parts or their functions,
mind or thinking, the faculty or the act of sensation, and so on?
If the investigation of the functions precedes that of the parts,
the further question suggests itself: ought we not before either to
consider the correlative objects, e.g. of sense or thought? It seems
not only useful for the discovery of the causes of the derived properties
of substances to be acquainted with the essential nature of those
substances (as in mathematics it is useful for the understanding of
the property of the equality of the interior angles of a triangle
to two right angles to know the essential nature of the straight and
the curved or of the line and the plane) but also conversely, for
the knowledge of the essential nature of a substance is largely promoted
by an acquaintance with its properties: for, when we are able to give
an account conformable to experience of all or most of the properties
of a substance, we shall be in the most favourable position to say
something worth saying about the essential nature of that subject;
in all demonstration a definition of the essence is required as a
starting-point, so that definitions which do not enable us to discover
the derived properties, or which fail to facilitate even a conjecture
about them, must obviously, one and all, be dialectical and futile.

A further problem presented by the affections of soul is this: are
they all affections of the complex of body and soul, or is there any
one among them peculiar to the soul by itself? To determine this is
indispensable but difficult. If we consider the majority of them,
there seems to be no case in which the soul can act or be acted upon
without involving the body; e.g. anger, courage, appetite, and sensation
generally. Thinking seems the most probable exception; but if this
too proves to be a form of imagination or to be impossible without
imagination, it too requires a body as a condition of its existence.
If there is any way of acting or being acted upon proper to soul,
soul will be capable of separate existence; if there is none, its
separate existence is impossible. In the latter case, it will be like
what is straight, which has many properties arising from the straightness
in it, e.g. that of touching a bronze sphere at a point, though straightness
divorced from the other constituents of the straight thing cannot
touch it in this way; it cannot be so divorced at all, since it is
always found in a body. It therefore seems that all the affections
of soul involve a body-passion, gentleness, fear, pity, courage, joy,
loving, and hating; in all these there is a concurrent affection of
the body. In support of this we may point to the fact that, while
sometimes on the occasion of violent and striking occurrences there
is no excitement or fear felt, on others faint and feeble stimulations
produce these emotions, viz. when the body is already in a state of
tension resembling its condition when we are angry. Here is a still
clearer case: in the absence of any external cause of terror we find
ourselves experiencing the feelings of a man in terror. From all this
it is obvious that the affections of soul are enmattered formulable
essences. 

Consequently their definitions ought to correspond, e.g. anger should
be defined as a certain mode of movement of such and such a body (or
part or faculty of a body) by this or that cause and for this or that
end. That is precisely why the study of the soul must fall within
the science of Nature, at least so far as in its affections it manifests
this double character. Hence a physicist would define an affection
of soul differently from a dialectician; the latter would define e.g.
anger as the appetite for returning pain for pain, or something like
that, while the former would define it as a boiling of the blood or
warm substance surround the heart. The latter assigns the material
conditions, the former the form or formulable essence; for what he
states is the formulable essence of the fact, though for its actual
existence there must be embodiment of it in a material such as is
described by the other. Thus the essence of a house is assigned in
such a formula as 'a shelter against destruction by wind, rain, and
heat'; the physicist would describe it as 'stones, bricks, and timbers';
but there is a third possible description which would say that it
was that form in that material with that purpose or end. Which, then,
among these is entitled to be regarded as the genuine physicist? The
one who confines himself to the material, or the one who restricts
himself to the formulable essence alone? Is it not rather the one
who combines both in a single formula? If this is so, how are we to
characterize the other two? Must we not say that there is no type
of thinker who concerns himself with those qualities or attributes
of the material which are in fact inseparable from the material, and
without attempting even in thought to separate them? The physicist
is he who concerns himself with all the properties active and passive
of bodies or materials thus or thus defined; attributes not considered
as being of this character he leaves to others, in certain cases it
may be to a specialist, e.g. a carpenter or a physician, in others
(a) where they are inseparable in fact, but are separable from any
particular kind of body by an effort of abstraction, to the mathematician,
(b) where they are separate both in fact and in thought from body
altogether, to the First Philosopher or metaphysician. But we must
return from this digression, and repeat that the affections of soul
are inseparable from the material substratum of animal life, to which
we have seen that such affections, e.g. passion and fear, attach,
and have not the same mode of being as a line or a plane.

Part 2

For our study of soul it is necessary, while formulating the problems
of which in our further advance we are to find the solutions, to call
into council the views of those of our predecessors who have declared
any opinion on this subject, in order that we may profit by whatever
is sound in their suggestions and avoid their errors. 

The starting-point of our inquiry is an exposition of those characteristics
which have chiefly been held to belong to soul in its very nature.
Two characteristic marks have above all others been recognized as
distinguishing that which has soul in it from that which has not-movement
and sensation. It may be said that these two are what our predecessors
have fixed upon as characteristic of soul. 

Some say that what originates movement is both pre-eminently and primarily
soul; believing that what is not itself moved cannot originate movement
in another, they arrived at the view that soul belongs to the class
of things in movement. This is what led Democritus to say that soul
is a sort of fire or hot substance; his 'forms' or atoms are infinite
in number; those which are spherical he calls fire and soul, and compares
them to the motes in the air which we see in shafts of light coming
through windows; the mixture of seeds of all sorts he calls the elements
of the whole of Nature (Leucippus gives a similar account); the spherical
atoms are identified with soul because atoms of that shape are most
adapted to permeate everywhere, and to set all the others moving by
being themselves in movement. This implies the view that soul is identical
with what produces movement in animals. That is why, further, they
regard respiration as the characteristic mark of life; as the environment
compresses the bodies of animals, and tends to extrude those atoms
which impart movement to them, because they themselves are never at
rest, there must be a reinforcement of these by similar atoms coming
in from without in the act of respiration; for they prevent the extrusion
of those which are already within by counteracting the compressing
and consolidating force of the environment; and animals continue to
live only so long as they are able to maintain this resistance.

The doctrine of the Pythagoreans seems to rest upon the same ideas;
some of them declared the motes in air, others what moved them, to
be soul. These motes were referred to because they are seen always
in movement, even in a complete calm. 

The same tendency is shown by those who define soul as that which
moves itself; all seem to hold the view that movement is what is closest
to the nature of soul, and that while all else is moved by soul, it
alone moves itself. This belief arises from their never seeing anything
originating movement which is not first itself moved. 

Similarly also Anaxagoras (and whoever agrees with him in saying that
mind set the whole in movement) declares the moving cause of things
to be soul. His position must, however, be distinguished from that
of Democritus. Democritus roundly identifies soul and mind, for he
identifies what appears with what is true-that is why he commends
Homer for the phrase 'Hector lay with thought distraught'; he does
not employ mind as a special faculty dealing with truth, but identifies
soul and mind. What Anaxagoras says about them is more obscure; in
many places he tells us that the cause of beauty and order is mind,
elsewhere that it is soul; it is found, he says, in all animals, great
and small, high and low, but mind (in the sense of intelligence) appears
not to belong alike to all animals, and indeed not even to all human
beings. 

All those, then, who had special regard to the fact that what has
soul in it is moved, adopted the view that soul is to be identified
with what is eminently originative of movement. All, on the other
hand, who looked to the fact that what has soul in it knows or perceives
what is, identify soul with the principle or principles of Nature,
according as they admit several such principles or one only. Thus
Empedocles declares that it is formed out of all his elements, each
of them also being soul; his words are: 

For 'tis by Earth we see Earth, by Water Water, 
By Ether Ether divine, by Fire destructive Fire, 
By Love Love, and Hate by cruel Hate. 

In the same way Plato in the Timaeus fashions soul out of his elements;
for like, he holds, is known by like, and things are formed out of
the principles or elements, so that soul must be so too. Similarly
also in his lectures 'On Philosophy' it was set forth that the Animal-itself
is compounded of the Idea itself of the One together with the primary
length, breadth, and depth, everything else, the objects of its perception,
being similarly constituted. Again he puts his view in yet other terms:
Mind is the monad, science or knowledge the dyad (because it goes
undeviatingly from one point to another), opinion the number of the
plane, sensation the number of the solid; the numbers are by him expressly
identified with the Forms themselves or principles, and are formed
out of the elements; now things are apprehended either by mind or
science or opinion or sensation, and these same numbers are the Forms
of things. 

Some thinkers, accepting both premisses, viz. that the soul is both
originative of movement and cognitive, have compounded it of both
and declared the soul to be a self-moving number. 

As to the nature and number of the first principles opinions differ.
The difference is greatest between those who regard them as corporeal
and those who regard them as incorporeal, and from both dissent those
who make a blend and draw their principles from both sources. The
number of principles is also in dispute; some admit one only, others
assert several. There is a consequent diversity in their several accounts
of soul; they assume, naturally enough, that what is in its own nature
originative of movement must be among what is primordial. That has
led some to regard it as fire, for fire is the subtlest of the elements
and nearest to incorporeality; further, in the most primary sense,
fire both is moved and originates movement in all the others.

Democritus has expressed himself more ingeniously than the rest on
the grounds for ascribing each of these two characters to soul; soul
and mind are, he says, one and the same thing, and this thing must
be one of the primary and indivisible bodies, and its power of originating
movement must be due to its fineness of grain and the shape of its
atoms; he says that of all the shapes the spherical is the most mobile,
and that this is the shape of the particles of fire and mind.

Anaxagoras, as we said above, seems to distinguish between soul and
mind, but in practice he treats them as a single substance, except
that it is mind that he specially posits as the principle of all things;
at any rate what he says is that mind alone of all that is simple,
unmixed, and pure. He assigns both characteristics, knowing and origination
of movement, to the same principle, when he says that it was mind
that set the whole in movement. 

Thales, too, to judge from what is recorded about him, seems to have
held soul to be a motive force, since he said that the magnet has
a soul in it because it moves the iron. 

Diogenes (and others) held the soul to be air because he believed
air to be finest in grain and a first principle; therein lay the grounds
of the soul's powers of knowing and originating movement. As the primordial
principle from which all other things are derived, it is cognitive;
as finest in grain, it has the power to originate movement.

Heraclitus too says that the first principle-the 'warm exhalation'
of which, according to him, everything else is composed-is soul; further,
that this exhalation is most incorporeal and in ceaseless flux; that
what is in movement requires that what knows it should be in movement;
and that all that is has its being essentially in movement (herein
agreeing with the majority). 

Alcmaeon also seems to have held a similar view about soul; he says
that it is immortal because it resembles 'the immortals,' and that
this immortality belongs to it in virtue of its ceaseless movement;
for all the 'things divine,' moon, sun, the planets, and the whole
heavens, are in perpetual movement. 

of More superficial writers, some, e.g. Hippo, have pronounced it
to be water; they seem to have argued from the fact that the seed
of all animals is fluid, for Hippo tries to refute those who say that
the soul is blood, on the ground that the seed, which is the primordial
soul, is not blood. 

Another group (Critias, for example) did hold it to be blood; they
take perception to be the most characteristic attribute of soul, and
hold that perceptiveness is due to the nature of blood. 

Each of the elements has thus found its partisan, except earth-earth
has found no supporter unless we count as such those who have declared
soul to be, or to be compounded of, all the elements. All, then, it
may be said, characterize the soul by three marks, Movement, Sensation,
Incorporeality, and each of these is traced back to the first principles.
That is why (with one exception) all those who define the soul by
its power of knowing make it either an element or constructed out
of the elements. The language they all use is similar; like, they
say, is known by like; as the soul knows everything, they construct
it out of all the principles. Hence all those who admit but one cause
or element, make the soul also one (e.g. fire or air), while those
who admit a multiplicity of principles make the soul also multiple.
The exception is Anaxagoras; he alone says that mind is impassible
and has nothing in common with anything else. But, if this is so,
how or in virtue of what cause can it know? That Anaxagoras has not
explained, nor can any answer be inferred from his words. All who
acknowledge pairs of opposites among their principles, construct the
soul also out of these contraries, while those who admit as principles
only one contrary of each pair, e.g. either hot or cold, likewise
make the soul some one of these. That is why, also, they allow themselves
to be guided by the names; those who identify soul with the hot argue
that sen (to live) is derived from sein (to boil), while those who
identify it with the cold say that soul (psuche) is so called from
the process of respiration and (katapsuxis). Such are the traditional
opinions concerning soul, together with the grounds on which they
are maintained. 

Part 3

We must begin our examination with movement; for doubtless, not only
is it false that the essence of soul is correctly described by those
who say that it is what moves (or is capable of moving) itself, but
it is an impossibility that movement should be even an attribute of
it. 

We have already pointed out that there is no necessity that what originates
movement should itself be moved. There are two senses in which anything
may be moved-either (a) indirectly, owing to something other than
itself, or (b) directly, owing to itself. Things are 'indirectly moved'
which are moved as being contained in something which is moved, e.g.
sailors in a ship, for they are moved in a different sense from that
in which the ship is moved; the ship is 'directly moved', they are
'indirectly moved', because they are in a moving vessel. This is clear
if we consider their limbs; the movement proper to the legs (and so
to man) is walking, and in this case the sailors tare not walking.
Recognizing the double sense of 'being moved', what we have to consider
now is whether the soul is 'directly moved' and participates in such
direct movement. 

There are four species of movement-locomotion, alteration, diminution,
growth; consequently if the soul is moved, it must be moved with one
or several or all of these species of movement. Now if its movement
is not incidental, there must be a movement natural to it, and, if
so, as all the species enumerated involve place, place must be natural
to it. But if the essence of soul be to move itself, its being moved
cannot be incidental to-as it is to what is white or three cubits
long; they too can be moved, but only incidentally-what is moved is
that of which 'white' and 'three cubits long' are the attributes,
the body in which they inhere; hence they have no place: but if the
soul naturally partakes in movement, it follows that it must have
a place. 

Further, if there be a movement natural to the soul, there must be
a counter-movement unnatural to it, and conversely. The same applies
to rest as well as to movement; for the terminus ad quem of a thing's
natural movement is the place of its natural rest, and similarly the
terminus ad quem of its enforced movement is the place of its enforced
rest. But what meaning can be attached to enforced movements or rests
of the soul, it is difficult even to imagine. 

Further, if the natural movement of the soul be upward, the soul must
be fire; if downward, it must be earth; for upward and downward movements
are the definitory characteristics of these bodies. The same reasoning
applies to the intermediate movements, termini, and bodies. Further,
since the soul is observed to originate movement in the body, it is
reasonable to suppose that it transmits to the body the movements
by which it itself is moved, and so, reversing the order, we may infer
from the movements of the body back to similar movements of the soul.
Now the body is moved from place to place with movements of locomotion.
Hence it would follow that the soul too must in accordance with the
body change either its place as a whole or the relative places of
its parts. This carries with it the possibility that the soul might
even quit its body and re-enter it, and with this would be involved
the possibility of a resurrection of animals from the dead. But, it
may be contended, the soul can be moved indirectly by something else;
for an animal can be pushed out of its course. Yes, but that to whose
essence belongs the power of being moved by itself, cannot be moved
by something else except incidentally, just as what is good by or
in itself cannot owe its goodness to something external to it or to
some end to which it is a means. 

If the soul is moved, the most probable view is that what moves it
is sensible things. 

We must note also that, if the soul moves itself, it must be the mover
itself that is moved, so that it follows that if movement is in every
case a displacement of that which is in movement, in that respect
in which it is said to be moved, the movement of the soul must be
a departure from its essential nature, at least if its self-movement
is essential to it, not incidental. 

Some go so far as to hold that the movements which the soul imparts
to the body in which it is are the same in kind as those with which
it itself is moved. An example of this is Democritus, who uses language
like that of the comic dramatist Philippus, who accounts for the movements
that Daedalus imparted to his wooden Aphrodite by saying that he poured
quicksilver into it; similarly Democritus says that the spherical
atoms which according to him constitute soul, owing to their own ceaseless
movements draw the whole body after them and so produce its movements.
We must urge the question whether it is these very same atoms which
produce rest also-how they could do so, it is difficult and even impossible
to say. And, in general, we may object that it is not in this way
that the soul appears to originate movement in animals-it is through
intention or process of thinking. 

It is in the same fashion that the Timaeus also tries to give a physical
account of how the soul moves its body; the soul, it is there said,
is in movement, and so owing to their mutual implication moves the
body also. After compounding the soul-substance out of the elements
and dividing it in accordance with the harmonic numbers, in order
that it may possess a connate sensibility for 'harmony' and that the
whole may move in movements well attuned, the Demiurge bent the straight
line into a circle; this single circle he divided into two circles
united at two common points; one of these he subdivided into seven
circles. All this implies that the movements of the soul are identified
with the local movements of the heavens. 

Now, in the first place, it is a mistake to say that the soul is a
spatial magnitude. It is evident that Plato means the soul of the
whole to be like the sort of soul which is called mind not like the
sensitive or the desiderative soul, for the movements of neither of
these are circular. Now mind is one and continuous in the sense in
which the process of thinking is so, and thinking is identical with
the thoughts which are its parts; these have a serial unity like that
of number, not a unity like that of a spatial magnitude. Hence mind
cannot have that kind of unity either; mind is either without parts
or is continuous in some other way than that which characterizes a
spatial magnitude. How, indeed, if it were a spatial magnitude, could
mind possibly think? Will it think with any one indifferently of its
parts? In this case, the 'part' must be understood either in the sense
of a spatial magnitude or in the sense of a point (if a point can
be called a part of a spatial magnitude). If we accept the latter
alternative, the points being infinite in number, obviously the mind
can never exhaustively traverse them; if the former, the mind must
think the same thing over and over again, indeed an infinite number
of times (whereas it is manifestly possible to think a thing once
only). If contact of any part whatsoever of itself with the object
is all that is required, why need mind move in a circle, or indeed
possess magnitude at all? On the other hand, if contact with the whole
circle is necessary, what meaning can be given to the contact of the
parts? Further, how could what has no parts think what has parts,
or what has parts think what has none? We must identify the circle
referred to with mind; for it is mind whose movement is thinking,
and it is the circle whose movement is revolution, so that if thinking
is a movement of revolution, the circle which has this characteristic
movement must be mind. 

If the circular movement is eternal, there must be something which
mind is always thinking-what can this be? For all practical processes
of thinking have limits-they all go on for the sake of something outside
the process, and all theoretical processes come to a close in the
same way as the phrases in speech which express processes and results
of thinking. Every such linguistic phrase is either definitory or
demonstrative. Demonstration has both a starting-point and may be
said to end in a conclusion or inferred result; even if the process
never reaches final completion, at any rate it never returns upon
itself again to its starting-point, it goes on assuming a fresh middle
term or a fresh extreme, and moves straight forward, but circular
movement returns to its starting-point. Definitions, too, are closed
groups of terms. 

Further, if the same revolution is repeated, mind must repeatedly
think the same object. 

Further, thinking has more resemblance to a coming to rest or arrest
than to a movement; the same may be said of inferring. 

It might also be urged that what is difficult and enforced is incompatible
with blessedness; if the movement of the soul is not of its essence,
movement of the soul must be contrary to its nature. It must also
be painful for the soul to be inextricably bound up with the body;
nay more, if, as is frequently said and widely accepted, it is better
for mind not to be embodied, the union must be for it undesirable.

Further, the cause of the revolution of the heavens is left obscure.
It is not the essence of soul which is the cause of this circular
movement-that movement is only incidental to soul-nor is, a fortiori,
the body its cause. Again, it is not even asserted that it is better
that soul should be so moved; and yet the reason for which God caused
the soul to move in a circle can only have been that movement was
better for it than rest, and movement of this kind better than any
other. But since this sort of consideration is more appropriate to
another field of speculation, let us dismiss it for the present.

The view we have just been examining, in company with most theories
about the soul, involves the following absurdity: they all join the
soul to a body, or place it in a body, without adding any specification
of the reason of their union, or of the bodily conditions required
for it. Yet such explanation can scarcely be omitted; for some community
of nature is presupposed by the fact that the one acts and the other
is acted upon, the one moves and the other is moved; interaction always
implies a special nature in the two interagents. All, however, that
these thinkers do is to describe the specific characteristics of the
soul; they do not try to determine anything about the body which is
to contain it, as if it were possible, as in the Pythagorean myths,
that any soul could be clothed upon with any body-an absurd view,
for each body seems to have a form and shape of its own. It is as
absurd as to say that the art of carpentry could embody itself in
flutes; each art must use its tools, each soul its body.

Part 4

There is yet another theory about soul, which has commended itself
to many as no less probable than any of those we have hitherto mentioned,
and has rendered public account of itself in the court of popular
discussion. Its supporters say that the soul is a kind of harmony,
for (a) harmony is a blend or composition of contraries, and (b) the
body is compounded out of contraries. Harmony, however, is a certain
proportion or composition of the constituents blended, and soul can
be neither the one nor the other of these. Further, the power of originating
movement cannot belong to a harmony, while almost all concur in regarding
this as a principal attribute of soul. It is more appropriate to call
health (or generally one of the good states of the body) a harmony
than to predicate it of the soul. The absurdity becomes most apparent
when we try to attribute the active and passive affections of the
soul to a harmony; the necessary readjustment of their conceptions
is difficult. Further, in using the word 'harmony' we have one or
other of two cases in our mind; the most proper sense is in relation
to spatial magnitudes which have motion and position, where harmony
means the disposition and cohesion of their parts in such a manner
as to prevent the introduction into the whole of anything homogeneous
with it, and the secondary sense, derived from the former, is that
in which it means the ratio between the constituents so blended; in
neither of these senses is it plausible to predicate it of soul. That
soul is a harmony in the sense of the mode of composition of the parts
of the body is a view easily refutable; for there are many composite
parts and those variously compounded; of what bodily part is mind
or the sensitive or the appetitive faculty the mode of composition?
And what is the mode of composition which constitutes each of them?
It is equally absurd to identify the soul with the ratio of the mixture;
for the mixture which makes flesh has a different ratio between the
elements from that which makes bone. The consequence of this view
will therefore be that distributed throughout the whole body there
will be many souls, since every one of the bodily parts is a different
mixture of the elements, and the ratio of mixture is in each case
a harmony, i.e. a soul. 

From Empedocles at any rate we might demand an answer to the following
question for he says that each of the parts of the body is what it
is in virtue of a ratio between the elements: is the soul identical
with this ratio, or is it not rather something over and above this
which is formed in the parts? Is love the cause of any and every mixture,
or only of those that are in the right ratio? Is love this ratio itself,
or is love something over and above this? Such are the problems raised
by this account. But, on the other hand, if the soul is different
from the mixture, why does it disappear at one and the same moment
with that relation between the elements which constitutes flesh or
the other parts of the animal body? Further, if the soul is not identical
with the ratio of mixture, and it is consequently not the case that
each of the parts has a soul, what is that which perishes when the
soul quits the body? 

That the soul cannot either be a harmony, or be moved in a circle,
is clear from what we have said. Yet that it can be moved incidentally
is, as we said above, possible, and even that in a sense it can move
itself, i.e. in the sense that the vehicle in which it is can be moved,
and moved by it; in no other sense can the soul be moved in space.

More legitimate doubts might remain as to its movement in view of
the following facts. We speak of the soul as being pained or pleased,
being bold or fearful, being angry, perceiving, thinking. All these
are regarded as modes of movement, and hence it might be inferred
that the soul is moved. This, however, does not necessarily follow.
We may admit to the full that being pained or pleased, or thinking,
are movements (each of them a 'being moved'), and that the movement
is originated by the soul. For example we may regard anger or fear
as such and such movements of the heart, and thinking as such and
such another movement of that organ, or of some other; these modifications
may arise either from changes of place in certain parts or from qualitative
alterations (the special nature of the parts and the special modes
of their changes being for our present purpose irrelevant). Yet to
say that it is the soul which is angry is as inexact as it would be
to say that it is the soul that weaves webs or builds houses. It is
doubtless better to avoid saying that the soul pities or learns or
thinks and rather to say that it is the man who does this with his
soul. What we mean is not that the movement is in the soul, but that
sometimes it terminates in the soul and sometimes starts from it,
sensation e.g. coming from without inwards, and reminiscence starting
from the soul and terminating with the movements, actual or residual,
in the sense organs. 

The case of mind is different; it seems to be an independent substance
implanted within the soul and to be incapable of being destroyed.
If it could be destroyed at all, it would be under the blunting influence
of old age. What really happens in respect of mind in old age is,
however, exactly parallel to what happens in the case of the sense
organs; if the old man could recover the proper kind of eye, he would
see just as well as the young man. The incapacity of old age is due
to an affection not of the soul but of its vehicle, as occurs in drunkenness
or disease. Thus it is that in old age the activity of mind or intellectual
apprehension declines only through the decay of some other inward
part; mind itself is impassible. Thinking, loving, and hating are
affections not of mind, but of that which has mind, so far as it has
it. That is why, when this vehicle decays, memory and love cease;
they were activities not of mind, but of the composite which has perished;
mind is, no doubt, something more divine and impassible. That the
soul cannot be moved is therefore clear from what we have said, and
if it cannot be moved at all, manifestly it cannot be moved by itself.

Of all the opinions we have enumerated, by far the most unreasonable
is that which declares the soul to be a self-moving number; it involves
in the first place all the impossibilities which follow from regarding
the soul as moved, and in the second special absurdities which follow
from calling it a number. How we to imagine a unit being moved? By
what agency? What sort of movement can be attributed to what is without
parts or internal differences? If the unit is both originative of
movement and itself capable of being moved, it must contain difference.

Further, since they say a moving line generates a surface and a moving
point a line, the movements of the psychic units must be lines (for
a point is a unit having position, and the number of the soul is,
of course, somewhere and has position). 

Again, if from a number a number or a unit is subtracted, the remainder
is another number; but plants and many animals when divided continue
to live, and each segment is thought to retain the same kind of soul.

It must be all the same whether we speak of units or corpuscles; for
if the spherical atoms of Democritus became points, nothing being
retained but their being a quantum, there must remain in each a moving
and a moved part, just as there is in what is continuous; what happens
has nothing to do with the size of the atoms, it depends solely upon
their being a quantum. That is why there must be something to originate
movement in the units. If in the animal what originates movement is
the soul, so also must it be in the case of the number, so that not
the mover and the moved together, but the mover only, will be the
soul. But how is it possible for one of the units to fulfil this function
of originating movement? There must be some difference between such
a unit and all the other units, and what difference can there be between
one placed unit and another except a difference of position? If then,
on the other hand, these psychic units within the body are different
from the points of the body, there will be two sets of units both
occupying the same place; for each unit will occupy a point. And yet,
if there can be two, why cannot there be an infinite number? For if
things can occupy an indivisible lace, they must themselves be indivisible.
If, on the other hand, the points of the body are identical with the
units whose number is the soul, or if the number of the points in
the body is the soul, why have not all bodies souls? For all bodies
contain points or an infinity of points. 

Further, how is it possible for these points to be isolated or separated
from their bodies, seeing that lines cannot be resolved into points?

Part 5

The result is, as we have said, that this view, while on the one side
identical with that of those who maintain that soul is a subtle kind
of body, is on the other entangled in the absurdity peculiar to Democritus'
way of describing the manner in which movement is originated by soul.
For if the soul is present throughout the whole percipient body, there
must, if the soul be a kind of body, be two bodies in the same place;
and for those who call it a number, there must be many points at one
point, or every body must have a soul, unless the soul be a different
sort of number-other, that is, than the sum of the points existing
in a body. Another consequence that follows is that the animal must
be moved by its number precisely in the way that Democritus explained
its being moved by his spherical psychic atoms. What difference does
it make whether we speak of small spheres or of large units, or, quite
simply, of units in movement? One way or another, the movements of
the animal must be due to their movements. Hence those who combine
movement and number in the same subject lay themselves open to these
and many other similar absurdities. It is impossible not only that
these characters should give the definition of soul-it is impossible
that they should even be attributes of it. The point is clear if the
attempt be made to start from this as the account of soul and explain
from it the affections and actions of the soul, e.g. reasoning, sensation,
pleasure, pain, &c. For, to repeat what we have said earlier, movement
and number do not facilitate even conjecture about the derivative
properties of soul. 

Such are the three ways in which soul has traditionally been defined;
one group of thinkers declared it to be that which is most originative
of movement because it moves itself, another group to be the subtlest
and most nearly incorporeal of all kinds of body. We have now sufficiently
set forth the difficulties and inconsistencies to which these theories
are exposed. It remains now to examine the doctrine that soul is composed
of the elements. 

The reason assigned for this doctrine is that thus the soul may perceive
or come to know everything that is, but the theory necessarily involves
itself in many impossibilities. Its upholders assume that like is
known only by like, and imagine that by declaring the soul to be composed
of the elements they succeed in identifying the soul with all the
things it is capable of apprehending. But the elements are not the
only things it knows; there are many others, or, more exactly, an
infinite number of others, formed out of the elements. Let us admit
that the soul knows or perceives the elements out of which each of
these composites is made up; but by what means will it know or perceive
the composite whole, e.g. what God, man, flesh, bone (or any other
compound) is? For each is, not merely the elements of which it is
composed, but those elements combined in a determinate mode or ratio,
as Empedocles himself says of bone, 

The kindly Earth in its broad-bosomed moulds 

Won of clear Water two parts out of eight, And four of Fire; and so
white bones were formed. 

Nothing, therefore, will be gained by the presence of the elements
in the soul, unless there be also present there the various formulae
of proportion and the various compositions in accordance with them.
Each element will indeed know its fellow outside, but there will be
no knowledge of bone or man, unless they too are present in the constitution
of the soul. The impossibility of this needs no pointing out; for
who would suggest that stone or man could enter into the constitution
of the soul? The same applies to 'the good' and 'the not-good', and
so on. 

Further, the word 'is' has many meanings: it may be used of a 'this'
or substance, or of a quantum, or of a quale, or of any other of the
kinds of predicates we have distinguished. Does the soul consist of
all of these or not? It does not appear that all have common elements.
Is the soul formed out of those elements alone which enter into substances?
so how will it be able to know each of the other kinds of thing? Will
it be said that each kind of thing has elements or principles of its
own, and that the soul is formed out of the whole of these? In that
case, the soul must be a quantum and a quale and a substance. But
all that can be made out of the elements of a quantum is a quantum,
not a substance. These (and others like them) are the consequences
of the view that the soul is composed of all the elements.

It is absurd, also, to say both (a) that like is not capable of being
affected by like, and (b) that like is perceived or known by like,
for perceiving, and also both thinking and knowing, are, on their
own assumption, ways of being affected or moved. 

There are many puzzles and difficulties raised by saying, as Empedocles
does, that each set of things is known by means of its corporeal elements
and by reference to something in soul which is like them, and additional
testimony is furnished by this new consideration; for all the parts
of the animal body which consist wholly of earth such as bones, sinews,
and hair seem to be wholly insensitive and consequently not perceptive
even of objects earthy like themselves, as they ought to have been.

Further, each of the principles will have far more ignorance than
knowledge, for though each of them will know one thing, there will
be many of which it will be ignorant. Empedocles at any rate must
conclude that his God is the least intelligent of all beings, for
of him alone is it true that there is one thing, Strife, which he
does not know, while there is nothing which mortal beings do not know,
for ere is nothing which does not enter into their composition.

In general, we may ask, Why has not everything a soul, since everything
either is an element, or is formed out of one or several or all of
the elements? Each must certainly know one or several or all.

The problem might also be raised, What is that which unifies the elements
into a soul? The elements correspond, it would appear, to the matter;
what unites them, whatever it is, is the supremely important factor.
But it is impossible that there should be something superior to, and
dominant over, the soul (and a fortiori over the mind); it is reasonable
to hold that mind is by nature most primordial and dominant, while
their statement that it is the elements which are first of all that
is. 

All, both those who assert that the soul, because of its knowledge
or perception of what is compounded out of the elements, and is those
who assert that it is of all things the most originative of movement,
fail to take into consideration all kinds of soul. In fact (1) not
all beings that perceive can originate movement; there appear to be
certain animals which stationary, and yet local movement is the only
one, so it seems, which the soul originates in animals. And (2) the
same object-on holds against all those who construct mind and the
perceptive faculty out of the elements; for it appears that plants
live, and yet are not endowed with locomotion or perception, while
a large number of animals are without discourse of reason. Even if
these points were waived and mind admitted to be a part of the soul
(and so too the perceptive faculty), still, even so, there would be
kinds and parts of soul of which they had failed to give any account.

The same objection lies against the view expressed in the 'Orphic'
poems: there it is said that the soul comes in from the whole when
breathing takes place, being borne in upon the winds. Now this cannot
take place in the case of plants, nor indeed in the case of certain
classes of animal, for not all classes of animal breathe. This fact
has escaped the notice of the holders of this view. 

If we must construct the soul out of the elements, there is no necessity
to suppose that all the elements enter into its construction; one
element in each pair of contraries will suffice to enable it to know
both that element itself and its contrary. By means of the straight
line we know both itself and the curved-the carpenter's rule enables
us to test both-but what is curved does not enable us to distinguish
either itself or the straight. Certain thinkers say that soul is intermingled
in the whole universe, and it is perhaps for that reason that Thales
came to the opinion that all things are full of gods. This presents
some difficulties: Why does the soul when it resides in air or fire
not form an animal, while it does so when it resides in mixtures of
the elements, and that although it is held to be of higher quality
when contained in the former? (One might add the question, why the
soul in air is maintained to be higher and more immortal than that
in animals.) Both possible ways of replying to the former question
lead to absurdity or paradox; for it is beyond paradox to say that
fire or air is an animal, and it is absurd to refuse the name of animal
to what has soul in it. The opinion that the elements have soul in
them seems to have arisen from the doctrine that a whole must be homogeneous
with its parts. If it is true that animals become animate by drawing
into themselves a portion of what surrounds them, the partisans of
this view are bound to say that the soul of the Whole too is homogeneous
with all its parts. If the air sucked in is homogeneous, but soul
heterogeneous, clearly while some part of soul will exist in the inbreathed
air, some other part will not. The soul must either be homogeneous,
or such that there are some parts of the Whole in which it is not
to be found. 

From what has been said it is now clear that knowing as an attribute
of soul cannot be explained by soul's being composed of the elements,
and that it is neither sound nor true to speak of soul as moved. But
since (a) knowing, perceiving, opining, and further (b) desiring,
wishing, and generally all other modes of appetition, belong to soul,
and (c) the local movements of animals, and (d) growth, maturity,
and decay are produced by the soul, we must ask whether each of these
is an attribute of the soul as a whole, i.e. whether it is with the
whole soul we think, perceive, move ourselves, act or are acted upon,
or whether each of them requires a different part of the soul? So
too with regard to life. Does it depend on one of the parts of soul?
Or is it dependent on more than one? Or on all? Or has it some quite
other cause? 

Some hold that the soul is divisible, and that one part thinks, another
desires. If, then, its nature admits of its being divided, what can
it be that holds the parts together? Surely not the body; on the contrary
it seems rather to be the soul that holds the body together; at any
rate when the soul departs the body disintegrates and decays. If,
then, there is something else which makes the soul one, this unifying
agency would have the best right to the name of soul, and we shall
have to repeat for it the question: Is it one or multipartite? If
it is one, why not at once admit that 'the soul' is one? If it has
parts, once more the question must be put: What holds its parts together,
and so ad infinitum? 

The question might also be raised about the parts of the soul: What
is the separate role of each in relation to the body? For, if the
whole soul holds together the whole body, we should expect each part
of the soul to hold together a part of the body. But this seems an
impossibility; it is difficult even to imagine what sort of bodily
part mind will hold together, or how it will do this. 

It is a fact of observation that plants and certain insects go on
living when divided into segments; this means that each of the segments
has a soul in it identical in species, though not numerically identical
in the different segments, for both of the segments for a time possess
the power of sensation and local movement. That this does not last
is not surprising, for they no longer possess the organs necessary
for self-maintenance. But, all the same, in each of the bodily parts
there are present all the parts of soul, and the souls so present
are homogeneous with one another and with the whole; this means that
the several parts of the soul are indisseverable from one another,
although the whole soul is divisible. It seems also that the principle
found in plants is also a kind of soul; for this is the only principle
which is common to both animals and plants; and this exists in isolation
from the principle of sensation, though there nothing which has the
latter without the former. 

----------------------------------------------------------------------

BOOK II

Part 1 

Let the foregoing suffice as our account of the views concerning
the soul which have been handed on by our predecessors; let us now
dismiss them and make as it were a completely fresh start, endeavouring
to give a precise answer to the question, What is soul? i.e. to formulate
the most general possible definition of it. 

We are in the habit of recognizing, as one determinate kind of what
is, substance, and that in several senses, (a) in the sense of matter
or that which in itself is not 'a this', and (b) in the sense of form
or essence, which is that precisely in virtue of which a thing is
called 'a this', and thirdly (c) in the sense of that which is compounded
of both (a) and (b). Now matter is potentiality, form actuality; of
the latter there are two grades related to one another as e.g. knowledge
to the exercise of knowledge. 

Among substances are by general consent reckoned bodies and especially
natural bodies; for they are the principles of all other bodies. Of
natural bodies some have life in them, others not; by life we mean
self-nutrition and growth (with its correlative decay). It follows
that every natural body which has life in it is a substance in the
sense of a composite. 

But since it is also a body of such and such a kind, viz. having life,
the body cannot be soul; the body is the subject or matter, not what
is attributed to it. Hence the soul must be a substance in the sense
of the form of a natural body having life potentially within it. But
substance is actuality, and thus soul is the actuality of a body as
above characterized. Now the word actuality has two senses corresponding
respectively to the possession of knowledge and the actual exercise
of knowledge. It is obvious that the soul is actuality in the first
sense, viz. that of knowledge as possessed, for both sleeping and
waking presuppose the existence of soul, and of these waking corresponds
to actual knowing, sleeping to knowledge possessed but not employed,
and, in the history of the individual, knowledge comes before its
employment or exercise. 

That is why the soul is the first grade of actuality of a natural
body having life potentially in it. The body so described is a body
which is organized. The parts of plants in spite of their extreme
simplicity are 'organs'; e.g. the leaf serves to shelter the pericarp,
the pericarp to shelter the fruit, while the roots of plants are analogous
to the mouth of animals, both serving for the absorption of food.
If, then, we have to give a general formula applicable to all kinds
of soul, we must describe it as the first grade of actuality of a
natural organized body. That is why we can wholly dismiss as unnecessary
the question whether the soul and the body are one: it is as meaningless
as to ask whether the wax and the shape given to it by the stamp are
one, or generally the matter of a thing and that of which it is the
matter. Unity has many senses (as many as 'is' has), but the most
proper and fundamental sense of both is the relation of an actuality
to that of which it is the actuality. We have now given an answer
to the question, What is soul?-an answer which applies to it in its
full extent. It is substance in the sense which corresponds to the
definitive formula of a thing's essence. That means that it is 'the
essential whatness' of a body of the character just assigned. Suppose
that what is literally an 'organ', like an axe, were a natural body,
its 'essential whatness', would have been its essence, and so its
soul; if this disappeared from it, it would have ceased to be an axe,
except in name. As it is, it is just an axe; it wants the character
which is required to make its whatness or formulable essence a soul;
for that, it would have had to be a natural body of a particular kind,
viz. one having in itself the power of setting itself in movement
and arresting itself. Next, apply this doctrine in the case of the
'parts' of the living body. Suppose that the eye were an animal-sight
would have been its soul, for sight is the substance or essence of
the eye which corresponds to the formula, the eye being merely the
matter of seeing; when seeing is removed the eye is no longer an eye,
except in name-it is no more a real eye than the eye of a statue or
of a painted figure. We must now extend our consideration from the
'parts' to the whole living body; for what the departmental sense
is to the bodily part which is its organ, that the whole faculty of
sense is to the whole sensitive body as such. 

We must not understand by that which is 'potentially capable of living'
what has lost the soul it had, but only what still retains it; but
seeds and fruits are bodies which possess the qualification. Consequently,
while waking is actuality in a sense corresponding to the cutting
and the seeing, the soul is actuality in the sense corresponding to
the power of sight and the power in the tool; the body corresponds
to what exists in potentiality; as the pupil plus the power of sight
constitutes the eye, so the soul plus the body constitutes the animal.

From this it indubitably follows that the soul is inseparable from
its body, or at any rate that certain parts of it are (if it has parts)
for the actuality of some of them is nothing but the actualities of
their bodily parts. Yet some may be separable because they are not
the actualities of any body at all. Further, we have no light on the
problem whether the soul may not be the actuality of its body in the
sense in which the sailor is the actuality of the ship. 

This must suffice as our sketch or outline determination of the nature
of soul. 

Part 2

Since what is clear or logically more evident emerges from what in
itself is confused but more observable by us, we must reconsider our
results from this point of view. For it is not enough for a definitive
formula to express as most now do the mere fact; it must include and
exhibit the ground also. At present definitions are given in a form
analogous to the conclusion of a syllogism; e.g. What is squaring?
The construction of an equilateral rectangle equal to a given oblong
rectangle. Such a definition is in form equivalent to a conclusion.
One that tells us that squaring is the discovery of a line which is
a mean proportional between the two unequal sides of the given rectangle
discloses the ground of what is defined. 

We resume our inquiry from a fresh starting-point by calling attention
to the fact that what has soul in it differs from what has not, in
that the former displays life. Now this word has more than one sense,
and provided any one alone of these is found in a thing we say that
thing is living. Living, that is, may mean thinking or perception
or local movement and rest, or movement in the sense of nutrition,
decay and growth. Hence we think of plants also as living, for they
are observed to possess in themselves an originative power through
which they increase or decrease in all spatial directions; they grow
up and down, and everything that grows increases its bulk alike in
both directions or indeed in all, and continues to live so long as
it can absorb nutriment. 

This power of self-nutrition can be isolated from the other powers
mentioned, but not they from it-in mortal beings at least. The fact
is obvious in plants; for it is the only psychic power they possess.

This is the originative power the possession of which leads us to
speak of things as living at all, but it is the possession of sensation
that leads us for the first time to speak of living things as animals;
for even those beings which possess no power of local movement but
do possess the power of sensation we call animals and not merely living
things. 

The primary form of sense is touch, which belongs to all animals.
just as the power of self-nutrition can be isolated from touch and
sensation generally, so touch can be isolated from all other forms
of sense. (By the power of self-nutrition we mean that departmental
power of the soul which is common to plants and animals: all animals
whatsoever are observed to have the sense of touch.) What the explanation
of these two facts is, we must discuss later. At present we must confine
ourselves to saying that soul is the source of these phenomena and
is characterized by them, viz. by the powers of self-nutrition, sensation,
thinking, and motivity. 

Is each of these a soul or a part of a soul? And if a part, a part
in what sense? A part merely distinguishable by definition or a part
distinct in local situation as well? In the case of certain of these
powers, the answers to these questions are easy, in the case of others
we are puzzled what to say. just as in the case of plants which when
divided are observed to continue to live though removed to a distance
from one another (thus showing that in their case the soul of each
individual plant before division was actually one, potentially many),
so we notice a similar result in other varieties of soul, i.e. in
insects which have been cut in two; each of the segments possesses
both sensation and local movement; and if sensation, necessarily also
imagination and appetition; for, where there is sensation, there is
also pleasure and pain, and, where these, necessarily also desire.

We have no evidence as yet about mind or the power to think; it seems
to be a widely different kind of soul, differing as what is eternal
from what is perishable; it alone is capable of existence in isolation
from all other psychic powers. All the other parts of soul, it is
evident from what we have said, are, in spite of certain statements
to the contrary, incapable of separate existence though, of course,
distinguishable by definition. If opining is distinct from perceiving,
to be capable of opining and to be capable of perceiving must be distinct,
and so with all the other forms of living above enumerated. Further,
some animals possess all these parts of soul, some certain of them
only, others one only (this is what enables us to classify animals);
the cause must be considered later.' A similar arrangement is found
also within the field of the senses; some classes of animals have
all the senses, some only certain of them, others only one, the most
indispensable, touch. 

Since the expression 'that whereby we live and perceive' has two meanings,
just like the expression 'that whereby we know'-that may mean either
(a) knowledge or (b) the soul, for we can speak of knowing by or with
either, and similarly that whereby we are in health may be either
(a) health or (b) the body or some part of the body; and since of
the two terms thus contrasted knowledge or health is the name of a
form, essence, or ratio, or if we so express it an actuality of a
recipient matter-knowledge of what is capable of knowing, health of
what is capable of being made healthy (for the operation of that which
is capable of originating change terminates and has its seat in what
is changed or altered); further, since it is the soul by or with which
primarily we live, perceive, and think:-it follows that the soul must
be a ratio or formulable essence, not a matter or subject. For, as
we said, word substance has three meanings form, matter, and the complex
of both and of these three what is called matter is potentiality,
what is called form actuality. Since then the complex here is the
living thing, the body cannot be the actuality of the soul; it is
the soul which is the actuality of a certain kind of body. Hence the
rightness of the view that the soul cannot be without a body, while
it csnnot he a body; it is not a body but something relative to a
body. That is why it is in a body, and a body of a definite kind.
It was a mistake, therefore, to do as former thinkers did, merely
to fit it into a body without adding a definite specification of the
kind or character of that body. Reflection confirms the observed fact;
the actuality of any given thing can only be realized in what is already
potentially that thing, i.e. in a matter of its own appropriate to
it. From all this it follows that soul is an actuality or formulable
essence of something that possesses a potentiality of being besouled.

Part 3

Of the psychic powers above enumerated some kinds of living things,
as we have said, possess all, some less than all, others one only.
Those we have mentioned are the nutritive, the appetitive, the sensory,
the locomotive, and the power of thinking. Plants have none but the
first, the nutritive, while another order of living things has this
plus the sensory. If any order of living things has the sensory, it
must also have the appetitive; for appetite is the genus of which
desire, passion, and wish are the species; now all animals have one
sense at least, viz. touch, and whatever has a sense has the capacity
for pleasure and pain and therefore has pleasant and painful objects
present to it, and wherever these are present, there is desire, for
desire is just appetition of what is pleasant. Further, all animals
have the sense for food (for touch is the sense for food); the food
of all living things consists of what is dry, moist, hot, cold, and
these are the qualities apprehended by touch; all other sensible qualities
are apprehended by touch only indirectly. Sounds, colours, and odours
contribute nothing to nutriment; flavours fall within the field of
tangible qualities. Hunger and thirst are forms of desire, hunger
a desire for what is dry and hot, thirst a desire for what is cold
and moist; flavour is a sort of seasoning added to both. We must later
clear up these points, but at present it may be enough to say that
all animals that possess the sense of touch have also appetition.
The case of imagination is obscure; we must examine it later. Certain
kinds of animals possess in addition the power of locomotion, and
still another order of animate beings, i.e. man and possibly another
order like man or superior to him, the power of thinking, i.e. mind.
It is now evident that a single definition can be given of soul only
in the same sense as one can be given of figure. For, as in that case
there is no figure distinguishable and apart from triangle, &c., so
here there is no soul apart from the forms of soul just enumerated.
It is true that a highly general definition can be given for figure
which will fit all figures without expressing the peculiar nature
of any figure. So here in the case of soul and its specific forms.
Hence it is absurd in this and similar cases to demand an absolutely
general definition which will fail to express the peculiar nature
of anything that is, or again, omitting this, to look for separate
definitions corresponding to each infima species. The cases of figure
and soul are exactly parallel; for the particulars subsumed under
the common name in both cases-figures and living beings-constitute
a series, each successive term of which potentially contains its predecessor,
e.g. the square the triangle, the sensory power the self-nutritive.
Hence we must ask in the case of each order of living things, What
is its soul, i.e. What is the soul of plant, animal, man? Why the
terms are related in this serial way must form the subject of later
examination. But the facts are that the power of perception is never
found apart from the power of self-nutrition, while-in plants-the
latter is found isolated from the former. Again, no sense is found
apart from that of touch, while touch is found by itself; many animals
have neither sight, hearing, nor smell. Again, among living things
that possess sense some have the power of locomotion, some not. Lastly,
certain living beings-a small minority-possess calculation and thought,
for (among mortal beings) those which possess calculation have all
the other powers above mentioned, while the converse does not hold-indeed
some live by imagination alone, while others have not even imagination.
The mind that knows with immediate intuition presents a different
problem. 

It is evident that the way to give the most adequate definition of
soul is to seek in the case of each of its forms for the most appropriate
definition. 

Part 4

It is necessary for the student of these forms of soul first to find
a definition of each, expressive of what it is, and then to investigate
its derivative properties, &c. But if we are to express what each
is, viz. what the thinking power is, or the perceptive, or the nutritive,
we must go farther back and first give an account of thinking or perceiving,
for in the order of investigation the question of what an agent does
precedes the question, what enables it to do what it does. If this
is correct, we must on the same ground go yet another step farther
back and have some clear view of the objects of each; thus we must
start with these objects, e.g. with food, with what is perceptible,
or with what is intelligible. 

It follows that first of all we must treat of nutrition and reproduction,
for the nutritive soul is found along with all the others and is the
most primitive and widely distributed power of soul, being indeed
that one in virtue of which all are said to have life. The acts in
which it manifests itself are reproduction and the use of food-reproduction,
I say, because for any living thing that has reached its normal development
and which is unmutilated, and whose mode of generation is not spontaneous,
the most natural act is the production of another like itself, an
animal producing an animal, a plant a plant, in order that, as far
as its nature allows, it may partake in the eternal and divine. That
is the goal towards which all things strive, that for the sake of
which they do whatsoever their nature renders possible. The phrase
'for the sake of which' is ambiguous; it may mean either (a) the end
to achieve which, or (b) the being in whose interest, the act is done.
Since then no living thing is able to partake in what is eternal and
divine by uninterrupted continuance (for nothing perishable can for
ever remain one and the same), it tries to achieve that end in the
only way possible to it, and success is possible in varying degrees;
so it remains not indeed as the self-same individual but continues
its existence in something like itself-not numerically but specifically
one. 

The soul is the cause or source of the living body. The terms cause
and source have many senses. But the soul is the cause of its body
alike in all three senses which we explicitly recognize. It is (a)
the source or origin of movement, it is (b) the end, it is (c) the
essence of the whole living body. 

That it is the last, is clear; for in everything the essence is identical
with the ground of its being, and here, in the case of living things,
their being is to live, and of their being and their living the soul
in them is the cause or source. Further, the actuality of whatever
is potential is identical with its formulable essence. 

It is manifest that the soul is also the final cause of its body.
For Nature, like mind, always does whatever it does for the sake of
something, which something is its end. To that something corresponds
in the case of animals the soul and in this it follows the order of
nature; all natural bodies are organs of the soul. This is true of
those that enter into the constitution of plants as well as of those
which enter into that of animals. This shows that that the sake of
which they are is soul. We must here recall the two senses of 'that
for the sake of which', viz. (a) the end to achieve which, and (b)
the being in whose interest, anything is or is done. 

We must maintain, further, that the soul is also the cause of the
living body as the original source of local movement. The power of
locomotion is not found, however, in all living things. But change
of quality and change of quantity are also due to the soul. Sensation
is held to be a qualitative alteration, and nothing except what has
soul in it is capable of sensation. The same holds of the quantitative
changes which constitute growth and decay; nothing grows or decays
naturally except what feeds itself, and nothing feeds itself except
what has a share of soul in it. 

Empedocles is wrong in adding that growth in plants is to be explained,
the downward rooting by the natural tendency of earth to travel downwards,
and the upward branching by the similar natural tendency of fire to
travel upwards. For he misinterprets up and down; up and down are
not for all things what they are for the whole Cosmos: if we are to
distinguish and identify organs according to their functions, the
roots of plants are analogous to the head in animals. Further, we
must ask what is the force that holds together the earth and the fire
which tend to travel in contrary directions; if there is no counteracting
force, they will be torn asunder; if there is, this must be the soul
and the cause of nutrition and growth. By some the element of fire
is held to be the cause of nutrition and growth, for it alone of the
primary bodies or elements is observed to feed and increase itself.
Hence the suggestion that in both plants and animals it is it which
is the operative force. A concurrent cause in a sense it certainly
is, but not the principal cause, that is rather the soul; for while
the growth of fire goes on without limit so long as there is a supply
of fuel, in the case of all complex wholes formed in the course of
nature there is a limit or ratio which determines their size and increase,
and limit and ratio are marks of soul but not of fire, and belong
to the side of formulable essence rather than that of matter.

Nutrition and reproduction are due to one and the same psychic power.
It is necessary first to give precision to our account of food, for
it is by this function of absorbing food that this psychic power is
distinguished from all the others. The current view is that what serves
as food to a living thing is what is contrary to it-not that in every
pair of contraries each is food to the other: to be food a contrary
must not only be transformable into the other and vice versa, it must
also in so doing increase the bulk of the other. Many a contrary is
transformed into its other and vice versa, where neither is even a
quantum and so cannot increase in bulk, e.g. an invalid into a healthy
subject. It is clear that not even those contraries which satisfy
both the conditions mentioned above are food to one another in precisely
the same sense; water may be said to feed fire, but not fire water.
Where the members of the pair are elementary bodies only one of the
contraries, it would appear, can be said to feed the other. But there
is a difficulty here. One set of thinkers assert that like fed, as
well as increased in amount, by like. Another set, as we have said,
maintain the very reverse, viz. that what feeds and what is fed are
contrary to one another; like, they argue, is incapable of being affected
by like; but food is changed in the process of digestion, and change
is always to what is opposite or to what is intermediate. Further,
food is acted upon by what is nourished by it, not the other way round,
as timber is worked by a carpenter and not conversely; there is a
change in the carpenter but it is merely a change from not-working
to working. In answering this problem it makes all the difference
whether we mean by 'the food' the 'finished' or the 'raw' product.
If we use the word food of both, viz. of the completely undigested
and the completely digested matter, we can justify both the rival
accounts of it; taking food in the sense of undigested matter, it
is the contrary of what is fed by it, taking it as digested it is
like what is fed by it. Consequently it is clear that in a certain
sense we may say that both parties are right, both wrong.

Since nothing except what is alive can be fed, what is fed is the
besouled body and just because it has soul in it. Hence food is essentially
related to what has soul in it. Food has a power which is other than
the power to increase the bulk of what is fed by it; so far forth
as what has soul in it is a quantum, food may increase its quantity,
but it is only so far as what has soul in it is a 'this-somewhat'
or substance that food acts as food; in that case it maintains the
being of what is fed, and that continues to be what it is so long
as the process of nutrition continues. Further, it is the agent in
generation, i.e. not the generation of the individual fed but the
reproduction of another like it; the substance of the individual fed
is already in existence; the existence of no substance is a self-generation
but only a self-maintenance. 

Hence the psychic power which we are now studying may be described
as that which tends to maintain whatever has this power in it of continuing
such as it was, and food helps it to do its work. That is why, if
deprived of food, it must cease to be. 

The process of nutrition involves three factors, (a) what is fed,
(b) that wherewith it is fed, (c) what does the feeding; of these
(c) is the first soul, (a) the body which has that soul in it, (b)
the food. But since it is right to call things after the ends they
realize, and the end of this soul is to generate another being like
that in which it is, the first soul ought to be named the reproductive
soul. The expression (b) 'wherewith it is fed' is ambiguous just as
is the expression 'wherewith the ship is steered'; that may mean either
(i) the hand or (ii) the rudder, i.e. either (i) what is moved and
sets in movement, or (ii) what is merely moved. We can apply this
analogy here if we recall that all food must be capable of being digested,
and that what produces digestion is warmth; that is why everything
that has soul in it possesses warmth. 

We have now given an outline account of the nature of food; further
details must be given in the appropriate place. 

Part 5

Having made these distinctions let us now speak of sensation in the
widest sense. Sensation depends, as we have said, on a process of
movement or affection from without, for it is held to be some sort
of change of quality. Now some thinkers assert that like is affected
only by like; in what sense this is possible and in what sense impossible,
we have explained in our general discussion of acting and being acted
upon. 

Here arises a problem: why do we not perceive the senses themselves
as well as the external objects of sense, or why without the stimulation
of external objects do they not produce sensation, seeing that they
contain in themselves fire, earth, and all the other elements, which
are the direct or indirect objects is so of sense? It is clear that
what is sensitive is only potentially, not actually. The power of
sense is parallel to what is combustible, for that never ignites itself
spontaneously, but requires an agent which has the power of starting
ignition; otherwise it could have set itself on fire, and would not
have needed actual fire to set it ablaze. 

In reply we must recall that we use the word 'perceive' in two ways,
for we say (a) that what has the power to hear or see, 'sees' or 'hears',
even though it is at the moment asleep, and also (b) that what is
actually seeing or hearing, 'sees' or 'hears'. Hence 'sense' too must
have two meanings, sense potential, and sense actual. Similarly 'to
be a sentient' means either (a) to have a certain power or (b) to
manifest a certain activity. To begin with, for a time, let us speak
as if there were no difference between (i) being moved or affected,
and (ii) being active, for movement is a kind of activity-an imperfect
kind, as has elsewhere been explained. Everything that is acted upon
or moved is acted upon by an agent which is actually at work. Hence
it is that in one sense, as has already been stated, what acts and
what is acted upon are like, in another unlike, i.e. prior to and
during the change the two factors are unlike, after it like.

But we must now distinguish not only between what is potential and
what is actual but also different senses in which things can be said
to be potential or actual; up to now we have been speaking as if each
of these phrases had only one sense. We can speak of something as
'a knower' either (a) as when we say that man is a knower, meaning
that man falls within the class of beings that know or have knowledge,
or (b) as when we are speaking of a man who possesses a knowledge
of grammar; each of these is so called as having in him a certain
potentiality, but there is a difference between their respective potentialities,
the one (a) being a potential knower, because his kind or matter is
such and such, the other (b), because he can in the absence of any
external counteracting cause realize his knowledge in actual knowing
at will. This implies a third meaning of 'a knower' (c), one who is
already realizing his knowledge-he is a knower in actuality and in
the most proper sense is knowing, e.g. this A. Both the former are
potential knowers, who realize their respective potentialities, the
one (a) by change of quality, i.e. repeated transitions from one state
to its opposite under instruction, the other (b) by the transition
from the inactive possession of sense or grammar to their active exercise.
The two kinds of transition are distinct. 

Also the expression 'to be acted upon' has more than one meaning;
it may mean either (a) the extinction of one of two contraries by
the other, or (b) the maintenance of what is potential by the agency
of what is actual and already like what is acted upon, with such likeness
as is compatible with one's being actual and the other potential.
For what possesses knowledge becomes an actual knower by a transition
which is either not an alteration of it at all (being in reality a
development into its true self or actuality) or at least an alteration
in a quite different sense from the usual meaning. 

Hence it is wrong to speak of a wise man as being 'altered' when he
uses his wisdom, just as it would be absurd to speak of a builder
as being altered when he is using his skill in building a house.

What in the case of knowing or understanding leads from potentiality
to actuality ought not to be called teaching but something else. That
which starting with the power to know learns or acquires knowledge
through the agency of one who actually knows and has the power of
teaching either (a) ought not to be said 'to be acted upon' at all
or (b) we must recognize two senses of alteration, viz. (i) the substitution
of one quality for another, the first being the contrary of the second,
or (ii) the development of an existent quality from potentiality in
the direction of fixity or nature. 

In the case of what is to possess sense, the first transition is due
to the action of the male parent and takes place before birth so that
at birth the living thing is, in respect of sensation, at the stage
which corresponds to the possession of knowledge. Actual sensation
corresponds to the stage of the exercise of knowledge. But between
the two cases compared there is a difference; the objects that excite
the sensory powers to activity, the seen, the heard, &c., are outside.
The ground of this difference is that what actual sensation apprehends
is individuals, while what knowledge apprehends is universals, and
these are in a sense within the soul. That is why a man can exercise
his knowledge when he wishes, but his sensation does not depend upon
himself a sensible object must be there. A similar statement must
be made about our knowledge of what is sensible-on the same ground,
viz. that the sensible objects are individual and external.

A later more appropriate occasion may be found thoroughly to clear
up all this. At present it must be enough to recognize the distinctions
already drawn; a thing may be said to be potential in either of two
senses, (a) in the sense in which we might say of a boy that he may
become a general or (b) in the sense in which we might say the same
of an adult, and there are two corresponding senses of the term 'a
potential sentient'. There are no separate names for the two stages
of potentiality; we have pointed out that they are different and how
they are different. We cannot help using the incorrect terms 'being
acted upon or altered' of the two transitions involved. As we have
said, has the power of sensation is potentially like what the perceived
object is actually; that is, while at the beginning of the process
of its being acted upon the two interacting factors are dissimilar,
at the end the one acted upon is assimilated to the other and is identical
in quality with it. 

Part 6

In dealing with each of the senses we shall have first to speak of
the objects which are perceptible by each. The term 'object of sense'
covers three kinds of objects, two kinds of which are, in our language,
directly perceptible, while the remaining one is only incidentally
perceptible. Of the first two kinds one (a) consists of what is perceptible
by a single sense, the other (b) of what is perceptible by any and
all of the senses. I call by the name of special object of this or
that sense that which cannot be perceived by any other sense than
that one and in respect of which no error is possible; in this sense
colour is the special object of sight, sound of hearing, flavour of
taste. Touch, indeed, discriminates more than one set of different
qualities. Each sense has one kind of object which it discerns, and
never errs in reporting that what is before it is colour or sound
(though it may err as to what it is that is coloured or where that
is, or what it is that is sounding or where that is.) Such objects
are what we propose to call the special objects of this or that sense.

'Common sensibles' are movement, rest, number, figure, magnitude;
these are not peculiar to any one sense, but are common to all. There
are at any rate certain kinds of movement which are perceptible both
by touch and by sight. 

We speak of an incidental object of sense where e.g. the white object
which we see is the son of Diares; here because 'being the son of
Diares' is incidental to the directly visible white patch we speak
of the son of Diares as being (incidentally) perceived or seen by
us. Because this is only incidentally an object of sense, it in no
way as such affects the senses. Of the two former kinds, both of which
are in their own nature perceptible by sense, the first kind-that
of special objects of the several senses-constitute the objects of
sense in the strictest sense of the term and it is to them that in
the nature of things the structure of each several sense is adapted.

Part 7

The object of sight is the visible, and what is visible is (a) colour
and (b) a certain kind of object which can be described in words but
which has no single name; what we mean by (b) will be abundantly clear
as we proceed. Whatever is visible is colour and colour is what lies
upon what is in its own nature visible; 'in its own nature' here means
not that visibility is involved in the definition of what thus underlies
colour, but that that substratum contains in itself the cause of visibility.
Every colour has in it the power to set in movement what is actually
transparent; that power constitutes its very nature. That is why it
is not visible except with the help of light; it is only in light
that the colour of a thing is seen. Hence our first task is to explain
what light is. 

Now there clearly is something which is transparent, and by 'transparent'
I mean what is visible, and yet not visible in itself, but rather
owing its visibility to the colour of something else; of this character
are air, water, and many solid bodies. Neither air nor water is transparent
because it is air or water; they are transparent because each of them
has contained in it a certain substance which is the same in both
and is also found in the eternal body which constitutes the uppermost
shell of the physical Cosmos. Of this substance light is the activity-the
activity of what is transparent so far forth as it has in it the determinate
power of becoming transparent; where this power is present, there
is also the potentiality of the contrary, viz. darkness. Light is
as it were the proper colour of what is transparent, and exists whenever
the potentially transparent is excited to actuality by the influence
of fire or something resembling 'the uppermost body'; for fire too
contains something which is one and the same with the substance in
question. 

We have now explained what the transparent is and what light is; light
is neither fire nor any kind whatsoever of body nor an efflux from
any kind of body (if it were, it would again itself be a kind of body)-it
is the presence of fire or something resembling fire in what is transparent.
It is certainly not a body, for two bodies cannot be present in the
same place. The opposite of light is darkness; darkness is the absence
from what is transparent of the corresponding positive state above
characterized; clearly therefore, light is just the presence of that.

Empedocles (and with him all others who used the same forms of expression)
was wrong in speaking of light as 'travelling' or being at a given
moment between the earth and its envelope, its movement being unobservable
by us; that view is contrary both to the clear evidence of argument
and to the observed facts; if the distance traversed were short, the
movement might have been unobservable, but where the distance is from
extreme East to extreme West, the draught upon our powers of belief
is too great. 

What is capable of taking on colour is what in itself is colourless,
as what can take on sound is what is soundless; what is colourless
includes (a) what is transparent and (b) what is invisible or scarcely
visible, i.e. what is 'dark'. The latter (b) is the same as what is
transparent, when it is potentially, not of course when it is actually
transparent; it is the same substance which is now darkness, now light.

Not everything that is visible depends upon light for its visibility.
This is only true of the 'proper' colour of things. Some objects of
sight which in light are invisible, in darkness stimulate the sense;
that is, things that appear fiery or shining. This class of objects
has no simple common name, but instances of it are fungi, flesh, heads,
scales, and eyes of fish. In none of these is what is seen their own
proper' colour. Why we see these at all is another question. At present
what is obvious is that what is seen in light is always colour. That
is why without the help of light colour remains invisible. Its being
colour at all means precisely its having in it the power to set in
movement what is already actually transparent, and, as we have seen,
the actuality of what is transparent is just light. 

The following experiment makes the necessity of a medium clear. If
what has colour is placed in immediate contact with the eye, it cannot
be seen. Colour sets in movement not the sense organ but what is transparent,
e.g. the air, and that, extending continuously from the object to
the organ, sets the latter in movement. Democritus misrepresents the
facts when he expresses the opinion that if the interspace were empty
one could distinctly see an ant on the vault of the sky; that is an
impossibility. Seeing is due to an affection or change of what has
the perceptive faculty, and it cannot be affected by the seen colour
itself; it remains that it must be affected by what comes between.
Hence it is indispensable that there be something in between-if there
were nothing, so far from seeing with greater distinctness, we should
see nothing at all. 

We have now explained the cause why colour cannot be seen otherwise
than in light. Fire on the other hand is seen both in darkness and
in light; this double possibility follows necessarily from our theory,
for it is just fire that makes what is potentially transparent actually
transparent. 

The same account holds also of sound and smell; if the object of either
of these senses is in immediate contact with the organ no sensation
is produced. In both cases the object sets in movement only what lies
between, and this in turn sets the organ in movement: if what sounds
or smells is brought into immediate contact with the organ, no sensation
will be produced. The same, in spite of all appearances, applies also
to touch and taste; why there is this apparent difference will be
clear later. What comes between in the case of sounds is air; the
corresponding medium in the case of smell has no name. But, corresponding
to what is transparent in the case of colour, there is a quality found
both in air and water, which serves as a medium for what has smell-I
say 'in water' because animals that live in water as well as those
that live on land seem to possess the sense of smell, and 'in air'
because man and all other land animals that breathe, perceive smells
only when they breathe air in. The explanation of this too will be
given later. 

Part 8

Now let us, to begin with, make certain distinctions about sound and
hearing. 

Sound may mean either of two things (a) actual, and (b) potential,
sound. There are certain things which, as we say, 'have no sound',
e.g. sponges or wool, others which have, e.g. bronze and in general
all things which are smooth and solid-the latter are said to have
a sound because they can make a sound, i.e. can generate actual sound
between themselves and the organ of hearing. 

Actual sound requires for its occurrence (i, ii) two such bodies and
(iii) a space between them; for it is generated by an impact. Hence
it is impossible for one body only to generate a sound-there must
be a body impinging and a body impinged upon; what sounds does so
by striking against something else, and this is impossible without
a movement from place to place. 

As we have said, not all bodies can by impact on one another produce
sound; impact on wool makes no sound, while the impact on bronze or
any body which is smooth and hollow does. Bronze gives out a sound
when struck because it is smooth; bodies which are hollow owing to
reflection repeat the original impact over and over again, the body
originally set in movement being unable to escape from the concavity.

Further, we must remark that sound is heard both in air and in water,
though less distinctly in the latter. Yet neither air nor water is
the principal cause of sound. What is required for the production
of sound is an impact of two solids against one another and against
the air. The latter condition is satisfied when the air impinged upon
does not retreat before the blow, i.e. is not dissipated by it.

That is why it must be struck with a sudden sharp blow, if it is to
sound-the movement of the whip must outrun the dispersion of the air,
just as one might get in a stroke at a heap or whirl of sand as it
was traveling rapidly past. 

An echo occurs, when, a mass of air having been unified, bounded,
and prevented from dissipation by the containing walls of a vessel,
the air originally struck by the impinging body and set in movement
by it rebounds from this mass of air like a ball from a wall. It is
probable that in all generation of sound echo takes place, though
it is frequently only indistinctly heard. What happens here must be
analogous to what happens in the case of light; light is always reflected-otherwise
it would not be diffused and outside what was directly illuminated
by the sun there would be blank darkness; but this reflected light
is not always strong enough, as it is when it is reflected from water,
bronze, and other smooth bodies, to cast a shadow, which is the distinguishing
mark by which we recognize light. 

It is rightly said that an empty space plays the chief part in the
production of hearing, for what people mean by 'the vacuum' is the
air, which is what causes hearing, when that air is set in movement
as one continuous mass; but owing to its friability it emits no sound,
being dissipated by impinging upon any surface which is not smooth.
When the surface on which it impinges is quite smooth, what is produced
by the original impact is a united mass, a result due to the smoothness
of the surface with which the air is in contact at the other end.

What has the power of producing sound is what has the power of setting
in movement a single mass of air which is continuous from the impinging
body up to the organ of hearing. The organ of hearing is physically
united with air, and because it is in air, the air inside is moved
concurrently with the air outside. Hence animals do not hear with
all parts of their bodies, nor do all parts admit of the entrance
of air; for even the part which can be moved and can sound has not
air everywhere in it. Air in itself is, owing to its friability, quite
soundless; only when its dissipation is prevented is its movement
sound. The air in the ear is built into a chamber just to prevent
this dissipating movement, in order that the animal may accurately
apprehend all varieties of the movements of the air outside. That
is why we hear also in water, viz. because the water cannot get into
the air chamber or even, owing to the spirals, into the outer ear.
If this does happen, hearing ceases, as it also does if the tympanic
membrane is damaged, just as sight ceases if the membrane covering
the pupil is damaged. It is also a test of deafness whether the ear
does or does not reverberate like a horn; the air inside the ear has
always a movement of its own, but the sound we hear is always the
sounding of something else, not of the organ itself. That is why we
say that we hear with what is empty and echoes, viz. because what
we hear with is a chamber which contains a bounded mass of air.

Which is it that 'sounds', the striking body or the struck? Is not
the answer 'it is both, but each in a different way'? Sound is a movement
of what can rebound from a smooth surface when struck against it.
As we have explained' not everything sounds when it strikes or is
struck, e.g. if one needle is struck against another, neither emits
any sound. In order, therefore, that sound may be generated, what
is struck must be smooth, to enable the air to rebound and be shaken
off from it in one piece. 

The distinctions between different sounding bodies show themselves
only in actual sound; as without the help of light colours remain
invisible, so without the help of actual sound the distinctions between
acute and grave sounds remain inaudible. Acute and grave are here
metaphors, transferred from their proper sphere, viz. that of touch,
where they mean respectively (a) what moves the sense much in a short
time, (b) what moves the sense little in a long time. Not that what
is sharp really moves fast, and what is grave, slowly, but that the
difference in the qualities of the one and the other movement is due
to their respective speeds. There seems to be a sort of parallelism
between what is acute or grave to hearing and what is sharp or blunt
to touch; what is sharp as it were stabs, while what is blunt pushes,
the one producing its effect in a short, the other in a long time,
so that the one is quick, the other slow. 

Let the foregoing suffice as an analysis of sound. Voice is a kind
of sound characteristic of what has soul in it; nothing that is without
soul utters voice, it being only by a metaphor that we speak of the
voice of the flute or the lyre or generally of what (being without
soul) possesses the power of producing a succession of notes which
differ in length and pitch and timbre. The metaphor is based on the
fact that all these differences are found also in voice. Many animals
are voiceless, e.g. all non-sanuineous animals and among sanguineous
animals fish. This is just what we should expect, since voice is a
certain movement of air. The fish, like those in the Achelous, which
are said to have voice, really make the sounds with their gills or
some similar organ. Voice is the sound made by an animal, and that
with a special organ. As we saw, everything that makes a sound does
so by the impact of something (a) against something else, (b) across
a space, (c) filled with air; hence it is only to be expected that
no animals utter voice except those which take in air. Once air is
inbreathed, Nature uses it for two different purposes, as the tongue
is used both for tasting and for articulating; in that case of the
two functions tasting is necessary for the animal's existence (hence
it is found more widely distributed), while articulate speech is a
luxury subserving its possessor's well-being; similarly in the former
case Nature employs the breath both as an indispensable means to the
regulation of the inner temperature of the living body and also as
the matter of articulate voice, in the interests of its possessor's
well-being. Why its former use is indispensable must be discussed
elsewhere. 

The organ of respiration is the windpipe, and the organ to which this
is related as means to end is the lungs. The latter is the part of
the body by which the temperature of land animals is raised above
that of all others. But what primarily requires the air drawn in by
respiration is not only this but the region surrounding the heart.
That is why when animals breathe the air must penetrate inwards.

Voice then is the impact of the inbreathed air against the 'windpipe',
and the agent that produces the impact is the soul resident in these
parts of the body. Not every sound, as we said, made by an animal
is voice (even with the tongue we may merely make a sound which is
not voice, or without the tongue as in coughing); what produces the
impact must have soul in it and must be accompanied by an act of imagination,
for voice is a sound with a meaning, and is not merely the result
of any impact of the breath as in coughing; in voice the breath in
the windpipe is used as an instrument to knock with against the walls
of the windpipe. This is confirmed by our inability to speak when
we are breathing either out or in-we can only do so by holding our
breath; we make the movements with the breath so checked. It is clear
also why fish are voiceless; they have no windpipe. And they have
no windpipe because they do not breathe or take in air. Why they do
not is a question belonging to another inquiry. 

Part 9

Smell and its object are much less easy to determine than what we
have hitherto discussed; the distinguishing characteristic of the
object of smell is less obvious than those of sound or colour. The
ground of this is that our power of smell is less discriminating and
in general inferior to that of many species of animals; men have a
poor sense of smell and our apprehension of its proper objects is
inseparably bound up with and so confused by pleasure and pain, which
shows that in us the organ is inaccurate. It is probable that there
is a parallel failure in the perception of colour by animals that
have hard eyes: probably they discriminate differences of colour only
by the presence or absence of what excites fear, and that it is thus
that human beings distinguish smells. It seems that there is an analogy
between smell and taste, and that the species of tastes run parallel
to those of smells-the only difference being that our sense of taste
is more discriminating than our sense of smell, because the former
is a modification of touch, which reaches in man the maximum of discriminative
accuracy. While in respect of all the other senses we fall below many
species of animals, in respect of touch we far excel all other species
in exactness of discrimination. That is why man is the most intelligent
of all animals. This is confirmed by the fact that it is to differences
in the organ of touch and to nothing else that the differences between
man and man in respect of natural endowment are due; men whose flesh
is hard are ill-endowed by nature, men whose flesh is soft, wellendowed.

As flavours may be divided into (a) sweet, (b) bitter, so with smells.
In some things the flavour and the smell have the same quality, i.e.
both are sweet or both bitter, in others they diverge. Similarly a
smell, like a flavour, may be pungent, astringent, acid, or succulent.
But, as we said, because smells are much less easy to discriminate
than flavours, the names of these varieties are applied to smells
only metaphorically; for example 'sweet' is extended from the taste
to the smell of saffron or honey, 'pungent' to that of thyme, and
so on. 

In the same sense in which hearing has for its object both the audible
and the inaudible, sight both the visible and the invisible, smell
has for its object both the odorous and the inodorous. 'Inodorous'
may be either (a) what has no smell at all, or (b) what has a small
or feeble smell. The same ambiguity lurks in the word 'tasteless'.

Smelling, like the operation of the senses previously examined, takes
place through a medium, i.e. through air or water-I add water, because
water-animals too (both sanguineous and non-sanguineous) seem to smell
just as much as land-animals; at any rate some of them make directly
for their food from a distance if it has any scent. That is why the
following facts constitute a problem for us. All animals smell in
the same way, but man smells only when he inhales; if he exhales or
holds his breath, he ceases to smell, no difference being made whether
the odorous object is distant or near, or even placed inside the nose
and actually on the wall of the nostril; it is a disability common
to all the senses not to perceive what is in immediate contact with
the organ of sense, but our failure to apprehend what is odorous without
the help of inhalation is peculiar (the fact is obvious on making
the experiment). Now since bloodless animals do not breathe, they
must, it might be argued, have some novel sense not reckoned among
the usual five. Our reply must be that this is impossible, since it
is scent that is perceived; a sense that apprehends what is odorous
and what has a good or bad odour cannot be anything but smell. Further,
they are observed to be deleteriously effected by the same strong
odours as man is, e.g. bitumen, sulphur, and the like. These animals
must be able to smell without being able to breathe. The probable
explanation is that in man the organ of smell has a certain superiority
over that in all other animals just as his eyes have over those of
hard-eyed animals. Man's eyes have in the eyelids a kind of shelter
or envelope, which must be shifted or drawn back in order that we
may see, while hardeyed animals have nothing of the kind, but at once
see whatever presents itself in the transparent medium. Similarly
in certain species of animals the organ of smell is like the eye of
hard-eyed animals, uncurtained, while in others which take in air
it probably has a curtain over it, which is drawn back in inhalation,
owing to the dilating of the veins or pores. That explains also why
such animals cannot smell under water; to smell they must first inhale,
and that they cannot do under water. 

Smells come from what is dry as flavours from what is moist. Consequently
the organ of smell is potentially dry. 

Part 10

What can be tasted is always something that can be touched, and just
for that reason it cannot be perceived through an interposed foreign
body, for touch means the absence of any intervening body. Further,
the flavoured and tasteable body is suspended in a liquid matter,
and this is tangible. Hence, if we lived in water, we should perceive
a sweet object introduced into the water, but the water would not
be the medium through which we perceived; our perception would be
due to the solution of the sweet substance in what we imbibed, just
as if it were mixed with some drink. There is no parallel here to
the perception of colour, which is due neither to any blending of
anything with anything, nor to any efflux of anything from anything.
In the case of taste, there is nothing corresponding to the medium
in the case of the senses previously discussed; but as the object
of sight is colour, so the object of taste is flavour. But nothing
excites a perception of flavour without the help of liquid; what acts
upon the sense of taste must be either actually or potentially liquid
like what is saline; it must be both (a) itself easily dissolved,
and (b) capable of dissolving along with itself the tongue. Taste
apprehends both (a) what has taste and (b) what has no taste, if we
mean by (b) what has only a slight or feeble flavour or what tends
to destroy the sense of taste. In this it is exactly parallel to sight,
which apprehends both what is visible and what is invisible (for darkness
is invisible and yet is discriminated by sight; so is, in a different
way, what is over brilliant), and to hearing, which apprehends both
sound and silence, of which the one is audible and the other inaudible,
and also over-loud sound. This corresponds in the case of hearing
to over-bright light in the case of sight. As a faint sound is 'inaudible',
so in a sense is a loud or violent sound. The word 'invisible' and
similar privative terms cover not only (a) what is simply without
some power, but also (b) what is adapted by nature to have it but
has not it or has it only in a very low degree, as when we say that
a species of swallow is 'footless' or that a variety of fruit is 'stoneless'.
So too taste has as its object both what can be tasted and the tasteless-the
latter in the sense of what has little flavour or a bad flavour or
one destructive of taste. The difference between what is tasteless
and what is not seems to rest ultimately on that between what is drinkable
and what is undrinkable both are tasteable, but the latter is bad
and tends to destroy taste, while the former is the normal stimulus
of taste. What is drinkable is the common object of both touch and
taste. 

Since what can be tasted is liquid, the organ for its perception cannot
be either (a) actually liquid or (b) incapable of becoming liquid.
Tasting means a being affected by what can be tasted as such; hence
the organ of taste must be liquefied, and so to start with must be
non-liquid but capable of liquefaction without loss of its distinctive
nature. This is confirmed by the fact that the tongue cannot taste
either when it is too dry or when it is too moist; in the latter case
what occurs is due to a contact with the pre-existent moisture in
the tongue itself, when after a foretaste of some strong flavour we
try to taste another flavour; it is in this way that sick persons
find everything they taste bitter, viz. because, when they taste,
their tongues are overflowing with bitter moisture. 

The species of flavour are, as in the case of colour, (a) simple,
i.e. the two contraries, the sweet and the bitter, (b) secondary,
viz. (i) on the side of the sweet, the succulent, (ii) on the side
of the bitter, the saline, (iii) between these come the pungent, the
harsh, the astringent, and the acid; these pretty well exhaust the
varieties of flavour. It follows that what has the power of tasting
is what is potentially of that kind, and that what is tasteable is
what has the power of making it actually what it itself already is.

Part 11

Whatever can be said of what is tangible, can be said of touch, and
vice versa; if touch is not a single sense but a group of senses,
there must be several kinds of what is tangible. It is a problem whether
touch is a single sense or a group of senses. It is also a problem,
what is the organ of touch; is it or is it not the flesh (including
what in certain animals is homologous with flesh)? On the second view,
flesh is 'the medium' of touch, the real organ being situated farther
inward. The problem arises because the field of each sense is according
to the accepted view determined as the range between a single pair
of contraries, white and black for sight, acute and grave for hearing,
bitter and sweet for taste; but in the field of what is tangible we
find several such pairs, hot cold, dry moist, hard soft, &c. This
problem finds a partial solution, when it is recalled that in the
case of the other senses more than one pair of contraries are to be
met with, e.g. in sound not only acute and grave but loud and soft,
smooth and rough, &c.; there are similar contrasts in the field of
colour. Nevertheless we are unable clearly to detect in the case of
touch what the single subject is which underlies the contrasted qualities
and corresponds to sound in the case of hearing. 

To the question whether the organ of touch lies inward or not (i.e.
whether we need look any farther than the flesh), no indication in
favour of the second answer can be drawn from the fact that if the
object comes into contact with the flesh it is at once perceived.
For even under present conditions if the experiment is made of making
a web and stretching it tight over the flesh, as soon as this web
is touched the sensation is reported in the same manner as before,
yet it is clear that the or is gan is not in this membrane. If the
membrane could be grown on to the flesh, the report would travel still
quicker. The flesh plays in touch very much the same part as would
be played in the other senses by an air-envelope growing round our
body; had we such an envelope attached to us we should have supposed
that it was by a single organ that we perceived sounds, colours, and
smells, and we should have taken sight, hearing, and smell to be a
single sense. But as it is, because that through which the different
movements are transmitted is not naturally attached to our bodies,
the difference of the various sense-organs is too plain to miss. But
in the case of touch the obscurity remains. 

There must be such a naturally attached 'medium' as flesh, for no
living body could be constructed of air or water; it must be something
solid. Consequently it must be composed of earth along with these,
which is just what flesh and its analogue in animals which have no
true flesh tend to be. Hence of necessity the medium through which
are transmitted the manifoldly contrasted tactual qualities must be
a body naturally attached to the organism. That they are manifold
is clear when we consider touching with the tongue; we apprehend at
the tongue all tangible qualities as well as flavour. Suppose all
the rest of our flesh was, like the tongue, sensitive to flavour,
we should have identified the sense of taste and the sense of touch;
what saves us from this identification is the fact that touch and
taste are not always found together in the same part of the body.
The following problem might be raised. Let us assume that every body
has depth, i.e. has three dimensions, and that if two bodies have
a third body between them they cannot be in contact with one another;
let us remember that what is liquid is a body and must be or contain
water, and that if two bodies touch one another under water, their
touching surfaces cannot be dry, but must have water between, viz.
the water which wets their bounding surfaces; from all this it follows
that in water two bodies cannot be in contact with one another. The
same holds of two bodies in air-air being to bodies in air precisely
what water is to bodies in water-but the facts are not so evident
to our observation, because we live in air, just as animals that live
in water would not notice that the things which touch one another
in water have wet surfaces. The problem, then, is: does the perception
of all objects of sense take place in the same way, or does it not,
e.g. taste and touch requiring contact (as they are commonly thought
to do), while all other senses perceive over a distance? The distinction
is unsound; we perceive what is hard or soft, as well as the objects
of hearing, sight, and smell, through a 'medium', only that the latter
are perceived over a greater distance than the former; that is why
the facts escape our notice. For we do perceive everything through
a medium; but in these cases the fact escapes us. Yet, to repeat what
we said before, if the medium for touch were a membrane separating
us from the object without our observing its existence, we should
be relatively to it in the same condition as we are now to air or
water in which we are immersed; in their case we fancy we can touch
objects, nothing coming in between us and them. But there remains
this difference between what can be touched and what can be seen or
can sound; in the latter two cases we perceive because the medium
produces a certain effect upon us, whereas in the perception of objects
of touch we are affected not by but along with the medium; it is as
if a man were struck through his shield, where the shock is not first
given to the shield and passed on to the man, but the concussion of
both is simultaneous. 

In general, flesh and the tongue are related to the real organs of
touch and taste, as air and water are to those of sight, hearing,
and smell. Hence in neither the one case nor the other can there be
any perception of an object if it is placed immediately upon the organ,
e.g. if a white object is placed on the surface of the eye. This again
shows that what has the power of perceiving the tangible is seated
inside. Only so would there be a complete analogy with all the other
senses. In their case if you place the object on the organ it is not
perceived, here if you place it on the flesh it is perceived; therefore
flesh is not the organ but the medium of touch. 

What can be touched are distinctive qualities of body as body; by
such differences I mean those which characterize the elements, viz,
hot cold, dry moist, of which we have spoken earlier in our treatise
on the elements. The organ for the perception of these is that of
touch-that part of the body in which primarily the sense of touch
resides. This is that part which is potentially such as its object
is actually: for all sense-perception is a process of being so affected;
so that that which makes something such as it itself actually is makes
the other such because the other is already potentially such. That
is why when an object of touch is equally hot and cold or hard and
soft we cannot perceive; what we perceive must have a degree of the
sensible quality lying beyond the neutral point. This implies that
the sense itself is a 'mean' between any two opposite qualities which
determine the field of that sense. It is to this that it owes its
power of discerning the objects in that field. What is 'in the middle'
is fitted to discern; relatively to either extreme it can put itself
in the place of the other. As what is to perceive both white and black
must, to begin with, be actually neither but potentially either (and
so with all the other sense-organs), so the organ of touch must be
neither hot nor cold. 

Further, as in a sense sight had for its object both what was visible
and what was invisible (and there was a parallel truth about all the
other senses discussed), so touch has for its object both what is
tangible and what is intangible. Here by 'intangible' is meant (a)
what like air possesses some quality of tangible things in a very
slight degree and (b) what possesses it in an excessive degree, as
destructive things do. 

We have now given an outline account of each of the several senses.

Part 12

The following results applying to any and every sense may now be formulated.

(A) By a 'sense' is meant what has the power of receiving into itself
the sensible forms of things without the matter. This must be conceived
of as taking place in the way in which a piece of wax takes on the
impress of a signet-ring without the iron or gold; we say that what
produces the impression is a signet of bronze or gold, but its particular
metallic constitution makes no difference: in a similar way the sense
is affected by what is coloured or flavoured or sounding, but it is
indifferent what in each case the substance is; what alone matters
is what quality it has, i.e. in what ratio its constituents are combined.

(B) By 'an organ of sense' is meant that in which ultimately such
a power is seated. 

The sense and its organ are the same in fact, but their essence is
not the same. What perceives is, of course, a spatial magnitude, but
we must not admit that either the having the power to perceive or
the sense itself is a magnitude; what they are is a certain ratio
or power in a magnitude. This enables us to explain why objects of
sense which possess one of two opposite sensible qualities in a degree
largely in excess of the other opposite destroy the organs of sense;
if the movement set up by an object is too strong for the organ, the
equipoise of contrary qualities in the organ, which just is its sensory
power, is disturbed; it is precisely as concord and tone are destroyed
by too violently twanging the strings of a lyre. This explains also
why plants cannot perceive. in spite of their having a portion of
soul in them and obviously being affected by tangible objects themselves;
for undoubtedly their temperature can be lowered or raised. The explanation
is that they have no mean of contrary qualities, and so no principle
in them capable of taking on the forms of sensible objects without
their matter; in the case of plants the affection is an affection
by form-and-matter together. The problem might be raised: Can what
cannot smell be said to be affected by smells or what cannot see by
colours, and so on? It might be said that a smell is just what can
be smelt, and if it produces any effect it can only be so as to make
something smell it, and it might be argued that what cannot smell
cannot be affected by smells and further that what can smell can be
affected by it only in so far as it has in it the power to smell (similarly
with the proper objects of all the other senses). Indeed that this
is so is made quite evident as follows. Light or darkness, sounds
and smells leave bodies quite unaffected; what does affect bodies
is not these but the bodies which are their vehicles, e.g. what splits
the trunk of a tree is not the sound of the thunder but the air which
accompanies thunder. Yes, but, it may be objected, bodies are affected
by what is tangible and by flavours. If not, by what are things that
are without soul affected, i.e. altered in quality? Must we not, then,
admit that the objects of the other senses also may affect them? Is
not the true account this, that all bodies are capable of being affected
by smells and sounds, but that some on being acted upon, having no
boundaries of their own, disintegrate, as in the instance of air,
which does become odorous, showing that some effect is produced on
it by what is odorous? But smelling is more than such an affection
by what is odorous-what more? Is not the answer that, while the air
owing to the momentary duration of the action upon it of what is odorous
does itself become perceptible to the sense of smell, smelling is
an observing of the result produced? 

----------------------------------------------------------------------

BOOK III

Part 1 

That there is no sixth sense in addition to the five enumerated-sight,
hearing, smell, taste, touch-may be established by the following considerations:

If we have actually sensation of everything of which touch can give
us sensation (for all the qualities of the tangible qua tangible are
perceived by us through touch); and if absence of a sense necessarily
involves absence of a sense-organ; and if (1) all objects that we
perceive by immediate contact with them are perceptible by touch,
which sense we actually possess, and (2) all objects that we perceive
through media, i.e. without immediate contact, are perceptible by
or through the simple elements, e.g. air and water (and this is so
arranged that (a) if more than one kind of sensible object is perceivable
through a single medium, the possessor of a sense-organ homogeneous
with that medium has the power of perceiving both kinds of objects;
for example, if the sense-organ is made of air, and air is a medium
both for sound and for colour; and that (b) if more than one medium
can transmit the same kind of sensible objects, as e.g. water as well
as air can transmit colour, both being transparent, then the possessor
of either alone will be able to perceive the kind of objects transmissible
through both); and if of the simple elements two only, air and water,
go to form sense-organs (for the pupil is made of water, the organ
of hearing is made of air, and the organ of smell of one or other
of these two, while fire is found either in none or in all-warmth
being an essential condition of all sensibility-and earth either in
none or, if anywhere, specially mingled with the components of the
organ of touch; wherefore it would remain that there can be no sense-organ
formed of anything except water and air); and if these sense-organs
are actually found in certain animals;-then all the possible senses
are possessed by those animals that are not imperfect or mutilated
(for even the mole is observed to have eyes beneath its skin); so
that, if there is no fifth element and no property other than those
which belong to the four elements of our world, no sense can be wanting
to such animals. 

Further, there cannot be a special sense-organ for the common sensibles
either, i.e. the objects which we perceive incidentally through this
or that special sense, e.g. movement, rest, figure, magnitude, number,
unity; for all these we perceive by movement, e.g. magnitude by movement,
and therefore also figure (for figure is a species of magnitude),
what is at rest by the absence of movement: number is perceived by
the negation of continuity, and by the special sensibles; for each
sense perceives one class of sensible objects. So that it is clearly
impossible that there should be a special sense for any one of the
common sensibles, e.g. movement; for, if that were so, our perception
of it would be exactly parallel to our present perception of what
is sweet by vision. That is so because we have a sense for each of
the two qualities, in virtue of which when they happen to meet in
one sensible object we are aware of both contemporaneously. If it
were not like this our perception of the common qualities would always
be incidental, i.e. as is the perception of Cleon's son, where we
perceive him not as Cleon's son but as white, and the white thing
which we really perceive happens to be Cleon's son. 

But in the case of the common sensibles there is already in us a general
sensibility which enables us to perceive them directly; there is therefore
no special sense required for their perception: if there were, our
perception of them would have been exactly like what has been above
described. 

The senses perceive each other's special objects incidentally; not
because the percipient sense is this or that special sense, but because
all form a unity: this incidental perception takes place whenever
sense is directed at one and the same moment to two disparate qualities
in one and the same object, e.g. to the bitterness and the yellowness
of bile, the assertion of the identity of both cannot be the act of
either of the senses; hence the illusion of sense, e.g. the belief
that if a thing is yellow it is bile. 

It might be asked why we have more senses than one. Is it to prevent
a failure to apprehend the common sensibles, e.g. movement, magnitude,
and number, which go along with the special sensibles? Had we no sense
but sight, and that sense no object but white, they would have tended
to escape our notice and everything would have merged for us into
an indistinguishable identity because of the concomitance of colour
and magnitude. As it is, the fact that the common sensibles are given
in the objects of more than one sense reveals their distinction from
each and all of the special sensibles. 

Part 2

Since it is through sense that we are aware that we are seeing or
hearing, it must be either by sight that we are aware of seeing, or
by some sense other than sight. But the sense that gives us this new
sensation must perceive both sight and its object, viz. colour: so
that either (1) there will be two senses both percipient of the same
sensible object, or (2) the sense must be percipient of itself. Further,
even if the sense which perceives sight were different from sight,
we must either fall into an infinite regress, or we must somewhere
assume a sense which is aware of itself. If so, we ought to do this
in the first case. 

This presents a difficulty: if to perceive by sight is just to see,
and what is seen is colour (or the coloured), then if we are to see
that which sees, that which sees originally must be coloured. It is
clear therefore that 'to perceive by sight' has more than one meaning;
for even when we are not seeing, it is by sight that we discriminate
darkness from light, though not in the same way as we distinguish
one colour from another. Further, in a sense even that which sees
is coloured; for in each case the sense-organ is capable of receiving
the sensible object without its matter. That is why even when the
sensible objects are gone the sensings and imaginings continue to
exist in the sense-organs. 

The activity of the sensible object and that of the percipient sense
is one and the same activity, and yet the distinction between their
being remains. Take as illustration actual sound and actual hearing:
a man may have hearing and yet not be hearing, and that which has
a sound is not always sounding. But when that which can hear is actively
hearing and which can sound is sounding, then the actual hearing and
the actual sound are merged in one (these one might call respectively
hearkening and sounding). 

If it is true that the movement, both the acting and the being acted
upon, is to be found in that which is acted upon, both the sound and
the hearing so far as it is actual must be found in that which has
the faculty of hearing; for it is in the passive factor that the actuality
of the active or motive factor is realized; that is why that which
causes movement may be at rest. Now the actuality of that which can
sound is just sound or sounding, and the actuality of that which can
hear is hearing or hearkening; 'sound' and 'hearing' are both ambiguous.
The same account applies to the other senses and their objects. For
as the-acting-and-being-acted-upon is to be found in the passive,
not in the active factor, so also the actuality of the sensible object
and that of the sensitive subject are both realized in the latter.
But while in some cases each aspect of the total actuality has a distinct
name, e.g. sounding and hearkening, in some one or other is nameless,
e.g. the actuality of sight is called seeing, but the actuality of
colour has no name: the actuality of the faculty of taste is called
tasting, but the actuality of flavour has no name. Since the actualities
of the sensible object and of the sensitive faculty are one actuality
in spite of the difference between their modes of being, actual hearing
and actual sounding appear and disappear from existence at one and
the same moment, and so actual savour and actual tasting, &c., while
as potentialities one of them may exist without the other. The earlier
students of nature were mistaken in their view that without sight
there was no white or black, without taste no savour. This statement
of theirs is partly true, partly false: 'sense' and 'the sensible
object' are ambiguous terms, i.e. may denote either potentialities
or actualities: the statement is true of the latter, false of the
former. This ambiguity they wholly failed to notice. 

If voice always implies a concord, and if the voice and the hearing
of it are in one sense one and the same, and if concord always implies
a ratio, hearing as well as what is heard must be a ratio. That is
why the excess of either the sharp or the flat destroys the hearing.
(So also in the case of savours excess destroys the sense of taste,
and in the case of colours excessive brightness or darkness destroys
the sight, and in the case of smell excess of strength whether in
the direction of sweetness or bitterness is destructive.) This shows
that the sense is a ratio. 

That is also why the objects of sense are (1) pleasant when the sensible
extremes such as acid or sweet or salt being pure and unmixed are
brought into the proper ratio; then they are pleasant: and in general
what is blended is more pleasant than the sharp or the flat alone;
or, to touch, that which is capable of being either warmed or chilled:
the sense and the ratio are identical: while (2) in excess the sensible
extremes are painful or destructive. 

Each sense then is relative to its particular group of sensible qualities:
it is found in a sense-organ as such and discriminates the differences
which exist within that group; e.g. sight discriminates white and
black, taste sweet and bitter, and so in all cases. Since we also
discriminate white from sweet, and indeed each sensible quality from
every other, with what do we perceive that they are different? It
must be by sense; for what is before us is sensible objects. (Hence
it is also obvious that the flesh cannot be the ultimate sense-organ:
if it were, the discriminating power could not do its work without
immediate contact with the object.) 

Therefore (1) discrimination between white and sweet cannot be effected
by two agencies which remain separate; both the qualities discriminated
must be present to something that is one and single. On any other
supposition even if I perceived sweet and you perceived white, the
difference between them would be apparent. What says that two things
are different must be one; for sweet is different from white. Therefore
what asserts this difference must be self-identical, and as what asserts,
so also what thinks or perceives. That it is not possible by means
of two agencies which remain separate to discriminate two objects
which are separate, is therefore obvious; and that (it is not possible
to do this in separate movements of time may be seen' if we look at
it as follows. For as what asserts the difference between the good
and the bad is one and the same, so also the time at which it asserts
the one to be different and the other to be different is not accidental
to the assertion (as it is for instance when I now assert a difference
but do not assert that there is now a difference); it asserts thus-both
now and that the objects are different now; the objects therefore
must be present at one and the same moment. Both the discriminating
power and the time of its exercise must be one and undivided.

But, it may be objected, it is impossible that what is self-identical
should be moved at me and the same time with contrary movements in
so far as it is undivided, and in an undivided moment of time. For
if what is sweet be the quality perceived, it moves the sense or thought
in this determinate way, while what is bitter moves it in a contrary
way, and what is white in a different way. Is it the case then that
what discriminates, though both numerically one and indivisible, is
at the same time divided in its being? In one sense, it is what is
divided that perceives two separate objects at once, but in another
sense it does so qua undivided; for it is divisible in its being but
spatially and numerically undivided. is not this impossible? For while
it is true that what is self-identical and undivided may be both contraries
at once potentially, it cannot be self-identical in its being-it must
lose its unity by being put into activity. It is not possible to be
at once white and black, and therefore it must also be impossible
for a thing to be affected at one and the same moment by the forms
of both, assuming it to be the case that sensation and thinking are
properly so described. 

The answer is that just as what is called a 'point' is, as being at
once one and two, properly said to be divisible, so here, that which
discriminates is qua undivided one, and active in a single moment
of time, while so far forth as it is divisible it twice over uses
the same dot at one and the same time. So far forth then as it takes
the limit as two' it discriminates two separate objects with what
in a sense is divided: while so far as it takes it as one, it does
so with what is one and occupies in its activity a single moment of
time. 

About the principle in virtue of which we say that animals are percipient,
let this discussion suffice. 

Part 3

There are two distinctive peculiarities by reference to which we characterize
the soul (1) local movement and (2) thinking, discriminating, and
perceiving. Thinking both speculative and practical is regarded as
akin to a form of perceiving; for in the one as well as the other
the soul discriminates and is cognizant of something which is. Indeed
the ancients go so far as to identify thinking and perceiving; e.g.
Empedocles says 'For 'tis in respect of what is present that man's
wit is increased', and again 'Whence it befalls them from time to
time to think diverse thoughts', and Homer's phrase 'For suchlike
is man's mind' means the same. They all look upon thinking as a bodily
process like perceiving, and hold that like is known as well as perceived
by like, as I explained at the beginning of our discussion. Yet they
ought at the same time to have accounted for error also; for it is
more intimately connected with animal existence and the soul continues
longer in the state of error than in that of truth. They cannot escape
the dilemma: either (1) whatever seems is true (and there are some
who accept this) or (2) error is contact with the unlike; for that
is the opposite of the knowing of like by like. 

But it is a received principle that error as well as knowledge in
respect to contraries is one and the same. 

That perceiving and practical thinking are not identical is therefore
obvious; for the former is universal in the animal world, the latter
is found in only a small division of it. Further, speculative thinking
is also distinct from perceiving-I mean that in which we find rightness
and wrongness-rightness in prudence, knowledge, true opinion, wrongness
in their opposites; for perception of the special objects of sense
is always free from error, and is found in all animals, while it is
possible to think falsely as well as truly, and thought is found only
where there is discourse of reason as well as sensibility. For imagination
is different from either perceiving or discursive thinking, though
it is not found without sensation, or judgement without it. That this
activity is not the same kind of thinking as judgement is obvious.
For imagining lies within our own power whenever we wish (e.g. we
can call up a picture, as in the practice of mnemonics by the use
of mental images), but in forming opinions we are not free: we cannot
escape the alternative of falsehood or truth. Further, when we think
something to be fearful or threatening, emotion is immediately produced,
and so too with what is encouraging; but when we merely imagine we
remain as unaffected as persons who are looking at a painting of some
dreadful or encouraging scene. Again within the field of judgement
itself we find varieties, knowledge, opinion, prudence, and their
opposites; of the differences between these I must speak elsewhere.

Thinking is different from perceiving and is held to be in part imagination,
in part judgement: we must therefore first mark off the sphere of
imagination and then speak of judgement. If then imagination is that
in virtue of which an image arises for us, excluding metaphorical
uses of the term, is it a single faculty or disposition relative to
images, in virtue of which we discriminate and are either in error
or not? The faculties in virtue of which we do this are sense, opinion,
science, intelligence. 

That imagination is not sense is clear from the following considerations:
Sense is either a faculty or an activity, e.g. sight or seeing: imagination
takes place in the absence of both, as e.g. in dreams. (Again, sense
is always present, imagination not. If actual imagination and actual
sensation were the same, imagination would be found in all the brutes:
this is held not to be the case; e.g. it is not found in ants or bees
or grubs. (Again, sensations are always true, imaginations are for
the most part false. (Once more, even in ordinary speech, we do not,
when sense functions precisely with regard to its object, say that
we imagine it to be a man, but rather when there is some failure of
accuracy in its exercise. And as we were saying before, visions appear
to us even when our eyes are shut. Neither is imagination any of the
things that are never in error: e.g. knowledge or intelligence; for
imagination may be false. 

It remains therefore to see if it is opinion, for opinion may be either
true or false. 

But opinion involves belief (for without belief in what we opine we
cannot have an opinion), and in the brutes though we often find imagination
we never find belief. Further, every opinion is accompanied by belief,
belief by conviction, and conviction by discourse of reason: while
there are some of the brutes in which we find imagination, without
discourse of reason. It is clear then that imagination cannot, again,
be (1) opinion plus sensation, or (2) opinion mediated by sensation,
or (3) a blend of opinion and sensation; this is impossible both for
these reasons and because the content of the supposed opinion cannot
be different from that of the sensation (I mean that imagination must
be the blending of the perception of white with the opinion that it
is white: it could scarcely be a blend of the opinion that it is good
with the perception that it is white): to imagine is therefore (on
this view) identical with the thinking of exactly the same as what
one in the strictest sense perceives. But what we imagine is sometimes
false though our contemporaneous judgement about it is true; e.g.
we imagine the sun to be a foot in diameter though we are convinced
that it is larger than the inhabited part of the earth, and the following
dilemma presents itself. Either (a while the fact has not changed
and the (observer has neither forgotten nor lost belief in the true
opinion which he had, that opinion has disappeared, or (b) if he retains
it then his opinion is at once true and false. A true opinion, however,
becomes false only when the fact alters without being noticed.

Imagination is therefore neither any one of the states enumerated,
nor compounded out of them. 

But since when one thing has been set in motion another thing may
be moved by it, and imagination is held to be a movement and to be
impossible without sensation, i.e. to occur in beings that are percipient
and to have for its content what can be perceived, and since movement
may be produced by actual sensation and that movement is necessarily
similar in character to the sensation itself, this movement must be
(1) necessarily (a) incapable of existing apart from sensation, (b)
incapable of existing except when we perceive, (such that in virtue
of its possession that in which it is found may present various phenomena
both active and passive, and (such that it may be either true or false.

The reason of the last characteristic is as follows. Perception (1)
of the special objects of sense is never in error or admits the least
possible amount of falsehood. (2) That of the concomitance of the
objects concomitant with the sensible qualities comes next: in this
case certainly we may be deceived; for while the perception that there
is white before us cannot be false, the perception that what is white
is this or that may be false. (3) Third comes the perception of the
universal attributes which accompany the concomitant objects to which
the special sensibles attach (I mean e.g. of movement and magnitude);
it is in respect of these that the greatest amount of sense-illusion
is possible. 

The motion which is due to the activity of sense in these three modes
of its exercise will differ from the activity of sense; (1) the first
kind of derived motion is free from error while the sensation is present;
(2) and (3) the others may be erroneous whether it is present or absent,
especially when the object of perception is far off. If then imagination
presents no other features than those enumerated and is what we have
described, then imagination must be a movement resulting from an actual
exercise of a power of sense. 

As sight is the most highly developed sense, the name Phantasia (imagination)
has been formed from Phaos (light) because it is not possible to see
without light. 

And because imaginations remain in the organs of sense and resemble
sensations, animals in their actions are largely guided by them, some
(i.e. the brutes) because of the non-existence in them of mind, others
(i.e. men) because of the temporary eclipse in them of mind by feeling
or disease or sleep. 

About imagination, what it is and why it exists, let so much suffice.

Part 4

Turning now to the part of the soul with which the soul knows and
thinks (whether this is separable from the others in definition only,
or spatially as well) we have to inquire (1) what differentiates this
part, and (2) how thinking can take place. 

If thinking is like perceiving, it must be either a process in which
the soul is acted upon by what is capable of being thought, or a process
different from but analogous to that. The thinking part of the soul
must therefore be, while impassible, capable of receiving the form
of an object; that is, must be potentially identical in character
with its object without being the object. Mind must be related to
what is thinkable, as sense is to what is sensible. 

Therefore, since everything is a possible object of thought, mind
in order, as Anaxagoras says, to dominate, that is, to know, must
be pure from all admixture; for the co-presence of what is alien to
its nature is a hindrance and a block: it follows that it too, like
the sensitive part, can have no nature of its own, other than that
of having a certain capacity. Thus that in the soul which is called
mind (by mind I mean that whereby the soul thinks and judges) is,
before it thinks, not actually any real thing. For this reason it
cannot reasonably be regarded as blended with the body: if so, it
would acquire some quality, e.g. warmth or cold, or even have an organ
like the sensitive faculty: as it is, it has none. It was a good idea
to call the soul 'the place of forms', though (1) this description
holds only of the intellective soul, and (2) even this is the forms
only potentially, not actually. 

Observation of the sense-organs and their employment reveals a distinction
between the impassibility of the sensitive and that of the intellective
faculty. After strong stimulation of a sense we are less able to exercise
it than before, as e.g. in the case of a loud sound we cannot hear
easily immediately after, or in the case of a bright colour or a powerful
odour we cannot see or smell, but in the case of mind thought about
an object that is highly intelligible renders it more and not less
able afterwards to think objects that are less intelligible: the reason
is that while the faculty of sensation is dependent upon the body,
mind is separable from it. 

Once the mind has become each set of its possible objects, as a man
of science has, when this phrase is used of one who is actually a
man of science (this happens when he is now able to exercise the power
on his own initiative), its condition is still one of potentiality,
but in a different sense from the potentiality which preceded the
acquisition of knowledge by learning or discovery: the mind too is
then able to think itself. 

Since we can distinguish between a spatial magnitude and what it is
to be such, and between water and what it is to be water, and so in
many other cases (though not in all; for in certain cases the thing
and its form are identical), flesh and what it is to be flesh are
discriminated either by different faculties, or by the same faculty
in two different states: for flesh necessarily involves matter and
is like what is snub-nosed, a this in a this. Now it is by means of
the sensitive faculty that we discriminate the hot and the cold, i.e.
the factors which combined in a certain ratio constitute flesh: the
essential character of flesh is apprehended by something different
either wholly separate from the sensitive faculty or related to it
as a bent line to the same line when it has been straightened out.

Again in the case of abstract objects what is straight is analogous
to what is snub-nosed; for it necessarily implies a continuum as its
matter: its constitutive essence is different, if we may distinguish
between straightness and what is straight: let us take it to be two-ness.
It must be apprehended, therefore, by a different power or by the
same power in a different state. To sum up, in so far as the realities
it knows are capable of being separated from their matter, so it is
also with the powers of mind. 

The problem might be suggested: if thinking is a passive affection,
then if mind is simple and impassible and has nothing in common with
anything else, as Anaxagoras says, how can it come to think at all?
For interaction between two factors is held to require a precedent
community of nature between the factors. Again it might be asked,
is mind a possible object of thought to itself? For if mind is thinkable
per se and what is thinkable is in kind one and the same, then either
(a) mind will belong to everything, or (b) mind will contain some
element common to it with all other realities which makes them all
thinkable. 

(1) Have not we already disposed of the difficulty about interaction
involving a common element, when we said that mind is in a sense potentially
whatever is thinkable, though actually it is nothing until it has
thought? What it thinks must be in it just as characters may be said
to be on a writingtablet on which as yet nothing actually stands written:
this is exactly what happens with mind. 

(Mind is itself thinkable in exactly the same way as its objects are.
For (a) in the case of objects which involve no matter, what thinks
and what is thought are identical; for speculative knowledge and its
object are identical. (Why mind is not always thinking we must consider
later., b) In the case of those which contain matter each of the
objects of thought is only potentially present. It follows that while
they will not have mind in them (for mind is a potentiality of them
only in so far as they are capable of being disengaged from matter)
mind may yet be thinkable. 

Part 5

Since in every class of things, as in nature as a whole, we find two
factors involved, (1) a matter which is potentially all the particulars
included in the class, (2) a cause which is productive in the sense
that it makes them all (the latter standing to the former, as e.g.
an art to its material), these distinct elements must likewise be
found within the soul. 

And in fact mind as we have described it is what it is what it is
by virtue of becoming all things, while there is another which is
what it is by virtue of making all things: this is a sort of positive
state like light; for in a sense light makes potential colours into
actual colours. 

Mind in this sense of it is separable, impassible, unmixed, since
it is in its essential nature activity (for always the active is superior
to the passive factor, the originating force to the matter which it
forms). 

Actual knowledge is identical with its object: in the individual,
potential knowledge is in time prior to actual knowledge, but in the
universe as a whole it is not prior even in time. Mind is not at one
time knowing and at another not. When mind is set free from its present
conditions it appears as just what it is and nothing more: this alone
is immortal and eternal (we do not, however, remember its former activity
because, while mind in this sense is impassible, mind as passive is
destructible), and without it nothing thinks. 

Part 6

The thinking then of the simple objects of thought is found in those
cases where falsehood is impossible: where the alternative of true
or false applies, there we always find a putting together of objects
of thought in a quasi-unity. As Empedocles said that 'where heads
of many a creature sprouted without necks' they afterwards by Love's
power were combined, so here too objects of thought which were given
separate are combined, e.g. 'incommensurate' and 'diagonal': if the
combination be of objects past or future the combination of thought
includes in its content the date. For falsehood always involves a
synthesis; for even if you assert that what is white is not white
you have included not white in a synthesis. It is possible also to
call all these cases division as well as combination. However that
may be, there is not only the true or false assertion that Cleon is
white but also the true or false assertion that he was or will he
white. In each and every case that which unifies is mind.

Since the word 'simple' has two senses, i.e. may mean either (a) 'not
capable of being divided' or (b) 'not actually divided', there is
nothing to prevent mind from knowing what is undivided, e.g. when
it apprehends a length (which is actually undivided) and that in an
undivided time; for the time is divided or undivided in the same manner
as the line. It is not possible, then, to tell what part of the line
it was apprehending in each half of the time: the object has no actual
parts until it has been divided: if in thought you think each half
separately, then by the same act you divide the time also, the half-lines
becoming as it were new wholes of length. But if you think it as a
whole consisting of these two possible parts, then also you think
it in a time which corresponds to both parts together. (But what is
not quantitatively but qualitatively simple is thought in a simple
time and by a simple act of the soul.) 

But that which mind thinks and the time in which it thinks are in
this case divisible only incidentally and not as such. For in them
too there is something indivisible (though, it may be, not isolable)
which gives unity to the time and the whole of length; and this is
found equally in every continuum whether temporal or spatial.

Points and similar instances of things that divide, themselves being
indivisible, are realized in consciousness in the same manner as privations.

A similar account may be given of all other cases, e.g. how evil or
black is cognized; they are cognized, in a sense, by means of their
contraries. That which cognizes must have an element of potentiality
in its being, and one of the contraries must be in it. But if there
is anything that has no contrary, then it knows itself and is actually
and possesses independent existence. 

Assertion is the saying of something concerning something, e.g. affirmation,
and is in every case either true or false: this is not always the
case with mind: the thinking of the definition in the sense of the
constitutive essence is never in error nor is it the assertion of
something concerning something, but, just as while the seeing of the
special object of sight can never be in error, the belief that the
white object seen is a man may be mistaken, so too in the case of
objects which are without matter. 

Part 7

Actual knowledge is identical with its object: potential knowledge
in the individual is in time prior to actual knowledge but in the
universe it has no priority even in time; for all things that come
into being arise from what actually is. In the case of sense clearly
the sensitive faculty already was potentially what the object makes
it to be actually; the faculty is not affected or altered. This must
therefore be a different kind from movement; for movement is, as we
saw, an activity of what is imperfect, activity in the unqualified
sense, i.e. that of what has been perfected, is different from movement.

To perceive then is like bare asserting or knowing; but when the object
is pleasant or painful, the soul makes a quasi-affirmation or negation,
and pursues or avoids the object. To feel pleasure or pain is to act
with the sensitive mean towards what is good or bad as such. Both
avoidance and appetite when actual are identical with this: the faculty
of appetite and avoidance are not different, either from one another
or from the faculty of sense-perception; but their being is different.

To the thinking soul images serve as if they were contents of perception
(and when it asserts or denies them to be good or bad it avoids or
pursues them). That is why the soul never thinks without an image.
The process is like that in which the air modifies the pupil in this
or that way and the pupil transmits the modification to some third
thing (and similarly in hearing), while the ultimate point of arrival
is one, a single mean, with different manners of being. 

With what part of itself the soul discriminates sweet from hot I have
explained before and must now describe again as follows: That with
which it does so is a sort of unity, but in the way just mentioned,
i.e. as a connecting term. And the two faculties it connects, being
one by analogy and numerically, are each to each as the qualities
discerned are to one another (for what difference does it make whether
we raise the problem of discrimination between disparates or between
contraries, e.g. white and black?). Let then C be to D as is to B:
it follows alternando that C: A:: D: B. If then C and D belong to
one subject, the case will be the same with them as with and B; and
B form a single identity with different modes of being; so too will
the former pair. The same reasoning holds if be sweet and B white.

The faculty of thinking then thinks the forms in the images, and as
in the former case what is to be pursued or avoided is marked out
for it, so where there is no sensation and it is engaged upon the
images it is moved to pursuit or avoidance. E.g.. perceiving by sense
that the beacon is fire, it recognizes in virtue of the general faculty
of sense that it signifies an enemy, because it sees it moving; but
sometimes by means of the images or thoughts which are within the
soul, just as if it were seeing, it calculates and deliberates what
is to come by reference to what is present; and when it makes a pronouncement,
as in the case of sensation it pronounces the object to be pleasant
or painful, in this case it avoids or persues and so generally in
cases of action. 

That too which involves no action, i.e. that which is true or false,
is in the same province with what is good or bad: yet they differ
in this, that the one set imply and the other do not a reference to
a particular person. 

The so-called abstract objects the mind thinks just as, if one had
thought of the snubnosed not as snub-nosed but as hollow, one would
have thought of an actuality without the flesh in which it is embodied:
it is thus that the mind when it is thinking the objects of Mathematics
thinks as separate elements which do not exist separate. In every
case the mind which is actively thinking is the objects which it thinks.
Whether it is possible for it while not existing separate from spatial
conditions to think anything that is separate, or not, we must consider
later. 

Part 8

Let us now summarize our results about soul, and repeat that the soul
is in a way all existing things; for existing things are either sensible
or thinkable, and knowledge is in a way what is knowable, and sensation
is in a way what is sensible: in what way we must inquire.

Knowledge and sensation are divided to correspond with the realities,
potential knowledge and sensation answering to potentialities, actual
knowledge and sensation to actualities. Within the soul the faculties
of knowledge and sensation are potentially these objects, the one
what is knowable, the other what is sensible. They must be either
the things themselves or their forms. The former alternative is of
course impossible: it is not the stone which is present in the soul
but its form. 

It follows that the soul is analogous to the hand; for as the hand
is a tool of tools, so the mind is the form of forms and sense the
form of sensible things. 

Since according to common agreement there is nothing outside and separate
in existence from sensible spatial magnitudes, the objects of thought
are in the sensible forms, viz. both the abstract objects and all
the states and affections of sensible things. Hence (1) no one can
learn or understand anything in the absence of sense, and (when the
mind is actively aware of anything it is necessarily aware of it along
with an image; for images are like sensuous contents except in that
they contain no matter. 

Imagination is different from assertion and denial; for what is true
or false involves a synthesis of concepts. In what will the primary
concepts differ from images? Must we not say that neither these nor
even our other concepts are images, though they necessarily involve
them? 

Part 9

The soul of animals is characterized by two faculties, (a) the faculty
of discrimination which is the work of thought and sense, and (b)
the faculty of originating local movement. Sense and mind we have
now sufficiently examined. Let us next consider what it is in the
soul which originates movement. Is it a single part of the soul separate
either spatially or in definition? Or is it the soul as a whole? If
it is a part, is that part different from those usually distinguished
or already mentioned by us, or is it one of them? The problem at once
presents itself, in what sense we are to speak of parts of the soul,
or how many we should distinguish. For in a sense there is an infinity
of parts: it is not enough to distinguish, with some thinkers, the
calculative, the passionate, and the desiderative, or with others
the rational and the irrational; for if we take the dividing lines
followed by these thinkers we shall find parts far more distinctly
separated from one another than these, namely those we have just mentioned:
(1) the nutritive, which belongs both to plants and to all animals,
and (2) the sensitive, which cannot easily be classed as either irrational
or rational; further (3) the imaginative, which is, in its being,
different from all, while it is very hard to say with which of the
others it is the same or not the same, supposing we determine to posit
separate parts in the soul; and lastly (4) the appetitive, which would
seem to be distinct both in definition and in power from all hitherto
enumerated. 

It is absurd to break up the last-mentioned faculty: as these thinkers
do, for wish is found in the calculative part and desire and passion
in the irrational; and if the soul is tripartite appetite will be
found in all three parts. Turning our attention to the present object
of discussion, let us ask what that is which originates local movement
of the animal. 

The movement of growth and decay, being found in all living things,
must be attributed to the faculty of reproduction and nutrition, which
is common to all: inspiration and expiration, sleep and waking, we
must consider later: these too present much difficulty: at present
we must consider local movement, asking what it is that originates
forward movement in the animal. 

That it is not the nutritive faculty is obvious; for this kind of
movement is always for an end and is accompanied either by imagination
or by appetite; for no animal moves except by compulsion unless it
has an impulse towards or away from an object. Further, if it were
the nutritive faculty, even plants would have been capable of originating
such movement and would have possessed the organs necessary to carry
it out. Similarly it cannot be the sensitive faculty either; for there
are many animals which have sensibility but remain fast and immovable
throughout their lives. 

If then Nature never makes anything without a purpose and never leaves
out what is necessary (except in the case of mutilated or imperfect
growths; and that here we have neither mutilation nor imperfection
may be argued from the facts that such animals (a) can reproduce their
species and (b) rise to completeness of nature and decay to an end),
it follows that, had they been capable of originating forward movement,
they would have possessed the organs necessary for that purpose. Further,
neither can the calculative faculty or what is called 'mind' be the
cause of such movement; for mind as speculative never thinks what
is practicable, it never says anything about an object to be avoided
or pursued, while this movement is always in something which is avoiding
or pursuing an object. No, not even when it is aware of such an object
does it at once enjoin pursuit or avoidance of it; e.g. the mind often
thinks of something terrifying or pleasant without enjoining the emotion
of fear. It is the heart that is moved (or in the case of a pleasant
object some other part). Further, even when the mind does command
and thought bids us pursue or avoid something, sometimes no movement
is produced; we act in accordance with desire, as in the case of moral
weakness. And, generally, we observe that the possessor of medical
knowledge is not necessarily healing, which shows that something else
is required to produce action in accordance with knowledge; the knowledge
alone is not the cause. Lastly, appetite too is incompetent to account
fully for movement; for those who successfully resist temptation have
appetite and desire and yet follow mind and refuse to enact that for
which they have appetite. 

Part 10

These two at all events appear to be sources of movement: appetite
and mind (if one may venture to regard imagination as a kind of thinking;
for many men follow their imaginations contrary to knowledge, and
in all animals other than man there is no thinking or calculation
but only imagination). 

Both of these then are capable of originating local movement, mind
and appetite: (1) mind, that is, which calculates means to an end,
i.e. mind practical (it differs from mind speculative in the character
of its end); while (2) appetite is in every form of it relative to
an end: for that which is the object of appetite is the stimulant
of mind practical; and that which is last in the process of thinking
is the beginning of the action. It follows that there is a justification
for regarding these two as the sources of movement, i.e. appetite
and practical thought; for the object of appetite starts a movement
and as a result of that thought gives rise to movement, the object
of appetite being it a source of stimulation. So too when imagination
originates movement, it necessarily involves appetite. 

That which moves therefore is a single faculty and the faculty of
appetite; for if there had been two sources of movement-mind and appetite-they
would have produced movement in virtue of some common character. As
it is, mind is never found producing movement without appetite (for
wish is a form of appetite; and when movement is produced according
to calculation it is also according to wish), but appetite can originate
movement contrary to calculation, for desire is a form of appetite.
Now mind is always right, but appetite and imagination may be either
right or wrong. That is why, though in any case it is the object of
appetite which originates movement, this object may be either the
real or the apparent good. To produce movement the object must be
more than this: it must be good that can be brought into being by
action; and only what can be otherwise than as it is can thus be brought
into being. That then such a power in the soul as has been described,
i.e. that called appetite, originates movement is clear. Those who
distinguish parts in the soul, if they distinguish and divide in accordance
with differences of power, find themselves with a very large number
of parts, a nutritive, a sensitive, an intellective, a deliberative,
and now an appetitive part; for these are more different from one
another than the faculties of desire and passion. 

Since appetites run counter to one another, which happens when a principle
of reason and a desire are contrary and is possible only in beings
with a sense of time (for while mind bids us hold back because of
what is future, desire is influenced by what is just at hand: a pleasant
object which is just at hand presents itself as both pleasant and
good, without condition in either case, because of want of foresight
into what is farther away in time), it follows that while that which
originates movement must be specifically one, viz. the faculty of
appetite as such (or rather farthest back of all the object of that
faculty; for it is it that itself remaining unmoved originates the
movement by being apprehended in thought or imagination), the things
that originate movement are numerically many. 

All movement involves three factors, (1) that which originates the
movement, (2) that by means of which it originates it, and (3) that
which is moved. The expression 'that which originates the movement'
is ambiguous: it may mean either (a) something which itself is unmoved
or (b) that which at once moves and is moved. Here that which moves
without itself being moved is the realizable good, that which at once
moves and is moved is the faculty of appetite (for that which is influenced
by appetite so far as it is actually so influenced is set in movement,
and appetite in the sense of actual appetite is a kind of movement),
while that which is in motion is the animal. The instrument which
appetite employs to produce movement is no longer psychical but bodily:
hence the examination of it falls within the province of the functions
common to body and soul. To state the matter summarily at present,
that which is the instrument in the production of movement is to be
found where a beginning and an end coincide as e.g. in a ball and
socket joint; for there the convex and the concave sides are respectively
an end and a beginning (that is why while the one remains at rest,
the other is moved): they are separate in definition but not separable
spatially. For everything is moved by pushing and pulling. Hence just
as in the case of a wheel, so here there must be a point which remains
at rest, and from that point the movement must originate.

To sum up, then, and repeat what I have said, inasmuch as an animal
is capable of appetite it is capable of self-movement; it is not capable
of appetite without possessing imagination; and all imagination is
either (1) calculative or (2) sensitive. In the latter an animals,
and not only man, partake. 

Part 11

We must consider also in the case of imperfect animals, sc. those
which have no sense but touch, what it is that in them originates
movement. Can they have imagination or not? or desire? Clearly they
have feelings of pleasure and pain, and if they have these they must
have desire. But how can they have imagination? Must not we say that,
as their movements are indefinite, they have imagination and desire,
but indefinitely? 

Sensitive imagination, as we have said, is found in all animals, deliberative
imagination only in those that are calculative: for whether this or
that shall be enacted is already a task requiring calculation; and
there must be a single standard to measure by, for that is pursued
which is greater. It follows that what acts in this way must be able
to make a unity out of several images. 

This is the reason why imagination is held not to involve opinion,
in that it does not involve opinion based on inference, though opinion
involves imagination. Hence appetite contains no deliberative element.
Sometimes it overpowers wish and sets it in movement: at times wish
acts thus upon appetite, like one sphere imparting its movement to
another, or appetite acts thus upon appetite, i.e. in the condition
of moral weakness (though by nature the higher faculty is always more
authoritative and gives rise to movement). Thus three modes of movement
are possible. 

The faculty of knowing is never moved but remains at rest. Since the
one premiss or judgement is universal and the other deals with the
particular (for the first tells us that such and such a kind of man
should do such and such a kind of act, and the second that this is
an act of the kind meant, and I a person of the type intended), it
is the latter opinion that really originates movement, not the universal;
or rather it is both, but the one does so while it remains in a state
more like rest, while the other partakes in movement. 

Part 12

The nutritive soul then must be possessed by everything that is alive,
and every such thing is endowed with soul from its birth to its death.
For what has been born must grow, reach maturity, and decay-all of
which are impossible without nutrition. Therefore the nutritive faculty
must be found in everything that grows and decays. 

But sensation need not be found in all things that live. For it is
impossible for touch to belong either (1) to those whose body is uncompounded
or (2) to those which are incapable of taking in the forms without
their matter. 

But animals must be endowed with sensation, since Nature does nothing
in vain. For all things that exist by Nature are means to an end,
or will be concomitants of means to an end. Every body capable of
forward movement would, if unendowed with sensation, perish and fail
to reach its end, which is the aim of Nature; for how could it obtain
nutriment? Stationary living things, it is true, have as their nutriment
that from which they have arisen; but it is not possible that a body
which is not stationary but produced by generation should have a soul
and a discerning mind without also having sensation. (Nor yet even
if it were not produced by generation. Why should it not have sensation?
Because it were better so either for the body or for the soul? But
clearly it would not be better for either: the absence of sensation
will not enable the one to think better or the other to exist better.)
Therefore no body which is not stationary has soul without sensation.

But if a body has sensation, it must be either simple or compound.
And simple it cannot be; for then it could not have touch, which is
indispensable. This is clear from what follows. An animal is a body
with soul in it: every body is tangible, i.e. perceptible by touch;
hence necessarily, if an animal is to survive, its body must have
tactual sensation. All the other senses, e.g. smell, sight, hearing,
apprehend through media; but where there is immediate contact the
animal, if it has no sensation, will be unable to avoid some things
and take others, and so will find it impossible to survive. That is
why taste also is a sort of touch; it is relative to nutriment, which
is just tangible body; whereas sound, colour, and odour are innutritious,
and further neither grow nor decay. Hence it is that taste also must
be a sort of touch, because it is the sense for what is tangible and
nutritious. 

Both these senses, then, are indispensable to the animal, and it is
clear that without touch it is impossible for an animal to be. All
the other senses subserve well-being and for that very reason belong
not to any and every kind of animal, but only to some, e.g. those
capable of forward movement must have them; for, if they are to survive,
they must perceive not only by immediate contact but also at a distance
from the object. This will be possible if they can perceive through
a medium, the medium being affected and moved by the perceptible object,
and the animal by the medium. just as that which produces local movement
causes a change extending to a certain point, and that which gave
an impulse causes another to produce a new impulse so that the movement
traverses a medium the first mover impelling without being impelled,
the last moved being impelled without impelling, while the medium
(or media, for there are many) is both-so is it also in the case of
alteration, except that the agent produces produces it without the
patient's changing its place. Thus if an object is dipped into wax,
the movement goes on until submersion has taken place, and in stone
it goes no distance at all, while in water the disturbance goes far
beyond the object dipped: in air the disturbance is propagated farthest
of all, the air acting and being acted upon, so long as it maintains
an unbroken unity. That is why in the case of reflection it is better,
instead of saying that the sight issues from the eye and is reflected,
to say that the air, so long as it remains one, is affected by the
shape and colour. On a smooth surface the air possesses unity; hence
it is that it in turn sets the sight in motion, just as if the impression
on the wax were transmitted as far as the wax extends. 

Part 13

It is clear that the body of an animal cannot be simple, i.e. consist
of one element such as fire or air. For without touch it is impossible
to have any other sense; for every body that has soul in it must,
as we have said, be capable of touch. All the other elements with
the exception of earth can constitute organs of sense, but all of
them bring about perception only through something else, viz. through
the media. Touch takes place by direct contact with its objects, whence
also its name. All the other organs of sense, no doubt, perceive by
contact, only the contact is mediate: touch alone perceives by immediate
contact. Consequently no animal body can consist of these other elements.

Nor can it consist solely of earth. For touch is as it were a mean
between all tangible qualities, and its organ is capable of receiving
not only all the specific qualities which characterize earth, but
also the hot and the cold and all other tangible qualities whatsoever.
That is why we have no sensation by means of bones, hair, &c., because
they consist of earth. So too plants, because they consist of earth,
have no sensation. Without touch there can be no other sense, and
the organ of touch cannot consist of earth or of any other single
element. 

It is evident, therefore, that the loss of this one sense alone must
bring about the death of an animal. For as on the one hand nothing
which is not an animal can have this sense, so on the other it is
the only one which is indispensably necessary to what is an animal.
This explains, further, the following difference between the other
senses and touch. In the case of all the others excess of intensity
in the qualities which they apprehend, i.e. excess of intensity in
colour, sound, and smell, destroys not the but only the organs of
the sense (except incidentally, as when the sound is accompanied by
an impact or shock, or where through the objects of sight or of smell
certain other things are set in motion, which destroy by contact);
flavour also destroys only in so far as it is at the same time tangible.
But excess of intensity in tangible qualities, e.g. heat, cold, or
hardness, destroys the animal itself. As in the case of every sensible
quality excess destroys the organ, so here what is tangible destroys
touch, which is the essential mark of life; for it has been shown
that without touch it is impossible for an animal to be. That is why
excess in intensity of tangible qualities destroys not merely the
organ, but the animal itself, because this is the only sense which
it must have. 

All the other senses are necessary to animals, as we have said, not
for their being, but for their well-being. Such, e.g. is sight, which,
since it lives in air or water, or generally in what is pellucid,
it must have in order to see, and taste because of what is pleasant
or painful to it, in order that it may perceive these qualities in
its nutriment and so may desire to be set in motion, and hearing that
it may have communication made to it, and a tongue that it may communicate
with its fellows. 

THE END

% chapter soul (end)